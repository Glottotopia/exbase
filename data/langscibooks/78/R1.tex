\chapter[References]{References}
\label{Para_R}
\begin{styleCitaviBibliographyEntry}
Abbot, Barbara. 2006. Definite and \isi{indefinite}. In Brown, Keith (ed.), \textit{Encyclopedia of language and linguistics}, 2\textsuperscript{nd} edn.. Amsterdam: Elsevier Science Ltd. 392–399.
\end{styleCitaviBibliographyEntry}

\begin{styleCitaviBibliographyEntry}
Abrams, Meyer H. \& Harpham, Geoffrey G. 2009. \textit{A glossary of literary terms}, 9\textsuperscript{th} edn. Boston: Wadsworth Cengage Learning.
\end{styleCitaviBibliographyEntry}

\begin{styleCitaviBibliographyEntry}
Adelaar, K. A. 1992. \textit{Proto \ili{Malayic}: The reconstruction of its \isi{phonology} and parts of its lexicon and morphology} (Pacific Linguistics C-119). Canberra: Research School of Pacific Studies, The Australian National University.
\end{styleCitaviBibliographyEntry}

\begin{styleCitaviBibliographyEntry}
Adelaar, K. A. 2001. Malay: A short history. In Citro, Maria \& Maria, Luigi S. (eds.), \textit{Alam melayu il mondo maolese: Lingua, storia, cultura} (Oriente Moderno XIX.2). Roma: Istituto per l’Oriente C.A. Nallino, 225–242.
\end{styleCitaviBibliographyEntry}

\begin{styleCitaviBibliographyEntry}
Adelaar, K. A. 2005a. \ili{Malayo-Sumbawan}. \textit{\ili{Oceanic} Linguistics} 44(2): 357–388. Online URL: \url{http://www.jstor.org/stable/3623345} (Accessed 8 Jan 2016).
\end{styleCitaviBibliographyEntry}

\begin{styleCitaviBibliographyEntry}
Adelaar, K. A. 2005b. Structural diversity in the \ili{Malayic} subgroup. In Adelaar, K. A. \& Himmelmann, Nikolaus P. (eds.), \textit{The \ili{Austronesian} languages of Asia and Madagascar} (Routledge Language Family Series). London: Routledge, 202–226.
\end{styleCitaviBibliographyEntry}

\begin{styleCitaviBibliographyEntry}
Adelaar, K. A. 2005c. The \ili{Austronesian} languages of Asia and Madagascar: A historical perspective. In Adelaar, K. A. \& Himmelmann, Nikolaus P. (eds.), \textit{The \ili{Austronesian} languages of Asia and Madagascar} (Routledge Language Family Series). London: Routledge, 1–42.
\end{styleCitaviBibliographyEntry}

\begin{styleCitaviBibliographyEntry}
Adelaar, K. A. \& Prentice, David J. 1996. Malay: Its history, role and spread. In Wurm, Stephen A. \& Mühlhäusler, Peter \& Tryon, Darrell T. (eds.), \textit{Atlas of languages of intercultural communication in the Pacific, Asia, and the Americas} (Trends in Linguistics: Documentation 13). Berlin: Mouton de Gruyter, 673–693.
\end{styleCitaviBibliographyEntry}

\begin{styleCitaviBibliographyEntry}
Agheyisi, Rebecca \& Fishman, Joshua A. 1970. Language attitude studies: A brief survey of methodological approaches. \textit{Anthropological Linguistics} 12(5): 137–157.
\end{styleCitaviBibliographyEntry}

\begin{styleCitaviBibliographyEntry}
Aikhenvald, Alexandra Y. 2007. Typological distinctions in \isi{word-formation}. In Shopen, Timothy (ed.), \textit{Language typology and syntactic description. Volume 3: Grammatical categories and the lexicon}, 2\textsuperscript{nd} edn.. Cambridge: Cambridge University Press, 1–65.
\end{styleCitaviBibliographyEntry}

\begin{styleCitaviBibliographyEntry}
Aikhenvald, Alexandra Y. \& Stebbins, Tonya. 2007. Languages of New Guinea. In Miyaoka, Osahito \& Sakiyama, Osamu \& Krauss, Michael E. (eds.), \textit{The vanishing languages of the Pacific rim} (Oxford Linguistics). Oxford: Oxford University Press, 239–266.
\end{styleCitaviBibliographyEntry}

\begin{styleCitaviBibliographyEntry}
Ajamiseba, Daniel C. 1984. Kebinekaan bahasa di Irian Jaya. In Koentjaraningrat (ed.), \textit{Irian Jaya: Membangun masyarakat majemuk} (Seri Etnografi Indonesia 5). Jakarta: Penerbit Djambatan, 119–135.
\end{styleCitaviBibliographyEntry}

\begin{styleCitaviBibliographyEntry}
Allerton, David J. 2006. Valency grammar. In Brown, Keith (ed.), \textit{Encyclopedia of language and linguistics}, 2\textsuperscript{nd} edn.. Amsterdam: Elsevier Science Ltd. 301–314.
\end{styleCitaviBibliographyEntry}

\begin{styleCitaviBibliographyEntry}
Alua, Agus A. 2006. \textit{Papua Barat dari pangkuan ke pangkuan: Suatu ikhtisar kronologis} (Seri Pendidikan Politik Papua 1). Jayapura: Sekretariat Presidium Dewan Papua dan Biro Penelitian STFT Fajar Timur.
\end{styleCitaviBibliographyEntry}

\begin{styleCitaviBibliographyEntry}
Ameka, Felix K. 2006. Interjections. In Brown, Keith (ed.), \textit{Encyclopedia of language and linguistics}, 2\textsuperscript{nd} edn.. Amsterdam: Elsevier Science Ltd. 743–746.
\end{styleCitaviBibliographyEntry}

\begin{styleCitaviBibliographyEntry}
Anceaux, Johannes C. no date. \textit{Nieuwguinees Maleis – Nederlands} (KITLV-inventaris 64. Or. 615: Collectie Anceaux, Johannes Cornelis Anceaux (1920-1988) 88). Leiden: KITLV Archives.
\end{styleCitaviBibliographyEntry}

\begin{styleCitaviBibliographyEntry}
Anceaux, Johannes C. \& Veldkamp, F. 1960. \textit{Woordenlijst Maleis-Nederlands-\ili{Dani}: Naar gegevens van F. Veldkamp bewerkt door J.C. Anceaux} (Rapport: Kantoor voor Bevolkingszaken 140). Hollandia: Dienst van Binnenlandse Zaken, Kantoor voor Bevolkingszaken, Gouvernement van Nederlands-Nieuw-Guinea.
\end{styleCitaviBibliographyEntry}

\begin{styleCitaviBibliographyEntry}
Andaya, Leonard Y. 1993. \textit{The world of Maluku: Eastern Indonesia in the early modern period}. Honolulu: University of Hawai’i Press.
\end{styleCitaviBibliographyEntry}

\begin{styleCitaviBibliographyEntry}
Anderbeck, Karl R. 2007. ISO 639-3 Registration authority: Request for new language code element in ISO 639-3 (Papuan Malay) (ISO 639-3 - Change request documentation for: 2007-183). Dallas: SIL International. Online URL: \url{http://www.sil.org/iso639-3/cr_files/2007-183.pdf} (Accessed 8 Jan 2016).
\end{styleCitaviBibliographyEntry}

\begin{styleCitaviBibliographyEntry}
Anderson, Gregory D. 2011. The velar nasal. In Haspelmath, Martin \& Dryer, Matthew S. \& Gil, David \& Comrie, Bernard (eds.), \textit{The world atlas of language structures}. München: Max Planck Digital Library, 1–8. Online URL: \url{http://wals.info/chapter/9} (Accessed 8 Jan 2016).
\end{styleCitaviBibliographyEntry}

\begin{styleCitaviBibliographyEntry}
Anderson, Stephen R. \& Keenan, Edward L. 1985. Deixis. In Shopen, Timothy (ed.), \textit{Language typology and syntactic description. Volume 3: Grammatical categories and the lexicon}. Cambridge: Cambridge University Press, 259–308.
\end{styleCitaviBibliographyEntry}

\begin{styleCitaviBibliographyEntry}
Andrews, Avery D. 2007. The major functions of the \isi{noun} phrase. In Shopen, Timothy (ed.), \textit{Language typology and syntactic description. Volume 1: Clause structure}, 2\textsuperscript{nd} edn.. Cambridge: Cambridge University Press, 132–223.
\end{styleCitaviBibliographyEntry}

\begin{styleCitaviBibliographyEntry}
Arakin, Vladimir D. 1963. \textit{Mal’ga\v{s}skij jazyk} (Jazyki zarube\v{z}nogo vostoka i Afriki). Moskva: Vosto\v{c}noj Literatury.
\end{styleCitaviBibliographyEntry}

\begin{styleCitaviBibliographyEntry}
Arakin, Vladimir D. 1981. \textit{Taitânskij âzyk} (\^{A}zyki Narodov Azii i Afriki). Moskva: Izdatel’stvo “Nauka”.
\end{styleCitaviBibliographyEntry}

\begin{styleCitaviBibliographyEntry}
Aronoff, Mark \& Schvaneveldt, Roger. 1978. Testing morphological productivity. \textit{Annals of the New York Academy of Sciences} 318: 106–114. Online URL: \url{http://dx.doi.org/10.1111/j.1749-6632.1978.tb16357.x} (Accessed 8 Jan 2016).
\end{styleCitaviBibliographyEntry}

\begin{styleCitaviBibliographyEntry}
Asher, Robert E. 1994. Glossary. In Asher, Robert E. (ed.), \textit{The Encyclopedia of language and linguistics}. Oxford: Pergamon Press, 5087–5188.
\end{styleCitaviBibliographyEntry}

\begin{styleCitaviBibliographyEntry}
Baayen, R. Harald. 1992. Quantitative aspects of morphological productivity. In Booij, Geert E. \& van Marle, Jaap (eds.), \textit{Yearbook of Morphology 1991}. Dordrecht: Kluwer Academic Publishers, 109–150.
\end{styleCitaviBibliographyEntry}

\begin{styleCitaviBibliographyEntry}
Baker, Colin. 1992. \textit{Attitudes and language} (Multilingual Matters 83). Clevedon: Multilingual Matters.
\end{styleCitaviBibliographyEntry}

\begin{styleCitaviBibliographyEntry}
Bao, Zhiming. 2009. One in Singapore English. \textit{Studies in Language} 33(2): 338–365. Online URL: \url{http://dx.doi.org/10.1075/sl.33.2.05bao} (Accessed 8 Jan 2016).
\end{styleCitaviBibliographyEntry}

\begin{styleCitaviBibliographyEntry}
Bauer, Laurie. 1983. \textit{English word-formation} (Cambridge Textbooks in Linguistics). Cambridge: Cambridge University Press.
\end{styleCitaviBibliographyEntry}

\begin{styleCitaviBibliographyEntry}
Bauer, Laurie. 2001. \textit{Morphological productivity}. Cambridge: Cambridge University Press.
\end{styleCitaviBibliographyEntry}

\begin{styleCitaviBibliographyEntry}
Bauer, Laurie. 2003. \textit{Introducing linguistic morphology}. Edinburgh: Edinburgh University Press.
\end{styleCitaviBibliographyEntry}

\begin{styleCitaviBibliographyEntry}
Bauer, Laurie. 2009. Typology of compounds. In Lieber, Rochelle \& Štekauer, Pavol (eds.), \textit{The Oxford handbook of compounding} (Oxford Handbooks in Linguistics). Oxford: Oxford University Press, 343–356.
\end{styleCitaviBibliographyEntry}

\begin{styleCitaviBibliographyEntry}
Besier, Dominik. 2012. Die Position und Bedeutung des Papua Malay in der Gesellschaft. Hamburg: University of Hamburg. (BA thesis.)
\end{styleCitaviBibliographyEntry}

\begin{styleCitaviBibliographyEntry}
Bhat, Darbhe N. S. 2007. \textit{Pronouns} (Oxford Studies in Typology and Linguistic Theory). Oxford: Oxford University Press.
\end{styleCitaviBibliographyEntry}

\begin{styleCitaviBibliographyEntry}
Bhat, Darbhe N. S. 2011. Third person pronouns and demonstratives. In Haspelmath, Martin \& Dryer, Matthew S. \& Gil, David \& Comrie, Bernard (eds.), \textit{The world atlas of language structures}. München: Max Planck Digital Library, Chapter 43. Online URL: \url{http://wals.info/chapter/43} (Accessed 8 Jan 2016).
\end{styleCitaviBibliographyEntry}

\begin{styleCitaviBibliographyEntry}
Biber, Douglas \& Conrad, Susan \& Leech, Geoffrey N. 2002. \textit{Longman student grammar of spoken and written English}. Harlow: Longman.
\end{styleCitaviBibliographyEntry}

\begin{styleCitaviBibliographyEntry}
Bickel, Balthasar \& Witzlack-Makarevich, Alena. 2008. Referential scales and case alignment: Reviewing the typological evidence. In Richards, Marc \& Malchukov, Andrej L. (eds.), \textit{Scales} (Linguistische Arbeitsberichte LAB 86). Leipzig: Institut für Linguistik, Universität Leipzig, 1–37.
\end{styleCitaviBibliographyEntry}

\begin{styleCitaviBibliographyEntry}
Bidang Neraca Wilayah dan Analisis Statistik. 2011a. \textit{Indikator pendidikan Provinsi Papua 2011}. Jayapura: Badan Pusat Statistik Provinsi Papua. Online URL: \url{http://papua.bps.go.id/arc/2012/indik2011/indik2011.html} (Accessed 22 Oct 2013).
\end{styleCitaviBibliographyEntry}

\begin{styleCitaviBibliographyEntry}
Bidang Neraca Wilayah dan Analisis Statistik. 2011b. \textit{Statistik daerah Provinsi Papua Barat 2011}. Manokwari: Badan Pusat Statistik Provinsi Papua Barat. Online URL: \url{http://papuabarat.bps.go.id/website/pdf_publikasi/Statistik-Daerah-Provinsi-Papua-Barat-2011.pdf} (Accessed 8 Jan 2016).
\end{styleCitaviBibliographyEntry}

\begin{styleCitaviBibliographyEntry}
Bidang Neraca Wilayah dan Analisis Statistik. 2012a. \textit{Papua dalam angka 2012 – Papua in Figure 2012}. Jayapura: Badan Pusat Statistik Provinsi Papua. Online URL: \url{http://papua.bps.go.id/arc/2012/dda2012/dda2012.html} (Accessed 22 Oct 2013).
\end{styleCitaviBibliographyEntry}

\begin{styleCitaviBibliographyEntry}
Bidang Neraca Wilayah dan Analisis Statistik. 2012b. \textit{Statistik daerah Provinsi Papua 2012}. Jayapura: Badan Pusat Statistik Provinsi Papua Barat. Online URL: \url{http://papua.bps.go.id/arc/2012/statda2012/statda2012.html} (Accessed 21 Oct 2013).
\end{styleCitaviBibliographyEntry}

\begin{styleCitaviBibliographyEntry}
Bink, G. L. 1894. Tocht van den zendeling Bink naar de Humboldts-baai (Met schetskaartje). \textit{Tijdschrift van het Koninklijk Nederlandsch Aardrijkskundig Genootschap} XI: 325–332. Online URL: \url{http://www.columbia.edu/cu/lweb/digital/collections/cul/texts/ldpd_10273444_001/ldpd_10273444_001.pdf} (Accessed 8 Jan 2016).
\end{styleCitaviBibliographyEntry}

\begin{styleCitaviBibliographyEntry}
Bisang, Walter. 2009. On the evolution of complexity: Sometimes less is more in East and mainland Southeast Asia. In Sampson, Geoffrey \& Gil, David \& Trudgill, Peter (eds.), \textit{Language complexity as an evolving variable} (Studies in the Evolution of Language 13). Oxford: Oxford University Press, 34–49.
\end{styleCitaviBibliographyEntry}

\begin{styleCitaviBibliographyEntry}
Blust, Robert A. 1994. The \ili{Austronesian} settlement of mainland Southeast Asia. In Adams, Karen L. \& Hudak, Thomas J. \& Southeast Asian Linguistics Society (eds.), \textit{Papers from the Second Annual Meeting of the Southeast Asian Linguistics Society, 1992}. Tempe: Program for Southeast Asian Studies, Arizona State University, 25–83. Online URL: \url{http://sealang.net/sala/archives/pdf8/blust1994austronesian.pdf} (Accessed 8 Jan 2016).
\end{styleCitaviBibliographyEntry}

\begin{styleCitaviBibliographyEntry}
Blust, Robert A. 1999. Subgrouping, circularity and extinction: Some issues in \ili{Austronesian} comparative linguistics. In Zeitoun, Elizabeth \& Li, Rengui (eds.), \textit{Selected papers from the Eighth International Conference on \ili{Austronesian} Linguistics} (Symposium Series of the Institute of Linguistics (Preparatory Office), Academia Sinica 1). Taipei: Academia Sinica, 31–94.
\end{styleCitaviBibliographyEntry}

\begin{styleCitaviBibliographyEntry}
Blust, Robert A. 2010. The Greater North Borneo Hypothesis. \textit{\ili{Oceanic} Linguistics} 49(1): 44–118. Online URL: \url{http://www.jstor.org/stable/40783586} (Accessed 8 Jan 2016).
\end{styleCitaviBibliographyEntry}

\begin{styleCitaviBibliographyEntry}
Blust, Robert A. 2012. One mark per word? Some patterns of dissimilation in \ili{Austronesian} and Australian languages. \textit{Phonology} 29(3): 355 381. Online URL: \url{http://dx.doi.org/10.1017/S0952675712000206} (Accessed 8 Jan 2016).
\end{styleCitaviBibliographyEntry}

\begin{styleCitaviBibliographyEntry}
Blust, Robert A. 2013. \textit{The \ili{Austronesian} languages} (Asia-Pacific Linguistics: Open Access Monographs A-PL 008), 2\textsuperscript{nd} edn. Canberra: Research School of Pacific and Asian Studies, The Australian National University. Online URL: \url{http://hdl.handle.net/1885/10191} (Accessed 8 Jan 2016).
\end{styleCitaviBibliographyEntry}

\begin{styleCitaviBibliographyEntry}
Booij, Geert E. 1986. Form and meaning in \isi{morphology}: The case of \ili{Dutch} ‘agent nouns’. \textit{Linguistics} 24: 503–517. Online URL: \url{http://dx.doi.org/10.1515/ling.1986.24.3.503} (Accessed 8 Jan 2016).
\end{styleCitaviBibliographyEntry}

\begin{styleCitaviBibliographyEntry}
Booij, Geert E. 2002. \textit{The \isi{morphology} of Dutch} (Oxford Linguistics). Oxford: Oxford University Press.
\end{styleCitaviBibliographyEntry}

\begin{styleCitaviBibliographyEntry}
Booij, Geert E. 2007. \textit{The grammar of words: An introduction to linguistic morphology} (Oxford Textbooks in Linguistics), 2\textsuperscript{nd} edn. Oxford: Oxford University Press.
\end{styleCitaviBibliographyEntry}

\begin{styleCitaviBibliographyEntry}
Booij, Geert E. 2013. Morphology in construction grammar. In Hoffmann, Thomas \& Trousdale, Graeme (eds.), \textit{The Oxford handbook of construction grammar}. Oxford: Oxford University Press, 255–273. Online URL: \url{https://geertbooij.files.wordpress.com/2014/02/booij-2013-morphology-in-cxg.pdf} (Accessed 8 Jan 2016).
\end{styleCitaviBibliographyEntry}

\begin{styleCitaviBibliographyEntry}
Bosch, C. J. 1995. Memorie van overgave van het bestuur der Residentie Ternate door C. J. Bosch aftredende Resident aan P. van der Crab Assistent Resident ter beschikking van den Gouverneur der Moluksche Eilanden, belast met het beheer van even genoemd gewest. In Overweel, Jeroen A. (ed.), \textit{Topics relating to Netherlands New Guinea in Ternate Residency memoranda of transfer and other assorted documents} (Irian Jaya Source Materials 13). Leiden: DSALCUL/IRIS, 28–29.
\end{styleCitaviBibliographyEntry}

\begin{styleCitaviBibliographyEntry}
Bowen, J. D. \& Philippine Center for Language Study. 1965. \textit{Beginning \ili{Tagalog}: A course for speakers of English}. Berkeley: University of California Press.
\end{styleCitaviBibliographyEntry}

\begin{styleCitaviBibliographyEntry}
Bublitz, Wolfram \& Bednarek, Monika. 2006. Reported speech: Pragmatic aspects. In Brown, Keith (ed.), \textit{Encyclopedia of language and linguistics}, 2\textsuperscript{nd} edn.. Amsterdam: Elsevier Science Ltd. 550–553.
\end{styleCitaviBibliographyEntry}

\begin{styleCitaviBibliographyEntry}
Bureau Cursussen en Vertalingen. 1950. \textit{Beknopte leergang Maleis voor Nieuw-Guinea}. Amsterdam: Koninklijk Instituut voor de Tropen.
\end{styleCitaviBibliographyEntry}

\begin{styleCitaviBibliographyEntry}
Burke, Edmund. 1831. \textit{The annual register, or a view of the history, politics, and literature of the year 1830} (The Annual Register 72). London: Dodsley. Online URL: \url{https://archive.org/details/annualregistero05burkgoog} (Accessed 8 Jan 2016).
\end{styleCitaviBibliographyEntry}

\begin{styleCitaviBibliographyEntry}
Burung, Willem. 2004. Comparisons in Melayu-Papua. Paper presented at the RCLT Workshop on Comparative Constructions La Trobe University. Melbourne, 6 October 2004.
\end{styleCitaviBibliographyEntry}

\begin{styleCitaviBibliographyEntry}
Burung, Willem. 2005. Discourse analysis. Kangaroo Ground: EQUIP Training.
\end{styleCitaviBibliographyEntry}

\begin{styleCitaviBibliographyEntry}
Burung, Willem. 2008a. Melayu Papua – A hidden treasure. Paper presented at the Second International Conference on Language Development, Language Revitalization and Multilingual Education in Ethnolinguistic Communities. Bangkok, 1-3 July 2008. Online URL: \url{http://www.seameo.org/_ld2008/doucments/Presentation_document/MicrosoftWord_Burung_MelayuPapuaAhidden_treasure_with_edits.pdf} (Accessed 8 Jan 2016).
\end{styleCitaviBibliographyEntry}

\begin{styleCitaviBibliographyEntry}
Burung, Willem. 2008b. The prime ‘FEEL’ in Melayu Papua: Cognition, emotion and body.
\end{styleCitaviBibliographyEntry}

\begin{styleCitaviBibliographyEntry}
Burung, Willem. 2009. Melayu Papua: Where have all its speakers gone. Jayapura: Universitas Cenderawasih.
\end{styleCitaviBibliographyEntry}

\begin{styleCitaviBibliographyEntry}
Burung, Willem \& Sawaki, Yusuf W. 2007. On syntactical paradigm of \isi{causative} constructions in Melayu Papua. Paper presented at the Eleventh International Symposium on Malay/Indonesian Linguistics – ISMIL 11. Manokwari, 6-8 August 2007. Online URL: \url{http://email.eva.mpg.de/~gil/ismil/11/abstracts/BurungSawaki.pdf} (Accessed 8 Jan 2016).
\end{styleCitaviBibliographyEntry}

\begin{styleCitaviBibliographyEntry}
Bussmann, Hadumod. 1996. \textit{Routledge dictionary of language and linguistics}. London: Routledge.
\end{styleCitaviBibliographyEntry}

\begin{styleCitaviBibliographyEntry}
Bussmann, Hadumod. 2000. \textit{Routledge dictionary of language and linguistics} (Routledge Reference). Beijing: Foreign Language Teaching and Research Press.
\end{styleCitaviBibliographyEntry}

\begin{styleCitaviBibliographyEntry}
Butler, Christopher. 2003. \textit{Structure and function: A guide to three major structural-functional theories. Part II: From clause to discourse and beyond} (Studies in Language Companion Series 64). Amsterdam: John Benjamins Publishing Company.
\end{styleCitaviBibliographyEntry}

\begin{styleCitaviBibliographyEntry}
Bybee, Joan L. 2006. From usage to grammar. \textit{Language} 82(4): 711–733. Online URL: \url{http://dx.doi.org/10.1353/lan.2006.0186} (Accessed 8 Jan 2016).
\end{styleCitaviBibliographyEntry}

\begin{styleCitaviBibliographyEntry}
Campbell, George L. 2000a. \textit{Compendium of the world’s languages. Vo. I: Abaza to Kurdish}, 2\textsuperscript{nd} edn. London: Routledge.
\end{styleCitaviBibliographyEntry}

\begin{styleCitaviBibliographyEntry}
Campbell, George L. 2000b. \textit{Compendium of the world’s languages. Volume 2: Ladakhi to Zuni}, 2\textsuperscript{nd} edn. London: Routledge.
\end{styleCitaviBibliographyEntry}

\begin{styleCitaviBibliographyEntry}
Chauvel, Richard. 2002. Papua and Indonesia. Where contending nationalisms meet. In Kingsbury, Damien \& Aveling, Harry (eds.), \textit{Autonomy and disintegration Indonesia}. London: Routledge Curzon Press, 115–127.
\end{styleCitaviBibliographyEntry}

\begin{styleCitaviBibliographyEntry}
Churchward, C. M. 1953. \textit{\ili{Tongan} grammar}. London: Oxford University Press.
\end{styleCitaviBibliographyEntry}

\begin{styleCitaviBibliographyEntry}
Clouse, Duane A. 2000. Papuan Malay – What is it! Paper presented at the SIL Malay Seminar. Yogyakarta, 8-10 January 2000.
\end{styleCitaviBibliographyEntry}

\begin{styleCitaviBibliographyEntry}
Clouse, Duane A. \& Donohue, Mark \& Ma, Felix. 2002. Survey report of the North Coast of Irian Jaya. \textit{SIL Electronic Survey Reports} 2002-078: 18 p. Online URL: \url{http://www.sil.org/silesr/2002/SILESR2002-078.pdf} (Accessed 8 Jan 2016).
\end{styleCitaviBibliographyEntry}


\begin{styleCitaviBibliographyEntry}
Collins, James T. 1980. \textit{Ambonese Malay and creolization theory} (Penerbitan Ilmiah 5). Kuala Lumpur: Dewan Bahasa dan Pustaka; Kementarian Pelajaran Malaysia.
\end{styleCitaviBibliographyEntry}

\begin{styleCitaviBibliographyEntry}
Collins, James T. 1998. \textit{Malay, world language: A short history}, 2\textsuperscript{nd} edn. Kuala Lumpur: Dewan Bahasa dan Pustaka.
\end{styleCitaviBibliographyEntry}

\begin{styleCitaviBibliographyEntry}
Commissie voor Nieuw Guinea \& van der Goes, H. D. A. \& Johan Hendrik Croockewit \& Commissie voor Nieuw Guinea, Netherlands. 1862. \textit{Nieuw Guinea, ethnographisch en natuurkundig onderzocht en beschreven in 1858}. Amsterdam: Frederik Muller. Online URL: \url{http://books.google.nl/books/download/Nieuw_Guinea_ethnographisch_en_natuurkun.pdf?id=CgAvAAAAYAAJ & output=pdf & sig=ACfU3U1Lv1gqcqbtPpihc6Yf8u8V_ipehQ} (Accessed 8 Jan 2016).
\end{styleCitaviBibliographyEntry}

\begin{styleCitaviBibliographyEntry}
Comrie, Bernard. 1989. \textit{Language universals and linguistic typology: Syntax and morphology}, 2\textsuperscript{nd} edn. Chicago: University of Chicago Press.
\end{styleCitaviBibliographyEntry}

\begin{styleCitaviBibliographyEntry}
Comrie, Bernard \& Thompson, Sandra A. 2007. Lexical nominalization. In Shopen, Timothy (ed.), \textit{Language typology and syntactic description. Volume 3: Grammatical categories and the lexicon}, 2\textsuperscript{nd} edn.. Cambridge: Cambridge University Press, 334–381.
\end{styleCitaviBibliographyEntry}

\begin{styleCitaviBibliographyEntry}
Cooper, Robert L. \& Fishman, Joshua A. 1974. The study of language attitudes. \textit{Linguistics} 136: 5–19.
\end{styleCitaviBibliographyEntry}

\begin{styleCitaviBibliographyEntry}
Cristofaro, Sonia. 2005. \textit{Subordination}. Oxford: Oxford University Press.
\end{styleCitaviBibliographyEntry}

\begin{styleCitaviBibliographyEntry}
Croft, William. 1991. \textit{Syntactic categories and grammatical relations: The cognitive organization of information}. Chicago: University of Chicago Press.
\end{styleCitaviBibliographyEntry}

\begin{styleCitaviBibliographyEntry}
Crystal, David. 2008. \textit{A dictionary of linguistics and phonetics} (The Language Library), 6\textsuperscript{th} edn. Malden: Basil Blackwell Publishers.
\end{styleCitaviBibliographyEntry}

\begin{styleCitaviBibliographyEntry}
Daniel, Michael. 2011. Plurality in independent personal pronouns. In Haspelmath, Martin \& Dryer, Matthew S. \& Gil, David \& Comrie, Bernard (eds.), \textit{The world atlas of language structures}. München: Max Planck Digital Library, Chapter 35. Online URL: \url{http://wals.info/chapter/35} (Accessed 8 Jan 2016).
\end{styleCitaviBibliographyEntry}

\begin{styleCitaviBibliographyEntry}
Daniel, Michael \& Moravcsik, Edith A. 2011. The \isi{associative plural}. In Haspelmath, Martin \& Dryer, Matthew S. \& Gil, David \& Comrie, Bernard (eds.), \textit{The world atlas of language structures}. München: Max Planck Digital Library, Chapter 36. Online URL: \url{http://wals.info/chapter/36} (Accessed 8 Jan 2016).
\end{styleCitaviBibliographyEntry}

\begin{styleCitaviBibliographyEntry}
De Lacy, Paul V. 2006. \textit{Markedness: Reduction and preservation in phonology} (Cambridge Studies in Linguistics 112). Cambridge: Cambridge University Press.
\end{styleCitaviBibliographyEntry}

\begin{styleCitaviBibliographyEntry}
De Saussure, Ferdinand. 1959. \textit{Course in general linguistics (edited by Charles Bally and Albert Sechehaye in collaboration with Albert Reidlinger; translated from the French by Wade Baskin)}, 3\textsuperscript{rd} edn. New York: Philosophical Library.
\end{styleCitaviBibliographyEntry}

\begin{styleCitaviBibliographyEntry}
De Vries, Lourens. 2005. Towards a typology of tail–head linkage in \ili{Papuan languages}. \textit{Studies in Language} 29(2): 363–384. Online URL: \url{http://dx.doi.org/10.1075/sl.29.2.04vri} (Accessed 8 Jan 2016).
\end{styleCitaviBibliographyEntry}

\begin{styleCitaviBibliographyEntry}
Diessel, Holger. 1999. \textit{Demonstratives: Form, function, and grammaticalization} (Typological Studies in Language 42). Amsterdam: John Benjamins Publishing Company.
\end{styleCitaviBibliographyEntry}

\begin{styleCitaviBibliographyEntry}
Diessel, Holger. 2006. Demonstratives. In Brown, Keith (ed.), \textit{Encyclopedia of language and linguistics}, 2\textsuperscript{nd} edn.. Amsterdam: Elsevier Science Ltd. 430–435.
\end{styleCitaviBibliographyEntry}

\begin{styleCitaviBibliographyEntry}
Dik, Simon C. \& Hengeveld, Kees. 1997. \textit{The theory of functional grammar. Part 2: Complex and derived constructions} (Functional Grammar Series 21), 2\textsuperscript{nd} edn. Berlin: Mouton de Gruyter.
\end{styleCitaviBibliographyEntry}

\begin{styleCitaviBibliographyEntry}
Dixon, Robert M. W. 1979. Ergativity. \textit{Language} 55: 59–138. Online URL: \url{http://www.jstor.org/stable/412519} (Accessed 8 Jan 2016).
\end{styleCitaviBibliographyEntry}

\begin{styleCitaviBibliographyEntry}
Dixon, Robert M. W. 1994. Adjectives. In Asher, Robert E. (ed.), \textit{The Encyclopedia of language and linguistics}. Oxford: Pergamon Press, 29–34.
\end{styleCitaviBibliographyEntry}

\begin{styleCitaviBibliographyEntry}
Dixon, Robert M. W. 2004. Adjective classes in typological perspective. In Dixon, Robert M. W. \& Aikhenvald, Alexandra Y. (eds.), \textit{Adjective classes: A cross-linguistic typology} (Explorations in Linguistic Typology 1). Oxford: Oxford University Press, 1–49.
\end{styleCitaviBibliographyEntry}

\begin{styleCitaviBibliographyEntry}
Dixon, Robert M. W. 2008. Comparative constructions: A cross-linguistic typology. \textit{Studies in Language} 32(4): 787–817. Online URL: \url{http://dx.doi.org/10.1075/sl.32.4.02dix} (Accessed 8 Jan 2016).
\end{styleCitaviBibliographyEntry}

\begin{styleCitaviBibliographyEntry}
Dixon, Robert M. W. 2010a. \textit{Basic linguistic theory. Volume 2: Grammatical topics}. Oxford: Oxford University Press.
\end{styleCitaviBibliographyEntry}

\begin{styleCitaviBibliographyEntry}
Dixon, Robert M. W. 2010b. \textit{Basic linguistic theory. Volume 3: Further grammatical topics}. Oxford: Oxford University Press.
\end{styleCitaviBibliographyEntry}

\begin{styleCitaviBibliographyEntry}
Dixon, Robert M. W. \& Aikhenvald, Alexandra Y. 2009. \textit{The semantics of clause linking: A cross-linguistic typology} (Explorations in Linguistic Typology 5). Oxford: Oxford University Press.
\end{styleCitaviBibliographyEntry}

\begin{styleCitaviBibliographyEntry}
Doetjes, Jenny. 2007. Adverbs and quantification: Degrees versus frequency. \textit{Lingua} 117: 685–720. Online URL: \url{http://dx.doi.org/10.1016/j.lingua.2006.04.003} (Accessed 8 Jan 2016).
\end{styleCitaviBibliographyEntry}

\begin{styleCitaviBibliographyEntry}
Donohue, Mark. 1997. \ili{Merauke Malay}: Some observations on contact and change. Manchester: Department of Linguistics, University of Manchester.
\end{styleCitaviBibliographyEntry}

\begin{styleCitaviBibliographyEntry}
Donohue, Mark. 2003. Papuan Malay. Singapore: National University of Singapore. Online URL: \url{http://papuan.linguistics.anu.edu.au/Donohue/downloads/Donohue_2003_PapuanMalay.pdf} (Accessed 8 Jan 2016).
\end{styleCitaviBibliographyEntry}

\begin{styleCitaviBibliographyEntry}
Donohue, Mark. 2005a. Numerals and their position in Universal Grammar. \textit{Journal of Universal Language} 6(2): 1–37.
\end{styleCitaviBibliographyEntry}

\begin{styleCitaviBibliographyEntry}
Donohue, Mark. 2005b. Voice in some \ili{eastern Malay varieties}. Paper presented at the Ninth International Symposium on Malay/Indonesian Linguistics – ISMIL 9. Ambun Pagi, 27-29 July 2005. Online URL: \url{http://email.eva.mpg.de/~gil/ismil/9/abstracts.zip} (Accessed 8 Jan 2016).
\end{styleCitaviBibliographyEntry}

\begin{styleCitaviBibliographyEntry}
Donohue, Mark. 2007a. Malay as a mirror of \ili{Austronesian}: Voice development and voice \isi{variation}. \textit{Lingua} 118(10): 1470–1499. Online URL: \url{http://dx.doi.org/10.1016/j.lingua.2007.08.007} (Accessed 8 Jan 2016).
\end{styleCitaviBibliographyEntry}

\begin{styleCitaviBibliographyEntry}
Donohue, Mark. 2007b. Variation in voice in Indonesian/Malay: Historical and synchronic perspectives. In Matsumoto, Yoshiko \& Oshima, David Y. \& Robinson, Orrin R. \& Sells, Peter (eds.), \textit{Diversity in language: Perspectives and implications} (CSLI Lecture Notes 176). Stanford: CSLI Publications, 71–129.
\end{styleCitaviBibliographyEntry}

\begin{styleCitaviBibliographyEntry}
Donohue, Mark. 2007c. Word order in \ili{Austronesian} from north to south and west to east. \textit{Linguistic Typology} 11(2): 349–391. Online URL: \url{http://dx.doi.org/10.1515/LINGTY.2007.026} (Accessed 8 Jan 2016).
\end{styleCitaviBibliographyEntry}

\begin{styleCitaviBibliographyEntry}
Donohue, Mark. 2011. Papuan Malay of New Guinea: \ili{Melanesian} influence on \isi{verb} and clause structure. In Lefebvre, Claire (ed.), \textit{Creoles, their substrates, and language typology} (Typological Studies in Language 95). Amsterdam: John Benjamins Publishing Company, 413–435.
\end{styleCitaviBibliographyEntry}

\begin{styleCitaviBibliographyEntry}
Donohue, Mark \& Grimes, Charles E. 2008. Yet more on the position of the languages of Eastern Indonesia and East Timor. \textit{\ili{Oceanic} Linguistics} 47(1): 114–158. Online URL: \url{http://www.jstor.org/stable/20172341} (Accessed 8 Jan 2016).
\end{styleCitaviBibliographyEntry}

\begin{styleCitaviBibliographyEntry}
Donohue, Mark \& Sawaki, Yusuf W. 2007. Papuan Malay pronominals: Forms and functions. \textit{\ili{Oceanic} Linguistics} 46(1): 253–276. Online URL: \url{http://www.jstor.org/stable/4499988} (Accessed 8 Jan 2016).
\end{styleCitaviBibliographyEntry}

\begin{styleCitaviBibliographyEntry}
Donohue, Mark \& Smith, John C. 1998. What’s happened to us? Some developments in the Malay \isi{pronoun} system. \textit{\ili{Oceanic} Linguistics} 37(1): 65–84. Online URL: \url{http://www.jstor.org/stable/3623280} (Accessed 8 Jan 2016).
\end{styleCitaviBibliographyEntry}

\begin{styleCitaviBibliographyEntry}
Dooley, Robert A. \& Levinsohn, Stephen H. 2001. \textit{Analyzing discourse: A manual of basic concepts}. Dallas: SIL International.
\end{styleCitaviBibliographyEntry}

\begin{styleCitaviBibliographyEntry}
Dryer, Matthew S. 2007a. Clause types. In Shopen, Timothy (ed.), \textit{Language typology and syntactic description. Volume 1: Clause structure}, 2\textsuperscript{nd} edn.. Cambridge: Cambridge University Press, 224–275.
\end{styleCitaviBibliographyEntry}

\begin{styleCitaviBibliographyEntry}
Dryer, Matthew S. 2007b. Noun phrase structure. In Shopen, Timothy (ed.), \textit{Language typology and syntactic description. Volume 2: Complex constructions}, 2\textsuperscript{nd} edn.. Cambridge: Cambridge University Press, 151–205.
\end{styleCitaviBibliographyEntry}

\begin{styleCitaviBibliographyEntry}
Dryer, Matthew S. 2007c. Word order. In Shopen, Timothy (ed.), \textit{Language typology and syntactic description. Volume 1: Clause structure}, 2\textsuperscript{nd} edn.. Cambridge: Cambridge University Press, 61–131.
\end{styleCitaviBibliographyEntry}

\begin{styleCitaviBibliographyEntry}
Dumont d‘Urville, Jules-Sébastien-César. 1833. \textit{Voyage de d\'{e}couvertes autour du monde et \`{a} la recherche de La P\'{e}rouse: Ex\'{e}cut\'{e} sous son commandement et par ordre du gouvernement, sur la corvette l’Astrolabe, pendant les ann\'{e}es 1826, 1827, 1828 et 1829. Histoire du voyage, Tome Quatrième}. Paris: La Librairie Encyclopédique de Roret. Online URL: \url{http://books.google.nl/books/download/Voyage_de_découvertes_autour_du_monde_e.pdf?id=0eg_AAAAcAAJ & output=pdf & sig=ACfU3U1rV1mJTLlb0VUe2dkx1ftK0l7nNQ} (Accessed 8 Jan 2016).
\end{styleCitaviBibliographyEntry}

\begin{styleCitaviBibliographyEntry}
Embassy of the Republic of Indonesia in London. 2009. Issues and perspectives. London: Embassy of the Republic of Indonesia in London. Online URL: \url{http://www.indonesianembassy.org.uk/transmigration-1.htm} (Accessed 8 Jan 2016).
\end{styleCitaviBibliographyEntry}

\begin{styleCitaviBibliographyEntry}
Encyclopædia Britannica Inc. 2001a-. New Guinea (island, Malay Archipelago). In Encyclopædia Britannica Inc. (ed.), \textit{Encyclopædia Britannica online}. Chicago: Encyclopædia Britannica. Online URL: \url{http://www.britannica.com/EBchecked/topic/411548/New-Guinea} (Accessed 8 Jan 2016).
\end{styleCitaviBibliographyEntry}

\begin{styleCitaviBibliographyEntry}
Encyclopædia Britannica Inc. 2001b-. Papua (province, Indonesia). In Encyclopædia Britannica Inc. (ed.), \textit{Encyclopædia Britannica online}. Chicago: Encyclopædia Britannica. Online URL: \url{http://www.britannica.com/EBchecked/topic/293960/Papua} (Accessed 8 Jan 2016).
\end{styleCitaviBibliographyEntry}

\begin{styleCitaviBibliographyEntry}
Englebretson, Robert. 2003. \textit{Searching for structure: The problem of complementation in colloquial Indonesian conversation} (Studies in Discourse and Grammar 13). Amsterdam: John Benjamins Publishing Company.
\end{styleCitaviBibliographyEntry}

\begin{styleCitaviBibliographyEntry}
Englebretson, Robert. 2007. Grammatical resources for social purposes: Some aspects of stancetaking in colloquial Indonesian conversation. In Englebretson, Robert (ed.), \textit{Stancetaking in discourse: Subjectivity, evaluation, interaction} (Pragmatics \& Beyond 164). Amsterdam: John Benjamins Publishing Company, 69–110.
\end{styleCitaviBibliographyEntry}

\begin{styleCitaviBibliographyEntry}
Errington, Joseph. 2001. Colonial linguistics. \textit{Annual Review of Anthropology} 30: 19–39. Online URL: \url{http://dx.doi.org/10.1146/annurev.anthro.30.1.19} (Accessed 8 Jan 2016).
\end{styleCitaviBibliographyEntry}

\begin{styleCitaviBibliographyEntry}
Fasold, Ralph W. 1984. \textit{The sociolinguistics of society} (Language in Society 5). Oxford: Basil Blackwell Publishers.
\end{styleCitaviBibliographyEntry}

\begin{styleCitaviBibliographyEntry}
Fearnside, Philip M. 1997. Transmigration in Indonesia: Lessons from its environmental and social impacts. \textit{Environmental Management} 21(4): 553–570. Online URL: \url{https://www.academia.edu/1196557/Transmigration_in_Indonesia_Lessons_from_its_environmental_and_social_impacts} (Accessed 8 Jan 2016).
\end{styleCitaviBibliographyEntry}

\begin{styleCitaviBibliographyEntry}
Ferguson, Charles A. 1972. Diglossia. In Giglioli, Pier P. (ed.), \textit{Language and social context: Selected readings}. Harmondsworth: Penguin Press, 232–251 [Reprint of Word 15: 325-340 (1959)].
\end{styleCitaviBibliographyEntry}

\begin{styleCitaviBibliographyEntry}
Filimonova, Elena. 2005. \textit{Clusivity: Typology and case studies of inclusive-exclusive distinction} (Typological Studies in Language 63). Amsterdam: John Benjamins Publishing Company.
\end{styleCitaviBibliographyEntry}

\begin{styleCitaviBibliographyEntry}
Fishman, Joshua A. 1965. Who speaks what language to whom and when? \textit{La Linguistique} 1(2): 67–88.
\end{styleCitaviBibliographyEntry}

\begin{styleCitaviBibliographyEntry}
Foley, William A. 1986. \textit{The \ili{Papuan languages} of New Guinea} (Cambridge Language Surveys). Cambridge: Cambridge University Press.
\end{styleCitaviBibliographyEntry}

\begin{styleCitaviBibliographyEntry}
Foley, William A. 2000. The languages of New Guinea. \textit{Annual Review of Anthropology} 29: 357–404. Online URL: \url{http://dx.doi.org/10.1146/annurev.anthro.29.1.357} (Accessed 8 Jan 2016).
\end{styleCitaviBibliographyEntry}

\begin{styleCitaviBibliographyEntry}
Gal, Susan \& Irvine, Judith T. 1995. The boundaries of languages and disciplines: How ideologies construct difference. \textit{Social Research} 62(4): 967–1001. Online URL: \url{http://www.jstor.org/stable/40971131} (Accessed 8 Jan 2016).
\end{styleCitaviBibliographyEntry}

\begin{styleCitaviBibliographyEntry}
Gil, David. 1994. The structure of \ili{Riau Indonesian}. \textit{Nordic Journal of Linguistics} 17: 179–200.
\end{styleCitaviBibliographyEntry}

\begin{styleCitaviBibliographyEntry}
Gil, David. 1999. The \isi{grammaticalization} of punya in Malay/Indonesian dialects. Paper presented at the Ninth Annual Meeting of the South-East Asian Linguistics Society. University of California, Berkeley, 22 May.
\end{styleCitaviBibliographyEntry}

\begin{styleCitaviBibliographyEntry}
Gil, David. 2001a. Commentary: Creoles, complexity, and \ili{Riau Indonesian}. \textit{Linguistic Typology} 5(2-3): 325–371. Online URL: \url{http://dx.doi.org/10.1515/lity.2001.002} (Accessed 8 Jan 2016).
\end{styleCitaviBibliographyEntry}

\begin{styleCitaviBibliographyEntry}
Gil, David. 2001b. Quantifiers. In Haspelmath, Martin \& König, Ekkehard \& Oesterreicher, Wulf \& Raible, Wolfgang (eds.), \textit{Language typology and language universals: An international handbook. Volume 2} (Handbücher zur Sprach- und Kommunikationswissenschaft 20). Berlin: Walter de Gruyter, 1275–1294.
\end{styleCitaviBibliographyEntry}

\begin{styleCitaviBibliographyEntry}
Gil, David. 2009. Associative plurals and inclusories in Malay/Indonesian. Paper presented at the Thirteenth Symposium on Malay and Indonesian Linguistics – ISMIL 13. Senggigi, 6-7 June 2009. Online URL: \url{http://email.eva.mpg.de/~gil/ismil/13/abstracts/Gil abstract ISMIL 13.pdf} (Accessed 8 Jan 2016).
\end{styleCitaviBibliographyEntry}

\begin{styleCitaviBibliographyEntry}
Gil, David. 2011. Conjunctions and universal quantifiers. In Haspelmath, Martin \& Dryer, Matthew S. \& Gil, David \& Comrie, Bernard (eds.), \textit{The world atlas of language structures}. München: Max Planck Digital Library, 1–4. Online URL: \url{http://wals.info/chapter/56} (Accessed 8 Jan 2016).
\end{styleCitaviBibliographyEntry}

\begin{styleCitaviBibliographyEntry}
Gil, David. 2013. \ili{Riau Indonesian}: A language without nouns and verbs. In Rijkhoff, Jan \& van Lier, Eva (eds.), \textit{Flexible word classes: Typological studies of underspecified parts of speech}. Oxford: Oxford University Press.
\end{styleCitaviBibliographyEntry}

\begin{styleCitaviBibliographyEntry}
Gil, David. 2014. Tanah Papua (Diversity Linguistics Comment - Language Structures throughout the World 655). Marseille: hypothesis. Online URL: \url{http://dlc.hypotheses.org/655} (Accessed 8 Jan 2016).
\end{styleCitaviBibliographyEntry}

\begin{styleCitaviBibliographyEntry}
Gil, David \& Tadmor, Uri. 1997. Towards a typology of Malay/Indonesian dialects. Paper presented at the First Symposium on Malay and Indonesian Linguistics – ISMIL 1. Penang, 14-15 January 1997. Online URL: \url{http://www.udel.edu/pcole/penang/abstracts.html#Towards a Typology of Malay/Ind} (Accessed 8 Jan 2016).
\end{styleCitaviBibliographyEntry}

\begin{styleCitaviBibliographyEntry}
Givón, Talmy. 2001. \textit{Syntax: An introduction. Volume 1}. Amsterdam: John Benjamins Publishing Company.
\end{styleCitaviBibliographyEntry}

\begin{styleCitaviBibliographyEntry}
Goodman, Thomas. 2002. The Rajas of Papua and East Seram during the early modern period (17\textsuperscript{th} – 18\textsuperscript{th} centuries): A bibliographic essay. \textit{Papuaweb’s Annotated Bibliographies}: 11 p. Online URL: \url{http://www.papuaweb.org/bib/abib/goodman.pdf} (Accessed 8 Jan 2016).
\end{styleCitaviBibliographyEntry}

\begin{styleCitaviBibliographyEntry}
Greenberg, Joseph H. 1978. Generalizations about numerals systems. In Greenberg, Joseph H. \& Ferguson, Charles A. \& Moravcsik, Andrew (eds.), \textit{Universals of human language. Volume 3: Word structure}. Stanford: Stanford University Press, 249–295.
\end{styleCitaviBibliographyEntry}

\begin{styleCitaviBibliographyEntry}
Grimes, Barbara D. 1991. The development and use of Ambonese Malay. In Steinhauer, Hein (ed.), \textit{Papers in \ili{Austronesian} linguistics. Volume 1} (Pacific Linguistics A-81). Canberra: Research School of Pacific Studies, The Australian National University, 83–123.
\end{styleCitaviBibliographyEntry}

\begin{styleCitaviBibliographyEntry}
Grimes, Charles E. \& Jacob, June A. (eds.) 2008. \textit{\ili{Kupang Malay} Online Dictionary}. Kupang: UBB-GMIT. Online URL: \href{http://e-kamus2.org/Kupang Malay Lexicon/lexicon/main.htm}{http://e-kamus2.org/Kupang Malay Lexicon/lexicon/main.htm} (Accessed 8 Jan 2016).
\end{styleCitaviBibliographyEntry}

\begin{styleCitaviBibliographyEntry}
Grimes, Joseph E. 1975. \textit{The thread of discourse} (Janua Linguarum: Series Minor 207). The Hague: Mouton de Gruyter.
\end{styleCitaviBibliographyEntry}

\begin{styleCitaviBibliographyEntry}
Grosjean, François. 1992. Another view of bilingualism. In Harris, Richard J. (ed.), \textit{Cognitive processing in bilinguals} (Advances in Psychology 83). Amsterdam: Elsevier Science Ltd. 51–62.
\end{styleCitaviBibliographyEntry}

\begin{styleCitaviBibliographyEntry}
Haga, A. 1884. \textit{Nederlandsch Nieuw Guinea en de Papoesche Eilanden: Historische Bijdrage. Eerste Deel ±1500-1817}. Batavia: W. Bruining \& Company. Online URL: \url{https://ia600300.us.archive.org/25/items/nederlandschnie00wetegoog/nederlandschnie00wetegoog.pdf} (Accessed 8 Jan 2016).
\end{styleCitaviBibliographyEntry}

\begin{styleCitaviBibliographyEntry}
Haga, A. 1885. Het rapport van H. Zwaardecroon en C. Chasteleijn betreffende de reis naar Nieuw Guinea in 1705 ondernomen door Jacob Weyland. \textit{Tijdschrift voor Indische Taal-, Land- en Volkenkunde} 30: 235–258.
\end{styleCitaviBibliographyEntry}

\begin{styleCitaviBibliographyEntry}
Hale, Kenneth L. 1973. Person marking in Walbiri. In Anderson, Stephen R. \& Kiparsky, Paul (eds.), \textit{A festschrift for Morris Halle}. New York: Holt, Rinehardt and Winston, 308–344.
\end{styleCitaviBibliographyEntry}

\begin{styleCitaviBibliographyEntry}
Hartanti. 2008. A sociolinguistics analysis on SMS texts on Papuan Malay: A case study of students SMS texts of semester VIII of faculty of letters. Manokwari: Universitas Negeri Papua.
\end{styleCitaviBibliographyEntry}

\begin{styleCitaviBibliographyEntry}
Haser, Verena \& Kortmann, Bernd. 2006. Adverbs. In Brown, Keith (ed.), \textit{Encyclopedia of language and linguistics}, 2\textsuperscript{nd} edn.. Amsterdam: Elsevier Science Ltd. 66–69.
\end{styleCitaviBibliographyEntry}

\begin{styleCitaviBibliographyEntry}
Haspelmath, Martin. 2004. Coordinating constructions: An overview. In Haspelmath, Martin (ed.), \textit{Coordinating constructions} (Typological Studies in Language 58). Amsterdam: John Benjamins Publishing Company, 3–39.
\end{styleCitaviBibliographyEntry}

\begin{styleCitaviBibliographyEntry}
Haspelmath, Martin. 2007a. Coordination. In Shopen, Timothy (ed.), \textit{Language typology and syntactic description. Volume 2: Complex constructions}, 2\textsuperscript{nd} edn.. Cambridge: Cambridge University Press, 1–51.
\end{styleCitaviBibliographyEntry}

\begin{styleCitaviBibliographyEntry}
Haspelmath, Martin. 2007b. Ditransitive alignment splits and inverse alignment. \textit{Functions of Language} 14(1): 79–102. Online URL: \url{http://dx.doi.org/10.1075/fol.14.1.06has} (Accessed 8 Jan 2016).
\end{styleCitaviBibliographyEntry}

\begin{styleCitaviBibliographyEntry}
Haspelmath, Martin. 2007c. Further remarks on reciprocal constructions. In Nedjalkov, Vladimir P. (ed.), \textit{Reciprocal constructions (Volume 4)} (Typological Studies in Language 71). Philadelphia: John Benjamins Publishing Company, 2087–2115.
\end{styleCitaviBibliographyEntry}

\begin{styleCitaviBibliographyEntry}
Hawkins, John A. 1983. \textit{Word order universals} (Quantitative Analyses of Linguistic Structure 3). New York: Academic Press.
\end{styleCitaviBibliographyEntry}

\begin{styleCitaviBibliographyEntry}
Hay, Jennifer. 2001. Lexical frequency in \isi{morphology}: Is everything relative? \textit{Linguistics} 39(6): 1041–1070. Online URL: \url{http://dx.doi.org/10.1515/ling.2001.041} (Accessed 8 Jan 2016).
\end{styleCitaviBibliographyEntry}

\begin{styleCitaviBibliographyEntry}
Hay, Jennifer \& Baayen, R. Harald. 2002. Parsing and productivity. In Booij, Geert E. \& van Marle, Jaap (eds.), \textit{Yearbook of \isi{morphology} 2001}. Dordrecht: Kluwer Academic Publishers, 203–235. Online URL: \url{http://www.sfs.uni-tuebingen.de/~hbaayen/publications/HayBaayenYoM2002.pdf} (Accessed 8 Jan 2016).
\end{styleCitaviBibliographyEntry}

\begin{styleCitaviBibliographyEntry}
Hayashi, Makoto \& Yoon, Kyung-Eun. 2006. A cross-linguistic exploration of demonstratives in interaction: With particular reference to the context of word-formulation trouble. \textit{Studies in Language} 30(3): 485–540. Online URL: \url{http://dx.doi.org/10.1075/sl.30.3.02hay} (Accessed 8 Jan 2016).
\end{styleCitaviBibliographyEntry}

\begin{styleCitaviBibliographyEntry}
Heine, Bernd \& Kuteva, Tania. 2002. \textit{World lexicon of grammaticalization}. Cambridge: Cambridge University Press.
\end{styleCitaviBibliographyEntry}

\begin{styleCitaviBibliographyEntry}
Helmbrecht, Johannes. 2004. Personal pronouns – Form, function, and \isi{grammaticalization}. Erfurt: University of Erfurt. (Habitilationsschrift.)
\end{styleCitaviBibliographyEntry}

\begin{styleCitaviBibliographyEntry}
Helmbrecht, Johannes. 2011. Politeness distinctions in pronouns. In Haspelmath, Martin \& Dryer, Matthew S. \& Gil, David \& Comrie, Bernard (eds.), \textit{The world atlas of language structures}. München: Max Planck Digital Library, Chapter 45. Online URL: \url{http://wals.info/chapter/45} (Accessed 14 Dec 2013).
\end{styleCitaviBibliographyEntry}

\begin{styleCitaviBibliographyEntry}
Hengeveld, Kees. 1992. \textit{Non-verbal predication: Theory, typology, diachrony} (Functional Grammar Series 15). Berlin: Mouton de Gruyter.
\end{styleCitaviBibliographyEntry}

\begin{styleCitaviBibliographyEntry}
Himmelmann, Nikolaus P. 1991. \textit{The Philippine challenge to Universal Grammar} (Arbeitspapier 15). Köln: Institut für Sprachwissenschaft, Universität Köln. Online URL: \url{http://publikationen.ub.uni-frankfurt.de/files/24331/AP15NF-Himmelmann(1991).pdf} (Accessed 8 Jan 2016).
\end{styleCitaviBibliographyEntry}

\begin{styleCitaviBibliographyEntry}
Himmelmann, Nikolaus P. 1996. Demonstratives in narrative discourse: A taxonomy of universal uses. In Fox, Barbara A. (ed.), \textit{Studies in anaphora} (Typological Studies in Language 33). Amsterdam: John Benjamins Publishing Company, 205–255.
\end{styleCitaviBibliographyEntry}

\begin{styleCitaviBibliographyEntry}
Himmelmann, Nikolaus P. 2005. The \ili{Austronesian} languages of Asia and Madagascar: Typological characteristics. In Adelaar, K. A. \& Himmelmann, Nikolaus P. (eds.), \textit{The \ili{Austronesian} languages of Asia and Madagascar} (Routledge Language Family Series). London: Routledge, 110–181.
\end{styleCitaviBibliographyEntry}

\begin{styleCitaviBibliographyEntry}
Himmelmann, Nikolaus P. 2008. Lexical categories and voice in \ili{Tagalog}. In Austin, Peter K. \& Musgrave, Simon (eds.), \textit{Voice and grammatical relations in \ili{Austronesian} languages} (Studies in Constraint-Based Lexicalism). Stanford: Center for the Study of Language and Information, 247–293. Online URL: \url{http://www.uni-muenster.de/imperia/md/content/allgemeine_sprachwissenschaft/dozenten-unterlagen/himmelmann/himmelmann_lexical_categories_and_voice_in_tagalog.pdf} (Accessed 8 Jan 2016).
\end{styleCitaviBibliographyEntry}

\begin{styleCitaviBibliographyEntry}
Hoeksema, Jack \& Zwarts, Frans. 1991. Some remarks on focus adverbs. \textit{Journal of Semantics}(8): 51–70. Online URL: \url{http://dx.doi.org/10.1093/jos/8.1-2.51} (Accessed 8 Jan 2016).
\end{styleCitaviBibliographyEntry}

\begin{styleCitaviBibliographyEntry}
Huizinga, F. 1998. Relations between Tidore and the north coast of New Guinea in the nineteenth century. In Miedema, Jelle \& Ode, Cecilia \& Dam, Rien A. C. (eds.), \textit{Perspectives on the Bird’s Head of Irian Jaya, Indonesia: Proceedings of the Conference, Leiden, 13-17 October 1997}. Amsterdam: Rodopi, 385–419.
\end{styleCitaviBibliographyEntry}

\begin{styleCitaviBibliographyEntry}
Hymes, Dell H. 1974. \textit{Foundations in sociolinguistics: An ethnographic approach}. Philadelphia: University of Pennsylvania Press.
\end{styleCitaviBibliographyEntry}

\begin{styleCitaviBibliographyEntry}
Jacob, June A. \& Grimes, Barbara D. 2006. Developing a role for \ili{Kupang Malay}: The contemporary politics of an eastern Indonesian \ili{creole}. Paper presented at the Tenth International Conference on \ili{Austronesian} Linguistics (10-ICAL). Puerto Princesa City, 17-20 January 2006. Online URL: \url{http://www.sil.org/asia/philippines/ical/papers/Jacob-Grimes \ili{Kupang Malay}.pdf} (Accessed 8 Jan 2016).
\end{styleCitaviBibliographyEntry}

\begin{styleCitaviBibliographyEntry}
Jacob, June A. \& Grimes, Charles E. 2011. Aspect and directionality in \ili{Kupang Malay} serial \isi{verb} constructions: Calquing on the grammars of substrate languages. In Lefebvre, Claire (ed.), \textit{Creoles, their substrates, and language typology} (Typological Studies in Language 95). Amsterdam: John Benjamins Publishing Company, 337–366.
\end{styleCitaviBibliographyEntry}

\begin{styleCitaviBibliographyEntry}
Johannessen, Janne B. 2006. Just any \isi{pronoun} anywhere? Pronouns and “new” demonstratives in \ili{Norwegian}. In Solstad, Torgrim \& Grønn, Atle \& Dag, Haug (eds.), \textit{A festschrift for Kjell Johan Sæbø}. Oslo: University of Oslo, 91–106.
\end{styleCitaviBibliographyEntry}

\begin{styleCitaviBibliographyEntry}
Jones, Russell (ed.) 2007. \textit{Loan-words in Indonesian and Malay}. Leiden: KITLV Press. Online URL: \url{http://sealang.net/indonesia/lwim/} (Accessed 8 Jan 2016).
\end{styleCitaviBibliographyEntry}

\begin{styleCitaviBibliographyEntry}
Jurafsky, Dan. 1993. Universals in the semantics of the diminutive. In Guenter, Joshua S. \& Kaiser, Barbara A. \& Zoll, Cheryl C. \& Berkeley Linguistics Society (eds.), \textit{Proceedings of the nineteenth annual meeting of the Berkeley Linguistics Society, 12-15 February, 1993: General session and parasession on semantic typology and semantic universals}. Berkeley: Berkeley Linguistics Society, 423–436. Online URL: \url{http://dx.doi.org/10.3765/bls.v19i1.1531}; \url{http://journals.linguisticsociety.org/proceedings/index.php/BLS/article/view/1531/1314} (Accessed 8 Jan 2016).
\end{styleCitaviBibliographyEntry}

\begin{styleCitaviBibliographyEntry}
Kacandes, Irene. 1994. Narrative \isi{apostrophe}: Reading, rhetoric, resistance in Michel Butor’s ‘La \isi{modification}’ and Julio Cortazar’s “Graffiti” (Second-Person Narrative). \textit{Style} 28(3): 329–349. Online URL: \url{http://www.jstor.org/stable/42946255} (Accessed 8 Jan 2016).
\end{styleCitaviBibliographyEntry}

\begin{styleCitaviBibliographyEntry}
Karam, Francis X. 2000. Investigating mutual intelligibility and language coalescence. \textit{International Journal of the Sociology of Language} 146: 119–136.
\end{styleCitaviBibliographyEntry}

\begin{styleCitaviBibliographyEntry}
Kaufmann, Stefan. 2006. Conditionals. In Brown, Keith (ed.), \textit{Encyclopedia of language and linguistics}, 2\textsuperscript{nd} edn.. Amsterdam: Elsevier Science Ltd. 6–9.
\end{styleCitaviBibliographyEntry}

\begin{styleCitaviBibliographyEntry}
Keenan, Edward L. \& Comrie, Bernard. 1977. Noun phrase accessibility and Universal Grammar. \textit{Linguistic Inquiry} 8(1): 63–99. Online URL: \url{http://www.jstor.org/stable/4177973} (Accessed 8 Jan 2016).
\end{styleCitaviBibliographyEntry}

\begin{styleCitaviBibliographyEntry}
Kelman, Herbert C. 1971. Language as an aid and barrier to the involvement in the national system. In Rubin, Joan \& Jernudd, Björn H. (eds.), \textit{Can language be planned? Sociolinguistic theory and practice for developing nations.}. Honolulu: University of Hawai’i Press, 21–51.
\end{styleCitaviBibliographyEntry}

\begin{styleCitaviBibliographyEntry}
Kemmer, Suzanne. 1993. \textit{The middle voice}. Amsterdam: John Benjamins Publishing Company.
\end{styleCitaviBibliographyEntry}

\begin{styleCitaviBibliographyEntry}
Kennedy, Christopher. 2006. Comparatives, semantics. In Brown, Keith (ed.), \textit{Encyclopedia of language and linguistics}, 2\textsuperscript{nd} edn.. Amsterdam: Elsevier Science Ltd. 690–697.
\end{styleCitaviBibliographyEntry}

\begin{styleCitaviBibliographyEntry}
Kenstowicz, Michael J. 1994. \textit{Phonology in generative grammar}. Cambridge MA: Basil Blackwell Publishers.
\end{styleCitaviBibliographyEntry}

\begin{styleCitaviBibliographyEntry}
Kersten, J. P. F. 1948. \textit{Balische grammatica}. ‘s-Gravenhage: W. van Hoeve.
\end{styleCitaviBibliographyEntry}

\begin{styleCitaviBibliographyEntry}
Kilian-Hatz, Christa. 2006. Ideophones. In Brown, Keith (ed.), \textit{Encyclopedia of language and linguistics}, 2\textsuperscript{nd} edn.. Amsterdam: Elsevier Science Ltd. 508–512.
\end{styleCitaviBibliographyEntry}

\begin{styleCitaviBibliographyEntry}
Kim, Hyun \& Nussy, Christian G. \& Rumaropen, Ben E. W. \& Scott, Eleonora L. \& Scott, Graham R. 2007. A survey of Papuan Malay: An interim report. Paper presented at the Eleventh International Symposium on Malay/Indonesian Linguistics – ISMIL 11. Manokwari, 6-8 August 2007. Online URL: \url{http://email.eva.mpg.de/~gil/ismil/11/abstracts/KimShonRumaropen.pdf} (Accessed 8 Jan 2016).
\end{styleCitaviBibliographyEntry}

\begin{styleCitaviBibliographyEntry}
King, Peter. 2002. Morning Star Rising? Indonesia Raya and the New Papuan Nationalism. \textit{Indonesia} 73: 89–127. Online URL: \url{http://www.jstor.org/stable/3351470} (Accessed 8 Jan 2016).
\end{styleCitaviBibliographyEntry}

\begin{styleCitaviBibliographyEntry}
King, Peter. 2004. \textit{West Papua \& Indonesia since Suharto: Independence, autonomy or chaos? }Sydney: University of New South Wales Press.
\end{styleCitaviBibliographyEntry}

\begin{styleCitaviBibliographyEntry}
Kingsbury, Damien \& Aveling, Harry (eds.) 2002. \textit{Autonomy and disintegration Indonesia}. London: Routledge Curzon Press.
\end{styleCitaviBibliographyEntry}

\begin{styleCitaviBibliographyEntry}
Kiyomi, Setsuku. 2009. A new approach to \isi{reduplication}: A semantic study of \isi{noun} and \isi{verb} \isi{reduplication} in the \ili{Malayo-Polynesian} languages. \textit{Linguistics} 33(6): 1145–1168. Online URL: \url{http://dx.doi.org/10.1515/ling.1995.33.6.1145} (Accessed 8 Jan 2016).
\end{styleCitaviBibliographyEntry}

\begin{styleCitaviBibliographyEntry}
Klamer, Marian. 2002. Typical features of \ili{Austronesian} languages in Central/Eastern Indonesia. \textit{\ili{Oceanic} Linguistics} 41(2): 363–383. Online URL: \url{http://dx.doi.org/10.1353/ol.2002.0007} (Accessed 8 Jan 2016).
\end{styleCitaviBibliographyEntry}

\begin{styleCitaviBibliographyEntry}
Klamer, Marian \& Ewing, Michael C. 2010. The languages of East Nusantara: An introduction. In Ewing, Michael C. \& Klamer, Marian (eds.), \textit{East Nusantara: Typological and areal analysis} (Pacific Linguistics 618). Canberra: College of Asia and the Pacific, School of Culture, History and Language, Pacific Linguistics, The Australian National University, 1–25. Online URL: \url{https://openaccess.leidenuniv.nl/bitstream/handle/1887/18306/Klamer & Ewing 2010 in Ewing & Klamer (eds).pdf?sequence=2} (Accessed 8 Jan 2016).
\end{styleCitaviBibliographyEntry}

\begin{styleCitaviBibliographyEntry}
Klamer, Marian \& Moro, Francesca R. 2013. ‘Give’-constructions in Heritage and Baseline Malay. Paper presented at the Workshop Structural Change in Heritage Languages. Noordwijkerhout, 23-25 January.
\end{styleCitaviBibliographyEntry}

\begin{styleCitaviBibliographyEntry}
Klamer, Marian \& Reesink, Gerard P. \& van Staden, Miriam. 2008. East Nusantara as a linguistic area. In Muysken, Pieter (ed.), \textit{From linguistic areas to areal linguistics} (Studies in Language Companion Series 90). Amsterdam: John Benjamins Publishing Company, 95–151.
\end{styleCitaviBibliographyEntry}

\begin{styleCitaviBibliographyEntry}
Kluge, Angela \& Rumaropen, Ben E. W. \& Aweta, Lodowik. 2014. \textit{Papuan Malay data – Word list}. Dallas: SIL International. Online URL: \url{http://www.sil.org/resources/archives/59649} (Accessed 8 Jan 2016).
\end{styleCitaviBibliographyEntry}

\begin{styleCitaviBibliographyEntry}
Krishnamurthy, Ramesh. 2006. Collocations. In Brown, Keith (ed.), \textit{Encyclopedia of language and linguistics}, 2\textsuperscript{nd} edn.. Amsterdam: Elsevier Science Ltd. 596–600.
\end{styleCitaviBibliographyEntry}

\begin{styleCitaviBibliographyEntry}
Kroeger, Paul R. 2005. \textit{Analyzing grammar: An introduction}. Cambridge: Cambridge University Press.
\end{styleCitaviBibliographyEntry}

\begin{styleCitaviBibliographyEntry}
Kroeger, Paul R. 2012. External vs. internal \isi{negation} in Indonesian verbal clauses. Paper presented at the Twelfth International Conference on \ili{Austronesian} Linguistics (12ICAL). Denpasar, 2-6 July 2012.
\end{styleCitaviBibliographyEntry}

\begin{styleCitaviBibliographyEntry}
Krupa, Viktor. 1967. \textit{Jazyk Maori}. Moskva: Izdatel’stvo “Nauka”.
\end{styleCitaviBibliographyEntry}

\begin{styleCitaviBibliographyEntry}
Kulikov, Leonid I. 2001. Causatives. In Haspelmath, Martin \& König, Ekkehard \& Oesterreicher, Wulf \& Raible, Wolfgang (eds.), \textit{Language typology and language universals: An international handbook. Volume 1} (Handbücher zur Sprach- und Kommunikationswissenschaft 20). Berlin: Walter de Gruyter, 886–898.
\end{styleCitaviBibliographyEntry}

\begin{styleCitaviBibliographyEntry}
Lakoff, Robin. 1974. Remarks on ‘this’ and ‘that’. \textit{Papers from the Regional Meetings, Chicago Linguistic Society} 10: 345–356.
\end{styleCitaviBibliographyEntry}

\begin{styleCitaviBibliographyEntry}
Lass, Roger. 1984. \textit{Phonology: An introduction to basic concepts} (Cambridge Textbooks in Linguistics). Cambridge: Cambridge University Press.
\end{styleCitaviBibliographyEntry}

\begin{styleCitaviBibliographyEntry}
Levinson, Stephen C. \& Wilkins, David P. (eds.) 2006. \textit{Grammars of space: Explorations in cognitive diversity} (Language, Culture and Cognition 6). Cambridge: Cambridge University Press.
\end{styleCitaviBibliographyEntry}

\begin{styleCitaviBibliographyEntry}
Lewis, M. Paul \& Simons, Gary F. \& Fennig, Charles D. (eds.) 2016a. \textit{Ethnologue: Languages of the world, Nineteenth edition}. Dallas: SIL International. Online URL: \url{http://www.ethnologue.com/} (Accessed 14 Mar 2016).
\end{styleCitaviBibliographyEntry}

\begin{styleCitaviBibliographyEntry}
Lewis, M. Paul \& Simons, Gary F. \& Fennig, Charles D. 2016b. The problem of language identification. In Lewis, M. Paul \& Simons, Gary F. \& Fennig, Charles D. (eds.), \textit{Ethnologue: Languages of the world, Nineteenth edition}. Dallas: SIL International. Online URL: \url{http://www.ethnologue.com/about/problem-language-identification} (Accessed 14 Mar 2016).
\end{styleCitaviBibliographyEntry}

\begin{styleCitaviBibliographyEntry}
Lichtenberk, Frantisek. 2000. Inclusory pronominals. \textit{\ili{Oceanic} Linguistics} 39(1): 1–32. Online URL: \url{http://www.jstor.org/stable/3623215} (Accessed 8 Jan 2016).
\end{styleCitaviBibliographyEntry}

\begin{styleCitaviBibliographyEntry}
Lieber, Rochelle \& Štekauer, Pavol. 2009. Introduction: Status and definition of \isi{compounding}. In Lieber, Rochelle \& Štekauer, Pavol (eds.), \textit{The Oxford handbook of compounding} (Oxford Handbooks in Linguistics). Oxford: Oxford University Press, 1–18.
\end{styleCitaviBibliographyEntry}

\begin{styleCitaviBibliographyEntry}
Lim, Sonny. 1988. Baba Malay: The language of the ‘Straits-born’ \ili{Chinese}. In Lim, Sonny \& Soemarmo, Marmo P. K. \& Blust, Robert A. \& Kroeger, Paul R. (eds.), \textit{Papers in western \ili{Austronesian} linguistics, No. 3} (Pacific Linguistics A-78). Canberra: Research School of Pacific Studies, The Australian National University.
\end{styleCitaviBibliographyEntry}

\begin{styleCitaviBibliographyEntry}
Litamahuputty, Betty. 1994. The use of biking and kasi in Ambonese Malay. \textit{Cakalele} 5: 11–31. Online URL: \url{http://scholarspace.manoa.hawaii.edu/bitstream/handle/10125/4137/UHM.CSEAS.Cakalele.v5.Litama.pdf} (Accessed 8 Jan 2016).
\end{styleCitaviBibliographyEntry}

\begin{styleCitaviBibliographyEntry}
Litamahuputty, Betty. 2012. \textit{\ili{Ternate Malay}: Grammar and texts} (LOT Dissertation Series 307). Utrecht: LOT. Online URL: \url{http://www.lotpublications.nl/ternate-malay-ternate-malay-grammar-and-texts} (Accessed 8 Jan 2016).
\end{styleCitaviBibliographyEntry}

\begin{styleCitaviBibliographyEntry}
Loos, Eugene E. \& Anderson, Susan \& Day, Dwight H. \& Jordan, Paul C. \& Wingate, J. Douglas. 2003. \textit{Glossary of linguistic terms}. Dallas: SIL International Digital Resources. Online URL: \url{http://www.sil.org/linguistics/GlossaryOfLinguisticTerms/} (Accessed 8 Jan 2016).
\end{styleCitaviBibliographyEntry}

\begin{styleCitaviBibliographyEntry}
Lumi, Johnli H. 2007. The typology of plural personal pronouns in Papuan, Ambonese and \ili{Manado Malay}: Malay varieties of Eastern Indonesian. Paper presented at the Eleventh International Symposium on Malay/Indonesian Linguistics – ISMIL 11. Manokwari, 6-8 August 2007. Online URL: \url{http://email.eva.mpg.de/~gil/ismil/11/abstracts/Lumi.pdf} (Accessed 8 Jan 2016).
\end{styleCitaviBibliographyEntry}

\begin{styleCitaviBibliographyEntry}
Lyons, Christopher. 1999. \textit{Definiteness} (Cambridge Textbooks in Linguistics). Cambridge: Cambridge University Press.
\end{styleCitaviBibliographyEntry}

\begin{styleCitaviBibliographyEntry}
Lyons, John. 1977. \textit{Semantics. Volume 2}. Cambridge: Cambridge University Press.
\end{styleCitaviBibliographyEntry}

\begin{styleCitaviBibliographyEntry}
MacDonald, R. R. 1976. \textit{Indonesian reference grammar}. Washington D.C.: Georgetown University Press.
\end{styleCitaviBibliographyEntry}

\begin{styleCitaviBibliographyEntry}
Maddieson, Ian. 2011a. Absence of common consonants. In Haspelmath, Martin \& Dryer, Matthew S. \& Gil, David \& Comrie, Bernard (eds.), \textit{The world atlas of language structures}. München: Max Planck Digital Library, Chapter 18. Online URL: \url{http://wals.info/chapter/18} (Accessed 8 Jan 2016).
\end{styleCitaviBibliographyEntry}

\begin{styleCitaviBibliographyEntry}
Maddieson, Ian. 2011b. Syllable structure. In Haspelmath, Martin \& Dryer, Matthew S. \& Gil, David \& Comrie, Bernard (eds.), \textit{The world atlas of language structures}. München: Max Planck Digital Library, Chapter 12. Online URL: \url{http://wals.info/chapter/12} (Accessed 8 Jan 2016).
\end{styleCitaviBibliographyEntry}

\begin{styleCitaviBibliographyEntry}
Malchukov, Andrej L. \& Haspelmath, Martin \& Comrie, Bernard. 2010. Ditransitive constructions: A typological overview. In Malchukov, Andrej L. \& Haspelmath, Martin \& Comrie, Bernard (eds.), \textit{Studies in ditransitive constructions: A comparative handbook}. Berlin: Mouton de Gruyter, 1–63.
\end{styleCitaviBibliographyEntry}

\begin{styleCitaviBibliographyEntry}
Marantz, Alec. 1982. Re \isi{reduplication}. \textit{Linguistic Inquiry} 13(3): 435–482. Online URL: \url{http://www.jstor.org/stable/4178287} (Accessed 8 Jan 2016).
\end{styleCitaviBibliographyEntry}

\begin{styleCitaviBibliographyEntry}
Margetts, Anna \& Austin, Peter K. 2007. Three-participant events in the languages of the world: Towards a crosslinguistic typology. \textit{Linguistics} 45(3): 393–451. Online URL: \url{http://dx.doi.org/10.1515/LING.2007.014} (Accessed 8 Jan 2016).
\end{styleCitaviBibliographyEntry}

\begin{styleCitaviBibliographyEntry}
Masinambow, Eduard K. M. \& Haenen, Paul. 2002. \textit{Bahasa Indonesia dan bahasa daerah}. Jakarta: Yayasan Obor Indonesia.
\end{styleCitaviBibliographyEntry}

\begin{styleCitaviBibliographyEntry}
Mattes, Veronika. 2007. Types of \isi{reduplication}: A case study of Bikol. Graz: Karl-Franzens-Universität Graz. (PhD dissertation.) Online URL: \url{http://\isi{reduplication}.uni-graz.at/texte/Dissertation_gesamt.pdf} (Accessed 8 Jan 2016).
\end{styleCitaviBibliographyEntry}

\begin{styleCitaviBibliographyEntry}
Matushansky, Ora. 2008. On the linguistic complexity of proper names. \textit{Linguistics and Philosophy} 31(5): 573–627. Online URL: \url{http://dx.doi.org/10.1007/s10988-008-9050-1} (Accessed 8 Jan 2016).
\end{styleCitaviBibliographyEntry}

\begin{styleCitaviBibliographyEntry}
McWhorter, John H. 2001. The worlds simplest grammars are \ili{creole} grammars. \textit{Linguistic Typology} 5(2-3): 125–166. Online URL: \url{http://dx.doi.org/10.1515/lity.2001.001} (Accessed 8 Jan 2016).
\end{styleCitaviBibliographyEntry}

\begin{styleCitaviBibliographyEntry}
McWhorter, John H. 2005. \textit{Defining creole}. New York: Oxford University Press.
\end{styleCitaviBibliographyEntry}

\begin{styleCitaviBibliographyEntry}
McWhorter, John H. 2007. \textit{Language interrupted: Signs of non-native acquisition in standard language grammars}. Oxford: Oxford University Press.
\end{styleCitaviBibliographyEntry}

\begin{styleCitaviBibliographyEntry}
Milner, George B. 1959. \textit{\ili{Fijian} grammar}. Fiji: Government Press.
\end{styleCitaviBibliographyEntry}

\begin{styleCitaviBibliographyEntry}
Mintz, Malcolm W. 1994. \textit{A student’s grammar of Malay \& Indonesian}. Singapore: EPB Publishers.
\end{styleCitaviBibliographyEntry}

\begin{styleCitaviBibliographyEntry}
Mintz, Malcolm W. 2002. \textit{An Indonesian and Malay grammar for students} (Malay and Indonesian Language Collection), 2\textsuperscript{nd} edn. Perth: Indonesian/Malay Texts and Resources.
\end{styleCitaviBibliographyEntry}

\begin{styleCitaviBibliographyEntry}
Mithun, Marianne. 1988. The grammaticization of coordination. In Haiman, John \& Thompson, Sandra A. (eds.), \textit{Clause combining in grammar and discourse} (Typological Studies in Language 18). Amsterdam: John Benjamins Publishing Company, 331–359.
\end{styleCitaviBibliographyEntry}

\begin{styleCitaviBibliographyEntry}
Moeliono, Anton M. 1963. Ragam bahasa di Irian Barat. In Koentjaraningrat \& Bachtiar, Harsja W. (eds.), \textit{Penduduk Irian Barat} (Projek Penelitian Universitas Indonesia 102). Jakarta: Penerbitan Universitas, 28–38.
\end{styleCitaviBibliographyEntry}

\begin{styleCitaviBibliographyEntry}
Moravcsik, Edith A. 1971. Some cross-linguistic generalizations about yes-no questions and their answers. Stanford: Stanford University. (PhD dissertation.)
\end{styleCitaviBibliographyEntry}

\begin{styleCitaviBibliographyEntry}
Moravcsik, Edith A. 1978. Reduplicative constructions. In Greenberg, Joseph H. \& Ferguson, Charles A. \& Moravcsik, Andrew (eds.), \textit{Universals of human language. Volume 3: Word structure}. Stanford: Stanford University Press, 297–334.
\end{styleCitaviBibliographyEntry}

\begin{styleCitaviBibliographyEntry}
Moravcsik, Edith A. 2003. A semantic analysis of \isi{associative} plurals. \textit{Studies in Language} 27(3): 469–503. Online URL: \url{http://dx.doi.org/10.1075/sl.27.3.02mor} (Accessed 8 Jan 2016).
\end{styleCitaviBibliographyEntry}

\begin{styleCitaviBibliographyEntry}
Moravcsik, Edith A. 2013. \textit{Introducing language typology} (Cambridge Introductions to Language and Linguistics). Cambridge: Cambridge University Press.
\end{styleCitaviBibliographyEntry}

\begin{styleCitaviBibliographyEntry}
Morley, G. D. 2000. \textit{Syntax in functional grammar: An introduction to lexicogrammar in systemic linguistics}. London: Continuum.
\end{styleCitaviBibliographyEntry}

\begin{styleCitaviBibliographyEntry}
Mosel, Ulrike. 2010. Ditransitive constructions and their alternatives in \ili{Teop}. In Malchukov, Andrej L. \& Haspelmath, Martin \& Comrie, Bernard (eds.), \textit{Studies in ditransitive constructions: A comparative handbook}. Berlin: Mouton de Gruyter, 486–509.
\end{styleCitaviBibliographyEntry}

\begin{styleCitaviBibliographyEntry}
Moussay, Gérard. 1981. \textit{La langue Minangkebau}. Paris: Association Archipel.
\end{styleCitaviBibliographyEntry}

\begin{styleCitaviBibliographyEntry}
Mühlhäusler, Peter. 1996. \textit{Linguistic ecology: Language change and linguistic imperialism in the Pacific region}. London: Routledge.
\end{styleCitaviBibliographyEntry}

\begin{styleCitaviBibliographyEntry}
Mundhenk, A. N. 2002. \textit{Final particles in Melayu Papua}. Jayapura: Lembaga Alkitab Internasional.
\end{styleCitaviBibliographyEntry}

\begin{styleCitaviBibliographyEntry}
Nedjalkov, Vladimir P. 2007. Overview of the research: Definitions of terms, framework, and related issues. In Nedjalkov, Vladimir P. (ed.), \textit{Reciprocal constructions (Volume 1)} (Typological Studies in Language 71). Philadelphia: John Benjamins Publishing Company, 3–114.
\end{styleCitaviBibliographyEntry}

\begin{styleCitaviBibliographyEntry}
Nordhoff, Sebastian \& Hammarström, Harald \& Forkel, Robert \& Haspelmath, Martin (eds.) 2013. \textit{Glottolog 2.6}. Leipzig: Max Planck Institute for Evolutionary Anthropology. Online URL: \url{http://glottolog.org} (Accessed 8 Jan 2016).
\end{styleCitaviBibliographyEntry}

\begin{styleCitaviBibliographyEntry}
Nothofer, Bernd. 2009. Malay. In Brown, Keith \& Ogilvie, Sarah (eds.), \textit{Concise encyclopedia of languages of the world} (Concise Encyclopedias of Language and Linguistics). Amsterdam: Elsevier Science Ltd. 677–680.
\end{styleCitaviBibliographyEntry}

\begin{styleCitaviBibliographyEntry}
Overweel, Jeroen A. 1995. Appendix I: Sultans of Tidore and residents of Ternate, 1850-1909. In Overweel, Jeroen A. (ed.), \textit{Topics relating to Netherlands New Guinea in Ternate Residency memoranda of transfer and other assorted documents} (Irian Jaya Source Materials 13). Leiden: DSALCUL/IRIS, 137–138.
\end{styleCitaviBibliographyEntry}

\begin{styleCitaviBibliographyEntry}
Oxford University Press. 2000-. \textit{Oxford English dictionary online}. Oxford: Oxford University Press. Online URL: \url{http://www.oed.com/} (Accessed 8 Jan 2016).
\end{styleCitaviBibliographyEntry}

\begin{styleCitaviBibliographyEntry}
Paauw, Scott H. 2003. What is \ili{Bazaar Malay}? Paper presented at the Seventh International Symposium on Malay/Indonesian Linguistics – ISMIL 7. Nijmegen, 27-29 June 2003. Online URL: \url{http://email.eva.mpg.de/~gil/ismil/7/abstracts/paauw.html} (Accessed 8 Jan 2016).
\end{styleCitaviBibliographyEntry}

\begin{styleCitaviBibliographyEntry}
Paauw, Scott H. 2005. Malay dialectology: A new analysis. Paper presented at the Niagara Linguistics Society. Buffalo, 1 October 2005.
\end{styleCitaviBibliographyEntry}

\begin{styleCitaviBibliographyEntry}
Paauw, Scott H. 2007. Malay contact varieties in eastern Indonesia. Paper presented at the Eleventh International Symposium on Malay/Indonesian Linguistics – ISMIL 11. Manokwari, 6-8 August 2007. Online URL: \url{http://email.eva.mpg.de/~gil/ismil/11/abstracts/Paauw.pdf} (Accessed 8 Jan 2016).
\end{styleCitaviBibliographyEntry}

\begin{styleCitaviBibliographyEntry}
Paauw, Scott H. 2009. The Malay contact varieties of eastern Indonesia: A typological comparison. Buffalo: State University of New York. (PhD dissertation.)
\end{styleCitaviBibliographyEntry}

\begin{styleCitaviBibliographyEntry}
Paauw, Scott H. 2013. The Malay varieties of eastern Indonesia - How, when and where they became isolating language varieties. Rochester: University of Rochester.
\end{styleCitaviBibliographyEntry}

\begin{styleCitaviBibliographyEntry}
Padgett, Jaye. 1994. Stricture and Nasal Place Assimilation. \textit{Natural Language and Linguistic Theory} 12(3): 465–513. Online URL: \url{http://www.jstor.org/stable/4047807} (Accessed 8 Jan 2016).
\end{styleCitaviBibliographyEntry}

\begin{styleCitaviBibliographyEntry}
Parker, Steve. 2008. Sound level protrusions as physical correlates of sonority. \textit{Journal of Phonetics} 36(1): 55–90. Online URL: \url{http://dx.doi.org/10.1016/j.wocn.2007.09.003} (Accessed 8 Jan 2016).
\end{styleCitaviBibliographyEntry}

\begin{styleCitaviBibliographyEntry}
Parkinson, Richard H. R. 1900. Die Berlinhafen-Section: Ein Beitrag zur Ethnographie der Neu-Guinea Küste. \textit{Internationales Archiv für Ethnographie} XIII: 18-54, Tafeln XV-XXII.
\end{styleCitaviBibliographyEntry}

\begin{styleCitaviBibliographyEntry}
Pawley, Andrew. 2005. The chequered career of the Trans New Guinea hypothesis. In Pawley, Andrew \& Attenborough, Robert \& Golson, Jack \& Hide, Robin (eds.), \textit{Papuan pasts: Cultural, linguistic and biological histories of Papuan-speaking peoples} (Pacific Linguistics 572). Canberra: Research School of Pacific and Asian Studies, The Australian National University, 67–107.
\end{styleCitaviBibliographyEntry}

\begin{styleCitaviBibliographyEntry}
Payne, Thomas E. 1997. \textit{Describing morphosyntax: A guide for field linguists}. Cambridge: Cambridge University Press.
\end{styleCitaviBibliographyEntry}

\begin{styleCitaviBibliographyEntry}
Pike, Kenneth L. 1967. \textit{Language in relation to a unified theory of the structure of human behavior} (Janua Linguarum: Series Maior 24), 2\textsuperscript{nd} edn. The Hague: Mouton de Gruyter.
\end{styleCitaviBibliographyEntry}

\begin{styleCitaviBibliographyEntry}
Plag, Ingo. 2006a. Productivity. In Brown, Keith (ed.), \textit{Encyclopedia of language and linguistics}, 2\textsuperscript{nd} edn.. Amsterdam: Elsevier Science Ltd. 121–128.
\end{styleCitaviBibliographyEntry}

\begin{styleCitaviBibliographyEntry}
Plag, Ingo. 2006b. Productivity. In Aarts, Bas \& McMahon, April (eds.), \textit{The handbook of English linguistics}. Malden: Basil Blackwell Publishers, 537–556.
\end{styleCitaviBibliographyEntry}

\begin{styleCitaviBibliographyEntry}
Podungge, Nurhayati. 2000. Slang in Papuan Malay (case study on the students at faculty of letters, the State University of Papua, Manokwari). Manokwari: Universitas Negeri Papua.
\end{styleCitaviBibliographyEntry}

\begin{styleCitaviBibliographyEntry}
Polinsky, Maria. 1998. A non-syntactic account of some asymmetries in the \isi{double-object} construction. In Koenig, Jean-Pierre (ed.), \textit{Discourse and cognition: Bridging the gap}. Stanford: Center for the Study of Language and Information, 403–422.
\end{styleCitaviBibliographyEntry}

\begin{styleCitaviBibliographyEntry}
Prentice, David J. 1994. \ili{Manado Malay}: Product and agent of language change. In Dutton, Thomas E. \& Tryon, Darrell T. (eds.), \textit{Language contact and change in the \ili{Austronesian} world} (Trends in Linguistics: Studies and Monographs 77). Berlin: Mouton de Gruyter, 411–441.
\end{styleCitaviBibliographyEntry}

\begin{styleCitaviBibliographyEntry}
Quirk, Randolph \& Leech, Geoffrey N. \& Svartvik, Jan. 1972. \textit{A grammar of contemporary English}. Harlow: Longman.
\end{styleCitaviBibliographyEntry}

\begin{styleCitaviBibliographyEntry}
Ramos, Teresita V. 1971. \textit{\ili{Tagalog} structures} (PALI Language Texts: Philippines 20). Honolulu: University of Hawai’i Press.
\end{styleCitaviBibliographyEntry}

\begin{styleCitaviBibliographyEntry}
Regier, Terry. 1994. \textit{A preliminary study of the semantics of reduplication} (Technical Report TR-94-019). Berkeley: International Computer Science Institute. Online URL: \url{ftp://ftp.icsi.berkeley.edu/pub/techreports/1994/tr-94-019.pdf} (Accessed 8 Jan 2016).
\end{styleCitaviBibliographyEntry}

\begin{styleCitaviBibliographyEntry}
Robidé van der Aa, Pieter J. B. C. 1879. \textit{Reizen naar Nederlandsch Nieuw-Guinea ondernomen op last der regeering van Nederlandsch-Indië in de jaren 1871, 1872, 1875-1876 door de heeren P. van der Crab en J. E. Teysmann, J. G. Coorengel en A. J. Langeveldt van Hemert en P. Swaan met geschied- en aardrijkskundige toelichtingen}. ‘s-Gravenhage: Martinus Nijhoff.
\end{styleCitaviBibliographyEntry}

\begin{styleCitaviBibliographyEntry}
Roehr, Dorian. 2005. Pronouns are determiners after all. In den Dikken, Marcel \& Tortora, Christina (eds.), \textit{The function of function words and functional categories} (Linguistik Aktuell / Linguistics Today 78). Philadelphia: John Benjamins Publishing Company, 251–285.
\end{styleCitaviBibliographyEntry}

\begin{styleCitaviBibliographyEntry}
Roosman, Raden S. 1982. Pidgin Malay as spoken in Irian Jaya. \textit{The Indonesian Quarterly} 10(2): 95–104.
\end{styleCitaviBibliographyEntry}

\begin{styleCitaviBibliographyEntry}
Ross, Malcolm D. 2001. Contact-induced change in \ili{Oceanic} languages in North-West Melanesia. In Aikhenvald, Alexandra Y. \& Dixon, Robert M. W. (eds.), \textit{Areal diffusion and genetic inheritance: Problems in comparative linguistics}. Oxford: Oxford University Press, 134–166.
\end{styleCitaviBibliographyEntry}

\begin{styleCitaviBibliographyEntry}
Rowley, Charles D. 1972. \textit{The New Guinea villager: A retrospect from 1964}. Melbourne: F.W. Cheshire.
\end{styleCitaviBibliographyEntry}

\begin{styleCitaviBibliographyEntry}
Rubino, Carlo. 2011. Reduplication. In Haspelmath, Martin \& Dryer, Matthew S. \& Gil, David \& Comrie, Bernard (eds.), \textit{The world atlas of language structures}. München: Max Planck Digital Library, Chapter 27. Online URL: \url{http://wals.info/chapter/27} (Accessed 8 Jan 2016).
\end{styleCitaviBibliographyEntry}

\begin{styleCitaviBibliographyEntry}
Rudolph, Elisabeth. 1996. \textit{Contrast: Adversative and \isi{concessive} relations and their expressions in English, \ili{German}, Spanish, \ili{Portuguese} on sentence and text level} (Research in Text Theory 23). Berlin: Walter de Gruyter.
\end{styleCitaviBibliographyEntry}

\begin{styleCitaviBibliographyEntry}
Rutherford, Danilyn. 2005. Frontiers of the \ili{lingua franca}: Ideologies of the linguistic contact zone in \ili{Dutch} New Guinea. \textit{Ethnos} 70(3): 387–412. Online URL: \url{http://dx.doi.org/10.1080/00141840500294490} (Accessed 8 Jan 2016).
\end{styleCitaviBibliographyEntry}

\begin{styleCitaviBibliographyEntry}
Sadock, Jerrold M. \& Zwicky, Arnold M. 1985. Speech act distinctions in syntax. In Shopen, Timothy (ed.), \textit{Language typology and syntactic description. Volume 1: Clause structure}. Cambridge: Cambridge University Press, 155–196.
\end{styleCitaviBibliographyEntry}

\begin{styleCitaviBibliographyEntry}
Samaun. 1979. The system of the contracted forms of the vernacular \ili{bahasa Indonesia} in Jayapura, Irian Jaya: A project presented to the SEAMEO Regional Language Centre. Singapore: SEAMEO Regional Language Centre.
\end{styleCitaviBibliographyEntry}

\begin{styleCitaviBibliographyEntry}
Saragih, Chrisma F. 2012. The practical use of person reference in Papuan Malay. Nijmegen: Radboud University Nijmegen. (MA thesis.) Online URL: \url{http://www.ru.nl/publish/pages/518697/thesis_the_practical_use_of_person_reference_in_papuan_malay.docx} (Accessed 8 Jan 2016).
\end{styleCitaviBibliographyEntry}

\begin{styleCitaviBibliographyEntry}
Sawaki, Yusuf W. 2004. Some morpho-syntax notes on Melayu Papua. Manokwari: Universitas Negeri Papua.
\end{styleCitaviBibliographyEntry}

\begin{styleCitaviBibliographyEntry}
Sawaki, Yusuf W. 2005a. An agreement between the head of \isi{noun} phrase and personal pronouns in Melayu-Papua syntax. Paper presented at the Ninth International Symposium on Malay/Indonesian Linguistics – ISMIL 9. Ambun Pagi, 27-29 July 2005.
\end{styleCitaviBibliographyEntry}

\begin{styleCitaviBibliographyEntry}
Sawaki, Yusuf W. 2005b. Melayu Papua: Tong pu bahasa. Manokwari: Universitas Negeri Papua.
\end{styleCitaviBibliographyEntry}

\begin{styleCitaviBibliographyEntry}
Sawaki, Yusuf W. 2007. Does passive exist in Melayu Papua? Paper presented at the Eleventh International Symposium on Malay/Indonesian Linguistics – ISMIL 11. Manokwari, 6-8 August 2007. Online URL: \url{http://email.eva.mpg.de/~gil/ismil/11/abstracts/Sawaki.pdf} (Accessed 8 Jan 2016).
\end{styleCitaviBibliographyEntry}

\begin{styleCitaviBibliographyEntry}
Schachter, Paul \& Shopen, Timothy. 2007. Parts-of-speech systems. In Shopen, Timothy (ed.), \textit{Language typology and syntactic description. Volume 1: Clause structure}, 2\textsuperscript{nd} edn.. Cambridge: Cambridge University Press, 1–60.
\end{styleCitaviBibliographyEntry}

\begin{styleCitaviBibliographyEntry}
Schütz, Albert J. \& Komaitai, Rusiate T. 1971. \textit{Spoken \ili{Fijian}: an intensive course in Bauan \ili{Fijian}, with grammatical notes and glossary} (PALI Language Texts: Melanesia 1). Honolulu: University of Hawai’i Press.
\end{styleCitaviBibliographyEntry}

\begin{styleCitaviBibliographyEntry}
Scott, Graham R. \& Kim, Hyun \& Rumaropen, Ben E. W. \& Scott, Eleonora L. \& Nussy, Christian G. \& Yumbi, Anita C. M. \& Cochran, Robert C. 2008. Tong pu bahasa: A preliminary report on some linguistic and sociolinguistic features of Papuan Malay. \ili{Sentani}: SIL International Indonesia.
\end{styleCitaviBibliographyEntry}

\begin{styleCitaviBibliographyEntry}
Seiler, Walter. 1982. The spread of Malay to Kaiser-Wilhelmsland. In Kähler, Hans \& Carle, Rainer (eds.), \textit{GAVA*: Studies in \ili{Austronesian} languages and cultures dedicated to Hans Kähler = Studien zu austronesischen Sprachen und Kulturen Hans Kähler gewidmet} (Veröffentlichungen des Seminars für Indonesische und Südseesprachen der Universität Hamburg 17). Berlin: Reimer, 67–85.
\end{styleCitaviBibliographyEntry}

\begin{styleCitaviBibliographyEntry}
Seiler, Walter. 1985. The Malay language of New Guinea. In Wurm, Stephen A. (ed.), \textit{Papers in Pidgin and Creole linguistics} (Pacific Linguistics A-72). Canberra: Research School of Pacific Studies, The Australian National University, 143–153.
\end{styleCitaviBibliographyEntry}

\begin{styleCitaviBibliographyEntry}
Shellabear, William G. 1904. \textit{A practical Malay grammar}, 2\textsuperscript{nd} edn. Singapore: The American Mission Press. Online URL: \url{http://ia600303.us.archive.org/18/items/practicalmalaygr00shelrich/practicalmalaygr00shelrich.pdf} (Accessed 8 Jan 2016).
\end{styleCitaviBibliographyEntry}

\begin{styleCitaviBibliographyEntry}
Siewierska, Anna. 2011. Gender distinctions in independent personal pronouns. In Haspelmath, Martin \& Dryer, Matthew S. \& Gil, David \& Comrie, Bernard (eds.), \textit{The world atlas of language structures}. München: Max Planck Digital Library, Chapter 44. Online URL: \url{http://wals.info/chapter/44} (Accessed 8 Jan 2016).
\end{styleCitaviBibliographyEntry}

\begin{styleCitaviBibliographyEntry}
Sigurðsson, Halldór Á. 2006. The \ili{Icelandic} \isi{noun} phrase: Central traits. \textit{Arkiv för Nordisk Filologi} 121: 193–236. Online URL: \url{http://lup.lub.lu.se/luur/download?func=downloadFile & recordOId=539373 & fileOId=625983} (Accessed 8 Jan 2016).
\end{styleCitaviBibliographyEntry}

\begin{styleCitaviBibliographyEntry}
SIL International. 1996–2008. IPA Help: A phonetics learning tool, Version 2.1 (SIL Computing). Dallas: SIL International. Online URL: \url{http://www-01.sil.org/computing/ipahelp/} (Accessed 8 Jan 2016).
\end{styleCitaviBibliographyEntry}

\begin{styleCitaviBibliographyEntry}
Silverstein, Michael. 1976. Hierarchy of features and ergativity. In Dixon, Robert M. W. (ed.), \textit{Grammatical categories in Australian languages} (Australian Institute of Aboriginal Studies Linguistic Series 22). Canberra: Humanities Press, 112–171. Online URL: \url{http://www.scribd.com/doc/36490729/silverstein-hierarchy-of-features-and-ergativity} (Accessed 8 Jan 2016).
\end{styleCitaviBibliographyEntry}

\begin{styleCitaviBibliographyEntry}
Silzer, Peter J. 1978. Notes on Irianese Indonesian. Jayapura: Summer Institute of Linguistics.
\end{styleCitaviBibliographyEntry}

\begin{styleCitaviBibliographyEntry}
Silzer, Peter J. 1979. Notes on Irianese Indonesian. Jayapura: Summer Institute of Linguistics; Universitas Cenderawasih.
\end{styleCitaviBibliographyEntry}

\begin{styleCitaviBibliographyEntry}
Slomanson, Peter. 2013. Sri Lankan Malay structure dataset. In Michaelis, Susanne M. \& Maurer, Philippe \& Haspelmath, Martin \& Huber, Magnus (eds.), \textit{The Atlas of Pidgin and Creole Language Structures (APiCS)}. Leipzig: Max Planck Institute for Evolutionary Anthropology. Online URL: \url{http://apics-online.info/contributions/66} (Accessed 8 Jan 2016).
\end{styleCitaviBibliographyEntry}

\begin{styleCitaviBibliographyEntry}
Smessaert, Hans \& ter Meulen, Alice G. B. 2004. Temporal reasoning with aspectual adverbs. \textit{Linguistics and Philosophy} 27(2): 209–261. Online URL: \url{http://dx.doi.org/10.1023/B:LING.0000016467.50422.63} (Accessed 8 Jan 2016).
\end{styleCitaviBibliographyEntry}

\begin{styleCitaviBibliographyEntry}
Sneddon, James N. 2003. \textit{The Indonesian language: Its history and role in modern society}. Sydney: UNSW Press.
\end{styleCitaviBibliographyEntry}

\begin{styleCitaviBibliographyEntry}
Sneddon, James N. 2006. \textit{Colloquial Jakartan Indonesian} (Pacific Linguistics 581). Canberra: Research School of Pacific and Asian Studies, The Australian National University.
\end{styleCitaviBibliographyEntry}

\begin{styleCitaviBibliographyEntry}
Sneddon, James N. 2010. \textit{Indonesian reference grammar}, 2\textsuperscript{nd} edn. Crows Nest: Allen and Unwin.
\end{styleCitaviBibliographyEntry}

\begin{styleCitaviBibliographyEntry}
Sommerfeldt, Karl-Ernst \& Schreiber, Herbert. 1983. \textit{Wörterbuch zur Valenz und Distribution der Substantive}. Leipzig: VEB Bibliographisches Institut.
\end{styleCitaviBibliographyEntry}

\begin{styleCitaviBibliographyEntry}
Song, Jae J. 2006. Causatives: Semantics. In Brown, Keith (ed.), \textit{Encyclopedia of language and linguistics}, 2\textsuperscript{nd} edn.. Amsterdam: Elsevier Science Ltd. 265–269.
\end{styleCitaviBibliographyEntry}

\begin{styleCitaviBibliographyEntry}
Song, Jae J. 2011. Non\isi{periphrastic causative} constructions. In Haspelmath, Martin \& Dryer, Matthew S. \& Gil, David \& Comrie, Bernard (eds.), \textit{The world atlas of language structures}. München: Max Planck Digital Library, Chapter 111. Online URL: \url{http://wals.info/chapter/111} (Accessed 8 Jan 2016).
\end{styleCitaviBibliographyEntry}

\begin{styleCitaviBibliographyEntry}
Stassen, Leon. 2000. $AN$D-languages and WITH-languages. \textit{Linguistic Typology} 4: 1–54. Online URL: \url{http://dx.doi.org/10.1515/lity.2000.4.1.1} (Accessed 8 Jan 2016).
\end{styleCitaviBibliographyEntry}

\begin{styleCitaviBibliographyEntry}
Stassen, Leon. 2009. \textit{Predicative possession} (Oxford Studies in Typology and Linguistic Theory). Oxford: Oxford University Press.
\end{styleCitaviBibliographyEntry}

\begin{styleCitaviBibliographyEntry}
Stassen, Leon. 2011a. Noun phrase \isi{conjunction}. In Haspelmath, Martin \& Dryer, Matthew S. \& Gil, David \& Comrie, Bernard (eds.), \textit{The world atlas of language structures}. München: Max Planck Digital Library, Chapter 63. Online URL: \url{http://wals.info/chapter/63} (Accessed 8 Jan 2016).
\end{styleCitaviBibliographyEntry}

\begin{styleCitaviBibliographyEntry}
Stassen, Leon. 2011b. Predicative possession. In Haspelmath, Martin \& Dryer, Matthew S. \& Gil, David \& Comrie, Bernard (eds.), \textit{The world atlas of language structures}. München: Max Planck Digital Library, Chapter 117. Online URL: \url{http://wals.info/chapter/117} (Accessed 8 Jan 2016).
\end{styleCitaviBibliographyEntry}

\begin{styleCitaviBibliographyEntry}
Steinhauer, Hein. 1983. Notes on the Malay of Kupang. In Collins, James T. (ed.), \textit{Studies in Malay dialects: Part II} (NUSA – Linguistic Studies of Indonesian and other Languages in Indonesia 17). Jakarta: Badan Penyelenggara Seri NUSA, Universitas Katolik Atma Jaya, 42–64.
\end{styleCitaviBibliographyEntry}

\begin{styleCitaviBibliographyEntry}
Steinhauer, Hein. 1991. Malay in east Indonesia: The case of Larantuka (Flores). In Steinhauer, Hein (ed.), \textit{Papers in \ili{Austronesian} linguistics. Volume 1} (Pacific Linguistics A-81). Canberra: Research School of Pacific Studies, The Australian National University, 177–195.
\end{styleCitaviBibliographyEntry}

\begin{styleCitaviBibliographyEntry}
Stoel, Ruben. 2005. \textit{Focus in \ili{Manado Malay}: Grammar, particles, and intonation}. Leiden: Department of Languages and Cultures of South-East Asia and Oceania, Leiden University.
\end{styleCitaviBibliographyEntry}

\begin{styleCitaviBibliographyEntry}
Suharno, Ignatius. 1979. Some notes on the teaching of \ili{Standard Indonesian} to speakers of Irianese Indonesian. In Tarumingkeng, Rudolf C. (ed.), \textit{Irian, Bulletin of Irian Jaya Development} (8.1). Jayapura: Institute for Anthropology, Universitas Cenderawasih, 3–32. Online URL: \url{http://www.papuaweb.org/dlib/irian/8-1.pdf} (Accessed 8 Jan 2016).
\end{styleCitaviBibliographyEntry}

\begin{styleCitaviBibliographyEntry}
Suharno, Ignatius. 1981. The reductive system of an Indonesian dialect – A study of Irian Jaya Case. Paper presented at the Third International Conference on \ili{Austronesian} Linguistics. National Center for Language Development, Ministry of Education and Culture, Denpasar, 19-24 January 1981.
\end{styleCitaviBibliographyEntry}

\begin{styleCitaviBibliographyEntry}
Sutri Narfafan \& Donohue, Mark. under review. Papuan Malay. \textit{Journal of the International Phonetic Association}.
\end{styleCitaviBibliographyEntry}

\begin{styleCitaviBibliographyEntry}
Tadmor, Uri. 2002. Language contact and the homeland of Malay. Paper presented at the Sixth International Symposium on Malay/Indonesian Linguistics – ISMIL 6. Bintan, 5 August 2002. Online URL: \url{http://lingweb.eva.mpg.de/jakarta/docs/homeland_handout.pdf} (Accessed 8 Jan 2016).
\end{styleCitaviBibliographyEntry}

\begin{styleCitaviBibliographyEntry}
Tadmor, Uri. 2009a. Indonesian vocabulary. In Haspelmath, Martin \& Tadmor, Uri (eds.), \textit{World Loanword Database (WOLD)}. München: Max Planck Digital Library. Online URL: \url{http://wold.clld.org/vocabulary/27} (Accessed 8 Jan 2016).
\end{styleCitaviBibliographyEntry}

\begin{styleCitaviBibliographyEntry}
Tadmor, Uri. 2009b. Malay-Indonesian. In Comrie, Bernard (ed.), \textit{The world’s major languages}, 2\textsuperscript{nd} edn.. London: Routledge, 791–818.
\end{styleCitaviBibliographyEntry}

\begin{styleCitaviBibliographyEntry}
Taylor, Paul M. 1983. North Moluccan Malay: Notes on a “substandard” dialect of Indonesian. In Collins, James T. (ed.), \textit{Studies in Malay dialects: Part II} (NUSA – Linguistic Studies of Indonesian and other Languages in Indonesia 17). Jakarta: Badan Penyelenggara Seri NUSA, Universitas Katolik Atma Jaya, 14–27.
\end{styleCitaviBibliographyEntry}

\begin{styleCitaviBibliographyEntry}
Tebay, Neles. 2005. \textit{West Papua: The struggle for peace with justice}. London: Catholic Institute for International Relations.
\end{styleCitaviBibliographyEntry}

\begin{styleCitaviBibliographyEntry}
Teeuw, Andries. 1961. \textit{A critical survey of studies on Malay and Bahasa Indonesia} (Koninklijk Instituut voor Taal-, Land- en Volkenkunde: Bibliographical Series 5). ‘S-Gravenhage: Martinus Nijhoff.
\end{styleCitaviBibliographyEntry}

\begin{styleCitaviBibliographyEntry}
Teutscher, Henk J. 1954. Wat gaat men straks op Nieuw-Guinea spreken? \textit{Wending – Manndblad voor Evangelie en Cultuur} 9: 113–129.
\end{styleCitaviBibliographyEntry}

\begin{styleCitaviBibliographyEntry}
The International Phonetic Association. 2005. IPA chart (available under a Creative Commons Attribution-Sharealike 3.0 Unported License). S.l.: The International Phonetic Association. Online URL: \url{https://www.internationalphoneticassociation.org/content/full-ipa-chart} (Accessed 8 Jan 2016).
\end{styleCitaviBibliographyEntry}

\begin{styleCitaviBibliographyEntry}
Thomason, Sarah G. \& Kaufman, Terrence. 1988. \textit{Language contact, creolization, and genetic linguistics}. Berkeley: University of California Press.
\end{styleCitaviBibliographyEntry}

\begin{styleCitaviBibliographyEntry}
Thompson, Sandra A. \& Longacre, Robert R. \& Hwang, Shin Ja J. 2007. Adverbial clauses. In Shopen, Timothy (ed.), \textit{Language typology and syntactic description. Volume 2: Complex constructions}, 2\textsuperscript{nd} edn.. Cambridge: Cambridge University Press, 237–300.
\end{styleCitaviBibliographyEntry}

\begin{styleCitaviBibliographyEntry}
Timmer, Jaap. 2002. A bibliographic essay on the southwestern Kepala Burung (Bird’s Head, Doberai) of Papua. \textit{Papuaweb’s Annotated Bibliographies}: 26 p. Online URL: \url{http://www.papuaweb.org/bib/abib/jt-kepala.pdf} (Accessed 8 Jan 2016).
\end{styleCitaviBibliographyEntry}

\begin{styleCitaviBibliographyEntry}
Topping, Donald M. \& Ogo, Pedro. 1960. \textit{Spoken \ili{Chamorro}: An intensive language course with grammatical notes and glossary}. Honolulu: University of Hawai’i Press.
\end{styleCitaviBibliographyEntry}

\begin{styleCitaviBibliographyEntry}
Van der Eng, Pierre. 2004. Irian Jaya (West Irian). In Ooi Keat Gin (ed.), \textit{Southeast Asia: A historical encyclopedia, from Angkor Wat to East Timor}. Santa Barbara: ABC-CLIO, 663–665.
\end{styleCitaviBibliographyEntry}

\begin{styleCitaviBibliographyEntry}
Van Durme, Karen \& Institut for Sprog og Kommunikation. 1997. \textit{The \isi{valency} of nouns} (Odense Working Papers in Language and Communication 15). Odense: Institute of Language and Communication, Odense University.
\end{styleCitaviBibliographyEntry}

\begin{styleCitaviBibliographyEntry}
Van Hasselt, Frans J. F. 1926. \textit{In het land van de Papoea’s}. Utrecht: Kemink \& Zoon.
\end{styleCitaviBibliographyEntry}

\begin{styleCitaviBibliographyEntry}
Van Hasselt, Frans J. F. 1936. Het Noemfoorsch als Eenheidstaal op het Noordwestelijk Deel van Nieuw Guinea. \textit{Tijdschrift Nieuw Guinea} 1: 114–117.
\end{styleCitaviBibliographyEntry}

\begin{styleCitaviBibliographyEntry}
Van Klinken, Catharina L. 1999. \textit{A grammar of the Fehan dialect of Tetun: An \ili{Austronesian} language of West Timor} (Pacific Linguistics 155). Canberra: Research School of Pacific and Asian Studies, The Australian National University.
\end{styleCitaviBibliographyEntry}

\begin{styleCitaviBibliographyEntry}
Van Minde, Don. 1997. \textit{Malayu Ambong: Phonology, \isi{morphology}, syntax}. Leiden: Department of Languages and Cultures of South-East Asia and Oceania, Leiden University.
\end{styleCitaviBibliographyEntry}

\begin{styleCitaviBibliographyEntry}
Van Oldenborgh, J. 1995. Memorie van overgave van het bestuur over de Residentie Ternate door den aftredenden Resident J. van Oldenborgh aan den optredenden \citet{Resident1895}. In Overweel, Jeroen A. (ed.), \textit{Topics relating to Netherlands New Guinea in Ternate Residency memoranda of transfer and other assorted documents} (Irian Jaya Source Materials 13). Leiden: DSALCUL/IRIS, 80–84.
\end{styleCitaviBibliographyEntry}

\begin{styleCitaviBibliographyEntry}
Van Valin, Robert D. 2001. \textit{An introduction to syntax}. Cambridge: Cambridge University Press.
\end{styleCitaviBibliographyEntry}

\begin{styleCitaviBibliographyEntry}
Van Valin, Robert D. 2005. \textit{Exploring the syntax-semantics interface}. Cambridge: Cambridge University Press.
\end{styleCitaviBibliographyEntry}

\begin{styleCitaviBibliographyEntry}
Van Valin, Robert D. \& LaPolla, Randy J. 1997. \textit{Syntax: Structure, meaning, and function}. Cambridge: Cambridge University Press.
\end{styleCitaviBibliographyEntry}

\begin{styleCitaviBibliographyEntry}
Van Velzen, Paul. 1995. Some notes on the variety of Malay used in Serui and vicinity. In Baak, Connie \& Bakker, Mary \& van der Meij, Dick (eds.), \textit{Tales from a concave world: Liber amicorum Bert Voorhoeve}. Leiden: Leiden University, 311-343 (265-296).
\end{styleCitaviBibliographyEntry}

\begin{styleCitaviBibliographyEntry}
Voorhoeve, Clemens L. 1983. Some observations on North-Moluccan Malay. In Collins, James T. (ed.), \textit{Studies in Malay dialects: Part II} (NUSA – Linguistic Studies of Indonesian and other Languages in Indonesia 17). Jakarta: Badan Penyelenggara Seri NUSA, Universitas Katolik Atma Jaya, 1–13.
\end{styleCitaviBibliographyEntry}

\begin{styleCitaviBibliographyEntry}
Walker, Roland W. 1982. Language use at Namatota: A \isi{sociolinguistic profile}. In Halim, Amran \& Carrington, Lois \& Wurm, Stephen A. (eds.), \textit{Papers from the Third International Conference on \ili{Austronesian} Linguistics} (Pacific Linguistics C-76). Canberra: Research School of Pacific Studies, The Australian National University, 79–94.
\end{styleCitaviBibliographyEntry}

\begin{styleCitaviBibliographyEntry}
Wallace, Alfred R. 1890. \textit{The Malay Archipelago and the land of the orang-utan and the bird of paradise: A narrative of travel with studies of man and nature}, 10\textsuperscript{th} edn. London: MacMillian and Co. Online URL: \url{https://archive.org/details/malayarchipelag04wallgoog} (Accessed 8 Jan 2016).
\end{styleCitaviBibliographyEntry}

\begin{styleCitaviBibliographyEntry}
Warami, Hugo. 2003. \textit{Wacana humor (mob) dialek Melayu Papua: Kumpulan pojok MOB “Warung Papeda” masyarakat Papua. (Surat Kabar Harian Cenderawasih Pos, 1994-200)}. Manokwari: Fakultas Sastra, Universitas Negeri Papua.
\end{styleCitaviBibliographyEntry}

\begin{styleCitaviBibliographyEntry}
Warami, Hugo. 2004. \textit{Bentuk dan peran humor (mob) dalam masyarakat Papua}. Manokwari: Fakultas Sastra, Universitas Negeri Papua.
\end{styleCitaviBibliographyEntry}

\begin{styleCitaviBibliographyEntry}
Warami, Hugo. 2005. Bentuk partikel bahasa Melayu Papua. \textit{Linguistika} 6(1): 87–112.
\end{styleCitaviBibliographyEntry}

\begin{styleCitaviBibliographyEntry}
Weinreich, Uriel. 1953. \textit{Languages in contact: Findings and problems} (Publications of the Linguistic Circle of New York 1). New York: Linguistic Circle of New York.
\end{styleCitaviBibliographyEntry}

\begin{styleCitaviBibliographyEntry}
Wichmann, Arthur. 1917. \textit{Bericht \"{u}ber eine im Jahre 1903 ausgef\"{u}hrte Reise nach Neu-Guinea} (Nova Guinea: R\'{e}sultats des exp\'{e}ditions scientifiques \`{a} la Nouvelle Guin\'{e}e en 1903 sous les auspices de Arthur Wichmann 4). Leiden: E. J. Brill.
\end{styleCitaviBibliographyEntry}

\begin{styleCitaviBibliographyEntry}
Wiltshire, Caroline \& Marantz, Alec. 1978. Reduplication. In Greenberg, Joseph H. \& Ferguson, Charles A. \& Moravcsik, Andrew (eds.), \textit{Universals of human language. Volume 3: Word structure}. Stanford: Stanford University Press, 557–567.
\end{styleCitaviBibliographyEntry}

\begin{styleCitaviBibliographyEntry}
Winstedt, Richard O. 1913. \textit{Malay grammar}. Oxford: Oxford University Press. Online URL: \url{http://ia600303.us.archive.org/18/items/malaygrammar00winsrich/malaygrammar00winsrich.pdf} (Accessed 8 Jan 2016).
\end{styleCitaviBibliographyEntry}

\begin{styleCitaviBibliographyEntry}
Winstedt, Richard O. 1938. \textit{Colloquial Malay: A simple grammar with conversations}, 4\textsuperscript{th} edn. Singapore: Kelly and Walsh, Limited.
\end{styleCitaviBibliographyEntry}

\begin{styleCitaviBibliographyEntry}
Wischer, Ilse. 2006. Grammaticalization. In Brown, Keith (ed.), \textit{Encyclopedia of language and linguistics}, 2\textsuperscript{nd} edn.. Amsterdam: Elsevier Science Ltd. 129–136.
\end{styleCitaviBibliographyEntry}

\begin{styleCitaviBibliographyEntry}
Wolff, John U. 1988. The contribution of Banjar Masin Malay to the reconstruction of Proto-Malay. In Ahmad, Mohammed T. \& Zain, Zaini M. (eds.), \textit{Rekonstruksi dan cabang-cabang bahasa Melayu Induk} (Siri Monograf Sejarah Bahasa Melayu). Kuala Lumpur: Dewan Bahasa dan Pustaka, 85–98.
\end{styleCitaviBibliographyEntry}

\begin{styleCitaviBibliographyEntry}
Woorden.org MMXI. 2010-. \textit{Woorden Nederlandse Taal}. Niebert: Woorden.org MMXI. Online URL: \url{http://www.woorden.org/woord/} (Accessed 8 Jan 2016).
\end{styleCitaviBibliographyEntry}

\begin{styleCitaviBibliographyEntry}
Yap, Foong H. 2007. On native and contact-induced \isi{grammaticalization}: The case of Malay empunya. Hong Kong: Department of Linguistics and Modern Languages, \ili{Chinese} University of Hong Kong. Online URL: \url{https://www.researchgate.net/publication/237259428_On_native_and_contact-induced_grammaticalization_The_case_of_Malay_empunya} (Accessed 8 Jan 2016).
\end{styleCitaviBibliographyEntry}

\begin{styleCitaviBibliographyEntry}
Yap, Foong H. \& Matthews, Stephen \& Horie, Kaoru. 2004. From pronominalizer to pragmatic marker – Implications for unidirectionality from a crosslinguistic perspective. In Fischer, Olga \& Norde, Muriel \& Perridon, Harry (eds.), \textit{Up and down the cline: The nature of grammaticalization} (Typological Studies in Language 59). Amsterdam: John Benjamins Publishing Company, 137–168.
\end{styleCitaviBibliographyEntry}

\begin{styleCitaviBibliographyEntry}
Yembise, Yohana S. 2011. Linguistic and cultural variations as barriers to the TEFL settings in Papua. \textit{TEFLIN Journal} 22(2): 201–225. Online URL: \url{http://journal.teflin.org/index.php/journal/article/download/27/28} (Accessed 8 Jan 2016).
\end{styleCitaviBibliographyEntry}

\begin{styleCitaviBibliographyEntry}
Zöller, Hugo. 1891. \textit{Deutsch-Neuguinea und meine Ersteigung des Finisterre-Gebirges: Eine Schilderung des ersten erfolgreichen Vordringens zu den Hochgebirgen Inner-Neuguineas, der Natur des Landes, der Sitten der Eingeborenen und des gegenwärtigen Standes der deutschen Kolonisationstätigkeit in Kaiser-Wilhelms-Land, Bismarck- und Salomo-Archipel, nebst einem Wortverzeichnis von 46 Papua-Sprachen}. Stuttgart: Union Deutsche Verlagsgesellschaft.
\end{styleCitaviBibliographyEntry}

\begin{styleCitaviBibliographyEntry}
Zsiga, Elizabeth. 2006. Assimilation. In Brown, Keith (ed.), \textit{Encyclopedia of language and linguistics}, 2\textsuperscript{nd} edn.. Amsterdam: Elsevier Science Ltd. 553–558.
\end{styleCitaviBibliographyEntry}

\begin{styleCitaviBibliographyEntry}
Zwanenburg, Wiecher. 2000. Correspondence between formal and semantic relations. In Booij, Geert E. \& Lehmann, Christian \& Mugdan, Joachim (eds.), \textit{Morphologie: Ein internationales Handbuch zur Flexion und Wortbildung = Morphology: An international handbook on inflection and \isi{word-formation}. Volume 1} (Handbooks of Linguistics and Communication Science 17.1). Berlin: Mouton de Gruyter, 840–850.
\end{styleCitaviBibliographyEntry}

