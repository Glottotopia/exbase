\chapter[Noun phrases]{Noun phrases}
\label{Para_8}
\section{Introduction}
\label{Para_8.1}
This chapter describes the Papuan Malay \isi{noun} phrase with its different types of structures. Also included is a description of \isi{noun} phrase \isi{apposition}; \isi{noun} phrase coordination is not discussed here but in \chapref{Para_14}.

An overview of the possible constituents of the Papuan Malay \isi{noun} phrase is given in \tabref{Table_8.1}
 (the parenthesis in the table header signal that the modifiers are optional). Modifying elements listed in the same column represent choices; constituents in the same row do not necessarily co-occur.


\begin{table}
\caption{Possible constituents of the Papuan Malay \isi{noun} phrase}\label{Table_8.1}

\begin{tabular}{llllll}
\lsptoprule
\multicolumn{1}{c}{(\textsc{mod})} & \multicolumn{1}{c}{\textsc{head}} & \multicolumn{4}{c}{ (\textsc{mod})}\\
&  & \multicolumn{1}{c}{Post-1} & \multicolumn{1}{c}{Post-2} & \multicolumn{1}{c}{Post-3} & \multicolumn{1}{c}{ Post-4}\\
\midrule
\textsc{num} & \textsc{n} & \textsc{v} & \textsc{pro} & \textsc{dem} & \textsc{dem}\\
\textsc{qt} &  & \textsc{n} &  & \textsc{loc} & \\
\textsc{possr-np} &  & \textsc{pp} &  & \textsc{int} & \\
&  & \textsc{rc} &  & \textsc{num} & \\
&  &  &  & \textsc{qt} & \\
& \textsc{pro} & \textsc{pp} &  & \textsc{loc} & \textsc{dem}\\
&  & \textsc{rc} &  & \textsc{num} & \\
&  &  &  & \textsc{qt} & \\
\textsc{possr-np} & \textsc{dem} & \textsc{rc} &  &  & \textsc{dem}\\
& \textsc{loc} & \textsc{rc} &  &  & \textsc{dem}\\
\textsc{possr-np} & \textsc{int} & \textsc{rc} &  &  & \textsc{dem}\\
\lspbottomrule
\end{tabular}
\end{table}

In the following, examples are presented for the different types of constituents that can function as the head of a \isi{noun} phrase. In giving examples, brackets are used to indicate the constituent structure within the \isi{noun} phrase, where deemed necessary. In (\ref{Example_8.1}) the head is a \isi{noun}, in (\ref{Example_8.2}) it is a personal \isi{pronoun}, in (\ref{Example_8.3}) a \isi{demonstrative}, in (\ref{Example_8.4}) a \isi{locative}, and in (\ref{Example_8.5}) an \isi{interrogative}. Head nouns allow the widest range of modifiers, while personal pronouns, \isi{demonstrative}, locatives, and interrogatives allow only a subset of modifiers, as shown throughout this chapter.



\begin{styleExampleTitle}
Types of constituents functioning as heads in \isi{noun} phrases
\end{styleExampleTitle}
\ea
\label{Example_8.1}
\gll {kitong} {cari} {\bluebold{ana}} {kecil} {itu}\\ %
 \textsc{1pl}  search  child  be.small  \textsc{d.dist}\\
\glt 
‘we were looking for that small \bluebold{kid}’ \textstyleExampleSource{[080921-004a-CvNP.0070]}
\z

\ea
\label{Example_8.2}
\gll {\bluebold{dong}} {dua} {tu} {ikut}\\ %
 \textsc{3pl}  two  \textsc{d.dist}  follow\\
\glt 
[About an upcoming event:] ‘both of \bluebold{them (}\blueboldSmallCaps{emph}\bluebold{)} are going to participate’ \textstyleExampleSource{[081115-001a-Cv.0115]}
\z

\ea
\label{Example_8.3}
\gll {\bluebold{itu}} {tu} {rahasia} {mo} {mo} {biking} {apa} {ka,} {mo} {\ldots}\\ %
 \textsc{d.dist}  \textsc{d.dist}  secret  want  want  make  what  or  want  \\
\glt 
[About raising children:] ‘\bluebold{that} (\textsc{emph}) is the secret (when we) want want to do something or want to {\ldots}’ \textstyleExampleSource{[080917-010-CvEx.0160]}
\z

\ea
\label{Example_8.4}
\gll {e,} {sa} {tinggal} {di} {\bluebold{situ}} {tu}\\ %
 uh  \textsc{1sg}  stay  at  \textsc{l.med}  \textsc{d.dist}\\
\glt 
‘uh, I lived \bluebold{there (}\blueboldSmallCaps{emph}\bluebold{)}’ \textstyleExampleSource{[080922-002-Cv.0112]}
\z

\ea
\label{Example_8.5}
\gll {ana} {laki{\Tilde}laki} {ini} {de} {mo} {ke} {\bluebold{mana}} {ni}?\\ %
 child  \textsc{rdp}{\Tilde}husband  \textsc{d.prox}  \textsc{3sg}  want  to  where  \textsc{d.prox}\\
\glt 
‘this boy, \bluebold{where} (\textsc{emph}) does he want to (go)?’ \textstyleExampleSource{[080922-004-Cv.0017]}
\z



The minimal \isi{noun} phrase consists of a bare head nominal. Modifiers are optional and occur in pre- and/or posthead position. Attested in the corpus is the co-occurrence of up to three posthead constituents. Modifiers listed in the same pre- or posthead slots in \tabref{Table_8.1}
 are unattested.\footnote{This observation also applies to adnominal possessive relations where a \isi{numeral} or \isi{quantifier} precedes a possessor. Such pre-possessor numerals or quantifiers are not analyzed as “co-occurring” with the \isi{possessor \isi{noun} phrase}. Instead they are analyzed as prehead modifiers of the \isi{possessor \isi{noun} phrase}. This is shown in the elicited example in (\ref{Footnote_Example_8.1}): the prehead modifier smua ‘all’ modifies the head nominal of the possessor phrase, ana ‘child’. (For details on \isi{adnominal possession} see §\ref{Para_8.4} and \chapref{Para_9}.)
 \vspace{-5pt}
 \ea
 \label{Footnote_Example_8.1}
 \gll \bluebold{smua} \bluebold{ana} \bluebold{ini} pu tugas su selesay\\
 [[all] child [\textsc{d.prox}]] \textsc{poss} duty already finish\\
 \glt  ‘the duties of all these children are already taken care of’ [Elicited ME151120.002]
 \z
 } There is one exception, however, namely the \isi{quantifier} \textitbf{brapa} ‘several’, as shown in in (\ref{Example_8.9}).



Prehead modifiers can be numerals such as \textitbf{empat} ‘four’ in (\ref{Example_8.6}), quantifiers such as \textitbf{smua} ‘all’ in (\ref{Example_8.7}), or possessor \isi{noun} phrases in adnominal possessive constructions such \textitbf{orang{\Tilde}orang besar} ‘big people’ in (\ref{Example_8.8}). The co-occurrence of prehead modifers is unattested, with one exception. The mid-range \isi{quantifier} \textitbf{brapa} ‘several’ co-occurs with certain numerals, such as \textitbf{ratus} ‘hundred’, or \textitbf{ribu} ‘thousand’, as in \textitbf{brapa ratus orang} ‘several hundred people’ in (\ref{Example_8.9}). 



\begin{styleExampleTitle}
\textsc{mod} – \textsc{head}
\end{styleExampleTitle}
\ea
\label{Example_8.6}
\gll {jadi} {saya} {\bluebold{empat}} {\bluebold{ana}}\\ %
 so  \textsc{1sg}  four  child\\
\glt 
\textsc{num} – \textsc{head}: ‘so, I (have) \bluebold{four children}’ \textstyleExampleSource{[081006-024-CvEx.0002]}
\z

\ea
\label{Example_8.7}
\gll {\bluebold{smua}} {\bluebold{buku}} {bisa} {basa}\\ %
 all  book  be.able  be.wet\\
\glt 
\textsc{qt} – \textsc{head}: ‘\bluebold{all books} could get wet’ \textstyleExampleSource{[080917-008-NP.0189]}
\z

\ea
\label{Example_8.8}
\gll {\ldots} {bukang} {\bluebold{orang{\Tilde}orang}} {\bluebold{besar}} {\bluebold{pu}} {\bluebold{ana}} {\ldots}\\ %
 { }  \textsc{neg}  \textsc{[rdp}{\Tilde}person  big]  \textsc{poss}  child  \\
\glt 
‘(she’s the child of farmers) not the \bluebold{child of big people} {\ldots}’ \textstyleExampleSource{[081110-005-Pr.0094]}
\z

\ea
\label{Example_8.9}
\gll {\ldots} {tentara} {itu} {ada} {\bluebold{brapa}} {\bluebold{ratus}} {\bluebold{orang}}\\ %
 { }  soldier  \textsc{d.dist}  exist  several  hundred  person\\
\glt 
\textsc{qt} – \textsc{num} – \textsc{head}: ‘[one time, I brought the military (into the forest),] those soldiers were \bluebold{several hundred people}’ \textstyleExampleSource{[081029-005-Cv.0131]}
\z


The posthead modifier slots attract a wider range of constituents: verbs, nouns, prepositional phrases, and relative clauses occur in slot Post-1, personal pronouns in slot Post-2, and demonstratives, locatives, interrogatives, numerals, and quantifiers in slot Post-3. In addition, the demonstratives also occur in slot Post-4. The modifiers occurring in slot Post-1 have \isi{attributive function}, while those in slot Post-2 to Post-4 have determining function.



Co-occurrences of modifiers listed in the same slot are unattested, whereas those listed in different slots are attested to co-occur, as demonstrated in (\ref{Example_8.10}) to (\ref{Example_8.16}). In \textitbf{tangang pendek satu tu} ‘that one short-handed (one)’ in (\ref{Example_8.10}), an adnominally used stative \isi{verb} co-occurs with a \isi{numeral} and a \isi{demonstrative}. In \textitbf{babi puti ko} ‘you white pig’ in (\ref{Example_8.11}), an adnominally used \isi{verb} co-occurs with a personal \isi{pronoun}. In \textitbf{pisang Sorong sana tu} ‘those bananas (from) Sorong over there’ in (\ref{Example_8.12}), an adnominally used \isi{noun} co-occurs with a \isi{locative} and a \isi{demonstrative}. In \textitbf{pace dorang dua ini} ‘the two men here’ in (\ref{Example_8.13}), an adnominally used personal \isi{pronoun} co-occurs with a \isi{numeral} and a \isi{demonstrative}. In \textitbf{kaka dari Mambramo satu} ‘a certain older brother from (the) Mambramo (area)’ in (\ref{Example_8.14}), an adnominally used \isi{prepositional phrase} co-occurs with a \isi{numeral}. In \textitbf{dong di Papua tu} ‘they in Papua there’ in (\ref{Example_8.15}), an adnominally used \isi{prepositional phrase} co-occurs with a \isi{demonstrative}. Finally, in \textitbf{kata itu tu} ‘those very words’ in (\ref{Example_8.16}) two adnominally used demonstratives co-occur.



\begin{styleExampleTitle}
\textsc{head} – \textsc{mod}
\end{styleExampleTitle}
\ea
\label{Example_8.10}
\gll {\bluebold{tangang}} {\bluebold{pendek}} {\bluebold{satu}} {\bluebold{tu}} {((laughter))}\\ %
 hand  be.short  one  \textsc{d.dist}  \\
\glt 
\textsc{head} – \textsc{v} – \textsc{num} – \textsc{dem}: [About an acquaintance:] ‘\bluebold{that one short-armed (one)} ((laughter))’ \textstyleExampleSource{[081006-016-Cv.0036]}
\z

\ea
\label{Example_8.11}
\gll {\bluebold{babi}} {\bluebold{puti}} {\bluebold{ko}} {dari} {atas} {turung}\\ %
 pig  be.white  \textsc{2sg}  from  top  descend\\
\glt 
\textsc{head} – \textsc{v} – \textsc{pro}: [About an acquaintance:] ‘\bluebold{you white pig} came down from up (there)’ \textstyleExampleSource{[081025-006-Cv.0260]}
\z

\ea
\label{Example_8.12}
\gll {\bluebold{pisang}} {\bluebold{Sorong}} {\bluebold{sana}} {\bluebold{tu},} {iii,} {besar{\Tilde}besar} {manis}\\ %
 banana  Sorong  \textsc{l.dist}  \textsc{d.prox}  oh!  \textsc{rdp}{\Tilde}be.big  be.sweet\\
\glt 
\textsc{head} – \textsc{n} – \textsc{loc} – \textsc{dem}: ‘\bluebold{those bananas (from) Sorong over there}, oooh, (they) are all big (and) sweet’ \textstyleExampleSource{[081011-003-Cv.0017]}
\z

\ea
\label{Example_8.13}
\gll {\bluebold{pace}} {\bluebold{dorang}} {\bluebold{dua}} {\bluebold{ini}} {ke} {atas}\\ %
 man  \textsc{3pl}  two  \textsc{d.prox}  to  top\\
\glt 
\textsc{head} – \textsc{pro} – \textsc{num} – \textsc{dem}: ‘\bluebold{both of the two men here} (went) up (there)’ \textstyleExampleSource{[081006-034-CvEx.0010]}
\z

\ea
\label{Example_8.14}
\gll {trus} {tamba} {\bluebold{kaka}} {\bluebold{dari}} {\bluebold{Mambramo}} {\bluebold{satu}}\\ %
 next  add  oSb  from  Mambramo  one\\
\glt 
\textsc{head} – \textsc{pp} – \textsc{num}: [About forming a sports team:] ‘then add \bluebold{a certain older brother from (the) Mambramo (area)}’ \textstyleExampleSource{[081023-001-Cv.0002]}
\z

\ea
\label{Example_8.15}
\gll {\bluebold{dong}} {\bluebold{di}} {\bluebold{Papua}} {\bluebold{tu}} {dong} {makang} {papeda}\\ %
 \textsc{3pl}  at  Papua  \textsc{d.dist}  \textsc{3pl}  eat  sagu.porridge\\
\glt 
\textsc{head} – \textsc{pp} – \textsc{dem}: ‘\bluebold{they in Papua there}, they eat sagu porridge’ \textstyleExampleSource{[081109-009-JR.0001]}
\z

\ea
\label{Example_8.16}
\gll {\bluebold{kata}} {\bluebold{itu}} {\bluebold{tu}} {yang} {biking} {sa} {bertahang}\\ %
 word  \textsc{d.dist}  \textsc{d.dist}  \textsc{rel}  make  \textsc{1sg}  hold.(out/back)\\
\glt 
\textsc{head} – \textsc{dem} – \textsc{dem} ‘(it was) \bluebold{those very words} that made me hold out’ \textstyleExampleSource{[081115-001a-Cv.0235]}
\z


This brief overview shows that Papuan Malay employs two distinct types of \isi{noun} phrase structures: (1) a head – modifier or ``\textsc{n-mod}'' structure, and (2) a modifier – head or ``\textsc{mod-n}'' structure. The particular structure of a \isi{noun} phrase depends on the syntactic properties of its adnominal constituents:


\begin{itemize}
\item 
\textsc{n-mod} structure with adnominally used verbs, nouns, personal pronouns, demonstratives, locatives, interrogatives, prepositional phrases, and relative clauses.
\item 
\textsc{n-mod} or \textsc{mod-n} structure with adnominally used numerals and quantifiers (the constituent order depends on the semantics of the phrasal structure).

\item 
\textsc{mod-n} structure in adnominal possessive constructions.

\end{itemize}

Noun phrases with an \textsc{n-mod} structure are examined in §\ref{Para_8.2} and those with an \textsc{n-mod} or \textsc{mod-n} structure in §\ref{Para_8.3}. Adnominal possessive constructions with a \textsc{mod-n} structure are briefly mentioned in §\ref{Para_8.4}, and fully discussed in \chapref{Para_9}. In addition, \isi{apposition} is discussed in §\ref{Para_8.5}. The main points of this chapter are summarized in §\ref{Para_8.6}.


\section[{\footnotesize N-MOD} structure]{\textsc{n-mod} structure}
\label{Para_8.2}
In \isi{noun} phrases with an \textsc{n-mod} structure, the head occurs in initial position followed by the modifying elements. The following modifiers are discussed: verbs (§\ref{Para_8.2.1}), nouns (§\ref{Para_8.2.2}), personal pronouns (§\ref{Para_8.2.3}), demonstratives (§\ref{Para_8.2.4}), locatives (§\ref{Para_8.2.5}), interrogatives (§\ref{Para_8.2.6}), prepositional phrases (§\ref{Para_8.2.7}), and relative clauses (§\ref{Para_8.2.8}).


\subsection{Verbs [\textsc{n} \textsc{v}]}
\label{Para_8.2.1}
Adnominally used verbs always follow their head nominals such that ``\textsc{n} \textsc{v}'', as shown in (\ref{Example_8.17}) to (\ref{Example_8.26}). Most often, the adnominal modifier is a stative \isi{verb}, as in (\ref{Example_8.17}) to (\ref{Example_8.20}), although \isi{noun} phrases with adnominally used dynamic verbs also occur, as in (\ref{Example_8.21}) to (\ref{Example_8.24}). (The distributional preferences of attributively used stative and dynamic verbs are discussed in §\ref{Para_5.3.2}).



In \isi{noun} phrases with adnominally used stative verbs, as in (\ref{Example_8.17}) to (\ref{Example_8.20}), the head nominal is typically a bare \isi{noun} as in (\ref{Example_8.17}), or a reduplicated \isi{noun} as in (\ref{Example_8.18}). The adnominal modifier is usually a bare stative \isi{verb}, such as \textitbf{besar} ‘be big’ in (\ref{Example_8.17}) or \textitbf{panjang} ‘be long’ in (\ref{Example_8.18}). However, the modifier can also be a multi-word phrase with an overt coordinator as in \textitbf{puti dengang hitam} ‘white and black’ in (\ref{Example_8.19}), or with juxtaposed constituents as in the elicited near contrastive example in (\ref{Example_8.20}). Overall, though, multi-word modifier phrases are rare and limited to phrases with two adnominally used verbs.



\begin{styleExampleTitle}
Noun phrases with adnominal stative verbs
\end{styleExampleTitle}
\ea
\label{Example_8.17}
\gll {sa} {su} {liat} {ada} {\bluebold{pohong}} {\bluebold{besar}} {di} {depang}\\ %
 \textsc{1sg}  already  see  exist  tree  be.big  at  front\\
\glt 
‘I already saw there was a \bluebold{big tree} in front’ \textstyleExampleSource{[081025-008-Cv.0019]}
\z

\ea
\label{Example_8.18}
\gll {langsung} {\bluebold{kuku{\Tilde}kuku}} {\bluebold{panjang}} {kluar}\\ %
 immediately  \textsc{rdp}{\Tilde}digit.nail  be.long  go.out\\
\glt 
‘immediately (his) \bluebold{long claws} came out’ \textstyleExampleSource{[081115-001a-Cv.0077]}
\z

\ea
\label{Example_8.19}
\gll {sa} {{pu}} {{bapa}} {kubur} {{sa}} {pu} {tete} {pu}\\ %
 \textsc{1sg}  {\textsc{poss}}  {father}  bury  {\textsc{1sg}}  \textsc{poss}  grandfather  \textsc{poss}\\
 \gll {[[\bluebold{kaing}]}  {[\bluebold{puti}}  {\bluebold{dengang}}  {\bluebold{hitam}]]}\\
 {cloth}  {be.white}  {with}  {be.black}\\
\glt 
‘my father buried my grandfather’s \bluebold{white and black cloth}’ \textstyleExampleSource{[081014-014-NP.0047]}
\z
\ea
\label{Example_8.20}
\gll {sa} {{pu}} {{bapa}} {kubur} {sa} {pu} {tete} {pu}\\ %
 \textsc{1sg}  {\textsc{poss}}  {father}  bury  \textsc{1sg}  \textsc{poss}  grandfather  \textsc{poss}\\
 \gll {[[\bluebold{kaing}]}  {[\bluebold{hitam}}  {\bluebold{puti}]]}\\
 {cloth}  {be.black}  {be.white}\\
\glt 
‘my father buried my grandfather’s \bluebold{white (and) black cloth}’ \textstyleExampleSource{[Elicited BR130221.036]}\footnote{According to one consultant, Papuan Malay speakers prefer \textitbf{hitam puti} ‘black (and) white’ over \textitbf{puti hitam} ‘white (and) black’, although both constructions are acceptable.}
\z



Adnominally used dynamic verbs denote activities, associated with the head nominal, as in (\ref{Example_8.21}) to (\ref{Example_8.24}). The head nominal can denote an agent who carries out the activity encoded by the \isi{verb}, as with \isi{monovalent} \textitbf{jalang} ‘walk’ in (\ref{Example_8.21}), or a patient who undergoes this activity, as with \isi{bivalent} \textitbf{bakar} ‘burn’ in (\ref{Example_8.22}). The head can also express a spatial or temporal location where the activity occurs as with \isi{monovalent} \textitbf{mandi} ‘bathe’ in (\ref{Example_8.23}) and \textitbf{bangung} ‘get up’ in (\ref{Example_8.24}), respectively.



\begin{styleExampleTitle}
Noun phrases with adnominal dynamic verbs
\end{styleExampleTitle}
\ea
\label{Example_8.21}
\gll {ana} {itu} {\bluebold{tukang}} {\bluebold{jalang}}\\ %
 child  \textsc{d.dist}  craftsman  walk\\
\glt 
‘that kid \bluebold{doesn’t like staying at home}’ (Lit. ‘\bluebold{specialist (in) walk(ing)}’) \textstyleExampleSource{[080927-001-Cv.0007]}
\z

\ea
\label{Example_8.22}
\gll {pi} {ambil} {\bluebold{kayu}} {\bluebold{bakar},} {\bluebold{kayu}} {\bluebold{bakar}} {buat} {Natal}\\ %
 go  fetch  wood  burn  wood  burn  for  Christmas\\
\glt 
‘(we) went to get \bluebold{firewood}, \bluebold{firewood} for Christmas’ (Lit. ‘\bluebold{wood to burn}’) \textstyleExampleSource{[081006-017-Cv.0014]}
\z

\ea
\label{Example_8.23}
\gll {tra} {ada} {\bluebold{kamar}} {\bluebold{mandi}}\\ %
 \textsc{neg}  exist  room  bathe\\
\glt 
‘there weren’t (any) \bluebold{bathrooms}’ (Lit. ‘\bluebold{room (where) to bathe}’) \textstyleExampleSource{[081025-009a-Cv.0059]}
\z

\ea
\label{Example_8.24}
\gll {sa} {pu} {\bluebold{jam{\Tilde}jam}} {\bluebold{bangung}} {bukang} {jam} {empat}\\ %
 \textsc{1sg}  \textsc{poss}  \textsc{rdp}{\Tilde}hour  get.up  \textsc{neg}  hour  four\\
\glt 
‘my \bluebold{time to get up} is not four o’clock’ (Lit ‘\bluebold{hours (when) to wake-up}’) \textstyleExampleSource{[081025-006-Cv.0061]}
\z

Noun phrases with adnominally used verbs can further be modified with numerals. In the corpus, the adnominally used \isi{numeral} is always the \isi{numeral} \textitbf{satu} ‘one’, as in (\ref{Example_8.25}) and (\ref{Example_8.26}) (the non-enumerating function of \textitbf{satu} ‘one’ as a marker of ``specific \isi{indefiniteness}'', as in (\ref{Example_8.26}), is discussed in §\ref{Para_5.9.4}).

\begin{styleExampleTitle}
Noun phrases with adnominal verbs and numerals
\end{styleExampleTitle}
\ea
\label{Example_8.25}
\gll {[[[\bluebold{tangang}} {\bluebold{pendek}]} {\bluebold{satu}]} {\bluebold{tu}]} {((laughter))}\\ %
 hand  be.short  one  \textsc{d.dist}  \\
\glt 
[About an acquaintance:] ‘\bluebold{that one short-handed (one)} ((laughter))’ \textstyleExampleSource{[081006-016-Cv.0036]}
\z

\ea
\label{Example_8.26}
\gll {[[\bluebold{kampung}} {\bluebold{tua}]} {\bluebold{satu}]} {yang} {perna} {om} {Wili} {\ldots}\\ %
 village  be.old  one  \textsc{rel}  once  uncle  Wili  \\
\glt
‘\bluebold{a certain old village} where uncle Wili once {\ldots}’ \textstyleExampleSource{[080922-010a-CvNF.0290]}
\z


\subsection{Nouns [\textsc{n} \textsc{n}]}
\label{Para_8.2.2}
In \isi{noun} phrases with adnominally used nouns, a posthead \isi{noun} \textsc{n2} modifies the head nominal \textsc{n1}, such that ``\textsc{n1n2}''. Such constructions are characterized by the semantic subordination of the \textsc{n2} modifier under the head nominal \textsc{n1}.



In Papuan Malay, the distinction between a \isi{noun} phrase with an adnominally used \isi{noun}, hereafter \textsc{n1n2-np}, and a \isi{compound} with juxtaposed nominal constituents is not clear-cut, however. Word combinations or collocations range from two word expressions with compositional transparent semantics such as \textitbf{air sagu} ‘liquid of the sago palm tree’, to less compositional two-word expressions, such as \textitbf{kampung-tana} ‘home village’ (literally ‘village-ground’). This section focuses on \textsc{n1n2-np}s; the demarcation of such \isi{phrasal expression} from compounds, and \isi{compounding} in general, are discussed in §\ref{Para_3.2.1}.



\textsc{n1n2-np}s denote important features for subclassification of the superordinate head nominal. Typically, the head of an \textsc{n1n2-np} is a \isi{noun}, as shown in (\ref{Example_8.27}) to (\ref{Example_8.40}). Less often, the head is a deverbal constituent as in (\ref{Example_8.41}) and (\ref{Example_8.42}). Semantically, \textsc{n1n2-np}s denote a wide range of \isi{associative} relations between the \textsc{n1} and the \textsc{n2}, as shown in (\ref{Example_8.27}) to (\ref{Example_8.42}): part-whole, property-of, affiliated-with, name-of, subtype-of, composed-of, and purpose-for relations, as well as \isi{locational}, temporal, and event relations. \textsc{n1n2-np}s encode inalienable and alienable concepts.



Inalienable ``part-whole'' relations of body parts and plants are given in (\ref{Example_8.27}) and (\ref{Example_8.28}), respectively, while (\ref{Example_8.29}) illustrates an alienable ``part-whole'' relation. (More types of ``part-whole'' relations are found in Table \ref{Table_8.2} and Table \ref{Table_8.2a}.)



\begin{styleExampleTitle}
``Part-whole'' relations
\end{styleExampleTitle}
\ea
\label{Example_8.27}
\gll {sa} {bilang,} {\bluebold{tulang}} {\bluebold{bahu}} {yang} {pata}\\ %
 \textsc{1sg}  say  bone  shoulder  \textsc{rel}  break\\
\glt 
‘I said, ``(it’s my) \bluebold{shoulder bone} that is broken''' \textstyleExampleSource{[081015-005-NP.0048]}
\z

\ea
\label{Example_8.28}
\gll {adu}, {sa} {pu} {\bluebold{daung}} {\bluebold{bawang}} {itu}\\ %
 oh.no!  \textsc{1sg}  \textsc{poss}  leaf  onion  \textsc{d.dist}\\
\glt 
[After someone had plucked some onion leaves:] ‘oh no!, my \bluebold{onion leaves} there!’ \textstyleExampleSource{[081006-024-CvEx.0043]}
\z

\ea
\label{Example_8.29}
\gll {\ldots} {pukul} {\ldots} {dengang} {\bluebold{blakang}} {\bluebold{kapak}} {juga} {bisa}\\ %
 { }   hit  { }  with  backside  axe  also  be.able\\
\glt 
[About killing dogs] ‘[(it’s) also possible to bow shoot him,] to beat (him to death) {\ldots} with the \bluebold{backside of an axe} is also possible’ \textstyleExampleSource{[081106-001-CvPr.0002]}
\z



\textsc{n1n2-np}s expressing ``property-of'' and ``affiliated-with'' relations are given in (\ref{Example_8.30}) and (\ref{Example_8.31}), respectively.



\begin{styleExampleTitle}
``Property-of'' and ``affiliated-with'' relations
\end{styleExampleTitle}
\ea
\label{Example_8.30}
\gll {dari} {situ} {kembali} {ambil} {\bluebold{seng}} {\bluebold{greja}}\\ %
 from  \textsc{l.med}  return  fetch  corrugated.iron  church\\
\glt 
‘from there (I) returned (and) took \bluebold{the corrugated iron (sheets) of the church}’ \textstyleExampleSource{[080927-004-CvNP.0005]}
\z

\ea
\label{Example_8.31}
\gll {\ldots} {sa} {su} {bakar} {ruma} {itu,} {\bluebold{ruma}} {\bluebold{setang}} {itu}\\ %
  { } \textsc{1sg}  already  burn  house  \textsc{d.dist}  house  evil.spirit  \textsc{d.dist}\\
\glt 
‘[(if) I, umh, for example, were in Aruswar or Niwerawar,] I would already have burnt that house, that \bluebold{evil spirit’s house}’ \textstyleExampleSource{[081025-009a-Cv.0198]}
\z



``Name-of'' relations are presented in (\ref{Example_8.32}) and (\ref{Example_8.33}). (Other types of ``name-of'' relations are found in Table \ref{Table_8.2} and Table \ref{Table_8.2a}.)




\ea 
``Name-of'' relations \\
\ea
\label{Example_8.32}
\gll {yo} {bapa,} {\bluebold{hari}} {\bluebold{minggu}} {sa} {datang}\\ %
 yes  father  day  Sunday  \textsc{1sg}  come\\
\glt 
‘yes father, on \bluebold{Sunday} I’ll come’ (Lit. ‘\bluebold{Sunday day}’) \textstyleExampleSource{[080922-001a-CvPh.0344]}
\vspace{5pt}
\ex
\label{Example_8.33}
\gll {knapa} {ko} {gambar} {monyet} {di} {bawa} {\bluebold{pohong}} {\bluebold{pisang}}?\\ %
 why  \textsc{2sg}  draw  monkey  at  under  tree  banana\\
\glt 
‘why did you draw the monkey under \bluebold{the banana tree}?’ \textstyleExampleSource{[081109-002-JR.0004]}
\z
\z



``Subtype-of'' relations are presented in (\ref{Example_8.34}) to (\ref{Example_8.35}).


\ea 
``Subtype-of'' relations\\
\ea
\label{Example_8.34}
\gll {\ldots} {maka} {pake} {[[\bluebold{bahasa}]} {\bluebold{orang}} {\bluebold{bisu}]}\\ %
 { }  therefore  use  language  person  be.mute\\
\glt ‘[she couldn’t speak the Indonesian language,] therefore (she) used \bluebold{sign language}’ (Lit. ‘\bluebold{language of mute people}’) \textstyleExampleSource{[081006-023-CvEx.0073]}
\vspace{5pt}
\ex
\label{Example_8.35}
\gll {\ldots} {supaya} {Sarmi} {ada} {[[\bluebold{petinju}} {\bluebold{prempuang}]} {satu]}\\ %
 { }  so.that  \ili{Sarmi}  exist  boxer  woman  one\\
\glt 
‘{\ldots} so that \ili{Sarmi} has a certain \bluebold{woman boxer}’ \textstyleExampleSource{[081023-003-Cv.0005]}
\z
\z


\textsc{n1n2-np}s expressing ``composed-of'' and ``purpose-for'' relations are illustrated in (\ref{Example_8.36}) and (\ref{Example_8.37}), respectively.




\ea 
``Composed-of'' and ``purpose-for'' relations\\
\ea
\label{Example_8.36}
\gll {smua} {jalang} {kaya} {\bluebold{kapal}} {\bluebold{kayu}}\\ %
 all  walk  like  ship  wood\\
\glt 
‘(they) all were strolling around like \bluebold{wooden boats}’ \textstyleExampleSource{[081025-009a-Cv.0188]}
\vspace{5pt}
\ex
\label{Example_8.37}
\gll {yo,} {\bluebold{net}} {\bluebold{laki{\Tilde}laki}} {tong} {yang} {bli}\\ %
 yes  (sport.)net  \textsc{rdp}{\Tilde}husband  \textsc{1pl}  \textsc{rel}  buy\\
\glt 
‘yes, the (volley-ball) \bluebold{net for men}, (it was) us who (bought it)’ \textstyleExampleSource{[081023-001-Cv.0012]}
\z
\z



Locational and temporal relations between the \textsc{n1} and \textsc{n2} are illustrated in (\ref{Example_8.38}) to (\ref{Example_8.40}). The \textsc{n2} denotes a \isi{locational} relation in (\ref{Example_8.38}), and a temporal relation in (\ref{Example_8.39}). In (\ref{Example_8.40}) the first two nominals express a \isi{locational} relation between the head \textitbf{ampas} ‘waste’ and its modifier \isi{noun}, the source \textitbf{pinang} ‘betel nut’. This \textsc{n1n2} construction is modified with the third nominal \textitbf{malam} ‘night’ which denotes a temporal relation (\textsc{n1n2-np}s with more than three nominal constituents are unattested in the corpus). (Other types of \isi{locational} relations are found in \tabref{Table_8.2}.)


\ea 
Locational and temporal relations \\
\ea
\label{Example_8.38}
\gll {orang} {Papua} {bilang} {\bluebold{jing}} {\bluebold{kayu}}\\ %
 person  Papua  say  genie  wood\\
\glt 
‘Papuans call (them) \bluebold{tree genies}’ \textstyleExampleSource{[081006-022-CvEx.0054]}
\vspace{5pt}
\ex
\label{Example_8.39}
\gll {[[\bluebold{jam}} {\bluebold{tiga}]} {\bluebold{pagi}]?}\\ %
 hour  three  morning\\
\glt 
‘(was it) \bluebold{three o’clock in the morning}?’ \textstyleExampleSource{[080918-001-CvNP.0042]}
\vspace{5pt}
\ex
\label{Example_8.40}
\gll {[[[\bluebold{ampas}} {\bluebold{pinang}]} {\bluebold{malam}]} {tu]} {sa} {taru}\\ %
 waste  betel.nut  night  \textsc{d.dist}  \textsc{1sg}  put\\
\glt 
‘that \bluebold{evening’s betel nut waste}, I put (it aside)’ \textstyleExampleSource{[081025-006-Cv.0294]}
\z
\z


An \textsc{n1n2-np} can also be formed with a deverbal nominal head as in (\ref{Example_8.41}) and (\ref{Example_8.42}). Semantically, the \textsc{n1n2-np} in (\ref{Example_8.41}) expresses an event relation in which adnominal \textitbf{tugu} ‘monument’ is affected by the event expressed by the deverbal head \textsc{n1}. The \textsc{n1n2-np} in (\ref{Example_8.42}) denotes a \isi{locational} relation with the deverbal head \textsc{n1} originating from the nominal spatial source \textsc{n2}.


\ea 
Subordinate \textsc{n1n2-np}s with deverbal constituent \\
\ea
\label{Example_8.41}
\gll {ada} {[[\bluebold{pasang}} {\bluebold{tugu}]} {\bluebold{itu}]}\\ %
 exist  install  monument  \textsc{d.dist}\\
\glt 
[Giving directions:] ‘there is \bluebold{that statue installation}’ \textstyleExampleSource{[080917-008-NP.0017]}
\vspace{5pt}
\ex
\label{Example_8.42}
\gll {kalo} {angkat} {air} {jemur} {di} {\bluebold{panas}} {\bluebold{mata-hari}}\\ %
 if  lift  water  dry  at  be.hot  sun\\
\glt 
‘when (you) fetch water, warm (it) up in \bluebold{the heat of the sun}’ \textstyleExampleSource{[081006-013-Cv.0005]}
\z
\z


\tabref{Table_8.2} and  \tabref{Table_8.2a}  give an overview of the different \isi{associative} meaning relations expressed with \textsc{n1n2-np}s.


\begin{table}
\caption{Associative meaning relations encoded by \textsc{n\oldstylenums{1}n\oldstylenums{2}-np}s
%\todo[inline]{Set n1 and n2 in oldstylenums?}
}\label{Table_8.2}
\begin{tabularx}{\textwidth}{llll}
\lsptoprule
 & \multicolumn{1}{c}{Papuan Malay \textsc{n\oldstylenums{1}n\oldstylenums{2}}} & \multicolumn{1}{c}{Glosses} &  \multicolumn{1}{c}{Free translations}\\
\midrule
\stepcounter{InTableCounter0} \arabic{InTableCounter0}. & \multicolumn{3}{p{11 cm}}{``Part-whole'' relation – \textsc{n}\oldstylenums{1} is a part of \textsc{n}\oldstylenums{2}: (a) human body part, (b) nonhuman body part, (c) plant part, (d) spatial location of a concrete object, (e) temporal location of an abstract object, (f) time segment within a time period, (g) member of an institution}\\
\stepcounter{InTableCounter1} (\alph{InTableCounter1}) & \textitbf{urat kaki} & tendon foot & ‘foot tendon’\\
\stepcounter{InTableCounter1} (\alph{InTableCounter1}) & \textitbf{duri ikang} & thorn fish & {‘fish bone’}\\
\stepcounter{InTableCounter1} (\alph{InTableCounter1}) & \textitbf{pelepa sagu} & stem sago & {‘sago stem’}\\
\stepcounter{InTableCounter1} (\alph{InTableCounter1}) & \textitbf{blakang kapak} & backside axe & {‘backside of an axe’}\\
\stepcounter{InTableCounter1} (\alph{InTableCounter1}) & \textitbf{tenga sembayang} & middle worship & {‘middle of the worship’}\\
\stepcounter{InTableCounter1} (\alph{InTableCounter1}) & \textitbf{malam hari} & night day & {‘evening (of the day)’}\\
\stepcounter{InTableCounter1} (\alph{InTableCounter1}) & \textitbf{petugas polisi} & official police & {‘police official’}\\
\midrule
\stepcounter{InTableCounter0} \arabic{InTableCounter0}. & \multicolumn{3}{p{11 cm}}{``Property-of''  relation – \textsc{n1} is a property of \textsc{n2}}\\
& \textitbf{ruma orang} & house person & {‘(other) people’s house’}\\
& \textitbf{cara orang Papua} & way person Papua & {‘Papuan traditions’}\\
\midrule
\stepcounter{InTableCounter0} \arabic{InTableCounter0}. & \multicolumn{3}{p{11 cm}}{``Affiliated-with'' relation: \textsc{n1} is affiliated with \textsc{n2}}\\
& \textitbf{ruma setang} & house evil.spirit & {‘house of an evil spirit’}\\
& \textitbf{ana{\Tilde}ana iblis} & \textsc{rdp}{\Tilde}child devil & {‘children of the devil’}\\
\midrule
\stepcounter{InTableCounter0} \arabic{InTableCounter0}. & \multicolumn{3}{p{11cm}}{``Name-of''relation – \textsc{n2} designates the name of \textsc{n1}: (a) animal, (b) plant, (c) personal name, (d) clan/ethnic group, (e) disease, (f) building/institution, (g) language, (h) religion, (i) spatial location, (j) temporal location}\\
\stepcounter{InTableCounter1} (\alph{InTableCounter1}) & \textitbf{ikang gurango} & fish shark & {‘shark fish’}\\
\stepcounter{InTableCounter1} (\alph{InTableCounter1}) & \textitbf{sayur bayam} & vegetable amaranth & {‘amaranth vegetable’}\\
\stepcounter{InTableCounter1} (\alph{InTableCounter1}) & \textitbf{nama Nofela} & name Nofela & {‘(of the) name Nofela}\\
\stepcounter{InTableCounter1} (\alph{InTableCounter1}) & \textitbf{marga Sope} & clan Sope & {‘Sope clan’}\\
\stepcounter{InTableCounter1} (\alph{InTableCounter1}) & \textitbf{penyakit malaria} & disease malaria & {‘malaria disease’}\\
\stepcounter{InTableCounter1} (\alph{InTableCounter1}) & \textitbf{greja Kema-Injil} & church Kema-Injil & {‘Kema-Injil church’}\\
\stepcounter{InTableCounter1} (\alph{InTableCounter1}) & \textitbf{bahasa Inggris} & language England & {‘English language’}\\
\stepcounter{InTableCounter1} (\alph{InTableCounter1}) & \textitbf{agama Kristen} & religion Christian & {‘Christian religion’}\\
\stepcounter{InTableCounter1} (\alph{InTableCounter1}) & \textitbf{kota Sarmi} & city \ili{Sarmi} & {‘\ili{Sarmi} city’}\\
\stepcounter{InTableCounter1} (\alph{InTableCounter1}) \setcounter{ItemCounter}{\value{InTableCounter0}} & \textitbf{hari kamis} & day Thursday & {‘Thursday’}\\
\lspbottomrule
\end{tabularx}
\end{table}

\begin{table}[t]
\caption{Associative meaning relations encoded by \textsc{n\oldstylenums{1}n\oldstylenums{2}-np}s continued}\label{Table_8.2a}
\begin{tabularx}{\textwidth}{llll}
\lsptoprule
 & \multicolumn{1}{c}{Papuan Malay \textsc{n\oldstylenums{1}n\oldstylenums{2}}} & \multicolumn{1}{c}{Glosses} &  \multicolumn{1}{c}{Free translation}\\
\midrule
\setcounter{InTableCounter0}{\value{ItemCounter}} \stepcounter{InTableCounter0} \arabic{InTableCounter0}. & \multicolumn{3}{p{11 cm}}{``Subtype-of'' relation – \textsc{n2} designates a specific type of \textsc{n1}}\\
& \textitbf{ana murit} & child school & {‘school kid’}\\
& \textitbf{kaing sprey} & cloth bed sheet & {‘bed sheets’}\\
\midrule
\stepcounter{InTableCounter0} \arabic{InTableCounter0}. & \multicolumn{3}{p{11 cm}}{``Composed-of'' relation – \textsc{n\oldstylenums{1}} is composed of / made from \textsc{n\oldstylenums{2}}}\\
& \textitbf{ruma batu} & house stone & {‘stone house’}\\
& \textitbf{kantong plastik} & bag plastic & {‘plastic bag’}\\
\midrule
\stepcounter{InTableCounter0} \arabic{InTableCounter0}. & \multicolumn{3}{p{11 cm}}{``Purpose-for'' relation: \textsc{n1} is intended for / at the disposal of \textsc{n2}}\\
& \textitbf{net laki{\Tilde}laki} & net \textsc{rdp}{\Tilde}husband & {‘(volleyball) net for men’}\\
& \textitbf{sikat gigi} & brush tooth & {‘toothbrush’}\\
\midrule
\stepcounter{InTableCounter0} \arabic{InTableCounter0}. & \multicolumn{3}{p{11 cm}}{Locational relation: (a) \textsc{n\oldstylenums{1}} contains \textsc{n\oldstylenums{2}}; (b) \textsc{n\oldstylenums{1}} is located at/in/on \textsc{n\oldstylenums{2}}; (c) \textsc{n\oldstylenums{1}} originates from spatial source \textsc{n\oldstylenums{2}}; (d) \textsc{n\oldstylenums{1}} originates from nonspatial source \textsc{n\oldstylenums{2}}}\\
\stepcounter{InTableCounter1} (\alph{InTableCounter1}) & \textitbf{lampu gas} & lamp gas & {‘gas lamp’}\\
\stepcounter{InTableCounter1} (\alph{InTableCounter1}) & \textitbf{jing{\Tilde}jing kayu} & \textsc{rdp}{\Tilde}genie wood & {‘tree genies’}\\
\stepcounter{InTableCounter1} (\alph{InTableCounter1}) & \textitbf{pisang Sorong} & banana Sorong & {‘bananas from Sorong’}\\
\stepcounter{InTableCounter1} (\alph{InTableCounter1}) & \textitbf{mop orang Sarmi} & joke people \ili{Sarmi} & {‘joke by the \ili{Sarmi} people’}\\
\midrule
\stepcounter{InTableCounter0} \arabic{InTableCounter0}. & \multicolumn{3}{p{11 cm}}{Temporal relation – \textsc{n\oldstylenums{2}} gives temporal specifications for \textsc{n\oldstylenums{1}}}\\
& \textitbf{jam dua pagi} & hour two morning & {‘two o’clock in the morning’}\\
& \textitbf{hari sening depang} & day Monday front & {‘next Monday’}\\
\midrule
\stepcounter{InTableCounter0} \arabic{InTableCounter0}. & \multicolumn{3}{p{11 cm}}{Event relation: \textsc{n\oldstylenums{2}} is affected by event \textsc{n\oldstylenums{1}}}\\
& \textitbf{pasang tugu} & install monument & {‘statue installation’}\\
\lspbottomrule

\end{tabularx}
\end{table}



\subsection{Personal pronouns [\textsc{n} \textsc{pro}]}
\label{Para_8.2.3}
Papuan Malay \isi{noun} phrases are often modified with personal pronouns in posthead position, such that ``\textsc{n} \textsc{pro}'', as illustrated in (\ref{Example_8.43}) to (\ref{Example_8.45}). Signaling the definiteness, person, and number of their referents, the adnominally used personal pronouns allow the unambiguous identification of their referents. They are available for all person-number values, except for first person singular \textitbf{saya}/\textitbf{sa} ‘\textsc{1sg}’ which is unattested; the long and short \isi{pronoun} forms are used interchangeably. (Personal pronouns and their adnominal uses are discussed in detail in \chapref{Para_6}; see also §\ref{Para_5.5}).



Adnominal singular personal pronouns indicate the singularity of their referents, as shown with \textitbf{ko} ‘\textsc{2sg}’ and \textitbf{dia} ‘\textsc{3sg}’ in (\ref{Example_8.43}). In addition, they have pertinent discourse functions, discussed in detail in §\ref{Para_6.2.1}. Noun phrases with adnominally used plural personal pronouns have two readings. With an \isi{indefinite} referent, such as \textitbf{laki{\Tilde}laki} ‘man’ in (\ref{Example_8.44}), the \isi{noun} phrase receives an additive plural reading. With a \isi{definite} referent such as \textitbf{Roni} in (\ref{Example_8.45}), the \isi{noun} phrase receives an \isi{associative} inclusory plural reading. Both readings are discussed in detail in §\ref{Para_6.2.2}.



\begin{styleExampleTitle}
Noun phrases with adnominal personal pronouns
\end{styleExampleTitle}
\ea
\label{Example_8.43}
\gll {\bluebold{Wili}} {\bluebold{ko}} {jangang} {gara{\Tilde}gara} {\bluebold{tanta}} {\bluebold{dia}} {\bluebold{itu}}!\\ %
 Wili  \textsc{2sg}  \textsc{neg.imp}  \textsc{rdp}{\Tilde}irritate  aunt  \textsc{3sg}  \textsc{d.dist}\\
\glt 
‘\bluebold{you Wili} don’t irritate \bluebold{that aunt}!’ \textstyleExampleSource{[081023-001-Cv.0038]}
\z

\ea
\label{Example_8.44}
\gll {jadi} {\bluebold{laki{\Tilde}laki}} {\bluebold{kitong}} {harus} {bayar} {spulu} {juta} {sama} {\ldots}\\ %
 so  \textsc{rdp}{\Tilde}husband  \textsc{1pl}  have.to  pay  ten  million  with  \\
\glt 
‘so \bluebold{we men} have to pay ten million to {\ldots}’ \textstyleExampleSource{[081110-005-CvPr.0107]}
\z

\ea
\label{Example_8.45}
\gll {\bluebold{Roni}} {\bluebold{dong}} {kas} {tinggal} {itu} {babi} {di} {sini}\\ %
 Roni  \textsc{3pl}  give  stay  \textsc{d.dist}  pig  at  \textsc{l.prox}\\
\glt
‘\bluebold{Roni and the others including Roni} left, what’s-its-name, the pig here’ \textstyleExampleSource{[080917-008-NP.0135]}
\z


\subsection{Demonstratives [\textsc{n} \textsc{dem}]}
\label{Para_8.2.4}
Within the \isi{noun} phrase, adnominally used demonstratives are placed at the right periphery, where they have scope over the entire \isi{noun} phrase, such that ``\textsc{n} \textsc{dem}'': proximal \textitbf{ini} ‘\textsc{d.prox}’ or distal \textitbf{itu} ‘\textsc{d.dist}’, or their respective reduced forms \textitbf{ni} ‘\textsc{d.prox}’ and \textitbf{tu} ‘\textsc{d.dist}’. Like adnominally used personal pronouns (§\ref{Para_8.2.3}), the adnominal demonstratives function as determiners. Unlike the personal pronouns, however, they signal specificity rather than definiteness.\footnote{Concerning the semantic distinctions between the notion of definiteness and the notion of specificity, see Footnote \ref{Footnote_5.164} in §\ref{Para_5.5} (p. \pageref{Footnote_5.164}).
}
% \todo{check crossref and link}



The head nominal can be a \isi{noun} such as \textitbf{ana} ‘child’ in (\ref{Example_8.46}), a personal \isi{pronoun} such as \textitbf{dia} ‘\textsc{3sg}’ in (\ref{Example_8.46}), a \isi{locative} such as \textitbf{sana} ‘\textsc{l.dist}’ in (\ref{Example_8.47}), or another \isi{demonstrative} such as \textitbf{itu} ‘\textsc{d.dist}’ in (\ref{Example_8.48}). (Demonstratives and their adnominal uses are discussed in detail in §\ref{Para_7.1}; see also §\ref{Para_5.6}).


\ea
\label{Example_8.46}
\gll {\bluebold{ana}} {\bluebold{itu}} {sa} {paling} {sayang} {\bluebold{dia}} {\bluebold{tu}}\\ %
 child  \textsc{d.dist}  \textsc{1sg}  most  love  \textsc{3sg}  \textsc{d.dist}\\
\glt 
‘\bluebold{that child}, I love \bluebold{her (}\blueboldSmallCaps{emph}\bluebold{)} most’ \textstyleExampleSource{[081011-023-Cv.0097]}
\z

\ea
\label{Example_8.47}
\gll {sana,} {te} {ada} {di} {\bluebold{sana}} {\bluebold{itu}}\\ %
 \textsc{l.dist}  tea  exist  at  \textsc{l.dist}  \textsc{d.dist}\\
\glt 
‘there, the tea is \bluebold{over there (}\blueboldSmallCaps{emph}\bluebold{)}’ \textstyleExampleSource{[081014-011-CvEx.0010]}
\z

\ea
\label{Example_8.48}
\gll {\ldots} {\bluebold{itu}} {\bluebold{tu}} {kata{\Tilde}kata} {dasar} {yang} {harusnya} {kamu} {taw}\\ %
 { }  \textsc{d.dist}  \textsc{d.dist}  \textsc{rdp}{\Tilde}word  base  \textsc{rel}  appropriately  \textsc{2pl}  know\\
\glt
[Addressing a school student:] ‘[do you know the (English) word ``please'' or not?,] \bluebold{that very} (word belongs to) the basic words that you should know’ \textstyleExampleSource{[081115-001a-Cv.0145]}
\z


\subsection{Locatives [\textsc{n} \textsc{loc}]}
\label{Para_8.2.5}
Adnominally used locatives occur in posthead position, such that ``\textsc{n} \textsc{loc}''. This is illustrated with proximal \textitbf{sini} ‘\textsc{l.prox}’ in (\ref{Example_8.49}), and distal \textitbf{sana} ‘\textsc{l.dist}’, as in (\ref{Example_8.50}). The head nominal may be a \isi{noun} such as \textitbf{ana} ‘child’ in (\ref{Example_8.49}), or a personal \isi{pronoun} such as \textitbf{dong} ‘\textsc{3pl}’ in (\ref{Example_8.50}). (A detailed discussion on locatives and their adnominal uses is found in §\ref{Para_7.2}; see also §\ref{Para_5.7}).


\ea
\label{Example_8.49}
\gll {kamu} {\bluebold{ana{\Tilde}ana}} {\bluebold{sini}} {\bluebold{tu}} {enak} {skali}\\ %
 \textsc{2pl}  \textsc{rdp}{\Tilde}child  \textsc{l.prox}  \textsc{d.dist}  be.pleasant  very\\
\glt 
‘you, \bluebold{the young people here (}\blueboldSmallCaps{emph}\bluebold{)}, (live) very pleasant (lives)’ \textstyleExampleSource{[081115-001b-Cv.0060]}
\z

\ea
\label{Example_8.50}
\gll {\bluebold{dong}} {\bluebold{sana}} {cari} {anging}\\ %
 \textsc{3pl}  \textsc{l.dist}  search  wind\\
\glt
‘\bluebold{they over there} are looking for a breeze’ \textstyleExampleSource{[081025-009b-Cv.0076]}
\z


\subsection{Interrogatives [\textsc{n} \textsc{int}]}
\label{Para_8.2.6}
In their adnominal uses, the interrogatives occur in posthead position, such that ``\textsc{n} \textsc{int}''. Syntactically, the interrogatives remain in-situ; that is, \isi{noun} phrases with adnominally used interrogatives correspond to their non-\isi{interrogative} expressions. This is illustrated with \textitbf{siapa} ‘who’ in (\ref{Example_8.51}), \textitbf{apa} ‘what’ in (\ref{Example_8.52}), and \textitbf{mana} ‘where, which’ in (\ref{Example_8.53}).



The interrogatives always occur in \isi{noun} phrases with a nominal head such as the common \isi{noun} \textitbf{kaka} ‘older sibling’ in (\ref{Example_8.51}). Modification of personal pronouns or other constituents is unattested. (More details on the interrogatives and their adnominal uses are found in §\ref{Para_5.8}).


\ea
\label{Example_8.51}
\gll {skarang} {sa} {tanya,} {[\bluebold{orang}} {\bluebold{siapa}]} {yang} {benar?}\\ %
 now  \textsc{1sg}  ask  person  who  \textsc{rel}  be.true\\
\glt 
‘now I asked, ``\bluebold{which person} (is the one) who is right?''' \textstyleExampleSource{[080917-010-CvEx.0197]}
\z

\ea
\label{Example_8.52}
\gll {[\bluebold{hari}} {\bluebold{apa}]} {yang} {sa} {ketemu} {dia} {e?}\\ %
 day  what  \textsc{rel}  \textsc{1sg}  meet  \textsc{3sg}  eh\\
\glt 
‘\bluebold{which day} (is the one) that I met her, eh?’ \textstyleExampleSource{[080922-004-Cv.0013]}
\z

\ea
\label{Example_8.53}
\gll {sa} {tanya} {dia,} {di} {[\bluebold{posisi}} {\bluebold{mana}]} {skarang?}\\ %
 1sg  ask  3sg  at  position  where  now\\
\glt
‘I asked him, ``\bluebold{which position} (is the one that are you) at now?''' \textstyleExampleSource{[081011-008-Cv.0023]}
\z


\subsection{Prepositional phrases [\textsc{n} \textsc{pp}]}
\label{Para_8.2.7}
Noun phrases can be modified with prepositional phrases, such that ``\textsc{n} \textsc{pp}''. Overall, however, such \isi{noun} phrases are uncommon. In the corpus, four prepositions occur in adnominally used prepositional phrases, namely \isi{locative} \textitbf{di} ‘at, in’ as in (\ref{Example_8.15}), repeated as (\ref{Example_8.54}), elative \textitbf{dari} ‘from’ as in (\ref{Example_8.55}), benefactive \textitbf{untuk} ‘for’ as in (\ref{Example_8.56}), and similative \textitbf{sperti} ‘like’ as in (\ref{Example_8.57}). (For a detailed discussion on prepositions and prepositional phrases see \chapref{Para_10}; see also §\ref{Para_5.11}).



\begin{styleExampleTitle}
Noun phrases with adnominal prepositional phrases
\end{styleExampleTitle}
\ea
\label{Example_8.54}
\gll {\bluebold{dong}} {\bluebold{di}} {\bluebold{Papua}} {\bluebold{tu}} {dong} {makang} {papeda}\\ %
 \textsc{3pl}  at  Papua  \textsc{d.dist}  \textsc{3pl}  eat  sagu.porridge\\
\glt 
‘\bluebold{they in Papua there}, they eat sagu porridge’ \textstyleExampleSource{[081109-009-JR.0001]}
\z

\ea
\label{Example_8.55}
\gll {itu} {\bluebold{iblis{\Tilde}iblis}} {\bluebold{dari}} {\bluebold{ruangang}} {\bluebold{ini}} {yang} {ganggu}\\ %
 \textsc{d.dist}  \textsc{rdp}{\Tilde}devil  from  room  \textsc{d.prox}  \textsc{rel}  disturb\\
\glt 
‘it’s \bluebold{the devils from this room} who are disturbing (you)’ \textstyleExampleSource{[081011-008-CvPh.0018]}
\z

\ea
\label{Example_8.56}
\gll {di} {sana} {kang} {masi} {\bluebold{tempat}} {\bluebold{untuk}} {\bluebold{kafir}}\\ %
 at  \textsc{l.dist}  you.know  still  place  for  unbeliever\\
\glt 
‘(the area) over there, you know, is still \bluebold{a location for unbelievers}’ \textstyleExampleSource{[081011-022-Cv.0238]}
\z

\ea
\label{Example_8.57}
\gll {\bluebold{orang{\Tilde}orang}} {\bluebold{sperti}} {\bluebold{begitu}} {yang} {tida} {mengenal} {Kristus}\\ %
 \textsc{rdp}{\Tilde}person  like  like.that  \textsc{rel}  \textsc{neg}  know  Kristus\\
\glt
‘(it’s) \bluebold{people like those} who don’t know Christ {\ldots}’ \textstyleExampleSource{[081006-023-CvEx.0034]}
\z


\subsection{Relative clauses [\textsc{n} \textsc{rc}]}
\label{Para_8.2.8}
Relative clauses are introduced with the \isi{relativizer} \textitbf{yang} ‘\textsc{rel}’. They always follow their head nominal, such that ``\textsc{n} \textsc{rc}''. The head nominal can be a \isi{noun} as in (\ref{Example_8.58}), a personal \isi{pronoun} as in (\ref{Example_8.59}), a \isi{demonstrative} as in (\ref{Example_8.60}), a \isi{locative} as in (\ref{Example_8.61}), or an \isi{interrogative} as in (\ref{Example_8.62}). The syntax of relatives clauses is discussed in detail in §\ref{Para_14.3.2} (see also \chapref{Para_5} for the respective word class sections, as well as \chapref{Para_6}, and \chapref{Para_7}).


\ea
\label{Example_8.58}
\gll {\ldots} {tapi} {di} {sini} {\bluebold{prempuang}} {\bluebold{yang}} {tokok}\\ %
  { }  but  at  \textsc{l.prox}  woman  \textsc{rel}  tap\\
\glt 
‘[at Pante-Timur all the men pound (sago),] but here (it’s) \bluebold{the women who} pound (sago)’ \textstyleExampleSource{[081014-007-CvEx.0073]}
\z

\ea
\label{Example_8.59}
\gll {a,} {\bluebold{ko}} {\bluebold{yang}} {tanya} {to?}\\ %
 ah!  \textsc{2sg}  \textsc{rel}  ask  right?\\
\glt 
‘ah, (it was) \bluebold{you who} asked, right?’ \textstyleExampleSource{[080923-014-CvEx.0010]}
\z

\ea
\label{Example_8.60}
\gll {\bluebold{itu}} {\bluebold{yang}} {orang} {Papua} {skarang} {maw}\\ %
 \textsc{d.dist}  \textsc{rel}  person  Papua  now  want\\
\glt 
‘(it’s) \bluebold{that what} Papuans want nowadays’ \textstyleExampleSource{[081025-004-Cv.0077]}
\z

\ea
\label{Example_8.61}
\gll {di} {\bluebold{sini}} {\bluebold{yang}} {tra} {banyak}\\ %
 at  \textsc{l.prox}  \textsc{rel}  \textsc{neg}  many\\
\glt 
[About logistic problems:] ‘(it’s) \bluebold{here where} there weren’t many (passengers)’ \textstyleExampleSource{[081025-008-Cv.0140]}
\z

\ea
\label{Example_8.62}
\gll {kamu} {tida} {perna} {dengar} {\bluebold{apa}} {\bluebold{yang}} {orang-tua} {bicara}\\ %
 \textsc{2pl}  \textsc{neg}  ever  listen  what  \textsc{rel}  parent  speak\\
\glt
‘because you never listened to \bluebold{what} the elders said’ \textstyleExampleSource{[081115-001a-Cv.0338]}
\z


\section[{\footnotesize N-MOD}/{\footnotesize MOD-N} structure]{\textsc{n-mod} / \textsc{mod-n} structure}
\label{Para_8.3}
Noun phrases with adnominally used numerals or quantifiers can have an \textsc{n-mod} or a \textsc{mod-n} structure, depending on the semantics of the phrasal structure. When preposed, adnominal numerals and quantifiers signal individuality, while postposed numerals and quantifiers express exhaustivity or positions within series. Posthead numerals and quantifiers have scope over their head nominal including its verbal and/or nominal modifiers, while they, in turn, are within the scope of the demonstratives. Adnominally used numerals are discussed in §\ref{Para_8.3.1}, and adnominal quantifiers in §\ref{Para_8.3.2}.


\subsection{Numerals [\textsc{n} \textsc{num} / \textsc{num} \textsc{n}]}
\label{Para_8.3.1}
Two types of \isi{noun} phrases with adnominally used numerals can be distinguished: (1) \isi{noun} phrases with prehead numerals, such that ``\textsc{n}um \textsc{n}'', are presented in (\ref{Example_8.63}) to (\ref{Example_8.66}), and (2) \isi{noun} phrases with posthead numerals, such that ``\textsc{n} \textsc{n}um'', are illustrated in (\ref{Example_8.67}) to (\ref{Example_8.70}). (For a discussion of numerals as a word class see §\ref{Para_5.9}.)



Noun phrases with preposed numerals (``\textsc{n}um\textsc{n-np}'') express a sense of individuality by signaling the composite nature of their referents. This is achieved in that \textsc{n}um\textsc{n-np}s denote absolute numbers of items expressed by their head nominals, including quantities as in (\ref{Example_8.63}) or periods of time as in (\ref{Example_8.64}).



\begin{styleExampleTitle}
\textsc{n}um\textsc{n-np}s denoting \isi{definite} quantities of countable referents: Individuality
\end{styleExampleTitle}
\ea
\label{Example_8.63}
\gll {\ldots} {brarti} {suda} {\bluebold{empat}} {\bluebold{orang}} {bisa} {masuk}\\ %
 { }  mean  already  four  person  be.able  enter\\
\glt 
[About local elections:] ‘{\ldots} that means that already \bluebold{four people} can be included (in the list of nominees)’ \textstyleExampleSource{[080919-001-Cv.0149]}
\z

\ea
\label{Example_8.64}
\gll {{ini}} {{untuk}} {{balita}} {{dang}} {{bayi}} {yang} {usia} {dari}\\ %
 {\textsc{d.prox}}  {for}  {children.under.five}  {and}  {baby}  \textsc{rel}  age  from\\
\gll \bluebold{lima}  {\bluebold{taung}}  {ke}  bawa  {sampe}  {\bluebold{dua}}  {\bluebold{bulang}}\\
 five  {year}  {to}  bottom  {until}  {two}  {month}\\
\glt 
‘this is for children and babies who are \bluebold{five years} down to \bluebold{two months}’ \textstyleExampleSource{[081010-001-Cv.0197]}
\z



If the exact absolute number of items is unknown, two numerals can be juxtaposed to indicate approximate quantities, as in (\ref{Example_8.65}) and (\ref{Example_8.66}). The approximated quantities are usually rather small, such as \textitbf{satu dua} ‘one or two’ in (\ref{Example_8.65}) or \textitbf{tiga empat} ‘three or four’ as in (\ref{Example_8.66}).


\begin{styleExampleTitle}
\textsc{n}um\textsc{n-n}um\textsc{n-np}s denoting approximate quantities
\end{styleExampleTitle}
\ea
\label{Example_8.65}
\gll {jangang} {ko} {lama} {ko} {\bluebold{satu}} {\bluebold{dua}} {\bluebold{hari}} {saja}\\ %
 \textsc{neg.imp}  \textsc{2sg}  be.long  \textsc{2sg}  one  two  day  just\\
\glt 
‘don’t (stay) long, just \bluebold{one} or \bluebold{two days}’ \textstyleExampleSource{[080922-001a-CvPh.0736]}
\z

\ea
\label{Example_8.66}
\gll {\bluebold{tiga}} {\bluebold{empat}} {\bluebold{kluarga}} {harus} {ada} {di} {situ}\\ %
 three  four  family  have.to  exist  at  \textsc{l.med}\\
\glt 
‘\bluebold{three} or \bluebold{four families} have to be there’ \textstyleExampleSource{[080923-007-Cv.0018]}
\z



Noun phrases with posthead numerals (``\textsc{nn}um\textsc{{}-np}'') signal exhaustivity of \isi{definite} referents, as in (\ref{Example_8.67}) and (\ref{Example_8.68}), or mark unique positions within series or sequences as in (\ref{Example_8.69}) and (\ref{Example_8.70}).



With head nominals undifferentiated in terms of their ranking, \textsc{nn}um\textsc{{}-np}s indicate exhaustivity of \isi{definite} referents. The head can be a \isi{noun}, as in the elicited example in (\ref{Example_8.67}), or a personal \isi{pronoun} as in (\ref{Example_8.68}).\footnote{The elicited example in (\ref{Example_8.67}) is based on the example in (\ref{Example_8.14}) in §\ref{Para_8.1} (p. \pageref{Example_8.14}).}



\begin{styleExampleTitle}
\textsc{nn}um\textsc{{}-np}s denoting \isi{definite} quantities of countable referents: Exhaustivity
\end{styleExampleTitle}
\ea
\label{Example_8.67}
\gll {trus} {tamba} {[[[\bluebold{kaka}} {\bluebold{dari}} {\bluebold{Mambramo}]} {\bluebold{tiga}]} {\bluebold{ni}]}\\ %
 next  add  oSb  from  Mambramo  one  \textsc{d.prox}\\
\glt 
[About forming a volleyball team:] ‘and then add \bluebold{these three older brothers from Mambramo}’ \textstyleExampleSource{[Elicited BR111018.004]}
\z

\ea
\label{Example_8.68}
\gll {nanti} {\bluebold{kitong}} {\bluebold{empat}} {su} {tidor} {di} {luar} {\ldots}\\ %
 very.soon  \textsc{1pl}  four  already  sleep  at  outside  \\
\glt 
‘after \bluebold{the four of us} had already been sleeping outside {\ldots}’ \textstyleExampleSource{[081025-009a-Cv.0004]}
\z



With head nominals differentiated in terms of their ranking within a series, \textsc{nn}um\textsc{{}-np}s signal the unique position of a referent within such a ranking as in (\ref{Example_8.69}), or they specify unique points in time as in (\ref{Example_8.70}).



\begin{styleExampleTitle}
\textsc{nn}um\textsc{{}-np}s denoting \isi{definite} quantities of countable referents: Unique positions or points in time
\end{styleExampleTitle}
\ea
\label{Example_8.69}
\gll {kitong} {lari{\Tilde}lari} {sampe} {di} {\bluebold{SP}} {\bluebold{tuju}}\\ %
 \textsc{1pl}  \textsc{rdp}{\Tilde}run  reach  at  transmigration.settlement  seven\\
\glt 
‘we drove all the way to \bluebold{transmigration settlement number seven}’ (Lit. ‘\bluebold{the seventh transmigration settlement}’) \textstyleExampleSource{[081006-033-Cv.0007]}
\z

\ea
\label{Example_8.70}
\gll {\bluebold{jam}} {\bluebold{dua},} {tong} {kluar} {dari} {sini} {\bluebold{jam}} {\bluebold{satu}}\\ %
 hour  two  \textsc{1pl}  go.out  from  \textsc{l.prox}  hour  one\\
\glt 
‘(we arrived at) \bluebold{two o’clock}, we left from here at \bluebold{one o’clock}’ \textstyleExampleSource{[081025-008-Cv.0099]}
\z


In (\ref{Example_8.71}) to (\ref{Example_8.73}), the opposition between the pre- and posthead positions is illustrated with (near) contrastive examples. In (\ref{Example_8.71}) prehead \textitbf{dua} ‘two’ designates the absolute number of items expressed by its head. In (\ref{Example_8.72}) posthead \textitbf{dua} ‘two’ modifies a head nominal undifferentiated in terms of its ranking, whereby it signals the exhaustivity of its referent. In (\ref{Example_8.73}) posthead \textitbf{dua} ‘two’ signals a unique position within a series.



\begin{styleExampleTitle}
Opposition between \textsc{n}um\textsc{n-np}s and \textsc{nn}um\textsc{{}-np}s
\end{styleExampleTitle}
\ea
\label{Example_8.71}
\gll {saya} {jaga} {\bluebold{dua}} {\bluebold{jam},} {yo} {kurang} {lebi} {\bluebold{dua}} {\bluebold{jam}} {\ldots}\\ %
 \textsc{1sg}  guard  two  hour  yes  lack  more  two  hour  \\
\glt 
‘I kept watch for \bluebold{two hours}, yes, more or less for \bluebold{two hours} {\ldots}’ \textstyleExampleSource{[080919-004-NP.0016]}
\z

\ea
\label{Example_8.72}
\gll {\bluebold{sidi}} {\bluebold{dua}} {dia} {potong}\\ %
 CD.player  two  \textsc{3sg}  cut\\
\glt 
‘\bluebold{both CD players}, he destroyed (them)’ \textstyleExampleSource{[081011-009-Cv.0006]}
\z

\ea
\label{Example_8.73}
\gll {ini} {suda} {\bluebold{jam}} {\bluebold{dua}} {malam}\\ %
 \textsc{d.prox}  already  hour  two  night\\
\glt 
‘this is already \bluebold{two o’clock} at night’ \textstyleExampleSource{[080916-001-CvNP.0001]}
\z


The data in (\ref{Example_8.63}) to (\ref{Example_8.73}) suggests that the \textsc{nn}um order is favored in more specific and \isi{definite} constructions, namely to signal exhaustivity of \isi{definite} referents or unique positions within series or sequences. The \textsc{n}um\textsc{n} order, by contrast, is associated with less specific or less \isi{definite} constructions which express the absolute number of items denoted by the head nominal. These patterns contrast with \citegen[284]{Greenberg.1978b} cross-linguistic findings concerning the word order in \isi{noun} phrases with adnominal numerals:


\begin{quote}
44. The order noun-\isi{numeral} is favored in \isi{indefinite} and approximative constructions.
\end{quote}


\citet[284]{Greenberg.1978b} does note, however, that this statement is a generalization rather than a universal, given cross-linguistic variations in quantifier-\isi{noun} [Q-N] order. Noting that “in some languages either QN or NQ may occur with any \isi{numeral}” and that this “contrast of order may then have semantic or syntactic function”, \citet[284]{Greenberg.1978b} presents a number of languages that, like Papuan Malay, employ \textsc{nn}um order in \isi{definite} constructions rather than in \isi{indefinite} ones.



Following \citet[284]{Greenberg.1978b}, the Papuan Malay \textsc{nn}um order in \isi{definite} constructions is a \isi{variation} of a much more common \textsc{n}um\textsc{n} order for these constructions. In his critique of  \citegen[284]{Greenberg.1978b} generalization \#44, \citet{Donohue.2005} demonstrates, however, that the \textsc{nn}um order in \isi{definite} constructions is not a mere “\isi{variation}” found in “some languages”. Rather, “there is a strong tendency for postnominal numerals to be interpreted in highly specific, highly \isi{definite} ways” (\citeyear*[34]{Donohue.2005}). The data presented here suggests that the Papuan Malay word order in \isi{noun} phrases with adnominally used numerals follows this same “strong tendency”.


\subsection{Quantifiers [\textsc{n} \textsc{qt} / \textsc{qt} \textsc{n}]}
\label{Para_8.3.2}
Noun phrases with adnominally used quantifiers have syntactic properties similar to those with adnominally used numerals. Noun phrases with prehead quantifiers (``\textsc{q}t\textsc{n-np}'') express nonnumeric amounts or quantities of the items indicated by their head nominals; they only modify countable referents. Noun phrases with posthead \isi{quantifier} (``\textsc{nq}t\textsc{{}-np}''), by contrast, either denote exhaustivity of \isi{indefinite} referents or signal unknown positions within series or sequences; they modify countable as well as uncountable referents. (For a discussion of quantifiers as a word class see §\ref{Para_5.10}).



The following adnominal quantifiers are attested: universal \textitbf{masing-masing} ‘each’, \textitbf{segala} ‘all’, \textitbf{sembarang} ‘any (kind of)’, \textitbf{(se)tiap} ‘every’, and \textitbf{smua} ‘all’, and mid-range \textitbf{banyak} ‘many’, \textitbf{brapa} ‘several’, \textitbf{sedikit} ‘few’, and \textitbf{stenga} ‘half’.



Five quantifiers can occur in pre- or posthead position, namely \textitbf{banyak} ‘many’, \textitbf{brapa} ‘several’, \textitbf{masing-masing} ‘each’, \textitbf{sedikit} ‘few’, and \textitbf{smua} ‘all’, as shown in (\ref{Example_8.78}) to (\ref{Example_8.92}). The other four quantifiers, that is, \textitbf{segala} ‘all’, \textitbf{sembarang} ‘any (kind of)’, \textitbf{(se)tiap} ‘every’, and \textitbf{stenga} ‘half’, only occur in prehead position where they signal nonnumeric quantities of countable referents, as illustrated in (\ref{Example_8.74}) to (\ref{Example_8.77}). While \textitbf{sembarang} ‘any (kind of)’ is only used with animate referents as in (\ref{Example_8.75}), \textitbf{(se)tiap} ‘every’ and \textitbf{stenga} ‘half’ are only used with inanimate referents as in (\ref{Example_8.76}) and (\ref{Example_8.77}), respectively.\footnote{To express the notion of ``every person'', speakers prefer quantification with \textitbf{masing-masing} ‘each’.} Quantifier \textitbf{segala} ‘all’ is always combined with the \isi{noun} \textitbf{macang} ‘variety’ with \textitbf{segala macang} expressing the notion of ``all kinds, whatever kind'' as in (\ref{Example_8.74}).



\begin{styleExampleTitle}
\textsc{q}t\textsc{n-np}s denoting \isi{indefinite} quantities of countable referents: Individuality
\end{styleExampleTitle}
\ea
\label{Example_8.74}
\gll {\bluebold{segala}} {\bluebold{macang}} {dia} {biking}\\ %
 all  variety  \textsc{3sg}  make\\
\glt 
[About an ancestor’s achievements:] ‘\bluebold{all kinds (of things)}, he made (them)’ \textstyleExampleSource{[080922-010a-CvNF.0297]}
\z

\ea
\label{Example_8.75}
\gll {sa} {tra} {bisa} {kasi} {\bluebold{sembarang}} {\bluebold{orang}}\\ %
 \textsc{1sg}  \textsc{neg}  be.able  give  any(.kind.of)  person\\
\glt 
‘I can’t give (the gasoline to just) \bluebold{any person}’ \textstyleExampleSource{[081110-002-Cv.0080]}
\z

\ea
\label{Example_8.76}
\gll {\bluebold{setiap}} {\bluebold{renungang}} {\bluebold{pagi}} {sa} {su} {kasi} {nasihat} {itu}\\ %
 every  meditation  morning  \textsc{1sg}  already  give  advice  \textsc{d.dist}\\
\glt 
‘(during) \bluebold{each morning devotions}, I already give (them) that (same) advice’ \textstyleExampleSource{[081115-001b-Cv.0008]}
\z

\ea
\label{Example_8.77}
\gll {mungking} {\bluebold{stenga}} {\bluebold{jam}} {saja} {sa} {tidor}\\ %
 maybe  half  hour  just  \textsc{1sg}  sleep\\
\glt 
‘I slept for maybe just \bluebold{half an hour}’ \textstyleExampleSource{[081115-001b-Cv.0056]}
\z



The quantifiers \textitbf{banyak} ‘many’, \textitbf{brapa} ‘several’, \textitbf{masing-masing} ‘each’, \textitbf{sedikit} ‘few’, and \textitbf{smua} ‘all’ can precede or follow their head nominals, as demonstrated in (\ref{Example_8.78}) to (\ref{Example_8.92}). Both phrasal structures serve distinct semantic functions similar to those of adnominal numerals, discussed in §\ref{Para_8.3.1}, although the contrast is more subtle. The examples presented in this section also illustrate that the quantifiers can be used with animate or inanimate referents.



\textsc{q}t\textsc{n-np}s with prehead \textitbf{banyak} ‘many’, \textitbf{brapa} ‘several’, \textitbf{masing-masing} ‘each’, \textitbf{sedi\-kit} ‘few’, and \textitbf{smua} ‘all’ denote the nonnumeric quantities of countable referents. Thereby, \textsc{q}t\textsc{n-np}s express the composite nature of their referents which conveys a sense of individuality, such that ``\textsc{q}t amount of \textsc{n}'' as in (\ref{Example_8.78}) to (\ref{Example_8.82}). The corpus includes only few \isi{noun} phrases with adnominally used \textitbf{sedikit} ‘few’ all of which have \textitbf{sedikit} ‘few’ in posthead position. According to one of the consultants, however, adnominal \isi{modification} with prehead \textitbf{sedikit} ‘few’ is natural and common, as illustrated with the elicited example in (\ref{Example_8.81}).



\begin{styleExampleTitle}
\textsc{q}t\textsc{n-np}s denoting \isi{indefinite} quantities of countable referents: Individuality
\end{styleExampleTitle}
\ea
\label{Example_8.78}
\gll {de} {itu} {kalo} {\bluebold{banyak}} {\bluebold{orang}} {de} {biasa} {begitu}\\ %
 \textsc{3sg}  \textsc{d.dist}  when  many  person  \textsc{3sg}  be.usual  like.that\\
\glt 
‘if there’re \bluebold{many people}, he’s usually like that’ \textstyleExampleSource{[081025-006-Cv.0272]}
\z
\ea
\label{Example_8.79}
\gll tentara  {itu}  ada  brapa  ratus  orang,  ada  sekitar\\
 soldier  {\textsc{d.dist}}  exist  several  hundred  person  exist  vicinity\\
 \gll {\bluebold{brapa}}  {\bluebold{pleton}}\\
 {several}  {platoon}\\
\glt 
‘those soldiers were several hundred people, (they) were approximately \bluebold{several platoons}’ \textstyleExampleSource{[081029-005-Cv.0131]}
\z
\ea
\label{Example_8.80}
\gll  bayar  mas-kawing  ini  laing  \bluebold{masing-masing}  \bluebold{budaya}\\
 pay  bride.price  \textsc{d.prox}  be.different  each  culture\\
\glt 
‘paying this bride price is different (for) \bluebold{each culture}’ \textstyleExampleSource{[081006-029-CvEx.0014]}
\z
\ea
\label{Example_8.81}
\gll {de} {itu} {kalo} {\bluebold{sedikit}} {\bluebold{orang}} {de} {biasa} {begitu}\\ %
 \textsc{3sg}  \textsc{d.dist}  when  few  person  \textsc{3sg}  be.usual  like.that\\
\glt 
‘if there’re \bluebold{few people}, he’s usually like that’ \textstyleExampleSource{[Elicited BR111021.004]}
\z
\ea
\label{Example_8.82}
\gll {\bluebold{smua}} {\bluebold{buku}} {bisa} {basa}\\ %
 all  book  be.able  be.wet\\
\glt 
‘\bluebold{all books} could get wet’ \textstyleExampleSource{[080917-008-NP.0189]}
\z


\textsc{nq}t\textsc{{}-np}s with posthead \textitbf{banyak} ‘many’, \textitbf{brapa} ‘several’, \textitbf{masing-masing} ‘each’, \textitbf{sedikit} ‘few’, and \textitbf{smua} ‘all’ typically signal exhaustivity of \isi{indefinite} countable referents, as shown in (\ref{Example_8.83}) to (\ref{Example_8.88}). Besides, \textsc{nq}t\textsc{{}-np}s with posthead \textitbf{brapa} ‘how many’ can denote unknown positions within series of countable referents, as in (\ref{Example_8.89}). While the head in \textsc{nq}t\textsc{{}-np}s is typically a \isi{noun}, as in (\ref{Example_8.83}), it can also be a personal \isi{pronoun} as in (\ref{Example_8.84}).
Most often,\textsc{ nq}t\textsc{{}-np}s signal a contrastive sense of exhaustivity: \textsc{n} \textitbf{banyak} translates with ``many (and not just a few) \textsc{n}'' as in (\ref{Example_8.83}), \textsc{n} \textitbf{masing-masing} with ``several (and not just a few) \textsc{n}'' as in (\ref{Example_8.84}), \textsc{n} \textitbf{masing-masing} with ``each \textsc{n }(with nobody missing)'' as in the elicited example in (\ref{Example_8.85}), \textsc{n} \textitbf{sedikit} with ``few (and not many) \textsc{n}'' as in (\ref{Example_8.86}), and \textsc{n} \textitbf{smua} with ``the entire collection of \textsc{n} (with nobody/nothing missing)'' as in (\ref{Example_8.88}). As mentioned above, the corpus includes only a few \isi{noun} phrases with adnominally used \textitbf{sedikit} ‘few’ one of which is presented in (\ref{Example_8.86}): \textitbf{ikang sedikit} ‘few fish’. Alternatively, however, \textitbf{ikang sedikit} could receive the predicative reading ‘the fish are few’. Therefore, an additional elicited example is given in (\ref{Example_8.87}).



\begin{styleExampleTitle}
\textsc{nq}t\textsc{{}-np}s denoting \isi{indefinite} quantities of countable referents: Exhaustivity
\end{styleExampleTitle}
\ea
\label{Example_8.83}
\gll {\ldots} {baca} {\bluebold{buku}} {\bluebold{banyak}} {\bluebold{skali}}\\ %
  { }  read  book  many  very\\
\glt 
‘{\ldots} (I’ve) read \bluebold{very many books}’ \textstyleExampleSource{[080917-010-CvEx.0172]}
\z
\ea
\label{Example_8.84}
\gll {sa} {maki} {\bluebold{dorang}} {\bluebold{brapa}} {itu}\\ %
 \textsc{1sg}  abuse.verbally  \textsc{3pl}  several  \textsc{d.dist}\\
\glt 
‘I verbally abused \bluebold{several of them} there’ \textstyleExampleSource{[080923-008-Cv.0012]}
\z
\ea
\label{Example_8.85}
\gll {{dong}} {antar} {{petatas}} {{dengang}} {sayur} {dulu} {taru} {tumpukang}\\ %
 {\textsc{3pl}}  bring  {sweet.potato}  {with}  vegetable  first  put  pile\\
\gll di  {\bluebold{klompok}}  {\bluebold{masing-masing}}  {begitu}\\
 at  {group}  {each}  {like.that}\\
\glt 
‘first they bring the sweet potatoes and vegetables (and) place the piles (of food) in (front of) \bluebold{each group} like that’ \textstyleExampleSource{[Elicited BR111021.001]}\footnote{The elicited example in (\ref{Example_8.85}) is the corrected version of the original recording \textitbf{tumpukang masing klompok masing-masing} ‘pile each[\textsc{tru}] group each’ [081014-017-CvPr.0043]. That is, the speaker started off by saying \textitbf{tumpukang masing-masing} but she corrected herself, resulting in the truncated \isi{quantifier} \textitbf{masing} ‘each[\textsc{tru}]’ and the missing \isi{locative} \isi{preposition} \textitbf{di} ‘at’.}
\z

\ea
\label{Example_8.86}
\gll {kalo} {\bluebold{ikang}} {\bluebold{sedikit},} {itu} {untuk} {tamu}\\ %
 if  fish  few  \textsc{d.dist}  for  guest\\
\glt 
‘as for the \bluebold{few fish}, those are for the guests’ \textstyleExampleSource{[081014-011-CvEx.0008]}
\z

\ea
\label{Example_8.87}
\gll {sa} {ada} {bawa} {\bluebold{kladi}} {\bluebold{sedikit}} {buat} {mama} {dong}\\ %
 \textsc{1sg}  exist  bring  taro.root  few  for  mother  \textsc{3pl}\\
\glt 
‘I’m bringing \bluebold{a few taro roots} for mother and the others’ \textstyleExampleSource{[Elicited }\textstyleExampleSource{BR111021.006}\textstyleExampleSource{]}
\z

\ea
\label{Example_8.88}
\gll {\bluebold{tong}} {\bluebold{smua}} {dari} {kampung}\\ %
 \textsc{1pl}  all  from  village\\
\glt 
‘\bluebold{we all} are from the village’ \textstyleExampleSource{[081010-001-Cv.0084]}
\z


Depending on the semantics of the head nominal, \textsc{nq}t\textsc{{}-np}s with posthead \textitbf{brapa} ‘several’ can also mark unknown positions within series expressed by their referents, as in (\ref{Example_8.89}).



\begin{styleExampleTitle}
\textsc{nq}t\textsc{{}-np}s denoting \isi{indefinite} quantities of countable referents: Exhaustivity or unknown positions within series
\end{styleExampleTitle}
\ea
\label{Example_8.89}
\gll {kalo} {di} {situ} {kang,} {\bluebold{jam}} {\bluebold{brapa}} {saja} {bisa}\\ %
 if  at  \textsc{l.med}  you.know  hour  several  just  be.able\\
\glt 
‘as for (the office) there, you know, (you) can (go there) \bluebold{any time}’ (Lit. ‘\bluebold{several hours}’) \textstyleExampleSource{[081005-001-Cv.0001]}
\z


Noun phrases with uncountable referents are modified with posthead quantifiers only, as shown in (\ref{Example_8.90}) to (\ref{Example_8.92}). This restriction is due to the semantics of mass nouns which, per se, do not convey the sense of individuality encoded by the prehead position of the quantifiers, presented in (\ref{Example_8.74}) to (\ref{Example_8.89}). Adnominal quantifiers for mass nouns are \textitbf{banyak} ‘many’ as in (\ref{Example_8.90}), \textitbf{sedikit} ‘few’ as in (\ref{Example_8.91}), or \textitbf{smua} ‘all’ as in (\ref{Example_8.92}).



\begin{styleExampleTitle}
\textsc{nq}t\textsc{{}-np}s denoting \isi{indefinite} quantities of uncountable referents: Exhaustivity
\end{styleExampleTitle}
\ea
\label{Example_8.90}
\gll {minum} {\bluebold{te}} {\bluebold{banyak},} {minum} {te} {dulu}\\ %
 drink  tea  many  drink  tea  first\\
\glt 
‘drink \bluebold{lots of tea}, drink tea for now!’ \textstyleExampleSource{[081011-001-Cv.0240]}
\z

\ea
\label{Example_8.91}
\gll {tida} {bisa} {\bluebold{air}} {\bluebold{sedikit}} {pung} {sentu} {sa} {pu} {mulut}\\ %
 \textsc{neg}  be.able  water  few  even  touch  \textsc{1sg}  \textsc{poss}  mouth\\
\glt 
[About a sickness:] ‘not even \bluebold{the least bit of water} could touch my mouth’ \textstyleExampleSource{[081006-035-CvEx.0050]}
\z

\ea
\label{Example_8.92}
\gll {\ldots} {buka} {\bluebold{de}} {\bluebold{pu}} {\bluebold{kulit}} {\bluebold{smua}}\\ %
  { } open  \textsc{3sg}  \textsc{poss}  skin  all\\
\glt 
‘(they) peel off \bluebold{his entire skin}’ \textstyleExampleSource{[081029-004-Cv.0047]}
\z



Typically, posthead \textitbf{smua} ‘all’ forms a constituent with the quantified nominal. Alternatively, however, it can float to a clause-final position, as shown in (\ref{Example_8.93}) and (\ref{Example_8.94}).



\begin{styleExampleTitle}
Floating adnominal \isi{quantifier} \textitbf{smua} ‘all’
\end{styleExampleTitle}
\ea
\label{Example_8.93}
\gll {\bluebold{makangang}} {kas} {tinggal} {\bluebold{smua}}\\ %
 food  give  stay  all\\
\glt 
‘(he was made) to leave \bluebold{all (his) food} (untouched)’ \textstyleExampleSource{[081025-008-Cv.0048]}
\z

\ea
\label{Example_8.94}
\gll {\bluebold{dong}} {diam} {\bluebold{smua}}\\ %
 \textsc{3pl}  be.quiet  all\\
\glt
‘\bluebold{they} were \bluebold{all} quiet’ \textstyleExampleSource{[080922-003-Cv.0095]}
\z


\section[{\footnotesize MOD-N} structure: Adnominal possession]{\textsc{mod-n} structure: Adnominal possession}
\label{Para_8.4}
In Papuan Malay, adnominal possessive relations between two \isi{noun} phrases are marked with the possessive ligature \textitbf{punya}; alternative realization of the ligature are reduced \textitbf{pu}, clitic \textitbf{=p}, or a zero morpheme.



Such possessive constructions have a \textsc{mod-n} constituent order which is opposite to the canonical \textsc{n-mod} structure. That is, the head nominal encoding the possessum (\textsc{possm}) takes the \textsc{n2} slot, following the possessive ligature (\textsc{lig}), whereas the modifier expressing the possessor (\textsc{possr}) takes the \textsc{n1} slot, such that ``\textsc{possr}{}-\textsc{np} – \textsc{lig} – \textsc{possm}{}-\textsc{np}''. This is shown with the \isi{adnominal possessive construction} in (\ref{Example_8.95}).


\ea
\label{Example_8.95}
\hspace{45pt} \textsc{possr} \hspace{10pt} \textsc{lig} \hspace{10pt}  { \textsc{possm}} \\ %
\gll  nanti  \bluebold{Hendro}  \bluebold{punya}  \bluebold{ade}  \bluebold{prempuang}  kawing  {\ldots}\\
 very.soon  Hendro  \textsc{poss}  ySb  woman  marry.inofficially  \\
\glt 
‘eventually \bluebold{Hendro’s younger sister} would marry {\ldots}’ \textstyleExampleSource{[081006-028-CvEx.0007]}
\z


Syntactically, a variety of constituents can encode the possessor and the possessum, as shown in (\ref{Example_8.96}). The possessor slot can be taken by a lexical \isi{noun} as in (\ref{Example_8.96a}, \ref{Example_8.96b},) a personal \isi{pronoun} as in (\ref{Example_8.96c}, \ref{Example_8.96d}), a \isi{demonstrative} as in (\ref{Example_8.96e}), the \isi{interrogative} \textitbf{siapa} ‘who’ as in (\ref{Example_8.96f}), or a \isi{noun} phrase as in (\ref{Example_8.96g}). The possessum can be encoded by a lexical \isi{noun} as in (\ref{Example_8.96c}, \ref{Example_8.96e}, \ref{Example_8.96f}), a \isi{demonstrative} as in (\ref{Example_8.96a}, \ref{Example_8.96b}), the \isi{interrogative} \textitbf{siapa} ‘who’ as in (\ref{Example_8.96d}), or a \isi{noun} phrase as in (\ref{Example_8.96g}). Possessive \isi{noun} phrases with a personal \isi{pronoun} possessum are unattested.



\begin{styleExampleTitle}
Syntactic constituents of adnominal possessive constructions\footnote{Documentation: 080919-006-CvNP.0028, 080921-009-Cv.0020, 080922-001a-CvPh.1123, 080925-004-Cv.0006, 081006-019-Cv.0002, 081025-006-Cv.0058, 081106-001-Ex.0007.}
\end{styleExampleTitle}
\ea
\label{Example_8.96}
\hspace{15pt} \textsc{possr}  \textsc{lig}  \textsc{possm}\hspace{15pt}  Adnominal possessive construction\\
 \ea
\label{Example_8.96a}
\gll {\fauxsc{n}}  {\textitbf{pu}}  {\fauxsc{dem}}  {\hspace{15.1mm}}  {\textitbf{ade pu itu}}\\ %
 {} {} {} {}  {‘younger sister’s (fish)’}\\
\ex
\label{Example_8.96b}
\gll \fauxsc{n}   \textitbf{pu}   \fauxsc{dem} {\hspace{15.1mm}} {\textitbf{Fitri pu ini}}\\
 {} {} {} {}  {‘Fitri’s (belongings)’}\\
\ex
\label{Example_8.96c}
\gll \fauxsc{pro}   \textitbf{punya}  \fauxsc{n}  {\hspace{10mm}} {\textitbf{de punya bulu{\Tilde}bulu}}\\
  {} {} {} {} {‘its (the dog’s) body hair’}\\
\ex
\label{Example_8.96d}
\gll \fauxsc{pro}   \textitbf{pu}  \fauxsc{int} {\hspace{13.3mm}}  {\textitbf{sa pu siapa}}\\
{} {} {} {}    {‘who of my (relatives)’ }\\
\ex
\label{Example_8.96e}
\gll \fauxsc{dem}   \textitbf{pu}   \fauxsc{n} {\hspace{15mm}}  {\textitbf{ini pu muka}}\\
 {} {} {} {}  {‘this (one’s) face’ }\\
\ex
\label{Example_8.96f}
\gll  \fauxsc{int}   \textitbf{pu}  \fauxsc{n} {\hspace{16.5mm}}  {\textitbf{siapa pu sandal}}\\
{} {} {} {}   {‘whose sandals’}\\
\ex
\label{Example_8.96g}
\gll \fauxsc{np}   \textitbf{pu}   \fauxsc{np} {\hspace{16mm}}  {\textitbf{mama Klara pu ana prempuang}} \\
 {} {} {} {} {‘mother Klara’s daughter’}\\
\z
\z


The examples in (\ref{Example_8.95}) and (\ref{Example_8.96}) show that adnominal possessive constructions designate possession of a \isi{definite} possessum. Adnominal possession, including the noncanonical functions of the possessive marker, is discussed in detail in \chapref{Para_9} (see also §\ref{Para_11.4.1} for the uses of adnominal possessive constructions in two-argument existential clauses). Possession of an \isi{indefinite} possessum is expressed with a \isi{two-argument existential clause} or with a nominal clause; details are presented in §\ref{Para_11.4.2} and §\ref{Para_12.2}, respectively.


\section{Apposition}
\label{Para_8.5}
In an \isi{apposition} two “or more \isi{noun} phrases” have “the same referent” and stand “in the same syntactical relation to the rest of the sentence” \citep[5093]{Asher.1994}. Papuan Malay employs two types of appositional constructions, namely \isi{apposition} of a \isi{noun} with another \isi{noun} or \isi{noun} phrase, such that ``\textsc{n} \textsc{np}'', and \isi{apposition} of a personal \isi{pronoun} with a \isi{noun} or \isi{noun} phrase, such that ``\textsc{pro} \textsc{np}''. This section describes ``\textsc{n} \textsc{np}'' appositions, while ``\textsc{pro} \textsc{np}'' appositions are discussed in §\ref{Para_6.1.6}.



Papuan Malay ``\textsc{n} \textsc{np}'' appositions are restrictive. That is, the apposited or juxtaposed \isi{noun} phrase is needed for the appropriate identification of the referent encoded by the initial \isi{noun}. There are no formal distinctions, though, between the ``\textsc{n} \textsc{np}'' appositions discussed here and \isi{noun} phrases with adnominally used nouns (\textsc{n1n2-np}), discussed in §\ref{Para_8.2.2}; the distinction is based on semantics.



In the corpus, ``\textsc{n} \textsc{np}'' appositions are rare, and in each case the initial \isi{noun} encodes a \isi{kinship term}, as in (\ref{Example_8.97}) and (\ref{Example_8.98}). The juxtaposed \isi{noun} phrase \textitbf{ibu pendeta} ‘Ms. Pastor’ in (\ref{Example_8.97}) is appositional to the first \isi{noun} \textitbf{kaka} ‘older sibling’. It provides information necessary for the identification of the referent. In (\ref{Example_8.98}), the appositional \isi{noun} phrase \textitbf{ketua klasis} ‘church district chairperson’ serves as an identifying explanation for the reference of the initial \isi{noun} \textitbf{bapa} ‘father’.


\ea
\label{Example_8.97}
\gll {bapa-ade} {{ini,}} {{kaka,}} {[\bluebold{kaka}]} {{[\bluebold{ibu}}} {\bluebold{pendeta}]} {dengang} {ini}\\ %
 uncle  {\textsc{d.prox}}  {oSb}  oSb  {woman}  pastor  with  \textsc{d.prox}\\
\gll {mama-tua,}  {nene}  {ini}  dong  {tertawa}\\
 {aunt}  {grandmother}  {\textsc{d.prox}}  \textsc{3pl}  {laugh}\\
\glt 
‘uncle here (and) older sibling, \bluebold{older sibling, Ms. Pastor}, and, what’s-her-name, aunt, grandmother here, they were laughing’ \textstyleExampleSource{[080922-001a-CvPh.0824]}
\z

\ea
\label{Example_8.98}
\gll {\ldots} {bapa} {di} {dalam,} {[\bluebold{bapa}]} {[\bluebold{ketua}} {\bluebold{klasis}]}\\ %
  { }  father  at  inside  father  chairperson  church.district\\
\glt
‘[that’s what I’ve never told older sibling, what’s-his-name,] father (who’s) inside, \bluebold{father, the church district chairperson}’ \textstyleExampleSource{[080922-010a-CvNF.0104]}
\z


\section{Summary}
\label{Para_8.6}
The head of a \isi{noun} phrase is typically a \isi{noun} or personal \isi{pronoun}. Further, although less common, demonstratives, locatives, or interrogatives can also function as heads. The canonical word order within the \isi{noun} phrase is \textsc{head-modifier}. Depending on the syntactic properties of the adnominal constituents, though, a \textsc{modifierhead} order is also common. Attested in the corpus is the co-occurr\-ence of up to three posthead modi\-fiers. The possible constituents of the maximally extended \isi{noun} phrase and the order of these constituents is summarized in the template in \tabref{Table_8.3}  (the items in parenthesis are optional).


\begin{table}
\caption{Template of the maximally extended \isi{noun} phrase}\label{Table_8.3}

\begin{tabular}{llllll}
\lsptoprule
(\textsc{num}) & \multirow{4}{*}{\textsc{head}} & (\textsc{v}) & \multirow{4}{*}{(\textsc{pro})} & (\textsc{dem}) & \multirow{4}{*}{(\textsc{dem})}\\
(\textsc{qt}) &  & (\textsc{n}) &  & (\textsc{loc}) & \\
(\textsc{possr-np}) &  & (\textsc{pp}) &  & (\textsc{int}) & \\
&  & (\textsc{rc}) &  & (\textsc{num}) & \\
&  &  &  & (\textsc{qt}) & \\
\lspbottomrule
\end{tabular}
\end{table}

The template in \tabref{Table_8.3} shows that \isi{noun} phrases with adnominally used verbs, nouns, personal pronouns, demonstratives, locatives, 
prepositional phrases, 
interrogatives, 
and relative clauses have an \textsc{n-mod} structure. Adnominal possessive constructions, by contrast, have a \textsc{mod-n} structure with the modifying possessor phrase occurring in prehead position. Noun phrases with adnominally used numerals and quantifiers have an \textsc{n-mod} or \textsc{mod-n} structure depending on the semantics of the phrasal structure. Adnominally used demonstratives can occur in two slots. They can take the same slot as adnominally used locatives, interrogatives, numerals, or quantifiers, and in addition they can occur at the right periphery of the \isi{noun} phrase where they have scope over the entire \isi{noun} phrase.



Papuan Malay uses two types of appositional constructions: those consisting of a \isi{noun} followed by another \isi{noun} or \isi{noun} phrase, and those consisting of a personal \isi{pronoun} followed by a \isi{noun} or \isi{noun} phrase, the latter being discussed in §\ref{Para_6.1.6}. Appositions with juxtaposed nouns or \isi{noun} phrases are restrictive.

