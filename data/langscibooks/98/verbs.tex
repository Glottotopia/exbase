\section{Verbs}\label{sec:8}



\subsection{Overview}\label{sec:8.1}


Ik verbs consist of a verbal root (written in this book with a hyphen, as in \textit{wèt-} ‘drink’) and at least one of a variety of available derivational and inflectional suffixes. The language has no prefixes except those borrowed centuries ago that no longer have any active function, for example the /a/ in \textit{áb\`{ʉ}b\`{ʉ}ƙ-} ‘bubble’ or the /i/ in \textit{iɓóɓór-} ‘hollow out’. Reduplicating a verb root, partially or totally, has long been a strategy for creating a sense of continuousness or repetitiveness, as when \textit{ɨtsán-} ‘disturb’ becomes \textit{ɨtsan{\Í}tsán-} ‘torment relentlessly’. 

Ik employs a large number of suffixes to create longer verb stems. Among these are the \textsc{infinitive} and other deverbalizing suffixes that change a verb into a noun that can take case endings, demonstratives, relative clauses, etc. One very key verb-building strategy of Ik is the \textsc{directional} suffixes that signify the direction of the verb’s movement to or away from the speaker. These two directionals have also been extended metaphorically to express the beginning or completion of actions or processes. Another set of verbal suffixes deal with \textsc{voice} and \textsc{valency}, that is, the number of objects the verb requires. Among these are the \textsc{passive}, \textsc{impersonal} \isi{passive}, \textsc{middle}, \textsc{causative}, and \textsc{reciprocal}.

Once a verb is taken from the mental lexicon and used in speech, it often requires \textsc{subject-agreement} marking, which Ik accomplishes through pronominal suffixes. Ik also has a special verbal suffix, the \textsc{dummy} \textsc{pronoun}, that goes on the verb whenever a peripheral argument, like a place or time designation, has been moved to the front of the clause or removed entirely.

The Ik verbal system has a variety of verbal paradigms based on \textsc{mood} and \textsc{aspect}. The basic distinction in mood is between \textsc{realis} and \textsc{irrealis}, or things that have happened and things that have not. Other modal distinctions include the \textsc{optative}, \textsc{subjunctive}, \textsc{imperative}, and \textsc{negative}. As for \isi{aspect}, the specification of the internal structure of a verb – complete or incomplete – Ik has suffixes that mark \textsc{present perfect}, \textsc{intentional}{}-\textsc{imperfective}, \textsc{pluractional}, \textsc{sequential}, and \textsc{simultaneous}. Lastly, Ik exhibits a special set of \textsc{adjectival} suffixes to cover the language’s need to express \isi{adjectival} concepts.




\subsection{Infinitives (\textsc{inf})}\label{sec:8.2}
\subsubsection{Intransitive}\label{sec:8.2.1}

\textsc{intransitive} verbs allow only a subject and possibly an indirect object – a direct object does not figure into their semantic schema. The Ik \isi{intransitive} \textsc{infinitive} suffix is \{-ònì-\}. It converts an \isi{intransitive} verb to a morphological noun that can be used as a noun in a \isi{noun phrase}. The \isi{infinitive} is the \textsc{citation} \textsc{form} of a verb, the form one cites in a dictionary or in isolation from other words. \tabref{tab:verbs:intrans} gives a few examples of \isi{intransitive} infinitives from the lexicon:


\begin{table}
\caption{Ik \isi{intransitive} infinitives}
\label{tab:verbs:intrans}


\begin{tabularx}{.66\textwidth}{XXX}
\lsptoprule

Root & \multicolumn{2}{l}{Intransitive infinitive}\\
\midrule
áƙáf- & áƙáfòn & ‘to yawn’\\
bòt- & bòtòn & ‘to migrate’\\
c{\Ì}- & c{\Ì}\`{ɔ}n & ‘to be satiated’\\
dód- & dódòn & ‘to hurt’\\
\`{ɛ}f- & \`{ɛ}fɔn & ‘to be tasty’\\
gw{\Ì}r- & gw{\Ì}rɔn & ‘to squirm’\\
iƙú- & iƙúón & ‘to howl’\\
\lspbottomrule
\end{tabularx}
\end{table}

\newpage 
Because the \isi{infinitive} is a noun morphologically, it can be fully declined for case as all nouns can. \tabref{tab:verbs:intrans:case1} gives the case declension of the verb \textit{wàtònì-} ‘to rain’, which shows some \isi{vowel assimilation} effects on [+\isi{ATR}] vowels, as when /io/ becomes /uo/ in the ablative and copulative cases. \tabref{tab:verbs:intrans:case2} does the same for the [-\isi{ATR}] verb \textit{w\'{ɛ}d\`{ɔ}n{\Ì}-} ‘to detour’. Note /ɨɔ/ becoming /ʉɔ/ there as well:


\begin{table}
\caption{Case declension of \textit{wàtònì-} ‘to rain’}
\label{tab:verbs:intrans:case1}


\begin{tabularx}{.66\textwidth}{XXX}
\lsptoprule

& Non-final & Final\\
\midrule
\textsc{nom} & wàtònà & wàtòn\\
\textsc{acc} & wàtònìà & wàtònìkᵃ\\
\textsc{dat} & wàtònìè & wàtònìkᵉ\\
\textsc{gen} & wàtònìè & wàtònì\\
\textsc{abl} & wàtònùò & wàtònù\\
\textsc{ins} & wàtònò & wàtònᵒ\\
\textsc{cop} & wàtònùò & wàtònùkᵒ\\
\textsc{obl} & wàtònì & wàtòn\\
\lspbottomrule
\end{tabularx}
\end{table}

\begin{table}
\caption{Case declension of \textit{w\'{ɛ}d\`{ɔ}n{\Ì}-} ‘to detour’}
\label{tab:verbs:intrans:case2}


\begin{tabularx}{.66\textwidth}{XXX}
\lsptoprule

& Non-final & Final\\
\midrule
\textsc{nom} & w\'{ɛ}d\`{ɔ}nà & w\'{ɛ}d\`{ɔ}n\\
\textsc{acc} & w\'{ɛ}d\`{ɔ}n{\Ì}à & w\'{ɛ}d\`{ɔ}n{\Ì}kᵃ\\
\textsc{dat} & w\'{ɛ}d\`{ɔ}n{\Ì}\`{ɛ} & w\'{ɛ}d\`{ɔ}n{\Ì}k\ᵋ\\
\textsc{gen} & w\'{ɛ}d\`{ɔ}n{\Ì}\`{ɛ} & w\'{ɛ}d\`{ɔ}n{\Ì}\\
\textsc{abl} & w\'{ɛ}d\`{ɔ}n\`{ʉ}\`{ɔ} & w\'{ɛ}d\`{ɔ}n\`{ʉ}\\
\textsc{ins} & w\'{ɛ}d\`{ɔ}n\`{ɔ} & w\'{ɛ}d\`{ɔ}nᵓ\\
\textsc{cop} & w\'{ɛ}d\`{ɔ}n\`{ʉ}\`{ɔ} & w\'{ɛ}d\`{ɔ}n\`{ʉ}kᵓ\\
\textsc{obl} & w\'{ɛ}d\`{ɔ}n{\Ì} & w\'{ɛ}d\`{ɔ}n\\
\lspbottomrule
\end{tabularx}
\end{table}

\subsubsection{Transitive}\label{sec:8.2.2}

\textsc{transitive} verbs are those that admit a subject \textit{and} a direct object into their schematic of an active event. The Ik transitive \isi{infinitive} suffix is \{-ésí-\}. It converts a transitive verb to a morphological noun that can be used as a noun in a \isi{noun phrase}. \tabref{tab:verbs:trans1} presents a few examples of transitive infinitives:


\begin{table}
\caption{Ik transitive infinitives}
\label{tab:verbs:trans1}


\begin{tabularx}{.66\textwidth}{XXX}
\lsptoprule

Root & \multicolumn{2}{l}{Transitive infinitive}\\
\midrule
ág\`{ʉ}ʝ- & ágʉʝ\'{ɛ}s & ‘to gulp’\\
ban- & ban\'{ɛ}s & ‘to sharpen’\\
c\'{ɛ}b- & c\'{ɛ}b\`{ɛ}s & ‘to roughen’\\
ɗóɗ- & ɗóɗés & ‘to point at’\\
erég- & erégès & ‘to employ’\\
g{\Í}ʝ- & g{\Í}ʝ\'{ɛ}s & ‘to shave’\\
ɨl\'{ɔ}ƙ- & ɨlɔƙɛs & ‘to dissolve’\\
\lspbottomrule
\end{tabularx}
\end{table}
%%please move \begin{table} just above \begin{tabular

\newpage 
\tabref{tab:verbs:trans2} gives the case declension of the deverbalized noun \textit{wetésí-} ‘to drink’, which shows \isi{vowel assimilation} effects on [+\isi{ATR}] vowels. \tabref{tab:verbs:trans3} does the same for the [-\isi{ATR}] verb for the [-\isi{ATR}] verb \textit{wɛtsʼ\'{ɛ}s{\Í}-} ‘to knap’.
 



\begin{table}
\caption{Case declension of \textit{wetésí-} ‘to drink’}
\label{tab:verbs:trans2}


\begin{tabularx}{.66\textwidth}{XXX}
\lsptoprule

& Non-final & Final\\
\midrule
\textsc{nom} & wetésá & wetés\\
\textsc{acc} & wetésíà & wetésíkᵃ\\
\textsc{dat} & wetésíè & wetésíkᵉ\\
\textsc{gen} & wetésíè & wetésí\\
\textsc{abl} & wetésúò & wetésú\\
\textsc{ins} & wetésó & wetésᵒ\\
\textsc{cop} & wetésúò & wetésúkᵒ\\
\textsc{obl} & wetésí & wetés\\
\lspbottomrule
\end{tabularx}
\end{table}

\begin{table}
\caption{Case declension of \textit{wɛtsʼ\'{ɛ}s{\Í}-} ‘to knap’}
\label{tab:verbs:trans3}


\begin{tabularx}{.66\textwidth}{XXX}
\lsptoprule

& Non-final & Final\\
\midrule
\textsc{nom} & wɛtsʼ\'{ɛ}sá & wɛtsʼ\'{ɛ}s\\
\textsc{acc} & wɛtsʼ\'{ɛ}s{\Í}à & wɛtsʼ\'{ɛ}s{\Í}kᵃ\\
\textsc{dat} & wɛtsʼ\'{ɛ}s{\Í}\`{ɛ} & wɛtsʼ\'{ɛ}s{\Í}k\ᵋ\\
\textsc{gen} & wɛtsʼ\'{ɛ}s{\Í}\`{ɛ} & wɛtsʼ\'{ɛ}s{\Í}\\
\textsc{abl} & wɛtsʼ\'{ɛ}s\'{ʉ}\`{ɔ} & wɛtsʼ\'{ɛ}s\'{ʉ}\\
\textsc{ins} & wɛtsʼ\'{ɛ}s\'{ɔ} & wɛtsʼ\'{ɛ}sᵓ\\
\textsc{cop} & wɛtsʼ\'{ɛ}s\'{ʉ}\`{ɔ} & wɛtsʼ\'{ɛ}s\'{ʉ}kᵓ\\
\textsc{obl} & wɛtsʼ\'{ɛ}s{\Í} & wɛtsʼ\'{ɛ}s\\
\lspbottomrule
\end{tabularx}
\end{table}

\subsubsection{Semi-transitive}\label{sec:8.2.3}

\textsc{semi-transitive} verbs fall between transitive and \isi{intransitive} in that they take an object, but the object is the \isi{reflexive} pronoun \textit{as{\Í}-} ‘-self’, referring to the subject. This means that semi-transitive verbs are morphologically transitive but almost \isi{intransitive} semantically. Another name for this is ‘middle’ (although see another Ik middle verb in \sectref{sec:8.6.3}). \tabref{tab:verbs:semitrans} provides a sample of semi-transitive verbs. No case declension is given for these because they decline the same way as the transitive infinitives shown in \tabref{tab:verbs:trans2} and \tabref{tab:verbs:trans3}:


\begin{table}
\caption{Ik semi-transitive infinitives}
\label{tab:verbs:semitrans}


\begin{tabularx}{\textwidth}{XXXlX}
\lsptoprule

Root & \multicolumn{2}{X}{} & \multicolumn{2}{X}{Semi-transitive}\\
\midrule
bal- & \multicolumn{2}{X}{‘ignore’} & bal\'{ɛ}sá as{\Í} & ‘to neglect -self’\\
\multicolumn{2}{X}{hoɗ-} & ‘free’ & hoɗésá as{\Í} & ‘to get freed’\\
\multicolumn{2}{X}{ɨr{\Í}ts-} & ‘keep’ & ɨrɨtsɛsa as{\Í} & ‘to control -self’\\
\multicolumn{2}{X}{ɨr\'{ʉ}ts-} & ‘fling’ & ɨrʉtsɛsa as{\Í} & ‘to race across’\\
\multicolumn{2}{X}{ɨt{\Í}ŋ-} & ‘force’ & ɨtɨŋɛsa as{\Í} & ‘to force -self’\\
\multicolumn{2}{X}{kɔk-} & ‘close’ & kɔk\'{ɛ}sá as{\Í} & ‘to cover -self’\\
\multicolumn{2}{X}{toɓ-} & ‘spear’ & toɓésá as{\Í} & ‘to shoot across’\\
\lspbottomrule
\end{tabularx}
\end{table}



\subsection{Deverbalizers}\label{sec:8.3}
\subsubsection{Abstractive (\textsc{abst})}\label{sec:8.3.1}

The \textsc{abstractive} suffix \{-ás{\Í}-\} can be used to replace the \isi{intransitive} suffix \{-ònì-\} for converting an \isi{intransitive} verb to an abstract noun, for example, when \textit{hábòn} ‘to be hot’ becomes \textit{hábàs} ‘heat’. \tabref{tab:verbs:abstr1} gives several examples of abstract nouns derived from \isi{intransitive} verbs:


\begin{table}
\caption{Ik abstract nouns derived from verbs}
\label{tab:verbs:abstr1}


\begin{tabularx}{\textwidth}{XXXXX}
\lsptoprule

\multicolumn{3}{l}{Intransitive infinitive} & \multicolumn{2}{X}{Abstract noun}\\
\midrule
ɓàŋ\`{ɔ}n & \multicolumn{2}{X}{ ‘to be loose’} & ɓaŋás & ‘looseness’\\
\multicolumn{2}{X}{\`{ɛ}f\`{ɔ}n} & ‘to be tasty’ & ɛfás & ‘(tasty) fat’\\
\multicolumn{2}{X}{gaanón} & ‘to be bad’ & gaánàs & ‘badness’\\
\multicolumn{2}{X}{ɦy\`{ɛ}t\`{ɔ}n} & ‘to be fierce’ & ɦyɛtás & ‘fierceness’\\
\multicolumn{2}{X}{kòmòn} & ‘to be many’ & komás & ‘manyness’\\
\multicolumn{2}{X}{ŋwàx\`{ɔ}n} & ‘to be disabled’ & ŋwaxás & ‘disability’\\
\multicolumn{2}{X}{x\`{ɛ}ɓ\`{ɔ}n} & ‘to be shy’ & xɛɓás & ‘shyness’\\
\lspbottomrule
\end{tabularx}
\end{table}

Because verbs deverbalized by the abstractive suffix are morphological nouns, they are fully declined for case. \tabref{tab:verbs:abstr2} gives one such case declension of the abstract noun \textit{kuɗás{\Í}-} ‘shortness’:


\begin{table}
\caption{Case declension of \textit{kuɗás{\Í}-} ‘shortness’}
\label{tab:verbs:abstr2}


\begin{tabularx}{.66\textwidth}{XXX}
\lsptoprule

& Non-final & Final\\
\midrule
\textsc{nom} & kuɗásá & kuɗás\\
\textsc{acc} & kuɗás{\Í}à & kuɗás{\Í}kᵃ\\
\textsc{dat} & kuɗás{\Í}\`{ɛ} & kuɗás{\Í}k\ᵋ\\
\textsc{gen} & kuɗás{\Í}\`{ɛ} & kuɗás{\Í}\\
\textsc{abl} & kuɗás\'{ʉ}\`{ɔ} & kuɗás\'{ʉ}\\
\textsc{ins} & kuɗás\'{ɔ} & kuɗásᵓ\\
\textsc{cop} & kuɗás\'{ʉ}\`{ɔ} & kuɗás\'{ʉ}kᵓ\\
\textsc{obl} & kuɗás{\Í} & kuɗás\\
\lspbottomrule
\end{tabularx}
\end{table}

\subsubsection{Behaviorative (\textsc{bhvr})}\label{sec:8.3.2}

The \textsc{behaviorative} suffix \{-nànèsì-\} is a complex suffix possibly consisting of the stative suffix \{-án-\} from \sectref{sec:8.11.4} and the transitive suffix (\sectref{sec:8.2.2}) or the abstractive suffix (\sectref{sec:8.3.1}). Regardless of its composition, the suffix as a whole creates abstract concepts based on simple nouns, like \textit{ámánànès} ‘personhood’ or ‘personality’ from \textit{ámá-} ‘person’. \tabref{tab:verbs:behave1} provides a few more examples:



\begin{table}
\caption{Ik behaviorative abstract nouns}
\label{tab:verbs:behave1}


\begin{tabularx}{\textwidth}{XXlX}
\lsptoprule

\multicolumn{2}{X}{Noun root} & \multicolumn{2}{X}{Behaviorative}\\
\midrule
babatí- & ‘his/her father’ & babatínánès & ‘fatherhood’\\
cekí- & ‘woman’ & cekínánès & ‘womanhood’\\
dzɔɗát{\Í}- & ‘rectum’ & dzɔɗát{\Í}nànès & ‘grabbiness’\\
dzúú- & ‘theft’ & dzúnánès & ‘thievery’\\
imá- & ‘child’ & imánánès & ‘childhood’\\
lɔŋ\'{ɔ}tá- & ‘enemy’ & lɔŋ\'{ɔ}tánànès & ‘enmity’\\
ŋókí- & ‘dog’ & ŋókínànès & ‘poverty’\\
\lspbottomrule
\end{tabularx}
\end{table}
Because behavioratives are nouns morphologically, they are declined for case. \tabref{tab:verbs:behave2} gives the case declension for the word \textit{eakwánánèsì}{}- ‘manhood’:


\begin{table}
\caption{Case declension of \textit{eakwánánèsì-} ‘manhood’}
\label{tab:verbs:behave2}


\begin{tabularx}{.66\textwidth}{XXX}
\lsptoprule

& Non-final & Final\\
\midrule
\textsc{nom} & eakwánánèsà & eakwánánès\\
\textsc{acc} & eakwánánèsìà & eakwánánèsìkᵃ\\
\textsc{dat} & eakwánánèsìè & eakwánánèsìkᵉ\\
\textsc{gen} & eakwánánèsìè & eakwánánèsì\\
\textsc{abl} & eakwánánèsùò & eakwánánèsù\\
\textsc{ins} & eakwánánèsò & eakwánánèsᵒ\\
\textsc{cop} & eakwánánèsùò & eakwánánèsùkᵒ\\
\textsc{obl} & eakwánánèsì & eakwánánès\\
\lspbottomrule
\end{tabularx}
\end{table}

\subsubsection{Patientive (\textsc{pat})}\label{sec:8.3.3}

The \textsc{patientive} suffix \{-amá-\} converts a verb to a noun that is characterized by the meaning of the verb. It is called ‘patientive’ because the derived noun usually has the meaning of ‘\isi{patient}’ or object of the original verb, as when \textit{meetés} ‘to give’ produces \textit{meetam} ‘gift’. \tabref{tab:verbs:pat1} gives some examples of patientive nouns:


\begin{table}
\caption{Ik patientive nouns}
\label{tab:verbs:pat1}


\begin{tabularx}{\textwidth}{XXXX}
\lsptoprule

\multicolumn{2}{X}{Verb root} & \multicolumn{2}{X}{Patientive noun}\\
\midrule
áts- & ‘chew’ & atsʼamá- & ‘chewy food’\\
ɓɛk- & ‘provoke’ & ɓɛkamá- & ‘provocation’\\
dʉb- & ‘knead’ & dʉbamá- & ‘dough’\\
dzígw- & ‘buy/sell’ & dzígwamá- & ‘merchandise’\\
gam- & ‘kindle’ & gamamá- & ‘kindling’\\
isúɗ- & ‘distort’ & isuɗamá- & ‘falsehood’\\
ŋƙ- & ‘eat’ & ŋƙamá- & ‘eatable’\\
\lspbottomrule
\end{tabularx}
\end{table}
Because patientives are nouns morphologically, they are fully declined for case. \tabref{tab:verbs:pat2} gives the full declension of the noun \textit{wetamá-} ‘drink(able)’:


\begin{table}
\caption{Case declension of \textit{wetamá-} ‘drink(able)’}
\label{tab:verbs:pat2}


\begin{tabularx}{.66\textwidth}{XXX}
\lsptoprule

& Non-final & Final\\
\midrule
\textsc{nom} & wetama & wetam\\
\textsc{acc} & wetamáá & wetamákᵃ\\
\textsc{dat} & wetaméé & wetamákᵉ\\
\textsc{gen} & wetaméé & wetamáᵉ\\
\textsc{abl} & wetamóó & wetamáᵒ\\
\textsc{ins} & wetamo & wetamᵒ\\
\textsc{cop} & wetamóó & wetamákᵒ\\
\textsc{obl} & wetama & wetam\\
\lspbottomrule
\end{tabularx}
\end{table}



\subsection{Directionals}\label{sec:8.4}
\subsubsection{Venitive (\textsc{ven})}\label{sec:8.4.1}

The \textsc{venitive} suffix \{-ét-\} denotes a direction \textit{toward} a deictic center, usually (but not always) the speaker. It can be translated variously as ‘here’, ‘this way’, ‘out’, or ‘up’, but it is the Middle English word ‘hither’ that captures its essence nicely. The venitive suffix comes between the verb root and the \isi{infinitive} suffix, whether \isi{intransitive} or transitive. It can be used to augment any verb whose meaning includes motion or movement of any kind. \tabref{tab:verbs:ven} gives a few examples:


\begin{table}
\caption{Ik venitive verbs}
\label{tab:verbs:ven}


\begin{tabularx}{\textwidth}{XlXl}
\lsptoprule

Intransitive &  & \multicolumn{2}{X}{Transitive}\\
\midrule
arétón & ‘to cross this way’ & béberetés & ‘to pull this way’\\
ɦyɔt\'{ɔ}g\`{ɛ}t\`{ɔ}n & ‘to approach here’ & ɗʉrɛt\'{ɛ}s & ‘to pull out’\\
ɨl\'{ɛ}\'{ɛ}tɔn & ‘to come visit’ & futetés & ‘to blow this way’\\
irímétòn & ‘to rotate this way’ & hɔnɛt\'{ɛ}s & ‘to drive out’\\
ŋkéétòn & ‘to get up’ & ɨrɨŋɛt\'{ɛ}s & ‘to turn this way’\\
t\'{ɛ}\'{ɛ}t\`{ɔ}n & ‘to fall down’ & iteletés & ‘to watch here’\\
tʉw\'{ɛ}t\'{ɔ}n & ‘to sprout up’ & seɓetés & ‘to sweep up’\\
\lspbottomrule
\end{tabularx}
\end{table}
Venitive infinitives are morphological nouns and thus are declined for case. See \sectref{sec:8.2.1} and \sectref{sec:8.2.2} for case declensions that show the relevant endings.


\subsubsection{An\isi{dative} (\textsc{and})}\label{sec:8.4.2}

The \textsc{andative} suffix \{-uƙot(í)-\} denotes direction \textit{away from} a deictic center, usually the speaker (but not always). It can be translated variously as ‘away’, ‘off’, ‘out’, ‘that way’, or ‘there’, but it is the Middle English word ‘thither’ that captures its essence nicely. Unlike the venitive suffix, the \isi{andative} comes after both the verbal root and the \isi{infinitive} suffix (in an infinitival construction). It can be used to augment any verb whose meaning includes motion or movement of any kind. \tabref{tab:verbs:and1} provides a few examples of \isi{andative} verbs.


\begin{table}
\caption{Ik \isi{andative} verbs}
\label{tab:verbs:and1}


\begin{tabularx}{\textwidth}{XlXl}
\lsptoprule

Intransitive &  & \multicolumn{2}{X}{Transitive}\\
\midrule
aronuƙotᵃ & ‘to cross that way’ & hɔn\'{ɛ}s\'{ʉ}ƙɔtᵃ & ‘to drive off/away’\\
botonuƙotᵃ & ‘to move away’ & ɨɗ\'{ɛ}\'{ɛ}sʉƙɔtᵃ & ‘to hide way’\\
bʉrɔnʉƙɔtᵃ & ‘to fly off/away’ & ídzesuƙotᵃ & ‘to shoot (away)’\\
ɨɓák\'{ɔ}nʉƙɔtᵃ & ‘to go next to’ & ígorésúƙotᵃ & ‘to cross over’\\
isépónuƙotᵃ & ‘to flow away’ & ƙanésúƙotᵃ & ‘to take away’\\
kúbonuƙotᵃ & ‘to go out of sight’ & maƙésúƙotᵃ & ‘to give away’\\
tʉl\'{ʉ}ŋ\'{ɔ}nʉƙɔtᵃ & ‘to storm off’ & tɔr\'{ɛ}s\'{ʉ}ƙɔtᵃ & ‘to toss away’\\
\lspbottomrule
\end{tabularx}
\end{table}
Because the \isi{andative} suffix comes after \isi{infinitive} suffixes, whenever an \isi{andative} \isi{infinitive} is declined for case, it is the \isi{andative} suffix that takes case endings. \tabref{tab:verbs:and2} gives a declension of the [+\isi{ATR}] \isi{andative} verb \textit{séɓésuƙotí-} ‘to sweep off’, while \tabref{tab:verbs:and3} does the same for the [-\isi{ATR}] verb \textit{sɛk\'{ɛ}s\'{ʉ}ƙɔt{\Í}-} ‘to scrub off’:


\begin{table}
\caption{Case declension of \textit{séɓésuƙotí-} ‘to sweep off’}
\label{tab:verbs:and2}


\begin{tabularx}{.66\textwidth}{XXX}
\lsptoprule

& Non-final & Final\\
\midrule
\textsc{nom} & séɓésuƙota & séɓésuƙotᵃ\\
\textsc{acc} & séɓésuƙotíá & séɓésuƙotíkᵃ\\
\textsc{dat} & séɓésuƙotíé & séɓésuƙotíkᵉ\\
\textsc{gen} & séɓésuƙotíé & séɓésuƙotí\\
\textsc{abl} & séɓésuƙotúó & séɓésuƙotú\\
\textsc{ins} & séɓésuƙoto & séɓésuƙotᵒ\\
\textsc{cop} & séɓésuƙotúó & séɓésuƙotúkᵒ\\
\textsc{obl} & séɓésuƙoti & séɓésuƙotⁱ\\
\lspbottomrule
\end{tabularx}
\end{table}

\begin{table}
\caption{Case declension of \textit{sɛk\'{ɛ}s\'{ʉ}ƙɔt{\Í}-} ‘to scrub off’}
\label{tab:verbs:and3}


\begin{tabularx}{.66\textwidth}{XXX}
\lsptoprule

& Non-final & Final\\
\midrule
\textsc{nom} & sɛk\'{ɛ}s\'{ʉ}ƙɔta & sɛk\'{ɛ}s\'{ʉ}ƙɔtᵃ\\
\textsc{acc} & sɛk\'{ɛ}s\'{ʉ}ƙɔt{\Í}á & sɛk\'{ɛ}s\'{ʉ}ƙɔt{\Í}kᵃ\\
\textsc{dat} & sɛk\'{ɛ}s\'{ʉ}ƙɔt{\Í}\'{ɛ} & sɛk\'{ɛ}s\'{ʉ}ƙɔt{\Í}k\ᵋ\\
\textsc{gen} & sɛk\'{ɛ}s\'{ʉ}ƙɔt{\Í}\'{ɛ} & sɛk\'{ɛ}s\'{ʉ}ƙɔt{\Í}\\
\textsc{abl} & sɛk\'{ɛ}s\'{ʉ}ƙɔt\'{ʉ}\'{ɔ} & sɛk\'{ɛ}s\'{ʉ}ƙɔt\'{ʉ}\\
\textsc{ins} & sɛk\'{ɛ}s\'{ʉ}ƙɔtɔ & sɛk\'{ɛ}s\'{ʉ}ƙɔtᵓ\\
\textsc{cop} & sɛk\'{ɛ}s\'{ʉ}ƙɔt\'{ʉ}\'{ɔ} & sɛk\'{ɛ}s\'{ʉ}ƙɔt\'{ʉ}kᵓ\\
\textsc{obl} & sɛk\'{ɛ}s\'{ʉ}ƙɔtɨ & sɛk\'{ɛ}s\'{ʉ}ƙɔt\ᶤ\\
\lspbottomrule
\end{tabularx}
\end{table}



\subsection{Aspectuals}\label{sec:8.5}
\subsubsection{Inchoative (\textsc{inch})}\label{sec:8.5.1}

The \textsc{inchoative} suffix \{-ét-\} is identical to the venitive suffix described in \sectref{sec:8.4.1}, and this is because its meaning is a metaphorical extension of the meaning of the venitive. That is, the venitive meaning of ‘hither’ was extended to mean the beginning of a state or activity (for intransitives) or the starting up of some action or process (for transitives). The \isi{inchoative} behaves morphologically (including case declensions) exactly the same as the venitive. \tabref{tab:verbs:inch} gives a few examples of \isi{intransitive} and transitive verbs in the \isi{inchoative} \isi{aspect}:


\begin{table}
\caption{Ik \isi{inchoative} verbs}
\label{tab:verbs:inch}


\begin{tabularx}{\textwidth}{XXXX}
\lsptoprule

Intransitive &  & \multicolumn{2}{X}{Transitive}\\
\midrule
aeétón & ‘to start ripening’ & balɛt\'{ɛ}s & ‘to ignore’\\
dikwétón & ‘to start dancing’ & ewanetés & ‘to take note of’\\
ɛkw\'{ɛ}t\'{ɔ}n & ‘to start early’ & hoɗetés & ‘to liberate’\\
ɨ\'{ɛ}ɓ\'{ɛ}t\`{ɔ}n & ‘to grow cold’ & ináƙúetés & ‘to destroy’\\
l\'{ɛ}ʝ\'{ɛ}t\`{ɔ}n & ‘to catch fire’ & rɛɛt\'{ɛ}s & ‘to coerce’\\
tsekétón & ‘to grow bushy’ & taʝaletés & ‘to relinquish’\\
was\'{ɛ}t\'{ɔ}n & ‘to refuse’ & tamɛt\'{ɛ}s & ‘to ponder’\\
\lspbottomrule
\end{tabularx}
\end{table}

\subsubsection{Completive (\textsc{comp})}\label{sec:8.5.2}

The \textsc{completive} suffix \{-uƙot(í)-\} is identical to the \isi{andative} suffix described in \sectref{sec:8.4.2}, and this is because its meaning is a metaphorical extension of the meaning of the \isi{andative}. That is, the \isi{andative} meaning of ‘thither’ was extended to mean the completion of a change of state or activity (for intransitives) or the fulfillment of some action or process (for transitives). The completive behaves morphologically (including case declensions) exactly the same as the \isi{andative}. \tabref{tab:verbs:comp} gives a few examples of lexical verbs in the \isi{completive aspect}:


\begin{table}
\caption{Ik completive verbs}
\label{tab:verbs:comp}


\begin{tabularx}{\textwidth}{XXXX}
\lsptoprule

Intransitive &  & \multicolumn{2}{X}{Transitive}\\
\midrule
aeonuƙotᵃ & ‘to become ripe’ & an\'{ɛ}s\'{ʉ}ƙɔtᵃ & ‘to remember’\\
barɔnʉƙɔtᵃ & ‘to become rich’ & dɔx\'{ɛ}s\'{ʉ}ƙɔtᵃ & ‘to reprimand’\\
hábonuƙotᵃ & ‘to become hot’ & ɦyeésúƙotᵃ & ‘to learn’\\
h\'{ɛ}ɗ\'{ɔ}nʉƙɔtᵃ & ‘to shrivel up’ & kurésúƙotᵃ & ‘to defeat’\\
mɨtɔnʉƙɔtᵃ & ‘to become’ & ŋábɛsʉƙɔtᵃ & ‘to finish up’\\
sɛkɔnʉƙɔtᵃ & ‘to fade away’ & ŋƙáƙésuƙotᵃ & ‘to devour’\\
zoonuƙotᵃ & ‘to become big’ & toɓésúƙotᵃ & ‘to plunder’\\
\lspbottomrule
\end{tabularx}
\end{table}

\subsubsection{Pluractional (\textsc{plur})}\label{sec:8.5.3}

The \textsc{pluractional} suffix \{-í-\} denotes an action or state that is construed as inherently \textit{plural} in its realization. This notion of plurality can mean any of the following: 1) an \isi{intransitive} action done more than once or done by more than one subject, 2) a state attributed more than once or of more than one subject, 3) a transitive action done more than once, done by more than one subject, or done to more than one object. In short, the pluractional suffix conveys the idea that the application of the verb is multiple. The pluractional suffix comes just before the \isi{infinitive} suffix and is a dominant [+\isi{ATR}] suffix that harmonizes [-\isi{ATR}] vowels. \tabref{tab:verbs:pluract} gives a few examples of \isi{intransitive} and transitive pluractional verbs:


\begin{table}
\caption{Ik pluractional verbs}
\label{tab:verbs:pluract}


\begin{tabularx}{\textwidth}{XlXl}
\lsptoprule

Intransitive &  & \multicolumn{2}{X}{Transitive}\\
\midrule
kóníón & ‘to be one-by-one’ & abutiés & ‘to sip continually’\\
ŋatíón & ‘to run (of many)’ & esetiés & ‘to interrogate’\\
ŋkáíón & ‘to get up (of many)’ & gafariés & ‘to stab repeatedly’\\
toɓéíón & ‘to be usually right’ & nesíbiés & ‘to obey habitually’\\
tatíón & ‘to drip constantly’ & tirifiés & ‘to investigate’\\
\lspbottomrule
\end{tabularx}
\end{table}



\subsection{Voice and valence}\label{sec:8.6}
\subsubsection{Passive (\textsc{pass})}\label{sec:8.6.1}

The Ik \textsc{passive} suffix \{-ósí-\} has the unusual distinction of being able to modify both \isi{intransitive} and transitive verbs. With \isi{intransitive} verbs, it adds the nuance of characteristicness to the meaning of the verb, often with the help of root \isi{reduplication}. With transitive verbs, it has the usual function of a \isi{passive}, which is to convert the object of a transitive verb into the subject of an \isi{intransitive} verb. \tabref{tab:verbs:pass} gives examples of both \isi{intransitive} and transitive passives:


\begin{table}
\caption{Ik passives}
\label{tab:verbs:pass}


\begin{tabularx}{\textwidth}{XlXX}
\lsptoprule

Intransitive &  & \multicolumn{2}{X}{Transitive}\\
\midrule
botibotos & ‘to be migratory’ & búdòs & ‘to be hidden’\\
ɓɛkɛsɔs & ‘to be mobile’ & cookós & ‘to be guarded’\\
deƙwideƙos & ‘to be quarrelsome’ & ɗɔts\'{ɔ}s & ‘to be joined’\\
ɗɛɲɨɗɛɲɔs & ‘to be restless’ & ʝʉ\'{ɔ}s & ‘to be roasted’\\
gúránós & ‘to be hot-tempered’ & ŋáɲ\'{ɔ}s & ‘to be open’\\
mɔɲɨmɔɲɔs & ‘to be gossipy’ & ógoós & ‘to be left’\\
tsuwoós & ‘to be active’ & tsáŋós & ‘to be anointed’\\
\lspbottomrule
\end{tabularx}
\end{table}
Another quirky feature of the Ik \isi{passive} \{-ósí-\} is that it can function both as a \isi{passive} \isi{infinitive} suffix (taking case) and as a regular inflectional suffix followed by subject-agreement pronouns. When it is declined for case, it declines just like the transitive suffix \{-ésí-\} in \sectref{sec:8.2.2}. Example \REF{ex:verbs:1} below illustrates this in a sentence where the \isi{passive} \isi{infinitive} \textit{búdòsì-} ‘to be hidden’ gets the \isi{accusative case}. Then, example \REF{ex:verbs:2} shows the same \isi{passive} acting as a verb proper, taking the 3\textsc{pl} subject-agreement pronominal suffix \{-át-\}:




\ea\label{ex:verbs:1}
\gll B\'{ɛ}ɗáta   búd\textbf{òsì}-kᵃ. \\
want:\textsc{3pl}   hidden:\textsc{pass-acc}    \\
\glt ‘They want to be hidden.’ 
\z




\ea\label{ex:verbs:2}
\gll Búd\textbf{os}-átᵃ. \\
hidden:\textsc{pass}{}-\textsc{3pl:real}    \\
\glt ‘They are hidden.’ 
\z




\subsubsection{Impersonal \isi{passive} (\textsc{ips})}\label{sec:8.6.2}

The \textsc{impersonal passive} suffix \{-àn-\} behaves like a typical \isi{passive} in that it eliminates the agent of a transitive verb and promotes the object to subject. However, unlike the \isi{passive} \{-ósí-\} described above, the impersonal \isi{passive} cannot be specified for the person or number of its subject. Instead, it remains marked for 3\textsc{sg} regardless of who or what the subject may be. Another strange property of \{-\`{a}n-\} is that it can be used with \isi{intransitive} verbs as well (just like the \isi{passive}). When used with \isi{intransitive} verbs, it has the function of downplaying the identity of the subject. For this reason, it can often be translated as ‘People {\dots}’ or ‘One {\dots}’, as in \textit{Tódian} ‘People say (it)’. The impersonal \isi{passive} is a grammatical morpheme not listed in the lexicon, and so it must be illustrated in examples like \REF{ex:verbs:3}-\REF{ex:verbs:4}:




\ea\label{ex:verbs:3}
\gll Ɨn\'{ɔ}m\'{ɛ}s\textbf{àn}{à}   bì. \\
beat:\textsc{fut:ips}  you[\textsc{sg}]:\textsc{nom}    \\
\glt ‘You will be beaten.’ 
\z




\ea\label{ex:verbs:4}
\gll Ƙaí\textbf{án}{à}   ƙàƙààƙ\`{ɔ}k\ᵋ. \\
go:\textsc{plur:ips} hunt:inside:\textsc{dat}    \\
\glt ‘People go hunting.’ (Lit. ‘It is gone for hunting.’) 
\z




\subsubsection{Middle (\textsc{mid})}\label{sec:8.6.3}

Ik has two \textsc{middle} suffixes: \{-m-\} and \{-ím-\}. Like the semi-transitive construction discussed in \sectref{sec:8.2.3}, the middle suffixes convert simple transitive verbs into something in the ‘middle’ of transitive and \isi{intransitive}. That is, the Ik middle verbs convey that idea that if an action is done to an entity, it is the entity itself – if anything – doing it to itself alone, apart from any other explicit agent. The middles eliminate the agent and promote the \isi{patient} to subject. 

The middle suffix \{-m-\} always has a vowel between it and the preceding verb root. This vowel is usually a copy of the root vowel, as when \textit{ɗusés} ‘cut’ becomes \textit{ɗusúmón} ‘to cut (alone/on its own)’, but it can also have a non-copy vowel as in \textit{bokímón} ‘to get caught’. For its part, the middle suffix \{-ím-\} – a dominant [+\isi{ATR}] suffix – is always paired with the \isi{inchoative} suffix \{-ét-\}, thereby forming the complex morpheme \{-ímét-\}. \tabref{tab:verbs:middle} below gives some examples of these two suffixes converting transitive verbs to middle verbs.


\begin{table}[t]
\caption{Ik middle verbs}
\label{tab:verbs:middle}


\begin{tabularx}{\textwidth}{XXll}
\lsptoprule

Transitive &  & \multicolumn{2}{X}{Middle \{-m-\}}\\
\midrule
ŋáɲ\'{ɛ}s & ‘to open’ & ŋáɲámòn & ‘to open (alone)’\\
pakés & ‘to split’ & pakámón & ‘to split (alone)’\\
pulés & ‘to pierce’ & pulúmón & ‘to go out’\\
raʝés & ‘to return’ & raʝámón & ‘to return (alone)’\\
terés & ‘to divide’ & terémón & ‘to divide (alone)’\\
\tablevspace
Transitive &  & \multicolumn{2}{X}{Middle \{-ímét-\}}\\
\midrule
átsʼ\'{ɛ}s & ‘to chew’ & atsʼímétòn & ‘to wear out (alone)’\\
iɓéléés & ‘to overturn’ & iɓéléìmètòn & ‘to overturn (alone)’\\
kɔk\'{ɛ}s & ‘to close’ & kokíméton & ‘to close (alone)’\\
rébès & ‘to deprive’ & rébìmètòn & ‘to be deprived (alone)’\\
tɔrɛɛs & \multicolumn{1}{X}{‘to coerce’} & toreimétòn & ‘to be coerced (alone)’\\
\lspbottomrule
\end{tabularx}
\end{table}

\subsubsection{Reciprocal (\textsc{recip})}\label{sec:8.6.4}

The \textsc{reciprocal} suffix \{-ínósí-\} denotes a \isi{reciprocal} relationship that a verb’s subject has with itself. That is, the \isi{reciprocal} collapses the subject and direct object of a transitive verb, or the subject and a secondary object of an \isi{intransitive} verb, into just the subject of a \isi{reciprocal} verb. In this regard, it is similar to the semi-transitive verbs from \sectref{sec:8.2.3} that use the \isi{reflexive} pronoun \textit{as{\Í}-} ‘-self’. \tabref{tab:verbs:recip1} provides a few examples of reciprocals derived from other verbs:


\begin{table}[t]
\caption{Ik \isi{reciprocal} verbs}
\label{tab:verbs:recip1}


\begin{tabularx}{\textwidth}{XlXl}
\lsptoprule

\multicolumn{2}{X}{Intransitive} & \multicolumn{2}{X}{Reciprocal}\\
\midrule
ɓɛƙ\'{ɛ}s & ‘to walk’ & ɓɛƙ\'{ɛ}s{\Í}n\'{ɔ}s & ‘to walk together’\\
ɨɓák\'{ɔ}n & ‘to be next to’ & ɨɓák{\Í}n\'{ɔ}s & ‘to be next to each other’\\
tódòn & ‘to speak’ & tódinós & ‘to speak to each other’\\
\tablevspace
\multicolumn{2}{X}{Transitive} & Reciprocal & \\
\midrule
ɦyeés & ‘to know’ & ɦyeínós & ‘to be related’\\
ɨŋaar\'{ɛ}s & ‘to help’ & ɨŋáár{\Í}n\'{ɔ}s & ‘to help each other’\\
m{\Í}n\'{ɛ}s & ‘to love’ & m{\Í}n{\Í}n\'{ɔ}s & ‘to love each other’\\
\lspbottomrule
\end{tabularx}
\end{table}

\newpage 
Like the \isi{passive} \{-ósí-\} discussed in \sectref{sec:8.6.1}, the \isi{reciprocal} suffix can take either case endings (as a morphological noun) or subject-agreement endings (as a morphological verb). A case declension of \textit{ínínósí-} ‘to cohabitate’ is shown in \tabref{tab:verbs:recip2}, and in example \REF{ex:verbs:5} below, the \isi{reciprocal} verb \textit{ɨɓák{\Í}n\'{ɔ}s{\Í}-} ‘to be next to each other’ gets the \isi{accusative case}. Then, example \REF{ex:verbs:6} shows the same verb acting as a verb proper, with the 3\textsc{pl} subject-agreement marker \{-át-\}:


\begin{table}
\caption{Case declension of \textit{ínínósí-} ‘to cohabitate’}
\label{tab:verbs:recip2}


\begin{tabularx}{.66\textwidth}{XXX}
\lsptoprule

& Non-final & Final\\
\midrule
\textsc{nom} & ínínósá & ínínós\\
\textsc{acc} & ínínósíà & ínínósíkᵃ\\
\textsc{dat} & ínínósíè & ínínósíkᵉ\\
\textsc{gen} & ínínósíè & ínínósí\\
\textsc{abl} & ínínósúò & ínínósú\\
\textsc{ins} & ínínósó & ínínósᵒ\\
\textsc{cop} & ínínósúò & ínínósúkᵒ\\
\textsc{obl} & ínínósí & ínínós\\
\lspbottomrule
\end{tabularx}
\end{table}



\ea\label{ex:verbs:5}
\gll B\'{ɛ}ɗáta     ɨɓák\textbf{{\Í}n\'{ɔ}s{\Í}}-kᵃ. \\
want:\textsc{3pl:real}   next.to:\textsc{recip-acc}    \\
\glt ‘They want to be next to each other.’ 
\z




\ea\label{ex:verbs:6}
\gll Ɨɓák\textbf{{\Í}n\'{ɔ}s}-átᵃ. \\
next.to:\textsc{recip-3pl:real}    \\
\glt ‘They are next to each other.’ 
\z




\subsubsection{Causative (\textsc{caus})}\label{sec:8.6.5}

Ik expresses causativity with a morphological \isi{causative}, the \textsc{causative} suffix \{-ìt-\}. When this suffix is added to a verb with meaning X, it changes the meaning of the verb to ‘cause/make (to) X’. This suffix can be used to causativize \isi{intransitive} and transitive verbs and comes right after the verb root, before the \isi{infinitive} marker (if present) and any other suffixes like an \isi{inchoative} or pluractional. If the last vowel of the verb root is /u/, the \isi{causative} may be assimilated to the form \{-ùt-\}. \tabref{tab:verbs:caus} gives several examples of causativized verbs.


\begin{table}
\caption{Ik \isi{causative} verbs}
\label{tab:verbs:caus}


\begin{tabularx}{\textwidth}{XXXX}
\lsptoprule

\multicolumn{2}{X}{Intransitive} & \multicolumn{2}{X}{Causative}\\
\midrule
bùkòn & ‘to be prostrate’ & bukites & ‘to lay prostrate’\\
itúrón & ‘to be proud’ & itúrútés & ‘to praise’\\
x\`{ɛ}ɓ\`{ɔ}n & ‘to be timid’ & xɛɓɨtɛs & ‘to intimidate’\\
\tablevspace
\multicolumn{2}{X}{Transitive} &  & \\
\midrule
dimés & ‘to refuse’ & dimités & ‘to prohibit’\\
naƙw\'{ɛ}s & ‘to suckle’ & naƙwɨt\'{ɛ}s & ‘to give suckle’\\
zízòn & ‘to be fat’ & zízités & ‘to fatten’\\
\lspbottomrule
\end{tabularx}
\end{table}



\subsection{Subject-agreement}\label{sec:8.7}


Whenever Ik grammar requires verbs to agree with their subjects, one of the seven pronominal suffixes in \tabref{tab:verbs:subj} are used. Note that if the verb contains [-\isi{ATR}] vowels, these suffixes will also be harmonized to [-\isi{ATR}]. Just like the free pronouns described back in \sectref{sec:5.2}, these bound pronominal suffixes are organized along three axes: 1) person (1/2/3), 2) number (singular/plural), and 3) \isi{clusivity} (exclusive/inclusive). The form these pronominals ultimately take depends on the grammatical mood of the verb to which they attach. If the verb is in the irrealis mood (see \sectref{sec:8.9.1}), the suffixes appear with their underlying forms. Whereas if they are in the realis mood (see \sectref{sec:8.9.2}), the realis suffix \{-a\} first subtracts or deletes their final vowel. The difference in the two mood-based paradigms is depicted in \tabref{tab:verbs:subj}. To see instances of the Ik subject-agreement suffixes in actual language use, you may refer back to example \REF{ex:case:11} in \sectref{sec:7.2}.


\begin{table}
\caption{Ik subject-agreement suffixes}
\label{tab:verbs:subj}


\begin{tabularx}{\textwidth}{XXXXX}
\lsptoprule

& \multicolumn{2}{X}{Irrealis} & \multicolumn{2}{X}{Realis}\\
& \textsc{nf} & \textsc{ff} & \textsc{nf} & \textsc{ff}\\
\midrule
\textsc{1sg} & {}-íí & {}-í & {}-íá & {}-í\\
\textsc{2sg} & {}-ídì & {}-îdⁱ & {}-ídà & {}-îdᵃ\\
\textsc{3sg} & {}-ì & {}-\textsuperscript{i} & {}-a & {}-\textsuperscript{a}\\
\textsc{1pl.exc} & {}-ímí & {}-ím & {}-ímá & {}-ím\\
\textsc{1pl.inc} & {}-ísínì & {}-ísín & {}-ísínà & {}-ísín\\
\textsc{2pl} & {}-ítí & {}-ítⁱ & {}-ítá & {}-ítᵃ\\
\textsc{3pl} & {}-átì & {}-átⁱ & {}-átà & {}-átᵃ\\
\lspbottomrule
\end{tabularx}
\end{table}






\subsection{Dummy pronoun (\textsc{dp})}\label{sec:8.8}


Ik has a special verbal affix called the \textsc{dummy pronoun} because it represents a secondary (indirect) object that has been (re)moved. That is, the \isi{dummy pronoun} is a form of object-marking on the verb, but not of direct object marking. For example, if an indirect object expressing location or time or means is moved to the front of a clause for emphasis, it leaves a trace on the verb in the form of the \isi{dummy pronoun}. Seen from another perspective, the \isi{dummy pronoun} is always a clue that there is a missing syntactic constituent in the clause.

The \isi{dummy pronoun} has the form \{-\'{}dè\} and is very volatile in terms of \isi{allomorphy}, dramatically changing its form in different morpho-phonological environments. Once the /d/ is lost in non-final forms, \isi{vowel assimilation} and \isi{vowel harmony} so distort the \isi{dummy pronoun} as to make it almost unrecognizable at times. \tabref{tab:verbs:dummy} below is given to illustrate its diverse \isi{allomorphy}:


\begin{table}
\caption{Allomorphs of the \isi{dummy pronoun} \{-\'{}dè\}}
\label{tab:verbs:dummy}


\begin{tabularx}{.5\textwidth}{XXX}
\lsptoprule

& Non-final & Final\\
\midrule
\{-\'{}dè\} & {}-\'{}è & {}-\'{}dᵉ\\
& {}-\'{}\`{ɛ} & {}-\'{}d\ᵋ\\
& {}-\'{}ì & \\
& {}-\'{}{\Ì} & \\
& {}-\'{}ò & \\
& {}-\'{}\`{ɔ} & \\
\lspbottomrule
\end{tabularx}
\end{table}
Examples \REF{ex:verbs:7}-\REF{ex:verbs:8} illustrate the \isi{dummy pronoun} in two different morphological forms: final and non-final. Note that the tones associated with the pronoun in these examples do not match what is shown in \tabref{tab:verbs:dummy}; this is because of local tonal interference. In terms of function, the \isi{dummy pronoun} in \REF{ex:verbs:7} indicates that an indirect object – the destination of the verb \textit{ƙáátà} ‘they go (went)’ – has been displaced from its usual spot after the verb to a place of focus at the beginning of the sentence (\textit{Ntsúó}). Then in \REF{ex:verbs:8}, the \isi{dummy pronoun} marks an indirect object – the location of staying – that is missing from the clause entirely. Since this sentence was taken out of context from a story, most likely the missing object had been already mentioned earlier in the discourse:




\ea\label{ex:verbs:7}
\gll Ntsúó=noo     Icéá     ƙáátà-\textbf{d\ᵉ}. \\
it:\textsc{cop}=\textsc{past}    Ik:\textsc{acc}   go:\textsc{3pl:real-dp}    \\
\glt ‘It’s where the Ik went (to).’ 
\z




\ea\label{ex:verbs:8}
\gll Jʼɛʝʉk\'{ɔ}-\textbf{\'{ɔ}}     sàà     ròɓàᵉ. \\
stay:\textsc{3sg:seq-dp}   other:\textsc{nom}   people:\textsc{gen}    \\
\glt ‘And other people stayed (there).’ 
\z






\subsection{Mood}\label{sec:8.9}
\subsubsection{Irrealis (\textsc{irr})}\label{sec:8.9.1}

A basic distinction in grammatical \textsc{mood} cleaves Ik verbal aspects and modalities right down the center, and this distinction is between \textsc{irrealis} and \textsc{realis}. As it applies specifically to Ik, the irrealis mood includes states and events whose \textit{actuality} or \textit{reality} are not expressly encoded in the grammar. Another way of saying this is that irrealis verbs in Ik can convey anything \textit{but} whether a state or event has happened, is happening, or will happen. The morphological manifestation of the irrealis is that the final suffix of an irrealis verb – a subject-agreement pronoun – surfaces with its underlying form (see \tabref{tab:verbs:subj}). 

The verbal aspects and modalities that fall under the irrealis mood include the \textsc{optative}, \textsc{subjunctive}, \textsc{imperative}, \textsc{negative}, \textsc{sequential}, and \textsc{simultaneous}. 


\subsubsection{Realis (\textsc{real})}\label{sec:8.9.2}

In contrast to irrealis, the \textsc{realis} mood includes states and events whose actuality or reality \textit{are} encoded in the grammar. That is to say, realis verbs in Ik include in their meaning the fact that something has taken place, is taking place, or will take place in the real world. The morphological manifestation of the realis mood is seen in the realis suffix \{-a\} that subtracts or deletes the final vowel of the subject-agreement suffix to which it attaches (again, see \tabref{tab:verbs:subj}). In terms of verb types, the realis mood includes declarative statements in the past or non-past, questions about the past or non-past, and, rather paradoxically, negative imperatives (which one might expect to fall under irrealis).



\newpage 
\subsection{Verb paradigms}\label{sec:8.10}
\subsubsection{Intentional-\isi{imperfective} (\textsc{int/ipfv})}\label{sec:8.10.1}

The \textsc{intentional-imperfective} \isi{aspect} suffix \{-és-\} has two functions, hence its hyphenated title. One is to denote either an intention on the part of animate subjects or an imminence on the part of inanimate subjects. It is in this role that it finds use as the usual translation for the English future \isi{tense}. It is also the answer to the question, “How do you express future \isi{tense} in Ik?” A second function is to denote grammatical imperfectivity: a sense that a state or event is ongoing or incomplete. The two concepts collapse into one when intention/imminence is viewed as the incomplete coming-to-be of a future state or event.  And even though intention or imperfectivity may seem to fall under an irrealis mood, \{-és-\} can actually be used with verbs in either the realis or irrealis mood. In \tabref{tab:verbs:int}, \{-és-\} is illustrated with the verb \textit{àts-} ‘come’ in its \isi{imperfective} sense with a recent past \isi{tense} marker (\textit{nák\ᵃ}) and then in its \isi{intentional} (English `future') sense:


\begin{table}
\caption{Ik intentional-\isi{imperfective} aspect}
\label{tab:verbs:int}


\begin{tabularx}{\textwidth}{XXX}
\lsptoprule

\multicolumn{2}{X}{Imperfective} & \\
\midrule
\textsc{1sg} & Atsésíà nàkᵃ. & ‘I was coming.’\\
\textsc{2sg} & Atsésídà nàkᵃ. & ‘You were coming.’\\
\textsc{3sg} & Atsesa nákᵃ. & ‘(S)he/it was coming.’\\
\textsc{1pl.exc} & Atsésímà nàkᵃ & ‘We were coming.’\\
\textsc{1pl.inc} & Atsésísìnà nàkᵃ. & ‘We all were coming.’\\
\textsc{2pl} & Atsésítà nàkᵃ. & ‘You all were coming.’\\
\textsc{3pl} & Atsésátà nàkᵃ. & ‘They were coming.’\\
\tablevspace
\multicolumn{2}{X}{Intentional} & \\
\midrule
\textsc{1sg} & Atsésí. & ‘I will come.’\\
\textsc{2sg} & Atsésîdᵃ. & ‘You will come.’\\
\textsc{3sg} & Atsés. & ‘(S)he/it will come.’\\
\textsc{1pl.exc} & Atsésím. & ‘We will come.’\\
\textsc{1pl.inc} & Atsésísìn. & ‘We all will come.’\\
\textsc{2pl} & Atsésítᵃ. & ‘You all will come.’\\
\textsc{3pl} & Atsésátᵃ. & ‘They will come.’\\
\lspbottomrule
\end{tabularx}
\end{table}

\subsubsection{Present perfect (\textsc{prf})}\label{sec:8.10.2}

The Ik \textsc{present perfect} suffix \{-\'{}ka\} denotes a state or event recently completed (‘perfected’) but still relevant in the present. The suffix has a ‘floating’ high tone that shows up on the preceding \isi{syllable} of 3\textsc{sg} verbs, for example in \textit{Nabʉƙɔták\ᵃ} ‘It is finished’. The /k/ in \{-\'{}ka\} disappears in non-final environments, making \{-\'{}a\} an allomorph. \tabref{tab:verbs:perf} presents the paradigm of the present perfect with the verb \textit{àts-} ‘come’ in both non-final and final environments:


\begin{table}
\caption{Ik present perfect aspect}
\label{tab:verbs:perf}


\begin{tabularx}{\textwidth}{XXXXl}
\lsptoprule

\multicolumn{2}{X}{} & Non-final & Final & \\
\midrule
\textsc{1sg} & \multicolumn{2}{X}{Atsíàà {\dots}} & Atsíàkᵃ. & ‘I have come’\\
\textsc{2sg} & \multicolumn{2}{X}{Atsídàà {\dots}} & Atsídàkᵃ. & ‘You have come’\\
\textsc{3sg} & \multicolumn{2}{X}{Atsáá {\dots}} & Atsákᵃ. & ‘She has come’\\
\textsc{1pl.exc} & \multicolumn{2}{X}{Atsímáà {\dots}} & Atsímákᵃ. & ‘We have come’\\
\textsc{1pl.inc} & \multicolumn{2}{X}{Atsísínàà {\dots}} & Atsísínàkᵃ. & ‘We all have come’\\
\textsc{2pl} & \multicolumn{2}{X}{Atsítáà {\dots}} & Atsítákᵃ. & ‘You all have come’\\
\textsc{3pl} & \multicolumn{2}{X}{Atsátàà {\dots}} & Atsátàkᵃ. & ‘They have come’\\
\lspbottomrule
\end{tabularx}
\end{table}

\subsubsection{Optative (\textsc{opt})}\label{sec:8.10.3}

The Ik \textsc{optative} mood is used to express wishes, even ironic ones like ‘Let the enemies comeǃ’. Optative verbs are often introduced with \isi{imperative} verbs like \textit{\'{O}goe} or \textit{Taláké}, both of which mean ‘Let {\dots}’. And all Ik optative verbs are translated into English with a sentence beginning with ‘Let {\dots}’ or ‘May {\dots}’. 

Morphologically, the optative is marked by a combination of tone and special irregular suffixes. All optative verbs except 3\textsc{pl} show a kind of high-tone ‘leveling’ in the subject-agreement suffixes. The leveled high tone is pushed out to the end, creating a floating high tone. This high tone is not seen except in the fact that the last \isi{syllable} of the subject-agreement suffixes remains at mid-tone level. Besides tone, special irregular suffixes mark the optative in \textsc{1sg}, 1\textsc{pl.exc}, and 1\textsc{pl.inc} verbs, while standard irrealis suffixes are used for the other paradigm members. Note that the 1\textsc{pl.inc} may also be called the ‘hortative’. Another peculiarity of the Ik optative is that there is no difference between its non-final and final forms. \tabref{tab:verbs:opt} presents the optative on the verb \textit{àts-} ‘come’:


\begin{table}
\caption{Ik optative mood}
\label{tab:verbs:opt}


\begin{tabularx}{.66\textwidth}{lXl}
\lsptoprule

\textsc{1sg} & Atsine. & ‘Let me come.’\\
\textsc{2sg} & Atsidi. & ‘May you come.’\\
\textsc{3sg} & Atsi. & ‘Let her come.’\\
\textsc{1pl.exc} & Atsima. & ‘Let us come.’\\
\textsc{1pl.inc} & Atsano. & ‘Let us all come.’\\
\textsc{2pl} & Atsiti. & ‘May you all come.’\\
\textsc{3pl} & Atsáti. & ‘Let them come.’\\
\lspbottomrule
\end{tabularx}
\end{table}

\subsubsection{Subjunctive (\textsc{subj})}\label{sec:8.10.4}

The Ik \textsc{subjunctive} mood is used to encode statements that are somehow contingent or temporally unrealized. In that regard, it is an essentially irrealis verb form because it captures states or events that have not yet happened. It is also essentially irrealis in that it is marked simply by the absence of any marking. In other words, the subject-agreement suffixes surface with their underlying forms in the \isi{subjunctive mood}, just as they appear in \tabref{tab:verbs:subj}. The subjunctive is usually introduced either by \textit{ɗɛmʉsʉ} ‘unless, until’ or \textit{damu (kóʝa)} ‘may’. \tabref{tab:verbs:subjct} gives the full subjunctive paradigm with the verb \textit{àts-} ‘come’:


\begin{table}
\caption{Ik subjunctive mood}
\label{tab:verbs:subjct}


\begin{tabularx}{\textwidth}{Xlll}
\lsptoprule

& Non-final & Final & \\
\midrule
\textsc{1sg} & ɗɛmʉsʉ atsíí {\dots} & ɗɛmʉsʉ atsí. & ‘unless I come’\\
\textsc{2sg} & ɗɛmʉsʉ atsídì {\dots} & ɗɛmʉsʉ atsîdⁱ. & ‘unless you come’\\
\textsc{3sg} & ɗɛmʉsʉ atsi {\dots} & ɗɛmʉsʉ atsⁱ. & ‘unless she comes’\\
\textsc{1pl.exc} & ɗɛmʉsʉ atsímí {\dots} & ɗɛmʉsʉ atsím. & ‘unless we come’\\
\textsc{1pl.inc} & ɗɛmʉsʉ atsísínì {\dots} & ɗɛmʉsʉ atsísín. & ‘unless we all come’\\
\textsc{2pl} & ɗɛmʉsʉ atsítí {\dots} & ɗɛmʉsʉ atsítⁱ. & ‘unless you all come’\\
\textsc{3pl} & ɗɛmʉsʉ atsátì {\dots} & ɗɛmʉsʉ atsátⁱ. & ‘unless they come’\\
\lspbottomrule
\end{tabularx}
\end{table}

\subsubsection{Imperative (\textsc{imp})}\label{sec:8.10.5}

The \textsc{imperative} mood is used to issue commands or instructions. If the \isi{recipient} of the command is singular, then the suffix used is \{-e\'{}\}, and if the \isi{recipient} is plural, the suffix is \{-úó\}. The singular \{-e\'{}\} has a floating high tone that raises any preceding low tones to mid. Both \isi{imperative} suffixes are appended to the end of the verb stem, and no subject-agreement markers are needed. Both \isi{imperative} suffixes are subject to vowel \isi{devoicing} before a pause, as shown in \tabref{tab:verbs:imp}:


\begin{table}
\caption{Ik \isi{imperative} mood}
\label{tab:verbs:imp}


\begin{tabularx}{\textwidth}{XXXXXX}
\lsptoprule

Singular &  &  & Plural &  & \\
\textsc{nf} & \textsc{ff} &  & \textsc{nf} & \textsc{ff} & \\
\midrule
Atse..ǃ & Atsᵉǃ & ‘Comeǃ’ & Atsúó..ǃ & Atsúǃ & ‘Comeǃ’\\
Ƙae..ǃ & Ƙaᵉǃ & ‘Goǃ’ & Ƙoyúó..ǃ & Ƙoyúǃ & ‘Goǃ’\\
Ŋƙɛ..ǃ & {Ŋƙ\ᵋ}ǃ & ‘Eatǃ’ & Ŋƙ\'{ʉ}\'{ɔ}..ǃ & Ŋƙ\'{ʉ}ǃ & ‘Eatǃ’\\
Zɛƙwɛ..ǃ & {Zɛƙw\ᵋ}ǃ & ‘Sitǃ’ & Zɛƙ\'{ʉ}\'{ɔ}..ǃ & Zɛƙ\'{ʉ}ǃ & ‘Sitǃ’\\
\lspbottomrule
\end{tabularx}
\end{table}

\subsubsection{Negative}\label{sec:8.10.6}

Ik negates clauses by means of verblike particles that come first in the negative clause. If the negated clause has a realis verb, then the negator \isi{particle} used is \textit{ńtá} ‘not’. If the negated clause has an irrealis verb, then the negator \isi{particle} is \textit{mòò} or \textit{nòò}. Lastly, if the negated clause is past \isi{tense} realis or present perfect realis, then the negator \isi{particle} used is \textit{máà} or \textit{náà}. In the negated clause, the negator \isi{particle} comes first, followed by the subject, and then the verb. Any negated verb takes the irrealis mood with the appropriate form of subject-agreement suffixes (see \tabref{tab:verbs:subj}). To make all this more concrete, \tabref{tab:verbs:neg} gives example of the different negator particles used with different types of clauses.


\begin{table}
\caption{Ik negative mood}
\label{tab:verbs:neg}


\begin{tabularx}{\textwidth}{XXl}
\lsptoprule

Realis &  & \\
\textsc{1sg} & \'{N}tá ɦyeí. & ‘I don’t know.’\\
\textsc{2sg} & \'{N}tá ɦyeîdⁱ. & ‘You don’t know.’\\
\textsc{3sg} & \'{N}tá ɦyèⁱ. & ‘She doesn’t know.’\\
\tablevspace
Sequential &  & \\
\textsc{1sg} & {\dots} moo ɦyeí. & ‘ {\dots} and I don’t know.’\\
\textsc{2sg} & {\dots} moo ɦyeîdⁱ. & ‘ {\dots} and you don’t know.’\\
\textsc{3sg} & {\dots} mòò ɦyèⁱ. & ‘ {\dots} and she doesn’t know.’\\
\tablevspace
Past realis &  & \\
\textsc{1sg} & Máa naa ɦyeí. & ‘I didn’t know.’\\
\textsc{2sg} & Máa naa ɦyeîdⁱ. & ‘You didn’t know.’\\
\textsc{3sg} & Máà nàà ɦyèⁱ. & ‘She didn’t know.’\\
\lspbottomrule
\end{tabularx}
\end{table}

\subsubsection{Sequential (\textsc{seq})}\label{sec:8.10.7}

The Ik \textsc{sequential} \isi{aspect} expresses states or events that happen in sequence. Usually a sequence of verbs starts with an anchoring non-sequential verb and/or \isi{time expression}, and then a \textsc{clause} \textsc{chain} begins in the \isi{sequential aspect}. For example, when someone tells a story, they may start with one or two past \isi{tense} realis verbs to set the stage and then continue the narrative with sequential verbs. Or if someone is giving a set of instructions, they may start with one or two \isi{imperative} verbs followed by a chain of sequential verbs. Because of its versatility, the Ik \isi{sequential aspect} is the language’s most frequently used verb form.

Morphologically, Ik sequential verbs are recognized by a combination of tone, irregular subject-agreement suffixes, and the \isi{sequential aspect} suffix \{-ko\}. Specifically, all 1 and 2-person sequential verbs exhibit high-tone leveling in their subject-agreement suffixes, which pushes a high tone out to the right of the verb. This floating high raises the preceding low tones to mid. These tone effects, plus the irregular suffixes, and the sequential marker \{-ko\} are shown in \tabref{tab:verbs:seq}. Note that the sequential paradigm also has an impersonal \isi{passive} marked with the suffix \{-ese\}. Its function is identical to that of the impersonal \isi{passive} described back in \sectref{sec:8.6.2}. For more on how the \isi{sequential aspect} works in actual language contexts, skip ahead to the discussion of clause-chaining in \sectref{sec:10.8.2}.


\begin{table}
\caption{Ik sequential aspect}
\label{tab:verbs:seq}


\begin{tabularx}{\textwidth}{XXXl}
\lsptoprule

& Non-final & Final & \\
\midrule
\textsc{1sg} & {\dots} atsiaa {\dots} & {\dots} atsiakᵒ. & ‘and I come’\\
\textsc{2sg} & {\dots} atsiduo {\dots} & {\dots} atsidukᵒ. & ‘and you come’\\
\textsc{3sg} & {\dots} àtsùò {\dots} & {\dots} àtsùkᵒ. & ‘and she comes’\\
\textsc{1pl.exc} & {\dots} atsimaa {\dots} & {\dots} atsimakᵒ. & ‘and we come’\\
\textsc{1pl.inc} & {\dots} atsisinuo {\dots} & {\dots} atsisinukᵒ. & ‘and we all come’\\
\textsc{2pl} & {\dots} atsituo {\dots} & {\dots} atsitukᵒ. & ‘and you all come’\\
\textsc{3pl} & {\dots} àtsìnì {\dots} & {\dots} àtsìn. & ‘and they come’\\
\textsc{ips} & {\dots} atsese {\dots} & {\dots} atses. & ‘and people come’\\
\lspbottomrule
\end{tabularx}
\end{table}



\subsubsection{Simultaneous (\textsc{sim})}\label{sec:8.10.8}

The Ik \textsc{simultaneous} \isi{aspect} is used to express states or events that are happening simultaneously to another state or event. In contrast to the sequential, the \isi{simultaneous aspect} can only be used in subordinate clauses. That is to say, simultaneous clauses usually cannot stand alone without a \isi{main clause} (with some exceptions). Because of its role of supporting sequential clauses, the \isi{simultaneous aspect} is also commonly found in narratives and other longer discourses. It can be given a perfective interpretation as in ‘when I came’ or an \isi{imperfective} one as in ‘while I was coming’. Morphologically, the \isi{simultaneous aspect} is marked by the suffix \{-ke\}, which is affixed to the subject-agreement suffixes in their irrealis forms. \tabref{tab:verbs:sim} presents the simultaneous paradigm of \textit{àts-} ‘come’:


\begin{table}
\caption{Ik simultaneous aspect}
\label{tab:verbs:sim}


\begin{tabularx}{\textwidth}{XXXl}
\lsptoprule

& Non-final & Final & \\
\midrule
\textsc{1sg} & {\dots} atsííkè {\dots} & {\dots} atsííkᵉ. & ‘while I was coming’\\
\textsc{2sg} & {\dots} atsídìè {\dots} & {\dots} atsídìkᵉ. & ‘while you were coming’\\
\textsc{3sg} & {\dots} àtsìè {\dots} & {\dots} àtsìkᵉ. & ‘while she was coming’\\
\textsc{1pl.exc} & {\dots} atsímíè {\dots} & {\dots} atsímíkᵉ. & ‘while we were coming’\\
\textsc{1pl.inc} & {\dots} atsísínìè {\dots} & {\dots} atsísínìkᵉ. & ‘while we all were coming’\\
\textsc{2pl} & {\dots} atsítíè {\dots} & {\dots} atsítíkᵉ. & ‘while you all were coming’\\
\textsc{3pl} & {\dots} atsátìè {\dots} & {\dots} atsátìkᵉ. & ‘while they were coming’\\
\lspbottomrule
\end{tabularx}
\end{table}



\subsection{Adjectival verbs}\label{sec:8.11}
\subsubsection{Overview}\label{sec:8.11.1}

Since Ik does not have a separate word class of adjectives, it conveys \isi{adjectival} concepts with \textsc{\isi{adjectival} verbs}. These verbs have \isi{adjectival} meanings but otherwise mostly behave like \isi{intransitive} verbs. One way they do differ from normal \isi{intransitive} verbs, though, is in the specific \isi{adjectival} suffixes they can take. The next four subsections briefly describe these special \isi{adjectival} suffixes.


\subsubsection{Physical property I (\textsc{phys1})}\label{sec:8.11.2}

The \textsc{physical property i} \isi{adjectival} suffix \{-\'{}d-\} is found on \isi{adjectival} verbs that express physical properties like appearance, size, shape, consistency, texture, and other tangible attributes. As a result, physical property I verbs are some of the language’s most colorful adjectivals. Physical property I verbs all contain two syllables with LH tone pattern, and in the \isi{infinitive}, they take the \isi{intransitive} suffix \{-ònì-\}. \tabref{tab:verbs:phys1} gives a sample of these colorful descriptive terms:


\begin{table}
\caption{Ik physical property I \isi{adjectival} verbs}
\label{tab:verbs:phys1}


\begin{tabularx}{\textwidth}{XX}
\lsptoprule

bufúdòn & ‘to be spongy’\\
ɗɔm\'{ɔ}d\`{ɔ}n & ‘to be gluey’\\
dirídòn & ‘to be compacted’\\
ʝamúdòn & ‘to be velvety’\\
lɛtsʼ\'{ɛ}d\`{ɔ}n & ‘to be bendy’\\
pɨɗ{\Í}d\`{ɔ}n & ‘to be sleek’\\
tsakádòn & ‘to be watery’\\
\lspbottomrule
\end{tabularx}
\end{table}

\subsubsection{Physical property II (\textsc{phys2})}\label{sec:8.11.3}

The \textsc{physical property ii} \isi{adjectival} suffix \{-m-\} is found in \isi{adjectival} verbs that also express physical properties like appearance, color, consistency, posture, shape, and texture. It can also express less tangible attributes like strength, weakness, quality, or personality traits. Physical property II verbs usually contain two syllables with a LH tone pattern or three syllables with a LHH tone pattern (without the \isi{infinitive} suffix), and in the \isi{infinitive}, they take the \isi{intransitive} suffix \{-ònì\}. \tabref{tab:verbs:phys2} gives a sample of these descriptive \isi{adjectival} verbs in two groupings.


\begin{table}[p]
\caption{Ik physical property II \isi{adjectival} verbs}
\label{tab:verbs:phys2}


\begin{tabularx}{\textwidth}{XX}
\lsptoprule

Bisyllabic & \\
\midrule
buɗámón & ‘to be black’\\
d\'{ʉ}g\`{ʉ}m\`{ɔ}n & ‘to be hunched’\\
firímón & ‘to be clogged’\\
kikímón & ‘to be stocky’\\
kwɛtsʼ\'{ɛ}m\'{ɔ}n & ‘to be damaged’\\
\tablevspace
Trisyllabic & \\
\midrule
bulúƙúmòn & ‘to be bulbous’\\
ʝʉr\'{ʉ}t\'{ʉ}m\`{ɔ}n & ‘to be slippery\\
pelérémòn & ‘to be squinty’\\
ságwàràmòn & ‘to be shadeless’\\
t\'{ɛ}ƙ\'{ɛ}z\`{ɛ}m\`{ɔ}n & ‘to be shallow\\
\lspbottomrule
\end{tabularx}
\end{table}

\subsubsection{Stative (\textsc{stat})}\label{sec:8.11.4}

The \textsc{stative} \isi{adjectival} suffix \{-án-\} forms \isi{adjectival} verbs that express an ongoing state characterized by the meaning of a noun or a transitive verb. Because \{-án-\} contains the vowel /a/, it prevents \isi{vowel harmony} from spreading between the verbal root and any suffixes that follow the stative suffix (for example, \isi{infinitive} or subject-agreement suffixes). \tabref{tab:verbs:stat1} and \tabref{tab:verbs:stat2} present a few examples of \isi{stative adjectival} verbs derived from nouns and verbs, respectively:


\begin{table}[p]
\caption{Ik stative verbs derived from nouns}
\label{tab:verbs:stat1}


\begin{tabularx}{\textwidth}{XXXX}
\lsptoprule

Noun &  & Stative verb & \\
\midrule
cué- & ‘water’ & cuanón & ‘to be liquid’\\
\'{ɛ}sá- & ‘drunkenness’ & ɛsánón & ‘to be drunk’\\
kirotí- & ‘sweat’ & kirotánón & ‘to be sweaty’\\
ɲ\`{ɛ}ƙ\`{ɛ}- & ‘hunger’ & ɲɛƙánón & ‘to be hungry’\\
ɲèrà- & ‘girls’ & iɲéráánón & ‘to be girl-crazy’\\
\lspbottomrule
\end{tabularx}
\end{table}




\begin{table}[p]
\caption{Ik stative verbs derived from transitive verbs}
\label{tab:verbs:stat2}


\begin{tabularx}{\textwidth}{XXXl}
\lsptoprule

Transitive &  & Stative & \\
\midrule
ɓɛk\'{ɛ}s & ‘to provoke’ & ɓɛkánón & ‘to be provocative’\\
dzɛr\'{ɛ}s & ‘to tear’ & dzɛrɛdzɛránón & ‘to be torn in shreds’\\
itáléés & ‘to forbid’ & itáléánón & ‘to be forbidden’\\
itukes & ‘to heap’ & itukánón & ‘to be congregated’\\
ɨraŋɛs & ‘to spoil’ & ɨráŋ\'{ʉ}nánón & ‘to be spoiled’\\
\lspbottomrule
\end{tabularx}
\end{table}

\subsubsection{Distributive (\textsc{distr})}\label{sec:8.11.5}

Ik has two \textsc{distributive} \isi{adjectival} suffixes: \{-aák-\} and \{-ìk-\}. These suffixes have the function of distributing the meaning of an \isi{adjectival} verb to more than one subject. The suffix \{-aák-\} can be used with all kinds of \isi{adjectival} verbs, including the physical property and stative varieties, while the suffix \{-ìk-\} has been found only with the two verbs of size, \textit{kwáts-} ‘small’ and \textit{zè-} ‘large’. Moreover, it commonly occurs together with \{-aák-\}, as in \textit{kwátsíkaakón} ‘to be small (of many)’ and \textit{zeikaakón} ‘to be large (of many)’. \tabref{tab:verbs:distr} gives a sampling of \isi{adjectival} verbs with the distributive suffix:


\begin{table}
\caption{Ik \isi{distributive adjectival} verbs}
\label{tab:verbs:distr}


\begin{tabularx}{.66\textwidth}{lX}
\lsptoprule

buɗúdaakón & ‘to be soft (of many)’\\
ɓetsʼaakón & ‘to be white (of many)’\\
gaanaakón & ‘to be bad (of many)’\\
kúɗaakón & ‘to be short (of many)’\\
maráŋaakón & ‘to be good (of many)’\\
nɔts\'{ɔ}daakón & ‘to be adhesive (of many)’\\
semélémaakón & ‘to be elliptical (of many)’\\
\lspbottomrule
\end{tabularx}
\end{table}

