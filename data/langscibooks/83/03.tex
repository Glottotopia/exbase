% CHAPTER 3 SUBSTANTIVES
\chapter{Substantives}\label{ch:substantives}
This chapter covers the various substantives in Southern Yauyos Quechua. It surveys their different classes and describes the patterns of inflection and derivation in the various dialects of the language.

\section{Parts of speech}
The parts of speech\index[sub]{parts of speech} in Southern Yauyos Quechua, as in other Quechuan languages, are substantives (\phono{warmi} ‘woman’), verbs (\phono{hamu-} ‘come’), ambivalents (\phono{para} ‘rain, to rain’), and particles (\phono{mana} ‘no, not’). Substantives and verbs are subject to different patterns of inflection; ambivalents may inflect either as substantives or verbs; particles do not inflect.

The class of substantives in Quechuan languages is usually defined as including nouns (\phono{wasi} ‘house’); pronouns (\phono{ñuqanchik} ‘we’); interrogative-indefinites (\phono{may} ‘where’); adjectives (\phono{sumaq} ‘pretty’); pre-adjectives (\phono{dimas} ‘too’); and numerals (\phono{kimsa} ‘three’). All substantives with the exception of dependent pronouns (\phono{Sapa} ‘alone’) may occur as free forms.

The class of verbs in Quechuan languages is usually defined to include transitive (\phono{qawa}- ‘see’), intransitive (\phono{tushu-} ‘dance’), and copulative (\phono{ka-} ‘be’) stems. A fourth class can be set apart: onomatopoetic verbs (\phono{chuqchuqya-} ‘nurse, make the sound of a calf nursing’). All verbs, with the exception of \phono{haku!} ‘let’s go!’, occur only as bound forms.

Ambivalents form a single class.

The class of particles is usually defined to include interjections (\phono{¡Alaláw!} ‘How cold!’); prepositions (\phono{asta} ‘until’); coordinators (\phono{icha} ‘or’); pre-numerals (\phono{la}, \phono{las}, occurring with expressions of time); negators (\phono{mana} ‘no, not’); assenters and greetings (\phono{aw} ‘yes’); adverbs (\phono{ayvis} ‘sometimes’).

The remainder of this section covers substantives; verbs are covered in Chapter~\ref{ch:verbs} and particles in Chapter~\ref{ch:particles}.

\section{Substantive classes}
In \SYQ, as in other Quechuan languages, the class of substantives\index[sub]{substantive!classes} comprises six subclasses: nouns, pronouns, interrogative-indefinites, adjectives, pre-adjectives, and numerals. \sectref{sec:nouns}--\ref{sec:numerals} cover each of these in turn. Multiple-class substantives and the dummy noun \phono{na} are covered in \sectref{sec:mcsub} and~\ref{sec:dummyna}, respectively.

\subsection{Nouns}\label{sec:nouns}
The class of nouns\index[sub]{nouns} may be divided into four sub-classes: regular nouns (\phono{wayta} ‘flower’), time nouns (\phono{kanan} ‘now’), gender nouns (\phono{tiya} ‘aunt’), and locative nouns (\phono{qipa} ‘behind’). \sectref{ssec:regnouns}--\ref{ssec:locnouns} cover each of these in turn.

\subsubsection{Regular nouns}\label{ssec:regnouns}
The class of regular nouns\index[sub]{nouns!regular} includes all nouns not included in the other three classes. Although in this sense it is defined negatively, as a kind of default class, it includes by far more members than any of the others. ~(\ref{Glo3:Warmi}--\ref{Glo3:Ukucha}) give examples.\\

% 1
\gloexe{Glo3:Warmi}{}{amv}%
{\pb{Warmi}npis qatiparun \pb{urqu}ta.}%amv que first line
{\morglo{warmi-n-pis}{woman-\lsc{3}-\lsc{acc}}\morglo{qati-pa-ru-n}{follow-\lsc{repet}-\lsc{urgt}-3}\morglo{urqu-ta}{hill-\lsc{acc}}}%morpheme+gloss
\glotran{His \pb{wife} herded him back to the \pb{hills}.}{}%eng+spa trans
{}{}%rec - time

% 2
\gloexe{Glo3:Qari}{}{amv}%
{\pb{Qari}ntash wañurachin, \pb{masha}ntash wañurachin.}%
{\morglo{qari-n-ta-sh}{man-3-\lsc{acc}-\lsc{evr}}\morglo{wañu-ra-chi-n}{die-\lsc{urgt}-\lsc{caus}-3}\morglo{masha-n-ta-sh}{son.in.law-3-\lsc{acc}-\lsc{evr}}\morglo{wañu-ra-chi-n}{die-\lsc{urgt}-\lsc{caus}-3}}%morpheme+gloss
\glotran{She killed her \pb{husband}, they say; she killed her \pb{son-in-law}, they say.}{}%eng+spa trans
{}{}%rec - time

% 3
\gloexe{Glo3:Lata}{}{ach}%
{\pb{Lata}wan yanushpataqshi \pb{runa}tapis mikurura.}%ach que first line
{\morglo{lata-wan}{tin.pot-\lsc{instr}}\morglo{yanu-shpa-taq-shi}{cook-\lsc{subis}-\lsc{seq}-\lsc{evr}}\morglo{runa-ta-pis}{person-\lsc{acc}-\lsc{add}}\morglo{miku-ru-ra}{eat-\lsc{urgt}-\lsc{pst}}}%morpheme+gloss
\glotran{They even cooked \pb{people} in metal \pb{pots}, they say, and ate them.}{}%eng+spa trans
{}{}%rec - time

% 4
\gloexe{Glo3:Unaykunaqa}{}{amv}%
{Unaykunaqa \pb{watu}ta ruwaq kayanchik \pb{llama}paqpis \pb{alpaka}paqpis.}%amv que first line
{\morglo{unay-kuna-qa}{before-\lsc{}pl-\lsc{}top}\morglo{watu-ta}{rope-\lsc{acc}}\morglo{ruwa-q}{make-\lsc{ag}}\morglo{ka-ya-nchik}{be-\lsc{prg}-\lsc{1pl}}\morglo{llama-paq-pis}{llama-\lsc{abl}-\lsc{add}}\morglo{alpaka-paq-pis}{alpaca-\lsc{abl}-\lsc{add}}}%morpheme+gloss
\glotran{In the old days, we used to make \pb{rope} from [the wool of] \pb{llamas} and \pb{alpacas}.}{}%eng+spa trans
{}{}%rec - time

% 5
\gloexe{Glo3:Ukucha}{}{ach}%
{\pb{Ukucha}pa \pb{trupa}llanta \pb{paluma}qa quykun.}%ach que first line
{\morglo{ukucha-pa}{mouse-\lsc{gen}}\morglo{trupa-lla-n-ta}{tail-\lsc{rstr}-\lsc{3}-\lsc{acc}}\morglo{paluma-qa}{dove-\lsc{top}}\morglo{qu-yku-n}{give-\lsc{excep}-\lsc{3}}}%morpheme+gloss
\glotran{The \pb{dove} gave them the \pb{tail} of a \pb{mouse}.}{}%eng+spa trans
{}{}%rec - time

\subsubsection{Time nouns}\label{ssec:timenouns}
Nouns referring to time\index[sub]{nouns!time} (\phono{kanan} ‘now’, \phono{wata} ‘year’) form a unique class in that they may occur adverbally without inflection, as in~(\ref{Glo3:Tukuy}--\ref{Glo3:Qaynahuk}).\\

% 1
\gloexe{Glo3:Tukuy}{}{amv}%
{Tukuy \pb{puntraw} yatramunanchikpaq.}%amv que first line
{\morglo{tukuy}{all}\morglo{puntraw}{day}\morglo{yatra-mu-na-nchik-paq}{know-\lsc{cisl}-\lsc{nmlz}-\lsc{1pl}-\lsc{purp}}}%morpheme+gloss
\glotran{So we can learn all \pb{day}.}{}%eng+spa trans
{}{}%rec - time

% 2
\gloexe{Glo3:Kanan}{}{amv}%
{\pb{Kanan} vakata pusillaman chawayanchik kabratahina.}%
{\morglo{kanan}{now}\morglo{vaka-ta}{cow-\lsc{acc}}\morglo{pusilla-man}{cup-\lsc{all}}\morglo{chawa-ya-nchik}{milk-\lsc{prog}-\lsc{1pl}}\morglo{kabra-ta-hina}{goat-\lsc{acc}-\lsc{comp}}}%morpheme+gloss
\glotran{\pb{These days} we milk a cow into just a cup, like a goat.}{}%eng+spa trans
{}{}%rec - time

% 3
\gloexe{Glo3:Pishi}{}{amv}%
{Pishiparullaniñam. Kutimunki \pb{paqarin}.}%amv que first line
{\morglo{pishipa-ru-lla-ni-ña-m}{tire-\lsc{urgt}-\lsc{rstr}-\lsc{1}-\lsc{disc}-\lsc{evd}}\morglo{kuti-mu-nki}{return-\lsc{cisl}-\lsc{2}}\morglo{paqarin}{tomorrow}}%morpheme+gloss
\glotran{I’m tired already. You’ll come back \pb{tomorrow}.}{}%eng+spa trans
{}{}%rec - time

% 4
\gloexe{Glo3:Rinrilla}{}{ach}%
{Rinrilla:pis uparura \pb{qayna wata}qa.}%ach que first line
{\morglo{rinri-lla-:-pis}{ear-\lsc{rstr}-\lsc{1}-\lsc{add}}\morglo{upa-ru-ra}{deaf-\lsc{urgt}-\lsc{pst}}\morglo{qayna}{previous}\morglo{wata-qa}{year-\lsc{top}}}%morpheme+gloss
\glotran{My ears went deaf \pb{last year}.}{}%eng+spa trans
{}{}%rec - time

% 5
\gloexe{Glo3:Qaynahuk}{}{amv}%
{\pb{Qayna huk wata}hina timblur yapa kaypa kaptinqa.}%amv que first line
{\morglo{qayna}{previous}\morglo{huk}{one}\morglo{wata-hina}{year-\lsc{comp}}\morglo{timblur}{earthquake}\morglo{yapa}{again}\morglo{kay-pa}{\lsc{dem.p}-\lsc{loc}}\morglo{ka-pti-n-qa}{be-\lsc{subds}-\lsc{3}-\lsc{top}}}%morpheme+gloss
\glotran{About \pb{a year ago}, when there was an earthquake here again.}{}%eng+spa trans
{}{}%rec - time

\subsubsection{Gender nouns}
Nouns indigenous to \SYQ{} do not inflect for gender\index[sub]{nouns!gender}. \SYQ{} indicates biological gender either with distinct noun roots (\phono{maqta} ‘young man’, \phono{pashña} ‘young woman’) or by modification with \phono{qari} ‘man’ or \phono{warmi} ‘woman’ in the case of people (\phono{qari wawa} ‘boy child’, \phono{warmi wawa} ‘girl child’) or \phono{urqu} ‘male’ or \phono{trina} ‘female’ in the case of animals. A few nouns, all borrowed from Spanish, are inflected for gender (masculine \pb{\textipa{/u/}} and feminine \textipa{/a/}).~(\ref{Glo3:Kayllata}--\ref{Glo3:Unayunay}) give examples.\\
\largerpage

% 1
\gloexe{Glo3:Kayllata}{}{amv}%
{¿Kayllata nisitanki, aw, \pb{tiyu}, llama wirata?}%
{\morglo{kay-lla-ta}{\lsc{dem.p}-\lsc{rstr}-\lsc{acc}}\morglo{nisita-nki}{need-2	}\morglo{aw}{yes}\morglo{tiyu}{uncle}\morglo{llama}{llama}\morglo{wira-ta}{fat-\lsc{acc}}}%morpheme+gloss
\glotran{You need only this, \pb{uncle}, llama fat?}%eng
{‘¿Vas a necesitar nada más esto, tío? ¿Sebo de llama?’}%eng+spa trans
{}{}%rec - time

% 2
\gloexe{Glo3:Chaytri}{}{amv}%
{Chaytri \pb{Tiya} Alejandraqa Shutcollapa yatrarqa.}%amv que first line
{\morglo{chay-tri}{\lsc{dem.d}-\lsc{evc}}\morglo{Tiya}{Aunt}\morglo{Alejandra-qa}{Alejandra-\lsc{top}}\morglo{Shutco-lla-pa}{Shutco-\lsc{rstr}-\lsc{loc}}\morglo{yatra-rqa}{reside-\lsc{pst}}}%morpheme+gloss
\glotran{That must be why \pb{Aunt} Alexandra lived just in Shutco.}{}%eng+spa trans
{}{}%rec - time

% 3
\gloexe{Glo3:Wak}{}{lt}%
{Wak karu purikushayta \pb{ansyana}ña kashayta.}%lt que first line
{\morglo{wak}{\lsc{dem.d}}\morglo{karu}{far}\morglo{puri-ku-sha-y-ta}{walk-\lsc{refl}-\lsc{prf}-\lsc{1}-\lsc{acc}}\morglo{ansyana-ña}{old.lady-\lsc{disc}}\morglo{ka-sha-y-ta}{be-\lsc{prf}-\lsc{1}-\lsc{acc}}}%morpheme+gloss
\glotran{There where I’ve walked far, an \pb{old lady} already.}{}%eng+spa trans
{}{}%rec - time

% 4
\gloexe{Glo3:Unayunay}{}{amv}%
{Unay unay blusataraqchu hinam ushturayachinpis \pb{awilita}qa. ¡Ve!}%amv que first line
{\morglo{unay}{before}\morglo{unay}{before}\morglo{blusa-ta-raq-chu}{blouse-\lsc{acc}-\lsc{cont}-\lsc{q}}\morglo{hina-m}{thus-\lsc{evd}}\morglo{ushtu-ra-ya-chi-n-pis}{dress-\lsc{unint}-\lsc{intens}-\lsc{caus}-\lsc{3}-\lsc{add}}\morglo{awilita-qa}{grandmother-\lsc{top}}\morglo{ve}{look}}%morpheme+gloss
\glotran{The \pb{old lady} is dressed in a blouse like the olden ones. Look!}{}%eng+spa trans
{}{}%rec - time
\newpage 

\subsubsection{Locative nouns}\label{ssec:locnouns}
Locative nouns\index[sub]{nouns!locative} indicate relative position (\phono{chimpa} ‘front’, \phono{hawa} ‘top’). They are inflected with the suffixes of the substantive (possessive) paradigm which indicate the person --~and, in the case of the first person, also the number~-- of the complement noun.~(\ref{Glo3:Hinas}--\ref{Glo3:Uma}) give examples.\\

% 1
\gloexe{Glo3:Hinas}{}{ach}%
{Hinashpaqa hatariru:. Allqukuna \pb{yata}npa kara.}%ach que first line
{\morglo{hinashpa-qa}{then-\lsc{top}}\morglo{hatari-ru-:}{get.up-\lsc{urgt}-\lsc{1}}\morglo{allqu-kuna}{dog-\lsc{pl}}\morglo{yata-n-pa}{side-\lsc{3}-\lsc{loc}}\morglo{ka-ra}{be-\lsc{pst}}}%morpheme+gloss
\glotran{Then I got up. Dogs were \pb{at his side}.}{}%eng+spa trans
{}{}%rec - time

% 2
\gloexe{Glo3:Kalaminahawa}{}{amv}%
{Kalamina \pb{hawa}nta pasarachisa \pb{uku}nman saqakuykusa.}%amv que first line
{\morglo{kalamina}{corrugated.iron}\morglo{hawa-n-ta}{above-\lsc{3}-\lsc{acc}}\morglo{pasa-ra-chi-sa}{pass-\lsc{urgt}-\lsc{caus}-\lsc{npst}}\morglo{uku-n-man}{inside-\lsc{3}-\lsc{all}}\morglo{saqa-ku-yku-sa}{go.down-\lsc{refl}-\lsc{excep}-\lsc{npst}}}%morpheme+gloss
\glotran{He made him go \pb{on top of} the tin roof and he fell \pb{inside}.}{}%eng+spa trans
{}{}%rec - time

% 3
\gloexe{Glo3:Planta}{}{amv}%
{Plantachaqa alfapa \pb{trawpi}npa wiñan.}%amv que first line
{\morglo{planta-cha-qa}{tree-\lsc{dim}-\lsc{top}}\morglo{alfa-pa}{alfalfa-\lsc{loc}}\morglo{trawpi-n-pa}{middle-\lsc{3}-\lsc{loc}}\morglo{wiña-n}{grow-\lsc{3}}}%morpheme+gloss
\glotran{The little plant grows in the \pb{middle} of alfalfa [fields].}{}%eng+spa trans
{}{}%rec - time

% 4
\gloexe{Glo3:Kalabira}{}{ach}%
{Kalabira, tullu, wama-wamaq chay \pb{uku}paq kakuyan.}%ach que first line
{\morglo{kalabira,}{skeleton}\morglo{tullu,}{bone}\morglo{wama-wamaq}{a.lot-a.lot}\morglo{chay}{\lsc{dem.d}}\morglo{uku-paq}{inside-\lsc{loc}}\morglo{ka-ku-ya-n}{be-\lsc{refl}-\lsc{prog}-\lsc{3}}}%morpheme+gloss
\glotran{Skeletons, bones -- there are a lot there \pb{inside}.}{}%eng+spa trans
{}{}%rec - time

% 5
\gloexe{Glo3:Uma}{}{amv}%
{Uma nanaypaq~\dots{} trurarunchik huk limuntam \pb{trawpi}paq partirunchik.}%amv que first line
{\morglo{uma}{head}\morglo{nana-y-paq}{hurt-\lsc{inf}-\lsc{purp}}\morglo{trura-ru-nchik}{put-\lsc{urgt}-\lsc{1pl}}\morglo{huk}{one}\morglo{limun-ta-m}{lime-\lsc{acc}-\lsc{evd}}\morglo{trawpi-paq}{middle-\lsc{loc}}\morglo{parti-ru-nchik}{split-\lsc{urgt}-\lsc{1pl}}}%morpheme+gloss
\glotran{For headaches~\dots{} we put a lime -- we cut it in the \pb{center}.}{}%eng+spa trans
{}{}%rec - time

\subsection{Pronouns}
In \SYQ, as in other Quechuan languages, pronouns\index[sub]{pronouns} may be sorted into four classes: personal pronouns, demonstrative pronouns, dependent pronouns and interroga\-tive-indefinite pronouns.

The personal pronouns\index[sub]{pronouns!personal} in \SYQ{} are \phono{ñuqa} ‘I’; \phono{qam} ‘you’; \phono{pay} ‘she/he’; \phono{ñuqa-nchik} ‘we’; \phono{qam-kuna} ‘you.\lsc{pl}’; and \phono{pay-kuna} ‘they’. \SYQ{} makes no distinction between subject, object, and possessive pronouns. With all three, case marking attaches to the same stem: \phono{ñuqa} (1) ‘I’; \phono{ñuqa-ta} (1-\lsc{acc}) ‘me’; \phono{ñuqa-pa} (1-\lsc{gen}) ‘my’ (nominative being zero-marked). Table~\ref{Tab8} summarizes this information.

The demonstrative pronouns\index[sub]{pronouns!demonstrative} are \phono{kay} ‘this’, \phono{chay} ‘that’, and \phono{wak} ‘that (other)’.

The dependent pronouns\index[sub]{pronouns!dependent} are \phono{kiki} ‘oneself’, \phono{Sapa} ‘only, alone’, \phono{llapa} ‘all’, and \phono{kuska} ‘together’. These occur only with substantive person inflection, which indicates the person and, in the case of the first person plural, number of the referent of the pronoun (\phono{kiki-y}/\phono{-:} ‘I myself’; \phono{sapa-yki} ‘you alone’). One additional pronoun may appear suffixed with substantive person inflection: \phono{wakin} ‘some~\dots’, ‘the rest of~\dots’

\sectref{ssec:ppnqp}--\ref{ssec:deppro} cover the personal pronouns, demonstrative pronouns, and dependent pronouns. Interrogative-indefinite pronouns are covered in \sectref{sec:IntInd}.

%%%%% Issue: 4 or 9?
\subsubsection{Personal pronouns \phono{ñuqa, qam, pay}}\label{ssec:ppnqp}\index[sub]{pronouns!personal}\index[sub]{fourth person}
\SYQ{} has three pronominal stems --~\phono{ñuqa}, \phono{qam}, and \phono{pay}, as in~(\ref{Glo3:Kala}),~(\ref{Glo3:Manamnuqa}) and~(\ref{Glo3:Payqa}). These correspond to the first, second and third persons. Table~\ref{Tab8} lists the personal pronouns.

% TABLE 8
\begin{table}
\small\centering
\caption{Personal pronouns}\label{Tab8}
\begin{tabular}{lll}
\lsptoprule
Person & Singular & Plural\\
\midrule
\multirow{3}{*}{1} 	& ñuqa 	& ñuqa-nchik (dual)		\\
	& 			& ñuqa-nchik-kuna (inclusive)	\\
	& 			& ñuqa-kuna (exclusive)	\\
%\midrule
2 	& qam 		& qam-kuna			\\
%\midrule
3 	& pay 		& pay-kuna\\
\lspbottomrule
\end{tabular}
\end{table}

% 1
\gloexe{Glo3:Kala}{}{ch}%
{Kala: Cañetepi chaypim uyarila: \pb{ñuqa}pis.}%ch que first line
{\morglo{ka-la-:}{be-\lsc{pst}-\lsc{1}}\morglo{Cañete-pi}{Cañete-\lsc{loc}}\morglo{chay-pi-m}{\lsc{dem.d}-\lsc{loc}-\lsc{evd}}\morglo{uyari-la-:}{hear-\lsc{pst}-\lsc{1}}\morglo{ñuqa-pis}{I-\lsc{add}}}%morpheme+gloss
\glotran{I was in Cañete. \pb{I}, too, heard it there.}{}%eng+spa trans
{}{}%rec - time

% 2
\gloexe{Glo3:Manamnuqa}{}{ch}%
{Manam \pb{ñuqa}qa Viñaqta riqsi:chu. ¿\pb{Qam} riqsinkichu, Min?}%ch que first line
{\morglo{mana-m}{no-\lsc{evd}}\morglo{ñuqa-qa}{I-\lsc{top}}\morglo{Viñaq-ta}{Viñac-\lsc{acc}}\morglo{riqsi-:-chu}{be.acquainted.with-\lsc{1}-\lsc{neg}}\morglo{qam}{you}\morglo{riqsi-nki-chu}{be.acquainted.with-\lsc{2}-\lsc{q}}\morglo{Min}{Min}}%morpheme+gloss
\glotran{\pb{I} don’t know Viñac. Do \pb{you} know it, Min?}{}%eng+spa trans
{}{}%rec - time

% 3
\gloexe{Glo3:Payqa}{}{lt}%
{\pb{Pay}qa hatarirushañam rikaq.}%lt que first line
{\morglo{pay-qa}{\lsc{3}-\lsc{top}}\morglo{hatari-ru-sha-ña-m}{get.up-\lsc{urgt}-\lsc{npst}-\lsc{disc}-\lsc{evd}}\morglo{rika-q}{see-\lsc{ag}}}%morpheme+gloss
\glotran{\pb{He} had already gotten up to see.}{}%eng+spa trans
{}{}%rec - time

\noindent
These may but need not inflect for number as \phono{ñuqa-kuna}\index[sub]{nuqakuna@\phono{ñuqakuna}}, \phono{qam-kuna}, and \phononb{pay}\phononb{-kuna}~(\ref{Glo3:Unay}), (\ref{Glo3:Qamkuna}) and~(\ref{Glo3:Manachupaykuna}).\\

% 9 or 4?
% 9
\gloexe{Glo3:Unay}{}{amv}%
{Unay \pb{ñuqakuna}qa manam qawarqanichu, paykunaqa alminus manam qawarqapischu.}%amv que first line
{\morglo{unay}{before}\morglo{ñuqa-kuna-qa}{I-\lsc{pl}-\lsc{top}}\morglo{mana-m}{no-\lsc{}evd}\morglo{qawa-rqa-ni-chu,}{see-\lsc{pst}-\lsc{1}-\lsc{neg}}\morglo{pay-kuna-qa}{\lsc{3}\lsc{pl}-\lsc{top}}\morglo{alminus}{at.least}\morglo{mana-m}{no-\lsc{evd}}\morglo{qawa-rqa-pis-chu}{see-\lsc{pst}-\lsc{add}-\lsc{neg}}}%morpheme+gloss
\glotran{Before, \pb{we} didn’t see, but they, at least, didn’t see either.}{}%eng+spa trans
{}{}%rec - time

% 5
\gloexe{Glo3:Qamkuna}{}{amv}%
{“\pb{Qamkuna} ashiptikim chinkakun”, ni:.}%
{\morglo{qam-kuna}{you-\lsc{pl}}\morglo{ashi-pti-ki-m}{look.for-\lsc{subds}-\lsc{2}-\lsc{evd}}\morglo{chinka-ku-n}{lose-\lsc{refl}-\lsc{3}}\morglo{ni-:}{say-\lsc{1}}}%morpheme+gloss
\glotran{“When \pb{you} looked for him, he got lost,” I said.}%eng
{‘“Cuando ustedes lo buscaron, él se perdió”, dije’.}%spa
{}{}

% 6
\gloexe{Glo3:Manachupaykuna}{}{amv}%
{¿Manachu \pb{paykuna} wakpa wasinpi mikun uqata?}%amv que first line
{\morglo{mana-chu}{no-\lsc{q}}\morglo{pay-kuna}{he-\lsc{pl}}\morglo{wak-pa}{\lsc{dem.d}-\lsc{loc}}\morglo{wasi-n-pi}{house-\lsc{3}-\lsc{loc}}\morglo{miku-n}{eat-\lsc{3}}\morglo{uqa-ta}{oca-\lsc{acc}}}%morpheme+gloss
\glotran{There in her house, don’t \pb{they} eat oca?}{}%eng+spa trans
{}{}%rec - time

\noindent
\SYQ{} makes available a three-way distinction in the first person plural among \phono{ñuqa-nchik} (dual), \phono{ñuqa-nchik-kuna}\index[sub]{nuqanchikkuna@\phono{ñuqanchikkuna}} (inclusive), and \phono{ñuqa-kuna} (exclusive)~(\ref{Glo3:Ishkaykashpalla}), (\ref{Glo3:Kaypi}), (\ref{Glo3:Unay}).\\

% 7
\gloexe{Glo3:Ishkaykashpalla}{}{amv}%
{Ishkay kashpallam, “\pb{ñuqanchik}” nin.}%amv que first line
{\morglo{ishkay}{two}\morglo{ka-shpa-lla-m}{be-\lsc{subis}-\lsc{rstr}-\lsc{evd}}\morglo{ñuqa-nchik}{I-\lsc{1pl}}\morglo{ni-n}{say-\lsc{3}}}%morpheme+gloss
\glotran{If there are only two people, they say \phono{ñuqanchik}.}{}%eng+spa trans
{}{}%rec - time

% 8
\gloexe{Glo3:Kaypi}{}{amv}%
{Kaypi \pb{ñuqanchikkuna}qa kustumbrawmi kanchik.}%amv que first line
{\morglo{kay-pi}{\lsc{dem.p}-\lsc{loc}}\morglo{ñuqa-nchik-kuna-qa}{we-\lsc{1pl}-\lsc{pl}-\lsc{top}}\morglo{kustumbraw-mi}{accustomed-\lsc{evd}}\morglo{ka-nchik}{be-\lsc{1pl}}}%morpheme+gloss
\glotran{Around here, \pb{we}’re used to it.}{}%eng+spa trans
{}{}%rec - time

%%%% 9 again or 4?
\noindent
\phono{ñuqa-kuna} is employed in all five dialects~(\ref{Glo3:Manamnuqkuna}--\ref{Glo3:Linchapi}).\\

% 10
\gloexe{Glo3:Manamnuqkuna}{}{ch}%
{Manam \pb{ñuqakuna}qa talpula:chu paypa wawinmi talpula.}%ch que first line
{\morglo{mana-m}{no-\lsc{evd}}\morglo{ñuqa-kuna-qa}{\lsc{1}-\lsc{pl}-\lsc{top}}\morglo{talpu-la-:-chu}{plant-\lsc{pst}-\lsc{1}-\lsc{neg}}\morglo{pay-pa}{he-\lsc{3}}\morglo{wawi-n-mi}{baby-\lsc{3}-\lsc{evd}}\morglo{talpu-la}{plant-\lsc{pst}}}%morpheme+gloss
\glotran{\pb{We} haven’t planted. Her children have planted.}{}%eng+spa trans
{}{}%rec - time

% 11
\gloexe{Glo3:Chaynakunam}{}{sp}%
{Chaynakunam \pb{ñuqakuna} kwintu: kara.}%sp que first line
{\morglo{chayna-kuna-m}{thus-\lsc{pl}-\lsc{evd}}\morglo{ñuqa-kuna}{I-\lsc{pl}}\morglo{kwintu-:}{story-\lsc{1}}\morglo{ka-ra}{be-\lsc{pst}}}%morpheme+gloss
\glotran{That’s how \pb{our} stories were.}{}%eng+spa trans
{}{}%rec - time

% 12
\gloexe{Glo3:Linchapi}{}{lt}%
{Linchapi \pb{ñuqakuna}pa kanchu.}%lt que first line
{\morglo{Lincha-pi}{Lincha-\lsc{loc}}\morglo{ñuqa-kuna-pa}{\lsc{1}-\lsc{pl}-\lsc{gen}}\morglo{ka-n-chu}{be-\lsc{3}-\lsc{neg}}}%morpheme+gloss
\glotran{\pb{We} don’t have any in Lincha.}{}%eng+spa trans
{}{}%rec - time

\noindent
In practice, except in \CH, \phono{ñuqa-nchik} is employed with dual, inclusive and exclusive interpretations to the virtual complete exclusion of the other two forms. Verbs and substantives appearing with the inclusive \phono{ñuqa-nchik-kuna} inflect in the same manner as verbs do and substantives appearing with the dual/default \phono{ñuqa-nchik}~(\ref{Glo3:Kriyi}); verbs and substantives appearing with the exclusive \phono{ñuqa-kuna} inflect in the manner as those appearing with the singular \phono{ñuqa}~(\ref{Glo3:Familyallan}),~(\ref{Glo3:Puntra}).\\

% 13
\gloexe{Glo3:Kriyi}{}{amv}%
{Kriyi\pb{nchik} \pb{ñuqanchikkuna}.}%amv que first line
{\morglo{kriyi-nchik}{believe-\lsc{1pl}}\morglo{ñuqa-nchik-kuna}{I-\lsc{1pl}-\lsc{pl}}}%morpheme+gloss
\glotran{\pb{We} believe.}{}%eng+spa trans
{}{}%rec - time

% 14
\gloexe{Glo3:Familyallan}{}{ch}%
{Familyallan \pb{ñuqakuna} suya:}%ch que first line
{\morglo{familya-lla-n}{family-\lsc{rstr}-\lsc{3}}\morglo{ñuqa-kuna}{I-\lsc{pl}}\morglo{suya-:}{wait-\lsc{1}}}%morpheme+gloss
\glotran{Only \pb{we}, their relatives, wait.}{}%eng+spa trans
{}{}%rec - time

% 15
\gloexe{Glo3:Puntra}{}{amv}%
{Puntrawyayanñam \pb{ñuqakuna}qa lluqsi\pb{ni}ñam.}%amv que first line
{\morglo{puntraw-ya-ya-n-ña-m}{day-\lsc{inch}-\lsc{prog}-\lsc{3}-\lsc{disc}-\lsc{evd}}\morglo{ñuqa-kuna-qa}{I-\lsc{pl}-\lsc{top}}\morglo{lluqsi-ni-ña-m}{go.out-\lsc{1}-\lsc{disc}-\lsc{evd}}}%morpheme+gloss
\glotran{It’s getting to be daytime -- \pb{we} leave already.}{}%eng+spa trans
{}{}%rec - time

\noindent
In the verbal and nominal paradigm tables, for reasons of space, I generally do not list \phono{ñuqa-nchik-kuna} and \phono{ñuqa-kuna} with the other first person pronouns in the headings; it can be assumed that the first patterns with \phononb{ñuqa}\phononb{-nchik}, the second with \phono{ñuqa}. In practice, where context does not adequately specify the referent, speakers of \SYQ{} make distinctions between the dual, inclusive and exclusive first-person plural exactly like speakers of English and Spanish do, indicating the dual, for example, with \phono{ishkay-ni-nchik} ‘the two of us’; the inclusive with \phono{llapa-nchik} ‘all of us’; and the exclusive with modifying phrases, as in \phono{ñuqa-nchik Viñac-pa} ‘we in Viñac’. \SYQ{} makes no distinction between subject, object~(\ref{Glo3:Nuqata}) and possessive~(\ref{Glo3:Manamka}) pronouns. With all three, case marking attaches to the same stem; nominative case is zero-marked.\\

% 16
\gloexe{Glo3:Nuqata}{}{amv}%
{\pb{Ñuqata} mikumuwananpaq kutimushpa traqnaruwan.}%amv que first line
{\morglo{ñuqa-ta}{I-\lsc{acc}}\morglo{miku-mu-wa-na-n-paq 			}{eat-\lsc{cisl}-\lsc{1.obj}-\lsc{nmlz}-\lsc{3}-\lsc{purp}}\morglo{kuti-mu-shpa}{return-\lsc{cisl}-\lsc{subis}}\morglo{traqna-ru-wa-n}{bind.limbs-\lsc{urgt}-\lsc{1.obj}-\lsc{3}}}%morpheme+gloss
\glotran{In order to me able to eat \pb{me} when he got back, he tied me up.}{}%eng+spa trans
{}{}%rec - time

% 17
\gloexe{Glo3:Manamka}{}{amv}%
{Manam kanchu. \pb{Ñuqapaq} puchukarun.}%amv que first line
{\morglo{mana-m}{no-\lsc{evd}}\morglo{ka-n-chu}{be-\lsc{3}-\lsc{neg}}\morglo{ñuqa-paq}{I-\lsc{gen}}\morglo{puchuka-ru-n}{finish-\lsc{urgt}-\lsc{3}}}%morpheme+gloss
\glotran{There aren’t any. \pb{Mine} finished off.}{}%eng+spa trans
{}{}%rec - time

% 4
\gloexe{Glo3:Hukqawaptinqa}{}{amv}%
{Huk qawaptinqa, \pb{ñuqanchik} qawanchikchu. Almanchik puriyanshi.}%
{\morglo{huk}{one}\morglo{qawa-pti-n-qa}{see-\lsc{subds}-\lsc{3}-\lsc{top}}\morglo{ñuqa-nchik}{I-\lsc{1pl}}\morglo{qawa-nchik-chu}{see-\lsc{1pl}-\lsc{neg}}\morglo{alma-nchik}{soul-1\lsc{pl}}\morglo{puri-ya-n-shi}{walk-\lsc{prog}-\lsc{3}-\lsc{evr}}}%morpheme+gloss
\glotran{“Although others see them, \pb{we} don’t see them. Our souls wander around,” they say.}%eng
{‘Nuestras almas andan, dicen. Aunque otros las vean, nosotros no las vemos’.}%spa
{}{}%

%%%%% END issue

\subsubsection{Demonstrative pronouns \phono{kay}, \phono{chay}, \phono{wak}}
\SYQ{} has three demonstrative pronouns\index[sub]{pronouns!demonstrative}: \phono{kay} ‘this’, \phono{chay} ‘that’, and \phono{wak} ‘that (other)’~(\ref{Glo3:Kayqa}--\ref{Glo3:Wakmulaqa}).\\

% 1
\gloexe{Glo3:Kayqa}{}{amv}%
{“\pb{Kay}qa manam balinchu mikunanchikpaq”, [nishpa] allquman qaraykurqani.}%
{\morglo{kay-qa}{\lsc{dem.p}-\lsc{top}}\morglo{mana-m}{no-\lsc{evd}}\morglo{bali-n-chu}{be.worth-\lsc{3}-\lsc{neg}}\morglo{miku-na-nchik-paq}{eat-\lsc{nmlz}-\lsc{1}\lsc{pl}-\lsc{purp}}\morglo{allqu-man}{dog-\lsc{all}}\morglo{qara-yku-rqa-ni}{serve-\lsc{excep}-\lsc{pst}-\lsc{1}}}%morpheme+gloss
\glotran{“\pb{This} is not good for us to eat,” I said and I served it to the dog.}{}%eng+spa trans
{}{}%rec - time

% 2
\gloexe{Glo3:Ollanta}{}{ch}%
{Ollanta Humala, “Kanan \pb{chay}kunakta wañuchishaq”, niyan.}%ch que first line
{\morglo{Ollanta}{Ollanta}\morglo{Humala}{Humala}\morglo{kanan}{now}\morglo{chay-kuna-kta}{\lsc{dem.d}-\lsc{pl}-\lsc{acc}}\morglo{wañu-chi-shaq}{die-\lsc{caus}-\lsc{1.fut}}\morglo{ni-ya-n}{say-\lsc{prog}-\lsc{3}}}%morpheme+gloss
\glotran{[President] Ollanta Humala is saying, “Now I’ll kill \pb{those}.”}{}%eng+spa trans
{}{}%rec - time

% 3
\gloexe{Glo3:Wakmulaqa}{}{amv}%
{Wak mulaqa manam mansuchu. Runatam \pb{wak} wañuchin.}%amv que first line
{\morglo{wak}{\lsc{dem.d}}\morglo{mula-qa}{mule-\lsc{top}}\morglo{mana-m}{no-\lsc{evd}}\morglo{mansu-chu}{tame-\lsc{neg}}\morglo{runa-ta-m}{person-\lsc{acc}-\lsc{evd}}\morglo{wak}{\lsc{dem.d}}\morglo{wañu-chi-n}{die-\lsc{caus}-\lsc{3}}}%morpheme+gloss
\glotran{That mule is not tame. \pb{That} kills people.}{}%eng+spa trans
{}{}%rec - time

\noindent
\phono{chay} may have both proximate and distal referents. \phono{wak} is consistently translated in Spanish as ‘\spanish{ese}’ (‘that’), not, perhaps contrary to expectation, as ‘aquel’. The demonstrative pronouns may substitute for any phrase or clause~(\ref{Glo3:Hinashpawawan}). They can but need not inflect for number~(\ref{Glo3:Ollanta}).\\

% 4
\gloexe{Glo3:Hinashpawawan}{}{ach}%
{Hinashpa achkaña wawan kayan. \pb{Chay}paq ñakanñataqtri mikuypaq.}%ach que first line
{\morglo{hinashpa}{then}\morglo{achka-ña}{a.lot-\lsc{disc}}\morglo{wawa-n}{baby-\lsc{3}}\morglo{ka-ya-n}{be-\lsc{prog}-\lsc{3}}\morglo{chay-paq}{\lsc{dem.d}-\lsc{abl}}\morglo{ñaka-n-ña-taq-tri}{suffer-\lsc{3}-\lsc{disc}-\lsc{seq}-\lsc{evc}}\morglo{miku-y-paq}{eat-\lsc{inf}-\lsc{abl}}}%morpheme+gloss
\glotran{Then she has a lot of babies. She’ll suffer, too, a lot from \pb{that}, from hunger.}{}%eng+spa trans
{}{}%rec - time

\noindent
They can appear simultaneously with possessive inflection~(\ref{Glo3:Kayninchik}).\\

% 5
\gloexe{Glo3:Kayninchik}{}{amv}%
{\pb{Kayninchik}.}%amv que first line
{\morglo{kay-ni-nchik}{\lsc{dem.p}-\lsc{euph}-\lsc{1pl}}}%morpheme+gloss
\glotran{\pb{These of ours}.}{}%eng+spa trans
{}{}%rec - time

\noindent
In complex phrases with demonstrative pronouns, case marking attaches to the final word in the phrase~(\ref{Glo3:Kayllan}).\\

% 6
\gloexe{Glo3:Kayllan}{}{amv}%
{\pb{Kay} \pb{llañutapis} puchkani kikiymi.}%amv que first line
{\morglo{kay}{\lsc{dem.p}}\morglo{llañu-ta-pis}{thin-\lsc{acc}-\lsc{add}}\morglo{puchka-ni}{spin-\lsc{1}}\morglo{kiki-y-mi}{self-\lsc{1}-\lsc{evd}}}%morpheme+gloss
\glotran{I spin \pb{this thin one}, too, myself.}{}%eng+spa trans
{}{}%rec - time

\noindent
\phono{chay} may be employed without deictic meaning, in particular when it figures in sentence-initial position~(\ref{Glo3:Chaymi}).\\

% 7
\gloexe{Glo3:Chaymi}{}{ach}%
{\pb{Chaymi} hampichira: hukwan, hukwan.}%ach que first line
{\morglo{chay-mi}{\lsc{dem.d}-\lsc{evd}}\morglo{hampi-chi-ra-:}{heal-\lsc{caus}-\lsc{pst}-\lsc{1}}\morglo{huk-wan,}{one-\lsc{instr}}\morglo{huk-wan}{one-\lsc{instr}}}%morpheme+gloss
\glotran{\pb{So} I had him cured with one and with another.}{}%eng+spa trans
{}{}%rec - time

\noindent
In this case, it is generally suffixed with one of the evidentials \phono{-mi} or \phono{-shi} and indicates that the sentence it heads is closely related to the sentence that precedes it.\footnote{As an anonymous reviewer points out, forms such as \phono{chay-mi} and \phono{chay-shi} are lexicalized discourse markers, and, as such “they do not take productive affixes such as \phono{-kuna}, \phono{-pi}, or \phono{-man}” among others.} \SYQ{} demonstrative pronouns are identical in form to the demonstrative determiners~(\ref{Glo3:Kaymillwa}--\ref{Glo3:Waktrakrayqa}).\\

% 8
\gloexe{Glo3:Kaymillwa}{}{ach}%
{\pb{Kay} millwapaqmi imapis lluqsimun.}%ach que first line
{\morglo{kay}{\lsc{dem.p}}\morglo{millwa-paq-mi}{wool-\lsc{abl}-\lsc{evd}}\morglo{ima-pis}{what-\lsc{add}}\morglo{lluqsi-mu-n}{come.out-\lsc{cisl}-\lsc{3}}}%morpheme+gloss
\glotran{Anything comes out of \pb{this} wool.}{}%eng+spa trans
{}{}%rec - time

% 9
\gloexe{Glo3:Manachuchay}{}{amv}%
{¿Manachu \pb{chay} qatra wambrayki rikarinraq?}%amv que first line
{\morglo{mana-chu}{no-\lsc{q}}\morglo{chay}{\lsc{dem.d}}\morglo{qatra}{dirty}\morglo{wambra-yki}{child-\lsc{2}}\morglo{rikari-n-raq}{appear-\lsc{3}-\lsc{cont}}}%morpheme+gloss
\glotran{Didn’t \pb{that} dirty kid of yours appear yet?}{}%eng+spa trans
{}{}%rec - time

% 10
\gloexe{Glo3:Waktrakrayqa}{}{amv}%
{\pb{Wak} trakrayqa hunta hunta kakuyan.}%amv que first line
{\morglo{wak}{\lsc{dem.d}}\morglo{trakra-y-qa}{field-\lsc{1}-\lsc{top}}\morglo{hunta}{full}\morglo{hunta}{full}\morglo{ka-ku-ya-n}{be-\lsc{refl}-\lsc{prog}-\lsc{3}}}%morpheme+gloss
\glotran{\pb{That} field of mine is really full.}{}%eng+spa trans
{}{}%rec - time
 
\paragraph{Determiners}
\SYQ{} does not have an independent class of determiners\index[sub]{pronouns!determiners}. \phono{huk} ‘one’, ‘once’, ‘other’ can be used to introduce new referents; in this capacity, it can be translated ‘a’~(\ref{Glo3:Hukpash}).\\

% 1
\gloexe{Glo3:Hukpash}{}{amv}%
{\pb{Huk} pashñash karqa ubihira. Chaymanshi trayarushqa \pb{huk} qari yuraq kurbatayuq.}%
{\morglo{huk}{one}\morglo{pashña-sh}{girl-\lsc{evr}}\morglo{ka-rqa}{be-\lsc{pst}}\morglo{ubihira}{shepherdess}\morglo{chay-man-shi}{\lsc{dem.d}-\lsc{all}-\lsc{evr}}\morglo{traya-ru-shqa}{arrive-\lsc{urgt}-\lsc{subis}}\morglo{huk}{one}\morglo{qari}{man}}%morpheme+gloss
\glotran{\pb{A} girl was a shepherdess. Then, they say, \pb{a} man with a white tie arrived.}{}%eng+spa trans
{}{}%rec - time

\noindent
\phono{kay} ‘this’, \phono{chay} ‘that’, and \phono{wak} ‘that (other)’ can be used to refer to established referents; in this capacity, they can be translated ‘the’~(\ref{Glo3:Yuraq}).\\

% 2
\gloexe{Glo3:Yuraq}{}{amv}%
{Yuraq kurbata-yuq yana tirnuyuq \pb{chay} pashñawan purirqa.}%
{\morglo{yuraq}{white}\morglo{kurbata-yuq}{tie-\lsc{poss}}\morglo{yana}{black}\morglo{tirnu-yuq}{suit-\lsc{poss}}\morglo{chay}{\lsc{dem.d}}\morglo{pashña-wan}{girl-\lsc{instr}}\morglo{puri-rqa}{walk-\lsc{pst}}}%morpheme+gloss
\glotran{With a white tie and a black suit, he walked about with the girl.}%
{‘Había una chica pastora. Luego llegó un hombre con corbata blanca. El con corbata blanca y terno negro andaba con la chica’.}%
{}{}%rec - time

% 3
\gloexe{Glo3:Runa}{}{amv}%
{Runa \pb{chay} maqtata wañurachin hanay urqupa.}%amv que first line
{\morglo{runa}{person}\morglo{chay}{\lsc{dem.d}}\morglo{maqta-ta}{young.man-\lsc{acc}}\morglo{wañu-ra-chi-n}{die-\lsc{urgt}-\lsc{caus}-\lsc{3}}\morglo{hanay}{above}\morglo{urqu-pa}{hill-\lsc{loc}}}%morpheme+gloss
\glotran{People killed \pb{the} boy up in the hills.}{}%eng+spa trans
{}{}%rec - time

\subsubsection{Dependent pronouns \phono{kiki-}, \phono{Sapa-}, \phono{llapa-}, \phono{kuska-}}\label{ssec:deppro}
\SYQ{} has four dependent pronouns\index[sub]{pronouns!dependent}: \phono{kiki-} ‘oneself’~(\ref{Glo3:hukkunapaq}), \phono{Sapa-} ‘alone’~(\ref{Glo3:Yatrarqani}), \phono{llapa-} ‘all’~(\ref{Glo3:Llapanta}), and \phono{kuska-} ‘together’~(\ref{Glo3:Mikuypaqpis}).\\

% 1
\gloexe{Glo3:hukkunapaq}{}{amv}%
{\pb{Kiki}ypaq ruwani hukkunapaq ruwani.}%amv que first line
{\morglo{kiki-y-paq}{self-\lsc{1}-\lsc{ben}}\morglo{ruwa-ni}{make-\lsc{1}}\morglo{huk-kuna-paq}{one-\lsc{pl}-\lsc{ben}}\morglo{ruwa-ni}{make-\lsc{1}}}%morpheme+gloss
\glotran{I make them for \pb{myself} and I make them for others.}{}%eng+spa trans
{}{}%rec - time

% 2
\gloexe{Glo3:Yatrarqani}{}{amv}%
{Yatrarqani \pb{sapa}llay.}%amv que first line
{\morglo{yatra-rqa-ni}{reside-\lsc{pst}-\lsc{1}}\morglo{sapa-lla-y}{alone-\lsc{rstr}-\lsc{1}}}%morpheme+gloss
\glotran{I lived all \pb{alone}.}{}%eng+spa trans
{}{}%rec - time

% 3
\gloexe{Glo3:Llapanta}{}{ch}%
{\pb{Llapa}nta apakunki.}%ch que first line
{\morglo{llapa-n-ta}{all-\lsc{3}-\lsc{acc}}\morglo{apa-ku-nki}{bring-\lsc{refl}-\lsc{2}}}%morpheme+gloss
\glotran{You’re going to take along them \pb{all}.}{}%eng+spa trans
{}{}%rec - time

% 4
\gloexe{Glo3:Mikuypaqpis}{}{amv}%
{Mikuypaqpis wañuyanki \pb{kuska}yki wawantin.}%amv que first line
{\morglo{miku-y-paq-pis}{eat-\lsc{inf}-\lsc{abl}-\lsc{add}}\morglo{wañu-ya-nki}{die-\lsc{prog}-\lsc{2}}\morglo{kuska-yki}{together-\lsc{2}}\morglo{wawa-ntin}{baby-\lsc{incl}}}%morpheme+gloss
\glotran{You’re going to be dying of hunger -- you \pb{together} with your children.}{}%eng+spa trans
{}{}%rec - time

\noindent
These pronouns are dependent in the sense that they cannot occur uninflected: the suffixes of the nominal (possessive) paradigm attach to dependent pronouns indicating the person and --~in the case of the first person~-- sometimes the number of the referent of the pronoun (\phono{llapa-nchik} ‘all of us’). Dependent pronouns function in the manner as personal pronouns do: they may refer to any of the participants in an event, subject~(\ref{Glo3:Sikya}) or object~(\ref{Glo3:Chaykuska}); they inflect obligatorily for case~(\ref{Glo3:Hukrunata}) and optionally for number; and they may be followed by enclitics~(\ref{Glo3:Kikinkamatr}).\\

% 5
\gloexe{Glo3:Sikya}{}{amv}%
{Sikya fayna kaptinmi liya: \pb{llapa}:.}%
{\morglo{sikya}{canal}\morglo{fayna}{work.day}\morglo{ka-pti-n-mi}{be-\lsc{subis}-\lsc{3}-\lsc{evd}}\morglo{li-ya-:}{go-\lsc{prog}-\lsc{1}}\morglo{llapa-:}{all-\lsc{1}}}%morpheme+gloss
\glotran{When there’s a community work day on the canal, we \pb{all} go.}%eng
{‘Cuando hay una faena en la acequia, todos vamos’.}%spa
{}{}%rec - time

% 6
\gloexe{Glo3:Chaykuska}{}{ach}%
{Chay \pb{kuska}nta wañurachisa chaypa.}%ach que first line
{\morglo{chay}{\lsc{dem.d}}\morglo{kuska-n-ta}{together-\lsc{3}-\lsc{acc}}\morglo{wañu-ra-chi-sa}{die-\lsc{urgt}-\lsc{caus}-\lsc{npst}}\morglo{chay-pa}{\lsc{dem.d}-\lsc{loc}}}%morpheme+gloss
\glotran{They killed those \pb{together} there.}{}%eng+spa trans
{}{}%rec - time

% 7
\gloexe{Glo3:Hukrunata}{}{amv}%
{Huk runata kaballun \pb{kiki}npi kaballun trakinta pakirusa.}%amv que first line
{\morglo{huk}{one}\morglo{runa-ta}{person-\lsc{acc}}\morglo{kaballu-n}{horse-\lsc{3}}\morglo{kiki-n-pi}{self-\lsc{3}-\lsc{gen}}\morglo{kaballu-n}{horse-\lsc{3}}\morglo{traki-n-ta}{foot-\lsc{3}\lsc{acc}}\morglo{paki-ru-sa}{break-\lsc{urgt}-\lsc{npst}}}%morpheme+gloss
\glotran{A man’s horse -- \pb{his own} horse -- broke his foot.}{}%eng+spa trans
{}{}%rec - time

% 8
\gloexe{Glo3:Kikinkamatr}{}{ach}%
{\pb{Kikinkamatr} wañuchinakura.}%ach que first line
{\morglo{kiki-n-kama-tr}{self-\lsc{3}-\lsc{lim}-\lsc{evc}}\morglo{wañu-chi-naku-ra}{die-\lsc{lim}-\lsc{recp}-\lsc{pst}}}%morpheme+gloss
\glotran{They must have killed each other \pb{themselves}.}{}%eng+spa trans
{}{}%rec - time
 
\noindent
All except \phono{kiki} may occur as free forms as well; it is, however, only as adjectives that they may occur uninflected; as pronouns~(\ref{Glo3:pantyunman}) or adverbs~(\ref{Glo3:Imaynachay}) all still demand inflection.\\


% 9
\gloexe{Glo3:pantyunman}{}{amv}%
{Hinashpa pantyunman apawanchik \pb{llapa} familyanchik kumpañawanchik.}%amv que first line
{\morglo{hinashpa}{then}\morglo{pantyun-man}{cemetery-\lsc{all}}\morglo{apa-wanchik}{bring-\lsc{3>1pl}}\morglo{llapa}{all}\morglo{familya-nchik}{family-\lsc{1pl}}\morglo{kumpaña-wanchik}{accompany-\lsc{3>1pl}}}%morpheme+gloss
\glotran{Then they take us to the cemetery. Our \pb{whole} family accompanies us.}{}%eng+spa trans
{}{}%rec - time

% 10
\gloexe{Glo3:Imaynachay}{}{ch}%
{¿Imayna chay lluqsilushpaqa mana \pb{kuska} lilachu?}%ch que first line
{\morglo{imayna}{why}\morglo{chay}{\lsc{dem.d}}\morglo{lluqsi-lu-shpa-qa}{go.out-\lsc{urgt}-\lsc{subis}-\lsc{top}}\morglo{mana}{no}\morglo{kuska}{together}\morglo{li-la-chu}{go-\lsc{pst}-\lsc{neg}}}%morpheme+gloss
\glotran{Why didn’t they go \pb{together} when they went out?}{}%eng+spa trans
{}{}%rec - time

\noindent
\phono{Sapa} is realized \phono{hapa} in the \CH{} and \LT{} dialects~(\ref{Glo3:Imaynatran}),~(\ref{Glo3:atindinqa}); \phono{sapa} in all others~(\ref{Glo3:Pampawanchik}).\\

% 11
\gloexe{Glo3:Imaynatran}{}{ch}%
{¿Imayna trankilu pulin \pb{hapa}llan?}%ch que first line
{\morglo{imayna}{how}\morglo{trankilu}{tranquil}\morglo{puli-n}{walk-\lsc{3}}\morglo{hapa-lla-n}{alone-\lsc{rstr}-\lsc{3}}}%morpheme+gloss
\glotran{How does she walk about calmly all \pb{alone}?}{}%eng+spa trans
{}{}%rec - time

% 12
\gloexe{Glo3:atindinqa}{}{lt}%
{Pitaq atindinqa \pb{hapa}llay kayaptiyqa.}%lt que first line
{\morglo{pi-taq}{who-\lsc{seq}}\morglo{atindi-nqa}{attend.to-\lsc{3.fut}}\morglo{hapa-lla-y}{alone-\lsc{rstr}-\lsc{1}}\morglo{ka-ya-pti-y-qa}{be-\lsc{prog}-\lsc{subds}-\lsc{1}-\lsc{top}}}%morpheme+gloss
\glotran{Who’s going to take care of him if I’m all \pb{alone}?}{}%eng+spa trans
{}{}%rec - time

% 13
\gloexe{Glo3:Pampawanchik}{}{amv}%
{Pampawanchik tardiqa diharamuwanchik \pb{sapa}llanchikta.}%
{\morglo{pampa-wanchik}{bury-\lsc{3>1pl}}\morglo{tardi-qa}{afternoon-\lsc{top}}\morglo{diha-ra-mu-wanchik}{leave-\lsc{urgt}-\lsc{cisl}-\lsc{3>1pl}}\morglo{sapa-lla-nchik-ta}{alone-\lsc{rstr}-\lsc{1pl}-\lsc{acc}}}%morpheme+gloss
\glotran{They bury us in the afternoon and then they leave us \pb{alone}.}%eng
{‘Nos sepultan en la tarde y después nos dejan solos’.}%spa
{}{}%rec - time

\noindent
One additional pronoun may appear inflected with possessive suffixes: \phono{wakin} ‘some, the rest of’~(\ref{Glo3:Wakintaq}),~(\ref{Glo3:Mamanqa}) (not attested in \CH).\\

% 14
\gloexe{Glo3:Wakintaq}{}{sp}%
{\pb{Wakin}taq intindiya:. Piru \pb{wakin}taq manam.}%sp que first line
{\morglo{wakin-taq}{some-\lsc{seq}}\morglo{intindi-ya-:}{understand-\lsc{prog}-\lsc{1}}\morglo{piru}{but}\morglo{wakin-taq}{some-\lsc{seq}}\morglo{mana-m}{no-\lsc{evd}}}%morpheme+gloss
\glotran{I’m catching [lit. understanding] \pb{some} of them. But the \pb{rest}, no.}{}%eng+spa trans
{}{}%rec - time

% 15
\gloexe{Glo3:Mamanqa}{}{ach}%
{Mamanqa kawsakunmi \pb{wakin}ninpaqqa.}%ach que first line
{\morglo{mama-n-qa}{mother-\lsc{3}-\lsc{top}}\morglo{kawsa-ku-n-mi}{live-\lsc{refl}-\lsc{3}\lsc{evd}}\morglo{wakin-ni-n-paq-qa}{some-\lsc{euph}-\lsc{3}-\lsc{abl}-\lsc{top}}}%morpheme+gloss
\glotran{His mother lived thanks to [lit. from] \pb{another} [man].}{}%eng+spa trans
{}{}%rec - time

\subsection[Interrogative-indefinites \phono{pi}, \phono{ima}, \phono{imay}, \phono{imayna}, \phono{mayqin}, \phono{imapaq}, \phono{ayka}]{\texorpdfstring{Interrogative-indefinites\\ \phono{pi}, \phono{ima}, \phono{imay}, \phono{imayna}, \phono{mayqin}, \phono{imapaq}, \phono{ayka}}{Interrogative-indefinites \phono{pi}, \phono{ima}, \phono{imay}, \phono{imayna}, \phono{mayqin}, \phono{imapaq}, \phono{ayka}}}\label{sec:IntInd}
\SYQ{} has seven interrogative-indefinite stems: \phono{pi} ‘who’, \phono{ima} ‘what’, \phono{imay} ‘when’, \phono{may} ‘where’, \phono{imayna} ‘how’, \phono{mayqin} ‘which’, \phono{imapaq} ‘why’, and \phono{ayka} ‘how much or how many’, as shown in Table~\ref{Tab9}. These form interrogative~(\ref{Glo3:Pitaq}--\ref{Glo3:Chaypaqa}), indefinite~(\ref{Glo3:Pipis}--\ref{Glo3:Ayvisdiman}), and negative indefinite pronouns~(\ref{Glo3:Manapipis}--\ref{Glo3:Rayaqa}). Interrogative pronouns\index[sub]{pronouns!interrogative} are formed by suffixing the stem --~generally but not obligatorily~-- with any of the enclitics \phono{-taq}, \phono{-raq}, \phono{-mI}, \phono{-shI} or \phono{-trI} (\phono{pi-taq} ‘who’, \phono{ima-raq} ‘what’); indefinite pronouns\index[sub]{pronouns!indefinite} are formed by attaching \phono{-pis} to the stem (\phono{pi-pis} ‘someone’, \phono{ima-pis} ‘something’); negative indefinite pronouns\index[sub]{pronouns!negative indefinite}, by preceding the indefinite pronoun with \phono{mana} ‘no’ (\phono{mana pi-pis} ‘no one’, \phono{mana ima-pis} ‘nothing’).

% TABLE 9
\begin{table}
\small\centering
\caption{Interrogative-indefinites}\label{Tab9}
\begin{tabular}{llll}
\lsptoprule
Stem			& Translation 	& (Negative) indefinite		& Translation 	\\
\midrule
\phono{pi}	 & who		 & \phono{(mana) pipis}		& some/anyone (no one)		\\
\phono{ima}	 & what		 & \phono{(mana) imapis}		& some/anything (nothing)	\\
\phono{imay} 	& when		 & \phono{(mana) imaypis}	& some/anytime (never)		\\
\phono{may}	 & where		 & \phono{(mana) maypis} 	& some/anywhere (nowhere)	\\
\phono{imapaq}	& why		 & \phono{(mana) imapaqpis} 	& some/any reason (no reason)	\\
\phono{imayna}	& how		 & \phono{(mana) imaynapis} 	& some/anyhow (no how)		\\
\phono{mayqin} 	& which		 & \phono{(mana) mayqinpis} 	& which ever (none)		\\
\phono{ayka}	& how many	 & \phono{(mana) aykapis}	& some/any amount (none)	\\
\lspbottomrule
\end{tabular}
\end{table}

% 1
\gloexe{Glo3:Pitaq}{}{ach}%
{¿\pb{Pi}taq willamanchik?}%ach que first line
{\morglo{pi-taq}{who-\lsc{seq}}\morglo{willa-ma-nchik}{tell-\lsc{1.obj}-\lsc{1pl}}}%morpheme+gloss
\glotran{\pb{Who}’s going tell us?}{}%eng+spa trans
{}{}%rec - time

% 2
\gloexe{Glo3:Imatam}{}{sp}%
{“¿\pb{Ima}tam maskakuyanki?” “Antaylumata maskakuya:”.}%sp que first line
{\morglo{ima-ta-m}{what-\lsc{acc}-\lsc{evd}}\morglo{maska-ku-ya-nki}{look.for-\lsc{refl}-\lsc{prog}-\lsc{2}}\morglo{antayluma-ta}{antayluma.berries-\lsc{acc}}\morglo{maska-ku-ya-:}{look.for-\lsc{prog}-\lsc{1}}}%morpheme+gloss
\glotran{“\pb{What} are you looking for?” “I’m looking for antayluma berries.”}{}%eng+spa trans
{}{}%rec - time

% 3
\gloexe{Glo3:Imayshi}{}{amv}%
{¿\pb{Imay}shi riyan Huancayota?}%amv que first line
{\morglo{imay-shi}{when-\lsc{evr}}\morglo{ri-ya-n}{go-\lsc{prog}-\lsc{3}}\morglo{\ili{Huancayo}-ta}{\ili{Huancayo}-\lsc{acc}}}%morpheme+gloss
\glotran{\pb{When} is he going to \ili{Huancayo}, did he say?}{}%eng+spa trans
{}{}%rec - time

% 4
\gloexe{Glo3:Maypaya}{}{amv}%
{¿\pb{May}payá Hildapa wakchan kayan?}%amv que first line
{\morglo{may-pa-yá}{where-\lsc{loc}-\lsc{emph}}\morglo{Hilda-pa}{Hilda-\lsc{gen}}\morglo{wakcha-n}{sheep-\lsc{3}}\morglo{ka-ya-n}{be-\lsc{prog}-\lsc{3}}}%morpheme+gloss
\glotran{\pb{Where} is Hilda’s sheep?}{}%eng+spa trans
{}{}%rec - time

% 5
\gloexe{Glo3:Chaymutuqa}{}{ach}%
{Chay mutuqa, ¿\pb{may}pitaq kayan?}%ach que first line
{\morglo{chay}{\lsc{demd}}\morglo{mutu-qa,}{motorcycle-\lsc{top}}\morglo{may-pi-taq}{where-\lsc{loc}-\lsc{top}}\morglo{ka-ya-n?}{be-\lsc{prog}-\lsc{3}}}%morpheme+gloss
\glotran{\pb{Where} is that motorbike?}{}%eng+spa trans
{}{}%rec - time

% 6
\gloexe{Glo3:Imapaqpapata}{}{amv}%
{¿\pb{Imapaq}~\dots{} papata apamuwarqanki?}%amv que first line
{\morglo{ima-paq}{what-\lsc{purp}}\morglo{papa-ta}{potato-\lsc{acc}}\morglo{apa-mu-wa-rqa-nki}{bring-\lsc{cisl}-\lsc{1.obj}-\lsc{pst}-\lsc{2}}}%morpheme+gloss
\glotran{\pb{Why}~\dots{} have you brought me potatoes?}{}%eng+spa trans
{}{}%rec - time

% 7
\gloexe{Glo3:Imapaqtaqcha}{}{ch}%
{¿\pb{Imapaq}taq chayna walmilla kidalun?}%ch que first line
{\morglo{ima-paq-taq}{what-\lsc{purp}-\lsc{seq}}\morglo{chayna}{thus}\morglo{walmi-lla}{woman-\lsc{rstr}}\morglo{kida-lu-n}{stay-\lsc{urgt}-\lsc{3}}}%morpheme+gloss
\glotran{\pb{Why} did just the woman stay like that?}{}%eng+spa trans
{}{}%rec - time

% 8
\gloexe{Glo3:Llakikuyan}{}{amv}%
{Llakikuyan atuqqa. “Diharuwan kumpadriy. ¿Kanan \pb{imayna}taq kutishaq?”}%amv que first line
{\morglo{llaki-ku-ya-n}{be.sad-\lsc{refl}-\lsc{prog}-\lsc{3}}\morglo{atuq-qa}{fox-\lsc{top}}\morglo{diha-ru-wa-n}{leave-\lsc{urgt}-\lsc{1.obj}-\lsc{3}}\morglo{kumpadri-y}{compadre-\lsc{1}}\morglo{kanan}{now}\morglo{imayna-taq}{how-\lsc{seq}}\morglo{kuti-shaq}{return-\lsc{1.fut}}}%morpheme+gloss
\glotran{The fox was sad. “My compadre left me. Now \pb{how} am I going to get back?”}{}%eng+spa trans
{}{}%rec - time

% 9
\gloexe{Glo3:Mayqinnin}{}{amv}%
{¿\pb{Mayqin}nin tunirun? ¿Kusinan?}%amv que first line
{\morglo{mayqin-ni-n}{which-\lsc{euph}-\lsc{3}}\morglo{tuni-ru-n}{crumble-\lsc{urgt}-\lsc{3}}\morglo{kusina-n}{kitchen-\lsc{3}}}%morpheme+gloss
\glotran{\pb{Which} of them crumbled? Her kitchen?}{}%eng+spa trans
{}{}%rec - time

% 10
\gloexe{Glo3:Lutuyuqmi}{}{lt}%
{Lutuyuqmi kayan wak runakuna. ¿Mamanchutr ñañanchutr? ¿\pb{Maqin}raq wañukun?}%lt que first line
{\morglo{lutu-yuq-mi}{mourning-\lsc{pos}-\lsc{evd}}\morglo{ka-ya-n}{be-\lsc{prog}-\lsc{evd}}\morglo{wak}{\lsc{dem.d}}\morglo{runa-kuna}{person-\lsc{pl}}\morglo{mama-n-chu-tr}{mother-\lsc{3}-\lsc{q}-\lsc{evc}}\morglo{ñaña-n-chu-tr}{sister-\lsc{3}-\lsc{q}-\lsc{evc}}\morglo{maqin-raq}{which-\lsc{cont}}\morglo{wañu-ku-n}{die-\lsc{refl}-\lsc{3}}}%morpheme+gloss
\glotran{Those people are wearing mourning. Would it be their mother or their sister? \pb{Which} died?}{}%eng+spa trans
{}{}%rec - time

% 11
\gloexe{Glo3:Aykanatr}{}{amv}%
{¿\pb{Ayka}ñatr awmintarun kabranqa?}%amv que first line
{\morglo{ayka-ña-tr}{how.many-\lsc{disc}-\lsc{evc}}\morglo{awminta-ru-n}{increase-\lsc{urgt}-\lsc{3}}\morglo{kabra-n-qa}{goat-\lsc{3}-\lsc{top}}}%morpheme+gloss
\glotran{\pb{How much} have her goats increased?}{}%eng+spa trans
{}{}%rec - time

% 12
\gloexe{Glo3:Chaypaqa}{}{ch}%
{Chaypaqa ¿\pb{Ayka}ktataq pagaya:?}%ch que first line
{\morglo{chay-pa-qa}{\lsc{dem.d}-\lsc{loc}-\lsc{top}}\morglo{ayka-kta-taq}{how.much-\lsc{acc}-\lsc{seq}}\morglo{paga-ya-:}{pay-\lsc{prog}-\lsc{1}}}%morpheme+gloss
\glotran{\pb{How much} am I paying there?}{}%eng+spa trans
{}{}%rec - time

% 13
\gloexe{Glo3:Pipis}{}{lt}%
{\pb{Pipis} fakultaykuwananpaq.}%lt que first line
{\morglo{pi-pis}{pi-\lsc{add}}\morglo{fakulta-yku-wa-na-n-paq}{faciliate-\lsc{excep}-\lsc{1.obj}-\lsc{nmlz}-\lsc{3}-\lsc{purp}}}%morpheme+gloss
\glotran{So \pb{someone} will help me out.}{}%eng+spa trans
{}{}%rec - time

% 14
\gloexe{Glo3:Wakchimpata}{}{sp}%
{Wak chimpata pasashpaqa \pb{ima}llata\pb{pis}.}%sp que first line
{\morglo{wak}{\lsc{dem.d}}\morglo{chimpa-ta}{opposite.side-\lsc{acc}}\morglo{pasa-shpa-qa}{pass-\lsc{subis}-\lsc{top}}\morglo{ima-lla-ta-pis}{what-\lsc{rstr}-\lsc{acc}-\lsc{add}}}%morpheme+gloss
\glotran{When you go by there on the opposite side -- [it could do] \pb{anything}.}{}%eng+spa trans
{}{}%rec - time

% 15
\gloexe{Glo3:Chaymuquykuna}{}{amv}%
{Chay muquykuna \pb{imaypis} nanaptin.}%amv que first line
{\morglo{chay}{\lsc{dem.d}}\morglo{muqu-y-kuna}{knee-\lsc{1}-\lsc{pl}}\morglo{imay-pis}{when-\lsc{add}}\morglo{nana-pti-n}{hurt-\lsc{subds}-\lsc{3}}}%morpheme+gloss
\glotran{\pb{Any time} my knees hurt.}{}%eng+spa trans
{}{}%rec - time

% 16
\gloexe{Glo3:Kayqullqita}{}{amv}%
{Kay qullqita qushqayki. ¡Ripukuy \pb{maytapis}!}%amv que first line
{\morglo{kay}{\lsc{dem.p}}\morglo{qullqi-ta}{money-\lsc{acc}}\morglo{qu-shqayki}{give-\lsc{3>1pl.fut}}\morglo{ripu-ku-y}{go-\lsc{refl}-\lsc{imp}}\morglo{may-ta-pis}{where-\lsc{acc}-\lsc{add}}}%morpheme+gloss
\glotran{I’m going to give you this money. Get going \pb{whereever}!}{}%eng+spa trans
{}{}%rec - time

% 17
\gloexe{Glo3:Kitrarun}{}{amv}%
{Kitrarun \pb{imaynapis} yaykurun Lluqi-Makiqa.}%amv que first line
{\morglo{kitra-ru-n}{open-\lsc{urgt}-\lsc{3}}\morglo{imayna-pis}{how-\lsc{add}}\morglo{yayku-ru-n}{enter-\lsc{urgt}-\lsc{3}}\morglo{Lluqi-Maki-qa}{Lluqi-Maki-\lsc{top}}}%morpheme+gloss
\glotran{Strong Arm opened it \pb{any way} [he could] and entered.}{}%eng+spa trans
{}{}%rec - time

% 18
\gloexe{Glo3:Manamkaytaqa}{}{ach}%
{Manam kaytaqa dihayta muna:chu. \pb{Imayna}paq\pb{pis} hinatam ruwakulla:.}%ach que first line
{\morglo{mana-m}{no-\lsc{evd}}\morglo{kay-ta-qa}{\lsc{dem.p}-\lsc{acc}-\lsc{top}}\morglo{diha-y-ta}{leave-\lsc{inf}-\lsc{acc}}\morglo{muna-:-chu}{want-\lsc{1}-\lsc{neg}}\morglo{imayna-paq-pis}{how-\lsc{abl}-\lsc{add}}\morglo{hina-ta-m}{thus-\lsc{acc}-\lsc{evd}}\morglo{ruwa-ku-lla-:}{make-\lsc{refl}-\lsc{rstr}-\lsc{1}}}%morpheme+gloss
\glotran{I don’t want to leave this. Like this, I just make \pb{whichever way}.}{}%eng+spa trans
{}{}%rec - time

% 19
\gloexe{Glo3:Imaynapis}{}{lt}%
{\pb{Imaynapis} yatrashaqmi. Limapaqa buskaq kanmiki.}%lt que first line
{\morglo{imayna-pis}{how-\lsc{add}}\morglo{yatra-shaq-mi}{know-\lsc{1.fut}-\lsc{evd}}\morglo{Lima-pa-qa}{Lima-\lsc{loc}-\lsc{top}}\morglo{buska-q}{look.for-\lsc{ag}}\morglo{ka-n-mi-ki}{be-\lsc{3}-\lsc{evd}-\lsc{ki}}}%morpheme+gloss
\glotran{\pb{Any way} about it, I’m going to find out. In Lima, there are people who read cards.}{}%eng+spa trans
{}{}%rec - time

% 20
\gloexe{Glo3:Chaywambra}{}{amv}%
{Chay wambra \pb{imapaqpis} rabyarirun.}%amv que first line
{\morglo{chay}{\lsc{dem.d}}\morglo{wambra}{child}\morglo{ima-paq-pis}{what-\lsc{purp}-\lsc{add}}\morglo{rabya-ri-ru-n}{be.mad-\lsc{incep}-\lsc{urgt}-\lsc{3}}}%morpheme+gloss
\glotran{That child gets mad for \pb{any reason}.}{}%eng+spa trans
{}{}%rec - time

% 21
\gloexe{Glo3:Ayvisdiman}{}{sp}%
{Ayvis dimandakurun tiyrayuqkuna trakrakunapaq \pb{imapaqpis}.}%sp que first line
{\morglo{ayvis}{sometimes}\morglo{dimanda-ku-ru-n}{denounce-\lsc{refl}-\lsc{urgt}-\lsc{3}}\morglo{tiyra-yuq-kuna}{land-\lsc{poss}-\lsc{pl}}\morglo{trakra-kuna-paq}{field-\lsc{pl}-\lsc{abl}}\morglo{ima-paq-pis}{what-\lsc{abl}-\lsc{add}}}%morpheme+gloss
\glotran{Sometimes they denounced landholders for their fields, \pb{for any thing at all}.}{}%eng+spa trans
{}{}%rec - time

% 22
\gloexe{Glo3:Manapipis}{}{amv}%
{\pb{Mana pipis} yachanchu.}%amv que first line
{\morglo{mana}{no}\morglo{pi-pis}{who-\lsc{add}}\morglo{yatra-n-chu}{know--\lsc{3}-\lsc{neg}}}%morpheme+gloss
\glotran{\pb{No one} lives here.}{}%eng+spa trans
{}{}%rec - time

% 23
\gloexe{Glo3:Puntrawqa}{}{sp}%
{Puntrawqa \pb{manam imapis} kanchu.}%sp que first line
{\morglo{puntraw-qa}{day-\lsc{top}}\morglo{mana-m}{no-\lsc{evd}}\morglo{ima-pis}{what-\lsc{add}}\morglo{ka-n-chu}{be-\lsc{3}-\lsc{neg}}}%morpheme+gloss
\glotran{In the day, there’s \pb{nothing}.}{}%eng+spa trans
{}{}%rec - time

% 24
\gloexe{Glo3:Pirumana}{}{amv}%
{Piru \pb{mana} \pb{imaypis} kaynaqa.}%amv que first line
{\morglo{piru}{but}\morglo{mana}{no}\morglo{imay-pis}{when-\lsc{add}}\morglo{kayna-qa}{thus-\lsc{top}}}%morpheme+gloss
\glotran{But \pb{never} like that.}{}%eng+spa trans
{}{}%rec - time

% 25
\gloexe{Glo3:Kasarakura}{}{ach}%
{Kasarakura: kayllapam hinallam kay lawpa kawsaku: tukuy watan watan \pb{manam maytapis} lluqsi:chu.}%ach que first line
{\morglo{kasara-ku-ra-:}{marry-\lsc{refl}-\lsc{pst}-\lsc{1}}
\morglo{kay-lla-pa-m}{\lsc{dem.p}-\lsc{rstr}-\lsc{loc}-\lsc{evd}}
\morglo{hina-lla-m}{thus-\lsc{rstr}-\lsc{evd}}
\morglo{kay}{\lsc{dem.p}}
\morglo{law-pa}{side-\lsc{loc}}
\morglo{kawsa-ku-:}{live-\lsc{refl}-\lsc{1}}
\morglo{tukuy}{all}
\morglo{wata-n}{year-\lsc{3}}
\morglo{wata-n}{year-\lsc{3}}
\morglo{mana-m}{no-\lsc{evd}}
\morglo{may-ta-pis}{where-\lsc{acc}-\lsc{add}}
\morglo{lluqsi-:-chu}{go.out-\lsc{1}-\lsc{neg}}}%morpheme+gloss
\glotran{I got married right here. Just like that, here I live, year in, year out, I do\pb{n’t} go \pb{anywhere}.}{}%eng+spa trans
{}{}%rec - time

% 26
\gloexe{Glo3:Manatalilachu}{}{ch}%
{\pb{Mana} talilachu \pb{maytrawpis}.}%ch que first line
{\morglo{mana}{no}\morglo{tali-la-chu}{find-\lsc{pst}-\lsc{neg}}\morglo{may-traw-pis}{where-\lsc{loc}-\lsc{add}}}%morpheme+gloss
\glotran{They have\pb{n’t} found him \pb{anywhere}.}{}%eng+spa trans
{}{}%rec - time

% 27
\gloexe{Glo3:Nakarinchikmi}{}{sp}%
{Ñakarinchikmi sapallanchikqa \pb{manam imaynapis}.}%sp que first line
{\morglo{ñaka-ri-nchik-mi}{suffer-\lsc{unint}-\lsc{1pl}-\lsc{evd}}\morglo{sapa-lla-nchik-qa}{alone-\lsc{rstr}-\lsc{1pl}-\lsc{}top}\morglo{mana-m}{no-\lsc{evd}}\morglo{imayna-pis}{how-\lsc{add}}}%morpheme+gloss
\glotran{We suffer alone \pb{without any way} [to make money].}{}%eng+spa trans
{}{}%rec - time

% 28
\gloexe{Glo3:Mayqinnikipis}{}{amv}%
{\pb{Mayqinnikipis} mana yuyachiwarqankichu.}%amv que first line
{\morglo{mayqin-ni-ki-pis}{which-\lsc{euph}-\lsc{2}-\lsc{add}}\morglo{mana}{no}\morglo{yuya-chi-wa-rqa-nki-chu}{remember-\lsc{caus}-\lsc{1.obj}-\lsc{pst}-\lsc{2}-\lsc{neg}}}%morpheme+gloss
\glotran{\pb{Neither} of you reminded me.}{}%eng+spa trans
{}{}%rec - time

% 29
\gloexe{Glo3:Rayaqa}{}{ach}%
{Rayaqa \pb{manam} \pb{aykas} kanchu.}%ach que first line
{\morglo{raya-qa}{row-\lsc{top}}\morglo{mana-m}{no-\lsc{evd}}\morglo{ayka-s}{how.many-\lsc{add}}\morglo{ka-n-chu}{be-\lsc{3}-\lsc{neg}}}%morpheme+gloss
\glotran{There is\pb{n’t even a small number} of rows.}{}%eng+spa trans
{}{}%rec - time

\noindent
Indefinite pronouns may figure in exclamations~(\ref{Glo3:Imamaldisyaw}).\\

% 30 (33)
\gloexe{Glo3:Imamaldisyaw}{}{amv}%
{¡\pb{Ima maldisyaw} chay Dimunyu! ¡Pudirniyuq!}%amv que first line
{\morglo{ima}{what}\morglo{maldisyaw}{damned}\morglo{chay}{\lsc{dem.d}}\morglo{dimunyu}{devil}\morglo{pudir-ni-yuq}{power-\lsc{euph}-\lsc{pos}}}%morpheme+gloss
\glotran{\pb{How damned} is the Devil! He’s powerful!}{}%eng+spa trans
{}{}%rec - time

\noindent
Interrogative pronouns are suffixed with the case markers corresponding to the questioned element~(\ref{Glo3:Runkuwanchu}).\\

% 31 (34)
\gloexe{Glo3:Runkuwanchu}{}{amv}%
{¿Runkuwanchu qaqurushaq? ¿\pb{Imawantaq} qaquruyman?}%amv que first line
{\morglo{runku-wan-chu}{sack-\lsc{instr}-\lsc{q}}\morglo{qaqu-ru-shaq}{rub-\lsc{urgt}-\lsc{1.fut}}\morglo{ima-wan-taq}{what-\lsc{instr}-\lsc{seq}}\morglo{qaqu-ru-y-man}{rub-\lsc{urgt}-\lsc{1}-\lsc{cond}}}%morpheme+gloss
\glotran{Should I rub it with a sack? \pb{With what} can I rub it?}{}%eng+spa trans
{}{}%rec - time

\noindent
Enclitics generally attach to the final word in the interrogative phrase: where the interrogative pronoun completes the phrase, the enclitic attaches directly to the interrogative (plus case suffixes, if any)~(\ref{Glo3:Imapaqmiqam}); where the phrase includes an \lsc{np}, the enclitic attaches to the \lsc{np} (\phono{pi-\pb{paq}-\pb{taq}} ‘for whom’ \phono{ima qullqi-\pb{tr}} ‘what money’)~(\ref{Glo3:Ukaliptuta}),~(\ref{Glo3:Aykawatanataq}).\\

% 32 (35)
\gloexe{Glo3:Imapaqmiqam}{}{sp}%
{“¿\pb{Imapaqmi} qam puka traki kanki?” nishpa.}%sp que first line
{\morglo{ima-paq-mi}{what-\lsc{purp}-\lsc{evd}}\morglo{qam}{you}\morglo{puka}{red}\morglo{traki}{foot}\morglo{ka-nki}{be-\lsc{2}}\morglo{ni-shpa}{	say-\lsc{subis}}}%morpheme+gloss
\glotran{“\pb{Why} are your feet red?” he said, they say.}{}%eng+spa trans
{}{}%rec - time

% 33 (36)
\gloexe{Glo3:Ukaliptuta}{}{amv}%
{¿Ukaliptuta pitaq simbranqa? ¿\pb{Pipaqñataq}?}%amv que first line
{\morglo{ukaliptu-ta}{eucaplyptus-\lsc{acc}}\morglo{pi-taq}{who-\lsc{seq}}\morglo{simbra-nqa}{plant-\lsc{3.fut}}\morglo{pi-paq-ña-taq}{who-\lsc{ben}-\lsc{disc}-\lsc{seq}}}%morpheme+gloss
\glotran{Who’s going to plant eucalyptus trees? \pb{For whom}?}{}%eng+spa trans
{}{}%rec - time

% 34 (37)
\gloexe{Glo3:Aykawatanataq}{}{amv}%
{¿\pb{Ayka watañataq} kanan nubinta i trispaq?}%amv que first line
{\morglo{ayka}{how.many}\morglo{wata-ña-taq}{year-\lsc{disc}-\lsc{seq}}\morglo{kanan}{now}\morglo{nubinta}{ninety}\morglo{i}{and}\morglo{tris-paq}{three-\lsc{abl}}}%morpheme+gloss
\glotran{\pb{How many years} is it already since ninety-three?}{}%eng+spa trans
{}{}%rec - time

\noindent
The interrogative enclitic is not employed in the interior of a subordinate clause but may attach to the final word in the clause (\phononb{¿Pi} \phononb{mishi-ta} \phononb{saru-ri-sa-n-ta} \phononb{qawa-rqa-nki?} ‘Who did you see trample the cat?’ \phononb{¿\pb{Pi}} \phononb{mishi-ta} \phononb{saru-ri-sa-n-ta-ta} \phononb{qawa-rqa-nki?} ‘Who did you see trample the cat?’).

\noindent
Interrogative phrases generally raise to sentence-initial position~(\ref{Glo3:Piwantumashpa}); they may, however, sometimes remain \emph{in-situ}, even in non-echo questions~(\ref{Glo3:Qaliqalikun}).\\

% 35 (38)
\gloexe{Glo3:Piwantumashpa}{}{amv}%
{¿\pb{Piwan} tumashpa\pb{tr} pay hamun?}%amv que first line
{\morglo{pi-wan}{who-\lsc{instr}}\morglo{tuma-shpa-tr}{take-\lsc{subis}-\lsc{evc}}\morglo{pay}{he}\morglo{hamu-n}{come-\lsc{3}}}%morpheme+gloss
\glotran{\pb{Who} did he come drinking \pb{with}?}{}%eng+spa trans
{}{}%rec - time

% 36 (39)
\gloexe{Glo3:Qaliqalikun}{}{ch}%
{¿Qaliqa likun \pb{maytataq}?}%ch que first line
{\morglo{qali-qa}{man-\lsc{top}}\morglo{li-ku-n}{come-\lsc{refl}-\lsc{3}}\morglo{may-ta-taq}{where-\lsc{acc}-\lsc{seq}}}%morpheme+gloss
\glotran{The man went \pb{where}?}{}%eng+spa trans
{}{}%rec - time

\noindent
Interrogative indefinites are sometimes employed as relative pronouns~(\ref{Glo3:Pashnaqapiwan}),~(\ref{Glo3:Familyanqa}).\\

% 37 (40)
\gloexe{Glo3:Pashnaqapiwan}{}{amv}%
{Pashñaqa \pb{piwan} trayaramun~\updag}%amv que first line
{\morglo{pashña-qa}{girl-\lsc{top}}\morglo{pi-wan}{who-\lsc{instr}}\morglo{traya-ra-mu-n}{arrive-\lsc{urgt}-\lsc{cisl}-\lsc{3}}}%morpheme+gloss
\glotran{The girl \pb{with whom} she came}{}%eng+spa trans
{}{}%rec - time

% 38 (41)
\gloexe{Glo3:Familyanqa}{}{amv}%
{Familyanqa qawarun \pb{imayna} wañukusam pustapa.}%amv que first line
{\morglo{familya-n-qa}{family-\lsc{3}-\lsc{top}}\morglo{qawa-ru-n}{see-\lsc{urgt}-\lsc{3}}\morglo{imayna}{how}\morglo{wañu-ku-sa-m}{die-\lsc{refl}-\lsc{npst}-\lsc{evd}}\morglo{pusta-pa}{clinic-\lsc{loc}}}%morpheme+gloss
\glotran{Her family saw \pb{how} she had died in the clinic.}{}%eng+spa trans
{}{}%rec - time

\noindent
Speakers use both \phono{ima ura} and \phono{imay ura} ‘what hour’ and ‘when hour’ to ask the time~(\ref{Glo3:Imayurataq}).\\

% 39 (42)
\gloexe{Glo3:Imayurataq}{}{lt}%
{¿\pb{Imay urataq} huntanqa kay yakuqa?}%lt que first line
{\morglo{imay}{when}\morglo{ura-taq}{hour-\lsc{seq}}\morglo{hunta-nqa}{fill-\lsc{3.fut}}\morglo{kay}{\lsc{dem.p}}\morglo{yaku-qa}{water-\lsc{top}}}%morpheme+gloss
\glotran{\pb{What time} will this water fill up?}{}%eng+spa trans
{}{}%rec - time

\noindent
Interrogative pronouns may be stressed with \phono{diyablu} ‘devil’ and like terms~(\ref{Glo3:Imadiyablu}).\\

% 40 (43)
\gloexe{Glo3:Imadiyablu}{}{amv}%
{¿\pb{Ima diyablu}yá ñuqanchik kanchik?}%amv que first line
{\morglo{ima}{what}\morglo{diyablu-yá}{devil-\lsc{emph}}\morglo{ñuqa-nchik}{I-\lsc{1pl}}\morglo{ka-nchik}{be-\lsc{1pl}}}%morpheme+gloss
\glotran{\pb{What the hell} are we?}{}%eng+spa trans
{}{}%rec - time

\largerpage
\noindent
Possessive suffixes attach to indefinites to yield phrases like ‘your things’ and ‘my people’~(\ref{Glo3:Manaimaykipis}--\ref{Glo3:Manavakanchik}); attaching to \phono{mayqin} ‘which’, they yield ‘which of \lsc{pron}’~(\ref{Glo3:Mayqinninchik}).\\

% 41 (44)
\gloexe{Glo3:Manaimaykipis}{}{ach}%
{\pb{Mana imaykipis} kaptin}%ach que first line
{\morglo{mana}{no}\morglo{ima-yki-pis}{what-\lsc{2}-\lsc{add}}\morglo{ka-pti-n}{be-\lsc{subds}-\lsc{3}}}%morpheme+gloss
\glotran{If you don’t have \pb{anything}}{}%eng+spa trans
{}{}%rec - time

% 42 (45)
\gloexe{Glo3:Yasqayaruptikimana}{}{ach}%
{Yasqayaruptiki \pb{mana pinikipis} kanqachu.}%ach que first line
{\morglo{yasqa-ya-ru-pti-ki}{old-\lsc{inch}-\lsc{urgt}-\lsc{subds}-\lsc{2}}\morglo{mana}{no}\morglo{pi-ni-ki-pis}{who-\lsc{euph}-\lsc{2}-\lsc{add}}\morglo{ka-nqa-chu}{be-\lsc{3.fut}-\lsc{neg}}}%morpheme+gloss
\glotran{When you’re old, you won’t have \pb{anyone}.}{}%eng+spa trans
{}{}%rec - time

% 43 (46)
\gloexe{Glo3:Manavakanchik}{}{amv}%
{\pb{Mana} vakanchik \pb{imanchik} kaptin hawkatr tiyakuchuwan.}%amv que first line
{\morglo{mana}{no}\morglo{vaka-nchik}{cow-\lsc{1pl}}\morglo{ima-nchik}{what-\lsc{1pl}}\morglo{ka-pti-n}{be-\lsc{subds}-3}\morglo{hawka-tr}{tranquil-\lsc{evc}}\morglo{tiya-ku-chuwan}{sit-\lsc{refl}-\lsc{1pl.cond}}}%morpheme+gloss
\glotran{\pb{Without} our cows and \pb{our stuff}, we could sit [live/be] in peace.}{}%eng+spa trans
{}{}%rec - time

% 44 (47)
\gloexe{Glo3:Mayqinninchik}{}{sp}%
{“¿\pb{Mayqinninchik} pirdirishun? Kusisam kayhina silbaku:” nin.}%sp que first line
{\morglo{mayqin-ni-nchik}{which-\lsc{euph}-\lsc{1pl}}\morglo{pirdi-ri-shun}{lose-\lsc{incep}-\lsc{1pl.fut}}\morglo{kusi-sa-m}{sew-\lsc{prf}-\lsc{evd}}\morglo{kay-hina}{\lsc{dem.p}-\lsc{comp}}\morglo{silba-ku-:}{whistle-\lsc{refl}-\lsc{1}}\morglo{ni-n}{say-\lsc{3}}}%morpheme+gloss
\glotran{“\pb{Which of us} will lose? Sewed up like this, I whistle,” he said.}{}%eng+spa trans
{}{}%rec - time

\noindent
\phono{Imapaq} ‘why’ is also sometimes realized as \phono{imapa} in \ACH{}~(\ref{Glo3:Imapamchayta}).\\

% 45 (48)
\gloexe{Glo3:Imapamchayta}{}{ach}%
{¿\pb{Imapam} chayta ruwara paytaq? ¿\pb{Imaparaq}?}%ach que first line
{\morglo{ima-pa-m}{what-\lsc{purp}-\lsc{evd}}\morglo{chay-ta}{\lsc{dem.d}\lsc{acc}}\morglo{ruwa-ra}{make-\lsc{pst}}\morglo{pay-taq}{he-\lsc{seq}}\morglo{ima-pa-raq}{what-\lsc{purp}-\lsc{cont}}}%morpheme+gloss
\glotran{\pb{Why} did they do that to him? \pb{Why ever}?}{}%eng+spa trans
{}{}%rec - time

\noindent
Negative indefinites may be formed with \phono{ni} ‘nor’ as well as \phono{mana}~(\ref{Glo3:Mananamkanan}); they may sometimes be formed with no negator at all~(\ref{Glo3:Katraykurun}),~(\ref{Glo3:Wakhina}).\\

% 46 (49)
\gloexe{Glo3:Mananamkanan}{}{ach}%
{Manañam kanan chay llamatapis qawanchikchu \pb{ni} \pb{imaypis} kanan unayñam.}%ach que first line
{\morglo{mana-ña-m}{no-\lsc{disc}-\lsc{evd}}\morglo{kanan}{now}\morglo{chay}{\lsc{dem.d}}\morglo{llama-ta-pis}{llama-\lsc{acc}-\lsc{add}}\morglo{qawa-nchik-chu}{see-\lsc{1pl}-\lsc{neg}}\morglo{ni}{nor}\morglo{imay-pis}{when-\lsc{add}}\morglo{kanan}{now}\morglo{unay-ña-m}{before-\lsc{disc}-\lsc{evd}}}%morpheme+gloss
\glotran{Now we don’t see llamas \pb{any more ever}. For a long time now.}{}%eng+spa trans
{}{}%rec - time

% 47 (50)
\gloexe{Glo3:Katraykurun}{}{sp}%
{Katraykurun. ¡\pb{Imapis} kanchu! “¡Ñuqata ingañamara!” nishpa.}%sp que first line
{\morglo{katra-yku-ru-n	}{release-\lsc{excep}-\lsc{urgt}-\lsc{3}}\morglo{ima-pis}{what-\lsc{add}}\morglo{ka-n-chu!}{be-\lsc{3}-\lsc{neg}}\morglo{ñuqa-ta}{I-\lsc{acc}}\morglo{ingaña-ma-ra}{trick-\lsc{1.obj}-\lsc{pst}}\morglo{ni-shpa}{say-\lsc{subis}}}%morpheme+gloss
\glotran{[The fox just] let it go and -- \pb{nothing}! “He tricked me!” said [the fox].}{}%eng+spa trans
{}{}%rec - time

% 48 (51)
\gloexe{Glo3:Wakhina}{}{lt}%
{Wakhina inutilisadu kakuyan \pb{imapaqpis} balinchu.}%lt que first line
{\morglo{wak-hina}{\lsc{dem.d}-\lsc{comp}}\morglo{inutilisadu}{unused}\morglo{ka-ku-ya-n}{be-\lsc{refl}-\lsc{prog}-\lsc{3}}\morglo{ima-paq-pis}{what-\lsc{purp}-\lsc{add}}\morglo{bali-n-chu}{be.worth-\lsc{3}-\lsc{neg}}}%morpheme+gloss
\glotran{It’s unused like that. It’s \pb{not} good for \pb{anything}.}{}%eng+spa trans
{}{}%rec - time

\noindent
Suffixed with the combining verb \phono{na-}, \phono{ima} ‘what’ forms a verb meaning ‘do what’ or ‘what happen’~(\ref{Glo3:Wanuqrunalla}--\ref{Glo3:Wanukunmantriki}).\\

% 49 (52)
\gloexe{Glo3:Wanuqrunalla}{}{sp}%
{Wañuq runalla hukvidata llakikuyan. “Kananqa prisutriki ñuqaqa rikushaq. ¿\pb{Imanashaq}?”}%sp que first line
{\morglo{wañu-q}{die-\lsc{ag}}\morglo{runa-lla}{person-\lsc{rstr}}\morglo{huk-vida-ta}{one-life-\lsc{acc}}\morglo{llaki-ku-ya-n}{sorrow-\lsc{refl}-\lsc{prog}-\lsc{3}}\morglo{kanan-qa}{now-\lsc{top}}\morglo{prisu-tri-ki}{imprisoned-\lsc{evc}-\lsc{ki}}\morglo{ñuqa-qa}{I-\lsc{top}}\morglo{riku-shaq}{go-\lsc{1.fut}}\morglo{ima-na-shaq}{what-\lsc{vrbz}-\lsc{1.fut}}}%morpheme+gloss
\glotran{She was very sorry for the deceased person. “Now I’m going to go to jail. \pb{What} \pb{will} \pb{I} \pb{do}?”}{}%eng+spa trans
{}{}%rec - time

% 50 (53)
\gloexe{Glo3:Karahu}{}{amv}%
{“¿Karahu-ta-taq \pb{imana}runtaq?” qawaykushpaqa huk utrpata qapikushpa kay kunkanman pasaykurun.}%amv
{\morglo{karahu-ta-taq}{jerk-\lsc{acc}-\lsc{seq}}\morglo{ima-na-ru-n-taq}{what-\lsc{vrbz}-\lsc{urgt}-\lsc{3}-\lsc{seq}}\morglo{qawa-yku-shpa-qa	}{look-\lsc{excep}-\lsc{subis}-\lsc{top}}\morglo{huk}{one}\morglo{utrpa-ta}{ash-\lsc{acc}}\morglo{qapi-ku-shpa}{grab-\lsc{refl}-\lsc{subis}}\morglo{kay}{kay}\morglo{kunka-n-man}{throat-\lsc{3}-\lsc{all}}\morglo{pasa-yku-ru-n}{pass-\lsc{excep}-\lsc{urgt}-\lsc{3}}}%morpheme+gloss
\glotran{She watched him then she said, “\pb{What happened} to that bastard?” and grabbed some ashes and stuffed them down his throat.}{}%eng+spa trans
{}{}%rec - time

%%%%%%%%%%%%%%%%%%%%%%%%%%%%%%%%%%%%%%%%%%%%%%%%%%%%%%
% Issue: .¿
% 51 (54)
\gloexe{Glo3:Wanukunmantriki}{}{ach}%
{Wañukunmantriki.¿\pb{Imana}nmantaq? ¿Imayna mana kutikamunmanchu?}%ach que first line
{\morglo{wañu-ku-n-man-tri-ki}{die-\lsc{refl}-\lsc{3}-\lsc{comp}-\lsc{evc}-\lsc{ki}}\morglo{ima-na-n-man-taq}{what-\lsc{vrbz}-\lsc{3}-\lsc{cond}-\lsc{seq}}\morglo{imayna}{why}\morglo{mana}{no}\morglo{kuti-ka-mu-n-man-chu}{return-\lsc{refl}-\lsc{cisl}-\lsc{3}-\lsc{cond}-\lsc{neg}}}%morpheme+gloss
\glotran{He could die, of course. What could \pb{happen}? Why can’t he come back?}{}%eng+spa trans
{}{}%rec - time

\noindent
In the \CH{} dialect, \phono{imayna} alternates with \phono{imamish}~(\ref{Glo3:Quniqunim}).\\

% 52 (55)
\gloexe{Glo3:Quniqunim}{}{ch}%
{Quni qunim ñuqa kaya:, kumadri. ¿Qam \pb{imamish} kayanki?}%ch que first line
{\morglo{quni}{warm}\morglo{quni-m}{warm-\lsc{evd}}\morglo{ñuqa}{I}\morglo{ka-ya-:}{be-\lsc{prog}-\lsc{1}}\morglo{kumadri}{comadre}\morglo{qam}{you}\morglo{imamish}{how}\morglo{ka-ya-nki}{\lsc{be}-\lsc{prog}-\lsc{2}}}%morpheme+gloss
\glotran{I’m really warm, comadre. \pb{How} are you?}{}%eng+spa trans
{}{}%rec - time

\subsection{Adjectives}
I follow the general practice in the treatment of adjectives in Quechuan languages and sort \SYQ{} adjectives\index[sub]{adjectives} into two classes: regular adjectives\index[sub]{adjectives!regular} (\phono{puka} ‘red’) and adverbial adjectives\index[sub]{adjectives!adverbial} (\phono{sumaq-ta} ‘nicely’). An additional class --~not native to \SYQ{} nor Quechua generally --~may be distinguished: gender adjectives\index[sub]{adjectives!gender} (\phono{kuntinta} ‘happy’). All three classes figure towards the end of the stack of potential noun modifiers, all of which precede the noun. Nouns may be modified by demonstratives (\phono{\pb{chay}} \phono{trakra} ‘that field’), quantifiers (\phono{\pb{ashlla}} \phono{trakra} ‘few fields’), numerals (\phono{\pb{trunka}} \phono{trakra} ‘ten fields’), negators (\phono{\pb{mana}} \phono{trakra-yuq} ‘person without fields’), pre-adjectives (\phono{\pb{dimas}} \phono{\pb{karu}} \phono{trakra} ‘field too far away’), adjectives (\phono{\pb{chaki}} \phono{trakra} ‘dry field’) and other nouns (\phono{\pb{sara}} \phono{trakra} ‘corn field’). Where modifiers appear in series, they appear in the order \lsc{dem}-\lsc{quant}-\lsc{num}-\lsc{neg}-pre\lsc{adj}-\lsc{adj}-\lsc{atr}-\lsc{nucleus} (\phono{chay} \phono{trunka} \phono{mana} \phono{dimas} \phono{chaki} \phono{sara} \phono{trakra} ‘these ten not-too-dry corn fields’).\footnote{Analysis and example taken from \citet{Parker76gram}\index[aut]{Parker, Gary J.}, confirmed in elicitation}. §\sectref{ssec:regadj}--\ref{ssec:preadj} cover regular adjectives, adverbial adjectives, gender adjectives, and preadjectives. Numeral adjectives are covered in \sectref{sec:numerals}

\subsubsection{Regular adjectives}\label{ssec:regadj}
The class of regular adjectives\index[sub]{adjectives!regular} includes all adjectives not included in the other two classes (\phono{trawa} ‘raw’, \phono{putka} ‘turbid’).~(\ref{Glo3:Wakpishqu}--\ref{Glo3:Wakumbruyan}) give examples. Adjectives are often repeated. The effect is augmentative (\phono{uchuk} ‘small’~→~\phono{uchuk-uchuk} ‘very small’). When adjectives are repeated, the last consonant or the last syllable of the first instance is generally elided (\phono{alli-allin} ‘very good’, \phono{hat-hatun} ‘very big’).\\

% 1
\gloexe{Glo3:Wakpishqu}{}{amv}%
{Wak pishqu mikukuyan mikunayta -- ¡\pb{qatra} pishqu!}%
{\morglo{wak}{\lsc{dem.d}}\morglo{pishqu}{bird}\morglo{miku-ku-ya-n}{eat-\lsc{refl}-\lsc{prog}-3}\morglo{miku-na-y-ta}{eat-\lsc{nmlz}-1-\lsc{acc}}\morglo{qatra}{dirty}\morglo{pishqu}{bird}}%morpheme+gloss
\glotran{That bird is eating my food -- dirty bird!}%eng
{‘Ese pájaro come mi comida -- ¡pájaro sucio!’}%spa
{}{}%rec - time

% 2
\gloexe{Glo3:Wakumbruyan}{}{amv}%
{Wak umbruyanñatr mamanta. \pb{Hat hatun} kayan.}%
{\morglo{wak}{\lsc{dem.d}}\morglo{umbru-ya-n-ña-tr}{carry.on.shoulder-\lsc{prog}-3-\lsc{disc}-\lsc{evc}}\morglo{mama-n-ta}{mother-3-\lsc{acc}}\morglo{hat-hatun}{big-big}\morglo{ka-ya-n}{be-\lsc{prog}-3}}%morpheme+gloss
\glotran{That one would be carrying his mother on his shoulders already -- he’s really big!}%eng
{‘Ese ya estará cargando a su mamá. Es tremendo.’}%spa
{}{}%rec - time

\subsubsection{Adverbial adjectives}
Adjectives\index[sub]{adjectives!adverbial} may occur adverbally, in which case they are generally but not necessarily inflected with \phono{-ta} (\phono{quyu} ‘ugly’~→~\phono{quyu-ta} ‘awfully’).~(\ref{Glo3:Aburikurun}--\ref{Glo3:Rupanchikta}) give examples.\\

% 1
\gloexe{Glo3:Aburikurun}{}{amv}%
{Aburikurun sakristanqa \pb{wama-wamaqta} kampanata suynachiptin}%
{\morglo{aburi-ku-ru-n}{annoy-\lsc{refl}-\lsc{urgt}-3}\morglo{sakristan-qa}{deacon-\lsc{top}}\morglo{wama-wamaq-ta}{a.lot-a.lot-\lsc{acc}}\morglo{kampana-ta}{bell-\lsc{acc}}\morglo{suyna-chi-pti-n}{sound-\lsc{caus}-\lsc{subds}-3}}%morpheme+gloss
\glotran{The deacon got annoyed that [Lluqi Maki] rang the bell \pb{so much}.}%eng
{‘El sacristán se aburrió cuando [Mano de Bastón] hacía sonar la campana constantamente’.}%spa
{}{}%rec - time

% 2
\gloexe{Glo3:Rupanchikta}{}{amv}%
{Rupanchikta trurakunchik \pb{qilluta}.}%
{\morglo{rupa-nchik-ta}{clothes-\lsc{1pl}-\lsc{acc}}\morglo{trura-ku-nchik}{put-\lsc{refl}-\lsc{1pl}}\morglo{qillu-ta.}{yellow-\lsc{acc}}}%morpheme+gloss
\glotran{We dress \pb{[in] yellow}.}{}%eng+spa trans
{}{}%rec - time

\subsubsection{Gender adjectives}
A few adjectives\index[sub]{adjectives!gender}, all borrowed from Spanish, may inflect for gender (masculine \pb{/u/} or feminine \pb{/a/}) (\phono{kuntintu} ‘happy’, \phono{luka} ‘crazy’) in case they modify nouns referring to animate male or female individuals, respectively. Some nouns indigenous to \SYQ{} specify the gender of the referent (\phono{masha} ‘son-in-law’, \phono{llumchuy} ‘daughter-in-law’)~(\ref{Glo3:Mashapis}).\\

% 1
\gloexe{Glo3:Mashapis}{}{ch}%
{\pb{masha}:pis qalipis walmipis wawi:kunapaq}%ch que first line
{\morglo{masha-:-pis}{son.in.law-\lsc{1}-\lsc{add}}\morglo{qali-pis}{man-\lsc{add}}\morglo{walmi-pis}{woman-\lsc{add}}\morglo{wawi-:-kuna-paq}{baby-\lsc{1}-\lsc{pl}-\lsc{gen}}}%morpheme+gloss
\glotran{my \pb{son-in-law}, too, my children’s sons and daughters}{}%eng+spa trans
{}{}%rec - time

\noindent
Indeed, some names of family relations specify the gender of both members of the relationship (\phono{wawqi} ‘brother of a male’, \phono{ñaña} ‘sister of a female’)~(\ref{Glo3:Wanurachin}--\ref{Glo3:Chayubihapa}).\\

% 2
\gloexe{Glo3:Wanurachin}{}{ach}%
{Wañurachin \pb{wawqi}nñataqa, “¡Ama wawqi:ta!” niptin.}%ach que first line
{\morglo{wañu-ra-chi-n}{die-\lsc{urgt}-\lsc{caus}-\lsc{3}}\morglo{wawqi-n-ña-ta-qa}{brother-\lsc{3}-\lsc{disc}-\lsc{acc}-\lsc{top}}\morglo{ama}{\lsc{proh}}\morglo{wawqi-:-ta}{brother-\lsc{1}-\lsc{acc}}\morglo{ni-pti-n}{say-\lsc{subds}-\lsc{3}}}%morpheme+gloss
\glotran{They killed his \pb{brother} when he said, “Don’t [kill] my \pb{brother}!”}{}%eng+spa trans
{}{}%rec - time

% 3
\gloexe{Glo3:nanypis}{}{amv}%
{\pb{Ñaña}ypis turiypis karqam piru wañukunña.}%amv que first line
{\morglo{ñaña-y-pis}{sister-\lsc{1}\lsc{add}}\morglo{turi-y-pis}{brother-\lsc{1}-\lsc{add}}\morglo{ka-rqa-m}{be-\lsc{pst}-\lsc{evd}}\morglo{piru}{but}\morglo{wañu-ku-n-ña}{die-\lsc{refl}-\lsc{3}-\lsc{disc}}}%morpheme+gloss
\glotran{I had a \pb{sister} and a \pb{brother}, but they died already.}{}%eng+spa trans
{}{}%rec - time

% 4
\gloexe{Glo3:Chayubihapa}{}{amv}%
{chay \pb{ubihapa wawan}ta chay \pb{karnirupa churin}ta}%amv que first line
{\morglo{chay}{\lsc{dem.d}}\morglo{ubiha-pa}{sheep-\lsc{gen}}\morglo{wawa-n-ta}{baby-\lsc{3}-\lsc{acc}}\morglo{chay}{\lsc{dem.d}}\morglo{karniru-pa}{ram-\lsc{gen}}\morglo{churi-n-ta}{child-\lsc{3}-\lsc{acc}}}%morpheme+gloss
\glotran{the \pb{baby of that sheep}, the \pb{baby of that ram}}{}%eng+spa trans
{}{}%rec - time

\noindent
Where it is necessary to specify the gender of the referent of a noun that does not indicate gender, \SYQ{} modifies that noun with \phono{qari} ‘man’ or \phono{warmi} ‘woman’ in the case of people (\phono{warmi wawa} ‘daughter’ \lit~‘girl child’) and \phono{urqu} ‘male’ or \phono{trina} ‘female’ in the case of animals~(\ref{Glo3:Pagashunnam}),~(\ref{Glo3:Wakvakanqa}).\\

% 5
\gloexe{Glo3:Pagashunnam}{}{lt}%
{“Pagashunñam rigarunanpaqmi. Balikurunki”, niwara ya chay \pb{wawi} \pb{warmi}.}%lt que first line
{\morglo{paga-shun-ña-m}{pay-\lsc{1pl.fut}-\lsc{disc}-\lsc{evd}}\morglo{riga-ru-na-n-paq-mi}{irrigate-\lsc{urgt}-\lsc{nmlz}-\lsc{3}-\lsc{purp}-\lsc{evd}}\morglo{bali-ku-ru-nki}{request.service-\lsc{refl}-\lsc{urgt}-\lsc{2}}\morglo{ni-wa-ra}{say-\lsc{1.obj}-\lsc{pst}}\morglo{ya}{\lsc{emph}}\morglo{chay}{\lsc{dem.d}}\morglo{wawi}{baby}\morglo{warmi}{woman}}%morpheme+gloss
\glotran{“We’re going to pay already to water. You’re going to request someone,” my \pb{daughter} said to me.}{}%eng+spa trans
{}{}%rec - time

% 6
\gloexe{Glo3:Wakvakanqa}{}{amv}%
{Wak vakanqa watrarusa. ¿Wak \pb{urqu}chu wawan, \pb{trina}chu?}%amv que first line
{\morglo{wak}{\lsc{dem.d}}\morglo{vaka-n-qa}{cow-\lsc{3}-\lsc{top}}\morglo{watra-ru-sa}{give.birth-\lsc{urgt}-\lsc{npst}}\morglo{wak}{\lsc{dem.d}}\morglo{urqu-chu}{male-\lsc{q}}\morglo{wawa-n}{baby-\lsc{3}}\morglo{trina-chu}{female-\lsc{q}}}%morpheme+gloss
\glotran{His cow gave birth. Is it a \pb{male} or a \pb{female}?}{}%eng+spa trans
{}{}%rec - time

\subsubsection{Preadjectives}\label{ssec:preadj}
Adjectives\index[sub]{adjectives!preadjectives} admit modification by adverbs~(\ref{Glo3:Pasaypaqchanchu}) and nouns functioning adjectivally; the latter are suffixed with \phono{-ta}.\\

% 1
\gloexe{Glo3:Pasaypaqchanchu}{}{lt}%
{\pb{Pasaypaq} chanchu sapatu \pb{pasaypaq} lapi chuku \pb{pasaypaq}shi ritamun paypis.}%lt que first line
{\morglo{pasaypaq}{completely}\morglo{chanchu}{old}\morglo{sapatu}{shoe}\morglo{pasaypaq}{completely}\morglo{lapi}{old}\morglo{chuku}{hat}\morglo{pasaypaq-shi}{completely-\lsc{evr}}\morglo{rita-mu-n}{go-\lsc{cisl}-\lsc{3}}\morglo{pay-pis}{he-\lsc{add}}}%morpheme+gloss
\glotran{He, too, went with \pb{totally} old shoes and a \pb{completely} worn hat, they say.}{}%eng+spa trans
{}{}%rec - time

\subsection{Numerals}\label{sec:numerals}
\SYQ{} employs two sets of cardinal numerals\index[sub]{numerals}. The first is native to Quechua; the second is borrowed from Spanish. The latter is always used for time and almost always for money. Also borrowed from Spanish are the ordinal numerals, \phono{primiru} ‘first’, \phono{sigundu} ‘second’, and so on. There is no set of ordinal numerals native to \SYQ. §\sectref{ssec:gennum}--\ref{ssec:timenum} cover general numerals, ordinal numerals, and time numerals in turn. \sectref{ssec:numallaff} and~\ref{ssec:huk} cover numerals inflected for possessive and the special case of \phono{huk} ‘one’, respectively.

\subsubsection{General numerals}\label{ssec:gennum}
The set of cardinal numerals\index[sub]{numerals!cardinal} native to \SYQ{} includes twelve members: \phono{huk} ‘one’; \phono{ishkay} ‘two’; \phono{kimsa} ‘three’; \phono{tawa} ‘four’; \phono{pichqa} ‘five’; \phono{suqta} ‘six’; \phono{qanchis} ‘seven’; \phono{pusaq} ‘eight’; \phono{isqun} ‘nine’; \phono{trunka} ‘ten’; \phono{patrak} ‘hundred’; and \phono{waranqa} ‘thousand’~(\ref{Glo3:IshkayWanka}--\ref{Glo3:Inganaykun}).\\

% 1
\gloexe{Glo3:IshkayWanka}{}{amv}%
{\pb{Ishkay} Wanka samakushqa huk matraypi.}%
{\morglo{ishkay}{two}\morglo{Wanka}{Huancayoan}\morglo{sama-ku-shqa}{rest-\lsc{refl}-\lsc{npst}}\morglo{huk}{one}\morglo{matray-pi}{cave-\lsc{loc}}}%morpheme+gloss
\glotran{\pb{Two} Huancayoans rested in a cave.}%eng
{‘Dos \ili{Huanca}ínos se alojaron en una cueva’.}%spa
{}{}%rec - time

% 2
\gloexe{Glo3:Kimsakillam}{}{amv}%
{\pb{Kimsa} killam kaypaq paranqa.}%amv que first line
{\morglo{kimsa}{three}\morglo{killa-m}{month-\lsc{evd}}\morglo{kay-paq}{\lsc{dem.p}-\lsc{loc}}\morglo{para-nqa}{rain-\lsc{3.fut}}}%morpheme+gloss
\glotran{It’s going to rain for \pb{three} months here.}{}%eng+spa trans
{}{}%rec - time

% 3
\gloexe{Glo3:Inganaykun}{}{ach}%
{Ingañaykun. Chay \pb{waranqa} kwistasantam~\dots}%ach que first line
{\morglo{ingaña-yku-n}{cheat-\lsc{excep}-\lsc{3}}\morglo{chay}{\lsc{dem.d}}\morglo{waranqa}{thousand}\morglo{kwista-sa-n-ta-m}{cost-\lsc{prf}-\lsc{3}-\lsc{acc}-\lsc{evd}}}%morpheme+gloss
\glotran{They cheat them. That which cost one \pb{thousand}~\dots}{}%eng+spa trans
{}{}%rec - time

\noindent
‘Twenty’, ‘thirty’ and so on are formed by placing a unit numeral --~\phono{ishkay} ‘two’, \phono{kimsa} ‘three’, and so on~-- in attributive construction with \phono{trunka} ‘ten’~(\ref{Glo3:Riganchik}).\\

% 4
\gloexe{Glo3:Riganchik}{}{amv}%
{Riganchik chay sarataqa \pb{ishkay trunka} \pb{kimsa trunka} puntrawniyuqtamá.}%amv que first line
{\morglo{riga-nchik}{irrigate-\lsc{1pl}}\morglo{chay}{\lsc{dem.d}}\morglo{sara-ta-qa}{corn-\lsc{acc}-\lsc{top}}\morglo{ishkay}{two}\morglo{trunka}{ten}\morglo{kimsa}{three}\morglo{trunka}{ten}\morglo{puntraw-ni-yuq-ta-m-á}{day-\lsc{euph-\lsc{poss}-\lsc{acc}-\lsc{evd}-\lsc{emph}}}}%morpheme+gloss
\glotran{We water the corn that’s \pb{twenty} or \pb{thirty} days old.}{}%eng+spa trans
{}{}%rec - time

\noindent
‘Forty-one’ and ‘forty-two’ and so on are formed by adding another unit numeral --~\phono{huk} ‘one’, \phono{ishkay} ‘two’, and so on~-- using \phono{-yuq} or, following a consonant, its allomorph, \phono{-ni-yuq} (\phono{ishkay trunka pusaq-ni-yuq} ‘twenty-eight’)~(\ref{Glo3:Trunka}).\\

% 5
\gloexe{Glo3:Trunka}{}{amv}%
{\pb{Trunka} \pb{ishkayniyuq}paqpis ruwanchik.}%amv que first line
{\morglo{trunka}{ten}\morglo{ishkay-ni-yuq-paq-pis}{two-\lsc{euph}-\lsc{poss}-\lsc{abl}-\lsc{add}}\morglo{ruwa-nchik}{make-\lsc{1pl}}}%morpheme+gloss
\glotran{We make them out of \pb{twelve} [strands], too.}{}%eng+spa trans
{}{}%rec - time

\noindent
General numerals are ambivalent, and may function as modifiers and as pronouns~(\ref{Glo3:Ishkayllata}).\\

% 6
\gloexe{Glo3:Ishkayllata}{}{lt}%
{\pb{Ishkay}llata apikunaypaq. Shantipa mana kashachu.}%lt que first line
{\morglo{ishkay-lla-ta}{two-\lsc{rstr}-\lsc{acc}}\morglo{api-ku-na-y-paq}{pudding-\lsc{refl}-\lsc{nmlz}-\lsc{1}-\lsc{purp}}\morglo{Shanti-pa}{Shanti-\lsc{gen}}\morglo{mana}{no}\morglo{ka-sha-chu}{be-\lsc{npst}-\lsc{neg}}}%morpheme+gloss
\glotran{Just \pb{two} so I can make pudding. Shanti didn’t have any.}{}%eng+spa trans
{}{}%rec - time

\subsubsection{Ordinal numerals}
\SYQ{} has no native system of ordinal numerals\index[sub]{numerals!ordinal}. It borrows the Spanish \spanish{primero} \spanish{segundo} and so on~(\ref{Glo3:Chaymamakuqta}),~(\ref{Glo3:Kwartulla}).\\

% 1
\gloexe{Glo3:Chaymamakuqta}{}{ach}%
{“Chay mamakuqta siqachinki \pb{primiru} yatrachishunaykipaq”, nin.}%ach que first line
{\morglo{chay}{\lsc{dem.d}}\morglo{mamakuq-ta}{old.lady-\lsc{acc}}\morglo{siqa-chi-nki}{go.up-\lsc{caus}-\lsc{2}}\morglo{primiru}{first}\morglo{yatra-chi-shu-na-yki-paq}{know-\lsc{caus}-\lsc{3>2}-\lsc{nmlz}-\lsc{3>2}-\lsc{purp}}\morglo{ni-n}{say-\lsc{3}}}%morpheme+gloss
\glotran{“Make the old woman go up \pb{first} in order to teach you,” they said.}{}%eng+spa trans
{}{}%rec - time

% 2
\gloexe{Glo3:Kwartulla}{}{amv}%
{\pb{Kwartulla} \pb{kintulla} manam puchukachiwarqapischu.}%amv que first line
{\morglo{kwartu-lla}{fourth-\lsc{rstr}}\morglo{kintu-lla}{fifth-\lsc{rstr}}\morglo{mana-m}{no-\lsc{evd}}\morglo{puchuka-chi-wa-rqa-pis-chu}{finish-\lsc{caus}-\lsc{1.obj}-\lsc{pst}-\lsc{add}-\lsc{neg}}}%morpheme+gloss
\glotran{They had me finish \pb{fourth} [grade], no more, \pb{fifth} [grade], no more.}{}%eng+spa trans
{}{}%rec - time

\noindent
The expression \phono{punta-taq} is sometimes employed for ‘first’~(\ref{Glo3:Qarinman}).\footnote{An anonymous reviewer points out that “most Quechuan languages express ordinals by attaching the enclitic \phono{-kaq} to the numeral,” as in \phono{ishkay-kaq} ‘second’, literally ‘that which is number two’. “The -\phono{kaq} enclitic derives historically from the copula *ka- plus agentive *-q.” This structure is not attested in Yauyos.}\\

% 3
\gloexe{Glo3:Qarinman}{}{amv}%
{Qarinman sirvirun \pb{puntataq} hinashpa kikinpis mikuruntriki.}%amv que first line
{\morglo{qari-n-man}{man-\lsc{3}-\lsc{all}}\morglo{sirvi-ru-n}{serve-\lsc{urgt}-\lsc{3}}\morglo{punta-taq}{point-\lsc{seq}}\morglo{hinashpa}{then}\morglo{kiki-n-pis}{self-\lsc{3}-\lsc{add}}\morglo{miku-ru-n-tri-ki}{eat-\lsc{urgt}-\lsc{3}-\lsc{evc}-\lsc{ki}}}%morpheme+gloss
\glotran{She served her husband [the poisoned tuna] \pb{first} then she herself must have eaten it.}{}%eng+spa trans
{}{}%rec - time

\subsubsection{Time numerals and pre-numerals}\label{ssec:timenum}
\SYQ{} makes use of the full set of Spanish cardinal numerals\index[sub]{numerals!time}: \phono{unu} ‘one’, \phono{dus} ‘two’, \phono{tris} ‘three’, \phono{kwatru} ‘four’, \phono{sinku} ‘five’, \phono{sis} ‘six’, \phono{siyti} ‘seven’, \phono{uchu} ‘eight’, \phono{nuybi} ‘nine’, \phono{dis} ‘ten’, and so on. It is this set that is used in telling time. As in Spanish, time numerals are preceded by the pre-numerals \phono{la} or \phono{las}~(\ref{Glo3:Punukun}).\\

% 1
\gloexe{Glo3:Punukun}{}{amv}%
{Puñukun tuta \pb{a las tris} di la mañanataqa.}%amv que first line
{\morglo{puñu-ku-n}{sleep-\lsc{refl}-\lsc{3}}\morglo{tuta}{night}\morglo{a}{at}\morglo{las}{the}\morglo{tris}{three}\morglo{di}{of}\morglo{la}{the}\morglo{mañana-ta-qa}{morning-\lsc{acc}-\lsc{top}}}%morpheme+gloss
\glotran{He went to sleep at night -- at \pb{three} in the morning.}{}%eng+spa trans
{}{}%rec - time

\newpage 
\noindent
Time expressions are usually suffixed with \phono{-ta} (\phono{a las dusi-ta} ‘at twelve o’clock’):% (\ref{Glo3:Lastris}).\\

% 2
\gloexe{Glo3:Lastris}{}{amv}%
{Las tris i midya\pb{ta} qaykuruni.}%amv que first line
{\morglo{las}{the}\morglo{tris}{three}\morglo{i}{and}\morglo{midya-ta}{middle-\lsc{acc}}\morglo{qayku-ru-ni}{corral-\lsc{urgt}-\lsc{1}}}%morpheme+gloss
\glotran{I threw him in the corral \pb{at} three thirty.}{}%eng+spa trans
{}{}%rec - time

\subsubsection{Numerals with possessive suffixes}\label{ssec:numallaff}
Any numeral, \lsc{num}, may be suffixed with any plural possessive suffix\index[sub]{numerals!with possessive suffixes} --~\phono{-nchik}, \phono{-Yki}, or \phono{-n}. These constructions translate ‘we/you/they \lsc{num}’ or ‘the \lsc{num} of us/you/them’ (\phono{kimsanchik} ‘we three’, ‘the three of us’)~(\ref{Glo3:Ishkaynin}).\\

% 1
\gloexe{Glo3:Ishkaynin}{}{amv}%
{\pb{Ishkaynin}, \pb{kimsan} kashpaqa mikunyá.}%amv que first line
{\morglo{ishkay-ni-n}{two-\lsc{euph}-\lsc{3}}\morglo{kimsa-n}{three-\lsc{3}}\morglo{ka-shpa-qa}{be-\lsc{subis}-\lsc{top}}\morglo{miku-n-yá}{eat-\lsc{3}-\lsc{emph}}}%morpheme+gloss
\glotran{If there are \pb{two of them} or \pb{three of them}, they eat.}{}%eng+spa trans
{}{}%rec - time

\noindent
In the case of \phono{ishkay} this translates ‘both of’~(\ref{Glo3:Ishkayninchik}).\\

% 2
\gloexe{Glo3:Ishkayninchik}{}{amv}%
{\pb{Ishkayninchik} ripukushun.}%
{\morglo{ishkay-ni-nchik}{two-\lsc{euph}-1\lsc{pl}}\morglo{ripu-ku-shun}{leave-\lsc{refl}-1\lsc{pl}.\lsc{fut}}}%morpheme+gloss
\glotran{Let’s go \pb{both of us}.}%eng
{‘Nos iremos los dos’.}%spa
{}{}%rec - time

\noindent
\phono{huknin} translates both ‘one of’ and ‘the other of’~(\ref{Glo3:Huknin}).\\

% 3
\gloexe{Glo3:Huknin}{}{ach}%
{\pb{Huknin}pis \pb{huknin}pis hinaptin sapalla: witrqarayachin.}%ach que first line
{\morglo{huk-ni-n-pis}{one-\lsc{euph}-\lsc{3}-\lsc{add}}\morglo{huk-ni-n-pis}{one-\lsc{euph}-\lsc{3}-\lsc{add}}\morglo{hinaptin}{then}\morglo{sapa-lla-:}{alone-\lsc{rstr}-\lsc{1}}\morglo{witrqa-ra-ya-chi-n}{close-\lsc{unint}-\lsc{intens}-\lsc{caus}-\lsc{3}}}%morpheme+gloss
\glotran{\pb{One of them} then the \pb{other of them} [leaves] and I’m closed in all alone.}{}%eng+spa trans
{}{}%rec - time

\subsubsection{\phono{huk}}\label{ssec:huk}
\phono{huk}\index[sub]{numerals!\phono{huk}} ‘one’ has several functions in addition to its function as a numeral~(\ref{Glo3:Pichqamulla}) and numeral adjective~(\ref{Glo3:Achka}).\\

% 1
\gloexe{Glo3:Pichqamulla}{}{ch}%
{Pichqa mulla. \pb{Huk}, ishkay, kimsa, tawa, pichqa.}%ch que first line
{\morglo{pichqa}{five}\morglo{mulla}{quota}\morglo{huk}{one}\morglo{ishkay}{two}\morglo{kimsa}{three}\morglo{tawa}{four}\morglo{pichqa}{five}}%morpheme+gloss
\glotran{Five quotas [of water]. \pb{One}, two, three, four, five.}{}%eng+spa trans
{}{}%rec - time

% 2
\gloexe{Glo3:Achka}{}{amv}%
{Achka~\dots{} lluqsin \pb{huk} pakayllapaq.}%amv que first line
{\morglo{achka}{a.lot}\morglo{lluqsi-n}{come.out-\lsc{3}}\morglo{huk}{one}\morglo{pakay-lla-paq}{pacay-\lsc{rstr}-\lsc{abl}}}%morpheme+gloss
\glotran{A lot [of seeds] come out of just \pb{one} pacay.}{}%eng+spa trans
{}{}%rec - time

\noindent
It may serve both as an indefinite determiner, as in~(\ref{Glo3:Hukinhiniyrush}) and~(\ref{Glo3:Hinaptinna}), and as a pronoun, as in~(\ref{Glo3:Puchkapaqa}) and~(\ref{Glo3:Ayvis}).\\

% 3
\gloexe{Glo3:Hukinhiniyrush}{}{ach}%
{\pb{Huk} inhiniyrush rikura. Chay ubsirvaq hinashpash~\dots}%ach que first line
{\morglo{huk}{one}\morglo{inhiniyru-sh}{engineer-\lsc{evr}}\morglo{riku-ra}{go-\lsc{pst}}\morglo{chay}{\lsc{dem.d}}\morglo{ubsirva-q}{observe-\lsc{ag}}\morglo{hinashpa-sh}{then-\lsc{evr}}}%morpheme+gloss
\glotran{\pb{An} engineer went. That observer, then, they say~\dots}{}%eng+spa trans
{}{}%rec - time

% 4
\gloexe{Glo3:Hinaptinna}{}{sp}%
{Hinaptinña \pb{huk} atrqay pasan, ismu atrqay. “\pb{Huk} turutam pagasayki”.}%sp que first line
{\morglo{hinaptin-ña}{then-\lsc{disc}}\morglo{huk}{one}\morglo{atrqay}{eagle}\morglo{pasa-n,}{pass-\lsc{3}}\morglo{ismu}{grey}\morglo{atrqay}{eagle}\morglo{huk}{one}\morglo{turu-ta-m}{bull-\lsc{acc}-\lsc{evd}}\morglo{paga-sayki}{pay-\lsc{1>2.fut}}}%morpheme+gloss
\glotran{Then \pb{an} eagle passed by, a gray eagle. “I’ll pay you \pb{a} bull,” [said the girl].}{}%eng+spa trans
{}{}%rec - time

% 5
\gloexe{Glo3:Puchkapaqa}{}{ach}%
{Puchka: paqarinninta \pb{huk}ta ruwa: minchanta \pb{huk}ta.}%ach que first line
{\morglo{puchka-:}{spin-\lsc{1}}\morglo{paqarin-ni-n-ta}{tomorrow-\lsc{euph}-\lsc{3}-\lsc{acc}}\morglo{huk-ta}{one-\lsc{acc}}\morglo{ruwa-:}{make-\lsc{1}}\morglo{mincha-n-ta}{day.after.tomorrow-\lsc{3}-\lsc{acc}}\morglo{huk-ta}{one-\lsc{acc}}}%morpheme+gloss
\glotran{I’ll spin tomorrow and make \pb{one}; the day after tomorrow, \pb{another}.}{}%eng+spa trans
{}{}%rec - time

% 6
\gloexe{Glo3:Ayvis}{}{ach}%
{Ayvis lliw chinkarun ayvis \pb{huk}lla ishkayllata tariru:.}%ach que first line
{\morglo{ayvis}{sometimes}\morglo{lliw}{all}\morglo{chinka-ru-n}{lose-\lsc{urgt}-\lsc{3}}\morglo{ayvis}{sometimes}\morglo{huk-lla}{one-\lsc{rstr}}\morglo{ishkay-lla-ta}{two-\lsc{rstr}-\lsc{acc}}\morglo{tari-ru-:}{find-\lsc{urgt}-\lsc{1}}}%morpheme+gloss
\glotran{Sometimes all get lost; sometimes I find just \pb{one} or two.}{}%eng+spa trans
{}{}%rec - time

\noindent
With ‘another’ interpretation, \phono{huk} may be inflected with plural \phono{-kuna}~(\ref{Glo3:Kikiypaqruwani}).\\

% 7
\gloexe{Glo3:Kikiypaqruwani}{}{amv}%
{Kikiypaq ruwani \pb{hukkuna}paq ruwani.}%amv que first line
{\morglo{kiki-y-paq}{self-\lsc{1}-\lsc{ben}}\morglo{ruwa-ni}{make-\lsc{1}}\morglo{huk-kuna-paq}{one-\lsc{pl}-\lsc{ben}}\morglo{ruwa-ni}{make-\lsc{1}}}%morpheme+gloss
\glotran{I make them for myself and I make them for \pb{others}.}{}%eng+spa trans
{}{}%rec - time

\noindent
Suffixed with allative/dative \phono{-man}, it may be interpreted ‘different’ or ‘differently’~(\ref{Glo3:Waytachaypis}).\\

% 8
\gloexe{Glo3:Waytachaypis}{}{amv}%
{Waytachaypis \pb{hukman} lluqsiruwan ishkay trakiyuqhina lluqsirun.}%amv que first line
{\morglo{wayta-cha-y-pis}{flower-\lsc{dim}-\lsc{1}-\lsc{add}}\morglo{huk-man}{one-\lsc{all}}\morglo{lluqsi-ru-wa-n}{come.out-\lsc{urgt}-\lsc{1.obj}-\lsc{3}}\morglo{ishkay}{two}\morglo{traki-yuq-hina}{foot-\lsc{poss}-\lsc{comp}}\morglo{lluqsi-ru-n}{come.out-\lsc{urgt}-\lsc{3}}}%morpheme+gloss
\glotran{My flower came out \pb{differently} on me. It came out like with two feet.}{}%eng+spa trans
{}{}%rec - time

\subsection{Multiple-class substantives}\label{sec:mcsub}
Some substantives are ambivalent. Regular nouns may appear as regular modifiers~(\ref{mcs:1}) and adverbial adjectives~(\ref{mcs:2}); interrogative pronouns as indefinite and relative pronouns~(\ref{mcs:3}); dependent pronouns as unit numerals~(\ref{mcs:4}); unit numerals as pronouns~(\ref{mcs:5}),~(\ref{mcs:6}); and dependent pronouns as adverbs~(\ref{mcs:7}) and quantitative~(\ref{mcs:8}) adjectives. Table~\ref{TabMCS} gives some examples.

% Table
\newcounter{mulclasub}\setcounter{mulclasub}{0}%
\newcommand{\mcstab}[3]{\refstepcounter{mulclasub}(\themulclasub)\label{mcs:#1} &\Qyell{\phono{#2}}& #3 \\}%
\begin{table}
\small\centering
\caption{Multiple-class substantives}\label{TabMCS}
\begin{tabular}{lll}
\lsptoprule
\mcstab{1}{mishki}{‘a sweet’, ‘sweet’}
\mcstab{2}{tardi}{‘afternoon’, ‘late’}
\mcstab{3}{ima}{‘thing’, ‘what’, ‘that’}
\mcstab{4}{sapa}{‘each’ ‘one alone’}
\mcstab{5}{huk}{‘one’, ‘I’}
\mcstab{6}{ishkay}{‘two[stones]’ ‘two[came]’}
\mcstab{7}{kuska}{‘we/you/they together’ ‘together’}
\mcstab{8}{llapa}{‘all of us/you/them’ ‘all’}
\lspbottomrule
\end{tabular}
\end{table}

\subsection{Dummy \phono{na}}\label{sec:dummyna}
\phono{na} is a dummy noun\index[sub]{dummy noun}, standing in for any substantive that doesn’t make it off the tip of the speaker’s tongue~(\ref{Glo3:Waknalawku}),~(\ref{Glo3:Wanqakunchik}).\\

% 1
\gloexe{Glo3:Waknalawku}{}{ach}%
{Wak \pb{na} lawkunapa Wañupisa. Yanak lawkunapatr.}%ach que first line
{\morglo{wak}{\lsc{dem.d}}\morglo{na}{\lsc{dmy}}\morglo{law-kuna-pa}{side-\lsc{pl}-\lsc{loc}}\morglo{Wañupisa}{Wañupisa}\morglo{Yanak}{Yanak}\morglo{law-kuna-pa-tr}{side-\lsc{pl}-\lsc{loc}-\lsc{evc}}}%morpheme+gloss
\glotran{Around that \pb{what-is-it} -- Wañupisa. Around Yanak, for sure.}{}%eng+spa trans
{}{}%rec - time

% 2
\gloexe{Glo3:Wanqakunchik}{}{ch}%
{Wanqakunchik \pb{na}kta papaktapis uqaktapis. Walmi.}%ch que first line
{\morglo{wanqa-ku-nchik}{turn-\lsc{refl}-\lsc{1pl}}\morglo{na-kta}{\lsc{dmy}-\lsc{acc}}\morglo{papa-kta-pis}{potato-\lsc{acc}-\lsc{add}}\morglo{uqa-kta-pis}{oca-\lsc{acc}-\lsc{add}}\morglo{walmi}{woman}}%morpheme+gloss
\glotran{We turn the \pb{what-do-you-call-them} -- the potatoes, the oca. [We] women.}{}%eng+spa trans
{}{}%rec - time

\noindent
\phono{na} inflects as does any other substantive --~for case~(\ref{Glo3:Waknatatr}), number, and possession~(\ref{Glo3:Waqayan}).\\

% 3
\gloexe{Glo3:Waknatatr}{}{amv}%
{Wak \pb{na}tatr qawanqa hinashpatr rimanqa.}%
{\morglo{wak}{\lsc{dem.d}}\morglo{na-ta-tr}{\lsc{dmy}-\lsc{acc}-\lsc{evc}}\morglo{qawa-nqa}{see-\lsc{3.fut}}\morglo{hinashpa-tr}{then-\lsc{evc}}\morglo{rima-nqa}{talk-\lsc{3.fut}}}%morpheme+gloss
\glotran{She’s going to look at that \pb{thingamajig}, then she’ll talk.}%eng
{‘Va a ver su cosita esa y después va a hablar’.}%spa
{}{}%rec - time

% 4
\gloexe{Glo3:Waqayan}{}{amv}%
{Waqayan. Uray lawpa apamunki chay \pb{na}nta.}%amv que first line
{\morglo{waqa-ya-n}{cry-\lsc{prog}-\lsc{3}}\morglo{uray}{down.hill}\morglo{law-pa}{side-\lsc{loc}}\morglo{apa-mu-nki}{bring-\lsc{cisl}-\lsc{2}}\morglo{chay}{\lsc{dem.d}}\morglo{na-n-ta}{\lsc{dmy}-\lsc{3}-\lsc{acc}}}%morpheme+gloss
\glotran{He’s crying. Bring his \pb{thingy} down there!}{}%eng+spa trans
{}{}%rec - time

\noindent
\phono{na} is ambivalent, serving also as a dummy verb~(\ref{Glo3:Chaykunarima}).\\

% 5
\gloexe{Glo3:Chaykunarima}{}{amv}%
{Chaykuna rimanqaña \pb{na}rushpaqa.}%amv que first line
{\morglo{chay-kuna}{\lsc{dem.d}-\lsc{pl}}\morglo{rima-nqa-ña}{talk-\lsc{3.fut}-\lsc{disc}}\morglo{na-ru-shpa-qa}{\lsc{dmy}-\lsc{urgt}-\lsc{subis}-\lsc{top}}}%morpheme+gloss
\glotran{They’ll talk after \pb{doing that}.}{}%eng+spa trans
{}{}%rec - time

\section{Substantive inflection}
Substantives in \SYQ, as in other Quechuan languages, inflect for person, number and case.\index[sub]{substantive!inflection} This introduction summarizes the more extended discussion to follow. 

The substantive (“possessive”) person suffixes of \SYQ{} are \phono{-y} (\AMV, \LT) or \phono{-:} (\ACH, \CH, \SP) (1\lsc{p}), \phono{-Yki} (2\lsc{p}), \phono{-n} (3\lsc{p}), and \phono{-nchik} (1\lsc{pl}) (\phono{mishi\pb{-y}}, \phono{mishi-\pb{:}} ‘my cat’; \phono{asnu-\pb{yki}} ‘your donkey’). Table~\ref{Tab10} below displays this paradigm.

The plural suffix of \SYQ{} is \phono{-kuna} (\phono{urqu-\pb{kuna}} ‘hills’).

\SYQ{} has ten case suffixes: comparative \phono{-hina} (\phono{María-hina} ‘like María’); limitative \phono{-kama} (\phono{marsu-kama} ‘until March’); allative, dative \phono{-man} (\phono{Cañete-man} ‘to Cañete’); genitive and locative \phono{-pa} (\phono{María-pa} ‘María’s’ \phono{Lima-pa} ‘in Lima’); ablative, benefactive, and purposive \phono{-paq} (\phono{Viñac-paq} ‘from Viñac’, \phono{María-paq} ‘for María,’ \phono{qawa-na-n-paq} ‘in order for her to see’); locative \phono{-pi} (\phono{Lima-pi} ‘in Lima’); exclusive \phono{-puRa} (\phono{amiga-pura} ‘among friends’); causative \phono{-rayku} (\phono{María-rayku} ‘on account of María’); accusative \phono{-ta} (\phono{María-ta} ‘María’ (direct object)), and comitative and instrumental \phono{-wan} (\phono{María-wan} ‘with María’, \phono{acha-wan} ‘with an axe’). Table~\ref{Tab11} below displays this paradigm.

All case marking attaches to the last word in the nominal phrase. When a stem bears suffixes of two or three classes, these appear in the order person-number-case~(\ref{Glo3:Blusalla}),~(\ref{Glo3:Kusasni}).\\

% 1
\gloexe{Glo3:Blusalla}{}{amv}%
{¡Blusalla\pb{ykunata} kayllaman warkurapuway!}%
{\morglo{blusa-lla-y-kuna-ta}{blusa-\lsc{rstr}-1-\lsc{pl}-\lsc{acc}}\morglo{kay-lla-man}{\lsc{dem.p}-\lsc{rstr}-\lsc{all}}\morglo{warku-ra-pu-wa-y}{hang-\lsc{urgt}-\lsc{ben}-1.\lsc{obj}-\lsc{imp}}}%morpheme+gloss
\glotran{Hang just \pb{my} blouse\pb{s} up just over there for me!}%eng
{‘¡Cuélgame mis blusas nada más hacia allá!’}%spa
{}{}%rec - time

% 2
\gloexe{Glo3:Kusasni}{}{amv}%
{Kusasni\pb{nchikkunallatatr} ñitinman.}%amv que first line
{\morglo{kusas-ni-nchik-kuna-lla-ta-tr}{things-\lsc{euph}-\lsc{1pl}-\lsc{pl}-\lsc{rstr}-\lsc{acc}-\lsc{evc}}\morglo{ñiti-n-man}{crush-\lsc{3}-\lsc{cond}}}%morpheme+gloss
\glotran{Just \pb{our} thing\pb{s} would crush.}{}%eng+spa trans
{}{}%rec - time

Sections \sectref{ssec:alloP}--\ref{ssec:case} cover inflection for possession, number and case, respectively. Most case suffixes are mutually exclusive; \sectref{ssec:limkama} gives some possible combinations.

\subsection{Possessive (person)}\label{ssec:alloP}
The possessive\index[sub]{substantive!possessive} suffixes of \SYQ{} are the same in all dialects for all persons except the first-person singular. Two of the five dialects --~\AMV{} and \LT~-- follow the \QII{} pattern, marking the first-person singular with \phono{-y}; three dialects --~\ACH, \CH, and \SP~-- follow the \QI{} pattern, marking it with \phono{-:} (vowel length). The \SYQ{} nominal suffixes, then, are: \phono{-y} or \phono{-:} (1\lsc{p}), \phono{-Yki} (2\lsc{p}), \phono{-n} (3\lsc{p}), \phono{-nchik} (1\lsc{pl}). Table~\ref{Tab10} lists the possessive suffixes.

% Table 10
\begin{table}
\small\centering
\caption{Possessive (substantive) suffixes}\label{Tab10}
\begin{tabular}{lll}
\lsptoprule
Person & Singular & Plural\\
\midrule
\multirow{3}{*}{1} 	& -y (\AMV, \LT) 		& -nchik (dual, inclusive) 		\\
	& -: (\ACH, \CH, \SP)	& -y (exclusive \AMV, \LT)		\\
	& 						& -: (exclusive \ACH, \CH, \SP)		\\
%\midrule
2 	& -Yki 					& -Yki 		\\
%\midrule
3 	& -n 					& -n		\\
\lspbottomrule
\end{tabular}
\end{table}

\noindent
Stems of the following substantive classes may be suffixed with person suffixes: nouns (\phono{wambra\pb{-yki}} ‘your child’)~(\ref{Glo3:Hinashpaqa}), general numerals (\phono{kimsa\pb{-nchik}} ‘the three of us’)~(\ref{Glo3:Kananqa}), dependent pronouns (\phono{kiki-\pb{n}} ‘she herself’)~(\ref{Glo3:Paysapallan}), demonstrative pronouns (\phono{chay-ni-\pb{y}} ‘this of mine’)~(\ref{Glo3:Chaynikita}) and interrogative-indefinites~(\ref{Glo3:mayqinniypis}).\\

% 1
\gloexe{Glo3:Hinashpaqa}{}{amv}%
{Hinashpaqa pubriqa kutimusa llapa animalni\pb{n}wan wasi\pb{n}man.}%amv que first line
{\morglo{hinashpa-qa}{then-\lsc{top}}\morglo{pubri-qa}{poor-\lsc{top}}\morglo{kuti-mu-sa}{return-\lsc{cisl}-\lsc{npst}}\morglo{llapa}{all}\morglo{animal-ni-n-wan}{animal-\lsc{euph}-\lsc{3}-\lsc{instr}}\morglo{wasi-n-man}{house-\lsc{3}-\lsc{acc}}}%morpheme+gloss
\glotran{Then the poor man returned to \pb{his} house with all \pb{his} animals.}{}%eng+spa trans
{}{}%rec - time

% 2
\gloexe{Glo3:Kananqa}{}{sp}%
{“Kananqa aysashun kay sugawan”, nishpa \pb{ishkaynin} aysapa:kun sanqaman.}%sp que first line
{\morglo{kanan-qa}{now-\lsc{top}}\morglo{aysa-shun}{pull-\lsc{1pl}}\morglo{kay}{\lsc{dem.p}}\morglo{suga-wan}{rope-\lsc{instr}}\morglo{ni-shpa}{say-\lsc{subis}}\morglo{ishkay-ni-n}{two-\lsc{euph}-\lsc{3}}\morglo{aysa-pa:-ku-n}{pull-\lsc{jtacc}-\lsc{3}}\morglo{sanqa-man}{ravine-\lsc{all}}}%morpheme+gloss
\glotran{“Now we’ll pull with this rope,” he said and \pb{the two of them} pulled it toward the ravine.}{}%eng+spa trans
{}{}%rec - time

% 3
\gloexe{Glo3:Paysapallan}{}{amv}%
{Pay \pb{sapallan} hamuyan kay llaqtataqa.}%amv que first line
{\morglo{pay}{she}\morglo{sapa-lla-n}{alone-\lsc{rstr}-\lsc{3}}\morglo{hamu-ya-n}{come-\lsc{prog}-\lsc{3}}\morglo{kay}{\lsc{dem.p}}\morglo{llaqta-ta-qa}{town-\lsc{acc}-\lsc{top}}}%morpheme+gloss
\glotran{She’s coming to this town \pb{all alone}.}{}%eng+spa trans
{}{}%rec - time

% 4
\gloexe{Glo3:Chaynikita}{}{amv}%
{\pb{Chaynikita} pristawanki.}%amv que first line
{\morglo{chay-ni-ki-ta}{\lsc{dem.d}-\lsc{euph}-\lsc{2}-\lsc{acc}}\morglo{prista-wa-nki}{lend-\lsc{1.obj}-\lsc{2}}}%morpheme+gloss
\glotran{Lend me that [thing] \pb{of yours}.}{}%eng+spa trans
{}{}%rec - time

% 5
\gloexe{Glo3:mayqinniypis}{}{amv}%
{Manam \pb{mayqinniypis} wañuniraqchu.}%amv que first line
{\morglo{mana-m}{no-\lsc{evd}}\morglo{mayqin-ni-y-pis}{which-\lsc{euph}-\lsc{1}-\lsc{add}}\morglo{wañu-ni-raq-chu}{die-\lsc{1}-\lsc{cont}-\lsc{neg}}}%morpheme+gloss
\glotran{\pb{None of us} has died yet.}{}%eng+spa trans
{}{}%rec - time


\noindent
In the case of words ending in a consonant, \phono{-ni} --~semantically vacuous~-- precedes the person suffix~(\ref{Glo3:Maynintapis}).\\

% 6
\gloexe{Glo3:Maynintapis}{}{amv}%
{¿\pb{Maynintapis} ripunqañatr? Gallu Rumi altuntapis ripunqañatr.}%amv que first line
{\morglo{may-ni-n-ta-pis}{where-\lsc{euph}-\lsc{3}-\lsc{acc}-\lsc{add}}\morglo{ripu-nqa-ña-tr}{go-\lsc{3.fut}-\lsc{disc}-\lsc{evc}}\morglo{Gallu}{Cock}\morglo{Rumi}{Rock}\morglo{altu-n-ta-pis}{high-\lsc{3}-\lsc{acc}-\lsc{add}}\morglo{ripu-nqa-ña-tr}{go-\lsc{3.fut}-\lsc{disc}-\lsc{evc}}}%morpheme+gloss
\glotran{\pb{Where abouts} will he go? He’ll go up above Gallu Rumi, for sure.}{}%eng+spa trans
{}{}%rec - time

\noindent
The third person possessive suffix, \phono{-n}, attaching to \phono{may} ‘where’ and other expressions of place, forms an idiomatic expression interpretable as ‘via’ or ‘around’ (\ref{Glo3:Hamuyaq}).\\

% 7
\gloexe{Glo3:Hamuyaq}{}{amv}%
{Hamuyaq \pb{kayninta}.}%amv que first line
{\morglo{hamu-ya-q}{come-\lsc{prog}-\lsc{ag}}\morglo{kay-ni-n-ta}{\lsc{dem.p}-\lsc{euph}-\lsc{3}-\lsc{acc}}}%morpheme+gloss
\glotran{He used to be coming \pb{around here}.}{}%eng+spa trans
{}{}%rec - time

\noindent
In the first person singular, the noun \phono{papa} ‘father’ inflects \phono{papa-ni-y} to refer to one’s biological or social father~, (\ref{Glo3:Vikunachayta}).\footnote{An anonymous reviewer writes, “As a loan word, most Central Quechuan languages have \emph{papa:} with final vowel length (reinterpretation of final accent in Spanish ‘\spanish{papá}’). As such, \phono{-ni} is required before a syllable-closing suffix, such as \phono{-y}. Though \phono{papa} does not end in a long vowel in SYQ, it probably did at one time, and the effect is retained.”}\\

% 8
\gloexe{Glo3:Vikunachayta}{}{amv}%
{Vikuñachayta diharuni \pb{papaniy}wan.}%amv que first line
{\morglo{vikuña-cha-y-ta}{vicuña-\lsc{dim}-\lsc{1}-\lsc{acc}}\morglo{diha-ru-ni}{leave-\lsc{urgt}-\lsc{1}}\morglo{papa-ni-y-wan}{father-\lsc{euph}-\lsc{1}-\lsc{instr}}}%morpheme+gloss
\glotran{I left my little vicuña with \pb{my father}.}{}%eng+spa trans
{}{}%rec - time

\noindent
\SYQ{} possessive constructions are formed \lsc{substantive}-\lsc{poss} \phono{ka-} (\pb{allqu-n ka-rqa} ‘she had a dog’ (lit. ‘her dog was’))~(\ref{Glo3:wambrayki})(\ref{Glo3:Yasqayaruptiki}).\\

% 9
\gloexe{Glo3:wambrayki}{}{ach}%
{Mana \pb{wambrayki} \pb{kan}chu mana \pb{qariyki} \pb{kan}chu.}%ach que first line
{\morglo{mana}{no}\morglo{wambra-yki}{child-\lsc{2}}\morglo{ka-n-chu}{	be-\lsc{3}-\lsc{neg}}\morglo{mana}{no}\morglo{qari-yki}{man-\lsc{2}}\morglo{ka-n-chu}{be-\lsc{3}-\lsc{neg}}}%morpheme+gloss
\glotran{\pb{You} don’t \pb{have children} and \pb{you} don’t \pb{have a husband}.}{}%eng+spa trans
{}{}%rec - time

\noindent
Finally, possessive suffixes attach to the subordinating suffix \phono{-pti} as well as to the nominalizing suffixes \phono{-na} and \phono{-sa} to form subordinate~(\ref{Glo3:Yasqayaruptiki}), purposive~(\ref{Glo3:Hampikunaykipaq}), complement~(\ref{Glo3:Atipasantatriki}), and relative~(\ref{Glo3:Chaywawqin}), (\ref{Glo3:Truraykun}) clauses.\\

% 10
\gloexe{Glo3:Yasqayaruptiki}{}{ach}%
{\pb{Yasqayaruptiki} mana pinikipis kanqachu.}%ach que first line
{\morglo{yasqa-ya-ru-pti-ki}{old-\lsc{inch}-\lsc{urgt}-\lsc{subds}-\lsc{2}}\morglo{mana}{no}\morglo{pi-ni-ki-pis}{who-\lsc{euph}-\lsc{2}-\lsc{add}}\morglo{ka-nqa-chu}{be-\lsc{3.fut}-\lsc{neg}}}%morpheme+gloss
\glotran{\pb{When you’re old}, you won’t have anyone.}{}%eng+spa trans
{}{}%rec - time

% 11
\gloexe{Glo3:Hampikunaykipaq}{}{amv}%
{\pb{Hampikunaykipaq} yatranki.}%amv que first line
{\morglo{hampi-ku-na-yki-paq}{cure-\lsc{refl}-\lsc{nmlz}-\lsc{2}-\lsc{purp}}\morglo{yatra-nki}{know-\lsc{2}}}%morpheme+gloss
\glotran{You’ll learn \pb{so that you can cure}.}{}%eng+spa trans
{}{}%rec - time

% 12
\gloexe{Glo3:Atipasantatriki}{}{ach}%
{\pb{Atipasantatriki} ruwan.}%ach que first line
{\morglo{atipa-sa-n-ta-tri-ki}{be.able-\lsc{prf}-\lsc{3}-\lsc{acc}-\lsc{evc}-\lsc{ki}}\morglo{ruwa-n}{make-\lsc{3}}}%morpheme+gloss
\glotran{They do \pb{what they can}.}{}%eng+spa trans
{}{}%rec - time

% 13
\gloexe{Glo3:Chaywawqin}{}{ach}%
{Chay wawqin ama \pb{nisantas} wañuchisataq.}%ach que first line
{\morglo{chay}{\lsc{dem.d}}\morglo{wawqi-n}{brother-\lsc{3}}\morglo{ama}{\lsc{proh}}\morglo{ni-sa-n-ta-s}{say-\lsc{prf}-\lsc{3}-\lsc{acc}-\lsc{add}}\morglo{wañu-chi-sa-taq}{die-\lsc{caus}-\lsc{npst}-\lsc{seq}}}%morpheme+gloss
\glotran{They also killed his brother \pb{who said} “No!”}{}%eng+spa trans
{}{}%rec - time

% 14
\gloexe{Glo3:Truraykun}{}{amv}%
{Truraykun frutachankunata -- llapa \pb{gustasan}.}%
{\morglo{trura-yku-n}{save-\lsc{excep}-3}\morglo{fruta-cha-n-kuna-ta}{fruit-\lsc{dim}-3-\lsc{pl}-\lsc{acc}}\morglo{llapa}{all}\morglo{gusta-sa-n}{like-\lsc{prf}-3}}%morpheme+gloss
\glotran{They put out their fruit and all -- everything \pb{they liked}.}%eng
{‘Ponen su fruta y todo -- todo lo que les gustaba’.}%spa
{}{}%rec - time

\subsection{Number \phono{-kuna}}
\phono{-kuna} pluralizes\index[sub]{substantive!number inflection} regular nouns, as in~(\ref{Glo3:Kabrakuna}), where it affixes to \phono{kabra} ‘goat’ to form \phono{kabra-\pb{kuna}} ‘goats’.

% 1
\gloexe{Glo3:Kabrakuna}{}{amv}%
{\pb{Kabrakuna}ta hapishpa mikukuyan.}%amv que first line
{\morglo{kabra-kuna-ta}{goat-\lsc{pl}-\lsc{acc}}\morglo{hapi-shpa}{grab-\lsc{subis}}\morglo{miku-ku-ya-n}{eat-\lsc{refl}-\lsc{prog}-\lsc{3}}}%morpheme+gloss
\glotran{Taking ahold of the \pb{goats}, [the puma] is eating them.}{}%eng+spa trans
{}{}%rec - time

\noindent
\phono{-kuna} also pluralizes the personal pronouns \phono{ñuqa}, \phono{qam}, and \phono{pay}~(\ref{Glo3:Awanmi}), demonstrative pronouns~(\ref{Glo3:Chaykuna}), and interrogative-indefinites~(\ref{Glo3:Imakuna}).\\

% 2
\gloexe{Glo3:Awanmi}{}{amv}%
{Awanmi \pb{paykuna}pisriki.}%amv que first line
{\morglo{awa-n-mi}{weave-\lsc{3}-\lsc{evd}}\morglo{pay-kuna-pis-r-iki}{he-\lsc{pl}-\lsc{add}-\lsc{r}-\lsc{iki}}}%morpheme+gloss
\glotran{\pb{They}, too, weave.}{}%eng+spa trans
{}{}%rec - time

% 3
\gloexe{Glo3:Chaykuna}{}{amv}%
{\pb{Chaykuna}pa algunusqa pamparayan.}%amv que first line
{\morglo{chay-kuna-pa}{\lsc{dem.d}-\lsc{pl}-\lsc{loc}}\morglo{algunus-qa}{some.people-\lsc{top}}\morglo{pampa-ra-ya-n}{bury-\lsc{unint}-\lsc{intens}-\lsc{3}}}%morpheme+gloss
\glotran{Some people are buried in \pb{those}.}{}%eng+spa trans
{}{}%rec - time

% 4
\gloexe{Glo3:Imakuna}{}{amv}%
{¿\pb{Imakuna}m ubihaykipa sutin?}%amv que first line
{\morglo{ima-kuna-m}{what-\lsc{pl}-\lsc{evd}}\morglo{ubiha-yki-pa}{sheep-\lsc{2}-\lsc{gen}}\morglo{suti-n}{name-\lsc{3}}}%morpheme+gloss
\glotran{\pb{What} are your sheep’s names?}{}%eng+spa trans
{}{}%rec - time

\noindent
\phono{-kuna} follows the stem and possessive suffix, if any, and precedes the case suffix, if any~(\ref{Glo3:Chamisninkuna}).\\

% 5
\gloexe{Glo3:Chamisninkuna}{}{amv}%
{\pb{Chamisninkuna}ta upyarin \pb{kukankuna}ta akun.}%amv que first line
{\morglo{chamis-ni-n-kuna-ta}{chamis-\lsc{euph}-\lsc{3}-\lsc{pl}-\lsc{acc}}\morglo{upya-ri-n}{drink-\lsc{incep}-\lsc{3}}\morglo{kuka-n-kuna-ta}{coca-\lsc{3}-\lsc{pl}-\lsc{acc}}\morglo{aku-n}{chew-\lsc{3}}}%morpheme+gloss
\glotran{They drink \pb{their \phono{chamis}} and they chew \pb{their coca}.}{}%eng+spa trans
{}{}%rec - time

\noindent
Number-marking in \SYQ{} is optional. Noun phrases introduced by numerals or quantifying adjectives generally are not inflected with \phono{-kuna}~(\ref{Glo3:Ishkayyatrarqa}).\\

% 6
\gloexe{Glo3:Ishkayyatrarqa}{}{amv}%
{\pb{Ishkay} yatrarqa, \pb{ishkay} warmi.}%amv que first line
{\morglo{ishkay}{two}\morglo{yatra-rqa}{live-\lsc{pst}}\morglo{ishkay}{two}\morglo{warmi}{woman}}%morpheme+gloss
\glotran{\pb{Two} lived [there], \pb{two} women.}{}%eng+spa trans
{}{}%rec - time

\noindent
\phono{-kuna} may receive non-plural interpretations and, like \phono{-ntin}, may indicate accompaniment or non-exhaustivity~(\ref{Glo3:Chaykwirpu}).\footnote{This example is, in fact, ambiguous between as reading in which \phono{-kuna} receives a non-plural interpretation and one in which it simply pluralizes the possessed item. Thus, \phono{kwirpu-y-kuna} could also refer to ‘your (plural) bodies’, as an anonymous reviewer points out.}\\

% 7
\gloexe{Glo3:Chaykwirpu}{}{amv}%
{Chay kwirpu\pb{ykikuna} mal kanman uma\pb{ykikuna} nananman.}%
{\morglo{chay}{\lsc{dem.d}}\morglo{kwirpu-yki-kuna}{body-\lsc{2}-\lsc{pl}}\morglo{mal}{bad}\morglo{ka-n-man}{be-\lsc{3}-\lsc{cond}}\morglo{uma-yki-kuna}{head-\lsc{2}-\pb{pl}}\morglo{nana-n-man}{hurt-\lsc{3}-\lsc{cond}}}%morpheme+gloss
\glotran{\pb{Your whole body} could be not well; \pb{your head and everything} could hurt.}%eng
{‘Tu cuerpo todo puede estar mal; tu cabeza todo puede doler’.}%spa
{}{}%rec - time

\noindent
Finally, words borrowed from Spanish already inflected for plural --~i.e.,~with Spanish plural \textit{s}~-- are generally still suffixed with \phono{-kuna} (\spanish{cosas}~→~\phononb{kusas}\phononb{-ni}\phononb{-nchik}\phononb{-kuna})~(\ref{Glo3:Qayashpa}).\\

% 8
\gloexe{Glo3:Qayashpa}{}{amv}%
{Qayashpa waqashpa purin animal\pb{isninchikuna}qa.}%
{\morglo{qaya-shpa}{scream-\lsc{subis}}\morglo{waqa-shpa-{m}}{cry-\lsc{subis}-\lsc{evd}}\morglo{puri-n}{walk-\lsc{3}}\morglo{animalis-ni-nchik-kuna-qa}{animals-\lsc{euph}-\lsc{1pl}-\lsc{pl}-\lsc{top}}}%morpheme+gloss
\glotran{Our animal\pb{s} walk around screaming, crying.}%eng
{‘Nuestros animales andan gritando, llorando’.}%spa
{}{}%rec - time

\subsection{Case}\label{ssec:case}\index[sub]{substantive!case}

% Table 11
\begin{table}[t]
% \renewcommand*\arraystretch{1.1}
\small\centering
\caption{Case suffixes with examples}\label{Tab11}
\begin{tabularx}{\textwidth}{l@{\hspace{1ex}}l@{\hspace{1ex}}L@{\hspace{1ex}}L}
\lsptoprule
\Qyell{\phono{-hina}} & comparative & \Qyell{Runa-\pb{hina}, uyqa-\pb{hina}} & ‘\pb{Like} people, \pb{like} sheep’	\\
\Qyell{\phono{-kama}} & limitative & \Qyell{Fibriru marsu-\pb{kama}-raq-tri para-nqa.} & ‘It will rain still \pb{until} February or march.’	\\
\Qyell{\phono{-man}} & allative, \mbox{dative}& \Qyell{Lima runa-kuna traya-mu-pti-n siyra-n-\pb{man}.} & ‘When people from Lima return \pb{to} their sierra.’ \\
\Qyell{\phono{-pa\tss{1}}} & genitive & \Qyell{Algunus-\pb{pa} puchka-n tipi-ku-ya-n-mi.} & ‘Some people\pb{’s} thread breaks on them.’	\\
\Qyell{\phono{-pa\tss{2}}} & locative & \Qyell{Urqu-lla-\pb{pa}-m chay-qa wiña-n.} & ‘It grows only \pb{in} the mountains.’	\\
\Qyell{\phono{-pi}} & locative & \Qyell{Yana-ya-sa qutra-pa pata-n-\pb{pi} qutra-pa tuna-n-\pb{pi}.} & ‘Blackened \pb{on} the banks of the lake, \pb{in} the corner of the lake.’	\\
\Qyell{\phono{-paq\tss{1}}} & ablative & \Qyell{Huangáscar-\pb{paq}-mi hamu-ra wama-wamaq polisiya-pis.} & ‘Lots of policemen came \pb{from} Huangáscar.’	\\
\Qyell{\phono{-paq\tss{2}}} & benefactive & \Qyell{Chay qari-kuna mana ishpa-y-ta atipa-q-\pb{paq}.} & ‘This is \pb{for} the men who can’t urinate.’	\\
\Qyell{\phono{-paq\tss{3}}} & purposive & \Qyell{Qawa-na-y-\pb{paq} ima-wan wañu-ru-n~\dots{} kitra-ni.} & ‘\pb{In order to} see what he died from~\dots{} I opened him up.’	\\
\Qyell{\phono{-puRa}} & reciprocal & \Qyell{Qam pay-wan wawqi ñaña-\pb{pura} ka-nki.} & ‘You and she are going to be true brothers and sisters.’	\\
\Qyell{\phono{-rayku}} & reason & \Qyell{Chawa-shi-q lichi-lla-n-rayku ri-y-man-tri.} & ‘I might go help milk \pb{on account} of her milk.’	\\
\Qyell{\phono{-ta}} & accusative & \Qyell{¿Maqta-kuna-\pb{ta} pusha-nki icha pashña-\pb{ta}?} & ‘Are you going to take the boys or the girl?’	\\
\Qyell{\phono{-wan\tss{1}}} & comitative & \Qyell{¿Imapaq-mi wak kundinaw-\pb{wan} puri-ya-nki?} & ‘Why are you walking around \pb{with} that zombie?’	\\
\Qyell{\phono{-wan\tss{2}}} & instrumental & \Qyell{Ichu-\pb{wan}-mi chay-ta ruwa-nchik.} & ‘We make this one \pb{with} straw.’	\\
\midrule
\multicolumn{4}{l}{in Cacra-Hongos dialect only:}\\[1ex]
\phono{-Kta} & \multicolumn{3}{l}{replaces \phono{-ta} to mark accusative}\\
\phono{-traw} & \multicolumn{3}{l}{alternates with \phono{-pa} and \phono{-pi} to mark the locative}\\
\lspbottomrule
\end{tabularx}
\end{table}

A set of ten suffixes constitutes the case system of \SYQ. Table~\ref{Tab11} gives glossed examples. These are: \phono{-hina} (comparative), \phono{-kama} (limitative), \phono{-man} (allative, dative), \phono{-pa/-pi} (genitive, locative), \phono{-paq} (ablative, benefactive, purposive), \phono{-puRa} (exclusive), \phono{-rayku} (reason), \phono{-ta} (accusative), and \phono{-wan} (comitative, instrumental). Genitive, instrumental and allative/dative may specify noun-verb in addition to noun-noun relations. \phono{-pa} is the default form for the locative, but \phono{-pi} is often and \phono{-paq} is sometimes used. The \CH{} dialect uses a fourth form, \phono{-traw}, common to the \QI{} languages. The \CH{} dialect is also unique among the five in its realization of accusative \phono{-ta} as \phono{-kta} after a short vowel. \phono{-puRa} --~attested only in Viñac~-- and \phono{-rayku} are employed only rarely. The genitive and accusative may form adverbs (\phono{tuta-pa} ‘at night’, \phono{allin-ta} ‘well’). Instrumental \phono{-wan} may coordinate \lsc{np}s (\phono{llama-wan} \phono{alpaka-wan} ‘the llama and the alpaca’). All case processes consist in adding a suffix to the last word in the nominal group. Most case suffixes are mutually exclusive. \sectref{sssec:simhina}--\ref{sssec:msnnr} cover each of the case suffixes in turn.
 
\subsubsection{Simulative \phono{-hina}}\label{sssec:simhina}
The simulative \phono{-hina}\index[sub]{simulative} generally indicates resemblance or comparison\index[sub]{comparative} (\phono{yawar-\pb{hi-}} \phono{\pb{na}} ‘like blood’)~(\ref{Glo3:Nawilla}--~\ref{Glo3:kunachatam}).\\

% 1
\gloexe{Glo3:Nawilla}{}{ach}%
{Ñawilla: pukayarura tutal puka. Yawar\pb{hina} ñawi: kara.}%ach que first line
{\morglo{ñawi-lla-:}{eye-\lsc{rstr}-\lsc{1}}\morglo{puka-ya-ru-ra}{red-\lsc{inch}-\lsc{urgt}-\lsc{pst}}\morglo{total}{completely}\morglo{puka}{red}\morglo{yawar-hina}{blood-\lsc{comp}}\morglo{ñawi-:}{eye-\lsc{1}}\morglo{ka-ra}{be-\lsc{pst}}}%morpheme+gloss
\glotran{My eyes turned red, totally red. My eyes were \pb{like} blood.}{}%eng+spa trans
{}{}%rec - time

% 2
\gloexe{Glo3:Karsilpa}{}{ach}%
{Karsilpa\pb{hina}m. Witrqamara wambra:kuna istudyaq pasan.}%ach que first line
{\morglo{karsil-pa-hina-m}{prison-\lsc{loc}-\lsc{comp}}\morglo{witrqa-ma-ra}{close.in-\lsc{1.obj}-\lsc{pst}}\morglo{wambra-:-kuna}{child-\lsc{1}-\lsc{pl}}\morglo{istudya-q}{study-\lsc{ag}}\morglo{pasa-n}{pass-\lsc{3}}}%morpheme+gloss
\glotran{It was \pb{like} in prison. When my children went to school, they closed me in.}{}%eng+spa trans
{}{}%rec - time

% 3
\gloexe{Glo3:Trakin}{}{lt}%
{Trakin, ishkaynin trakin kayan maniyasha\pb{hina}.}%lt que first line
{\morglo{traki-n,}{foot-\lsc{3}}\morglo{ishkay-ni-n}{two-\lsc{euph}-\lsc{3}}\morglo{traki-n}{foot-\lsc{3}}\morglo{ka-ya-n}{be-\lsc{prog}-\lsc{3}}\morglo{maniya-sha-hina}{bind.feet-\lsc{prf}-\lsc{comp}}}%morpheme+gloss
\glotran{His feet, it’s \pb{like} both are shackled.}{}%eng+spa trans
{}{}%rec - time

% 4
\gloexe{Glo3:purikuni}{}{lt}%
{Wak\pb{hina}llam purikuni. ¿Imanashaqmi?}%lt que first line
{\morglo{wak-hina-lla-m}{\lsc{dem.d}-\lsc{comp}-\lsc{rstr}-\lsc{evd}}\morglo{puri-ku-ni}{walk-\lsc{refl}-\lsc{1}}\morglo{ima-na-shaq-mi}{what-\lsc{vrbz}-\lsc{1.fut}-\lsc{evd}}}%morpheme+gloss
\glotran{Just \pb{like} that I go about. What am I going to do?}{}%eng+spa trans
{}{}%rec - time

% 5
\gloexe{Glo3:Hukrumikayan}{}{sp}%
{Huk rumi kayan warmi\pb{hina}. Chaypish inkantara unay unay.}%sp que first line
{\morglo{huk}{one}\morglo{rumi}{stone}\morglo{ka-ya-n}{be-\lsc{prog}-\lsc{3}}\morglo{warmi-hina}{woman-\lsc{comp}}\morglo{chay-pi-sh}{\lsc{dem.d}-\lsc{loc}-\lsc{evr}}\morglo{inkanta-ra}{enchant-\lsc{pst}}\morglo{unay}{before}\morglo{unay}{before}}%morpheme+gloss
\glotran{There’s a stone \pb{like} [in the form of] a woman. A long, long time ago, it bewitched [people] there, they say.}{}%eng+spa trans
{}{}%rec - time

% 6
\gloexe{Glo3:Tutakuna}{}{amv}%
{Tutakuna puriyan qarqarya\pb{hina}.}%
{\morglo{tuta-kuna}{night-\lsc{pl}}\morglo{puri-ya-n}{walk-\lsc{prog}-\lsc{3}}\morglo{qariya-hina}{zombie-\lsc{comp}}}%morpheme+gloss
\glotran{At night, he walks around like a zombie.}%eng
{‘De noche anda como condenado’.}%spa
{}{}%rec - time

% 7
\gloexe{Glo3:kunachatam}{}{amv}%
{Kay\pb{hina}kunachatam (=kay\pb{hina}chakunatam) ruwani.}%amv que first line
{\morglo{kay-hina-kuna-cha-ta-m}{\lsc{dem.p}-\lsc{comp}-\lsc{pl}-\lsc{dim}-\lsc{acc}-\lsc{evd}}\morglo{(=kay-hina-cha-kuna-ta-m)}{\lsc{dem.p}-\lsc{comp}-\lsc{dim}-\lsc{pl}-\lsc{acc}-\lsc{evd}}\morglo{ruwa-ni}{make-\lsc{1}}}%morpheme+gloss
\glotran{I make all of them just \pb{like} this.}{}%eng+spa trans
{}{}%rec - time

\noindent
It can generally be translated ‘like’. In Cacra and sometimes in Hongos, \phono{-mish} is employed in place of \phono{-hina}~(\ref{Glo3:paqwalun}),~(\ref{Glo3:mish}).\\

% 8
\gloexe{Glo3:paqwalun}{}{ch}%
{Kilun paqwalun. Mikuyta atipanchu. Awila\pb{mish}.}%ch que first line
{\morglo{kilu-n}{tooth-\lsc{3}}\morglo{paqwa-lu-n}{finish.off-\lsc{urgt}-\lsc{3}}\morglo{miku-y-ta}{eat-\lsc{inf}-\lsc{acc}}\morglo{atipa-n-chu}{be.able-\lsc{3}-\lsc{neg}}\morglo{awila-mish}{grandmother-\lsc{comp}}}%morpheme+gloss
\glotran{Her teeth finished off. He can’t eat. \pb{Like} an old lady.}{}%eng+spa trans
{}{}%rec - time

% 9
\gloexe{Glo3:mish}{}{ch}%
{¿Ima\pb{mish} wawipaq takin?}%ch que first line
{\morglo{ima-mish}{what-\lsc{comp}}\morglo{wawi-paq}{baby-\lsc{gen}}\morglo{taki-n}{song-\lsc{3}}}%morpheme+gloss
\glotran{What is a baby’s song \pb{like}?}{}%eng+spa trans
{}{}%rec - time


\subsubsection{Limitative \phono{-kama}}\label{ssec:limkama}
The limitative \phono{-kama}\index[sub]{limitative} --~sometimes realized as \phono{kaman}~-- generally indicates a limit in space~(\ref{Glo3:Qatimushaq}),~(\ref{Glo3:wambraykita}) or time~(\ref{Glo3:Fibriru}--\ref{Glo3:Kandawniypis}).\\

% 1
\gloexe{Glo3:Qatimushaq}{}{amv}%
{Qatimushaq vakata kay\pb{kama}.}%amv que first line
{\morglo{qati-mu-shaq}{follow-\lsc{cisl}-\lsc{1.fut}}\morglo{vaka-ta}{cow-\lsc{acc}}\morglo{kay-\pb{kama}}{\lsc{dem.p}-\lsc{lim}}}%morpheme+gloss
\glotran{I’m going to drive the cows \pb{over} here.}{}%eng+spa trans
{}{}%rec - time

% 2
\gloexe{Glo3:wambraykita}{}{lt}%
{Chay wambraykita katrarunki mayurniki\pb{kama} wawqiki\pb{kama}qa.}%lt que first line
{\morglo{chay}{\lsc{dem.d}}\morglo{wambra-yki-ta}{child-\lsc{2}-\lsc{acc}}\morglo{katra-ru-nki}{release-\lsc{urgt}-\lsc{2}}\morglo{mayur-ni-ki-kama}{eldest-\lsc{euph}-\lsc{2}-\lsc{all}}\morglo{wawqi-ki-kama-qa}{brother-\lsc{2}-\lsc{all}-\lsc{top}}}%morpheme+gloss
\glotran{You sent your children \pb{over to} your older brother, \pb{over to} your brother.}{}%eng+spa trans
{}{}%rec - time

% 3
\gloexe{Glo3:Fibriru}{}{amv}%
{Fibriru marsu\pb{kama}raqtri paranqa.}%
{\morglo{fibriru}{February}\morglo{marsu-kama-raq-tri}{March-\lsc{lim}-\lsc{cont}-\lsc{evc}}\morglo{para-nqa}{rain-\lsc{3.fut}}}%morpheme+gloss
\glotran{It will rain still until February or March.}%eng
{‘Lloverá todavía hasta febrero o marzo’.}%spa
{}{}%rec - time

% 4
\gloexe{Glo3:Imaykama}{}{amv}%
{¿Imay\pb{kama} kanki?}%amv que first line
{\morglo{imay-kama}{when-\lsc{lim}}\morglo{ka-nki}{be-\lsc{2}}}%morpheme+gloss
\glotran{\pb{Until} when are you going to be (here)?}{}%eng+spa trans
{}{}%rec - time

% 5
\gloexe{Glo3:Kandawniypis}{}{lt}%
{Kandawniypis warkurayan altupam. Manam kanan\pb{kama}pis trurachinichu.}%lt que first line
{\morglo{kandaw-ni-y-pis}{padlock-\lsc{euph}-\lsc{1}-\lsc{add}}\morglo{warku-raya-n}{hang-\lsc{pass}-\lsc{3}}\morglo{altu-pa-m}{high-\lsc{loc}-\lsc{evd}}\morglo{mana-m}{no-\lsc{evd}}\morglo{kanan-kama-pis}{now-\lsc{lim}-\lsc{add}}\morglo{trura-chi-ni-chu}{put-\lsc{caus}-\lsc{1}-\lsc{neg}}}%morpheme+gloss
\glotran{My padlock, too, is hung up there. \pb{Until} now I haven’t had it put on.}{}%eng+spa trans
{}{}%rec - time

\noindent
In case time is delimited by an event, the usual structure is \lsc{stem}-\lsc{nmlz}-\lsc{poss}\phono{-kama} (\phono{puri-na-yki-\pb{kama}} (‘so you can walk’)~(\ref{Glo3:Trakipaltanchikpis}),~(\ref{Glo3:Apuraw}).\\

% 6
\gloexe{Glo3:Trakipaltanchikpis}{}{amv}%
{Traki paltanchikpis pushllunan\pb{kama} purinchik. Trakipis ampulla hatarinan\pb{kaman} rirqani.}%amv que first line
{\morglo{traki}{foot}\morglo{palta-nchik-pis}{sole-\lsc{1pl}-\lsc{add}}\morglo{pushllu-na-n-kama}{blister-\lsc{nmlz}-\lsc{3}-\lsc{all}}\morglo{puri-nchik}{walk-\lsc{1pl}}\morglo{traki-pis}{foot-\lsc{add}}\morglo{ampulla}{blister}\morglo{hatari-na-n-kaman}{get.up-\lsc{nmlz}-\lsc{3}-\lsc{all}}\morglo{ri-rqa-ni}{go-\lsc{pst}-\lsc{1}}}%morpheme+gloss
\glotran{We walked \pb{while} blisters formed on the souls of our feet. I went \pb{while} blisters came up on my feet.}{}%eng+spa trans
{}{}%rec - time

% 7
\gloexe{Glo3:Apuraw}{}{sp}%
{Apuraw mikunan\pb{kama} turuqa kayna tuksirikusa.}%sp que first line
{\morglo{apuraw}{quickly}\morglo{miku-na-n-kama}{eat-\lsc{nmlz}-\lsc{3}-\lsc{all}}\morglo{turu-qa}{bull-\lsc{top}}\morglo{kayna}{thus}\morglo{tuksi-ri-ku-sa}{prick-\lsc{incep}-\lsc{refl}-\lsc{npst}}}%morpheme+gloss
\glotran{\pb{Until} the bull ate quickly, she pricked him like this.}{}%eng+spa trans
{}{}%rec - time

\noindent
\phono{-kama} can appear simultaneously with \phono{asta} (\Sp~\spanish{hasta} ‘up to’, ‘until’)~(\ref{Glo3:SanJeronimopaq}).\\

% 8
\gloexe{Glo3:SanJeronimopaq}{}{amv}%
{San Jerónimopaq \pb{asta} kay\pb{kama}.}%amv que first line
{\morglo{San}{San}\morglo{Jerónimo-paq}{Jerónimo-\lsc{abl}}\morglo{asta}{until}\morglo{kay-kama}{\lsc{dem.p-\lsc{all}}}}%morpheme+gloss
\glotran{From San Jerónino \pb{to} here.}{}%eng+spa trans
{}{}%rec - time

\noindent
\phono{-kama} can form distributive expressions: in this case, \phono{-kama} attaches to the quality or characteristic that is distributed~(\ref{Glo3:Unachayuq}),~(\ref{Glo3:Trayaramun}). In case it indicates a limit, \phono{-kama} can usually be translated as ‘up to’ or ‘until’; in case it indicates distribution, it can usually be translated as ‘each’.\\

% 9
\gloexe{Glo3:Unachayuq}{}{amv}%
{Uñachayuq\pb{kama} kayan.}%amv que first line
{\morglo{uña-cha-yuq-kama}{calf-\lsc{dim}-\lsc{poss}-\lsc{all}}\morglo{ka-ya-n}{be-\lsc{prog}-\lsc{3}}}%morpheme+gloss
\glotran{They \pb{all [each]} have their little \pb{young}.}{}%eng+spa trans
{}{}%rec - time

% 10
\gloexe{Glo3:Trayaramun}{}{ach}%
{Trayaramun arman qipikusa\pb{kama}. Manchaku:.}%ach que first line
{\morglo{traya-ra-mu-n}{arrive-\lsc{urgt}-\lsc{cisl}-\lsc{3}}\morglo{arma-n}{weapon-\lsc{3}}\morglo{qipi-ku-sa-kama}{carry-\lsc{refl}-\lsc{prf}-\lsc{all}}\morglo{mancha-ku-:}{scare-\lsc{refl}-\lsc{1}}}%morpheme+gloss
\glotran{They arrived \pb{each} carrying weapons. I got scared.}{}%eng+spa trans
{}{}%rec - time

  
\subsubsection{Allative, dative \phono{-man}}
The allative and dative (directional) \phono{-man}\index[sub]{allative} generally indicates movement toward a point~(\ref{Glo3:Qinwalman}),~(\ref{Glo3:Hinashpachaypaq}) or the end-point of movement or action more generally~(\ref{Glo3:Wakwasikuna}),~(\ref{Glo3:Kabrataqaqa}).\\

% 1
\gloexe{Glo3:Qinwalman}{}{amv}%
{Qiñwal\pb{man} trayarachiptiki wañukunman.}%amv que first line
{\morglo{qiñwal-man}{quingual.grove-\lsc{all}}\morglo{traya-ra-chi-pti-ki}{arrive-\lsc{urgt}-\lsc{caus}-\lsc{subds}-\lsc{2}}\morglo{wañu-ku-n-man}{die-\lsc{refl}-\lsc{3}-\lsc{cond}}}%morpheme+gloss
\glotran{If you make her go \pb{to} the quingual grove, she could die.}{}%eng+spa trans
{}{}%rec - time

% 2
\gloexe{Glo3:Hinashpachaypaq}{}{ach}%
{Hinashpa chaypaq wichay\pb{man} pasachisa chay Amador kaq\pb{man}ñataq.}%ach que first line
{\morglo{hinashpa}{then}\morglo{chay-paq}{\lsc{dem.d}-\lsc{abl}}\morglo{wichay-man}{up.hill-\lsc{all}}\morglo{pasa-chi-sa}{pass-\lsc{caus}-\lsc{npst}}\morglo{chay}{\lsc{dem.d}}\morglo{Amador}{Amador}\morglo{ka-q-man-ña-taq}{be-\lsc{ag}-\lsc{all}-\lsc{disc}-\lsc{seq}}}%morpheme+gloss
\glotran{Then, from there they made them go up high \pb{to} Don Amador’s place.}{}%eng+spa trans
{}{}%rec - time

% 3
\gloexe{Glo3:Wakwasikuna}{}{sp}%
{Wak wasikuna\pb{man}shi yaykurun kundinawqa.}%sp que first line
{\morglo{wak}{\lsc{dem.d}}\morglo{wasi-kuna-man-shi}{house-\lsc{pl}-\lsc{all}-\lsc{evr}}\morglo{yayku-ru-n}{enter-\lsc{urgt}-\lsc{3}}\morglo{kundinaw-qa}{zombie-\lsc{top}}}%morpheme+gloss
\glotran{The zombie entered those houses, they say.}{}%eng+spa trans
{}{}%rec - time

% 4
\gloexe{Glo3:Kabrataqaqa}{}{sp}%
{“¿Kabrata qaqa\pb{man} imapaq qarquranki?” nishpa.}%sp que first line
{\morglo{kabra-ta}{goat-\lsc{acc}}\morglo{qaqa-man}{cliff-\lsc{all}}\morglo{ima-paq}{what-\lsc{purp}}\morglo{qarqu-ra-nki}{toss-\lsc{pst}-\lsc{2}}\morglo{ni-shpa}{say-\lsc{subis}}}%morpheme+gloss
\glotran{“Why did you let the goats loose \pb{onto} the cliff?” he said.}{}%eng+spa trans
{}{}%rec - time

\noindent
It may function as a dative\index[sub]{dative}, indicating a non-geographical goal~(\ref{Glo3:Pashnaqa}),~(\ref{Glo3:Chaylliw}).\\

% 5
\gloexe{Glo3:Pashnaqa}{}{amv}%
{Pashñaqa quykurusa mushuqta watakurusa chumpita wiqawnin\pb{man}.}%amv que first line
{\morglo{pashña-qa}{girl-\lsc{top}}\morglo{qu-yku-ru-sa}{give-\lsc{excep}-\lsc{urgt}-\lsc{npst}}\morglo{mushuq-ta}{new-\lsc{acc}}\morglo{wata-ku-ru-sa}{tie-\lsc{refl}-\lsc{urgt}-\lsc{npst}}\morglo{chumpi-ta}{sash-\lsc{acc}}\morglo{wiqaw-ni-n-man}{waist-\lsc{euph}-\lsc{3}-\lsc{all}}}%morpheme+gloss
\glotran{The girl gave [the young man] a sash, a new one, and she tied it \pb{around} his waist.}{}%eng+spa trans
{}{}%rec - time

% 6
\gloexe{Glo3:Chaylliw}{}{ach}%
{Chay lliw lliw lista\pb{man}shi trurara. Chay lista\pb{man} trurasan rikura.}%ach que first line
{\morglo{chay}{\lsc{dem.d}}\morglo{lliw}{all}\morglo{lliw}{all}\morglo{lista-man-shi}{list-\lsc{all}-\lsc{evr}}\morglo{trura-ra}{put-\lsc{pst}}\morglo{chay}{\lsc{dem.d}}\morglo{lista-man}{list-\lsc{all}}\morglo{trura-sa-n}{put-\lsc{prf}-\lsc{3}}\morglo{riku-ra}{go-\lsc{pst}}}%morpheme+gloss
\glotran{[The Shining Path] put everyone \pb{on} the list. Those who were put \pb{on} the list left.}{}%eng+spa trans
{}{}%rec - time

\noindent
With verbs of giving, it marks the recipient~(\ref{Glo3:Imatataqqunki}),~(\ref{Glo3:Urquman}); with verbs of communication, the person receiving the communication~(\ref{Glo3:Chayshi}),~(\ref{Glo3:Chayllapaq}).\\

% 7
\gloexe{Glo3:Imatataqqunki}{}{amv}%
{¿Imatataq qunki kay pubri\pb{man}?}%amv que first line
{\morglo{ima-ta-taq}{what-\lsc{acc}-\lsc{seq}}\morglo{qu-nki}{give-\lsc{2}}\morglo{kay}{\lsc{dem.p}}\morglo{pubri-man}{poor.person-\lsc{all}}}%morpheme+gloss
\glotran{What are you going to give \pb{to} this poor man?}{}%eng+spa trans
{}{}%rec - time

% 8
\gloexe{Glo3:Urquman}{}{amv}%
{¿Urqu\pb{man} qapishuptiki imatataq qaranki?}%amv que first line
{\morglo{urqu-man}{hill-\lsc{all}}\morglo{qapi-shu-pti-ki}{grab-\lsc{3>1}-\lsc{subds}-\lsc{3>1}}\morglo{ima-ta-taq}{what-\lsc{acc}-\lsc{seq}}\morglo{qara-nki?}{serve-\lsc{2}}}%morpheme+gloss
\glotran{What are you going to serve \pb{to} the hill when it grabs you?}{}%eng+spa trans
{}{}%rec - time

% 9
\gloexe{Glo3:Chayshi}{}{amv}%
{Chayshi maman\pb{man} willakun.}%
{\morglo{chay-shi}{\lsc{dem.d}-\lsc{evr}}\morglo{mama-n-man}{mother-3-\lsc{all}}\morglo{willa-ku-n}{tell-\lsc{refl}-3}}%morpheme+gloss
\glotran{With that, she told her mother.}%eng
{‘En eso, se lo contó a su mamá’.}%spa
{}{}%rec - time

% 10
\gloexe{Glo3:Chayllapaq}{}{ach}%
{Chayllapaq willakurusa tirrurista\pb{man} hinaptin chayta wañurachin.}%ach que first line
{\morglo{chay-lla-paq}{\lsc{dem.d}-\lsc{rstr}-\lsc{abl}}\morglo{willa-ku-ru-sa}{tell-\lsc{refl}-\lsc{urgt}-\lsc{npst}}\morglo{tirrurista-man}{terrorist-\lsc{all}}\morglo{hinaptin}{then}\morglo{chay-ta}{\lsc{dem.d}-\lsc{acc}}\morglo{wañu-ra-chi-n}{die-\lsc{urgt}-\lsc{caus}-\lsc{3}}}%morpheme+gloss
\glotran{So they told it \pb{to} the terrorists and then they killed him.}{}%eng+spa trans
{}{}%rec - time

\noindent
It may indicate a very approximate time specification~(\ref{Glo3:Trayanqa}).\\

% 11
\gloexe{Glo3:Trayanqa}{}{amv}%
{Trayanqa sabadu\pb{man}.}%amv que first line
{\morglo{traya-nqa}{arrive-\lsc{3.fut}}\morglo{sabadu-man}{Saturday-\lsc{all}}}%morpheme+gloss
\glotran{She’ll arrive \pb{on} Saturday [\pb{or around there}].}{}%eng+spa trans
{}{}%rec - time

\noindent
With verbs indicating change of state, quantity or number, it may indicate the result or extent of change~(\ref{Glo3:Pasaypaqrunapaq}),~(\ref{Glo3:Winarun}).\\

% 12
\gloexe{Glo3:Pasaypaqrunapaq}{}{lt}%
{Pasaypaq runapaq kunvirtirun kabra\pb{man}.}%lt que first line
{\morglo{pasaypaq}{completely}\morglo{runa-paq}{person-\lsc{abl}}\morglo{kunvirti-ru-n}{convert-\lsc{urgt}-\lsc{3}}\morglo{kabra-man}{goat-\lsc{all}}}%morpheme+gloss
\glotran{Completely, from people they \pb{turned into} goats.}{}%eng+spa trans
{}{}%rec - time

% 13
\gloexe{Glo3:Winarun}{}{amv}%
{Wiñarun hatun\pb{man}.}%amv que first line
{\morglo{wiña-ru-n}{grow-\lsc{urgt}-\lsc{3}}\morglo{hatun-man}{big-\lsc{all}}}%morpheme+gloss
\glotran{She grew tall.}{}%eng+spa trans
{}{}%rec - time

\noindent
It may also indicate the goal in the sense of purpose of movement~(\ref{Glo3:Karukarum}),~(\ref{Glo3:Chaypaqrishaq}). It can usually be translated as ‘to’, ‘toward’.\\

% 14
\gloexe{Glo3:Karukarum}{}{amv}%
{Karu karum. ¿Imaynataq, ima\pb{man}taq hamuranki?}%amv que first line
{\morglo{karu}{far}\morglo{karu-m}{far-\lsc{evd}}\morglo{imayna-taq}{how-\lsc{seq}}\morglo{ima-man-taq}{what-\lsc{all}-\lsc{seq}}\morglo{hamu-ra-nki}{come-\lsc{pst}-\lsc{2}}}%morpheme+gloss
\glotran{Very far. How, \pb{for} what did you come?}{}%eng+spa trans
{}{}%rec - time

% 15
\gloexe{Glo3:Chaypaqrishaq}{}{lt}%
{Chaypaq rishaq wak animalniy\pb{man} wak infirmuykuna\pb{man}.}%lt que first line
{\morglo{chay-paq}{\lsc{dem.d}-\lsc{abl}}\morglo{ri-shaq}{go-\lsc{1.fut}}\morglo{wak}{\lsc{dem.d}}\morglo{animal-ni-y-man}{animal-\lsc{euph}-\lsc{y}-\lsc{all}}\morglo{wak}{\lsc{dem.d}}\morglo{infirmu-y-kuna-man}{sick.person-\lsc{1}-\lsc{pl}-\lsc{ll}}}%morpheme+gloss
\glotran{I’m going to go \pb{to} my animals and \pb{to} my sick [husband] and all.}{}%eng+spa trans
{}{}%rec - time

\subsubsection{Genitive, locative \phono{-pa\tss{1}}, \phono{-pa\tss{2}}}\label{ssec:genlocpa12}
As a genitive\index[sub]{genitive!\phono{-pa}}, \phono{-pa} indicates possession~(\ref{Glo3:Runapa}),~(\ref{Glo3:Imaynataqqam}); it is often paired with possessive inflection~(\ref{Glo3:Mananammiran}),~(\ref{Glo3:Puchkanchik}).\\

% 1
\gloexe{Glo3:Runapa}{}{amv}%
{Runa\pb{pa} umallaña trakillaña kayashqa.}%
{\morglo{runa-pa}{person-\lsc{gen}}\morglo{uma-lla-ña}{head-\lsc{rstr}-\lsc{disc}}\morglo{traki-lla-ña}{leg-\lsc{rstr}-\lsc{disc}}\morglo{ka-ya-shqa}{be-\lsc{prog}-\lsc{npst}}}%morpheme+gloss
\glotran{There was only the head and the hand \pb{of} the person.}%eng
{‘Nada más quedaba la cabeza y el pie de la persona’.}%spa
{}{}%rec - time

% 2
\gloexe{Glo3:Imaynataqqam}{}{sp}%
{¿Imaynataq qam\pb{pa} trakikiqa kayan qillu qillucha?}%sp que first line
{\morglo{imayna-taq}{how-\lsc{seq}}\morglo{qam-pa}{you-\lsc{gen}}\morglo{traki-ki-qa}{foot-\lsc{2}-\lsc{top}}\morglo{ka-ya-n}{be-\lsc{prog}-\lsc{3}}\morglo{qillu}{yellow}\morglo{qillu-cha}{yellow-\lsc{dim}}}%morpheme+gloss
\glotran{How are \pb{your} feet nice and yellow?}{}%eng+spa trans
{}{}%rec - time

% 3
\gloexe{Glo3:Mananammiran}{}{amv}%
{Manañam miranñachu ganawni\pb{n}qa pay\pb{pa}qa.}%
{\morglo{mana-ña-m}{no-\lsc{disc}-\lsc{evd}}\morglo{mira-n-ña-chu}{reproduce-\lsc{3}-\lsc{disc}-\lsc{neg}}\morglo{ganaw-ni-n-qa}{cattle-\lsc{euph}-\lsc{3}-\lsc{top}}\morglo{pay-pa-qa}{he-\lsc{gen}-\lsc{top}}}%morpheme+gloss
\glotran{\pb{His} animals no longer reproduce.}%eng
{‘Ya no aumentan sus animales.’}%spa
{}{}%rec - time

% 4
\gloexe{Glo3:Puchkanchik}{}{amv}%
{Puchkanchik. Vakata harkanchik vaka\pb{pa} qipa\pb{n}pa millwinchik.}%amv que first line
{\morglo{puchka-nchik}{spin-\lsc{1pl}}\morglo{vaka-ta}{cow-\lsc{acc}}\morglo{harka-nchik}{herd-\lsc{1pl}}\morglo{vaka-pa}{cow-\lsc{gen}}\morglo{qipa-n-pa}{behind-\lsc{3}-\lsc{loc}}\morglo{millwi-nchik}{wool-\lsc{1pl}}}%morpheme+gloss
\glotran{We spin. We herd the cows and \pb{behind} the cows, we [twist] our yarn.}{}%eng+spa trans
{}{}%rec - time

\noindent
As a locative\index[sub]{locative!\phono{-pa}}, \phono{-pa} indicates temporal~(\ref{Glo3:Manambiranu}) and spatial location~(\ref{Glo3:Trabahu}--\ref{Glo3:Takllawan}).\\

% 5
\gloexe{Glo3:Manambiranu}{}{amv}%
{Manam biranu\pb{pa}hinachu.}%amv que first line
{\morglo{mana-m}{no-\lsc{evd}}\morglo{biranu-pa-hina-chu}{summer-\lsc{loc}-\lsc{comp}-\lsc{neg}}}%morpheme+gloss
\glotran{Not like \pb{in} summer.}{}%eng+spa trans
{}{}%rec - time

% 6
\gloexe{Glo3:Trabahu}{}{ach}%
{Trabahu: may\pb{pa}pis may\pb{pa}pis.}%ach que first line
{\morglo{trabahu-:}{work-\lsc{1}}\morglo{may-pa-pis}{where-\lsc{loc}-\lsc{add}}\morglo{may-pa-pis}{where-\lsc{loc}-\lsc{add}}}%morpheme+gloss
\glotran{I work \pb{where}ever, \pb{where}ever.}{}%eng+spa trans
{}{}%rec - time

% 7
\gloexe{Glo3:Filapa}{}{amv}%
{Fila\pb{pa} trurakurun mana hukllachu.}%amv que first line
{\morglo{fila-pa}{line-\lsc{loc}}\morglo{trura-ku-ru-n}{put-\lsc{refl}-\lsc{urgt}-\lsc{3}}\morglo{mana}{no}\morglo{huk-lla-chu}{one-\lsc{rstr}-\lsc{neg}}}%morpheme+gloss
\glotran{They put themselves \pb{in} a line -- not just one.}{}%eng+spa trans
{}{}%rec - time

% 8
\gloexe{Glo3:Iskwila}{}{sp}%
{Iskwila\pb{pa}m niytu:kunaqa wawa:kunaqa rinmi. ñuqallam ka: analfabitu.}%sp que first line
{\morglo{iskwila-pa-m}{school-\lsc{loc}-\lsc{evd}}\morglo{niytu-:-kuna-qa}{nephew-\lsc{1}-\lsc{pl}-\lsc{top}}\morglo{wawa-:-kuna-qa}{baby-\lsc{1}-\lsc{pl}-\lsc{top}}\morglo{ri-n-mi}{go-\lsc{3}-\lsc{evd}}\morglo{ñuqa-lla-m}{I-\lsc{rstr}-\lsc{evd}}\morglo{ka-:}{be-\lsc{1}}\morglo{analfabitu}{illiterate}}%morpheme+gloss
\glotran{My grandchildren and my children are \pb{in} school. Only I am illiterate.}{}%eng+spa trans
{}{}%rec - time

% 9
\gloexe{Glo3:Takllawan}{}{ch}%
{Takllawan haluyanchik chay\pb{pa}qa. Uqa trakla. Yakuwan ichashpa chay\pb{pa}qa.}%ch que first line
{\morglo{taklla-wan}{plow-\lsc{instr}}\morglo{halu-ya-nchik}{plow-\lsc{prog}-\lsc{1pl}}\morglo{chay-pa-qa}{\lsc{dem.d}-\lsc{loc}-\lsc{top}}\morglo{uqa}{oca}\morglo{trakla}{field}\morglo{yaku-wan}{water-\lsc{instr}}\morglo{icha-shpa}{toss-\lsc{subis}}\morglo{chay-pa-qa}{\lsc{dem.d}-\lsc{loc-\lsc{top}}}}%morpheme+gloss
\glotran{We’re plowing with a [foot] plow \pb{in} there. The oca fields. Adding water \pb{in} there.}{}%eng+spa trans
{}{}%rec - time

\largerpage
\noindent
In all dialects, \phono{-paq} is often used in place of \phono{-pa} and \phono{-pi} as both a locative~(\ref{Glo3:Dimunyum}) and genitive~(\ref{Glo3:Imapaypaq}); in the \CH{} dialect, \phono{-traw} is used in addition to \phono{-pa} and \phono{-pi} as a locative~(\ref{Glo3:Pustatrawshi}),~(\ref{Glo3:Nuqakunaqa}). As a genitive, \phono{-pa} can usually be translated ‘of’ or with a possessive pronoun; as a locative, it can usually translated ‘in’ or ‘on’.\\

% 10
\gloexe{Glo3:Dimunyum}{}{ach}%
{Dimunyum chayqa. Chay~\dots{} altu rumi\pb{paq} ukun\pb{paq} yatran.}%ach que first line
{\morglo{Dimunyu-m}{Devil-\lsc{evd}}\morglo{chay-qa}{\lsc{dem.d}-\lsc{top}}\morglo{chay}{\lsc{dem.d}}\morglo{altu}{high}\morglo{rumi-paq}{stone-\lsc{loc}}\morglo{uku-n-paq}{inside-\lsc{3}-\lsc{loc}}\morglo{yatra-n}{live-\lsc{3}}}%morpheme+gloss
\glotran{It was a devil. It~\dots{} lives \pb{in} the stone up \pb{inside} it.}{}%eng+spa trans
{}{}%rec - time

% 11
\gloexe{Glo3:Imapaypaq}{}{amv}%
{¿Ima pay\pb{paq} huchan? Qaykuruptinqa hawkam sayakun uñankunata fwiraman diharuptinchik.}%amv que first line
{\morglo{ima}{what}\morglo{pay-paq}{she-\lsc{gen}}\morglo{hucha-n}{fault-\lsc{3}}\morglo{qayku-ru-pti-n-qa}{corral-\lsc{urgt}-\lsc{subds}-\lsc{3}-\lsc{top}}\morglo{hawka-m}{tranquil-\lsc{evd}}\morglo{saya-ku-n}{stand-\lsc{refl}-\lsc{3}}\morglo{uña-n-kuna-ta}{calf-\lsc{3}-\lsc{pl}-\lsc{acc}}\morglo{fwira-man}{outside-\lsc{all}}\morglo{diha-ru-pti-nchik}{leave-\lsc{urgt}-\lsc{subds}-\lsc{1pl}}}%morpheme+gloss
\glotran{What fault is it \pb{of her}s? When you toss her into the corral, she stands there calmly when we leave her babies outside.}{}%eng+spa trans
{}{}%rec - time

% 12
\gloexe{Glo3:Pustatrawshi}{}{ch}%
{Pusta\pb{traw}shi chay mutu.}%ch que first line
{\morglo{pusta-traw-shi}{clinic-\lsc{loc}-\lsc{evr}}\morglo{chay}{\lsc{dem.d}}\morglo{mutu}{motorcycle}}%morpheme+gloss
\glotran{That motorcycle is \pb{in} the health clinic.}{}%eng+spa trans
{}{}%rec - time

% 13
\gloexe{Glo3:Nuqakunaqa}{}{ch}%
{Ñuqakunaqa fayna\pb{traw}mi kaya:.}%ch que first line
{\morglo{ñuqa-kuna-qa}{I-\lsc{pl}-\lsc{top}}\morglo{fayna-traw-mi}{community.work.day-\lsc{loc}-\lsc{evd}}\morglo{ka-ya-:}{be-\lsc{prog}-\lsc{1}}}%morpheme+gloss
\glotran{We’re \pb{in} the middle of community work days.}{}%eng+spa trans
{}{}%rec - time

% 14
\gloexe{Glo3:Chaytam}{}{amv}%
{Chaytam nin kichwa\pb{pa}: “Wichayman qatishaq”.}%amv que first line
{\morglo{chay-ta-m}{\lsc{dem.d}-\lsc{acc}-\lsc{evd}}\morglo{ni-n}{say-\lsc{3}}\morglo{kichwa-\pb{pa}}{Quechua-\lsc{loc}}\morglo{wichay-man}{up.hill-\lsc{all}}\morglo{qati-shaq}{follow-\lsc{1.fut}}}%morpheme+gloss
\glotran{They say that \pb{in} Quechua: “I’ll herd it up hill.”}{}%eng+spa trans
{}{}%rec - time

\subsubsection{Ablative, benefactive, purposive \phono{-paq}}\label{ssec:ablbenpur}
As an ablative\index[sub]{ablative}, \phono{-paq} indicates provenance in space~(\ref{Glo3:Imaytaq}--\ref{Glo3:suwamuranki}) or time~(\ref{Glo3:Uchuklla}),~(\ref{Glo3:riqsinakushun}); origin or cause~(\ref{Glo3:walmitaqa}),~(\ref{Glo3:Wambray}); or the material of which an item is made~(\ref{Glo3:Llikllakuna}),~(\ref{Glo3:Ayvisruwani}).\\

% 1
\gloexe{Glo3:Imaytaq}{}{ch}%
{¿Imaytaq llaqtayki\pb{paq} lluqsimulanki?}%ch que first line
{\morglo{imay-taq}{when-\lsc{seq}}\morglo{llaqta-yki-paq}{town-\lsc{2}-\lsc{abl}}\morglo{lluqsi-mu-la-nki}{go.out-\lsc{cisl}-\lsc{pst}-\lsc{2}}}%morpheme+gloss
\glotran{When did you go out \pb{from} your country?}{}%eng+spa trans
{}{}%rec - time

% 2
\gloexe{Glo3:altuta}{}{amv}%
{Kusta\pb{paq} altuta siqaptinchik umanchik nanan.}%amv que first line
{\morglo{kusta-paq}{coast-\lsc{abl}}\morglo{altu-ta}{high-\lsc{acc}}\morglo{siqa-pti-nchik}{go.up-\lsc{subds}-\lsc{1pl}}\morglo{uma-nchik}{head-\lsc{1pl}}\morglo{nana-n}{hurt-\lsc{3}}}%morpheme+gloss
\glotran{When we come up \pb{from} the coast, our heads hurt.}{}%eng+spa trans
{}{}%rec - time

% 3
\gloexe{Glo3:suwamuranki}{}{lt}%
{“¿May\pb{paq}taqmi suwamuranki?” nishpa.}%lt que first line
{\morglo{may-paq-taq-mi}{where-\lsc{abl}-\lsc{seq}-\lsc{evd}}\morglo{suwa-mu-ra-nki}{steal-\lsc{cisl}-\lsc{pst}-\lsc{2}}\morglo{ni-shpa}{say-\lsc{subis}}}%morpheme+gloss
\glotran{“Where did you steal it \pb{from}?” he said.}{}%eng+spa trans
{}{}%rec - time

% 4
\gloexe{Glo3:Uchuklla}{}{ach}%
{Uchuklla kasa:\pb{paq}.}%ach que first line
{\morglo{uchuk-lla}{small-\lsc{rstr}}\morglo{ka-sa-:-paq}{be-\lsc{prf}-\lsc{1}-\lsc{abl}}}%morpheme+gloss
\glotran{\pb{From} [the time when] I was little.}{}%eng+spa trans
{}{}%rec - time

% 5
\gloexe{Glo3:riqsinakushun}{}{ch}%
{Kanan\pb{paq} riqsinakushun.}%ch que first line
{\morglo{kanan-paq}{now-\lsc{abl}}\morglo{riqsi-naku-shun}{know-\lsc{recip}-\lsc{1pl.fut}}}%morpheme+gloss
\glotran{\pb{From} now on, we’re going to get to know each other.}{}%eng+spa trans
{}{}%rec - time

% 6
\gloexe{Glo3:walmitaqa}{}{ch}%
{Chay huk walmitaqa talilushpaqa apalunñam uspitalman. Pasaypaq mikuy\pb{paq} alalay\pb{paq}, ¿aw?}%ch que first line
{\morglo{chay}{\lsc{dem.d}}\morglo{huk}{one}\morglo{walmi-ta-qa}{woman-\lsc{acc}-\lsc{top}}\morglo{tali-lu-shpa-qa}{find-\lsc{urgt}-\lsc{subis}-\lsc{top}}\morglo{apa-lu-n-ña-m}{bring-\lsc{urgt}-\lsc{3}-\lsc{disc}-\lsc{evd}}\morglo{uspital-man}{hospital-\lsc{all}}\morglo{pasaypaq}{completely}\morglo{miku-y-paq}{eat-\lsc{inf}-\lsc{abl}}\morglo{alala-y-paq}{cold-\lsc{inf}-\lsc{abl}}\morglo{aw}{yes}}%morpheme+gloss
\glotran{When they found the other woman they brought her to the hospital -- completely [sick] \pb{from} hunger and cold, no?}{}%eng+spa trans
{}{}%rec - time

% 7
\gloexe{Glo3:Wambray}{}{lt}%
{Wambray lichi\pb{paq}, kisu\pb{paq} waqaptin ñuqa rikurani urquta.}%lt que first line
{\morglo{wambra-y}{child-\lsc{acc}}\morglo{lichi-paq,}{milk-\lsc{abl}}\morglo{kisu-paq}{cheese-\lsc{abl}}\morglo{waqa-pti-n}{cry-\lsc{subds}-\lsc{3}}\morglo{ñuqa}{I}\morglo{riku-ra-ni}{go-\lsc{pst}-\lsc{1}}\morglo{urqu-ta}{hill-\lsc{acc}}}%morpheme+gloss
\glotran{When my children cried \pb{for} [because they had no] milk or cheese, I went to the hill.}{}%eng+spa trans
{}{}%rec - time

% 8
\gloexe{Glo3:Llikllakuna}{}{ach}%
{Llikllakuna, punchukuna, puñunakuna, ruwa: lliw lliw imatapis ruwa: kay\pb{paq}mi, kay millwa\pb{paq}mi.}%ach que first line
{\morglo{lliklla-kuna,}{shawl-\lsc{pl}}\morglo{punchu-kuna,}{poncho-\lsc{pl}}\morglo{puñu-na-kuna}{sleep-\lsc{nmlz}-\lsc{pl}}\morglo{ruwa-:}{make-\lsc{1}}\morglo{lliw}{all}\morglo{lliw}{all}\morglo{ima-ta-pis}{what-\lsc{acc}-\lsc{add}}\morglo{ruwa-:}{make-\lsc{1}}\morglo{kay-paq-mi}{\lsc{dem.p}-\lsc{abl}-\lsc{evd}}\morglo{kay}{\lsc{dem.p}}\morglo{millwa-paq-mi}{wool-\lsc{abl}-\lsc{evd}}}%morpheme+gloss
\glotran{Shawls, ponchos, blankets -- everything, everything I make \pb{from} this, \pb{from} this yarn.}{}%eng+spa trans
{}{}%rec - time

% 9
\gloexe{Glo3:Ayvisruwani}{}{amv}%
{Ayvis ruwani wiqa\pb{paq} uviha\pb{paq}.}%amv que first line
{\morglo{ayvis}{sometimes}\morglo{ruwa-ni}{make-\lsc{1}}\morglo{wiqa-paq}{twisted.wool-\lsc{abl}}\morglo{uviha-paq}{sheep-\lsc{abl}}}%morpheme+gloss
\glotran{Sometimes I make them \pb{out of} twisted wool, \pb{out of} sheep’s wool.}{}%eng+spa trans
{}{}%rec - time

\largerpage
\noindent
As a benefactive\index[sub]{benefactive}, \phono{-paq} indicates the individual who benefits from --~or suffers as a result of~-- an event~(\ref{Glo3:qarikuna}).\\

% 10
\gloexe{Glo3:qarikuna}{}{amv}%
{Chay allin chay qarikuna mana ishpayta atipaq\pb{paq}.}%
{\morglo{chay}{\lsc{dem.d}}\morglo{allin}{good}\morglo{chay}{\lsc{dem.d}}\morglo{qari-kuna}{man-\lsc{pl}}\morglo{mana}{no}\morglo{ishpa-y-ta}{urinate-\lsc{inf}-\lsc{acc}}\morglo{atipa-q-paq}{be.able-\lsc{ag}-\lsc{ben}}}%morpheme+gloss
\glotran{This is good \pb{for} men who can’t urinate.}%eng
{‘éso es bueno para los hombres que no pueden orinar’.}%spa
{}{}%rec - time

\noindent
As a purposive\index[sub]{purposive}, \phono{-paq} indicates the purpose of an event~(\ref{Glo3:plantam}),~(\ref{Glo3:Qawanay}).\\

% 11
\gloexe{Glo3:plantam}{}{amv}%
{Quni quni plantam chayqa. Chiri\pb{paq}mi allin.}%amv que first line
{\morglo{quni}{warm}\morglo{quni}{warm}\morglo{planta-m}{plant-\lsc{evd}}\morglo{chay-qa}{\lsc{dem.d}-\lsc{top}}\morglo{chiri-paq-mi}{cold-\lsc{purp}-\lsc{evd}}\morglo{allin}{good}}%morpheme+gloss
\glotran{This plant is really warm. It’s good \pb{for} (fighting) the cold.}{}%eng+spa trans
{}{}%rec - time

% 12
\gloexe{Glo3:Qawanay}{}{amv}%
{Qawanay\pb{paq} imawan wañurun nishpa kitrani.}%
{\morglo{qawa-na-y-paq}{see-\lsc{nmlz}-\lsc{1}-\lsc{purp}}\morglo{ima-wan}{what-\lsc{instr}}\morglo{wañu-ru-n}{die-\lsc{urgt}-\lsc{3}}\morglo{ni-shpa}{say-\lsc{subis}}\morglo{kitra-ni}{open-\lsc{1}}}%morpheme+gloss
\glotran{‘\pb{To} see what he died from, I said, and I opened him up.}%eng
{‘Para ver con qué murió, lo abrí’.}%spa
{}{}%rec - time

\noindent
\phono{-paq} may also alternate with \phono{-pa} and \phono{-pi} to indicate the genitive~(\ref{Glo3:puchukarun}) or locative~(\ref{Glo3:Asnu}),~(\ref{Glo3:kundinawmi}).\\

% 13
\gloexe{Glo3:puchukarun}{}{amv}%
{Manam kanchu ñuqa\pb{paq} puchukarun.}%amv que first line
{\morglo{mana-m}{no-\lsc{evd}}\morglo{ka-n-chu}{be-\lsc{3}-\lsc{neg}}\morglo{ñuqa-paq}{I-\lsc{gen}}\morglo{puchuka-ru-n}{finish-\lsc{urgt}-\lsc{3}}}%morpheme+gloss
\glotran{There aren’t any -- \pb{mine} are all finished up.}{}%eng+spa trans
{}{}%rec - time

% 14
\gloexe{Glo3:Asnu}{}{sp}%
{Asnu alla-allita atuq watakun kunka\pb{paq} traki\pb{paq} sugawan watarun.}%sp que first line
{\morglo{asnu}{donkey}\morglo{alla-alli-ta}{a.lot-a.lot-\lsc{acc}}\morglo{atuq}{fox}\morglo{wata-ku-n}{tie-\lsc{refl}-\lsc{3}}\morglo{kunka-paq}{throat-\lsc{abl}}\morglo{traki-paq}{foot-\lsc{abl}}\morglo{suga-wan}{rope-\lsc{instr}}\morglo{wata-ru-n}{tie-\lsc{urgt}-\lsc{3}}}%morpheme+gloss
\glotran{The fox tied the donkey up really well. He tied him up with a rope \pb{on} his neck and \pb{on} his foot.}{}%eng+spa trans
{}{}%rec - time

% 15
\gloexe{Glo3:kundinawmi}{}{amv}%
{Kay llaqta\pb{paq} kundinawmi lliw lliw runata puchukayan.}%amv que first line
{\morglo{kay}{\lsc{dem.d}}\morglo{llaqta-paq}{town-\lsc{loc}}\morglo{kundinaw-mi}{zombie-\lsc{evd}}\morglo{lliw}{all}\morglo{lliw}{all}\morglo{runa-ta}{person-\lsc{acc}}\morglo{puchuka-ya-n}{finish-\lsc{prog}-\lsc{3}}}%morpheme+gloss
\glotran{\pb{In} this town, a zombie is finishing off all the people.}{}%eng+spa trans
{}{}%rec - time

\noindent
\phono{-paq} also figures in a number of fixed expressions~(\ref{Glo3:Pasaypaq}),~(\ref{Glo3:Kuyaylla}).\\

% 16
\gloexe{Glo3:Pasaypaq}{}{amv}%
{\pb{Pasaypaq} uyqaytapis puchukarun. ¿Imatataq mikushaq?}%amv que first line
{\morglo{pasaypaq}{completely}\morglo{uyqa-y-ta-pis}{sheep-\lsc{1}-\lsc{acc}-\lsc{add}}\morglo{puchuka-ru-n}{finish-\lsc{urgt}-\lsc{3}}\morglo{ima-ta-taq}{what-\lsc{acc}-\lsc{seq}}\morglo{miku-shaq}{eat-\lsc{1.fut}}}%morpheme+gloss
\glotran{My sheep are \pb{completely} finished. What will I eat?}{}%eng+spa trans
{}{}%rec - time

% 17
\gloexe{Glo3:Kuyaylla}{}{sp}%
{Kuyaylla\pb{paq} waqakuyan yutuqa, kuyakuylla\pb{paq} chay waychawwan yutuqa.}%sp que first line
{\morglo{kuya-y-lla-paq}{love-\lsc{inf}-\lsc{rstr}-\lsc{abl}}\morglo{waqa-ku-ya-n}{cry-\lsc{refl}-\lsc{prog}-\lsc{3}}\morglo{yutu-qa}{partridge-\lsc{top}}\morglo{kuya-ku-y-lla-paq}{love-\lsc{refl}-\lsc{inf}-\lsc{abl}}\morglo{chay}{\lsc{dem.d}}\morglo{waychaw-wan}{waychaw.bird-\lsc{instr}}\morglo{yutu-qa}{partridge-\lsc{top}}}%morpheme+gloss
\glotran{The partridge is singing beautiful\pb{ly}. The \phono{waychaw} and the partridge [sing] beautiful\pb{ly}.}{}%eng+spa trans
{}{}%rec - time

\noindent
Suffixed to the distal demonstrative \phono{chay}, \phono{-paq} indicates a close temporal or causal connection between two events, translating ‘then’ or ‘so’~(\ref{Glo3:Balinaku}).\\

% 18
\gloexe{Glo3:Balinaku}{}{ch}%
{Balinaku:. “¡Paqarin yanapamay!” u “Paqarin ñuqakta chay\pb{paq} talpushun qampaktañataq”, ninaku:mi.}%ch que first line
{\morglo{bali-naku:}{request.a.service-\lsc{recip}-\lsc{1}}\morglo{paqarin}{tomorrow}\morglo{yanapa-ma-y}{help-\lsc{1.obj}-\lsc{imp}}\morglo{u}{or}\morglo{paqarin}{tomorrow}\morglo{ñuqa-kta}{I-\lsc{acc}}\morglo{chay-\pb{paq}}{\lsc{dem.d}-\lsc{abl}}\morglo{talpu-shun}{plant-\lsc{1pl.fut}}\morglo{qam-pa-kta-ña-taq}{you-\lsc{gen}-\lsc{acc}-\lsc{disc}-\lsc{seq}}\morglo{ni-naku-:-mi}{say-\lsc{recip}-\lsc{1}-\lsc{evd}}}%morpheme+gloss
\glotran{We ask for each other’s services. “Help me tomorrow!” or, “Tomorrow mine \pb{then} we’ll plant yours,” we say to each other.}{}%eng+spa trans
{}{}%rec - time

\largerpage
\noindent
In comparative expressions, \phono{-paq} attaches to the base of comparison~(\ref{Glo3:Qayna}),~(\ref{Glo3:Celia}); it may be combined with the Spanish-origin comparatives \phono{mihur} (\phono{mejor} ‘better’) and \phono{piyur} (\phono{peor} ‘worse’)~(\ref{Glo3:Pular}). It can generally be translated ‘for’; in its capacity as a purposive, it can generally be translated ‘in order to’.\\

% 19
\gloexe{Glo3:Qayna}{}{amv}%
{Qayna puntraw\pb{paq} masmi.}%amv que first line
{\morglo{qayna}{previous}\morglo{puntraw-paq}{day-\lsc{abl}}\morglo{mas-mi}{more-\lsc{evd}}}%morpheme+gloss
\glotran{It’s more \pb{than} yesterday.}{}%eng+spa trans
{}{}%rec - time

% 20
\gloexe{Glo3:Celia}{}{sp}%
{Celia\pb{paq}pis masta chawan.}%sp que first line
{\morglo{Celia-paq-pis}{Celia-\lsc{abl}-\lsc{add}}\morglo{mas-ta}{more-\lsc{acc}}\morglo{chawa-n}{milk-\lsc{3}}}%morpheme+gloss
\glotran{She milks more \pb{than} Celia.}{}%eng+spa trans
{}{}%rec - time

% 21
\gloexe{Glo3:Pular}{}{ach}%
{Pular\pb{paq}pis mas \pb{mihur}tam chayqa allukun.}%ach que first line
{\morglo{pular-paq-pis}{fleece-\lsc{abl}-\lsc{add}}\morglo{mas}{more}\morglo{mihur-ta-m}{better-\lsc{acc}-\lsc{evd}}\morglo{chay-qa}{\lsc{dem.d}-\lsc{top}}\morglo{allu-ku-n}{wrap-\lsc{refl}-\lsc{3}}}%morpheme+gloss
\glotran{\pb{Better} \pb{than} fleece -- this bundles you up.}{}%eng+spa trans
{}{}%rec - time

\subsubsection{Locative \phono{-pi}}\label{ssec:genlocpi}
As a locative\index[sub]{locative!\phono{-pi}}, \phono{-pi} indicates temporal~(\ref{Glo3:Kananpuntraw}),~(\ref{Glo3:Uktubripaqway}) and spatial location~(\ref{Glo3:chakirusa}--\ref{Glo3:atuqta}).\\

% 1
\gloexe{Glo3:Kananpuntraw}{}{amv}%
{Kanan puntraw\pb{pi} rishaq.}%amv que first line
{\morglo{kanan}{now}\morglo{puntraw-pi}{day-\lsc{loc}}\morglo{ri-shaq}{go-\lsc{1.fut}}}%morpheme+gloss
\glotran{I’ll go today.}{}%eng+spa trans
{}{}%rec - time

% 2
\gloexe{Glo3:Uktubripaqway}{}{ch}%
{¿Uktubri paqway\pb{pi}ñachu hamunki?}%ch que first line
{\morglo{uktubri}{October}\morglo{paqwa-y-pi-ña-chu}{finish-\lsc{inf}-\lsc{loc}-\lsc{disc}-\lsc{q}}\morglo{hamu-nki}{come-\lsc{2}}}%morpheme+gloss
\glotran{Are you coming \pb{at} the end of October?}{}%eng+spa trans
{}{}%rec - time

% 3
\gloexe{Glo3:chakirusa}{}{ach}%
{Chay\pb{pi} chakirusa walantin vistiduntinshi.}%ach que first line
{\morglo{chay-pi}{\lsc{dem.d}-\lsc{loc}}\morglo{chaki-ru-sa}{dry-\lsc{urgt}-\lsc{npst}}\morglo{wala-ntin}{skirt-\lsc{incl}}\morglo{vistidu-ntin-shi}{dress-\lsc{incl}-\lsc{evr}}}%morpheme+gloss
\glotran{\pb{There} she dried out with her skirt and her dress.}{}%eng+spa trans
{}{}%rec - time

% 4
\gloexe{Glo3:Chaylaguna}{}{sp}%
{Chay laguna\pb{pi} yatraqñataq nira, “¿Imaynam qam kayanki puka traki?”}%sp que first line
{\morglo{chay}{\lsc{dem.d}}\morglo{laguna-pi}{lake-\lsc{loc}}\morglo{yatra-q-ña-taq}{live-\lsc{ag}-\lsc{disc}-\lsc{seq}}\morglo{ni-ra}{say-\lsc{pst}}\morglo{imayna-m}{how-\lsc{evd}}\morglo{qam}{you}\morglo{ka-ya-nki}{be-\lsc{prog}-\lsc{2}}\morglo{puka}{red}\morglo{traki}{foot}}%morpheme+gloss
\glotran{The one that lives \pb{in} the lake said, “How do you have red feet?”}{}%eng+spa trans
{}{}%rec - time

% 5
\gloexe{Glo3:atuqta}{}{sp}%
{Kundurñataq atuqta apustirun, “¿Mayqinninchik lasta\pb{pi} urqu\pb{pi} wañurushun?”}%sp que first line
{\morglo{kundur-ña-taq}{condor-\lsc{disc}-\lsc{seq}}\morglo{atuq-ta}{fox-\lsc{acc}}\morglo{apusti-ru-n}{bet-\lsc{urgt}-\lsc{3}}\morglo{mayqin-ni-nchik}{which-\lsc{euph}-\lsc{1pl}}\morglo{lasta-pi}{snow-\lsc{loc}}\morglo{urqu-pi}{hill-\lsc{loc}}\morglo{wañu-ru-shun}{die-\lsc{urgt}-\lsc{1pl.fut}}}%morpheme+gloss
\glotran{The condor bet the fox, “Which of us will die \pb{in} the snow, \pb{in} the hills?”}{}%eng+spa trans
{}{}%rec - time

\noindent
It is used in the expression to speak in a language~(\ref{Glo3:Kastillanu}).\\

% 6
\gloexe{Glo3:Kastillanu}{}{amv}%
{Kastillanu\pb{pi} rimaq chayllamanñam shimin riyan manayá kay kichwa.}%amv que first line
{\morglo{kastillanu-pi}{Spanish-\lsc{loc}}\morglo{rima-q}{talk-\lsc{ag}}\morglo{chay-lla-man-ña-m}{\lsc{dem.d}-\lsc{rstr}-\lsc{all}-\lsc{disc}-\lsc{evd}}\morglo{shimi-n}{mouth-\lsc{3}}\morglo{ri-ya-n}{go-\lsc{prog}-\lsc{3}}\morglo{mana-yá}{no-\lsc{emph}}\morglo{kay}{\lsc{dem.p}}\morglo{kichwa}{Quechua}}%morpheme+gloss
\glotran{Those who speak \pb{in} Spanish, their mouths are running just there. Not [those who speak in?] Quechua.}{}%eng+spa trans
{}{}%rec - time

\noindent
It can be translated as ‘in’, ‘on’, or ‘at’. \phono{-pi} has a marginal use as a genitive\index[sub]{genitive!\phono{-pi}} indicating subordinative relations --~including, prominently, relationships of possession~-- between nouns referring to different items~(\ref{Glo3:planta}). In this capacity it is translated as ‘of’ or with a possessive.\\

% 7
\gloexe{Glo3:planta}{}{amv}%
{Chay planta\pb{pi} yatan.}%amv que first line
{\morglo{chay}{\lsc{dem.d}}\morglo{planta-pi}{tree-\lsc{gen}}\morglo{yata-n}{side-\lsc{3}}}%morpheme+gloss
\glotran{The side \pb{of} that tree.}{}%eng+spa trans
{}{}%rec - time

\subsubsection{Exclusive \phono{-puRa}}
\phono{-puRa}\index[sub]{exclusive} --~realized \phono{-pula} in the \CH{} dialect~(\ref{Glo3:Walmipula}) and \phono{-pura} in all others~-- indicates the inclusion of the marked individual among other individuals of the same kind. It can be translated as ‘among’ or ‘between’. \phono{-puRa} is not commonly employed; more commonly employed is the particle \phono{intri} ‘between’, borrowed from Spanish (\spanish{entre} ‘between’)~(\ref{Glo3:warmiqa}).\\

% 1
\gloexe{Glo3:Walmipula}{}{ch}%
{Walmi\pb{pula} qutunakulanchik.}%ch que first line
{\morglo{walmi-pula}{woman-\lsc{excl}}\morglo{qutu-naku-la-nchik}{gather-\lsc{recip}-\lsc{pst}-\lsc{1pl}}}%morpheme+gloss
\glotran{We women gathered amongst ourselves.}{}%eng+spa trans
{}{}%rec - time

% 1
\gloexe{Glo3:warmiqa}{}{amv}%
{\pb{Intri} warmiqa ¿Imatatr ruwanman hapinakushpa?}%amv que first line
{\morglo{intri}{between}\morglo{warmi-qa}{woman-\lsc{top}}\morglo{ima-ta-tr}{what-\lsc{acc}-\lsc{evc}}\morglo{ruwa-n-man}{make-\lsc{3}-\lsc{cond}}\morglo{hapi-naku-shpa}{grab-\lsc{recip}-\lsc{subis}}}%morpheme+gloss
\glotran{\pb{Between} women, what are they going to do when they grab each other?}{}%eng+spa trans
{}{}%rec - time

\subsubsection{Reason \phono{-rayku}}
\phono{-rayku}\index[sub]{causative} indicates motivation~(\ref{Glo3:Chawashiq}),~(\ref{Glo3:Papallayki}) or reason~(\ref{Glo3:Waynayki}),~(\ref{Glo3:Mikunallan}). It generally but not obligatorily follows possessive inflection~(\ref{Glo3:Chawashiq}--\ref{Glo3:Mikunallan}).\\

% 1
\gloexe{Glo3:Chawashiq}{}{amv}%
{Chawashiq lichillan\pb{rayku} riymantri.}%
{\morglo{chawa-shi-q}{milk-\lsc{acmp}-\lsc{ag}}\morglo{lichi-lla-n-rayku}{milk-\lsc{rstr}-\lsc{3}-\lsc{reasn}}\morglo{ri-y-man-tri}{go-\lsc{1}-\lsc{cond}-\lsc{evc}}}%morpheme+gloss
\glotran{I could go help milk on account of her milk.}%eng
{‘Podría ir a ayudar a lechear a cuenta de su leche’.}%spa
{}{}%rec - time

% 2
\gloexe{Glo3:Papallayki}{}{amv}%
{Papallayki\pb{rayku}pis awapakuruyman.}%amv que first line
{\morglo{papa-lla-yki-rayku-pis}{potato-\lsc{rstr}-\lsc{2}-\lsc{reasn}-\lsc{add}}\morglo{awa-paku-ru-y-man}{weave-\lsc{mutben}-\lsc{urgt}-\lsc{1}-\lsc{cond}}}%morpheme+gloss
\glotran{Even \pb{for} your potatoes, I’d weave.}{}%eng+spa trans
{}{}%rec - time

% 3
\gloexe{Glo3:Waynayki}{}{ch}%
{Waynayki shamunan\pb{rayku}.}%ch que first line
{\morglo{wayna-yki}{lover-\lsc{2}}\morglo{shamu-na-n-rayku}{come-\lsc{nmlz}-\lsc{3}-\lsc{reasn}}}%morpheme+gloss
\glotran{\pb{On account of} your lover’s coming.}{}%eng+spa trans
{}{}%rec - time

% 4
\gloexe{Glo3:Mikunallan}{}{amv}%
{Mikunallan\pb{rayku}pis yanukunqatr.}%amv que first line
{\morglo{miku-na-lla-n-rayku-pis}{eat-\lsc{nmlz}-\lsc{rstr}-\lsc{3}-\lsc{reasn}-\lsc{add}}\morglo{yanu-ku-nqa-tr}{cook-\lsc{refl}-\lsc{3.fut}-\lsc{evc}}}%morpheme+gloss
\glotran{\pb{On account of} her food, she’ll probably cook.}{}%eng+spa trans
{}{}%rec - time

\noindent
It can generally be translated ‘because’, ‘because of’ or ‘on account of’. \phono{-rayku} is not frequently employed: ablative \phono{-paq} is more frequently employed to indicate motivation or reason~(\ref{Glo3:Qatra}), although this \phono{-paq} does not, as an anonymous reviewer points out, mark the same relation. \phono{-kawsu} (\Sp~\spanish{causa} ‘cause’) may be employed in place of \phono{-rayku}~(\ref{Glo3:Manamlichi}). Recognized but not attested spontaneously outside \AMV{} and \CH.\\

% 5
\gloexe{Glo3:Qatra}{}{amv}%
{Qatra vakaqa wanuyan qutranman. Sikintin qaykusan\pb{paq}.}%amv que first line
{\morglo{qatra}{dirty}\morglo{vaka-qa}{cow-\lsc{top}}\morglo{wanu-ya-n}{excrete-\lsc{prog}-\lsc{3}}\morglo{qutra-n-man}{lake-\lsc{3}-\lsc{all}}\morglo{siki-ntin}{calf-\lsc{incl}}\morglo{qayku-sa-n-paq}{corral-\lsc{prf}-\lsc{3}-\lsc{abl}}}%morpheme+gloss
\glotran{That dirty cow is pissing in the reservoir! \pb{For} having been let out with her calf.}{}%eng+spa trans
{}{}%rec - time

% 6
\gloexe{Glo3:Manamlichi}{}{amv}%
{Manam lichi kanchu. Pastu \pb{kawsu}.}%amv que first line
{\morglo{mana-m}{no-\lsc{evd}}\morglo{lichi}{milk-\lsc{3}}\morglo{ka-n-chu}{be-\lsc{3}-\lsc{neg}}\morglo{pastu-kawsu}{pasture.grass-cause}}%morpheme+gloss
\glotran{There’s no milk. \pb{Because of} the grass.}{}%eng+spa trans
{}{}%rec - time

\subsubsection{Accusative \phono{-Kta} and \phono{-ta}}
In the \CH{} dialect, the accusative\index[sub]{accusative} is realized \phono{-kta} after a short vowel and \phono{-ta} after a long vowel or consonant~(\ref{Glo3:piluta}),~(\ref{Glo3:apakunki}); in all other dialects it is realized as \phono{-ta} in all environments. \phono{-ta} indicates the object or goal of a transitive verb~(\ref{Glo3:Asnuqa}),~(\ref{Glo3:maqarura}).\\

% 1
\gloexe{Glo3:piluta}{}{ch}%
{\pb{Tilivisyunta} likakuyan, piluta \pb{pukllaq}\pb{kuna}\pb{kta}m.}%ch que first line
{\morglo{tilivisyun-ta}{television-\lsc{acc}}\morglo{lika-ku-ya-n}{look-\lsc{refl}-\lsc{prog}-\lsc{3}}\morglo{piluta}{ball}\morglo{puklla-q-kuna-kta-m}{play-\lsc{ag}-\lsc{pl}-\lsc{acc}-\lsc{evd}}}%morpheme+gloss
\glotran{They’re watching \pb{television}, ball \pb{players}.}{}%eng+spa trans
{}{}%rec - time

% 2
\gloexe{Glo3:apakunki}{}{ch}%
{“\pb{Suti:ta}m apakunki”, ¡niy! “\pb{Llapanta} apakunki”.}%ch que first line
{\morglo{suti-:-ta-m}{name-\lsc{1}-\lsc{acc}-\lsc{evd}}\morglo{apa-ku-nki}{bring-\lsc{refl}-\lsc{2}}\morglo{ni-y}{say	-\lsc{imp}}\morglo{llapa-n-ta}{all-\lsc{3}-\lsc{acc}}\morglo{apa-ku-nki}{bring-\lsc{refl}-\lsc{2}}}%morpheme+gloss
\glotran{Say, “You’re going to take along my \pb{name}. You’re going to take along \pb{them all}.”}{}%eng+spa trans
{}{}%rec - time

% 3
\gloexe{Glo3:Asnuqa}{}{sp}%
{Asñuqa nin, “Ñuqa tarisisayki \pb{sugaykita}qa”.}%sp que first line
{\morglo{asnu-qa}{donkey-\lsc{top}}\morglo{ni-n,}{say-\lsc{3}}\morglo{ñuqa}{I}\morglo{tari-si-sayki}{find-\lsc{acmp}-\lsc{1>2.fut}}\morglo{suga-yki-ta-qa}{rope-\lsc{2}-\lsc{acc}-\lsc{top}}}%morpheme+gloss
\glotran{The mule said, “I’m going to help you find \pb{your rope}.”}{}%eng+spa trans
{}{}%rec - time

% 4
\gloexe{Glo3:maqarura}{}{lt}%
{Wak Kashapatapiñam maqarura \pb{César Mullidata}.}%lt que first line
{\morglo{wak}{\lsc{dem.d}}\morglo{Kashapata-pi-ña-m}{Kashapata-\lsc{loc}-\lsc{disc}-\lsc{evd}}\morglo{maqa-ru-ra}{beat-\lsc{urgt}-\lsc{pst}}\morglo{César}{César}\morglo{Mullida-ta}{Mullida-\lsc{acc}}}%morpheme+gloss
\glotran{They beat \pb{César Mullida} there in Kashapata.}{}%eng+spa trans
{}{}%rec - time

\noindent
\phono{-ta} may occur more than once in a clause, marking multiple objects~(\ref{Glo3:Maqtakunata}),~(\ref{Glo3:Vakata}) or both object and goal. In case one noun modifies another, case-marking on the head \lsc{n} is obligatory~(\ref{Glo3:Sibadata}); on the modifying \lsc{n}, optional~(\ref{Glo3:Asnuqa}).\\

% 5
\gloexe{Glo3:Maqtakunata}{}{amv}%
{¿\pb{Maqtakunata} pushanki icha \pb{pashñata}?}%amv que first line
{\morglo{maqta-kuna-ta}{young.man-\lsc{pl}-\lsc{acc}}\morglo{pusha-nki}{bring.along-\lsc{2}}\morglo{icha}{or}\morglo{pashña-ta}{girl-\lsc{acc}}}%morpheme+gloss
\glotran{Are you going to take \pb{the boys} or \pb{the girl}?}{}%eng+spa trans
{}{}%rec - time

% 6
\gloexe{Glo3:Vakata}{}{amv}%
{¡\pb{Vakata} \pb{lliwta} qaquruy! Rikurushaq hanaypim.}%amv que first line
{\morglo{vaka-ta}{cow-\lsc{acc}}\morglo{lliw-ta}{all-\lsc{acc}}\morglo{qaqu-ru-y}{toss.out-\lsc{urgt}-\lsc{imp}}\morglo{riku-ru-shaq}{go-\lsc{urgt}-\lsc{1.fut}}\morglo{hanay-pi-m}{up.hill-\lsc{loc}-\lsc{evd}}}%morpheme+gloss
\glotran{Toss out \pb{the cows}, \pb{all of them}! I’m going to go up hill.}{}%eng+spa trans
{}{}%rec - time

% 7
\gloexe{Glo3:Sibadata}{}{amv}%
{\pb{Sibadata} \pb{trakrata} kwidanchik.}%amv que first line
{\morglo{sibada-ta}{barley-\lsc{acc}}\morglo{trakra-ta}{field-\lsc{acc}}\morglo{kwida-nchik}{care.for-\lsc{1pl}}}%morpheme+gloss
\glotran{We take care of the \pb{barley field}.}{}%eng+spa trans
{}{}%rec - time

\noindent
Complement clauses are suffixed with \phono{-ta}~(\ref{Glo3:Qaqapaq}--\ref{Glo3:willasuptiki}).\\

% 8
\gloexe{Glo3:Qaqapaq}{}{sp}%
{\pb{Qaqapaq} \pb{lluqsiyta} atipanchu. Qayakun, “¿Imaynataq kanan lluqsishaq?”}%sp que first line
{\morglo{qaqa-paq}{cliff-\lsc{abl}}\morglo{lluqsi-y-ta}{go.out-\lsc{inf}-\lsc{acc}}\morglo{atipa-n-chu}{be.able-\lsc{3}-\lsc{neg}}\morglo{qaya-ku-n}{shout-\lsc{refl}-\lsc{3}}\morglo{imayna-taq}{how-\lsc{seq}}\morglo{kanan}{now}\morglo{lluqsi-shaq}{go.out-\lsc{1.fut}}}%morpheme+gloss
\glotran{She couldn’t \pb{get off the cliff}. She shouted, “Now, how am I going to get down?”}{}%eng+spa trans
{}{}%rec - time

% 9
\gloexe{Glo3:Chaypaq}{}{sp}%
{Chaypaq \pb{kabrata} \pb{mikuyta} qallakuykun.}%sp que first line
{\morglo{chay-paq}{\lsc{dem.d}-\lsc{abl}}\morglo{kabra-ta}{goat-\lsc{acc}}\morglo{miku-y-ta}{eat-\lsc{inf}-\lsc{acc}}\morglo{qalla-ku-yku-n}{begin-\lsc{refl}-\lsc{excep}-\lsc{3}}}%morpheme+gloss
\glotran{So, the fox started \pb{to eat the goat}.}{}%eng+spa trans
{}{}%rec - time

% 10
\gloexe{Glo3:willasuptiki}{}{lt}%
{Wambra willasuptiki \pb{imayna} \pb{kutirimusanta}.}%lt que first line
{\morglo{wambra}{child}\morglo{willa-su-pti-ki}{tell-\lsc{3>2}-\lsc{subds}-\lsc{3>2}}\morglo{imayna}{how}\morglo{kuti-ri-mu-sa-n-ta}{return-\lsc{incep}-\lsc{urgt}-\lsc{prf}-\lsc{3}-\lsc{acc}}}%morpheme+gloss
\glotran{When the children told you \pb{how they had returned}.}{}%eng+spa trans
{}{}%rec - time

\noindent
\phono{-ta} always attaches to the last word in a multi-word phrase~(\ref{Glo3:Chayshiyatrarun}).\\

% 11
\gloexe{Glo3:Chayshiyatrarun}{}{amv}%
{Chayshi yatrarun \pb{kundur} \pb{kashanta}.}%
{\morglo{chay-shi}{\lsc{dem.d}-\lsc{evr}}\morglo{yatra-ru-n}{know-\lsc{urgt}-\lsc{3}}\morglo{kundur}{condor}\morglo{ka-sha-n-ta}{be-\lsc{prf}-\lsc{3}-\lsc{acc}}}%morpheme+gloss
\glotran{That’s how they found out \pb{he was a condor}.}%eng
{‘En éso supieron que era condor’.}%spa
{}{}%rec - time

\noindent
With \phono{-na} nominalizations, \phono{-ta} may be omitted. In many instances, \phono{-ta} does not indicate accusative case. \phono{-ta} may indicate the goal of movement of a person, as in~(\ref{Glo3:Siqashpaqa}) and~(\ref{Glo3:rirqani}), \phono{-n-ta} may indicate \lsc{path}~(\ref{Glo3:shamushpa}) (see also~§\sectref{sssec:simhina}, ex.(~\ref{Glo3:kunachatam})).\footnote{Thanks to Willem Adelaar for pointing this out to me.}\\

% 12
\gloexe{Glo3:Siqashpaqa}{}{amv}%
{Siqashpaqa chuqaykaramun \pb{ukuta} almataqa.}%
{\morglo{siqa-shpa-qa}{ascend-\lsc{subis}-\lsc{top}}\morglo{chuqa-yka-ra-mu-n}{throw-\lsc{excep}-\lsc{urgt}-\lsc{cisl}-\lsc{3}}\morglo{uku-ta}{inside-\lsc{acc}}\morglo{alma-ta-qa}{	soul-\lsc{acc}-\lsc{top}}}%morpheme+gloss
\glotran{Going up, he threw the ghost \pb{inside}.}%eng
{‘Subiendo botó al alma al dentro’.}%spa
{}{}%rec - time

% 13
\gloexe{Glo3:rirqani}{}{amv}%
{\pb{Qiñwalta}m rirqani yanta qipikuq.}%amv que first line
{\morglo{qiñwal-ta-m}{quingual.grove-\lsc{acc}-\lsc{evd}}\morglo{ri-rqa-ni}{go-\lsc{pst}-\lsc{1}}\morglo{yanta}{firewood}\morglo{qipi-ku-q}{carry-\lsc{refl}-\lsc{ag}}}%morpheme+gloss
\glotran{I went \pb{to the quingual grove} to carry firewood.}{}%eng+spa trans
{}{}%rec - time

% 14
\gloexe{Glo3:shamushpa}{}{ch}%
{\pb{Ukunta} shamushpa. \pb{Qaqunanta} shamushpapis.}%ch que first line
{\morglo{uku-n-ta}{inside-\lsc{3}-\lsc{acc}}\morglo{shamu-shpa}{come-\lsc{subis}}\morglo{Qaquna-n-ta}{Qaquna-\lsc{3}-\lsc{acc}}\morglo{shamu-shpa-pis}{come-\lsc{subis}-\lsc{add}}}%morpheme+gloss
\glotran{Coming \pb{via} the interior. Coming \pb{via} Qaquna.}{}%eng+spa trans
{}{}%rec - time

\noindent
\phono{-ta} marks substantives --~nouns, adjectives, numerals, derived nouns~-- when they function as adverbs~(\ref{Glo3:Kikinqa}--\ref{Glo3:Kumpadri}).\\

% 15
\gloexe{Glo3:Kikinqa}{}{sp}%
{Kikinqa \pb{allinta}raqtaq gusaq.}%sp que first line
{\morglo{kiki-n-qa}{self-\lsc{3}-\lsc{top}}\morglo{allin-ta-raq-taq}{good-\lsc{acc}-\lsc{cont}-\lsc{seq}}\morglo{gusa-q}{enjoy-\lsc{ag}}}%morpheme+gloss
\glotran{They themselves enjoyed them \pb{well} still.}{}%eng+spa trans
{}{}%rec - time

% 16
\gloexe{Glo3:trurakunchik}{}{amv}%
{Rupanchikta trurakunchik \pb{qilluta}.}%amv que first line
{\morglo{rupa-nchik-ta}{clothes-\lsc{1pl}-\lsc{acc}}\morglo{trura-ku-nchik}{put-\lsc{refl}-\lsc{1pl}}\morglo{qillu-ta}{yellow-\lsc{acc}}}%morpheme+gloss
\glotran{We dress ourselves \pb{in yellow}.}{}%eng+spa trans
{}{}%rec - time

% 17
\gloexe{Glo3:Ishkayishkayta}{}{amv}%
{\pb{Ishkay ishkayta}m plantaramuni.}%amv que first line
{\morglo{ishkay}{two}\morglo{ishkay-ta-m}{two-\lsc{acc}-\lsc{evd}}\morglo{planta-ra-mu-ni}{plant-\lsc{urgt}-\lsc{cisl}-\lsc{1}}}%morpheme+gloss
\glotran{I planted them \pb{two by two}.}{}%eng+spa trans
{}{}%rec - time

% 18
\gloexe{Glo3:Kumpadri}{}{sp}%
{“Kumpadri, ¿Imaynataq waqayanki qamqa? ¡\pb{Kuyayllata} waqanki!” nin.}%sp que first line
{\morglo{kumpadri,}{compadre}\morglo{imayna-taq}{why-\lsc{seq}}\morglo{waqa-ya-nki}{cry-\lsc{prog}-\lsc{2}}\morglo{qam-qa}{you-\lsc{top}}\morglo{kuya-y-lla-ta}{love-\lsc{inf}-\lsc{rstr}-\lsc{acc}}\morglo{waqa-nki}{cry-\lsc{2}}\morglo{ni-n}{say-\lsc{3}}}%morpheme+gloss
\glotran{“\phono{Compadre}, why are you crying? How \pb{lovely} you sing!” he said.}{}%eng+spa trans
{}{}%rec - time

\noindent
It may also mark an item directly affected by an event or time period culminating in an event~(\ref{Glo3:madrugaw}).\\

% 19
\gloexe{Glo3:madrugaw}{}{amv}%
{Chay huk madrugaw \pb{trinta i unu di abrilta} lluqsirun waway.}%amv que first line
{\morglo{chay}{\lsc{dem.d}}\morglo{huk}{one}\morglo{madrugaw}{morning}\morglo{trinta}{thirty}\morglo{i}{and}\morglo{unu}{one}\morglo{di}{of}\morglo{abril-ta}{April-\lsc{acc}}\morglo{lluqsi-ru-n}{go.out-\lsc{urgt}-\lsc{3}}\morglo{wawa-y}{baby-\lsc{1}}}%morpheme+gloss
\glotran{On that morning, \pb{the thirty-first of April}, my son left the house [and was kidnapped].}{}%eng+spa trans
{}{}%rec - time

\noindent
With verbs referring to natural phenomena, \phono{-ta} may mark a place affected by an event~(\ref{Glo3:Yakupis}),~(\ref{Glo3:Llaqtaykita}).\\

% 20
\gloexe{Glo3:Yakupis}{}{amv}%
{Yakupis tukuy \pb{pampata} rikullaq.}%
{\morglo{yaku-pis}{water-\lsc{add}}
\morglo{tukuy}{all}
\morglo{pampa-ta}{ground-\lsc{acc}}
\morglo{ri-ku-lla-q}{go-\lsc{refl}-\lsc{rstr}-\lsc{ag}}}%morpheme+gloss
\glotran{The water, too, would go all over the ground.}%eng
{‘El agua también iba por todos lados en la pampa’.}%spa
{}{}%rec - time

% 21
\gloexe{Glo3:Llaqtaykita}{}{amv}%
{¿\pb{Llaqtaykita} paranchu?}%amv que first line
{\morglo{llaqta-yki-ta}{town-\lsc{2}-\lsc{acc}}\morglo{para-n-chu?}{rain-\lsc{3}-\lsc{q}}}%morpheme+gloss
\glotran{Does it rain \pb{on your town}?}{}%eng+spa trans
{}{}%rec - time

\noindent
With verbs of communication, it may mark the person receiving the communication~(\ref{Glo3:swirupis}),~(\ref{Glo3:Tarpuriptinchikpis}).\\

% 22
\gloexe{Glo3:swirupis}{}{lt}%
{“Kay swirupis allquypaqpis. Faltan”, nikurunshi \pb{subrinunta}qa.}%lt que first line
{\morglo{kay}{\lsc{dem.d}}\morglo{swiru-pis}{whey-\lsc{add}}\morglo{allqu-y-paq-pis}{dog-\lsc{1}-\lsc{ben}-\lsc{add}}\morglo{falta-n}{lack-\lsc{3}}\morglo{ni-ku-ru-n-shi}{say-\lsc{refl}-\lsc{urgt}-\lsc{3}-\lsc{evr}}\morglo{subrinu-n-ta-qa}{nephew-\lsc{3}-\lsc{acc}-\lsc{top}}}%morpheme+gloss
\glotran{“This whey of mine, too, is for my dog. There isn’t enough,” he said \pb{to his nephew}.}{}%eng+spa trans
{}{}%rec - time

% 23
\gloexe{Glo3:Tarpuriptinchikpis}{}{amv}%
{Tarpuriptinchikpis mikunchu wak \pb{Shullita} wak \pb{Erminiota} nini.}%amv que first line
{\morglo{tarpu-ri-pti-nchik-pis}{plant-\lsc{incep}-\lsc{subds}-\lsc{1pl}-\lsc{add}}\morglo{miku-n-chu}{eat-\lsc{3}-\lsc{neg}}\morglo{wak}{\lsc{dem.d}}\morglo{Shulli-ta}{Shulli-\lsc{acc}}\morglo{wak}{\lsc{dem.d}}\morglo{Erminio-ta}{Erminio-\lsc{acc}}\morglo{ni-ni}{say-\lsc{1}}}%morpheme+gloss
\glotran{If we plant it, they won’t eat it, I said \pb{to my younger brother}, \pb{to Erminio}.}{}%eng+spa trans
{}{}%rec - time

\subsubsection{Instrumental, comitative \phono{-wan}}
\phono{-wan} indicates means or company\index[sub]{comitative}. \phono{-wan} may mark an instrument\index[sub]{instrumental} or item which is essential to the event~(\ref{Glo3:qalatuykushpa}),~(\ref{Glo3:Qaliqa}).\\

% 1
\gloexe{Glo3:qalatuykushpa}{}{lt}%
{Chaymi qalatuykushpa kuriyan\pb{wan} alli-allita chikutita qura.}%lt que first line
{\morglo{chay-mi}{\lsc{dem.d}-\lsc{evd}}\morglo{qalatu-yku-shpa}{strip.naked-\lsc{excep}-\lsc{subis}}\morglo{kuriya-n-wan}{belt-\lsc{3}-\lsc{instr}}\morglo{alli-alli-ta}{good-good-\lsc{acc}}\morglo{chikuti-ta}{whip-\lsc{acc}}\morglo{qu-ra}{give-\lsc{pst}}}%morpheme+gloss
\glotran{Then they stripped him naked and gave him a whipping \pb{with} his belt.}{}%eng+spa trans
{}{}%rec - time

% 2
\gloexe{Glo3:Qaliqa}{}{ch}%
{Qaliqa taklla\pb{wan}mi halun. Qipantañataq kulpakta maqanchik piku\pb{wan}.}%ch que first line
{\morglo{qali-qa}{man-\lsc{top}}\morglo{taklla-wan-mi}{plow-\lsc{instr}-\lsc{evd}}\morglo{halu-n}{turn.earth-\lsc{3}}\morglo{qipa-n-ta-ña-taq}{behind-\lsc{3}-\lsc{acc}-\lsc{disc}-\lsc{seq}}\morglo{kulpa-kta}{clod-\lsc{acc}}\morglo{maqa-nchik}{hit-\lsc{1pl}}\morglo{piku-wan}{pick-\lsc{instr}}}%morpheme+gloss
\glotran{Men turn the earth \pb{with} a [foot] plow. Behind them, we break up the clods \pb{with} a pick.}{}%eng+spa trans
{}{}%rec - time

\noindent
\phono{-wan} marks all means of transportation~(\ref{Glo3:kapas}).\\

% 3
\gloexe{Glo3:kapas}{}{ch}%
{Karru\pb{wan}tri kapas trayamunña. Mutu\pb{wan}shi hamula.}%ch que first line
{\morglo{karru-wan-tri}{car-\lsc{instr}-\lsc{evc}}\morglo{kapas}{maybe}\morglo{traya-mu-n-ña}{arrive-\lsc{cisl}-\lsc{3}-\lsc{disc}}\morglo{mutu-wan-shi}{motorcycle-\lsc{instr}-\lsc{evr}}\morglo{hamu-la}{come-\lsc{pst}}}%morpheme+gloss
\glotran{Maybe she came \pb{on} the bus. She came \pb{by} motorbike, she says.}{}%eng+spa trans
{}{}%rec - time

\noindent
It may mark illnesses~(\ref{Glo3:Prustata}).\\

% 4
\gloexe{Glo3:Prustata}{}{ch}%
{¿Prustata\pb{wan}tri kayanki?}%ch que first line
{\morglo{prustata-wan-tri}{prostate-\lsc{instr}-\lsc{evc}}\morglo{ka-ya-nki}{be-\lsc{prog}-\lsc{2}}}%morpheme+gloss
\glotran{Would you have prostate [problems]?}{}%eng+spa trans
{}{}%rec - time

\noindent
\phono{-wan} may mark any animate individual who takes part in an event together with the performer~(\ref{Glo3:Taytachalla}),~(\ref{Glo3:Imapaqmi}); it may also mark the actor in an event referred to by a causative verb~(\ref{Glo3:Manaraqmi}).\\

% 5
\gloexe{Glo3:Taytachalla}{}{ach}%
{Taytachalla:\pb{wan} kawsakura: mamachalla:\pb{wan} kawsakura:. Mama:qa huk kumprumisu\pb{wan} rikun huk lawta.}%ach que first line
{\morglo{tayta-cha-lla-:-wan}{father-\lsc{dim}-\lsc{rstr}-\lsc{1}-\lsc{instr}}\morglo{kawsa-ku-ra-:}{live-\lsc{refl}-\lsc{pst}-\lsc{1}}\morglo{mama-cha-lla-:-wan}{mother-\lsc{dim}-\lsc{rstr}-\lsc{1}-\lsc{instr}}\morglo{kawsa-ku-ra-:}{live-\lsc{refl}-\lsc{pst}-\lsc{1}}\morglo{mama-:-qa}{mother-\lsc{1}-\lsc{top}}\morglo{huk}{one}\morglo{kumprumisu-wan}{commitment-\lsc{instr}}\morglo{ri-ku-n}{go-\lsc{refl}-\lsc{3}}\morglo{huk}{one}\morglo{law-ta}{side-\lsc{acc}}}%morpheme+gloss
\glotran{I lived \pb{with} just my grandfather and my grandmother. My mother went to another place \pb{with} another commitment.}{}%eng+spa trans
{}{}%rec - time

% 6
\gloexe{Glo3:Imapaqmi}{}{amv}%
{¿Imapaqmi wak kundinaw\pb{wan} puriyanki?}%
{\morglo{ima-paq-mi}{what-\lsc{purp}-\lsc{evd}}\morglo{wak}{\lsc{dem.d}}\morglo{kundinaw-wan}{zombie-\lsc{instr}}\morglo{puri-ya-nki}{zombie-\lsc{prog}-\lsc{2}}}%morpheme+gloss
\glotran{Why are you walking around \pb{with} that zombie?}%eng
{‘¿Por qué estás andando con ese condenado?’}%spa
{}{}%{Florida\_JH\_Condor\_Condenados}{05:48--05:53}

% 7
\gloexe{Glo3:Manaraqmi}{}{ach}%
{Manaraqmi qari:pis kararaqchu. Sapalla: wak wasipa puñukura: \pb{vaka:wan}.}%ach que first line
{\morglo{mana-raq-mi}{no-\lsc{cont}-\lsc{evd}}\morglo{qari-:-pis}{man-\lsc{1}-\lsc{add}}\morglo{ka-ra-raq-chu}{be-\lsc{pst}-\lsc{cont}-\lsc{neg}}\morglo{sapa-lla-:}{alone-\lsc{rstr}-\lsc{1}}\morglo{wak}{\lsc{dem.d}}\morglo{wasi-pa}{house-\lsc{loc}}\morglo{puñu-ku-ra-:}{sleep\lsc{refl}-\lsc{pst}-\lsc{1}}\morglo{vaka-:-wan}{cow-\lsc{1}-\lsc{instr}}}%morpheme+gloss
\glotran{I still didn’t have my husband. I slept alone in my house \pb{with} my cows.}{}%eng+spa trans
{}{}%rec - time

\noindent
\phono{wan} may mark coordinate relations between nouns or nominal groups; case matching attaches to all items except the last in a coordinate series~(\ref{Glo3:Alicia}). It can usually be translated ‘with’.\\

% 8
\gloexe{Glo3:Alicia}{}{amv}%
{Mila\pb{wan} Alicia\pb{wan} Hilda trayaramun.~\updag}%amv que first line
{\morglo{Mila-wan}{Mila-\lsc{instr}}\morglo{Alicia-wan}{Alicia-\lsc{instr}}\morglo{Hilda}{Hilda}\morglo{traya-ra-mu-n}{arrive-\lsc{urgt}-\lsc{cisl}-\lsc{3}}}%morpheme+gloss
\glotran{Hilda arrived \pb{with} Mila and Alicia.}{}%eng+spa trans
{}{}%rec - time

\subsubsection{Possible combinations}
Combinations\index[sub]{case!combinations} of case suffixes are rare. They do occur, however, notably with \phono{-pa}, \phono{-wan}, and \phono{-hina}. Where a noun phrase marked with genitive \phono{-pa} or \phono{-paq} functions as an anaphor, the phrase may be case marked as its referent would be~(\ref{Glo3:Paqarin}),~(\ref{Glo3:Piluntaqa}). Note that in ~(\ref{Glo3:Piluntaqa}) the accusative has no phonological reflex in the English gloss.\\

% 1
\gloexe{Glo3:Paqarin}{}{ch}%
{Paqarin yanapamay u paqarin \pb{ñuqapakta} chaypaq talpashun \pb{qampakta}ñataq.}%ch que first line
{\morglo{paqarin}{tomorrow}\morglo{yanapa-ma-y}{help-\lsc{1.obj}-\lsc{imp}}\morglo{u}{or}\morglo{paqarin}{tomorrow}\morglo{ñuqa-pa-kta}{I-\lsc{gen}-\lsc{acc}}\morglo{chay-paq}{\lsc{dem.d}-\lsc{abl}}\morglo{talpu-shun}{plant-\lsc{1pl.fut}}\morglo{qam-pa-kta-ña-taq}{you-\lsc{gen}-\lsc{acc}-\lsc{disc}-\lsc{seq}}}%morpheme+gloss
\glotran{Help me tomorrow or tomorrow \pb{mine} and then we’ll plant \pb{yours}.}{}%eng+spa trans
{}{}%rec - time

% 2
\gloexe{Glo3:Piluntaqa}{}{amv}%
{Piluntaqa yupayanshari chay chapu\pb{paqta}. Ushachinchu yupayta.}%amv que first line
{\morglo{pilu-n-ta-qa}{hair-\lsc{3}-\lsc{acc}-\lsc{seq}}\morglo{yupa-ya-n-sh-ari}{count-\lsc{prog}-\lsc{3}-\lsc{evr}-\lsc{ari}}\morglo{chay}{\lsc{dem.d}}\morglo{chapu-paq-ta}{dog-\lsc{gen}-\lsc{acc}}\morglo{ushachi-n-chu}{be.able-\lsc{3}-\lsc{neg}}\morglo{yupa-y-ta}{count-\lsc{inf}-\lsc{acc}}}%morpheme+gloss
\glotran{He’s counting the hairs of that small [hairless] dog, but he can’t count them.}{}%eng+spa trans
{}{}%rec - time

\noindent
In addition to functioning as a case marker, \phono{-wan} also serves to conjoin noun phrases. In this capacity, \phono{-wan} may follow other case markers~(\ref{Glo3:Mishkita}),~(\ref{Glo3:kabranpawan}).\\

% 3
\gloexe{Glo3:Mishkita}{}{amv}%
{Mishkita yawarnintam mikurunchik mutin\pb{tawan} papan\pb{tawan}.}%amv que first line
{\morglo{mishki-ta}{sweet-\lsc{acc}}\morglo{yawar-ni-n-ta-m}{blood-\lsc{euph}-\lsc{3}-\lsc{acc}-\lsc{evd}}\morglo{miku-ru-nchik}{eat-\lsc{urgt}-\lsc{1pl}}\morglo{muti-n-ta-wan}{hominy-\lsc{3}-\lsc{acc}-\lsc{instr}}\morglo{papa-n-ta-wan}{potato-\lsc{3}-\lsc{acc}-\lsc{instr}}}%morpheme+gloss
\glotran{We eat its delicious blood with hominy \pb{and} with potatoes.}{}%eng+spa trans
{}{}%rec - time

% 4
\gloexe{Glo3:kabranpawan}{}{amv}%
{Chay kabranpawan vakan\pb{pawan}tri kisuchan.}%
{\morglo{chay}{\lsc{dem.d}}\morglo{kabr-n-pa-wan}{goat-\lsc{3}-\lsc{gen}-\lsc{instr}}\morglo{vaka-n-pa-wan}{cow-\lsc{3}-\lsc{gen}-\lsc{instr}}\morglo{kisu-cha-n}{cheese-\lsc{dim}-\lsc{3}}}%morpheme+gloss
\glotran{Her cheese would be from her goats’ [milk] \pb{and} from her cows’[milk].}%eng
{‘Su quesito será de [la leche] de sus cabras y de sus vacas.’}%spa
{}{}%rec - time

\noindent
Elicited examples~(\ref{Glo3:purawan}),~(\ref{Glo3:Piliyarachin}) follow \citet{Parker76gram}.\index[aut]{Parker, Gary J.}\\

% 5
\gloexe{Glo3:purawan}{}{amv}%
{Qari\pb{purawan} kambyashun.~\updag}%amv que first line
{\morglo{qari-pura-wan}{man-\lsc{excl}-\lsc{instr}}\morglo{kambya-shun}{change-\lsc{1pl.fut}}}%morpheme+gloss
\glotran{Let’s exchange husbands [\pb{for one another}].}{}%eng+spa trans
{}{}%rec - time

% 6
\gloexe{Glo3:Piliyarachin}{}{amv}%
{Piliyarachin wambra\pb{purata}.~\updag}%amv que first line
{\morglo{piliya-ra-chi-n}{fight-\lsc{urgt}-\lsc{caus}-\lsc{3}}\morglo{wambra-pura-ta}{child-\lsc{excl}-\lsc{acc}}}%morpheme+gloss
\glotran{He made the boys fight \pb{among} themselves.}{}%eng+spa trans
{}{}%rec - time

\noindent
Comparative \phono{-hina} may also combine with other case markers~(\ref{Glo3:Karsil}),~(\ref{Glo3:vakata}).\\

% 7
\gloexe{Glo3:Karsil}{}{ach}%
{Karsil\pb{pahina}m witrqamara. Wambra:kuna istudyaq pasan.}%ach que first line
{\morglo{karsil-pa-hina-m}{prison-\lsc{loc}-\lsc{comp}-\lsc{evd}}\morglo{witrqa-ma-ra}{close.in-\lsc{1.obj}-\lsc{pst}}\morglo{wambra-:-kuna}{child-\lsc{1}-\lsc{pl}}\morglo{istudya-q}{study-\lsc{ag}}\morglo{pasa-n}{pass-\lsc{3}}}%morpheme+gloss
\glotran{They closed me in \pb{like in} a jail. My children leave to study.}{}%eng+spa trans
{}{}%rec - time

% 8
\gloexe{Glo3:vakata}{}{amv}%
{Kanan vakata pusillaman chawayanchik kabra\pb{tahina}.}%amv que first line
{\morglo{kanan}{now}\morglo{vaka-ta}{cow-\lsc{acc}}\morglo{pusi-lla-man}{cup-\lsc{rstr}-\lsc{all}}\morglo{chawa-ya-nchik}{milk-\lsc{prog}-\lsc{1pl}}\morglo{kabra-ta-hina}{goat-\lsc{acc}-\lsc{comp}}}%morpheme+gloss
\glotran{Now we milk a cow into a cup \pb{like} a goat.}{}%eng+spa trans
{}{}%rec - time
 
\subsubsection{More specific noun-noun relations}\label{sssec:msnnr}
Noun-noun relations more specific than the ‘in’ and ‘of’, for example, of \phono{-pi} and \phono{-pa} are expressed by noun phrases headed by nouns which name relative positions (see \sectref{ssec:locnouns} on locative nouns)~(\ref{Glo3:karutam}--\ref{Glo3:pasayan}). Such nouns include, for example, \phono{qipa} ‘rear’; \phono{hawa} ‘top’; and \phono{trawpi} ‘center’. The head (relational) noun is inflected for person, agreeing with the noun to which it is related; this noun may be inflected with genitive \phono{-pa} (\phono{pantyun\pb{-pa} qipa-\pb{n}} ‘behind the cemetery’ lit. ‘of the cemetery its behind’).\\

% 1
\gloexe{Glo3:karutam}{}{amv}%
{Wak urqu \pb{qipa}npa karu karutam muyumunchik.}%amv que first line
{\morglo{wak}{\lsc{dem.d}}\morglo{urqu}{hill}\morglo{qipa-n-pa}{behind-\lsc{3}-\lsc{loc}}\morglo{karu}{far}\morglo{karu-ta-m}{far-\lsc{acc}-\lsc{evd}}\morglo{muyu-mu-nchik}{circle-\lsc{cisl}-\lsc{1pl}}}%morpheme+gloss
\glotran{We circle around very far \pb{behind} that hill.}{}%eng+spa trans
{}{}%rec - time


% 2
\gloexe{Glo3:Kundur}{}{sp}%
{Kundur tiya-ya-n rumi \pb{hawa}-n-pa ima-tri-ki.}%
{\morglo{kundur}{condor}\morglo{tiya-ya-n}{sit-\lsc{prog}-\lsc{3}}\morglo{rumi}{rock}\morglo{hawa-n-pa}{top-\lsc{3}-\lsc{loc}}\morglo{ima-tri-ki}{what-\lsc{evc}-\lsc{iki}}}%morpheme+gloss
\glotran{The condor must be sitting \pb{on top} of a rock.}%eng
{‘El cóndor estará sentado encima de una piedra’.}%spa
{}{}%rec - time
 

% 3
\gloexe{Glo3:npatriki}{}{ach}%
{Waka \pb{uku}npatriki runa wañura unay.}%ach que first line
{\morglo{waka}{ruins}\morglo{uku-n-pa-tri-ki}{inside-\lsc{3}-\lsc{loc}-\lsc{evc}-\lsc{iki}}\morglo{runa}{person}\morglo{wañu-ra}{die-\lsc{pst}}\morglo{unay}{before}}%morpheme+gloss
\glotran{\pb{Inside} the ruins, people must have died before.}{}%eng+spa trans
{}{}%rec - time

% 4
\gloexe{Glo3:pasayan}{}{amv}%
{Wak wambra qaqa \pb{trawpi}ntam pasayan manam manchakuyan.}%amv que first line
{\morglo{wak}{\lsc{dem.d}}\morglo{wambra}{child}\morglo{qaqa}{cliff}\morglo{trawpi-n-ta-m}{center-\lsc{3}-\lsc{acc}-\lsc{evd}}\morglo{pasa-ya-n}{pass-\lsc{prog}-\lsc{3}}\morglo{mana-m}{no-\lsc{evd}}\morglo{mancha-ku-ya-n}{scare-\lsc{refl}-\lsc{prog}-\lsc{3}}}%morpheme+gloss
\glotran{That boy passes \pb{between} the cliffs. He’s not afraid.}{}%eng+spa trans
{}{}%rec - time

\section{Substantive derivation}
In \SYQ, as in other Quechuan languages, suffixes deriving substantives\index[sub]{substantive!derivation} may be divided into two classes, governing and restrictive. Governing suffixes may be further divided into two subclasses: those which derive substantives from verbs (\phono{-na}, \phono{-q}, \phono{-sHa}, \phono{-y}) and those which derive substantives from other substantives (\phono{-ntin}, \phono{-sapa}, \phono{-yuq}, \phono{-masi}). \SYQ{} has a single restrictive suffix deriving substantives, diminutive \phono{-cha}. \phono{-lla} also functions to restrict substantives, but it is treated here not as a derivational morpheme but as an enclitic. §\sectref{ssec:sdfv} and~\ref{ssec:sdfs} cover the the governing suffixes deriving substantives from verbs and those deriving substantives from other substantives, respectively.

\subsection{Substantive derived from verbs}\label{ssec:sdfv}\index[sub]{substantive!derivation from verbs}
Four suffixes derive substantives from verbs in \SYQ{}: \phono{-na}, \phono{-q}, \phono{-sHa}, and \phono{-y}. All four form both relative and complement clauses. \phono{-na}, \phono{-q}, \phono{-sHa}, and \phono{-y} form subjunctive, agentive, indicative, and infinitive clauses, respectively. The nominalizing suffixes attach directly to the verb stem, with the exception that the first- and second-person object suffixes, \phono{-wa}/\phono{ma} and \phono{-sHa}, may intervene. \sectref{sssc:con}--\ref{sssc:inf} cover \phono{-na}, \phono{-q}, \phono{-sHa}, and \phono{-y} in turn.

\subsubsection{\phono{-na}}\label{sssc:con}\index[sub]{substantive!concretizing}
\phono{-na} derives nouns that refer to (a) the instrument with which the action named by the base is realized (\phono{alla-na} ‘harvesting tool’)~(\ref{Glo3:Mulinchik}),~(\ref{Glo3:punchukuna}); (b) the place in which the event referred to occurs (\phono{michi-na} ‘pasture’)~(\ref{Glo3:Iskina}); and (c) the object in which the action named by the base is realized (\phono{upya-na} ‘drinking water’, \phono{milla-na} ‘nausea’)~(\ref{Glo3:Mamayqa}),~(\ref{Glo3:lliwshi}).\\

% 1
\gloexe{Glo3:Mulinchik}{}{amv}%
{Mulinchik makinapaq kamcharinchik \pb{kallana}pa.}%amv que first line
{\morglo{muli-nchik}{grind-\lsc{1pl}}\morglo{makina-paq}{machine-\lsc{loc}}\morglo{kamcha-ri-nchik}{toast-\lsc{incep}-\lsc{1pl}}\morglo{kalla-na-pa}{toast-\lsc{nmlz}-\lsc{loc}}}%morpheme+gloss
\glotran{We grind it in a machine and then we toast it in the \pb{toasting pan}.}{}%eng+spa trans
{}{}%rec - time

% 2
\gloexe{Glo3:punchukuna}{}{ach}%
{Llikllakuna, punchukuna, \pb{puñuna}kuna ruwa:.}%ach que first line
{\morglo{lliklla-kuna,}{shawl-\lsc{pl}}\morglo{punchu-kuna,}{poncho-\lsc{pl}}\morglo{puñu-na-kuna,}{sleep-\lsc{nmlz}-\lsc{pl}}\morglo{ruwa-:}{make-\lsc{1}}}%morpheme+gloss
\glotran{I make shawls, ponchos and \pb{blankets}.}{}%eng+spa trans
{}{}%rec - time

% 3
\gloexe{Glo3:Iskina}{}{amv}%
{Iskina hawanpa \pb{michina}yki.}%amv que first line
{\morglo{iskina}{corner}\morglo{hawa-n-pa}{above-3-\lsc{loc}}\morglo{michi-na-yki}{pasture-\lsc{nmlz}-\lsc{2}}}%morpheme+gloss
\glotran{Above the corner \pb{where you pasture}.}{}%eng+spa trans
{}{}%rec - time

% 4
\gloexe{Glo3:Mamayqa}{}{amv}%
{Mamayqa wichayta \pb{mikuna}yta apashpa asnuchanwan kargachakusa hamuq.}%amv que first line
{\morglo{mama-y-qa}{mother-\lsc{1}-\lsc{top}}\morglo{wichay-ta}{up.hill-\lsc{acc}}\morglo{miku-na-y-ta}{eat-\lsc{nmlz}-\lsc{1}-\lsc{acc}}\morglo{apa-shpa}{bring-\lsc{subis}}\morglo{asnu-cha-n-wan}{donkey-\lsc{dim}-\lsc{3}-\lsc{instr}}\morglo{karga-cha-ku-sa}{carry-\lsc{dim}-\lsc{refl}-\lsc{prf}}\morglo{hamu-q}{come-\lsc{ag}}}%morpheme+gloss
\glotran{My mother would come up hill bringing my \pb{food}, carrying it with her donkey.}{}%eng+spa trans
{}{}%rec - time

% 5
\gloexe{Glo3:lliwshi}{}{ach}%
{\pb{Mikuna}ntapis lliw lliwshi sibadanta trigunta ima kaqtapis katriwan takurachisa.}%ach que first line
{\morglo{miku-na-n-ta-pis}{eat-\lsc{nmlz}-\lsc{3}-\lsc{acc}-\lsc{add}}\morglo{lliw}{all}\morglo{lliw-shi}{all-\lsc{evr}}\morglo{sibada-n-ta}{barley-\lsc{3}-\lsc{acc}}\morglo{trigu-n-ta}{wheat-\lsc{3}-\lsc{add}}\morglo{ima}{what}\morglo{ka-q-ta-pis}{be-\lsc{ag}-\lsc{acc}-\lsc{add}}\morglo{katri-wan}{salt-\lsc{instr}}\morglo{taku-ra-chi-sa}{mix-\lsc{urgt}-\lsc{caus}-\lsc{npst}}}%morpheme+gloss
\glotran{Their \pb{food}, too, everything, everything, their barley, their wheat, anything, they mixed it with salt.}{}%eng+spa trans
{}{}%rec - time

  
\noindent
Followed by a possessive suffix plus the copula auxiliary inflected for third person (null just in case tense/aspect are not specified), \phono{-na} indicates necessity (i.e.,~it forms a universal deontic/teleological modal) (\phono{taqsa-na-yki} ‘you have to wash’) (\ref{Glo3:Sibadayta}), (\ref{Glo3:Hinata}).\\

% 6
\gloexe{Glo3:Sibadayta}{}{amv}%
{Sibadayta wayrachishaq abasniyta \pb{pallanay} kayan.}%amv que first line
{\morglo{sibada-y-ta}{barley-\lsc{1}-\lsc{acc}}\morglo{wayra-chi-shaq}{wind-\lsc{caus}-\lsc{1.fut}}\morglo{abas-ni-y-ta}{broad.beans-\lsc{euph}-\lsc{1}-\lsc{acc}}\morglo{palla-na-y}{pick-\lsc{nmlz}-\lsc{1}}\morglo{ka-ya-n}{be-\lsc{prog}-\lsc{3}}}%morpheme+gloss
\glotran{I’m going to winnow my barley -- \pb{I have to pick} my broad beans.}{}%eng+spa trans
{}{}%rec - time

% 7
\gloexe{Glo3:Hinata}{}{lt}%
{Hinata risani \pb{yanukunay} kakuyaptin.}%lt que first line
{\morglo{hina-ta}{thus-\lsc{acc}}
\morglo{risa-ni}{pray-\lsc{1}}
\morglo{yanu-ku-na-y}{cook-\lsc{refl}-\lsc{nmlz}-\lsc{1}}
\morglo{ka-ku-ya-pti-n}{be-\lsc{refl}-\lsc{prog}-\lsc{subds}-\lsc{3}}}%morpheme+gloss
\glotran{I pray like that -- when he’s there, \pb{I have to cook}.}{}%eng+spa trans
{}{}%rec - time

\noindent
The past tense of necessity is formed by adding \phono{ka-RQa}, the third person simple past tense form of \phono{ka-} ‘be’ (\phono{palla-na-y ka-ra} ‘I had to pick’)~(\ref{Glo3:Kutikamura}),~(\ref{Glo3:Shinkakunaqa}).\\

% 8
\gloexe{Glo3:Kutikamura}{}{ach}%
{Kutikamura qari wambra: \pb{yaykunan} kara manaña atiparachu.}%ach que first line
{\morglo{kuti-ka-mu-ra}{return-\lsc{passacc}-\lsc{cisl}-\lsc{pst}}
\morglo{qari}{man}
\morglo{wambra-:}{child-\lsc{1}}
\morglo{yayku-na-n}{enter-\lsc{nmlz}-\lsc{3}}
\morglo{ka-ra}{be-\lsc{pst}}
\morglo{mana-ña}{no-\lsc{disc}}
\morglo{atipa-ra-chu}{be.able-\lsc{pst}-\lsc{neg}}}%morpheme+gloss
\glotran{My son came back -- he \pb{was supposed to} enter [university] but he couldn’t any more.}{}%eng+spa trans
{}{}%rec - time

% 9
\gloexe{Glo3:Shinkakunaqa}{}{amv}%
{Shinkakunaqa {kasunan} \pb{kara} madriqa rabyasatr kutin.}%xx que first line
{\morglo{shinka-kuna-qa}{drunk-\lsc{pl}-\lsc{top}}\morglo{kasu-na-n}{pay.attention-\lsc{nmlz}-\lsc{3}}\morglo{ka-ra}{be-\lsc{pst}}\morglo{madri-qa}{nun-\lsc{top}}\morglo{rabya-sa-tr}{be.mad-\lsc{prf}-\lsc{evc}}\morglo{kuti-n}{return-\lsc{3}}}%morpheme+gloss
\glotran{The drunks \pb{had to} pay [\pb{should have} paid] attention. The nun must have gotten mad.}{}%eng+spa trans
{}{}%rec - time

\noindent
In combination with the purposive case suffix \phono{-paq}, \phono{-na} forms subordinate clauses that indicate the purpose of the action in the main clause (\phono{qawa-na-y-paq} ‘so I can see’)~(\ref{Glo3:Ganawkuna}--\ref{Glo3:kunihuqa}).\\

% 10
\gloexe{Glo3:Ganawkuna}{}{sp}%
{Ganawkuna \pb{michina:paq} chay chaytam trakra \pb{trabahana:paq}.}%sp que first line
{\morglo{ganaw-kuna}{cattle-\lsc{pl}}\morglo{michi-na-:-paq}{pasture-\lsc{nmlz}-\lsc{1}-\lsc{purp}}\morglo{chay}{\lsc{dem.d}}\morglo{chay-ta-m}{\lsc{dem.d}-\lsc{acc}-\lsc{evd}}\morglo{trakra}{field}\morglo{trabaha-na-:-paq}{work-\lsc{nmlz}-\lsc{1}-\lsc{purp}}}%morpheme+gloss
\glotran{\pb{So I can herd} the cows, \pb{so I can work} in the fields.}{}%eng+spa trans
{}{}%rec - time

% 11
\gloexe{Glo3:Tambopaq}{}{amv}%
{Tambopaq apamuq kani, “¡Mikuy! ¡\pb{Hampishuna}ykipaq!” nini.}%amv que first line
{\morglo{Tambo-paq}{Tambo-\lsc{abl}}\morglo{apa-mu-q}{bring-\lsc{cisl}-\lsc{ag}}\morglo{ka-ni,}{be-\lsc{1}}\morglo{miku-y}{eat-\lsc{imp}}\morglo{hampi-shu-na-yki-paq}{cure-\lsc{3>2}-\lsc{nmlz}-\lsc{3>2}-\lsc{purp}}\morglo{ni-ni}{say-\lsc{1}}}%morpheme+gloss
\glotran{I used to bring it from Tambopata. “Eat it \pb{so it can} cure you!” I said.}{}%eng+spa trans
{}{}%rec - time

% 12
\gloexe{Glo3:maqashunaykipaq}{}{amv}%
{\pb{Mana}ña yapa \pb{maqashunaykipaq}.}%amv que first line
{\morglo{mana-ña}{no-\lsc{disc}}\morglo{yapa}{again}\morglo{maqa-shu-na-yki-paq}{hit-\lsc{3>2}-\lsc{nmlz}-\lsc{3>2}-\lsc{purp}}}%morpheme+gloss
\glotran{\pb{So she doesn’t hit you} again.}{}%eng+spa trans
{}{}%rec - time

% 13
\gloexe{Glo3:kunihuqa}{}{sp}%
{“¿Imay ura chay kunihuqa kutimunqa \pb{yanapamananpaq}?” nin.}%sp que first line
{\morglo{imay}{when}\morglo{ura}{hour}\morglo{chay}{\lsc{dem.d}}\morglo{kunihu-qa}{rabbit-\lsc{top}}\morglo{kuti-mu-nqa}{return-\lsc{cisl}-\lsc{3.fut}}\morglo{yanapa-ma-na-n-paq}{help-\lsc{1.obj}-\lsc{nmlz}-\lsc{3}-\lsc{purp}}\morglo{ni-n}{say-\lsc{3}}}%morpheme+gloss
\glotran{“What time is that rabbit going to come back \pb{so he can} help me?” said [the fox].}{}%eng+spa trans
{}{}%rec - time

\noindent
\phono{-na} forms subjunctive complement clauses with the verb \phono{muna-} ‘want’ (\phononb{tushu}\phononb{-na}\phononb{-n}\phononb{-ta} \phononb{muna-ni} ‘I want her to dance’)~(\ref{Glo3:Pagananta}),~(\ref{Glo3:Hinaptinshi}).\\

% 14
\gloexe{Glo3:Pagananta}{}{ach}%
{\pb{Pagananta munayan}, \pb{rantinanta} gasolinata.}%ach que first line
{\morglo{paga-na-n-ta}{pay-\lsc{nmlz}-\lsc{3}-\lsc{acc}}\morglo{muna-ya-n}{want-\lsc{prog}-\lsc{3}}\morglo{ranti-na-n-ta}{buy-\lsc{nmlz}-\lsc{3}-\lsc{acc}}\morglo{gasolina-ta}{gasoline-\lsc{acc}}}%morpheme+gloss
\glotran{\pb{He wants her to} \pb{pay}, \pb{to buy} gasoline.}{}%eng+spa trans
{}{}%rec - time

% 15
\gloexe{Glo3:Hinaptinshi}{}{ach}%
{Hinaptinshi paytaqa mana \pb{tarpunanta munasa}chu.}%ach que first line
{\morglo{hinaptin-shi}{then-\lsc{evr}}\morglo{pay-ta-qa}{he-\lsc{acc}-\lsc{top}}\morglo{mana}{no}\morglo{tarpu-na-n-ta}{plant-\lsc{nmlz}-\lsc{3}-\lsc{acc}}\morglo{muna-sa-chu}{want-\lsc{npst}-\lsc{neg}}}%morpheme+gloss
\glotran{Then, they say, they didn’t \pb{want him to plant}.}{}%eng+spa trans
{}{}%rec - time

\noindent
\phono{-na} nominalizations, relative to the event of the main clause, refer to actions still to be completed~(\ref{Glo3:Mansanapaq}),~(\ref{Glo3:laqyarushaq}).\\

% 16
\gloexe{Glo3:Mansanapaq}{}{amv}%
{\pb{Mansanapaq}ña wak turun kayan.}%amv que first line
{\morglo{mansa-na-paq-ña}{tame-\lsc{nmlz}-\lsc{purp}-\lsc{disc}}\morglo{wak}{\lsc{dem.d}}\morglo{turu-n}{bull-\lsc{3}}\morglo{ka-ya-n}{be-\lsc{prog}-\lsc{3}}}%morpheme+gloss
\glotran{That bull is \pb{to be tamed}/for taming already.}{}%eng+spa trans
{}{}%rec - time

% 17
\gloexe{Glo3:laqyarushaq}{}{amv}%
{Ñuqa laqyarushaq sikipaq. Kiputaqa. \pb{Laqyapana}sh kayan.}%amv que first line
{\morglo{ñuqa}{I}\morglo{laqya-ru-shaq}{slap-\lsc{urgt}-\lsc{1.fut}}\morglo{siki-paq}{behind-\lsc{loc}}\morglo{Kipu-ta-qa}{Kipu-\lsc{acc}-\lsc{top}}\morglo{laqya-pa-na-sh}{slap-\lsc{repet}-\lsc{nmlz}-\lsc{evr}}\morglo{ka-ya-n}{be-\lsc{prog}-\lsc{3}}}%morpheme+gloss
\glotran{I’m going to slap him on the behind. Kipu [a dog]. It’s there to be hit.}{}%eng+spa trans
{}{}%rec - time

\subsubsection{Agentive \phono{-q}}\index[sub]{substantive!agentive}
\phono{-q} is agentive, deriving nouns that refer to the agent of the verb to which it attaches (\phono{michi-q} ‘shepherd’, \phono{ara-q} ‘plower’)~(\ref{Glo3:Qaripis}--\ref{Glo3:Imaynataqwak}).\\

% 1
\gloexe{Glo3:Qaripis}{}{ach}%
{Qaripis kanmi \pb{wawachikuq}. Wawachin hapishpa.}%ach que first line
{\morglo{qari-pis}{man-\lsc{add}}\morglo{ka-n-mi}{be-\lsc{3}-\lsc{evd}}\morglo{wawa-chi-ku-q}{give.birth-\lsc{caus}-\lsc{refl}-\lsc{ag}}\morglo{wawa-chi-n}{give.birth-\lsc{caus}-\lsc{3}}\morglo{hapi-shpa}{grab-\lsc{subis}}}%morpheme+gloss
\glotran{There are also men \pb{midwives}. Holding on, they birth the baby.}{}%eng+spa trans
{}{}%rec - time

% 2
\gloexe{Glo3:Manammunaq}{}{ch}%
{\pb{Manam munaq}kunakta pushakuyan.}%ch que first line
{\morglo{mana-m}{no-\lsc{evd}}\morglo{muna-q-kuna-kta}{want-\lsc{ag}-\lsc{pl}-\lsc{acc}}\morglo{pusha-ku-ya-n}{bring.along-\lsc{refl}-\lsc{prog}-\lsc{3}}}%morpheme+gloss
\glotran{They’re bringing along \pb{people who don’t want to}.}{}%eng+spa trans
{}{}%rec - time

% 3
\gloexe{Glo3:bandiduqa}{}{ach}%
{Wak bandiduqa munarqachu manash wawayuqta. \pb{Wawapakuq}triki kidarqa.}%ach que first line
{\morglo{wak}{\lsc{dem.d}}\morglo{bandidu-qa}{bastard-\lsc{top}}\morglo{muna-rqa-chu}{want-\lsc{pst}-\lsc{neg}}\morglo{mana-sh}{want-\lsc{evr}}\morglo{wawa-yuq-ta}{baby-\lsc{poss}-\lsc{acc}}\morglo{wawa-paku-q-tri-ki}{baby-\lsc{mutben}-\lsc{ag}-\lsc{evc}-\lsc{ki}}\morglo{kida-rqa}{remain-\lsc{pst}}}%morpheme+gloss
\glotran{That bastard didn’t want [a woman] with a baby, they say. She remained a \pb{single mother}, for sure.}{}%eng+spa trans
{}{}%rec - time

% 4
\gloexe{Glo3:Imaynataqwak}{}{amv}%
{¿Imaynataq wak \pb{miyrdaq} ganayawan?}%amv que first line
{\morglo{imayna-taq}{how-\lsc{seq}}\morglo{wak}{\lsc{dem.d}}\morglo{miyrda-q}{shit-\lsc{ag}}\morglo{gana-ya-wa-n?}{win-\lsc{prog}-\lsc{1.obj}-\lsc{3}}}%morpheme+gloss
\glotran{How is that \pb{shithead} beating me?}{}%eng+spa trans
{}{}%rec - time

\noindent
\phono{-q} nominalizations may form adjectival and relative clauses (\phono{chinka-ku-q} \phono{pashña} ‘the lost girl’, ‘the girl who was lost’)~(\ref{Glo3:Trabahapakuya}--\ref{Glo3:Manarikchaq}).\\

% 5
\gloexe{Glo3:Trabahapakuya}{}{ch}%
{Trabahapakuya: llapan \pb{rigakuq} \pb{luna}. Trabahaya:.}%ch que first line
{\morglo{trabaha-paku-ya-:}{work-\lsc{mutben}-\lsc{prog}-\lsc{1}}\morglo{llapa-n}{all-\lsc{3}}\morglo{riga-ku-q}{irrigate-\lsc{refl}-\lsc{ag}}\morglo{luna}{person}\morglo{trabaha-ya-:}{work-\lsc{prog}-\lsc{1}}}%morpheme+gloss
\glotran{All the \pb{people who water} are working, we’re working.}{}%eng+spa trans
{}{}%rec - time

% 6
\gloexe{Glo3:Istudyaq}{}{amv}%
{\pb{Istudyaq} \pb{wambra}kunapaqshi mas mimuryanpaq.}%amv que first line
{\morglo{istudya-q}{study-\lsc{ag}}\morglo{wambra-kuna-paq-shi}{child-\lsc{pl}-\lsc{ben}-\lsc{evr}}\morglo{mas}{more}\morglo{mimurya-n-paq}{memory-\lsc{3}-\lsc{purp}}}%morpheme+gloss
\glotran{For the \pb{children who study}, they say, so that they have more memory.}{}%eng+spa trans
{}{}%rec - time

% 7
\gloexe{Glo3:Maqtawan}{}{ach}%
{\pb{maqtawan pashña chinkakuq}qa}%ach que first line
{\morglo{maqta-wan}{young.man-\lsc{instr}}\morglo{pashña}{girl}\morglo{chinka-ku-q-qa}{get.lost-\lsc{refl}-\lsc{ag}-\lsc{top}}}%morpheme+gloss
\glotran{\pb{the boy and the girl who were lost}}{}%eng+spa trans
{}{}%rec - time

% 8
\gloexe{Glo3:Manarikchaq}{}{sp}%
{\pb{mana rikchaq} \pb{runa}kuna}%sp que first line
{\morglo{mana}{no}\morglo{rikcha-q}{go-\lsc{ag}}\morglo{runa-kuna}{person-pl}}%morpheme+gloss
\glotran{the \pb{people who aren’t going}}{}%eng+spa trans
{}{}%rec - time

\noindent
With verbs of movement, \phono{-q} forms complement clauses indicating the purpose of the displacement (\phono{taki-q} \phono{hamu-nqa} ‘they will come to sing’)~(\ref{Glo3:Maskakuq}--\ref{Glo3:Hakumichiq}).\\

% 9
\gloexe{Glo3:Maskakuq}{}{lt}%
{\pb{Maskakuq} wak vikuñachatam \pb{wakchakuq} \pb{ri}tamunki.}%lt que first line
{\morglo{maska-ku-q}{look.for-\lsc{refl}-\lsc{ag}}\morglo{wak}{\lsc{dem.d}}\morglo{vikuña-cha-ta-m}{vicuña-\lsc{dim}-\lsc{acc}-\lsc{evd}}\morglo{wakcha-ku-q}{raise-\lsc{refl}-\lsc{ag}}\morglo{ri-tamu-nki}{go-\lsc{irrev}-\lsc{1}}}%morpheme+gloss
\glotran{You \pb{left to look} for that little vicuña to \pb{domesticate}.}{}%eng+spa trans
{}{}%rec - time

% 10
\gloexe{Glo3:Misa}{}{ch}%
{Misa \pb{lulaq shamu}n.}%ch que first line
{\morglo{misa}{mass}\morglo{lula-q}{make-\lsc{ag}}\morglo{shamu-n}{come-\lsc{3}}}%morpheme+gloss
\glotran{They \pb{come to hold} mass.}{}%eng+spa trans
{}{}%rec - time

% 11
\gloexe{Glo3:Hakumichiq}{}{lt}%
{¡Haku michiq! Michimushun chay llamata.}%lt que first line
{\morglo{haku}{let’s}\morglo{michi-q}{pasture-\lsc{ag}}\morglo{michi-mu-shun}{pasture-\lsc{cisl}-\lsc{1pl.fut}}\morglo{chay}{\lsc{dem.d}}\morglo{llama-ta}{llama-\lsc{acc}}}%morpheme+gloss
\glotran{Let’s \pb{[go to] herd}! We’ll herd those llamas.}{}%eng+spa trans
{}{}%rec - time

\noindent
With the verb \phono{kay} ‘be’ \phono{-q} forms the habitual past (\phono{asi-ku-q} \phono{ka-nki} ‘you used to laugh’)~(\ref{Glo3:Unayqa}--\ref{Glo3:Sirdallawan}) (see \sectref{par:iterative}).\\

% 12
\gloexe{Glo3:Unayqa}{}{amv}%
{Unayqa paykunaqa~\dots{} mantilta ruwa\pb{q}, mantilta burda\pb{q}, unayqa.}%
{\morglo{unay-qa}{long.ago-\lsc{top}}\morglo{pay-kuna-qa}{he-\lsc{pl}-\lsc{top}}\morglo{mantil-ta}{table.cloth-\lsc{acc}}\morglo{ruwa-q}{make-\lsc{ag}}\morglo{mantil-ta}{table.cloth-\lsc{acc}}\morglo{burda-q}{embroider-\lsc{ag}}\morglo{unay-qa}{long.ago-\lsc{top}}}%morpheme+gloss
\glotran{Formerly, they \pb{used to} make table cloths; they \pb{used to} embroider table cloths, formerly.}%eng
{‘Antés ellas hacían manteles; ellas bordaban manteles, antes’.}%spa
{}{}%{Madean\_VDE\_Various}{06:02--06:09}

% 13
\gloexe{Glo3:Huybisninpa}{}{ach}%
{Huybisninpa dumingunpa kisuta \pb{apaq ka:} ishkay.}%ach que first line
{\morglo{huybis-ni-n-pa}{Thursday-\lsc{euph}-\lsc{3}-\lsc{loc}}\morglo{dumingu-n-pa}{Sunday-\lsc{3}-\lsc{loc}}\morglo{kisu-ta}{cheese-\lsc{acc}}\morglo{apa-q}{bring-\lsc{ag}}\morglo{ka-:}{be-\lsc{1}}\morglo{ishkay}{two}}%morpheme+gloss
\glotran{On Thursdays and Sundays, I \pb{used to bring} two cheeses [to sell].}{}%eng+spa trans
{}{}%rec - time

% 14
\gloexe{Glo3:Sirdallawan}{}{amv}%
{Sirdallawan \pb{chumakuq} \pb{kanchik}, kaspichallawan \pb{aychiq} \pb{kanchik}. Winku purucham \pb{kaq}. Antis.}%amv que first line
{\morglo{sirda-lla-wan}{bristle-\lsc{rstr}-\lsc{instr}}\morglo{chuma-ku-q}{strain-\lsc{refl}-\lsc{ag}}\morglo{ka-nchik,}{be-\lsc{1pl}}\morglo{kaspi-cha-lla-wan}{stick-\lsc{dim}-\lsc{rstr}-\lsc{instr}}\morglo{aychi-q}{stir-\lsc{ag}}\morglo{ka-nchik}{	be-\lsc{1pl}}\morglo{winku}{crooked}\morglo{puru-cha-m}{pot-\lsc{dim}-\lsc{evd}}\morglo{ka-q}{be-\lsc{ag}}\morglo{antis}{before}}%morpheme+gloss
\glotran{\pb{We used to strain} it with just bristles, \pb{we used to stir} it with just a stick. \pb{There used to be} a crooked little bottle. Before.}{}%eng+spa trans
{}{}%rec - time

\subsubsection{Perfective \phono{-sHa}}\index[sub]{substantive!perfective}
\phono{-sHa} is perfective, deriving stative participles. It is realized as \phono{-sa} in \ACH, \AMV{}, and \SP{} and as \phono{-sha} in \LT{} and \CH. \phono{-sHa} nominalizations form adjectives (\phono{chaki-sa} ‘dried’)~(\ref{Glo3:Mandilllaykunaqa}--\ref{Glo3:Wakrunapa}) as well as relative (\phono{apa-sa-y} ‘that I bring’)~(\ref{Glo3:dividisan}--\ref{Glo3:Ratuskamanshi}), and complement clauses (\phono{atipa-sha-y-ta} ‘what I can’)~(\ref{Glo3:Imatataq}--\ref{Glo3:Nuqapataqa}).\\

% 1
\gloexe{Glo3:Mandilllaykunaqa}{}{amv}%
{Mandilllaykunaqa \pb{chakisa} kayan.}%amv que first line
{\morglo{mandil-lla-y-kuna-qa}{apron-\lsc{rstr}-\lsc{1}-\lsc{pl}-\lsc{top}}\morglo{chaki-sa}{dry-\lsc{prf}}\morglo{ka-ya-n}{be-\lsc{prog}-\lsc{3}}}%morpheme+gloss
\glotran{My aprons and things with them are \pb{dry}.}{}%eng+spa trans
{}{}%rec - time

% 2
\gloexe{Glo3:Wakrunapa}{}{ach}%
{Wak runapa trakinqa \pb{punkisa}m kayan tulluntri \pb{kuyusa} kayan.}%ach que first line
{\morglo{wak}{\lsc{dem.d}}\morglo{runa-pa}{person-\lsc{gen}}\morglo{traki-n-qa}{foot-\lsc{3}-\lsc{top}}\morglo{punki-sa-m}{swell-\lsc{prf}-\lsc{evd}}\morglo{ka-ya-n}{be-\lsc{prog}-\lsc{3}}\morglo{tullu-n-tri}{bone-\lsc{3}-\lsc{evc}}\morglo{kuyu-sa}{move-\lsc{prf}}\morglo{ka-ya-n}{be-\lsc{prog}-\lsc{3}}}%morpheme+gloss
\glotran{That person’s foot is \pb{swollen}, the bone must be \pb{moved} [out of place].}{}%eng+spa trans
{}{}%rec - time

% 3
\gloexe{Glo3:dividisan}{}{sp}%
{Chay ganaw \pb{dividisan}wan rikisiyantri.}%sp que first line
{\morglo{chay}{\lsc{dem.d}}\morglo{ganaw}{cattle}\morglo{dividi-sa-n-wan}{devide-\lsc{prf}-\lsc{3}-\lsc{instr}}\morglo{rikisi-ya-n-tri}{get.rich-\lsc{prog}-\lsc{3}-\lsc{evc}}}%morpheme+gloss
\glotran{They must be getting rich with the cattle \pb{that they divided up} [among themselves].}{}%eng+spa trans
{}{}%rec - time

% 4
\gloexe{Glo3:Pampakurun}{}{amv}%
{Pampakurun matraymanqa chay \pb{wañusan} tardiqa.}%amv que first line
{\morglo{pampa-ku-ru-n}{bury-\lsc{refl}-\lsc{urgt}-\lsc{3}}\morglo{matray-man-qa}{cave-\lsc{all}-\lsc{top}}\morglo{chay}{\lsc{dem.d}}\morglo{wañu-sa-n}{die-\lsc{prf}-\lsc{3}}\morglo{tardi-qa}{afternoon}}%morpheme+gloss
\glotran{They buried him in a cave the afternoon \pb{that he died}.}{}%eng+spa trans
{}{}%rec - time

% 5
\gloexe{Glo3:pasamashanchik}{}{ch}%
{Unay imas \pb{pasamashanchik}~\dots}%ch que first line
{\morglo{unay}{before}\morglo{ima-s}{what-\lsc{add}}\morglo{pasa-ma-sha-nchik}{pass-\lsc{1.obj}-\lsc{prf}-\lsc{1pl}}}%morpheme+gloss
\glotran{Before, anything \pb{that happened} to us~\dots}{}%eng+spa trans
{}{}%rec - time

% 6
\gloexe{Glo3:Kalamina}{}{lt}%
{kalamina \pb{rantishanchikkuna}}%lt que first line
{\morglo{kalamina}{corrugated.iron}\morglo{ranti-sha-nchik-kuna}{buy-\lsc{prf}-\lsc{1pl}-\lsc{pl}}}%morpheme+gloss
\glotran{the tin roofing \pb{that we bought}}{}%eng+spa trans
{}{}%rec - time

% 7
\gloexe{Glo3:Ratuskamanshi}{}{amv}%
{Ratuskamanshi kisuta \pb{ruwasayki}ta qawanqa.}%amv que first line
{\morglo{ratus-kaman-shi}{moments-\lsc{lim}-\lsc{evr}}\morglo{kisu-ta}{cheese-\lsc{acc}}\morglo{ruwa-sa-yki-ta}{make-\lsc{prf}-\lsc{2}-\lsc{acc}}\morglo{qawa-nqa}{see-\lsc{3.fut}}}%morpheme+gloss
\glotran{A little later, she says, she’ll see the cheese \pb{that you made}.}{}%eng+spa trans
{}{}%rec - time

% 8
\gloexe{Glo3:Imatataq}{}{amv}%
{¿Imatataq kanan ñuqa Lutupa ubihawan \pb{yatrasa}yta willakushaq?}%amv que first line
{\morglo{ima-ta-taq}{what-\lsc{acc}-\lsc{seq}}\morglo{kanan}{now}\morglo{ñuqa}{I}\morglo{Lutu-pa}{Lutu-\lsc{loc}}\morglo{ubiha-wan}{sheep-\lsc{instr}}\morglo{yatra-sa-y-ta}{live-\lsc{prf}-\lsc{1}-\lsc{acc}}\morglo{willa-ku-shaq}{tell-\lsc{refl}-\lsc{1.fut}}}%morpheme+gloss
\glotran{Now what am I going to tell you about \pb{what I lived} in Lutu with my sheep?}{}%eng+spa trans
{}{}%rec - time

% 9
\gloexe{Glo3:Luchashaq}{}{lt}%
{Luchashaq. \pb{Atipashay}tatrik ruwakushaq.}%lt que first line
{\morglo{lucha-shaq}{fight-\lsc{1.fut}}\morglo{atipa-sha-y-ta-tri-k}{be.able-\lsc{prf}-\lsc{1}-\lsc{acc}-\lsc{evc}-\lsc{ik}}\morglo{ruwa-ku-shaq}{make-\lsc{refl}-\lsc{1.fut}}}%morpheme+gloss
\glotran{I’ll fight. I’ll do \pb{what I can}.}{}%eng+spa trans
{}{}%rec - time

% 10
\gloexe{Glo3:Nuqapataqa}{}{lt}%
{Ñuqapataqa silinsyu kaptin \pb{munashan}taña ruwayan.}%lt que first line
{\morglo{ñuqa-pa-ta-qa}{I-\lsc{gen}-\lsc{acc}-\lsc{top}}\morglo{silinsyu}{abandoned}\morglo{ka-pti-n}{be-\lsc{subds}-\lsc{3}}\morglo{muna-sha-n-ta-ña}{want-\lsc{prf}-\lsc{3}-\lsc{acc}-\lsc{disc}}\morglo{ruwa-ya-n}{make-\lsc{prog}-\lsc{3}}}%morpheme+gloss
\glotran{When it falls silent, they’re doing \pb{what they want} to my things.}{}%eng+spa trans
{}{}%rec - time

\noindent
\phono{-sHa} complement clauses are common with the verbs \phono{yatra-} ‘know’, \phono{qunqa-} ‘forget’, \phono{qawa} ‘see’ and \phono{uyaRi-} ‘hear’ (\phono{upya-sa-n-ta} \phono{uyari-rqa-ni} ‘I heard that he drank’)~(\ref{Glo3:Nuqaqa}).\\

% 11
\gloexe{Glo3:Nuqaqa}{}{amv}%
{Ñuqaqa wambran \pb{qipikusan}ta qawarqanichu.}%amv que first line
{\morglo{ñuqa-qa}{I-\lsc{top}}\morglo{wambra-n}{child-\lsc{3}}\morglo{qipi-ku-sa-n-ta}{carry-\lsc{refl}-\lsc{prf}-\lsc{3}-\lsc{acc}}\morglo{qawa-rqa-ni-chu}{see-\lsc{pst}-\lsc{1}-\lsc{neg}}}%morpheme+gloss
\glotran{I didn’t see \pb{that she carried} her baby.}{}%eng+spa trans
{}{}%rec - time

\noindent
As substantives, they are inflected with possessive suffixes, not verbal suffixes (\phono{ranti-sa-\pb{yki}} \phononb{*ranti-sa-\pb{nki}} ‘that you sold’); these may be reinforced with possessive pronouns (\phono{\pb{qam-pa}} \phononb{ranti-sa-yki} ‘that \emph{you} sold’)~(\ref{Glo3:rantikurasayki}).\\

% 12
\gloexe{Glo3:rantikurasayki}{}{amv}%
{Qam\pb{pa} \pb{rantikurasayki}yá chay shakash.}%amv que first line
{\morglo{qam-pa}{you-\lsc{gen}}\morglo{rantiku-ra-sayki-yá}{sell-\lsc{urgt}-\lsc{2>1}-\lsc{emph}}\morglo{chay}{\lsc{dem.d}}\morglo{shakash}{guinea.pig}}%morpheme+gloss
\glotran{That guinea pig \pb{that \emph{you} sold me}.}{}%eng+spa trans
{}{}%rec - time

\noindent
\phono{-sHa} may also form nouns referring to the place where an event, \lsc{e}, occurs (\phono{dipurti} \phono{ka-sha-n} ‘where there are sports’)~(\ref{Glo3:Wambraqa}--\ref{Glo3:Riyasan}).\\

% 13
\gloexe{Glo3:Wambraqa}{}{sp}%
{Wambraqa \pb{pukllayasan}pa tutaykarachin.}%sp que first line
{\morglo{wambra-qa}{child-\lsc{top}}\morglo{puklla-ya-sa-n-pa}{play-\lsc{prog}-\lsc{prf}-\lsc{3}-\lsc{loc}}\morglo{tuta-yka-ra-chi-n}{night-\lsc{excep}-\lsc{urgt}-\lsc{caus}-\lsc{3}}}%morpheme+gloss
\glotran{Night fell \pb{where the girls were playing}.}{}%eng+spa trans
{}{}%rec - time

% 14
\gloexe{Glo3:Tilivisyunta}{}{ch}%
{Tilivisyunta likakuyan piluta pukllaqkunaktam maytraw \pb{dipurti kashan}kunakta.}%ch que first line
{\morglo{tilivisyun-ta}{television-\lsc{acc}}\morglo{lika-ku-ya-n}{look-\lsc{refl}-\lsc{prog}-\lsc{3}}\morglo{piluta}{ball}\morglo{puklla-q-kuna-kta-m}{play-\lsc{ag}-\lsc{pl}-\lsc{acc}-\lsc{evd}}\morglo{may-traw}{where-\lsc{loc}}\morglo{dipurti}{sport}\morglo{ka-sha-n-kuna-kta}{be-\lsc{prf}-\lsc{3}-\lsc{pl}-\lsc{acc}}}%morpheme+gloss
\glotran{They’re watching television -- the ball-players and \pb{where there are sports}.}{}%eng+spa trans
{}{}%rec - time

% 15
\gloexe{Glo3:Riyasan}{}{amv}%
{\pb{Riyasan}piqa trayarun, pwintiman.}%amv que first line
{\morglo{ri-ya-sa-n-pi-qa}{go-\lsc{prog}-\lsc{prf}-\lsc{3}-\lsc{loc}-\lsc{top}}\morglo{traya-ru-n,}{arrive-\lsc{urgt}-\lsc{3}}\morglo{pwinti-man}{bridge-\lsc{all}}}%morpheme+gloss
\glotran{He arrived \pb{where he was going}, at a bridge.}{}%eng+spa trans
{}{}%rec - time

\noindent
\phono{-sHa} nominalizations, relative to the \lsc{e} of the main clause, refer to actions already completed~(\ref{Glo3:kutishqa}),~(\ref{Glo3:yaykukuntri}).\\

% 16
\gloexe{Glo3:kutishqa}{}{amv}%
{Yapa kutishqa \pb{awakusa}nman.}%amv que first line
{\morglo{yapa}{again}\morglo{kuti-shqa}{return-\lsc{subis}}\morglo{awa-ku-sa-n-man}{weave-\lsc{refl}-\lsc{prf}-\lsc{3}-\lsc{all}}}%morpheme+gloss
\glotran{When she returned again to \pb{what/where she had woven}.}{}%eng+spa trans
{}{}%rec - time

% 17
\gloexe{Glo3:yaykukuntri}{}{amv}%
{¿Pi yaykukuntri? Mana ya yatranichu pi \pb{kasha}ntapis.}%xx que first line
{\morglo{pi}{who}\morglo{yayku-ku-n-tri}{enter-\lsc{refl}-\lsc{3}-\lsc{evc}}\morglo{mana}{mana}\morglo{ya}{\lsc{emph}}\morglo{yatra-ni-chu}{know-\lsc{1}-\lsc{neg}}\morglo{pi}{who}\morglo{ka-sha-n-ta-pis}{be-\lsc{prf}-\lsc{3}-\lsc{acc}-\lsc{add}}}%morpheme+gloss
\glotran{Who would have entered? I don’t know \pb{who it was}, either.}{}%eng+spa trans
{}{}%rec - time

\subsubsection{Infinitive \phono{-y}}\label{sssc:inf}\index[sub]{substantive!infinitive}
\phono{-y} indicates the infinitive or what in English would be a gerund (\phono{tushu-y} ‘to dance, dancing’)~(\ref{Glo3:punuy}),~(\ref{Glo3:Paqwayannam}).\\

% 1
\gloexe{Glo3:punuy}{}{amv}%
{Ni \pb{puñuy} ni \pb{mikuy}.}%amv que first line
{\morglo{ni}{nor}\morglo{puñu-y}{sleep-\lsc{inf}}\morglo{ni}{nor}\morglo{miku-y}{eat-\lsc{inf}}}%morpheme+gloss
\glotran{Neither sleep\pb{ing} nor eat\pb{ing}.}{}%eng+spa trans
{}{}%rec - time

% 2
\gloexe{Glo3:Paqwayannam}{}{ch}%
{Paqwayanñam \pb{talpukuy}.}%ch que first line
{\morglo{paqwa-ya-n-ña-m}{finish-\lsc{prog}-\lsc{3}-\lsc{disc}-\lsc{evd}}\morglo{talpu-ku-y}{plant-\lsc{refl}-\lsc{inf}}}%morpheme+gloss
\glotran{The plant\pb{ing} is finishing up.}{}%eng+spa trans
{}{}%rec - time

\noindent
\phono{-y} nominalizations may refer to the object or event in which the verb stem is realized (\phono{ishpa-} ‘urinate’~→~\phono{ishpa-y} ‘urine’; \phono{nana-} ‘hurt’~→~\phono{nana-y} ‘pain’; \phono{rupa-} ‘burn’~→ \phono{rupa-y} ‘sunshine’)~(\ref{Glo3:Warminpa}--\ref{Glo3:Aligrakuyan}).\\

% 3
\gloexe{Glo3:Warminpa}{}{amv}%
{Warminpa \pb{ishpay}nintash tuman.}%amv que first line
{\morglo{warmi-n-pa}{woman-\lsc{3}-\lsc{gen}}\morglo{ishpa-y-ni-n-ta-sh}{urinate-\lsc{inf}-\lsc{euph}-\lsc{3}-\lsc{acc}-\lsc{evr}}\morglo{tuma-n}{drink-\lsc{3}}}%morpheme+gloss
\glotran{He drinks his wife’s \pb{urine}, they say.}{}%eng+spa trans
{}{}%rec - time

% 4
\gloexe{Glo3:Traki}{}{amv}%
{Traki \pb{nanay}wan karqani.}%amv que first line
{\morglo{traki}{foot}\morglo{nana-y-wan}{hurt-\lsc{inf}-\lsc{instr}}\morglo{ka-rqa-ni}{be-\lsc{pst}-\lsc{1}}}%morpheme+gloss
\glotran{I’ve had foot \pb{pain}.}{}%eng+spa trans
{}{}%rec - time

% 5
\gloexe{Glo3:Tutal}{}{ach}%
{Tutal \pb{suday}llaña hamukuyan kwirpunchikpapis “¡Chaq! ¡Chaq! ¡Chaq!” sutukuyan \pb{suday}niki.}%ach que first line
{\morglo{tutal}{completely}\morglo{suda-y-lla-ña}{sweat-\lsc{inf}-\lsc{rstr}-\lsc{disc}}\morglo{hamu-ku-ya-n}{come-\lsc{refl}-\lsc{prog}-\lsc{3}}\morglo{kwirpu-nchik-pa-pis}{body-\lsc{1pl}-\lsc{loc}-\lsc{add}}\morglo{chaq}{tak}\morglo{chaq}{tak}\morglo{chaq}{tak}\morglo{sutu-ku-ya-n}{drip-\lsc{refl}-\lsc{prog}-\lsc{3}}\morglo{suda-y-ni-ki}{sweat-\lsc{inf}-\lsc{euph}-\lsc{2}}}%morpheme+gloss
\glotran{Just a whole lot of \pb{sweat} is coming out on our bodies --~“\emph{Chak! Chak! Chak!}”~-- your \pb{sweat} is dripping.}{}%eng+spa trans
{}{}%rec - time

% 6
\gloexe{Glo3:Uktubri}{}{ch}%
{¿Uktubri \pb{paqway}piñachu hamunki?}%ch que first line
{\morglo{uktubri}{October}\morglo{paqwa-y-pi-ña-chu}{finish-\lsc{inf}-\lsc{loc}-\lsc{disc}-\lsc{q}}\morglo{hamu-nki}{come-\lsc{2}}}%morpheme+gloss
\glotran{Are you coming at \pb{the end} of October?}{}%eng+spa trans
{}{}%rec - time

% 7
\gloexe{Glo3:Aligrakuyan}{}{amv}%
{Aligrakuyan suygran wañukusantatr. Manayá \pb{pampakuyninpa} karqachu, ¿aw?}%amv que first line
{\morglo{aligra-ku-ya-n}{happy-\lsc{refl}-\lsc{prog}-\lsc{3}}\morglo{suygra-n}{mother.in.law-\lsc{3}}\morglo{wañu-ku-sa-n-ta-tr}{die-\lsc{refl}-\lsc{prf}-\lsc{3}-\lsc{acc}-\lsc{evc}}\morglo{mana-yá}{no-\lsc{emph}}\morglo{pampa-ku-y-ni-n-pa}{bury-\lsc{refl}-\lsc{inf}-\lsc{euph}-\lsc{3}-\lsc{loc}}\morglo{ka-rqa-chu}{be-\lsc{pst}-\lsc{q}}\morglo{aw}{yes}}%morpheme+gloss
\glotran{He must be very happy his mother-in-law died. He wasn’t at her \pb{burial}, was he?}{}%eng+spa trans
{}{}%rec - time

\noindent
\phono{-y} nominalizations form adjectival and relative clauses (\phono{ranti-y kahun} ‘bought casket’, \phono{yanu-ku-y tardi} ‘the afternoon that we cook’)~(\ref{Glo3:Rantiy}--\ref{Glo3:yanukuy}) and infinitive complement clauses (\phono{waqa-y-ta qalla-ku-n} ‘it started to wail’)~(\ref{Glo3:qallakun}).\\

% 8
\gloexe{Glo3:Rantiy}{}{amv}%
{\pb{Rantiy} kahun mana yaykunchu.}%amv que first line
{\morglo{ranti-y}{buy-\lsc{inf}}\morglo{kahun}{coffin}\morglo{mana}{no}\morglo{yayku-n-chu}{enter-\lsc{3}-\lsc{neg}}}%morpheme+gloss
\glotran{\pb{Bought} coffins won’t fit it.}{}%eng+spa trans
{}{}%rec - time

% 9
\gloexe{Glo3:Waqtakunata}{}{ach}%
{Waqtakunata lluqsishpa runas \pb{puñuy}.}%ach que first line
{\morglo{waqta-kuna-ta}{hillside-\lsc{pl}-\lsc{acc}}\morglo{lluqsi-shpa}{go.out-\lsc{subis}}\morglo{runa-s}{person-\lsc{add}}\morglo{puñu-y}{sleep-\lsc{inf}}}%morpheme+gloss
\glotran{The people, too, \pb{asleep}, they came out on the hillsides.}{}%eng+spa trans
{}{}%rec - time

% 10
\gloexe{Glo3:yanukuy}{}{amv}%
{Chay \pb{yanukuy} tardish almaqa trayamun.}%amv que first line
{\morglo{chay}{\lsc{dem.d}}\morglo{yanu-ku-y}{cook-\lsc{refl}-\lsc{inf}}\morglo{tardi-sh}{afternoon-\lsc{evr}}\morglo{alma-qa}{soul-\lsc{top}}\morglo{traya-mu-n}{arrive-\lsc{cisl}-\lsc{3}}}%morpheme+gloss
\glotran{The souls arrive on the afternoon \pb{that we cook}, they say.}{}%eng+spa trans
{}{}%rec - time

% 11
\gloexe{Glo3:qallakun}{}{sp}%
{\pb{Waqay}ta qallakun, “¡Oooh oooohh oooohhhh ooh ooh!”}%sp que first line
{\morglo{waqa-y-ta}{cry-\lsc{inf}-\lsc{acc}}\morglo{qalla-ku-n}{start-\lsc{refl}-\lsc{3}}\morglo{oooh}{oooh}\morglo{oooohh}{oooohh}\morglo{oooohhhh}{oooohhhh}\morglo{ooh}{ooh}\morglo{ooh}{ooh}}%morpheme+gloss
\glotran{It started \pb{to wail}, “\emph{Oooh oooohh oooohhhh ooh ooh!}”}{}%eng+spa trans
{}{}%rec - time

\noindent
The latter are particularly common with the auxiliary verbs \phono{muna-} ‘want,’ \phono{atipa-} ‘be able,’ and \phono{yatra-} ‘know’ (\phono{iskribi-y-ta muna-ni} ‘I want to write’)~(\ref{Glo3:Mananam}--\ref{Glo3:risakuyta}).\\

% 12
\gloexe{Glo3:Mananam}{}{lt}%
{Manañam \pb{diskutiy}ta ñuqa \pb{muna}nichu kayna.}%lt que first line
{\morglo{mana-ña-m}{no-\lsc{disc}-\lsc{evd}}\morglo{diskuti-y-ta}{dispute-\lsc{inf}-\lsc{acc}}\morglo{ñuqa}{I}\morglo{muna-ni-chu}{want-\lsc{1}-\lsc{neg}}\morglo{kayna}{thus}}%morpheme+gloss
\glotran{I don’t \pb{want to fight} about it like this any more.}{}%eng+spa trans
{}{}%rec - time

% 13
\gloexe{Glo3:Kukata}{}{amv}%
{¿Kukata \pb{akuykuyta muna}nkichu?}%amv que first line
{\morglo{kuka-ta}{coca-\lsc{acc}}\morglo{aku-yku-y-ta}{chew-\lsc{excep}-\lsc{inf}-\lsc{acc}}\morglo{muna-nki-chu}{want-\lsc{2}-\lsc{q}}}%morpheme+gloss
\glotran{Do you \pb{want to chew} coca?}{}%eng+spa trans
{}{}%rec - time

% 14
\gloexe{Glo3:vakaypa}{}{amv}%
{Wak vakaypa atakanmi mal kayan \pb{puriyta} \pb{atipanchu}.}%amv que first line
{\morglo{wak}{\lsc{dem.d}}\morglo{vaka-y-pa}{cow-\lsc{1}-\lsc{gen}}\morglo{ataka-n-mi}{leg-\lsc{3}-\lsc{evd}}\morglo{mal}{bal}\morglo{ka-ya-n}{be-\lsc{prog}-\lsc{3}}\morglo{puri-y-ta}{walk-\lsc{inf}-\lsc{acc}}\morglo{atipa-n-chu}{be.able-\lsc{3}-\lsc{neg}}}%morpheme+gloss
\glotran{My cow’s leg is hurt -- she \pb{can’t walk}.}{}%eng+spa trans
{}{}%rec - time

% 15
\gloexe{Glo3:Iskribiy}{}{ch}%
{\pb{Iskribiy}tapis \pb{usachi}nichu ni \pb{firmay}tapis. Total analfabitu.}%ch que first line
{\morglo{iskribi-y-ta-pis}{write-\lsc{inf}-\lsc{acc}-\lsc{add}}\morglo{usachi-ni-chu}{be.able-\lsc{1}-\lsc{neg}}\morglo{ni}{nor}\morglo{firma-y-ta-pis}{sign-\lsc{inf}-\lsc{acc}-\lsc{add}}\morglo{total}{totally}\morglo{analfabitu}{illiterate}}%morpheme+gloss
\glotran{I \pb{can’t write or sign} [my name], either. Completely illiterate.}{}%eng+spa trans
{}{}%rec - time

% 16
\gloexe{Glo3:risakuyta}{}{sp}%
{Mana \pb{risakuy}ta \pb{yatra}rachu. Satanaswan yatrara.}%sp que first line
{\morglo{mana}{no}\morglo{risa-ku-y-ta}{pray-\lsc{refl}-\lsc{inf}-\lsc{acc}}\morglo{yatra-ra-chu}{know-\lsc{pst}-\lsc{neg}}\morglo{Satanas-wan}{Satan-\lsc{instr}}\morglo{yatra-ra}{live-\lsc{pst}}}%morpheme+gloss
\glotran{They didn’t \pb{know how to pray}. They lived with Satan.}{}%eng+spa trans
{}{}%rec - time

\noindent
Infinitive complements are case-marked with accusative \phono{-ta}~(\ref{Glo3:Wakhinamana}).\\

% 17
\gloexe{Glo3:Wakhinamana}{}{amv}%
{Wakhina mana vininu tuma\pb{yta} munashpatri manam yayku\pb{yta} munanchu ubihaqa.}%amv que first line
{\morglo{wak-hina}{\lsc{dem.d}-\lsc{comp}}\morglo{mana}{no}\morglo{vininu}{poison}\morglo{tuma-y-ta}{take-\lsc{inf}-\lsc{acc}}\morglo{muna-shpa-tri}{want-\lsc{subis}-\lsc{evc}}\morglo{mana-m}{no-\lsc{evd}}\morglo{yayku-y-ta}{enter-\lsc{inf}-\lsc{acc}}\morglo{muna-n-chu}{want-\lsc{3}-\lsc{neg}}\morglo{ubiha-qa}{sheep-\lsc{top}}}%morpheme+gloss
\glotran{Like that, not wanting to drink poison, the sheep don’t want to go in.}{}%eng+spa trans
{}{}%rec - time

\noindent
In the \CH{} dialect, accusative marking in this structure is sometimes elided,~(\ref{Glo3:munanchu}).\\

% 18
\gloexe{Glo3:munanchu}{}{ch}%
{Manam lulay munanchu.}%ch que first line
{\morglo{mana-m}{no-\lsc{evd}}\morglo{lula-y}{make-\lsc{inf}}\morglo{muna-n-chu}{want-\lsc{3}-\lsc{neg}}}%morpheme+gloss
\glotran{He doesn’t want to do it.}{}%eng+spa trans
{}{}%rec - time

\subsection{Substantives derived from substantives}\label{ssec:sdfs}\index[sub]{substantive!derivation from substantives}
Four suffixes derive substantives from substantives in \SYQ{}: \phono{-kuna}, \phono{-ntin}, \phono{-sapa}, and \phono{-yuq}. The first two of these --~\phono{-kuna} and \phono{-ntin}~-- indicate accompaniment, adjacency, or completeness (\phono{llama-n-kuna} ‘with her llama’, \phono{amiga-ntin} ‘with her friends’); \phono{-yuq} and \phono{-sapa} indicate possession (\phono{llama-yuq} ‘person with llamas’, \phono{llama-sapa} ‘person with more llamas than usual’). \sectref{sssc:nonex}--\ref{sssc:poss} cover \phono{-kuna}, \phono{-ntin}, \phono{-sapa}; and \phono{-yuq}, in turn.

\subsubsection{Non-exhaustivity \phono{-kuna\tss{2}}}\label{sssc:nonex}\index[sub]{substantive!non-exhaustivity}
\phono{-kuna\tss{2}} indicates that the referent of its base is accompanied by another entity, generally of the same class (\phono{qusa-yki-\pb{kuna}} ‘your husband and all’)~(\ref{Glo3:Ispusu}--\ref{Glo3:Pachamanka}).\\

% 1
\gloexe{Glo3:Ispusu}{}{ch}%
{Ispusu:ta mama:\pb{kuna} tayta:\pb{kuna}kta qayakushpa manam~\dots{} hiwyaku:chu.}%ch que first line
{\morglo{ispusu-:-ta}{husband-\lsc{1}-\lsc{acc}}\morglo{mama-:-kuna}{mother-\lsc{1}-\lsc{pl}}\morglo{tayta-:-kuna-kta}{father-\lsc{1}-\lsc{pl}-\lsc{acc}}\morglo{qaya-ku-shpa}{call-\lsc{refl}-\lsc{subis}}\morglo{mana-m}{no-\lsc{evd}}\morglo{hiwya-ku-:-chu}{scare-\lsc{refl}-\lsc{1}-\lsc{neg}}}%morpheme+gloss
\glotran{Calling on my husbands and on my mothers and my fathers, I’m not scared.}{}%eng+spa trans
{}{}%rec - time

% 2
\gloexe{Glo3:kwirpuyki}{}{amv}%
{Chay kwirpuyki\pb{kuna} mal kanman umayki\pb{kuna} nananman.}%amv que first line
{\morglo{chay}{\lsc{dem.d}}\morglo{kwirpu-yki-kuna}{body-\lsc{2}-\lsc{pl}}\morglo{mal}{bad}\morglo{ka-n-man}{be-\lsc{3}-\lsc{cond}}\morglo{uma-yki-kuna}{head-\lsc{2}-\lsc{pl}}\morglo{nana-n-man}{hurt-\lsc{3}-\lsc{cond}}}%morpheme+gloss
\glotran{Your body \pb{among other things} could be sick; your head \pb{among other things} could hurt.}{}%eng+spa trans
{}{}%rec - time

% 3
\gloexe{Glo3:rikisunnin}{}{amv}%
{Wak rikisunnin\pb{kuna}ta narun warkurun.}%amv que first line
{\morglo{wak}{\lsc{dem.d}}\morglo{rikisun-ni-n-kuna-ta}{cheese.curd-\lsc{euph}-\lsc{3}-\lsc{pl}-\lsc{acc}}\morglo{na-ru-n}{\lsc{dmy}-\lsc{urgt}-\lsc{3}}\morglo{warku-ru-n}{hang-\lsc{urgt}-\lsc{3}}}%morpheme+gloss
\glotran{She did that, she hung up her cheese curd \pb{along with other things}.}{}%eng+spa trans
{}{}%rec - time

% 4
\gloexe{Glo3:Pachamanka}{}{sp}%
{“Pachamanka\pb{kuna} kayan alli allin mikushun kanan tardi”, nishpa.}%sp que first line
{\morglo{pachamanka-kuna}{barbecue-\lsc{pl}}\morglo{ka-ya-n}{be-\lsc{prog}-\lsc{3}}\morglo{alli}{good}\morglo{allin}{good}\morglo{miku-shun}{eat-\lsc{1pl.fut}}\morglo{kanan}{now}\morglo{tardi}{afternoon}\morglo{ni-shpa}{say-\lsc{subis}}}%morpheme+gloss
\glotran{“There’s a barbecue \pb{and all} -- we’re going to eat really, really well this afternoon,” said [the rabbit].}{}%eng+spa trans
{}{}%rec - time

\subsubsection{Accompaniment, adjacency \phono{-ntin}}\label{ssec:accomp}\index[sub]{substantive!accompaniment}
\phono{-ntin} indicates that the referent of the base accompanies or is adjacent to another entity (\phono{allqu-\pb{ntin}} ‘with her dog’)~(\ref{Glo3:Vistigashpaqa}--\ref{Glo3:Trayamura}).\\

% 1
\gloexe{Glo3:Vistigashpaqa}{}{sp}%
{Vistigashpaqa pasakun vistigaq lliw gwardya\pb{ntin} huysni\pb{ntin}.}%sp que first line
{\morglo{vistiga-shpa-qa}{investigate-\lsc{subis}-\lsc{top}}\morglo{pasa-ku-n}{pass-\lsc{refl}-\lsc{3}}\morglo{vistiga-q}{investigate-\lsc{ag}}\morglo{lliw}{all}\morglo{gwardya-ntin}{police-\lsc{acmp}}\morglo{huys-ni-ntin}{judge-\lsc{euph}-\lsc{acmp}}}%morpheme+gloss
\glotran{After they investigated, the investigators left \pb{with} the policemen \pb{and} judges.}{}%eng+spa trans
{}{}%rec - time

% 2
\gloexe{Glo3:Hinashpash}{}{amv}%
{Hinashpash pwirtanta kandawni\pb{ntin}ta kuchurusa, ¿aw?}%amv que first line
{\morglo{hinashpa-sh}{then-\lsc{evr}}\morglo{pwirta-n-ta}{door-\lsc{3}-\lsc{acc}}\morglo{kandaw-ni-ntin-ta}{lock-\lsc{euph}-\lsc{3}-\lsc{acc}}\morglo{kuchu-ru-sa}{cut-\lsc{urgt}-\lsc{npst}}\morglo{aw}{yes}}%morpheme+gloss
\glotran{Then, they say, they cut the door \pb{along with} its lock, no?}{}%eng+spa trans
{}{}%rec - time

% 3
\gloexe{Glo3:Qullqi}{}{amv}%
{Qullqi\pb{ntin} riptin krusni\pb{ntin}shi qullqi\pb{ntin}shi.}%amv que first line
{\morglo{qullqi-ntin}{money-\lsc{acmp}}\morglo{ri-pti-n}{go-\lsc{subds}-\lsc{3}}\morglo{krus-ni-ntin-shi}{cross-\lsc{euph}-\lsc{incl}-\lsc{evr}}\morglo{qullqi-ntin-shi}{money-\lsc{acmp}-\lsc{evr}}}%morpheme+gloss
\glotran{Leaving \pb{with} her money -- \pb{with} her cross and \pb{with} her money, they say.}{}%eng+spa trans
{}{}%rec - time

% 4
\gloexe{Glo3:Trayamura}{}{sp}%
{Trayamura punta\pb{ntin} punta\pb{ntin} payqa.}%sp que first line
{\morglo{traya-mu-ra}{arrive-\lsc{urgt}-\lsc{pst}}\morglo{punta-\pb{ntin}}{point-\lsc{acmp}}\morglo{punta-\pb{ntin}}{point-\lsc{acmp}}\morglo{pay-qa}{he-\lsc{top}}}%morpheme+gloss
\glotran{He arrived \pb{peak by peak}, he did.}{}%eng+spa trans
{}{}%rec - time

\subsubsection{Multiple possession \phono{-sapa}}\index[sub]{substantive!multi-possessive}
\phono{-sapa} derives a nouns referring to the possessor of the referent of the base. It differs from \phono{-yuq} in that what is possessed is possessed in greater proportion than is usual\footnote{Thanks to an anonymous reviewer for correcting my understanding of this structure.} (\phono{uma} ‘head’~→~\phono{uma-sapa} ‘person with a head bigger than usual’, \phono{yuya-y} ‘memory’~→~\phono{yuya-y-sapa} ‘person with a memory better than usual’. In the literature on Quechua it is sometimes referred to as “super” possession (posession of more than usual).\\

% 1
\gloexe{Glo3:tukuchkanina}{}{amv}%
{“¡Ñam tukuchkaniña!” puk, puk, puk \pb{sikisapa} sapu.}%amv que first line
{\morglo{ña-m}{\lsc{disc}-\lsc{evd}}\morglo{tuku-chka-ni-ña}{finish-\lsc{dur}-\lsc{1}-\lsc{disc}}\morglo{puk}{puk}\morglo{puk}{puk}\morglo{puk}{puk}\morglo{siki-sapa}{behind-\lsc{mult.all}}\morglo{sapu}{frog}}%morpheme+gloss
\glotran{“I’m already finishing up!” --~\emph{puk, puk, puk}~-- [said] the frog \pb{with the rear bigger than usual}.}{}%eng+spa trans
{}{}%rec - time

% 2
\gloexe{Glo3:Figura}{}{amv}%
{Figura alli-allin \pb{waqrasapa} ukunpa, iglisyapash.}%amv que first line
{\morglo{figura}{figure}\morglo{alli-allin}{good-good}\morglo{waqra-sapa}{horn-\lsc{mult.all}}\morglo{uku-n-pa}{inside-3-loc}\morglo{iglisya-pa-sh}{church-\lsc{gen}-\lsc{evr}}}%morpheme+gloss
\glotran{Inside the church, they say, a statue \pb{with horns bigger than usual}.}{}%eng+spa trans
{}{}%rec - time

% 3
\gloexe{Glo3:Qamqa}{}{lt}%
{Qamqa \pb{wawasapa} kayanki paypis \pb{wawasapa}sh \pb{churisapa}sh.}%lt que first line
{\morglo{qam-qa}{you-\lsc{top}}\morglo{wawa-sapa}{baby-\lsc{mult.all}}\morglo{ka-ya-nki}{be-\lsc{prog}-\lsc{2}}\morglo{pay-pis}{he-\lsc{add}}\morglo{wawa-sapa-sh}{baby-\lsc{mult.all}-\lsc{evr}}\morglo{churi-sapa-sh}{son-\lsc{mult.all}-\lsc{evr}}}%morpheme+gloss
\glotran{You have \pb{more children than usual}. He, too, has \pb{more children than usual}, \pb{more sons than usual}, they say.}{}%eng+spa trans
{}{}%rec - time

\subsubsection{Possession \phono{-yuq}}\label{sssc:poss}\index[sub]{substantive!possessive}
\phono{-yuq} derives nouns referring to the possessor of the referent of the base~(\ref{Glo3:dimandakurun}--\ref{Glo3:altupam}).\\

% 1
\gloexe{Glo3:dimandakurun}{}{sp}%
{Ayvis dimandakurun \pb{tiyrayuqkunata}.}%sp que first line
{\morglo{ayvis}{sometimes}\morglo{dimanda-ku-ru-n}{denounce-\lsc{refl}-\lsc{urgt}-\lsc{3}}\morglo{tiyra-yuq-kuna-ta}{	land-\lsc{poss}-\lsc{pl}-\lsc{acc}}}%morpheme+gloss
\glotran{Sometimes they denounced \pb{the ones with land}.}{}%eng+spa trans
{}{}%rec - time

% 2
\gloexe{Glo3:Kwirpu}{}{ch}%
{Kwirpu:mi hutra\pb{yuq}.}%ch que first line
{\morglo{kwirpu-:-mi}{body-\lsc{1}-\lsc{evd}}\morglo{hutra-yuq}{fault-\lsc{poss}}}%morpheme+gloss
\glotran{My body is \pb{the guilty one}.}{}%eng+spa trans
{}{}%rec - time

% 3
\gloexe{Glo3:altupam}{}{amv}%
{Wiñan altupam puka \pb{waytachayuq}mi.}%amv que first line
{\morglo{wiña-n}{grow-\lsc{3}}\morglo{altu-pa-m}{high-\lsc{loc}-\lsc{evd}}\morglo{puka}{red}\morglo{wayta-cha-yuq-mi}{flower-\lsc{dim}-\lsc{poss}-\lsc{evd}}}%morpheme+gloss
\glotran{\pb{The one with a little red flower} grows in the hills.}{}%eng+spa trans
{}{}%rec - time

\noindent
Ownership applies to substantives, including interrogative indefinites~(\ref{Glo3:Imayuq}), numerals~(\ref{Glo3:Kimsayuq}), pronouns~(\ref{Glo3:Chayyuq}), and so on.\\

% 4
\gloexe{Glo3:Imayuq}{}{lt}%
{\pb{Imayuq}pis kankichu.}%lt que first line
{\morglo{ima-yuq-pis}{what-\lsc{poss}-\lsc{add}}\morglo{ka-nki-chu}{be-\lsc{2}-\lsc{neg}}}%morpheme+gloss
% \glotrannq{‘You don’t \pb{have anything}.’ (lit. ‘you aren’t one with something’)}{}%eng+spa trans
\glotran{‘You don’t \pb{have anything}.’ (lit. ‘you aren’t one with something’)}{}%eng+spa trans
{}{}%rec - time

% 5
\gloexe{Glo3:Kimsayuq}{}{amv}%
{\pb{Kimsayuq} kayan.}%amv que first line
{\morglo{kimsa-yuq}{three-\lsc{poss}}\morglo{ka-ya-n}{be-\lsc{prog}-\lsc{3}}}%morpheme+gloss
\glotran{She \pb{has three}.’ (\lit~‘she is one with three’)}{}%eng+spa trans
{}{}%rec - time

% 6
\gloexe{Glo3:Chayyuq}{}{ch}%
{\pb{Chayyuq}triki chayqa.}%ch que first line
{\morglo{chay-yuq-tri-ki}{\lsc{dem.d}-\lsc{poss}-\lsc{evc}-\lsc{iki}}\morglo{chay-qa}{\lsc{dem.d}-\lsc{top}}}%morpheme+gloss
\glotran{It must \pb{have that}.}{}%eng+spa trans
{}{}%rec - time

\noindent
In case the base ends in a consonant, the semantically vacuous particle \phono{-ni} precedes \phono{-yuq}~(\ref{Glo3:Kuknin}).\\

% 7
\gloexe{Glo3:Kuknin}{}{amv}%
{Kuknin kasa \pb{kaqniqu} huknin mana \pb{kaqni}\pb{qu}.}%amv que first line
{\morglo{huk-ni-n}{one-\lsc{euph}-\lsc{3}}\morglo{ka-sa}{be-\lsc{npst}}\morglo{ka-q-ni-qu}{be-\lsc{ag}-\lsc{euph}-\lsc{poss}}\morglo{huk-ni-n}{one-\lsc{euph}-\lsc{3}}\morglo{mana}{no}\morglo{ka-q-ni-qu}{be-\lsc{ag}-\lsc{euph}-\lsc{poss}}}%morpheme+gloss
\glotran{One was \pb{wealthy}, one \pb{had nothing}.}{}%eng+spa trans
{}{}%rec - time

\noindent
\textipa{[yuq]} is in free variation with \textipa{[qu]} following \textipa{[i]}~(\ref{Glo3:Ayka}).\\

% 8
\gloexe{Glo3:Ayka}{}{amv}%
{¿Ayka \pb{watayuq} nishurankitaqqa?}%amv que first line
{\morglo{ayka}{how.many}\morglo{wata-yuq}{year-\lsc{poss}}\morglo{ni-shu-ra-nki-taq-qa?}{say-\lsc{3>2}-\lsc{pst}-\lsc{3>2}-\lsc{seq}-\lsc{top}}}%morpheme+gloss
\glotran{How \pb{old} did she tell you she was?}{}%eng+spa trans
{}{}%rec - time

\noindent
\phono{-yuq} is used in the expression ‘to be \lsc{n} years old’~(\ref{Glo3:trunka}) as well as in the construction of compound numerals~(\ref{Glo3:kankichu}).\\

% 9
\gloexe{Glo3:trunka}{}{amv}%
{Chay \pb{trunka pichqa}\pb{yuq} puntrawnintaqa ñam trakrantañam tapamun.}%amv que first line
{\morglo{chay}{\lsc{dem.d}}\morglo{trunka}{ten}\morglo{pichqa-yuq}{five-\lsc{pos}}\morglo{puntraw-ni-n-ta-qa}{day-\lsc{euph}-\lsc{3}-\lsc{acc}-\lsc{top}}\morglo{ña-m}{\lsc{disc}-\lsc{evd}}\morglo{trakra-n-ta-ña-m}{field-\lsc{3}-\lsc{acc}-\lsc{evd}}\morglo{tapa-mu-n}{cover-\lsc{cisl}-\lsc{3}}}%morpheme+gloss
\glotran{At \pb{fifteen} days they cover the field.}{}%eng+spa trans
{}{}%rec - time

% 10
\gloexe{Glo3:kankichu}{}{lt}%
{Ima\pb{yuq}pis kankichu chay wambraykita katrarunki mayurnikikama.}%
{\morglo{ima-yuq-pis}{what-\lsc{poss}-\lsc{add}}\morglo{ka-nki-chu}{be-\lsc{2}-\lsc{neg}}\morglo{chay}{\lsc{dem.d}}\morglo{wambra-yki-ta}{child-\lsc{2}-\lsc{acc}}\morglo{katra-ru-nki}{release-\lsc{urgt}-\lsc{2}}\morglo{mayur-ni-ki-kama}{older-\lsc{euph}-\lsc{2}-\lsc{lim}}}%morpheme+gloss
\glotran{You don’t have anything and you sent your son to your older brother.}%eng
{‘No tienes nada y mandaste a tu hijo donde tu hermano mayor’.}%spa
{}{}%rec - time

\noindent
\phono{-yuq} nouns may function adverbially without case-marking or other modification~(\ref{Glo3:Puntantam}),~(\ref{Glo3:Pallayara}).\\

% 11
\gloexe{Glo3:Puntantam}{}{sp}%
{Puntantam hamullarqani kuka kintu \pb{quqawniyuq}llam.}%sp que first line
{\morglo{punta-n-ta-m}{point-\lsc{3}-\lsc{acc}-\lsc{evd}}\morglo{hamu-lla-rqa-ni}{come-\lsc{rstr}-\lsc{pst}-\lsc{1}}\morglo{kuka}{coca}\morglo{kintu}{leaf}\morglo{quqaw-ni-yuq-lla-m}{picnic-\lsc{euph}-\lsc{poss}-\lsc{rstr}-\lsc{evd}}}%morpheme+gloss
\glotran{I’ve come by the peak \pb{with just a picnic} of coca leaves.}{}%eng+spa trans
{}{}%rec - time

% 12
\gloexe{Glo3:Pallayara}{}{lt}%
{Pallayara puka \pb{pantalunniyuq} ginduntaqa nini.}%lt que first line
{\morglo{palla-ya-ra}{pick-\lsc{prog}-\lsc{pst}}\morglo{puka}{red}\morglo{pantalun-ni-yuq}{pants-\lsc{euph}-\lsc{poss}}\morglo{gindun-ta-qa}{peach-\lsc{acc}-\lsc{top}}\morglo{ni-ni}{say-\lsc{1}}}%morpheme+gloss
\glotran{She was picking peaches \pb{in red pants}, I said.}{}%eng+spa trans
{}{}%rec - time

\subsubsection{Partnership \phono{-masi}}\index[sub]{substantive!partnership}
\largerpage
\phono{-masi} indicates partnership. It attaches to \lsc{n}s to derive \lsc{n}s generally translated ‘\lsc{n}-mate’ ‘fellow \lsc{n}’~(\ref{Glo3:mikurunchik}),~(\ref{Glo3:Chaywan}), or ‘co-\lsc{n}’ (\phono{puñu-q}~→~\phono{puñu-q-masi} ‘bedmate’). \phono{-masi} is not very widely employed.

% 1
\gloexe{Glo3:mikurunchik}{}{amv}%
{¡Runa\pb{masi}nchikta mikurunchik, wawqi!}%amv que first line
{\morglo{runa-masi-nchik-ta}{person-\lsc{part}-\lsc{1pl}-\lsc{acc}}\morglo{miku-ru-nchik,}{eat-\lsc{urgt}-\lsc{1pl}}\morglo{wawqi}{brother}}%morpheme+gloss
\glotran{We ate our \pb{fellow} people, brother!}{}%eng+spa trans
{}{}%rec - time

% 2
\gloexe{Glo3:Chaywan}{}{lt}%
{Chaywan apakatrakushpam rikakayachin runa\pb{masi}nchiktaqa.}%lt que first line
{\morglo{chay-wan}{\lsc{dem.d}-\lsc{instr}}\morglo{apa-katra-ku-shpa-m}{bring-\lsc{freq}-\lsc{refl}-\lsc{subis}-\lsc{evd}}\morglo{rika-ka-ya-chi-n}{see-\lsc{passacc}-\lsc{prog}-\lsc{caus}-\lsc{3}}\morglo{runa-masi-nchik-ta-qa}{person-\lsc{part}-\lsc{1pl}-\lsc{acc}-\lsc{top}}}%morpheme+gloss
\glotran{Carrying those [their arms], they made our \pb{fellow} people look.}{}%eng+spa trans
{}{}%rec - time

% 3
\gloexe{Glo3:ayqikuyan}{}{amv}%
{Chay yatraq\pb{masi}nqa ayqikuyan.}%amv que first line
{\morglo{chay}{\lsc{dem.d}}\morglo{yatra-q-masi-n-qa}{live-\lsc{ag}-\lsc{part}-\lsc{3}-\lsc{top}}\morglo{ayqi-ku-ya-n}{escape-\lsc{refl}-\lsc{prog}-\lsc{3}}}%morpheme+gloss
\glotran{Her \pb{neighbor} is escaping.}{}%eng+spa trans
{}{}%rec - time

% 4
\gloexe{Glo3:Qunqaytaqqa}{}{ach}%
{Qunqaytaqqa, chay ukucha\pb{masi}n apamun trupataqa.}%ach que first line
{\morglo{qunqaytaq-qa,}{suddenly-\lsc{top}}\morglo{chay}{\lsc{dem.d}}\morglo{ukucha-masi-n}{mouse-\lsc{part}-\lsc{3}}\morglo{apa-mu-n}{bring-\lsc{cisl}-\lsc{3}}\morglo{trupa-ta-qa}{tail-\lsc{acc}-\lsc{top}}}%morpheme+gloss
\glotran{Suddenly, the mouse’s \pb{companion} [arrived and] took away the tail.}{}%eng+spa trans
{}{}%rec - time

\subsubsection{Restrictive suffix: \phono{-cha}}\index[sub]{substantive!restrictive suffix}
\phono{-cha} attaches to \lsc{n}s to derive \lsc{n}s with the meaning ‘little~\lsc{n}’~(\ref{Glo3:Wambra}--\ref{Glo3:trabaha}).\\

% 1
\gloexe{Glo3:Wambra}{}{lt}%
{Wambra, uchuchuk wambra. Kayna wambra\pb{cha}kunalla.}%lt que first line
{\morglo{wambra}{child}\morglo{uch-uchuk}{small-small}\morglo{wambra}{child}\morglo{kayna}{thus}\morglo{wambra-cha-kuna-lla}{child-\lsc{dim}-\lsc{pl}-\lsc{rstr}}}%morpheme+gloss
\glotran{Little, little children --~like this~-- just \pb{small} children.}{}%eng+spa trans
{}{}%rec - time

% 2
\gloexe{Glo3:Santupa}{}{amv}%
{Santupa karqa kuruna\pb{cha}nkuna.}%amv que first line
{\morglo{Santu-pa}{Saint-\lsc{gen}}\morglo{ka-rqa}{be-\lsc{pst}}\morglo{kuruna-cha-n-kuna}{crown-\lsc{dim}-\lsc{3}-\lsc{pl}}}%morpheme+gloss
\glotran{The saints had their \pb{little} crowns.}{}%eng+spa trans
{}{}%rec - time

% 3
\gloexe{Glo3:trabaha}{}{ch}%
{Turnu\pb{cha}wan ñuqakunaqa trabaha:.}%ch que first line
{\morglo{turnu-cha-wan}{turn-\lsc{dim}-\lsc{instr}}\morglo{ñuqa-kuna-qa}{I-\lsc{pl}-\lsc{top}}\morglo{trabaha-:}{work-\lsc{1}}}%morpheme+gloss
\glotran{We work by \pb{short} turns.}{}%eng+spa trans
{}{}%rec - time

\noindent
It may also express an affectionate attitude toward the referent of \lsc{n}~(\ref{Glo3:Katraramuy}).\\

% 4
\gloexe{Glo3:Katraramuy}{}{amv}%
{Katraramuy indikananpaq, Hilda\pb{cha}.}%amv que first line
{\morglo{katra-ra-mu-y}{send-\lsc{urgt}-\lsc{cisl}-\lsc{imp}}\morglo{indika-na-n-paq}{indicate-\lsc{nmlz}-\lsc{3}-\lsc{purp}}\morglo{Hilda-cha}{Hilda-\lsc{dim}}}%morpheme+gloss
\glotran{Send him so that he shows him, Hilda, \pb{dear}.}{}%eng+spa trans
{}{}%rec - time

\noindent
(\ref{Glo3:tapaykullasa}) is taken from a song in which a girl addresses her lover.\\

% 5
\gloexe{Glo3:tapaykullasa}{}{sp}%
{Pulvu\pb{cha}paq tapaykullasa, wayra\pb{cha}paq apaykullasa, kay sityu\pb{cha}man trayaykamunki.}%sp que first line
{\morglo{pulvu-cha-paq}{dust-\lsc{dim}-\lsc{abl}}\morglo{tapa-yku-lla-sa}{cover-\lsc{excep}-\lsc{rstr}-\lsc{prf}}\morglo{wayra-cha-paq}{wind-\lsc{dim}-\lsc{abl}}\morglo{apa-yku-lla-sa}{bring-\lsc{excep}-\lsc{rstr}-\lsc{prf}}\morglo{kay}{\lsc{dem.p}}\morglo{sityu-cha-man}{place-\lsc{dim}-\lsc{all}}\morglo{traya-yka-mu-nki}{arrive-\lsc{excep}-\lsc{cisl}-\lsc{2}}}%morpheme+gloss
\glotran{Covered with dust, carried by the wind, you’re going to come to this place.}{}%eng+spa trans
{}{}%rec - time

\noindent
Applied to other substantives \phono{-cha} may function as a limitative. In these cases, it is generally translated ‘just’ or ‘only’~(\ref{Glo3:kakullayan}).\\

% 6
\gloexe{Glo3:kakullayan}{}{amv}%
{Chay\pb{cha}pam kakullayan.}%amv que first line
{\morglo{chay-cha-pa-m}{\lsc{dem.d}-\lsc{dim}-\lsc{loc}-\lsc{evd}}\morglo{ka-ku-lla-ya-n}{be-\lsc{refl}-\lsc{rstr}-\lsc{prog}-\lsc{3}}}%morpheme+gloss
\glotran{It’s \pb{just} right there.}{}%eng+spa trans
{}{}%rec - time

\noindent
The forms \phono{Mama-cha} (mother-\lsc{dim}) and \phono{tayta-cha} (father-\lsc{dim}) are lexicalized, meaning ‘grandmother’ and ‘grandfather’ respectively~(\ref{Glo3:sirvintin}).\\

% 7
\gloexe{Glo3:sirvintin}{}{amv}%
{Tiyu:pa sirvintin mama\pb{cha}:pis sirvintin ñuqa kara:.}%xx que first line
{\morglo{tiyu-:-pa}{uncle-\lsc{1}-\lsc{gen}}\morglo{sirvinti-n}{servant-\lsc{3}}\morglo{mama-cha-:-pis}{mother-\lsc{dim}-\lsc{1}-\lsc{add}}\morglo{sirvinti-n}{servant-\lsc{3}}\morglo{ñuqa}{I}\morglo{ka-ra-:}{be-\lsc{pst}-\lsc{1}}}%morpheme+gloss
\glotran{I was my uncles’s and my \pb{grandmother’s} servant.}{}%eng+spa trans
{}{}%rec - time

\noindent
In addition to \phono{-cha}, speakers sometimes employ the borrowed Spanish diminutive suffix, \emph{-itu/a} (or its post-consonant form \emph{-citu/a})~(\ref{Glo3:urunguy}).\\

% 8
\gloexe{Glo3:urunguy}{}{amv}%
{Chay urunguy\pb{situ} lluqsiramushqa chay kahapaq.}%amv que first line
{\morglo{chay}{\lsc{dem.d}}\morglo{urunguy-situ}{fly-\lsc{dim}}\morglo{lluqsi-ra-mu-shqa}{go.out-\lsc{urgt}-\lsc{cisl}-\lsc{subis}}\morglo{chay}{\lsc{dem.d}}\morglo{kaha-paq}{coffin-\lsc{abl}}}%morpheme+gloss
\glotran{That \pb{little} fly came out of the coffin.}{}%eng+spa trans
{}{}%rec - time
