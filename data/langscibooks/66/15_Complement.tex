\chapter{Complementation}\label{compl}

Yakkha has a number of  verbs that embed  clausal complements. These com\-ple\-ment-taking verbs (\textsc{ctp}s, cf. \citealt{Noonan2007Complementation}) license several complement constructions, defined by the type of the embedded clause. The most salient formal distinction can be drawn between infinitival clauses and inflected clauses.\footnote{I deliberately avoid the terms \emph{finite} and \emph{nonfinite} here, since finiteness is a problematic concept. The defining criteria are different across languages and across theoretical frameworks. In this case, finiteness would only be defined by the presence or absence of the verbal inflection, as inflected complement clauses are less finite than independent clauses. They lack many features such as clause-final marking for evidentiality or mirativity, certain mood inflections like the imperative, and detached positions. The infinitival clauses, on the other hand, are potentially equal to finite clauses in Yakkha: in one infinitival complement construction the main verb is optional (the deontic construction), so that the infinitive can present a full (deontic) predication (see \sectref{obl}).} A typical infinitival complement construction is shown in \Next[a]. Inflected complement clauses show person, tense/aspect, mood inflection (indicative  or subjunctive) and the nominalizing clitics \emph{=na} or \emph{=ha}  (see \Next[b], cf. \sectref{nmlz-uni}). In the case of embedded direct speech they may show any marking that is found on independent clauses as well, for instance the imperative (see \Next[c]). 

\ex.\ag.ka kheʔ-ma mit-a-ŋ-na\\
{\sc 1sg} go{\sc -inf} think{\sc -pst-1sg=nmlz.sg}\\
\rede{I want to go.}
\bg.nda lam-me-ka=na mi-nuŋ-nen=na\\
{\sc 2sg} return{\sc -npst-2=nmlz.sg} think{\sc -prf-1>2=nmlz.sg}\\
\rede{I thought that you will come back.} 
\bg.haku bagdata=ca   py-a-ŋ-ni-ŋ  bhoŋ eko binti   bisa       cog-wa-ci\\
now finalization\_of\_marriage{\sc =add} give{\sc -imp-1sg-pl-1sg} {\sc comp} one request request make{\sc -npst-3nsg.P}\\
\rede{“Now give me the \emph{bagdata}, too!” she requests from them.}\footnote{The form \emph{pyaŋniŋ} is more complex than expected from the Tumok data (see \sectref{mood}).The speaker's natal home is in Hombong village.}\source{26\_tra\_02.019 } 


\tabref{overview-all} provides an overview of the complement-taking predicates of Yakkha. They are grouped according to the  inflectional features of the embedded clauses and according to their semantics.  \citet[120]{Noonan2007Complementation} lists the following semantic classes of complement-taking predicates: utterance predicates, propositional attitude predicates, pretence predicates, comment (factitive) predicates, predicates of knowledge or acquisition of knowledge, predicates of fearing, desiderative predicates, manipulative predicates, modal predicates, achievement predicates, phasal predicates and predicates of perception. In Yakkha, inflected complement clauses occur with utterance predicates, propositional attitude predicates and  with predicates of perception. Infinitival  complement clauses occur with a broader semantic range of \textsc{ctp}s, as the Table shows. 

Most complement-taking verbs may also take nominal arguments, e.g. \emph{tokma}, which can mean \rede{get something} or \rede{get to do something}. In addition to the complement-taking verbs, there are also some nouns that can embed clausal complements, such as \emph{ceʔya} \rede{talk, matter} or \emph{kisiʔma} \rede{fear}. 


\begin{table}  
\small 
\resizebox{\textwidth}{!}{
\begin{tabular}{p{6cm}p{6cm}}
\lsptoprule
{\sc infinitival complement} & {\sc inflected complement}   \\
\midrule
\multicolumn{2}{c}{{\sc Phasal}}\\
\midrule
\emph{tarokma} \rede{begin} &\\
\emph{lepnima} \rede{stop, abandon} &\\
\emph{tama} \rede{be time to} (lit. \rede{come})& \\
\midrule
\multicolumn{2}{c}{{\sc Utterance Predicates}}\\
\midrule
& \emph{kama} \rede{say}\\
&\emph{luʔma} \rede{tell}\\
& \emph{chimma} \rede{ask}\\
&\emph{yokmeʔma} \rede{tell (about something)}\\ 
\midrule
\multicolumn{2}{c}{{\sc Propositional Attitude, Desiderative, Experiential}}\\
\midrule
\emph{miʔma} \rede{like doing, want} (lit. \rede{think})&\emph{miʔma} \rede{think, hope, remember}\\								
\emph{kaŋma} \rede{give in, surrender} (lit. \rede{fall})	&\emph{yemma \ti emma} \rede{agree (to propositions)}\\
\emph{\textsc{poss}-niŋsaŋ puŋma} \rede{have enough, be fed up}  & \emph{consiʔma} \rede{be happy about}\\
\emph{sukma} \rede{intend, aim}&\emph{kuma} \rede{wait, expect}\\
\emph{leŋma} \rede{be acceptable, be alright}& \emph{\textsc{poss}-niŋwa wama} \rede{hope}\\
\emph{kisiʔma} \rede{be afraid}&\emph{niŋwa hupma} \rede{decide collectively}\\
\midrule
\multicolumn{2}{c}{{\sc Perception, Cognition, Knowledge}}\\
\midrule
\emph{nima} \rede{know how to do}& \emph{nima} \rede{see/get to know}\\
\emph{muʔnima} \rede{forget to do}  & \emph{muʔnima} \rede{forget about something}\\
&\emph{khemma} \rede{hear}\\
&\emph{eʔma} \rede{have the impression that}\\
\midrule
\multicolumn{2}{c}{{\sc Modals}}\\
\midrule
\emph{yama} \rede{be able}& \\
(copula/zero) \rede{have to}& \\
\emph{cokma} \rede{try} (lit. \rede{do})&\\
\midrule
\multicolumn{2}{c}{{\sc Achievement}}\\
\midrule
\emph{tokma} \rede{get to do} (lit. \rede{get})&\\
\midrule
\multicolumn{2}{c}{{\sc Pretence}}\\
\midrule
\emph{loʔwa cokma} \rede{pretend}&\\
\midrule
\multicolumn{2}{c}{{\sc Permissive, Assistive (S-to-P )}}\\
\midrule
\emph{piʔma} \rede{allow} (lit. \rede{give})&\\
\emph{phaʔma} \rede{help doing}&\\
\emph{cimma} \rede{teach}&\\
\emph{soʔmeʔma} \rede{show}&\\
\lspbottomrule
\end{tabular} 
}
\caption{Overview of complement-taking predicates}\label{overview-all}
\end{table}

%\emph{luʔma} \rede{tell to do} &\\ all examples have finite compl!
%\emph{nakma} \rede{ask}&\\ -> the only verb that worked well was 'phaʔma', probably a nominalized infinitive!

Many complement constructions, especially those with infinitival complements, are characterized by the referential identity between an argument of the embedded clause and an argument of the matrix clause. This may be reflected by leaving an argument unexpressed in one clause (Equi-deletion). Traditionally, complement constructions with shared arguments are divided into raising and control constructions.\footnote{Arguments can be shared  with respect to morphological marking, agreement, or constituency properties, following Serdobolskaya's definition of raising \citep[278]{Serdobolskaya2009_Raising}.} In control constructions, the shared argument belongs to both clauses semantically,  whereas in raising constructions the shared argument belongs to the embedded clause semantically, despite being coded as argument of the main clause. Hence, in raising constructions an \rede{unraised} alternative expressing the same propositional content is usually available, while in control constructions, there is only one option. 

The \textsc{ctp}s taking infinitival complements are discussed in \sectref{nonfin-comp}, the verbs and nouns that take inflected complement clauses are dealt with in \sectref{fin-comp}. 


\section{Infinitival complement clauses}\label{nonfin-comp}
\subsection{Overview}

In this section I discuss all constructions that follow the basic pattern of an infinitive followed by an inflected verb. I found 20 predicates  that occur with embedded infinitives in Yakkha. 

Infinitival complement clauses are marked by the infinitive \emph{-ma} to which, optionally, the genitive marker \emph{=ga}  may attach (see \Next). Other inflectional categories are not possible on these clauses. Apart from these basic features, the behavior of the various verbs that  embed infinitives is far from homogenous, as will be shown below. 

 \ex. \ag. ka kheʔ-ma(=ga) cog-a-ŋ=na\\
		{\sc 1sg}  go-\sc{inf(=gen)} do\sc{-pst-1sg=nmlz.sg}\\
		\rede{I tried to go.}
		\bg. kucuma=ŋa ka haʔ-ma(=ga) sukt-a-ŋ=na\\
		dog{\sc =erg} {\sc 1sg} bite{\sc -inf(=gen)} aim\sc{[3sg.A]-pst-1sg.P=nmlz.sg} \\
		\rede{The dog intended to bite me.} 
		%\bg. uŋci-ŋa 	uŋ 			kam 			cok-ma cim-ma=na mit-a-ma-ci=ha (CHECK)\\
	%{\sc 3ns=erg} {\sc 3sg} work-\sc{} do-\sc{inf} learn-\sc{inf} want-\sc{pst-prf-nsg=nmlz.nsg}\\
%			\rede{They wanted him to learn some work./They thought - he has to do some work. - CHECK} 
	
	
The embedded infinitive depends on the main verb for all inflectional categories. One construction (expressing deontic modality) is exceptional in this respect, though (cf. \sectref{obl}). In this construction, the nonsingular marker \emph{=ci}  attaches to the infinitive when the object has third person nonsingular reference. The marker is aligned with the primary object (compare the intransitive example \Next[a] with transitive  \Next[b] and ditransitive \Next[c]).  
			
\ex. \ag. yapmi  paŋ-paŋ=be nak-saŋ kheʔ-m=ha\\
		people house-house{\sc =loc} ask-{\sc sim} go-{\sc inf[deont]=nmlz.nsg}\\
	\rede{The people have to go from house to house, asking (for food).}  (S) \source{14\_nrr\_02.029}
\bg. thim-m=ha=ci\\
scold-{\sc inf[deont]=nmlz.nsg=nsg}\\
\rede{They (the young people) have to be scolded.} (P)
 \bg. wa=ci piʔ-m=ha=ci\\
chicken{\sc =nsg} give-{\sc inf[deont]=nmlz.nsg=nsg}\\
\rede{It has to be given to the chicken.} (G)



 The complement-taking verbs can be distinguished according to their valency and their argument realization. Some matrix verbs always have intransitive morphology, some always have transitive morphology, and some verbs assimilate to the valency of their embedded predicates.   According  to this, the agreement properties have the potential to differ. Yakkha infinitival complement constructions show long distance agreement (\textsc{lda}), i.e. the matrix verb shows agreement with an argument of the embedded clause. This is illustrated  by example \LLast[b] above: the constituent \emph{ka} \rede{1sg} is the P argument of the embedded clause and triggers object agreement in the matrix verb \emph{sukma} \rede{intend}. \textsc{lda} is common for Kiranti languages; it has been described  e.g. for Belhare \citep{Bickeletal2001Syntactic, Bickel2004Hidden} and for Puma \citep{Schackow2008Clause}. An overview of the infinitival complement types and their properties is provided in \tabref{overview-infinitival} below. In the following sections, these properties will be treated in detail.
 
The case assignment has to be determined independently from the agreement properties. In the majority of the complement-taking verbs, the case assignment for the embedded arguments comes from the embedded verb. This means that in cases of shared reference it is not necessarily the embedded argument that is omitted, but rather the matrix argument (discussed below). Such structures are known as backward control, and received attention in the literature on control phenomena especially after an article on the Nakh-Dagestanian language Tsez by \citet{Polinskyetal2002_Backward}. 

Thus, in several infinitival complement constructions the arguments show relations to both clauses simultaneously, so that the whole structure is better analyzed as monoclausal rather than as matrix clause and embedded clause.\footnote{See \citet{Haspelmath1999Long-distance} for a discussion on clause union in the Nakh-Dagestanian language Godoberi.}  This is the case when, for instance,  case is assigned by the embedded verb, but agreement is triggered on the  main verb.\footnote{Yakkha also has a periphrastic imperfective construction that  has developed from an infinitival complement construction, see \sectref{tense}.}  

%Among others, it is found in Romance languages, in Urdu \citep{Butt2008Revisiting}, in Caucasian languages like Tsez \citep{Polinskyetal1999Agreement} and  Godoberi \citep{Haspelmath1999Long-distance}. 

 
\todo{This table's width was set to .95\textbackslash linewidth. I did not see the point and hence changed it to \textbackslash linewidth only. Also, it was generating a float to large.}
\begin{table}[htp]
{\small
% \resizebox{.95\textwidth}{!}{
\resizebox{\textwidth}{!}{
\begin{tabular}{p{2.2cm}lllll}
\lsptoprule
{\sc verb} 							&{\sc morph.}  & {\sc argu-}&{\sc case}&\multicolumn{2}{c}{{\sc agreement}}\\%&{\sc referent.}  \\
										&   {\sc valency} & {\sc ments}& {\sc of overt A} &{\sc emb. arg.}&{\sc matr. slot}\\%&  {\sc identity }\\
\midrule
\emph{yama} \newline \rede{be able}			& 				& 2		&by emb. V	&[S]&→S\\%&[S]=S\\
\emph{miʔma} \newline \rede{want}			&as emb. V	& 2		&by emb. V	&[A]&→A\\%&[A]=A\\
\emph{cokma} \newline \rede{try}				& 				& 2		&by emb. V	&[P]&→P\\%&[P]=P\\
\midrule
\emph{niŋsaŋ puŋma} \newline \rede{be fed up}&as emb. V& 2& ? 			&[A]&→A\\%&[A]=A\\
													&					&		&						&[P]&→P\\%&[P]=P or\\
													&					&		&						&[clause]	&→S[3sg]\\%&-\\
\midrule
\emph{sukma} \newline \rede{aim, intend}	& 			&2 		&by emb. V	&[S/A]		&→A\\%&[S/A]=A\\
\emph{tarokma} \newline \rede{begin} 		& 			&2 		&by emb. V	&[P]			&→P\\%&[P]=P\\
\emph{lepnima} \newline \rede{stop}			&trans.		& 2		&by emb. V	&[clause]	&→P[3sg]\\%&\\
\emph{tokma} \newline \rede{get to do}		& 			& 2		&by emb. V	&				&			\\%&\\
\emph{muʔnima} \newline \rede{forget}		& 			& 2		&?	&				&			\\%&\\
\midrule
\emph{kaŋma} \newline \rede{agree, give in}&intrans.	&2		&  ?&[S/A/P]&→S\\%&[S/P]=S\\
%\emph{kisiʔma} \newline \rede{be afraid}	&intrans.	&2		&?				&[S/A]&→S\\%&[S/A]=S\\
\midrule
\emph{piʔma} \newline \rede{allow} & 						&3		&					&[S/A]&→P\\%&[S/A]=P\\
\emph{phaʔma} \newline \rede{help doing}& 	ditrans.				&3		&	by matrix V				&&\\%&\\
\emph{cimma} \newline \rede{teach}&			&3		&					&&\\%&\\
\emph{soʔmeʔma} \newline \rede{show}& 					&3		&					&&\\%&\\
\midrule
\emph{leŋma} \newline \rede{be alright}&		&1		& by emb. V	&[S/A/P]& →S or \\%&-\\
												&					&			&					&[clause]&→S[3sg]\\%&-\\
\emph{tama} \newline \rede{be time to}&intrans.		&1		& ? 				&[clause]&→S[3sg]\\%&-\\
copula/zero \rede{have to}&			&1 		& by emb. V	&[S/P]&→S\\%&-\\
											&						&			& 				&[1/2]&→S\\%&-\\
\lspbottomrule
\end{tabular}
}
}
\caption{Argument realization in infinitival complement constructions}\label{overview-infinitival}
\end{table}

 

\subsection{Predicates with variable valency}

Some complement-taking verbs assimilate in valency to the embedded verb. If the embedded verb is intransitive, the matrix verb shows intransitive agreement morphology; if the embedded verb is transitive, the matrix verb shows transitive agreement morphology. The verbs \emph{yama} \rede{be able}, \emph{miʔma} \rede{like doing} and \emph{cokma} \rede{try} belong to this class. The S/A arguments of embedded and matrix verb have to be coreferential. Sentences like \rede{I want him to go} cannot be expressed by this construction. These predicates exhibit backward control. The S/A arguments are case-marked according to the properties of the embedded verb, i.e. nominative with intransitive verbs and ergative with transitive verbs. This indicates that the overtly realized arguments are those of the embedded verb.

The verb \emph{yama} is a modal verb, expressing abilities, as opposed to \emph{tokma}, which expresses the possibility of an event as determined by other circumstances or participants. Example \Next[a] shows an intransitive complement clause. The respective transitive form \emph{nyaswaŋanna} (supposing \rede{dummy} object inflection of third person singular) would be ungrammatical here. Example \Next[b] and \Next[c] show transitive complement clauses.  Both arguments trigger agreement in the matrix verb, and case is assigned by the embedded verb (notice the ergative in \Next[c]). In \Next[d], a three-argument verb from the double object class is embedded. The person inflection of the matrix verb is analogous to the inflection that is usually found in the embedded verb  \emph{soʔmeʔma} \rede{show}, aligned with the primary object.

\ex. \ag. ka nda=maʔniŋ hiŋ-ma n-ya-me-ŋa-n=na\\  
	 {\sc 1sg} {\sc 2sg=}without survive-\sc{inf}  {\sc neg-}be\_able-\sc{npst-1sg-neg=nmlz.sg}\\
	\rede{I cannot survive without you.}
%\bg.na=go nda kheʔ-ma  n-yas-wa-ga-n\\
%this{\sc =top} {\sc 2sg[erg]} carry\_off{\sc -inf} {\sc neg-}be\_able{\sc -npst-2sg.A[3.P]-neg}\\
%\rede{As for this, you cannot carry it off.}\source{37\_nrr\_07.006}
\bg. ŋkha (...) them-ma     n-yas-uks-u-n-ci-ni-n\\
those (...) lift{\sc -inf} {\sc neg-}be\_able{\sc -prf[pst]-3.P-neg-3nsg.P-3pl.A-neg}\\
\rede{They could not lift those (stones).}\source{37\_nrr\_07.029}
\bg.na   mamu=ŋa   luŋkhwak pok-ma        n-yas-u-n\\
this girl{\sc =erg} stone wake\_up{\sc -inf} {\sc neg-}be\_able{\sc [3sg.A]-3.P[pst]-neg}\\
\rede{This girl was not able to wake up the stone.} \source{37\_nrr\_07.036}
	\bg. ka nda a-den soʔ-meʔ-ma ya-meʔ-nen=na\\
	{\sc 1sg[erg]} {\sc 2sg} {\sc 1sg.poss}-village  show\sc{-inf} be\_able-\sc{npst-1>2=nmlz.sg}\\
	\rede{ I can show you my village.}
	
When  \emph{yama} is negated, occasionally an alternative infinitive marker \emph{-sa} is found, as in \Next. It is only attested with \emph{yama}.

\exg.kanciŋ  i=ca ka-sa y-yas-a-n-ci-ŋa-n\\
{\sc 1du} what{\sc =add} say{\sc -inf} {\sc neg-}be\_able{\sc -pst-neg-du-excl-neg}\\
\rede{We could not say anything.} \source{40\_leg\_08.027}

%	\bg. nda ka=ca chim-ma ya-me-{\bf ŋ-ga}=na\\
%	{\sc 1sg=add} ask-\sc{inf} be\_able-\sc{npst-1sg.P-2sg.A=nmlz.sg}  \\
%	\rede{You can also ask me.}

Example \Next  and \NNext illustrate  the same properties for  the verbs \emph{miʔma} \rede{want} (lit.: \rede{think}) and  \emph{cokma} \rede{try} (lit.: \rede{do}). 

\ex. \ag.   paŋ=be     ap-ma            mit-a-ma-ŋ=na\\  
	house{\sc =loc} come-\sc{inf} like-\sc{pst-prf-1sg=nmlz.sg}\\
	\rede{I want to come home.} (intransitive) \source{01\_leg\_07.096}
\bg.suman=ŋa    limlim inca-ma    mit-uks-u\\
Suman{\sc =erg} sweets  buy{\sc -inf}  like{\sc [3sg.A]-prf-3.P[pst]}\\
\rede{Suman wanted to buy sweets.}	 (transitive) \source{01\_leg\_07.040}

\ex.\ag.  siŋ=be    thaŋ-ma=ga       cog-a-ŋ\\
tree{\sc =loc} climb{\sc -inf=gen} try{\sc -pst-1sg}\\
\rede{I tried to climb the tree.}  (intransitive) \source{42\_leg\_10.020}
\bg. ŋkha them-ma     n-jog-uks-u-ci\\
those lift{\sc -inf} {\sc 3pl.A-}try{\sc -prf-3.P[pst]-3nsg.P}\\
\rede{They tried to lift those (stones).}  (transitive) \source{37\_nrr\_07.029}


The agreement pattern with the embedded object may also apply when the agreement triggering arguments and the embedded verb are not overt, as example \Next from a conversation illustrates.

\exg. khuʔ-nen? – khatniŋgo n-yas-wa-ŋ-ga-n=naǃ\\
carry-{\sc 1>2[sbjv]}  – but {\sc neg}-be\_able-{\sc npst-1sg.P-2sg.A-neg=nmlz.sg}\\
A: \rede{Shall I carry you?} B: \rede{But you can't!}


The  verb \emph{niŋsaŋ puŋma} \rede{have enough, lose interest}, an experiencer-as-possessor predicate (cf. \sectref{nv-comp-poss}) may also alternate in valency. In most cases, it is intransitively inflected, invariably with third person singular inflection \Next[a]. With transitive embedded verbs, however, it may optionally mirror the inflection of the embedded verb  \Next[b]. 

\ex. \ag. nda=nuŋ kon-ca-ma a-niŋsaŋ puŋ-a-by-a=naǃ ikhiŋ koʔ-ma=le haʔloǃ? \\
	{\sc 2sg=com} walk{\sc -V2.eat-inf} {\sc 1sg.poss-[stem]}  lose\_interest{\sc [3sg]-pst-V2.give-pst=nmlz.sg} how\_much walk{\sc -inf[deont]=ctr}  {\sc excla}\\
	\rede{I have enough of walking with youǃ How much do we have to walk?ǃ} 
	\bg. kha philim=ci (soʔ-ma) a-niŋsaŋ puŋ-y-uks-u-ŋ-ci-ŋ=ha\\
		{\sc those} film{\sc =nsg} (watch{\sc -inf}) {\sc 1sg.poss-[stem]} lose\_interest{\sc -compl-prf-3.P[pst]-1sg.A-3nsg.P-1sg.A=nmlz.nsg}	\\
	\rede{I have enough of (watching) those films.} 
 


\subsection{Invariably transitive predicates}

Some complement-taking verbs always show transitive person inflection. Both subject and object of transitive embedded verbs are indexed on the matrix verb. When intransitive verbs are embedded, the embedded S argument triggers transitive subject agreement in the matrix verb, while the object agreement is default third person singular. Verbs belonging to this class are  \emph{tokma} \rede{get to do}, \emph{tarokma} \rede{begin}, \emph{sukma} \rede{aim, intend},  \emph{lepnima} \rede{stop, abandon} and \emph{muʔnima} \rede{forget}. Except for \emph{tarokma} all verbs may also take nominal complements. 

Example \Next illustrates this pattern with the verb \emph{tokma}. It expresses possibilities that are determined by other participants or conditioned by outer circumstances beyond the power and control of the subject (S or A). When three-argument verbs are embedded,  the person marking on the matrix verb is analogous to the person marking usually found on  the embedded verb (see \Next[b] from the double object class). The long distance agreement is obligatory, as evidenced by ungrammatical \Next[c], where the number features of the embedded object and the person inflection of the matrix verb do not match. 
%\emph{yemma} \rede{agree}, - no corpus evidence!

\ex. \ag. nda nhe uŋ-ma n-dokt-wa-ga-n=na\\
{\sc 2sg} here come\_down-{\sc inf} {\sc neg}-get{\sc -npst[3.P]-2sg.A-neg=nmlz.sg}\\
\rede{You will not get the chance to come down here (i.e. we will not let you come down here).} \source{21\_nrr\_04.035}
\bg. ka kamniwak=ci sandisa  khuʔ-ma tokt-u-ŋ-ci-ŋ=ha\\
{\sc 1sg[erg]} friend{\sc =nsg} present bring-{\sc inf} get-{\sc 3.P[pst]-1sg.A-3nsg.P-1sg.A=nmlz.nsg}\\
\rede{I got the chance to bring my friends presents.} 
	\bg. *ka muŋ=ci im-ma tokt-u-ŋ\\
		{\sc 1sg[erg]} mushroom{\sc =nsg} buy{\sc -inf} get-{\sc 3.P-1sg.A}\\
	Intended: \rede{I got the chance to buy mushrooms.}  (correct: \emph{imma toktuŋciŋ})


Example \Next shows the transitive agreement with the embedded object for the verb \emph{tarokma} \rede{begin}. In \Next[a], the verbal person marking is the only hint given on the number of the object argument, as the nonsingular marker on the noun is optional and often omitted, in this case because of  the unspecific reference of the argument \emph{yakpuca}. Example \Next[b] and \Next[c] serve to show  that \emph{tarokma} is  not restricted to verbs with intentional agents. 
Case is assigned by the embedded verb, since the subjects of embedded intransitive verbs are in the nominative and not in the ergative. 


\ex.\ag. yakpuca yok-ma tarokt-a-ma-c-u-ci\\
porcupine search-{\sc inf} begin{\sc -pst-prf-du.A-3.P-3nsg.P}\\
\rede{They (dual) began to look for porcupines.}\source{22\_nrr\_05.014}
\bg. khap khap-ma=niŋa=go chiŋdaŋ  khoŋ-ma,  khap yoŋ-ma tarokt-uks-u\\
		roof cover{\sc -inf=ctmp=top} main\_pillar collapse{\sc -inf} 	roof  shake{\sc -inf} begin{\sc [3sg.A]-prf-3.P[pst]}	\\
	\rede{While (Tumhang) made the roof, the pillar began to collapse, the roof began to shake.} \source{27\_nrr\_06.031}
\bg. nna  mamu=ga    o-phok            tuk-ma           tarokt-uks-u\\
that girl{\sc =gen} {\sc 3sg.poss-}stomach hurt{\sc -inf} begin{\sc [3sg.A]-prf-3.P[pst]}\\
\rede{That girl's stomach began to hurt.} \source{37\_nrr\_07.020}


Another member of this class is the phasal verb \emph{lepnima}  \rede{stop, abandon} (see \Next). The verb consists of two stems: the lexical stem \emph{lept} \rede{throw} and the marker \emph{ni \ti i} contributing completive semantics. This verb indicates the terminal point of an event, regardless of whether  they are activities or states (as in \Next[a]), or actual or habitual (as \Next[b] shows).  The embedded object triggers object agreement on the matrix verb.

\ex. \ag. a-nabhuk hup-ma n-lept-iʔ-wa-n=na\\
{\sc 1sg.poss}-nose be\_blocked{\sc -inf} {\sc neg}-stop{\sc [3sg.A]-compl-npst[3.P]-neg=nmlz.sg}\\
\rede{My nose continues to be blocked.}
\bg. ka uŋci cim-ma lept-i-ŋ-ci-ŋ=ha\\
{\sc 1sg[erg]} {\sc 3nsg} teach{\sc -inf} stop-{\sc compl-c-3nsg.P-1sg.A=nmlz.nsg}\\
\rede{I stopped teaching them.}

As with \emph{tarokma} above, the semantics of \emph{lepnima} do not imply conscious decisions and actions, as both sentences in \Next show. Given the etymological relation to the lexical verb \emph{lepma} \rede{throw}, this is an unexpected finding. 

\ex. \ag. paŋ  yoŋ-ma     lept-i-uks-u\\
house shake{\sc -inf} stop{\sc [3sg.A]-compl-prf-3.P[pst]}\\
\rede{The house stopped shaking.} \source{27\_nrr\_06.036}
\bg. luŋdaŋ  lupluŋ=be    hom-ma n-lept-i-ci\\
cave den{\sc =loc} fit\_into-{\sc inf} {\sc 3pl.A-}stop-{\sc compl[pst]-3nsg.P}\\
\rede{They did not fit into caves and dens any more.} \source{27\_nrr\_06.004}



\subsection{Three-argument constructions}	
		
 Complement-taking  verbs with assistive or permissive semantics such as \emph{phaʔma} \rede{help}, \emph{piʔma} \rede{give} (\rede{allow}), \emph{cimma} \rede{teach} and \emph{soʔmeʔma} \rede{show} follow a pattern where the G argument of the matrix clause has identical reference to the subject of the embedded clause. In other words, the matrix G argument controls the reference of the embedded S or A argument. The patients of teaching, allowing etc. are arguments of the  matrix clause; hence, the agreement is not optional. For instance, a form like \emph{soʔmetuŋna} (1>3) would not be acceptable in \Next[b]. 
 These verbs realize their arguments according to the double object frame, i.e. the nominal G argument triggers object agreement on the verb; and the embedded clause has the role of the T argument \Next. An example from natural speech is provided in \Next[c]. 
 

\ex.\ag. im-ma pim-meʔ-nen=na\\
	sleep-{\sc inf} give-{\sc npst-1>2=nmlz.sg}\\
	\rede{I allow you to sleep.}
	\bg. kondarik ok-ma soʔmeʔ-meʔ-nen=na\\
	spade dig{\sc -inf} show{\sc -npst-1>2=nmlz.sg}		\\
	\rede{I will show you how to plough.}  
	\bg.na   yakkha=ga ceʔyamumma ghak     heko=ci=ŋa      ŋ-khus-het-u, cek-meʔ-ma m-bi-n-ci-nin=niŋa,  ...  \\
this Yakkha{\sc =gen}  speech all other{\sc =nsg=erg} {\sc 3pl.A-}steal{\sc -V2.carry.off-3.P[pst]} talk{\sc -caus-inf} {\sc 3pl.A-}give{\sc -neg-3nsg.P-3pl.A=ctmp} \\
\rede{When this language of the Yakkha people was all taken away by the others, when they did not allow them to speak it ...} \source{18\_nrr\_03.006}
	
	
All verbs of this class can have either nominal or infinitival complements, as exemplified with \emph{cimma} \rede{teach} in \Next.

\ex. \ag. kamala=ŋa ka yakkha ceʔya cind-a-ŋ=na\\
		K.{\sc =erg} {\sc 1sg} Yakkha language teach{\sc [3sg.A]-pst-1sg.P=nmlz.sg}	\\
	\rede{Kamala taught me the Yakka language.}  
 	\bg. kei-lak-ma cim-meʔ-nen=na\\
	drum-dance-{\sc inf} teach{\sc -npst-1>2=nmlz.sg}\\
	\rede{I will teach you to dance the drum dance.} 

	
\subsection{The intransitively inflected verb  \emph{kaŋma} \rede{agree, give in}}

The verb \emph{kaŋma} \rede{agree, be willing to, give in}  (lit.: \rede{fall}) always shows intransitive person inflection, regardless of the valency of the embedded verb. Hence, there is only one agreement slot and (at least) two potential candidates to trigger  agreement when transitive verbs are embedded. For this  verb, the choice of the agreement triggering argument is determined by pragmatics, not by syntax. It shows agreement with whatever argument of the embedded clause is more salient in the current stretch of discourse (see \Next). This is also the case for the subject-complement construction with the matrix verb \emph{leŋma} \rede{be all right} (cf. \sectref{subjectcomplement} below). With certain embedded verbs, agreement with A is pragmatically more common (e.g. with \emph{cama} \rede{eat}), while with other  verbs agreement with P is more common (e.g. with \emph{cameʔma} \rede{feed}). Interestingly, A arguments in the ergative case are not allowed with this complement-taking verb, indicating that the matrix S argument controls the embedded arguments.

This pragmatically conditioned behavior stands in contrast to Chintang and Belhare, where certain complement-taking verbs are restricted to P arguments alone, following a purely syntactic constraint \citep{Bickeletal2001Syntactic, Bickeletal2010Ditransitives}. 

\ex. \ag. picha im-ma ŋ-gaŋ-me-n=na\\
child sleep-{\sc inf} {\sc neg}-agree{\sc [3sg]-npst-neg=nmlz.sg}\\
\rede{The child is not willing to sleep.} (S)
\bg. lukt-a-khy-a=ŋ, chuʔ-ma  ŋ-gaks-a-n=oŋ\\
run{\sc [3sg]-pst-V2.go-pst=seq} tie{\sc -inf} {\sc neg-}agree{\sc [3sg]-pst-neg=seq}\\
\rede{As it (the cow) ran away, as was not willing to be tied, ...} (P) \source{11\_nrr\_01.011} 
\bg.ka uŋ phaʔ-ma ŋ-gaks-a-ŋa-n=na\\
{\sc 1sg} {\sc 3sg} help{\sc -inf} {\sc neg-}agree{\sc -pst-1sg-neg=nmlz.sg}\\
\rede{I was not willing to help him.} (A)


\subsection{Subject complement constructions}\label{subjectcomplement}

Two verbs, namely \emph{tama} \rede{be time to} (lit.: \rede{come}) and \emph{leŋma} \rede{be allright, be accepted} (lit.: \rede{become}), take the whole proposition as their sole argument, usually showing third person singular person marking regardless of the referential properties of the embedded arguments, as shown in \Next. This type of complement construction is referred to as subject complement construction.

\ex. \ag. uŋ mit-a:       haku eko paŋ  cok-ma    ta-ya=na\\
	{\sc 3sg} think{\sc [3sg]-pst}: now one house make-{\sc inf} come{\sc [3sg]-pst=nmlz.sg}\\
	\rede{He thought: Now the time has come to build a house.} \source{27\_nrr\_06.006}
 	\bg. yun-ma leŋ-meʔ=n=em n-leŋ-me-n=n=em?\\
	sit\_down{\sc -inf} be\_allright{\sc [3sg]-npst=nmlz.sg=alt} {\sc neg-}be\_allright{\sc [3sg]-npst-neg=nmlz.sg=alt}\\
	\rede{Is it allright to sit down (or is it not alright)?}
	\bg.liŋkha=ci hoŋliŋwa ca-ma n-leŋ-me-n=na\\
	a\_clan{\sc =nsg} a\_kind\_of\_fish eat{\sc -inf}	{\sc neg-}be\_allright{\sc [3sg]-npst-neg=nmlz.sg}\\ 
	\rede{The Linkhas are not allowed to eat the Honglingwa fish.}

It is also possible that one embedded argument gets raised and triggers agreement on the matrix verb (compare \Next[a] and \Next[b]). 

\ex.\ag.n-dokhumak im-ma n-leŋ-me-n=na\\
{\sc 2sg.poss-}alone sleep{\sc -inf} {\sc neg-}be\_allright{\sc [3sg]-npst-neg=nmlz.sg}\\
\bg.n-dokhumak im-ma n-leŋ-me-ka-n=na\\
{\sc 2sg.poss-}alone sleep{\sc -inf} {\sc neg-}be\_allright{\sc -npst-2sg-neg=nmlz.sg}\\
Both: \rede{You should not sleep alone.}


When transitive verbs are embedded, the choice of which argument to raise is determined by pragmatics, i.e. by the question which argument is pragmatically more salient \Next.\footnote{Information structure has not been studied in depth for Yakkha yet. Impressionistically, an argument can be raised either when the discourse is about that argument (i.e., the topic), but also when the reference of that argument is singled out against other possible referents (i.e., focus).} Thus, transitive embedded clauses are potentially  ambiguous, when there is no overt A argument in the ergative (see \Next[b]). Such ambiguities are resolved by context.

\ex.\ag.kaniŋ kha siau=ci ca-ma n-leŋ-me=ha=ci\\
{\sc 1pl} these apple{\sc =nsg} eat{\sc -inf} {\sc neg-}be\_allright{\sc -npst=nmlz.nsg=nsg}\\
\rede{We are not allowed to eat these apples.} (P)
\bg.ka mok-ma n-leŋ-meʔ-ŋa-n=na!\\
{\sc 1sg[erg/nom?]} beat{\sc -inf} {\sc neg-}be\_allright{\sc -npst-1sg=nmlz.sg}\\
\rede{It is not allright to beat me!} (P) OR\\
\rede{I should not beat (others).} (A)


The verb \emph{leŋma} does not only occur with infinitival complements. It can be found following non-embedded structures such as converbal clauses (see \Next[a]), and it is also found with nominal objects (see \Next[b]), where it basically just means \rede{be, become}. \emph{leŋma} does not only express social acceptability, but also personal attitudes (see \Next[c]). The clause providing the context for the expressed attitude is marked with a sequential marker followed by the additive focus particle, so that the clause acquires a concessive reading. The verb \emph{leŋma} in the third person singular nonpast inflection (\emph{leŋmeʔna}) has also developed into a fixed expression \rede{It's allright}. 
	

\ex. \ag. ceʔya=ŋa=se ŋ-khas-iʔ-wa-n=na, cama men-ja-le n-leŋ-me-n=na\\
		language{\sc =erg=restr} {\sc neg}-be\_full-{\sc 1pl-npst-neg=nmlz.sg}, food	 {\sc neg-}eat{\sc -cvb}	{\sc neg-}be\_alright{\sc [3sg]-npst-neg=nmlz.sg}	\\ 
\rede{One will not be satisfied just by talking, there is no way but to eat food.}
\bg. paŋ=be     ta-meʔ-ma           n-yas-u-ga-n bhoŋ,  aniŋga=ca n-leŋ-me-n, ŋga=ca n-leŋ-me-n\\
	house{\sc =loc} arrive{\sc -caus-inf} {\sc neg-}be\_able-{\sc 3.P-2sg.A-neg} {\sc cond} {\sc 1p.excl.poss=add} {\sc neg}-be\_alright{\sc [3sg]-npst-neg} {\sc 2sg.poss=add} {\sc neg}-be\_alright{\sc [3sg]-npst-neg} 	\\
	\rede{If you cannot bring it (the stone) home, it will neither belong to us nor to you.} \source{37\_nrr\_07.013}
	\bg. ka sa matniŋ=hoŋ=ca leŋ-meʔ-ŋa=na\\
{\sc 1sg} meat without{\sc =seq=add} be\_alright{\sc -npst-1sg=nmlz.sg}\\
\rede{I am fine also/even without (eating) meat.}



\subsection{The Necessitative construction}\label{obl}

\subsubsection{Introduction}
Yakkha has an infinitival construction that expresses necessities, either with a deontic or with a dynamic reading. Deontic modality is understood as the expression of a moral obligation of an event, as assessed by the speaker or by someone else, if one reports on someone else's assessments of a situation (following the distinctions made e.g. in \citet[2]{Nuyts2006_Modality} or in \citealt[12]{Vanlinden2012_Modal}). In  dynamic readings, the expressed necessity is not grounded in the attitudes of the speaker, but in the external circumstances of a situation.  This construction is henceforth referred to as Necessitative Construction, since the deontic/dynamic distinction does not have syntactic consequences.

In most cases, this construction simply consists of an infinitive  to which one of the nominalizing clitics is added (either \emph{=ha} or \emph{=na}, see \Next[a]).\footnote{See \sectref{nmlz-uni-3} for the functions and etymology of these nominalizers.}  Alternatively, the infinitive can also be followed by a copular auxiliary.\footnote{See \sectref{cop-infl} for the forms of the copula. The third person present affirmative is zero.} This is found when the argument is singled out pragmatically, which is tentatively analyzed as focus here \Next[b]. The auxiliary is obligatory in scenarios with first or second person objects. Although the occurrence of the auxiliary in this construction is conditioned by reference and pragmatics, it exhibits an interesting alignment pattern that is conditioned by syntactic roles and referential properties of the arguments (for details cf. below). Both verbs constitute a tightly-knit unit in the necessitative construction. Formally,  it is thus rather a simple clause consisting of an infinitive and an auxiliary. 

\ex.\ag. ka kheʔ-ma=na\\
\sc{1sg} go-\sc{inf[deont]=nmlz.sg} \\
\bg. ka kheʔ-ma ŋan\\
\sc{1sg} go-\sc{inf[deont]} \sc{cop.1sg} \\
Both: \rede{I have to go.} 


The following examples provide an overview of the different readings that the Necessitative Construction may have.
The dynamic reading is exemplified in \Next. Here, the conditions for the necessity of the event lie in the circumstances of the situation: in the fact that people are starving in \Next[a], and in the fact that the stone should not get wet in \Next[b], both from mythical narratives. Deontic examples are shown in \NNext. It is straightforward that both utterances express the attitude of the speakers. What these examples also show is that the infinitives in this construction can be inflected by the nonsingular marker \emph{=ci} when the object has nonsingular number (see \NNext, also mentioned above). In three-argument constructions of the double-object frame, the G argument triggers   \emph{=ci}  \NNext[c].


\ex. \ag. wasik  ta-ni bhoŋ wasik=phaŋ luŋkhwak leŋ-ma=na\\
rain come{\sc [3sg]-opt} {\sc cond} rain{\sc =abl} stone turn\_over-{\sc inf[deont]=nmlz.nsg}\\
\rede{In case it rains, one has to turn the stone away from the rain.}\source{37\_nrr\_07.112}
\bg. yapmi  paŋ-paŋ=be nak-saŋ kheʔ-m=ha\\
		people house-house{\sc=loc} ask-{\sc sim} go-{\sc inf[deont]=nmlz.nsg}\\
	\rede{The people have to go from house to house, asking (for food).} \source{14\_nrr\_02.029}

\ex.\ag.nna=haŋ=maŋ kaniŋ=ca    eŋ=ga  pama=ci pyak luŋma tuk-ma=ha=ci\\
that{\sc =abl=emph} {\sc 1pl=add} {\sc 1pl.incl.poss=gen} parents{\sc =nsg} much liver pour{\sc -inf[deont]=nmlz.nsg=nsg} \\
\rede{So that is why we too have to love our parents very much.}	\source{01\_leg\_07.086}
\bg. thim-m=ha=ci\\
	scold{\sc -inf[deont]=nmlz.nsg=nsg}\\
	\rede{They (the young people) have to be scolded.}
\bg. wa=ci piʔ-m=ha=ci\\
chicken{\sc =nsg} give-{\sc inf[deont]=nmlz.nsg=nsg}\\
\rede{It has to be given to the chicken.} (G)
	
	
It is not uncommon for constructions expressing deontic modality to develop further meanings. A directive speech act is shown in  \Next[a]. The directive is somewhat related to the expression of an attitude, with the difference lying not in the semantics, but in the type of speech act (assertion vs. command). At least one example, taken from a narrative, also points towards an evidential usage of the Necessitative Construction \Next[b]. The context of this utterance is that a king named Helihang is said to have set out to search for the lost language of the Yakkha people. Before doing so, he erected a marble stele at the foot of Mount Kumbhakarna, in order to let his people know whether he was still alive. As long as this stele does not topple over, the people shall know that he is still alive, searching for their language. This example provides a bridging context between deontic and epistemic modality, as one can read it in two ways: either the people infer from the stele standing upright that their king is still alive (epistemic modality), or the people know that they are expected to think that he is still alive, as he told them so when he departed on his search for the language (deontic modality). 


\ex.\ag. ŋkha=ci ham-biʔ-ma=ci,  ŋ-ga-ya=oŋ,\\
those{\sc =nsg} distribute{\sc -V2.give-inf[deont]=nsg} {\sc 3pl-}say{\sc -pst=seq}, \\
\rede{After they said: It has to be distributed among those (who do not have food), ...}\source{14\_nrr\_02.031}
\bg.nna  ceŋ    waiʔ=na=ŋa      haku=ca   kaniŋ i    miʔ=m=ha baŋniŋ: (...)\\
that upright be{\sc [3sg]=nmlz.sg=erg} now{\sc =add} {\sc 1pl} what think{\sc -inf=nmlz.nsg} about\\
\rede{Since that (stele) stands upright, what we have to think even now, ...} \source{18\_nrr\_03.031}


The negated forms of this construction, marked by prefix \emph{men-}, express the necessity for an event not to happen (see \Next[a] and \Next[b]). The deontic meaning has scope over the negation, not the other  way around. Negating  the  necessity of an event is expressed by a construction involving a Nepali loan \emph{parnu} \rede{fall/have to} and the light verb \emph{cokma} \rede{do} \Next[c].  The negation of the infinitive is formally different from the negation in the verbal inflection, and identical to the negation in another uninflected form, the negative converb.

\ex.\ag.nhaŋ     cautara=ca  men-jok-m=ha=ci,  barpipal=ca  men-lip-m=ha=ci\\
and\_then resting\_place{\sc =add} {\sc neg-}make{\sc -inf=nmlz.nsg=nsg} banyan\_tree{\sc =add} {\sc neg-}plant{\sc -inf=nmlz.nsg=nsg} \\
\rede{And neither are they allowed to build resting places, nor to plant banyan trees.}\source{11\_nrr\_01.017}
\bg.chubuk=ka    caleppa=chen   isa=ŋa=ca          heʔniŋ=ca        men-ja-ma=na\\
ashes{\sc =gen} bread{\sc =top} who{\sc =erg=add} when{\sc =add} {\sc neg-}eat{\sc -inf[deont]=nmlz.sg}\\
\rede{No one should ever have to eat the bread made of ashes.} \source{40\_leg\_08.082}
\bg.iha=ca  im-ma     por-a           n-joŋ-me-ŋa-n\\
what{\sc =add} buy{\sc -inf} have\_to{\sc -nativ} {\sc neg-}do{\sc -npst-1sg-neg}\\
\rede{I do not have to buy anything.} \source{28\_cvs\_04.187} 



\subsubsection{Alignment patterns}

After the basic morphological and semantic properties have been introduced, let us now turn to the argument realization in this construction. I will show how the nominalizing clitics attaching to the infinitive are aligned and how the copular auxiliary is aligned. 

\tabref{cop-options} shows the distribution of the two constructions over participant scenarios. The default option in my Yakkha corpus  is the construction without the auxiliary.\footnote{As for the auxiliary, elicited paradigms from different speakers on various occasions and plenty examples from unrecorded spontaneous discourse exist to illustrate its alignment.} In scenarios with third person acting on third, it is the only option. In scenarios with first or second person objects, however, only the construction with the auxiliary is acceptable.

\begin{table}
\centering
\begin{tabular}{llll}
\lsptoprule
{\bf A>P}&{\bf 1}&{\bf 2}&{\bf 3}\\
\midrule
 {\bf 1}	&-&\textsc{inf}+\textsc{cop}&\textsc{inf}+\textsc{cop} or\\
 &&&\textsc{inf}\\
 \midrule
 {\bf 2}&\textsc{inf}+\textsc{cop}&-&\textsc{inf}+\textsc{cop} or\\
 &&&\textsc{inf}\\
 \midrule
 {\bf 3-	extsc{erg}}&\textsc{inf}+\textsc{cop}&\textsc{inf}+\textsc{cop}&\textsc{inf}\\
 \lspbottomrule
\end{tabular}
\caption{Two options of the Necessitative Construction}\label{cop-options}
\end{table}


\tabref{copula} shows the suppletive inflectional paradigm of the copula (present, affirmative). The forms resemble the agreement suffixes in the verbal inflection, so that I assume that there was a phonologically light stem in a earlier stage that got lost over time.\footnote{Related person suffixes in Yakkha are \emph{-ŋ(a)} for first person (exclusive), \emph{-ka \ti -ga} for second person, and \emph{-ci} for the dual. The initial  \emph{/s/} of the plural  forms and the dual forms starting in \emph{nci} cannot be related to the agreement morphology of Yakkha. Limbu though, the eastern neighbour of Yakkha, has a 3nsg object agreement marker \emph{-si} \citep[76]{Driem1987A-grammar}.}  In the past paradigms, there is a stem \emph{sa}, and the person inflection is regular (see \sectref{cop-infl}), but in the present forms the stem is zero. This copula does not have infinitival forms. An equational copula as such does not belong to the general morphological profile of Kiranti languages \citep[276]{Bickel1999Nominalization}. What is crucial for the following discussion of the data is that the copula has only one agreement slot.

\begin{table}[htp]
\begin{center}
\begin{tabular}{llll}
\lsptoprule
&{\sc 1.excl}&{\sc 1.incl}&{\sc 2}\\
\midrule
{\sc sg}&  ŋan& &gan\\
{\sc du} & nciŋan&ncin& ncigan \\
{\sc pl}  & siŋan& sin&sigan\\
\lspbottomrule
\end{tabular}
\caption{Inflection of the copular auxiliary (present indicative, affirmative)}\label{copula}
\end{center}
\end{table}


As for the agreement of the auxiliary, with intransitive verbs it simply agrees with the embedded S argument, as shown in \Next[a] below. If transitive verbs are embedded, the agreement exhibits an intricate combination of hierarchical\footnote{Hierarchical alignment is the \rede{morphological and syntactic treatment of arguments according to their relative ranking on the referential (...) hierarchies} \citep[10]{Siewierska1998On-nominal}. With reference to agreement, this means that \rede{access to inflectional slots for subject and/or object is based on person, number, and/or animacy rather than (or no less than) on syntactic relations}  \citep[66]{Nichols1992Language}.} and ergative alignment,\footnote{Ergative alignment is given when S and P arguments are treated alike and differently from A arguments. 
} as shown schematically in \tabref{align-sum-cop}.  Since there are no forms for the third person, non-local scenarios (3>3) are always marked just by the infinitive and the nominalizers (\emph{=na} or \emph{=ha}). 


\begin{table}[htp]
\begin{center}
\begin{tabular}{ccc}
\lsptoprule
{\bf A>P}	&{\bf 1/2}&{\bf 3}\\
\midrule
 {\bf 1/2}&P&A\\
 {\bf 3-	extsc{erg}}	&P&-\\
\lspbottomrule
\end{tabular}
\todo[inline]{Please provide a caption}
\caption{}
\label{align-sum-cop}
\end{center} 
\end{table}


In mixed scenarios (either 3>\textsc{sap} or \textsc{sap}>3)  the speech-act participant rules out third person (i.e., hierarchical agreement; compare \Next[b] with \Next[c] and \Next[d]). 

\ex. \ag. {\bf nda}  ap-ma {\bf gan}\\
\sc{2sg} go-\sc{inf[deont]} {\sc cop.2sg} \\
\rede{You have to come.}
\bg. {\bf ka} uŋci soʔ-ma  {\bf ŋan}\\
\sc{1sg} \sc{3nsg} watch-\sc{inf[deont]}  \sc{cop.1sg}\\  
\rede{I have to watch them.}	(\textsc{sap}>3: A)
\bg. uŋ=ŋa {\bf nda}  soʔ-ma {\bf gan}\\
 \sc{3s=erg}  \sc{2sg} watch-\sc{inf[deont]}  \sc{cop.2sg}\\  
\rede{He has to watch you.}	(3>\textsc{sap}: P)
\bg. uŋ=ŋa {\bf ka}  soʔ-ma {\bf ŋan}\\
 \sc{3s=erg}  \sc{1sg} watch-\sc{inf[deont]}  \sc{cop.1sg}\\  
\rede{He has to watch me.}	(3>\textsc{sap}: P)

In local scenarios (\textsc{sap}>\textsc{sap}), the copula always agrees with the P argument, as illustrated by \Next. This is a rigid syntactic constraint; the agreement is not manipulable by changes in the information structure. There is no context in which a clause like \Next[a] could mean \rede{you have to watch me}. Comparing local scenarios to intransitive verbs, we can see that S and P are treated alike and differently from  A arguments, hence this is a case of ergative alignment. Ergativity in complement constructions was also found in the neighboring languages Belhare \citep{Bickel2004Hidden, Bickeletal2001Syntactic} and Chintang \citep{Bickeletal2010Ditransitives}. Still, it is crosslinguistically pretty quirky in complement constructions, or at least it is not documented well enough, which has even led linguists like \citet[135]{Dixon1994Ergativity}  to the conclusion that complement clauses are universally accusatively aligned. 
In local scenarios (\textsc{sap}>\textsc{sap}), the copula always agrees with the P argument, as illustrated by \Next. This is a rigid syntactic constraint; the agreement is not manipulable by changes in the information structure. There is no context in which a clause like \Next[a] could mean \rede{you have to watch me}. Comparing local scenarios to intransitive verbs, we can see that S and P are treated alike and differently from  A arguments, hence this is a case of ergative alignment. Ergativity in complement constructions was also found in the neighboring languages Belhare \citep{Bickel2004Hidden, Bickeletal2001Syntactic} and Chintang \citep{Bickeletal2010Ditransitives}. Still, it is crosslinguistically pretty quirky in complement constructions, or at least it is not documented well enough, which has even led \citet[135]{Dixon1994Ergativity}  to the conclusion that complement clauses are universally accusatively aligned. 


\ex. \ag. ka {\bf nda} soʔ-ma {\bf gan}\\
\sc{1sg} \sc{2sg} watch-\sc{inf[deont]} \sc{cop.2sg}\\
\rede{I have to watch you.}	(\textsc{sap}>\textsc{sap}: P)
\bg. nda {\bf ka} soʔ-ma {\bf ŋan}\\
 \sc{2sg}  \sc{1sg} watch-\sc{inf[deont]} \sc{cop.1sg}\\
\rede{You have to watch me.}	(\textsc{sap}>\textsc{sap}: P) 



\tabref{cop-agree} summarizes the copula forms in the Necessitative Construction. It shows the hierarchical alignment according to an \textsc{sap}>3 hierarchy, and in local (\textsc{sap}>\textsc{sap}) scenarios, it shows the ergative alignment. Some alternations between speakers should be mentioned. For one speaker (out of four) not the whole paradigm was possible. The brackets indicate those forms that were rejected and replaced by the construction without the copula. A possible explanation for the rejection of these forms could be a Face-preserving strategy. Explicit reference to a second person agent or a first person patient is avoided in necessitative contexts. Scenarios with a second person A and with a first person P are socially sensitive (e.g. \rede{You have to [give/serve/help] us.}), and therefore speakers prefer to leave reference to any participant unexpressed. An exception to this strategy are those scenarios where both actants have singular number and  are thus clearly identifiable anyway. The avoidance of explicit reference to first person nonsingular patients as a face-preserving strategy is not surprising at all in light of the verbal inflectional paradigms of Yakkha (cf. \sectref{detrans-polite} for details and a possible historical scenario). However, avoiding reference to a second person agent is innovative and limited to this construction.



\begin{table}[htp]
\begin{center}
\begin{tabular}{llllllll}
\lsptoprule
{\bf A>P/G} 	&	{\sc 1sg}&	{\sc 1du}& {\sc 1pl} &{\sc 2sg}&{\sc 2du}&{\sc 2pl}&{\sc 3 }\\
\midrule
{\sc 1sg}  		&\multicolumn{3}{c||}{\cellcolor[gray]{.8}}	& &&				& ŋan  \\
{\sc  1du.ex}   &\multicolumn{3}{c||}{\cellcolor[gray]{.8}}	&gan&ncigan&sigan	& nciŋan  \\
{\sc 1pl.ex}   	&\multicolumn{3}{c||}{\cellcolor[gray]{.8}}	& &&				& siŋan \\
\cline{5-7}
{\sc 1du.in}   	&\multicolumn{6}{c||}{\cellcolor[gray]{.8}}				& ncin  \\
{\sc 1pl.in}   	&\multicolumn{6}{c||}{\cellcolor[gray]{.8}}	 					& sin  \\
\midrule
{\sc 2sg}  		& ŋan	& & &\multicolumn{3}{c||}{\cellcolor[gray]{.8}}	& (gan) \\
\cline{2-2}
{\sc 2du}  	& (ŋan)& (nciŋan) &  (siŋan)	&\multicolumn{3}{c||}{\cellcolor[gray]{.8}}&  (ncigan) \\
{\sc 2pl}  	& (ŋan) &  &	&\multicolumn{3}{c||}{\cellcolor[gray]{.8}}			&  (sigan) \\
\midrule
{\sc 3sg}  	& ŋan	&  (nciŋan/&(siŋan/	 & gan&ncigan& sigan&\cellcolor[gray]{.8}\\
\cline{2-2}
{\sc 3nsg}  &  (ŋan)&ncin)  & 	sin)				&&&							&\cellcolor[gray]{.8} \\
\lspbottomrule
\end{tabular}
\caption{Complete paradigm of the auxiliary in the Necessitative Construction}\label{cop-agree}
\end{center}
\end{table}


The following examples illustrate the alignment in verbs of the double object class by means of  \emph{piʔma} \rede{give}, which is usually aligned with the primary object (treating G identically to the P of monotransitive verbs). This is also reflected in the Necessitative Construction. In mixed scenarios  it is always the speech-act participant that triggers agreement in the auxiliary according to a referential hierarchy [\textsc{sap}>3], as examples \Next[a] and \Next[b] show. In local scenarios it is the G argument that triggers the agreement, as examples \Next[c] and \Next[d] illustrate. Thus, we have again a combination of reference-based and role-based alignment. 
 
\ex. \ag.  uŋ=ŋa {\bf ka} cuwa piʔ-ma {\bf ŋan}\\
 \sc{3nsg=erg}  \sc{1sg} beer\sc{}  give-\sc{inf[deont]} \sc{cop.1sg}\\
\rede{He has to give me beer.} (3>\textsc{sap}: G) 
\bg. {\bf ka} uŋci cuwa  piʔ-ma {\bf ŋan}\\
\sc{1sg[erg]} \sc{3nsg} beer\sc{}  give-\sc{inf[deont]} \sc{cop.1sg}\\
\rede{I have to give them beer.} (\textsc{sap}>3: A)
\bg. ka {\bf njiŋda} cuwa piʔ-ma {\bf cigan}\\
\sc{1sg[erg]} \sc{2du} beer\sc{} give-\sc{inf[deont]} \sc{cop.2du}\\ 
\rede{I have to give you beer.}	(\textsc{sap}>\textsc{sap}: G)
\bg. nda {\bf ka} cuwa piʔ-ma {\bf ŋan}\\
\sc{2sg[erg]} \sc{1sg} beer\sc{} give-\sc{inf[deont]} \sc{cop.1sg}\\ 
\rede{You have to give me beer.}	(\textsc{sap}>\textsc{sap}: G)


If the reasoning behind hierarchical alignment systems is transferred to the objects of three-argument verbs, as e.g. suggested in \citet{Haspelmath2004Explaining}, \citet{Malchukovetal2010Ditrans-overview}, and \citet{ Siewierska2003Person}, one should also find hierarchical alignment of person marking with respect to the T and the G of three-argument verbs. The expected, typical ditransitive scenario contains a referentially high G argument and a referentially low T argument. The interesting question is what happens when this relation is reversed, i.e. when the T argument is higher on a referential hierarchy than the G argument. This is illustrated by the verb \emph{soʔmeʔma}  \rede{show} (see \Next). If T is a speech-act participant  and G is a third person, the auxiliary indexes T instead of G, and G shows a strong tendency to receive locative case marking. The locative case marking is expectable, as “the construction which is more marked in terms of the direction of information flow should also be more marked formally” \citep[128]{Comrie1989Language}. 

\exg.ka \textbf{nda} appa-ama=be soʔmeʔ-ma \textbf{gan}\\
		 {\sc 1sg[erg]}  {\sc 2sg[nom]}  mother-father{\sc =loc} show{\sc -inf} {\sc cop.2sg}\\
		\rede{I have to show you to my parents.} (T[\textsc{sap}]→G[3])
		


The infinitival form of the lexical verb  in the Necessitative Construction usually hosts one of the nominalizing clitics. Their  function is best described as focus or lending authority to the assertion (cf. \sectref{nmlz-uni-3}). Let us now turn to their alignment. 

There are two different alignment patterns, depending on whether we are dealing with the construction  with or without the auxiliary. The construction without the auxiliary is  found only when the P argument has third person reference, because the auxiliary is obligatory in scenarios with first or second person P arguments. The  alignment here is clearly ergative, as shown in \tabref{nom-align-without} and in example \Next. The choice of \emph{=na} vs. \emph{=ha} is conditioned by the number of the S argument in intransitive environments (for all persons) and by the number of the (third person) P argument in transitive environments. 


\ex. \ag. kaniŋ kheʔ-m{\bf =ha},                       nhaŋa   hum-se         kheʔ-ma{\bf =na}\\
{\sc 1pl} go{\sc -inf[deont]=nmlz.nsg} and\_then bury{\sc -sup} carry\_off{\sc -inf[deont]=nmlz.sg}\\
\rede{We have to go, and we have to carry him off to bury him.} (P=sg) \source{29\_cvs\_05.049}
\bg. {\bf na}   {\bf khibum}  imin kaŋ-nhaŋ-ma{\bf =na}...\\
this cotton\_ball how drop-{\sc V2.send-inf=nmlz.sg} \\
\rede{As for how this cotton ball has to be dropped...} (P=sg) \source{22\_nrr\_05.092} 
\bg.  khaʔniŋgo, kaniŋ=go       cekci=be    niʔ-m{\bf =ha} \\
but {\sc 1pl[erg]=top}  iron{\sc =loc} fry-{\sc inf[deont]=nmlz.nsg} \\
\rede{But we have to fry (bread) in something made of iron.} (P=nsg/mass) \source{29\_cvs\_05.075}


\begin{table}[htp]
\begin{center}
\begin{tabular}{lll}
\lsptoprule
{\bf A>P(=S)} 	&	{\sc 3sg}&{\sc 3nsg}\\
\midrule
1		&=na &  =ha\\
\midrule
2		&=na& =ha\\
\midrule
3		&=na&=ha\\
\lspbottomrule
\end{tabular}
\caption{Alignment of the nominalizers, construction without auxiliary}\label{nom-align-without}
\end{center}
\end{table}
 
 
The picture is slightly more complex in the construction with the auxiliary. It is summarized in 
\tabref{nom-align}. When the lexical verb is intransitive, first and second person subjects always trigger \emph{=ha} (see also \Next). In transitive verbs, first and second person P arguments also trigger \emph{=ha}, again resulting in ergative alignment, since speech-act participant S and P arguments are treated identically, but differently from A arguments. 

\exg. ka hoŋkoŋ kheʔ-m{\bf =ha} ŋan\\
{\sc 1sg} Hong\_Kong go-{\sc inf[deont]=nmlz.nsg} {\sc cop.1sg}\\
\rede{I have to go to Hong Kong.}


\begin{table}[htp]
\begin{center}
\begin{tabular}{llllll}
\lsptoprule
{\bf A>P/G} 	&	1&2&{\sc 3sg}&{\sc 3nsg}&S\\
\midrule
{\sc }1sg  		& \cellcolor[gray]{.8}&&=na &  &=ha \\
% \cline{4-4}
 {\sc 1nsg.excl}   &	\cellcolor[gray]{.8}&=ha 				&\multicolumn{2}{c}{}&=ha\\
%\cline{3-3}
{\sc 1nsg.incl}   	&\multicolumn{2}{c}{\cellcolor[gray]{.8}}		& \multicolumn{2}{c}{=ha}&=ha\\
\midrule
{\sc 2sg}  		&    & \cellcolor[gray]{.8}	&=na& & =ha\\
%\cline{4-4} 
{\sc 2nsg}  		& =ha  & \cellcolor[gray]{.8}	&\multicolumn{2}{c}{=ha}&=ha\\
\midrule
{\sc 3sg}  	& =ha&=ha &=na&&=na\\
%\cline{4-4}
{\sc 3nsg} 	&	& 	&\multicolumn{2}{c}{=ha}&=ha \\
\lspbottomrule
\end{tabular}
\caption{Alignment  of the nominalizers, construction with auxiliary}\label{nom-align}
\end{center}
\end{table}

The singular clitic \emph{=na} is only found when a singular A acts on a third person singular P argument. As soon as one participant, no matter which one, has nonsingular reference, \emph{=ha} attaches to the infinitive, illustrated here with {\sc 1sg.A} acting on {\sc 3.sg/nsgP} in \Next and with {\sc 1pl.A} acting on {\sc 3.sg/nsgP}  in \NNext. This is a hierarchical pattern, according to a number hierarchy [nsg>sg].

\ex. \ag. ka {\bf na} sambakhi ca-ma{\bf =na} ŋan\\
{\sc 1sg[erg]} this potato eat-{\sc inf=nmlz.sg} {\sc cop.1sg}\\
\rede{I have to eat this potato.}
\bg.  ka {\bf kha} sambakhi(=ci) ca-m{\bf =ha} ŋan \\
{\sc 1sg[erg]} these potato{\sc =nsg} eat-{\sc inf=nmlz.nsg} {\sc cop.1sg}\\
\rede{I have to eat these potatoes.}

\ex. \ag. kaniŋ {\bf na} phʌrsi ca-m{\bf =ha} siŋan\\
{\sc 1pl[erg]} this pumpkin eat-{\sc inf=nmlz.nsg} {\sc cop.1pl.excl}\\
\rede{We have to eat this pumpkin.}
\bg.  kaniŋ {\bf kha} sambakhi(=ci) ca-m{\bf =ha} siŋan \\
{\sc 1pl[erg]} these potato{\sc =nsg} eat-{\sc inf=nmlz.nsg} {\sc cop.1pl.excl}\\
\rede{We have to eat these potatoes.}

\subsubsection{Comparative notes and discussion}


The alignment in the necessitative construction is centered around two factors: person (\textsc{sap}) and syntactic role (P), and the question of which selection principle applies is conditioned by different scenarios: speech-act participant reference is the relevant factor in mixed scenarios, and P is the relevant factor in local scenarios. 

The  person marking in regular verbs supports this reasoning, as the alignment is more consistent across the columns (representing P) than across the rows (representing A) in the paradigm (cf. \citealt{Schackow2012_Grammatical}). Speech-act participant markers are aligned different than the third person markers. \textsc{sap} markers show ergative or hierarchical alignment, while the third person is largely aligned accusatively. As I have shown above, the alignment of the nominalizing clitics \emph{=na} and \emph{=ha} is sensitive to participant scenarios.  The distribution of tense-marking allomorphs is scenario-based, too (cf. \sectref{tense}).  The distinction between speech-act participants and third persons figures as a factor in other constructions: the ergative case is not overtly marked on first and second person pronouns, and the verbal inflection has person markers that have underspecified speech-act participant reference, e.g. \emph{-m} (for first and second person plural A arguments), \emph{-i} (for first and second person plural S and P arguments). Hence, although the alignment of the necessitative construction is unique, the selection principles leading to this pattern can also be found in other domains of the grammar of Yakkha. Alignment splits serve as scenario classifiers in Yakkha (cf. also \citealt{Bickel1992Motivation} on Belhare).\footnote{In search for a  Yakkha-internal explanation of the pattern at hand it is tempting to attribute the hierarchical alignment to the defective paradigm of the copula: the absence of third person forms provides an empty slot that is filled by the material marking a speech-act participant. The absence of third person forms can, however, not explain the entire picture.  Firstly, it does not account for the fact that the copula is obligatory when the object is a speech-act participant, but not when the subject is a speech-act participant. Secondly, there are also traces of hierarchical alignment in the verbal paradigm, although there are markers for the third person available. Scenarios of a 3.A>2.P type, for instance, largely neglect reference to the third person A argument (cf. \sectref{verb-infl}). Furthermore, the defective paradigm of the copula can also not explain the ergative alignment in local scenarios, i.e. why it is always the object that is indexed on the verb in these scenarios.}


Hierarchical patterns are a recurrent feature  in other Tibeto-Burman languages that show person marking, too, e.g. in Rawang, Kham and in many of the Kiranti languages (cf. \citealt[317]{DeLancey1989Verb}, \citealt[311]{LaPolla1992Dating}, \citealt[30]{LaPolla2003_Overview}, \citealt{LaPolla2007Hierarchical}, \citealt[473]{Ebert1987Grammatical}, \citealt[1]{DeLancey2011_Notes}, \citealt[398]{Watters2002A-grammar}). The particular pattern that was found in Yakkha has been attested, with varying degrees of transparency, in Tibeto-Burman languages of various geographic origins and sub-branches, reaching back to Tangut sources as early as the 12th century \citep{Kepping1975_Subject, Kepping1994The-conjugation}. \citet{Watters2002A-grammar}, who compared the person marking system of the Kham dialects with person marking in other Tibeto-Burman languages, found hierarchical patterns in Gyarong, Nocte and Western Kiranti languages with \textsc{sap} outranking third person and, crucially, with a preference for the object in conflicting scenarios (\citet{Nagano1984A-historical}, cited in \citealt[388]{Watters2002A-grammar}). The alignment in the Yakkha necessitative construction also resembles Proto-Tibeto-Burman agreement as reconstructed in \citet{DeLancey1989Verb}, who characterizes the system as  “a split ergative agreement pattern in which agreement is always with a 1st or 2nd person argument in preference to 3rd person, regardless of which is subject or object.” \citep[317]{DeLancey1989Verb}.\footnote{As noted earlier, Tibeto-Burman reconstruction and subgrouping are far from being settled, see, e.g., \citet{Thurgood1984The-Rung, DeLancey1989Verb, DeLancey2010_Towards, DeLancey2011_Notes, LaPolla1992Dating, LaPolla2012_Comments, Jacques2012_Agreement}. Under the assumption that  a sub-branch \emph{Rung} exists (see the references from Thurgood and LaPolla), those Tibeto-Burman languages showing agreement would be related on a lower level than Proto-Tibeto-Burman.}


A potentially interesting side note in this respect is that the same alignment split (ergative and hierarchical, conditioned by a distinction of speech-act participant vs. third person) is also found in the verbal agreement of the Caucasian (Nakh-Daghestanian) language Dargwa \citep[208]{Zuniga2007_From}. The pattern is neither a one-off case, nor is it restricted to Tibeto-Burman. Of course, more cross-linguistic data would be necessary to be able to corroborate any functional-typological explanation for this alignment split. Indeed, it would be exciting to discover that this pattern is more widespread than  assumed at this stage. According to \citet{Serdobolskaya2009_Raising}, referential properties also play a role  in raising constructions in some Uralic, Turkic and Mongolic languages languages, and she tentatively suggests this to be an areal feature. More data on other languages could help to answer the question if this feature really has an areal distribution.  


On a final note, the auxiliary in the necessitative construction  is often replaced by the third person form of the Nepali auxiliary \emph{pʌrnu} \rede{fall/have to}, as shown in \Next. In this calqued structure, the peculiar alignment pattern of the necessitative construction gets lost.

\exg.kʌhile-kʌhile mamu=ŋa=ca  taʔ-ma=ci       pʌrchʌ\\ 
sometimes girl{\sc =erg=add} bring-{\sc inf=nsg} {\sc has\_to}\\
\rede{Sometimes, the girl has to bring them (further wives), too.} \source{06\_cvs\_01.044}

 		
\section{Inflected complement clauses} \label{fin-comp}

Inflected complement clauses are found with predicates of  cognition like \emph{nima} \rede{see, get to know}, \emph{khemma} \rede{hear}, \emph{miʔma} \rede{think, hope, consider, like}  (see \sectref{cognition-pred}) and utterance predicates such as  \emph{luʔma} \rede{tell, say}, \emph{yokmepma} \rede{tell (about)}, \emph{chimma} \rede{ask} (see \sectref{utterance-pred}). \tabref{overview-all} (on page \pageref{overview-all}) provides the list of all complement-taking predicates that embed clauses with inflected verbs in Yakkha. 

Complement clauses in Yakkha distinguish direct speech and indirect speech. In case of indirect speech, the  clauses contain nominalized inflected verbs.  In case of direct speech, the verbs are able to express the full range of verbal categories found in independent clauses. Some types of complement clauses are marked by the conjunction \emph{bhoŋ} that also functions as a marker of conditional clauses,  direct speech and purpose clauses. The latter seem to have developed out of clauses containing direct  speech (see \sectref{adv-cl-fin-purp}). Some complement-taking verbs are polysemous, their semantics depending on whether they occur with an infinitival or with an inflected complement. The verb \emph{miʔma}, for instance, means \rede{like (doing)} with infinitival complements but \rede{think, hope, consider} with inflected complements. The verb \emph{nima} means \rede{know (how to do)} with infinitival complements, but \rede{see, get to know} with inflected complements. Some nouns may embed propositions  as well, by means of the complementizer \emph{baŋna/baŋha} (cf. \sectref{noun-compl}). 


\subsection{Predicates of cognition and experience}\label{cognition-pred}

The complement-taking predicates of cognition can be further classified into predicates of knowledge, perception, experience and propositional attitude. 
 The most common predicate embedding inflected complements is \emph{nima} \rede{to see or get to know something}, exemplified in \Next. The embedded verb is mostly in one of the indicative past tenses and has to carry one of the nominalizers \emph{=na} or \emph{=ha} (cf. Chapter \ref{ch-nmlz}). The embedded subject obligatorily triggers agreement both on the embedded verb and on the main verb. On the main verb, it triggers object marking (see \Next[b]). This can be explained from a semantic perspective: perceiving that a person is doing something implies perceiving that person. The argument realization in the complement clause is identical to that of independent clauses. The subject of the embedded clause receives ergative case marking when the embedded verb is transitive, although the A argument simultaneously triggers object agreement on the matrix verb. 
 
 The perspective in complement clauses is that of the speaker, and the zero point is the speech situation; hence the embedded clause contains indirect speech or indirect questions.

\ex. \ag.khy-a-ma=hoŋ          so-ks-u=niŋa                 eko yapmi  bhirik=pe    het-u=na                      nis-uks-u.\\
go{\sc -pst-prf=seq}  watch{\sc -prf-3.P[pst]=ctmp} one person cliff{\sc =loc} get\_stuck{\sc -3.P[pst]=nmlz.sg} see{\sc -prf-3.P[pst]}\\
\rede{He went there, and when he looked, he saw a man caught in a cliff.} \source{01\_leg\_07.319}
\bg. ka uŋci=ŋa toŋba ŋ-uŋ-ma-si-a=ha=ci nis-u-ŋ-ci-ŋ=ha\\
{\sc 1sg[erg]} {\sc 3nsg=erg} beer  {\sc 3pl}-drink-{\sc inf-ipfv-pst=nmlz.nsg=nsg} see{\sc -3.P[pst]-1sg.A-3nsg.P-1sg.A=nmlz.nsg}\\
\rede{I saw that they were drinking beer.}


Other verbs of this kind are  \emph{miʔma} \rede{think, consider, hope} and the propositional attitude verb \emph{eʔma} \rede{perceive, have impression}, shown in \Next and \NNext[a], respectively.  Complement clauses embedded to  \emph{eʔma} can also be marked by the equative case  \emph{loʔa} \rede{like} (see \NNext[b]). This marker is also used in equative constructions, where it subcategorizes nouns (NPs) and adjectives.  Again, one can see that the embedded subject is cross-referenced by subject marking on the embedded verb and by object marking on the main verb. The form \emph{*etuŋna} (assuming third person singular \rede{dummy} object agreement) would be ungrammatical in \NNext[b] and \NNext[c]. 

\ex.\ag. nda cama ca-ya-ga=na mi-nuŋ-nen=na\\
{\sc 2sg[erg]} rice eat-{\sc pst-2=nmlz.sg} think-{\sc prf-1>2=nmlz.sg}\\
\rede{I have thought you ate the rice.} %example for spec. mass noun and =na
\bg.ŋkhatniŋ=ŋa ka=ca sy-a=na mit-a\\
that\_time{\sc =ins} {\sc 1sg=add} die{\sc [3sg]-pst=nmlz.sg} think{\sc -imp[1.P]}\\
\rede{At that time, consider me dead (think that I am dead), too.} \source{18\_nrr\_03.018}
	

\ex. \ag. mi mi=na et-u-ŋ=na, khatniŋgo ma leks-a-bhoks-a=na\\
	fire small{\sc =nmlz.sg} perceive{\sc -3.P[pst]-1sg.A=nmlz.sg}, but big become{\sc [3sg]-pst-v2.split-pst=nmlz.sg}	\\
	\rede{It seemed to me that the fire was small, but suddenly it flamed up.}  
	\bg. haŋ=ha c-wa-g=ha loʔa eʔ-nen=na\\
be\_spicy{\sc =nmlz.nc} eat-{\sc npst-2.A[3.P]=nmlz.nc} like perceive{\sc [pst]-1>2=nmlz.sg}  \\
\rede{It seems to me that you eat spicy things.} OR
\rede{You seem to me like someone who eats spicy things.}
\bg. pyak ŋ-waiʔ=ya loʔa et-u-ŋ-ci-ŋ=ha, khatniŋgo m-man=ha=ciǃ\\
many {\sc 3sg-}exist{\sc =nmlz.nsg} like perceive{\sc -3.P[pst]-1sg.A-3nsg.P-1sg.A=nmlz.nsg} but {\sc neg-cop[neg]=nmlz.nsg=nsg}\\
\rede{It seemed to me that there were many, but there were none!}


When the embedded clause has hypothetical or irrealis status, it is in the optative,  and the main verb \emph{miʔma} \rede{think} acquires the reading \rede{hope}.

\exg.hakt-a-ŋ-ga-ni bhoŋ mit-a-masa-ŋ=na\\
send{\sc -sbjv-1sg.P-2-opt} {\sc comp} think{\sc -pst-pst.prf-1sg=nmlz.sg}\\
\rede{I had hoped that you would send me something.}


Another use of \emph{bhoŋ} is marking complements of experiential verbs like \emph{consiʔma} \rede{be happy} and verbs like \emph{niŋwa hupma} \rede{make a plan together, decide collectively}. In such sentences, the embedded clause contains direct speech, with the full range of inflectional categories being possible on the embedded verb. The speaker shifts the perspective to the subject of the embedded clause (see e.g. first person singular A in \Next[a], hortative mood in \Next[b]), interrogative mood \Next[c]).  The literal translation of \Next[a] would be \rede{He rejoiced: I saved another man's life!}. 
 
 
\ex.\ag.lamdikpa  cahi yapmi  hiŋ-u-ŋ=na       bhoŋ cond-a-sy-a-ma.\\
traveller {\sc top} person save{\sc [pst]-3.P-1sg.A=nmlz.sg} {\sc comp} be\_happy{\sc [3sg]-pst-mddl-pst-prf}\\
\rede{The traveller was happy that he saved another person's life.} \source{01\_leg\_07.331}
\bg.yaŋchalumba   a-phu=nuŋ                      khi khon-se         puŋda=be    khe-ci   bhoŋ niŋwa hupt-a-ŋ-c-u-ŋ.\\
third-born\_boy {\sc 1sg.poss-}elder\_brother{\sc =com} yam dig{\sc -sup} jungle{\sc =loc} go{\sc -du[sbjv]} {\sc comp} mind unite{\sc -pst-excl-du-3.P-excl}\\
\rede{So my third-born brother and I decided to go to the jungle to dig yams.}\source{40\_leg\_08.006}
\bg.haku imin nak-ma=ci                  bhoŋ, uŋci-niŋwa       ŋ-hupt-u.\\
now how ask{\sc -inf[deont]=nsg} {\sc comp} {\sc 3nsg.poss-}mind {\sc 3pl.A-}unite{\sc -3.P[pst]}\\
\rede{They decided about how to ask them (for food).} \source{14\_nrr\_02.14}


Since some complement-taking predicates are also able to take nominal objects,  they yield a potential ambiguity of interpretations as inflected  complement clauses or circumnominal relative clauses (see also \citet[272]{Bickel1999Nominalization}, \citealt[120, 143]{Noonan2007Complementation}): the two propositions \rede{I hear the one who is singing} and \rede{I hear that someone is singing} refer to exactly the same situation in the world. The sentences in example \Next are potentially ambiguous, and there is no structural difference that could resolve the ambiguity (see also \sectref{internally-headed-rc}). It seems that internally headed relative clauses have developed out of complements of perception verbs and were then extended analogically to other kinds of verbs which rule out a complement reading, such as \emph{tumma} \rede{find}, allowing only nominal arguments.

\ex. \ag.nna  ten=be=ha=nuŋ waleŋ    ten=be=ha    kholoŋba=ci    chem-khusa    uŋ-khusa    n-ja-ya    nis-u-ŋ-ci-ŋ.\\
that village{\sc =loc=nmlz.nsg=com} Waleng village{\sc =loc=nmlz.nsg} guests{\sc =nsg} tease{\sc -recip} pull{\sc -recip} {\sc 3pl-eat.aux-pst.nmlz.nsg} see{\sc [pst]-3.P-1sg.A-3nsg.P-1sg.A}\\
\rede{I saw the people of the village and the guests teasing and pulling each other (jokingly in a dance).} OR\\
\rede{I saw how the people of the village and the guests were teasing and pulling each other (jokingly in a dance).} \source{41\_leg\_09.017}
\bg.  ka  haŋcaŋcaŋ chu-ya-ŋ=na                  nis-a-ma.\\
{\sc 1sg} dangling hang{\sc -pst-1sg=nmlz.sg} see{\sc -pst-prf[3A;1.P]}\\
\rede{They saw me dangling (there).} OR \\
\rede{They saw how I dangled there.} \source{42\_leg\_10.040}


 
\subsection{Utterance predicates}\label{utterance-pred}

Utterance predicates distinguish between those that embed indirect speech (as already introduced in the predicates of cognition) and those that embed direct speech. Some verbs may occur with both  indirect speech and direct speech, for instance \emph{miʔma} \rede{think}. Predicates embedding indirect speech or indirect questions are, for instance, \emph{yokmeʔma} \rede{tell (about)} and \emph{khemmeʔma} \rede{tell, make hear}  (see \Next), while direct speech is embedded mainly by the predicates  \emph{kama} \rede{say, call} and \emph{luʔma} \rede{tell}. 

The complement clauses of both types of utterance predicates can be marked by \emph{bhoŋ}, the complementizer that is also found as a quotative marker and as conjunction on purpose clauses and  conditional clauses.\footnote{These uses of \emph{bhoŋ} are parallel to the functional distribution of the  Nepali conjunction \emph{bhʌne}.} This complementizer is frequently found in clauses containing indirect speech and indirect questions (see \Next), and rarely on clauses containing direct speech (see \NNext[e]).  

\ex.\ag. ak=ka kamniwak=ci,  ka isa om,  isa om bhoŋ     uŋci=ŋa=ca  n-yokme-me-ka-nin.\\
  {\sc 1sg.poss=gen} friend{\sc =nsg} {\sc 1sg} who {\sc cop} who {\sc cop} {\sc comp} {\sc 3nsg=erg=add} {\sc neg}-tell\_about-{\sc npst-2-pl.neg}\\
 \rede{My friends, they  will also not tell you who I am.} \source{14\_nrr\_02.25}
%\bg.   ka=ca         isa=ca        yapmi=ci=ca            n-yokmet-a-n-u-m-ci-m-nin!              \\
 % {\sc 1sg=add} who{\sc =add} person{\sc =nsg=add} {\sc neg-}tell\_about{\sc -imp-pl-3.P-2pl.A-3nsgP-pl.neg}  \\
  % \rede{Do not tell the people who I am, too!} \source{14\_nrr\_02.26}
\bg. philm=be=cen       haku i=ya       i=ya           cok-ma-sy-a=na        bhoŋ khem-meʔ-meʔ-nen=ba,       ani.\\
film{\sc =loc=top}  now  what{\sc =nmlz.nc} what{\sc =nmlz.nc}  do{\sc -inf-aux.prog[3sg]-pst=nmlz.sg} {\sc comp}  hear{\sc -caus-npst-1>2=emph} aunt\\ 
\rede{I will tell you now what he was doing in the film, auntie.} \source{34\_pea\_04.007}
\bg.nna  cuʔlumphi isa=ŋa   thukt-u=na bhoŋ pyak pyak yapmi=ci=nuŋ ceʔya leks-a=ha.\\
that stele who{\sc =erg} erect{\sc -3.P[pst]=nmlz.sg} {\sc comp} many many people{\sc =nsg=com} matter become{\sc [3sg]-pst=nmlz.nc}	\\
\rede{Many people have discussed who erected that stele.} \source{18\_nrr\_03.002--3}

Direct speech (quotes) may show the full range of tense, aspect and modal marking, and also any type of illocutionary force marking, clause-final exclamative particles and the like. The person marking on the embedded verb follows the perspective of the embedded subject, not the perspective of the speaker. Direct speech is generally embedded without any complementizer, as shown in the following examples, where \emph{kama} \rede{say}, \emph{luʔma} \rede{tell},  \emph{miʔma} \rede{think} and \emph{chimma} \rede{ask} function as matrix verbs (see \Next, and note that  \Next[e] presents an exceptional case without \emph{bhoŋ}).

\ex. \ag. khamba=ŋa a-tokhumak yep-ma n-ya-me-ŋa-n=na ka-saŋ por-a-khy-a=na\\
pillar{\sc =erg} {\sc 1sg.poss}-alone stand{\sc -inf} {\sc neg-}be\_able{\sc -npst-1sg-neg=nmlz.sg} say{\sc -sim} fall{\sc [3sg]-pst-V2.go-pst=nmlz.sg}\\
\rede{I cannot stand alone, said the pillar and fell down.} \source{27\_nrr\_06.17}
\bg. heko=ha nwak=ci=ŋa haku nda nhe uŋ-ma n-dokt-wa-ga-n=na n-lu-ks-u\\
	other{\sc =nmlz.nsg} bird{\sc =nsg=erg}	now {\sc 2sg} here come\_down{\sc -inf} {\sc neg-}get\_to\_do{\sc -npst-2.A[3.P]-neg=nmlz.sg} 	{\sc 3pl.A}-tell-{\sc prf-3.P[pst]}\\
	\rede{The other birds told him: Now you will not get the chance to come down here any more.} \source{21\_nrr\_04.035}
	\bg. hetne=le ta-ʔi=ya m-mit-a-ma=hoŋ uŋci m-maks-a-by-a-ma=ca\\
	where{\sc =ctr} come{\sc -1pl=nmlz.nsg} {\sc 3pl-}think{\sc -pst-prf=seq} {\sc 3nsg} {\sc 3pl}-be\_surprised-{\sc pst-V2.give-pst-prf=add}\\
	\rede{They also wondered: Where on earth did we arrive?} \source{22\_nrr\_05.30}
\bg. hetniŋ-hetniŋ om=em men=em mit-wa-m=ha\\
sometimes yes{\sc =alt} no{\sc =alt} think{\sc -npst[3.P]-1pl.A=nmlz.nc}\\
\rede{Sometimes we think: Is it (true) or not?}
\bg.nani,  i=na           yubak?, n-chimd-uks-u.\\
child what{\sc =nmlz.sg} property {\sc 3pl.A-}ask{\sc -prf-3.P[pst]}\\
\rede{Child, what thing (do you want)?, they asked her.} \source{37\_nrr\_07.006}
\bg.imin kaniŋ cin-a             cok-meʔ-nen-in?             bhoŋ elaba=ci=ŋa               n-lu-ks-u-ci.\\
how {\sc 1pl} recognize{\sc -nativ} do{\sc -npst-1>2-2pl} {\sc quot} a\_clan{\sc =nsg=erg} {\sc 3pl.A-}tell{\sc -prf-3.P[pst]-3nsg.P}\\
\rede{How will we recognize you?, asked the Elabas (people of Elaba clan).} \source{39\_nrr\_08.11}


In a few cases, the reportative particle \emph{=bu} is also found on embedded speech (see \Next[a] for direct speech, and \Next[b] which could be either direct or indirect speech). 

\ex.\ag.por-a        cog-a=bu         lu-ya-n-u-m.\\
study{\sc -nativ} do{\sc -imp=rep} tell{\sc -imp-pl-3.P-2.A}\\
\rede{Tell him to study.} (\rede{Tell him: study!}) \source{01\_leg\_07.094}
\bg.kham=be=ca             n-yuŋ-me-n,                    siŋ=be=ca           man,     hiʔwa=ga    hiʔwa wait=na=bu            ŋ-ga-me\\
earth{\sc =loc=add} {\sc neg-}live{\sc [3sg]-npst-neg} tree{\sc =loc=add} {\sc neg.cop} air{\sc =gen} air exist{\sc [npst;3]=nmlz.sg=rep} {\sc 3pl-}say{\sc -npst}\\
\rede{It is neither living on the ground nor in the trees, it just lives in the air, they say.}\source{21\_nrr\_04.052}


Example \Next serves to illustrate that question words remain in situ in direct speech; they do not get extracted. Semantically, the question word belongs to the embedded clause because it is an argument of \emph{khuʔma} \rede{bring}. It is, however, the focus of the question and hence appears as question word. The particle \emph{=le} serves as contrastive focus marker, and it emphasizes the cluelessness of the speaker  in questions and his eagerness to get to know the answer.

\exg. nda ka ina=le khut-a-ŋ ly-a-ŋ-ga=na?\\
		{\sc 2sg[erg]} {\sc 1sg} what{\sc =ctr}	bring{\sc -imp-1sg} tell{\sc -pst-1sg.P-2.A=nmlz.sg}\\
	\rede{What did you tell me to bring?} 




%42_leg_10.039 %yo         amanuŋ                anaŋachen       heʔne khyana     bhoŋ ibebe    yogama.
%42_leg_10.018% iha                ayuksaha       bhoŋ khaʔla   siŋchoŋbe        soŋniŋago              phopcibale          weʔna.
%04_leg_03.007%nhaŋgo        mima  uyana                      hoŋbe    mima  lommeʔnabhoŋ                         kusuksu



\subsection{Complement-taking nouns}\label{noun-compl}

Nouns can embed finite complement clauses via the use of the complementizers \emph{baŋna} (if the head noun has singular number) and \emph{baŋha} (if the head noun has nonsingular number or non-countable reference). 

Example \Next[a] shows how   \emph{baŋna} links  propositions to a head noun from the semantic domain of saying.  In \Next[b] the head noun is from the experiential domain. As \Next[c] illustrates, the complement clause can be of considerable length and complexity. Narratives often conclude with a structure as shown in \Next[d]. 

\ex. \ag.  na   chuʔlumphi helihaŋ=ŋa  thukt-u=na  baŋna ceʔya\\
this stele Helihang{\sc =erg} erect{\sc -3.P[pst]=nmlz.sg} {\sc comp} matter\\
\rede{the matter about how/when this stele was erected by Helihang} \source{18\_nrr\_03.004}
\bg.a-ma=ŋa                moŋ-meʔ=ha                baŋna kisiʔma=ŋa\\
{\sc 1sg.poss-}mother{\sc =erg} beat{\sc -npst[1.P]=nmlz.nsg} {\sc comp} fear{\sc =erg}\\
\rede{Out of fear that my mother would beat us, ...} \source{40\_leg\_08.030}
\bg. nhaŋroŋ-hoknam=be    me-wasiʔ-le      me-wacek-le        cameŋba  ca-ma     bhoŋ samba-nwak leks-iʔ-wa,          phak men-semeʔ-le    ca-ya  bhoŋ phak loʔwa pombrekpa,  chippakeppa    leks-iʔ-wa baŋna eko lu-yukt-a=na        ceʔya wɛʔ=na\\
a\_festival{\sc =loc} {\sc neg}-wash{\sc -cvb} {\sc neg}-wash{\sc -cvb} food eat{\sc -inf} {\sc cond} vulture become{\sc -1pl-npst} pig {\sc neg}-greet{\sc -cvb} eat{\sc -sbjv} {\sc cond} pig like lazy disgusting become{\sc -1pl-npst} {\sc compl} one tell{\sc -V2.put-3.P=nmlz.sg} story exist{\sc [3sg]=nmlz.sg}\\
\rede{We have a saying that during Maghe Sankranti, if we eat without bathing, we will become a vulture, and if we eat without
bowing down to the pig, we will become lazy and disgusting like a pig.} \source{40\_leg\_08.047}
\bg.nnakha lalubaŋ phalubaŋ=ci    ŋkha liŋkha=ci      iya            ŋ-wa-ya,        ŋkha=ci ghak eko n-leks-a=bu baŋna taplik om\\
those Lalubang Phalubang{\sc =nsg} those Linkha\_person{\sc =nsg} what {\sc 3pl-}exist{\sc -sbjv} those{\sc =nsg} all one {\sc 3pl-}become{\sc -pst=rep} {\sc compl} story {\sc cop}\\
\rede{Those Lalubang and Phalubang, those Linkhas, whatever happens, they all became one (they say), (this) is what the story is about.} \source{22\_nrr\_05.134}

Such structures do not necessarily need a head noun, as example \Next shows.

\exg. keŋ-khuwa            u-thap-ka              kolem=na         sumphak baŋna        na=maŋ     om.\\
bear\_fruit{\sc -nmlz} {\sc 3sg.poss-}plant{\sc =gen} smooth{\sc =nmlz.sg} leaf    comp this{\sc =emph} {\sc cop}\\
\rede{This is what it means if we say: the high-yielding plant has smooth leaves.}\footnote{From the Nepali saying: \emph{hune biruwāko cillo pāt}, for people who show a promising behavior from an early age on.}\source{01\_leg\_07.236}



The complementizer is not only used to embed clauses, it may also link names to a head noun, translating as \rede{called} or \rede{so-called} (see \Next[a] and \Next[b]). As  \Next[c] shows, this structure does not necessarily need a nominal head either. The phrase marked by \emph{baŋna} has nominal properties, like a headless relative clause. It can, for instance, host the nonsingular marker \emph{=ci} and case markers. This shows again how complementation and nominalization are related in Yakkha (see Chapter \ref{ch-nmlz}).\footnote{Etymologically, the complementizer can be deconstructed into a root \emph{baŋ} (of unknown origin) and the nominalizers \emph{=na} and \emph{=ha}.}

\ex.\ag. eko selele-phelele baŋna nwak\\
one Selele-Phelele so-called bird\\
\rede{the bird called Selele-Phelele} \source{21\_nrr\_04.001}
\bg. haŋsewa       baŋha         yakkha=ci\\
Hangsewa {\sc comp} Yakkha\_person{\sc =nsg}\\
\rede{the Yakkha people called Hangsewa (a clan name)} \source{39\_nrr\_08.06}
\bg.   jalangaja, mendenbarik baŋna=be   \\
Jalan-Gaja Mendenbarik so-called{\sc =loc}\\
\rede{in a place called Jalan Gaja, Mendenbarik (in Malaysia)} \source{13\_cvs\_02.063} 


 	