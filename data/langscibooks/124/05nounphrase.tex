\chapter[The noun phrase]{The noun phrase}\label{ch:5}
\section{The structure of the common noun phrase}\label{sec:5.1}
\is{Noun phrase|(}
As discussed in \sectref{sec:3.2}, Rapa Nui has three types of nominal elements: common nouns\is{Noun!common}, personal nouns and locationals\is{Locational}. This chapter discusses the different elements occurring in the noun phrase. The largest part (sections \sectref{sec:5.1}–\ref{sec:5.12}) is devoted to the common noun phrase and its constituents. Proper noun phrases may only contain a subset of these constituents; these are discussed in \sectref{sec:5.13}. 

A prototypical\is{Prototype} common noun phrase consists of a noun, preceded by a determiner and possibly other elements, and possibly followed by adjectives and other elements. Within the noun phrase, a large number of different positions can be distinguished. Some of these contain a single word, others may contain a phrase. Each position may be empty, including – under certain circumstances – the nucleus. Some positions are mutually exclusive; for example, of the three possessive positions, only one (occasionally two) can be filled in a given noun phrase.

\tabref{tab:36} and \tabref{tab:37} represent the structure of the common noun phrase. 

\begin{table} 
\fittable{
\begin{tabular}{lp{1cm}p{1.3cm}p{1.3cm}p{1.8cm}p{1.5cm}lp{1.2cm}p{1.7cm}} 
\lsptoprule
& 0& 1& 2& 3& 4& 5& 6& 7\\
& prep. & quantif. phrase& coll. marker & determiner & quantif. phrase & poss. & plural mkr. & nucleus\\
\midrule
& \textit{e,\newline 
{\ꞌ}i,\newline
 mai} etc.& 
QtfP& 
\textit{kuā}\is{kua (collective)@kuā (collective)}& 
  \textit{te}; \newline
  \textit{t}{}-poss.\is{Pronoun!possessive!t-class}; \newline
  dem.det.; \newline
  NumP; \newline
  \textit{he}& 
QtfP&
 Ø-poss.\is{Pronoun!possessive!Ø-class}& 
\textit{ŋā}\is{nza (plural marker)@ŋā (plural marker)};
  \newline \textit{mau}& 
noun;\newline
  compound\is{Compound} noun\\
\midrule
 §&  & \ref{sec:4.4}& \ref{sec:5.2}& \ref{sec:5.3}; \ref{sec:4.6.2};\newline \ref{sec:4.6.4}; \ref{sec:6.2.1}& \ref{sec:4.4}; \ref{sec:5.4}& \ref{sec:6.2.1}& \ref{sec:5.5}& \ref{sec:5.6}; \ref{sec:5.7}\\
\lspbottomrule
\end{tabular} 
}
\caption{The common noun phrase: prenominal elements}
\label{tab:36}
\end{table}

\begin{table}
\small{
\fittable{
\begin{tabularx}{139mm}{p{2mm}p{9mm}p{10mm}p{11mm}p{8mm}p{8mm}p{7mm}p{10mm}p{10mm}p{10mm}p{10mm}} 
\lsptoprule
& 8& 9& 10& 11& 12& 13& 14& 15& 16& 17\\
& modif. & quantif. phrase & adverb & emph. mkr & limit. mkr & PND & ident. mkr & num. phrase & poss. & deictic part.\\
\midrule
& AdjP& QtfP& \textit{haka{\ꞌ}o}\textit{u}; \textit{tako{\ꞌ}a}& \textit{mau}\is{mau ‘really’}& \textit{nō}\is{no ‘just’@nō ‘just’}& \textit{nei}; \textit{ena}; \textit{era}& \textit{{\ꞌ}ā}; \textit{{\ꞌ}ana}\is{a (identity)@{\ꞌ}ā (identity)}& NumP& Ø\nobreakdash-poss\is{Pronoun!possessive!Ø-class}; poss. phrase& \textit{ai}\is{ai (deictic)}\\
\midrule
 §& \ref{sec:5.7}& \ref{sec:4.4}& \ref{sec:5.8}& \ref{sec:5.8}& \ref{sec:5.8}& \ref{sec:4.6.3}& \ref{sec:5.9}& \ref{sec:5.4}& \ref{sec:6.2.1}& \ref{sec:5.10}\\
\lspbottomrule
\end{tabularx}
}
}
\caption{The common noun phrase: postnominal elements}
\label{tab:37}
\end{table}

Apart from these, the noun phrase may contain the following elements:

\begin{itemize}
\item 
appositions\is{Apposition} (\sectref{sec:5.12})

\item 
relative clauses\is{Clause!relative} (\sectref{sec:11.4})

\item 
vocative\is{Vocative} particles (\sectref{sec:8.9})

\end{itemize}

Below are examples illustrating the different noun phrase positions. The constituents are numbered according to the numbering in the tables.

\ea\label{ex:5.1}
\gll e\textsubscript{0} kuā\textsubscript{2} tō{\ꞌ}oku\textsubscript{3} pāpā\textsubscript{7} era\textsubscript{13}\\
\textsc{ag} \textsc{coll} \textsc{poss.1sg.o} father \textsc{dist}\\

\glt 
‘My father and others (said)’ \textstyleExampleref{[R412.383]} 
\z

\ea\label{ex:5.2}
\gll mo\textsubscript{0} te\textsubscript{3} nu{\ꞌ}u\textsubscript{7} pa{\ꞌ}ari\textsubscript{8} ta{\ꞌ}ato{\ꞌ}a\textsubscript{9} mau\textsubscript{11} nei\textsubscript{13} {\ꞌ}ā\textsubscript{14}\\
for \textsc{art} people adult all really \textsc{prox} \textsc{ident}\\

\glt 
‘for all the older people here’ \textstyleExampleref{[R207.017]} 
\z

\ea\label{ex:5.3}
\gll ki\textsubscript{0} tū\textsubscript{3} taŋata\textsubscript{7} haka{\ꞌ}ou\textsubscript{10} era\textsubscript{13} \\
to \textsc{dem} man again \textsc{dist} \\

\glt 
‘(he said) to that other man’ \textstyleExampleref{[R102.020]} 
\z

\ea\label{ex:5.4}
\gll i\textsubscript{0} tā{\ꞌ}ana\textsubscript{3} {\ob}poki vahine\,{\cb}\textsubscript{7} mau\textsubscript{11} nō\textsubscript{12} {\ob}e tahi\,{\cb}\textsubscript{15}\\
\textsc{acc} \textsc{poss.3sg.a} {\db}child female really just {\db}\textsc{num} one\\

\glt 
‘(to look at) his really only daughter’ \textstyleExampleref{[Luke 8:41-42]}
\z

\ea\label{ex:5.5}
\gll i\textsubscript{0} te\textsubscript{3} pāpā\textsubscript{7} era\textsubscript{13} {\ob}o Pētero\,{\cb}\textsubscript{16} ai\textsubscript{17}\\
\textsc{acc} \textsc{art} father \textsc{dist} {\db}of Peter there\\

\glt 
‘(look at) Petero’s father there’ \textstyleExampleref{[Notes]}
\z

\ea\label{ex:5.6}
\gll rauhuru\textsubscript{1} te\textsubscript{3} me{\ꞌ}e\textsubscript{7} mātāmu{\ꞌ}a\textsubscript{8}\\
diverse \textsc{art} thing past\\

\glt 
‘(he saw) many things from the past’ \textstyleExampleref{[R423.021]}
\z

\ea\label{ex:5.7}
\gll {\ob}me{\ꞌ}e rahi nō atu\,{\cb}\textsubscript{4} rāua\textsubscript{5} ŋā\textsubscript{6} poki\textsubscript{7}  \\
{\ob}thing many just away \textsc{3pl} \textsc{pl} child  \\

\glt 
‘many children of theirs (were born)’ \textstyleExampleref{[R438.049]} 
\z

In \sectref{sec:5.2}–\ref{sec:5.12}, different elements in the common noun phrase will be discussed in turn. Some elements are discussed in other chapters: quantifiers\is{Quantifier} and demonstratives\is{Demonstrative} are discussed in Chapter 4, possessors in Chapter 6. See the references in \tabref{tab:36} and \tabref{tab:37}.

\section{The collective marker \textit{kuā}}\label{sec:5.2}
\is{kua (collective)@kuā (collective)|(}
\textit{Kuā}\is{kua (collective)@kuā (collective)} (etymology unknown; there is also a less common variant \textit{koā}\is{koa (collective)@koā (collective)}, which does not occur in older texts) indicates a human collectivity, a group of people belonging together. With a singular noun, as in examples \REF{ex:5.8} and \REF{ex:5.9} below, it has an associative sense \citep[50]{Dixon2012}: \textit{kuā} \textit{N} means ‘N and the ones around him/her, N and the others’. When the noun itself has plural reference, as in (\ref{ex:5.10}–\ref{ex:5.13}) below, \textit{kuā} refers to ‘the group of N’.

In older texts \textit{kuā} is rare and only occurs before proper names. Nowadays its use has increased in frequency and it occurs before the following elements:

\subparagraph{Proper names}

\ea\label{ex:5.8}
\gll Pē ira a \textbf{kuā} \textbf{Tiare} i iri ai ki {\ꞌ}uta e tahi mahana. \\
like \textsc{ana} \textsc{prop} \textsc{coll} Tiare \textsc{pfv} ascend \textsc{pvp} to inland \textsc{num} one day \\

\glt
‘Thus Tiare and the others went to the countryside one day.’ \textstyleExampleref{[R151.048]} 
\z

\subparagraph{Kinship terms} Examples are\is{Kinship term} \textit{koro} ‘Dad’, \textit{nua} ‘Mum’:

\ea\label{ex:5.9}
\gll He nonoho \textbf{a} \textbf{kuā} \textbf{koro} he kakai. \\
\textsc{ntr} \textsc{pl}:sit \textsc{prop} \textsc{coll} Dad \textsc{ntr} \textsc{pl}:eat \\

\glt
‘Dad and the others sat down and ate.’ \textstyleExampleref{[R333.538]} 
\z

\subparagraph{Other personal nouns} An example is \textit{māhaki} ‘that person’:

\ea\label{ex:5.10}
\gll E Tiare, e hāpa{\ꞌ}o rivariva koe i a \textbf{kuā} \textbf{māhaki}. \\
\textsc{voc} flower \textsc{exh} care\_for good:\textsc{red} \textsc{2sg} \textsc{acc} \textsc{prop} \textsc{coll} companion \\

\glt
‘Tiare, take good care of the little ones.’ \textstyleExampleref{[R496.015]} 
\z

\subparagraph{Pronouns}

\ea\label{ex:5.11}
\gll ¿O \textbf{kuā} \textbf{kōrua} i aŋa? \\
~of \textsc{coll} \textsc{2pl} \textsc{pfv} make \\

\glt
‘Did you (pl.) make it together?’ \textstyleExampleref{[R415.808]} 
\z

In short, \textit{kuā} occurs before the same words which can also be preceded by the proper article\is{a (proper article)} \textit{a}, i.e. proper nouns (\sectref{sec:3.3.2}). This may have led \citet[474]{DuFeu1987} to classify \textit{kuā} as a proper article\is{a (proper article)} as well. However, \textit{kuā} is different from the proper article\is{a (proper article)}. As examples \REF{ex:5.8}, \REF{ex:5.9} and \REF{ex:5.10} show, \textit{kuā} can co-occur with the proper article\is{a (proper article)}. This indicates that the two do not belong to the same class of particles, but occupy different slots in the noun phrase.

In fact, the syntactic behavior of \textit{kuā} shows it to be quite different from \textit{a}. \textit{Kuā} occurs in a number of contexts in which \textit{a} is impossible:

In the first place, even though \textit{kuā} usually occurs before proper nouns, it occasionally occurs before common nouns. \textit{Repa} ‘friend’ is a common noun which never takes the proper article\is{a (proper article)} \textit{a}, but it can be preceded by \textit{kuā}:

\ea\label{ex:5.12}
\gll Ka oho mai, e \textbf{kuā} \textbf{repa} ē. \\
\textsc{imp} go hither \textsc{voc} \textsc{coll} young\_man \textsc{voc} \\

\glt
‘Come, my friends.’ \textstyleExampleref{[R313.004]} 
\z

Secondly, unlike the proper article \textit{a}, \textit{kuā} occurs after the preposition \textit{o}, as in \REF{ex:5.11} above.

Thirdly, unlike the proper article, \textit{kuā} can be followed by a possessive pronoun:

\ea\label{ex:5.13}
\gll Ko \textbf{kuā} \textbf{tō{\ꞌ}oku} ŋā poki taina rikiriki era ko tō{\ꞌ}oku  pāpārū{\ꞌ}au era.\\
\textsc{prom} \textsc{coll} \textsc{poss.1sg.o} \textsc{pl} child sibling small:\textsc{pl}:\textsc{red} \textsc{dist} \textsc{prom} \textsc{poss.1sg.o}  grandfather \textsc{dist}\\

\glt
‘We were with my little brothers and my grandfather.’ \textstyleExampleref{[R123.014]} 
\z

In the fourth place, unlike the proper article, \textit{kuā} can be followed by the plural marker \textit{ŋā}\is{nza (plural marker)@ŋā (plural marker)}, as in \REF{ex:5.13} above.

Finally, \textit{kuā} occurs in the vocative\is{Vocative}, as in \REF{ex:5.12} above, something which is not possible with \textit{a}.

All of this shows that \textit{kuā} not only occurs in the proper noun phrase (\sectref{sec:5.13.1}), but also in the common noun phrase. The fact that \textit{kuā} mostly occurs before the same nouns as the proper article\is{a (proper article)}, may have semantic rather than syntactic reasons. The proper article\is{a (proper article)} \textit{a} precedes nouns which have a unique referent, and similarly, \textit{kuā} indicates reference to a group which is identified by a unique referent. This unique referent is either a central member of the group (\textit{kuā koro} ‘father and company’, \textit{kuā Tiare} ‘Tiare and the others’), or identified with the group as such (\textit{e kuā repa ē} ‘you friends’, \textit{kuā ŋā kope} ‘guys’).
\is{kua (collective)@kuā (collective)|)}
\section{Determiners}\label{sec:5.3}
\is{te (article)|(}\subsection{Introduction}\label{sec:5.3.1}
\is{Determiner|(}
The term \textsc{determiner} is used for a category which includes articles, as well as other elements which occur in the same position in the noun phrase, such as demonstratives and possessive pronouns \citep[161]{Dryer2007Noun}. This means that ‘determiner’ is a purely structural category; the function of elements in determiner positions may vary. 

\tabref{tab:38} lists the elements occurring in the determiner position in Rapa Nui. 

\begin{table}
\begin{tabularx}{\textwidth}{p{4mm}p{48mm}p{35mm}p{15mm}}
\lsptoprule
 & {element} & {§} & {referential}\\
\midrule
1. & the article \textit{te} & \ref{sec:5.3.3} & yes\\
2. & demonstrative determiners & \ref{sec:4.6.2}, \ref{sec:4.6.4} & yes\\
3. & {possessive pronouns\is{Pronoun!possessive} (\textit{t-}series)} & \ref{sec:4.2.2} (forms); \ref{sec:6.2.1} (use) & yes\\
4. & numeral phrases & \ref{sec:5.4.1} & –\\
5. & the predicate marker \textit{he} & \ref{sec:5.3.4} & no\\
\lspbottomrule
\end{tabularx}
\caption{Determiners}
%\todo[inline]{top-align}
\label{tab:38}
\end{table}

These elements are in complementary distribution; for example, the article \textit{te}, demonstrative determiners and possessive pronouns\is{Pronoun!possessive} of the \textit{t}{}-series never occur together.

There is a fundamental distinction between the predicate marker \textit{he} and the other determiners. Categories 1–3 introduce referential noun phrases; in fact, it can be argued that the determiner serves to make the noun phrase referential (see \sectref{sec:5.3.3}). Categories 1–3 will be referred to as \textit{t}{}-determiners, as most of these elements start with \textit{t-}. 

The predicate marker \textit{he}, on the other hand, marks non-referential noun phrases, such as nominal predicates. Nevertheless, \textit{he} is analysed as a determiner, because it occurs in the same structural position (\sectref{sec:5.3.4.2}). 

Category 4, numeral phrases, is neutral with respect to referentiality\is{Referentiality}. Numeral phrases in determiner position occur both in referential noun phrases and in certain types of non-referential noun phrases; in both cases, they provide the noun phrase with an obligatory determiner.

In the following subsections, first the distribution of \textit{t-}determiners will be examined (\sectref{sec:5.3.2}). \sectref{sec:5.3.3} then examines the function of the determiner \textit{te}, leading to the conclusion that \textit{te} marks referentiality\is{Referentiality}, not specificity or definiteness. \sectref{sec:5.3.4} discusses the use of the predicate marker \textit{he} and shows that, despite its different character compared to other determiners, it should be analysed as a determiner. Finally, \sectref{sec:5.3.5} shows that numeral phrases in determiner position occur in both referential and non-referential noun phrases, which shows that they are neutral with respect to referentiality\is{Referentiality}.

\subsection{The syntax of \textit{t}{}-determiners}\label{sec:5.3.2}

In this section I will show that the use of \textit{te} is largely determined by syntax. In many contexts, a \textit{t}{}-determiner is needed; if no other determiner is present, \textit{te} is used as default determiner. 

In the following sections the syntactic conditions are listed under which determiners are or are not used. First the conditions will be listed under which determiners are obligatory (\sectref{sec:5.3.2.1}), then contexts in which determiners are excluded (\sectref{sec:5.3.2.2}), and finally contexts in which the determiner is optional (\sectref{sec:5.3.2.3}). 

According to \citet{Chapin1974}, Rapa Nui is much more flexible than other Polynesian languages in the omission of the article, and the circumstances under which the article can be omitted are not completely clear. A close look reveals, however, that determiners can only be omitted in a limited set of specific contexts.

\subsubsection{Obligatory \textit{t}-determiners}\label{sec:5.3.2.1}

In Rapa Nui discourse, the vast majority of all noun phrases is preceded by a determiner. The most neutral determiner is the article \textit{te}; in fact, \textit{te} is the most common word overall in Rapa Nui discourse.

In many cases, determiners are obligatory. Two main rules can be formulated:\footnote{\label{fn:246}NB These rules only apply to common nouns\is{Noun!common}, which have a determiner position in the noun phrase. With personal nouns, the proper article\is{a (proper article)} \textit{a} is used, but not in exactly the same contexts as common noun determiners (\sectref{sec:5.13.2.1}).}

\begin{enumerate}
\item
Noun phrases having a core grammatical role in the clause must contain a \textit{t}-determiner or a numeral phrase in determiner position. 
\item 
Noun phrases preceded by a preposition (with the exception of \textit{pē} ‘like’ and \textit{hai} ‘with (instrumental)’) must immediately be followed by a \textit{t}-determiner. 
\end{enumerate}

Rule 1 is illustrated in the following examples, both with a subject noun phrase:

\ea\label{ex:5.14}
\gll He oho *(\textbf{te}) taŋata. \\
\textsc{ntr} go \textsc{~~pred} man \\

\glt
‘(The) people came.’ \textstyleExampleref{[R648.165]} 
\z

\ea\label{ex:5.15}
\gll He poreko \textbf{e} \textbf{toru} poki. \\
\textsc{ntr} born \textsc{num} three child \\

\glt
‘Three children were born.’ \textstyleExampleref{[R352.010]} 
\z

Rule 2 is true for all prepositions, including the Agent marker \textit{e} (which often marks subjects) and the accusative marker \textit{i}, but with the exception of \textit{pē}\is{pe ‘like’@pē ‘like’} ‘like’ and \textit{hai}\is{hai (instrumental prep.)} ‘with’ (\sectref{sec:4.7.1}). It is stricter than rule 1: the determiner must follow the preposition immediately and it must be a \textit{t}-determiner. This means that noun phrases preceded by a preposition cannot have a prenominal numeral, as the latter excludes a \textit{t}{}-determiner. In some cases a preposition is omitted to allow for a prenominal numeral. In the example below, the emphasised noun phrase is a direct object, which would normally be preceded by the accusative marker \textit{i}; in this case however, the accusative marker must be omitted, as the noun phrase does not have a \textit{t}-determiner.

\ea\label{ex:5.16}
\gll He aŋa (*\textbf{i}) \textbf{e} \textbf{tahi} \textbf{paepae}. \\
\textsc{ntr} make \textsc{acc} \textsc{num} one shack \\

\glt
‘He built a shelter.’ \textstyleExampleref{[Blx-2-1.015]}
\z

Most other prepositions cannot be omitted, however. After these prepositions, numerals in determiner position are excluded, so numerals must be placed after the noun. See \sectref{sec:5.4} for a fuller discussion and examples.

The rule that prepositions are immediately followed by a \textit{t}{}-determiner, also has consequences for noun phrases containing a quantifier. As \tabref{tab:30} on p.~\pageref{tab:30} shows, certain quantifiers may precede the determiner (\textit{ta{\ꞌ}ato{\ꞌ}a} \textit{te taŋata} ‘all the people’) or occur without determiner (\textit{ta{\ꞌ}ato{\ꞌ}a taŋata} ‘every man’). But when the noun phrase is preceded by a preposition, nothing may precede the determiner, so these constructions are excluded. The quantifier must either occur after the determiner (\textsc{prep} \textit{te ta}\textit{{\ꞌ}ato{\ꞌ}a taŋata}), or after the noun (\textsc{prep} \textit{te taŋata ta{\ꞌ}ato{\ꞌ}a}). 

The quantifier \textit{me{\ꞌ}e rahi} ‘many/much’ (\sectref{sec:4.4.7.1}) excludes the use of a determiner; this means it cannot occur at all in noun phrases preceded by a preposition. (In this case, speakers may resort to using the adjective \textit{rahi} ‘many/much’ as an alternative strategy, see \REF{ex:4.102} on p.~\pageref{ex:4.102}.)

Exceptions to rules 1 and 2 are only found in a few well-defined contexts, which are described in the next two subsections. Most of these exceptions only apply to rule 1; there are very few situations in which rule 2 is violated.

\subsubsection{Contexts in which determiners are excluded}\label{sec:5.3.2.2}

Determiners are excluded in the following contexts:

\begin{enumerate}
\setcounter{enumi}{0}
\item
When the noun is preceded by the quantifier \textit{me{\ꞌ}e rahi} ‘many/much’ (\sectref{sec:4.4.7.1}).\footnote{\label{fn:247}Alternatively, \textit{me}\textit{{\ꞌ}e rahi} itself could be analysed as a determiner.}  As discussed above, this means that \textit{me{\ꞌ}e rahi} cannot be preceded by a preposition.
\end{enumerate}

\ea\label{ex:5.17}
\gll \textbf{Me{\ꞌ}e} \textbf{rahi} \textbf{nu{\ꞌ}u} i māmate. \\
thing many/much people \textsc{pfv} \textsc{pl}:die \\

\glt
‘Many people died.’ \textstyleExampleref{[R532-05.002]}
\z

\begin{enumerate}
\setcounter{enumi}{1}
\item
When the noun is followed by \textit{aha}\is{aha ‘what’} ‘what’ or \textit{hē}\is{he (content question marker)@hē (content question marker)} ‘which’ (\sectref{sec:10.3.2.2}, \sectref{sec:10.3.2.3}), even when preceded by a preposition (an exception to rule 2 formulated above):
\end{enumerate}

\ea\label{ex:5.18}
\gll ¿{\ꞌ}I \textbf{kona} \textbf{hē} te māmoe nei? \\
~at place \textsc{cq} \textsc{art} sheep \textsc{prox} \\

\glt
‘Where is this sheep?’ \textstyleExampleref{[R536.037]} 
\z

\begin{enumerate}
\setcounter{enumi}{2}
\item
When the noun phrase is preceded by the instrumental preposition \textit{hai}\is{hai (instrumental prep.)} (\sectref{sec:4.7.9}):
\end{enumerate}

\ea\label{ex:5.19}
\gll He pu{\ꞌ}apu{\ꞌ}a \textbf{hai} \textbf{pāoa}; he mate. \\
\textsc{ntr} beat:\textsc{red} \textsc{ins} war\_club \textsc{ntr} die \\

\glt
‘They beat her with a club and she died.’ \textstyleExampleref{[Egt-01.082]}
\z

\begin{enumerate}
\setcounter{enumi}{3}
\item
In a few expressions in which the noun phrase is non-referential, even when preceded by a preposition\is{Referentiality} (another exception to rule 2). These expressions are typically repeated noun phrases with a distributive\is{Distributive} sense:
\end{enumerate}

\ea\label{ex:5.20}
\gll He e{\ꞌ}a rā matu{\ꞌ}a Iporito \textbf{ki} \textbf{hare} \textbf{era} \textbf{ki} \textbf{hare} \textbf{era}. \\
\textsc{ntr} go\_out \textsc{dist} parent Hippolyte to house \textsc{dist} to house \textsc{dist} \\

\glt 
‘Father Hippolyte went here and there (lit. to that house to that house).’ \textstyleExampleref{[R231.282]} 
\z

\ea\label{ex:5.21}
\gll He oho \textbf{hare} \textbf{{\ꞌ}iti} \textbf{hare} \textbf{nui} ki te taŋata. \\
\textsc{ntr} go house small house big to \textsc{art} man \\

\glt
‘He went to all the houses (lit. small house big house) to the people.’ \textstyleExampleref{[R368.056]} 
\z

\begin{enumerate}
\item[]
In these cases the noun phrase does not refer to any house in particular: \textit{hare era} in \REF{ex:5.20} does not refer to a certain house, but to houses in general. In other words, the noun phrase is non-referential\is{Referentiality}. If the article were used (\textit{ki te hare era}), the noun phrase would refer to a \is{Specific reference}specific house.
\end{enumerate}

\subsubsection{Contexts in which \textit{t}{}-determiners are optional}\label{sec:5.3.2.3}

In the following situations, \textit{t}{}-determiners are optional:

\begin{enumerate}
\setcounter{enumi}{0}
\item
In a somewhat informal style, the determiner can be left out in the second and following items of enumerations or lists:
\end{enumerate}

\ea\label{ex:5.22}
\gll Māuruuru ki a rāua {\ꞌ}e ki te nu{\ꞌ}u era hua{\ꞌ}ai, \textbf{matu{\ꞌ}a}, \textbf{nu{\ꞌ}u} pa{\ꞌ}ari...\\
thank fo \textsc{prop} \textsc{3pl} and to \textsc{art} people \textsc{dist} family parent people old\\

\glt 
‘Thanks to them, and to the family members, parents, old people...’ \textstyleExampleref{[R202.004]} 
\z

\begin{enumerate}
\setcounter{enumi}{1}
\item
Occasionally, the determiner is omitted when the noun phrase contains the plural marker \textit{ŋā}\is{nza (plural marker)@ŋā (plural marker)} (\sectref{sec:5.5.2}).

\item
Sometimes the determiner is omitted in sentence-initial noun phrases which contain a postnominal demonstrative\is{Demonstrative!postnominal} (\textit{nei}, \textit{ena} or \textit{era}). The noun phrase may be the subject of a verbal \REF{ex:5.23} or nominal clause\is{Clause!nominal} \REF{ex:5.24}, or a left-dislocated\is{Dislocation!left} constituent \REF{ex:5.25}:
\end{enumerate}

\ea\label{ex:5.23}
\gll \textbf{Nu{\ꞌ}u} \textbf{nei} ko hoki mai {\ꞌ}ā mai Tahiti ki Rapa Nui.  \\
people \textsc{prox} \textsc{prf} return hither \textsc{cont} from Tahiti to Rapa Nui  \\

\glt 
‘These people had returned from Tahiti to Rapa Nui.’ \textstyleExampleref{[R231.086]} 
\z

\ea\label{ex:5.24}
\gll \textbf{Kai} \textbf{ena} i a kōrua, kai rivariva. \\
food \textsc{med} at \textsc{prop} \textsc{2pl} food good:\textsc{red} \\

\glt 
‘That food you have is good food.’ \textstyleExampleref{[R310.262]} 
\z

\ea\label{ex:5.25}
\gll \textbf{Taŋata} \textbf{nei} ko Pāpu{\ꞌ}e, {\ꞌ}i Mā{\ꞌ}ea Makohe tō{\ꞌ}ona hare. \\
man \textsc{prox} \textsc{prom} Papu’e at Ma’ea Makohe \textsc{poss.3sg.o} house \\

\glt
‘This man Papu’e, his house was in Ma’ea Makohe.’ \textstyleExampleref{[R372.035]} 
\z

\begin{enumerate}
\setcounter{enumi}{3}
\item[]
In these cases, the absence of the article makes no difference in meaning; the omission is a purely stylistic matter, and limited to a somewhat informal style.\footnote{\label{fn:248}Note that the same happens in verb phrases: aspect markers are occasionally left out at the beginning of a sentence, but only if the verb phrase has a postverbal particle (\sectref{sec:7.2.2}).}

\item
The determiner is sometimes omitted with the quantifiers\is{Quantifier} \textit{\mbox{ta{\ꞌ}ato{\ꞌ}a}} ‘all’ (\sectref{sec:4.4.2}) and \textit{rauhuru} ‘diverse’ (\sectref{sec:4.4.5}), as well as the old quantifier \textit{ana\-nake} ‘all’ (\sectref{sec:4.4.4.2}).

\item 
The determiner can be left out after \textit{koia ko}\is{koia ko ‘with’} ‘with’, which indicates attendant circumstances (\sectref{sec:8.10.4.2}).
\end{enumerate}

\subsection{The function of the article \textit{te}}\label{sec:5.3.3}

The article \textit{te} is widespread in Polynesian languages.\footnote{\label{fn:249}In fact, cognates of \textit{te} occur in all Polynesian languages, though in some language \is{Proto-Polynesian}PPN \textit{*te} underwent an irregular change (e.g. \ili{Tongan} \textit{he}, \ili{Samoan} \textit{le}). Interestingly, \textit{te} as a definite or specific article is not reconstructed for any protolanguage prior to \is{Proto-Polynesian}PPN; however, \citet{Clark2015} shows that possible cognates occur in various Oceanic languages, mostly as an indefinite article. If these are indeed cognates, this article extended its use to definite NPs in PPN.} Older descriptions characterise it as a definite\is{Definiteness} article, while \textit{he} is described as an indefinite article. According to \citet[11]{DuFeu1996}, \textit{te} in Rapa Nui is a [+specific] \is{Specific reference}article, while \textit{he} is [±specific]. However, in actual fact \textit{he} and \textit{te} are not two articles which can be substituted for each other. They occur in quite different syntactic contexts. \textit{He} mainly introduces noun phrases which serve as predicates of non-verbal clauses (this will be discussed in \sectref{sec:5.3.4} below). It does not occur, for example, in noun phrases serving as subject or object of a verbal clause:

\ea\label{ex:5.26}
\gll *He oho \textbf{he} taŋata ki te hare. \\
\textsc{~~ntr} go \textsc{pred} man to \textsc{art} house \\

\glt 
‘A man went home.’
\z

\ea\label{ex:5.27}
\gll *Ko tike{\ꞌ}a {\ꞌ}ā a au (i) \textbf{he} honu. \\
\textsc{~~prf} see \textsc{cont} \textsc{prop} \textsc{1sg} \textsc{acc} \textsc{pred} turtle \\

\glt
‘I have seen a turtle.’
\z

This means that \textit{te} is the only full-fledged article in Rapa Nui. As indicated in \sectref{sec:5.3.1}, it is in complementary distribution with the other \textit{t}{}-determiners: demonstratives\is{Demonstrative} and possessive pronouns\is{Pronoun!possessive} of the \textit{t-}series. 

The article \textit{te} occurs with all common nouns\is{Noun!common}, that is, all nouns which do not take the proper article\is{a (proper article)} \textit{a} (\sectref{sec:5.13.2}). It is not specified for gender or case. Neither is it specified for number:\footnote{\label{fn:250}This is different from the situation in some other \is{Eastern Polynesian}EP languages, where the plural marker is in determiner position and in complementary distribution with \textit{te}; see Footnote \ref{fn:261} on p.~\pageref{fn:261}.} both singular and plural nouns are introduced by \textit{te}. Number is indicated by the plural marker \textit{ŋā}\is{nza (plural marker)@ŋā (plural marker)}, by numerals or understood from the context.

\textit{Te} can be used with count nouns as in \REF{ex:5.28}, mass nouns as in \REF{ex:5.29}, and abstract concepts as in \REF{ex:5.30}:

\ea\label{ex:5.28}
\gll He tu{\ꞌ}u mai \textbf{te} taŋata, \textbf{te} vi{\ꞌ}e, he popo mai ki roto ki \textbf{te} hare.\\
\textsc{ntr} arrive hither \textsc{art} man \textsc{art} woman \textsc{ntr} pack hither to inside to \textsc{art} house\\

\glt 
‘Men and women arrived and crowded into the house.’ \textstyleExampleref{[Ley-5-34.009]}
\z

\ea\label{ex:5.29}
\gll Ko mate atu {\ꞌ}ana ki \textbf{te} \textbf{vai} mo unu.\\
\textsc{prf} die away \textsc{cont} to \textsc{art} water for drink\\

\glt 
‘I’m dying for water to drink.’ \textstyleExampleref{[R303.032]} 
\z

\ea\label{ex:5.30}
\gll \textbf{Te} \textbf{haŋa} rahi pa he manu era he paloma...\\
\textsc{art} love much like \textsc{pred} bird \textsc{dist} \textsc{pred} dove\\

\glt
‘Great love is like a dove...’ \textstyleExampleref{[R222.036–037]}
\z

Is \textit{te} a definite article, as older descriptions suggest? \citet{Lyons1999} defines definiteness\is{Definiteness} in terms of \textsc{identifiability}\is{Identifiability}: the definite article signals that the hearer is in a position to identify the referent of a noun phrase. When a speaker says ‘Pass me the hammer’, the hearer infers that there is a single hammer that he/she is able to identify.\footnote{\label{fn:251}As the notion of identifiability\is{Identifiability} is not without problems, \citet{Lyons1999} also uses the notion of \textsc{inclusiveness}: the definite article signals that there is only one entity satisfying the description used, relative to the context. Thus in ‘There was a wedding. The bride was radiant,’ the hearer cannot identify the bride (he does not know who she is), and yet ‘the bride’ is definite: the article indicates that in the given situation there is only one bride.}

In this sense, \textit{te} cannot be considered a definite article. In many cases, \textit{te} introduces noun phrases with indefinite reference.

\ea\label{ex:5.31}
\gll Ko tu{\ꞌ}u {\ꞌ}ana a au ki ruŋa i \textbf{te} \textbf{henua} e hitu. \\
\textsc{prf} arrive \textsc{cont} \textsc{prop} \textsc{1sg} to above at \textsc{art} land \textsc{num} seven \\

\glt
‘(In my dream) I arrived on seven islands.’ \textstyleExampleref{[R420.014]} 
\z

Even when not definite, \textit{te} usually refers to a \is{Specific reference}\textsc{specific} entity. Thus in the following example, \textit{te taŋata e tahi} refers to a specific man; his name is mentioned straight afterwards.

\ea\label{ex:5.32}
\gll {\ꞌ}I te noho iŋa tuai era {\ꞌ}ā \textbf{te} \textbf{taŋata} \textbf{e} \textbf{tahi} te {\ꞌ}īŋoa  ko Tu{\ꞌ}uhakararo.\\
at \textsc{art} stay \textsc{nmlz} ancient \textsc{dist} \textsc{ident} \textsc{art} man \textsc{num} one \textsc{art} name  \textsc{prom} Tu’uhakararo\\

\glt
‘In the old times (there was) a man called Tu’uhakararo.’ \textstyleExampleref{[R477.002]} 
\z

However, \textit{te} can also be used in non-specific\is{Specific reference!non-specific reference} contexts. This is for example the case in general statements, in which the noun phrases have generic reference:

\ea\label{ex:5.33}
\gll E tano nō mo ma{\ꞌ}u i \textbf{te} \textbf{mōai} e ho{\ꞌ}e {\ꞌ}ahuru toneladas ...  e \textbf{te} \textbf{taŋata} e ho{\ꞌ}e hānere va{\ꞌ}u {\ꞌ}ahuru.\\
\textsc{ipfv} correct just for carry \textsc{acc} \textsc{art} statue \textsc{num} one ten tons ~   \textsc{ag} \textsc{art} man \textsc{num} one hundred eight ten\\

\glt
‘It is possible to transport a statue of ten tons... by one hundred and eighty men.’ \textstyleExampleref{[R376.062]} 
\z

This sentence does not refer to any specific situation involving a specific statue and specific people, but to statues and people in general. 

A noun phrase is also non-specific\is{Specific reference!non-specific reference} when its referent is hypothetical. This happens for example when the item is desired or sought as in (\ref{ex:5.34}–\ref{ex:5.35}), denied as in \REF{ex:5.36}, or its existence is questioned as in \REF{ex:5.37}. In all these examples, the referent has not been mentioned in the preceding context, but even so, \textit{te} is used:

\ea\label{ex:5.34}
\gll ...mo ai o te moni mo ho{\ꞌ}o mai i \textbf{te} \textbf{haraoa}. \\
~~~for exist of \textsc{art} money for trade hither \textsc{acc} \textsc{art} bread \\

\glt 
‘(He sells food) in order to have money to buy bread.’ \textstyleExampleref{[R156.023]} 
\z

\ea\label{ex:5.35}
\gll He kī ō{\ꞌ}oku ki tō{\ꞌ}oku ŋā poki taina era mo oho o mātou mo kimi i \textbf{te} \textbf{pipi}.\\
\textsc{ntr} say \textsc{poss.1sg.o} to \textsc{poss.1sg.o} \textsc{pl} child sibling \textsc{dist} for go of \textsc{1pl.excl} for search \textsc{acc} \textsc{art} shell\\

\glt 
‘I told my brothers and sisters that we would go to look for shells.’ \textstyleExampleref{[R125.002]} 
\z

\ea\label{ex:5.36}
\gll {\ꞌ}Ina ko kai i \textbf{te} \textbf{kai} mata. \\
\textsc{neg} \textsc{neg.ipfv} eat \textsc{acc} \textsc{art} food raw \\

\glt 
‘Don’t eat raw food.’ \textstyleExampleref{(\citealt[61]{WeberR2003})} 
\z

\ea\label{ex:5.37}
\gll —¿E ai rō {\ꞌ}ā \textbf{te} \textbf{ika} o roto? —{\ꞌ}Ina e tahi. \\
~~~~\textsc{ipfv} exist \textsc{emph} \textsc{cont} \textsc{art} fish of inside ~~~\textsc{neg~} \textsc{num} one \\

\glt
‘—Are there any fish inside? —Not one.’ \textstyleExampleref{[R241.058]} 
\z

We may conclude that the article \textit{te} in Rapa Nui does not indicate definiteness\is{Definiteness} or specificity\is{Specific reference}.\footnote{\label{fn:252}One could wonder whether an element not encoding definiteness\is{Definiteness} or specificity\is{Specific reference} still qualifies as an article. \citet[157]{Dryer2007Noun} answers this question in the affirmative.} Rapa Nui does have other devices to indicate these:

\begin{itemize}
\item 
The article in combination with a postnominal demonstrative\is{Demonstrative!postnominal} indicates definiteness\is{Definiteness} (\sectref{sec:4.6.3.1}). 

\item 
To indicate a specific number, numerals are used. The numeral\is{tahi ‘one’} \textit{e tahi} ‘one’ may function almost as the equivalent of an indefinite article (\sectref{sec:5.4.3}).

\end{itemize}

This raises the question whether \textit{te} has any meaning at all. Its role seems to be purely syntactic as a default determiner: whenever a determiner is needed and the noun phrase has no other determiner, \textit{te} is used. However, this begs the question why the syntax requires a determiner at all in the contexts discussed in \sectref{sec:5.3.2}. To recapitulate: \textit{te} (or another \textit{t}-determiner) is generally required in core grammatical roles and after prepositions. On the other hand, \textit{t}-determiners do not occur when the noun phrase serves as a predicate; in that case \textit{he} is used (\sectref{sec:5.3.4}. This suggests that \textit{te} does have a semantic function: the article \textit{te} (and other \textit{t}-determiners) signals \textsc{referentiality}; it turns a common noun into a referential expression. 

A common noun as such is not \is{Referentiality}referential; in itself it does not refer to an entity, but denotes a certain property which defines a class of entities. A determiner is needed to create an expression which refers to one or more entities belonging to this class, and only on this condition can the noun be used as a subject or object, or as complement of a preposition. The noun phrase may refer to a specific entity (whether known to the hearer or not, i.e. definite or indefinite) or to some unspecified one: referentiality\is{Referentiality} is not the same as specificity\is{Specific reference}. 

A noun phrase in argument position or after a preposition is referential, so it needs a determiner. Any determiner will do: a demonstrative, a possessive pronoun, or – by default – the article \textit{te}.\footnote{\label{fn:253}\citet[467]{RigoVernaudon2004} apply the same analysis to the \ili{Tahitian} article \textit{te}. They refer to \citet{Lemaréchal1989}, who analyses such elements as translating a “nom” into a “substantif”. A “nom” expresses a quality (e.g. ‘doctor’ = the quality of being a doctor), while a determiner converts this into a referring expression (a person who has the quality of being a doctor). \citet{GorrieKellner2010} give a partly similar analysis for determiners in \ili{Niuean}: determiners are the obligatory elements which allow a noun to function as an argument. They separate this function from referentiality\is{Referentiality}, which in their analysis is provided by other noun phrase elements.} On the other hand, in order for a noun to function as predicate, \textit{t}{}-determiners are excluded: the predicate should not be referential, but denote a property.

This analysis explains why determiners are used with noun phrases in argument positions and after prepositions, while \textit{he} is used with predicate noun phrases. It is further confirmed by a number of other phenomena.

In the first place, as discussed in the section on quantifiers\is{Quantifier}, the article is often omitted with a prenominal quantifier\is{Quantifier} (\sectref{sec:4.4.2} on \textit{ta}\textit{{\ꞌ}ato{\ꞌ}a}\is{taatoa ‘all’@ta{\ꞌ}ato{\ꞌ}a ‘all’} ‘all’; \sectref{sec:4.4.5} on \textit{rauhuru}\is{rauhuru ‘diverse’} ‘diverse’; \sectref{sec:4.4.7.1} on \textit{me}\textit{{\ꞌ}e rahi}\is{mee rahi ‘many’@me{\ꞌ}e rahi ‘many’} ‘many’). While referential noun phrases presume the existence of an entity, noun phrases with the universal quantifier\is{Quantifier} ‘all’ do not; in other words, they can be considered non-referential. 

It is thus not surprising that with the universal quantifier\is{Quantifier}, the article can be left out. Now this does not explain yet why the article can also be left out with \textit{rauhuru} ‘diverse’ and \textit{me{\ꞌ}e rahi} ‘many’: unlike the universal quantifier\is{Quantifier}, these do imply the existence of a set of entities. However, they do not single out a definite number of individuals from a set; rather, they denote an unspecified subset from the total set of entities denoted by the noun. ‘Many people went’ implies that there exists a subset from the class of ‘people’ for whom the predicate ‘went’ is true, but without being specific about the extent of this subset. While these expressions are not strictly non-referential, they appear to be lower on the referentiality\is{Referentiality} scale than expressions referring to individuated entities. These quantifiers\is{Quantifier} are somewhat similar to distributional expressions (\sectref{sec:5.3.2.2}), which likewise exclude the article \textit{te}:

\ea\label{ex:5.38}
\gll He e{\ꞌ}a rā matu{\ꞌ}a Iporito \textbf{ki} \textbf{hare} \textbf{era} \textbf{ki} \textbf{hare} \textbf{era}. \\
\textsc{ntr} go\_out \textsc{dist} parent Hippolytus to house \textsc{dist} to house \textsc{dist} \\

\glt 
‘Father Hippolytus went here and there (lit. to that house to that house).’ \textstyleExampleref{[R231.282]} 
\z

Secondly, as will be discussed below (\sectref{sec:5.3.4}), \textit{t}{}-determiners are excluded – and the predicate marker \textit{he}\is{he (nominal predicate marker)} is used – not only with nominal predicates, but with other non-referential noun phrases as well: 

\begin{itemize}
\item 
noun phrases in apposition\is{Apposition}. Noun phrases in apposition\is{Apposition} do not refer to an entity or set of entities, but depend for their reference on the preceding head noun. The function of the apposition\is{Apposition} is to specify a further property of this head noun; they are more like predicates than referential expressions.

\item 
noun phrases after the comparative preposition \textit{pē}\is{pe ‘like’@pē ‘like’} ‘like’ (\sectref{sec:4.7.8}). Interestingly, the same constraint applies to the preposition \textit{me} ‘like’, which occurs in \ili{Marquesan} (\citealt[136–137]{Cablitz2006}), \ili{Hawaiian} (\citealt[53]{Cook1999}; \citealt[156]{ElbertPukui1979}), \ili{Māori} (\citealt[237]{Polinsky1992}; \citealt[356]{Bauer1993}) and \ili{Tuvaluan} \citep[224]{Besnier2000}. In all these languages, \textit{me} must be followed by the predicate marker \textit{se} or \textit{he}. The use of \textit{he}\is{he (nominal predicate marker)} after prepositions meaning ‘like’ can be explained from the non-referential character of the noun phrase after these prepositions: the noun phrase does not refer to any concrete entity, but involves a comparison with a class of entities; hence the predicate marker \textit{he} is appropriate rather than referential \textit{te} (cf. \citealt[236]{Polinsky1992}; cf. the discussion of \textit{he} in \sectref{sec:5.3.4}). 

\end{itemize}

Thirdly, noun phrases in interrogative\is{Question} and negative\is{Negation} sentences are less referential than those in positive declarative sentences: in both cases, the noun phrase does not refer to an entity whose existence is presupposed.\footnote{\label{fn:254}\citet[437]{ChungMason1995} explain the use of \textit{he} in \ili{Māori} in (among others) interrogative and negative constructions precisely from the \is{Referentiality}non-referential character of the noun phrase in these contexts.} Now the use of \textit{t-}determiners is not excluded in interrogative and negative contexts per se. Two examples from the previous section are repeated here:

\ea\label{ex:5.39}
\gll {\ꞌ}Ina ko kai i \textbf{te} \textbf{kai} mata. \\
\textsc{neg} \textsc{neg.ipfv} eat \textsc{acc} \textsc{art} food raw \\

\glt 
‘Don’t eat raw food.’ \textstyleExampleref{(\citealt[610]{WeberR2003})} 
\z

\ea\label{ex:5.40}
\gll —¿E ai rō {\ꞌ}ā \textbf{te} \textbf{ika} o roto? —{\ꞌ}Ina e tahi. \\
~~~~~\textsc{ipfv} exist \textsc{emph} \textsc{cont} \textsc{art} fish of inside ~~~~\textsc{neg~~} \textsc{num} one \\

\glt
‘—Are there any fish inside? —Not one.’ \textstyleExampleref{[R241.058]} 
\z

On the other hand, there is one negative and one interrogative construction in which \textit{t-}determiners are excluded:

\begin{itemize}
\item 
when the noun itself is questioned by the interrogative adjective \textit{hē} ‘which’: see example \REF{ex:5.18} in \sectref{sec:5.3.2.2}.

\item 
when the noun phrase occurs immediately after the negator \textit{{\ꞌ}ina} (\sectref{sec:10.5.1}, esp. (\ref{ex:10.94}–\ref{ex:10.96})):\footnote{Notice that the use of \textit{he}\is{he (nominal predicate marker)} rather than \textit{te} in this example cannot be explained as an existential construction. This sentence is not a negation of ‘there was an old woman who went down to Hanga Roa’, but refers to a definite woman, as the demonstrative \textit{nei} indicates. Even so, the negation triggers the use of the predicate marker instead of the referential article.}

\end{itemize}

\ea\label{ex:5.41}
\gll \textbf{{\ꞌ}Ina} \textbf{he} rū{\ꞌ}au nei he turu mai ki Haŋa Roa. \\
\textsc{neg} \textsc{pred} old\_woman \textsc{prox} \textsc{ntr} go\_down hither to Hanga Roa \\

\glt
‘This old woman did not go down to Hanga Roa.’ \textstyleExampleref{[R380.006]} 
\z

Recapitulating: \textit{t-}determiners are excluded (and \textit{he} used instead) in non-referential noun phrases: nominal predicates, appositions, after \textit{p}\textit{ē} ‘like’. They are also commonly omitted (though not excluded) with the universal quantifier \textit{\mbox{ta{\ꞌ}ato{\ꞌ}a}}. Finally, \textit{t-}determin\-ers are excluded in a number of contexts which are low on the referentiality scale (though not strictly non-referential): with \textit{me{\ꞌ}e rahi} ‘many’, in questioned noun phrases, and after the negation \textit{{\ꞌ}ina}. 
\is{te (article)|)}
\subsection{The predicate marker \textit{he}}\label{sec:5.3.4}
\is{he (nominal predicate marker)|(}\subsubsection[Uses of he]{Uses of \textit{he}}\label{sec:5.3.4.1}

The determiner \textit{he} reflects \is{Proto-Polynesian}PPN \textit{*sa} ({\textgreater} PNP \textit{*se}); its cognates occur in most Polynesian languages. In the past these have often been analysed as indefinite articles (see references in \citealt[230]{Polinsky1992}). For Rapa Nui, \citet[18]{Englert1978} already realised that \textit{he} is something different from an indefinite article: \textit{he} ‘se emplea cuando se trata de denominaciones generales de personas u objetos’ (is used when general designations of persons and objects are concerned). 

The \is{Proto-Polynesian}Proto-Polynesian ancestors of \textit{te} and \textit{he} did function as definite and indefinite (or \is{Specific reference}specific and non-specific\is{Specific reference!non-specific reference}) article respectively (see \citealt[47–50]{Clark1976}; \citealt[411]{Hamp1977}). In Samoic and Tongic languages, \textit{he} continued to function as an indefinite article: it is  commonly used to introduce referential noun phrases functioning as verb arguments.\footnote{\label{fn:255}See e.g. \citet[261–264]{MoselHovdhaugen1992} on \ili{Samoan}, \citet[365]{Besnier2000} on \ili{Tuvaluan}, and \citet[22]{AndersonOtsuka2006} on \ili{Tongan}.} In \is{Eastern Polynesian}Eastern Polynesian languages, however, \textit{he} mainly functions as nominal predicate marker, though in some languages it is occasionally used to mark argument noun phrases\footnote{\label{fn:256}In \ili{Māori} (\citealt{Polinsky1992}; \citealt{ChungMason1995}) and \ili{Hawaiian} \citep{Cook1999}, \textit{*he} occasionally marks objects or non-agentive subjects. In Rapa Nui its use with argument noun phrases is very marginal, though not limited to non-agentives: it may mark topicalised\is{Topicalisation} noun phrases, regardless the nature of the verb (\sectref{sec:8.6.2.2}). \citet{Clark1997} provides a reconstruction of the shifts in the use of \textit{he} in \is{Eastern Polynesian}PEP.} (see \sectref{sec:5.3.3} on referentiality\is{Referentiality}). 

As explained in the previous section, Rapa Nui \textit{he} occurs in non-referential noun phrases and is excluded in referential noun phrases (with a single exception, see Footnote \ref{fn:256} above). It is mainly used to mark noun phrases as \textsc{predicates} of a verbless clause. In the following example, \textit{he taŋata} is the predicate of the clause: ‘man’ is predicated of the subject \textit{tau manu era}. The clause is \textsc{classifying}\is{Clause!classifying} (\sectref{sec:9.2.1}): it expresses that the subject belongs to the class of human beings. \textit{Taŋata} does not refer to any man in particular, nor to a group of men or even to men in general; rather, it denotes the property of ‘being man’.

\ea\label{ex:5.42}
\gll \textbf{He} \textbf{taŋata} tau manu era. \\
\textsc{pred} man \textsc{dem} bird \textsc{dist} \\

\glt
‘That bird was a human being.’ \textstyleExampleref{[Mtx-7-12.069]}
\z

Besides classifying clauses, \textit{he} is also used in \textsc{existential} clauses\is{Clause!existential} (\sectref{sec:9.3.1}):

\ea\label{ex:5.43}
\gll He taŋata ko Eŋo. \\
\textsc{pred} man \textsc{prom} Engo \\

\glt
‘There was a man (called) Engo.’ \textstyleExampleref{[Mtx-7-28.001]}
\z

As a nominal predicate marker, \textit{he} also marks the complement of the \textsc{copula verbs}\is{Verb!copula} \textit{riro} ‘become’ and \textit{ai} ‘be’ (\sectref{sec:9.6}).

Apart from marking the predicate of a verbless clause, \textit{he} has the following other uses:

\subparagraph{In appositions} \textit{He} optionally occurs before common nouns in appositions\is{Apposition} (\sectref{sec:5.12}):

\ea\label{ex:5.44}
\gll He kī e te matu{\ꞌ}a tane era o Te Rau, \textbf{he} \textbf{taŋata} pū{\ꞌ}oko  o te nu{\ꞌ}u o Kapiti...\\
\textsc{ntr} say \textsc{ag} \textsc{art} parent male \textsc{dist} of Te Rau \textsc{pred} man head  of \textsc{art} people of Kapiti\\

\glt 
‘The father of Te Rau, the leader of the people of Kapiti, said...’ \textstyleExampleref{[R347.089]} 
\z

\subparagraph{In isolation} \textit{He} is used before common nouns\is{Noun!common} in isolation (i.e. without a semantic role in the clause), for example in titles:

\ea\label{ex:5.45}
\gll \textbf{He} \textbf{aŋa} vaka, {\ꞌ}e \textbf{he} \textbf{e{\ꞌ}a} \textbf{iŋa} ki haho i te tai \\
\textsc{pred} make boat and \textsc{pred} go\_out \textsc{nmlz} to outside at \textsc{art} sea \\

\glt 
‘Building boats, and going out to sea’ \textstyleExampleref{[R200 title]}
\z

\ea\label{ex:5.46}
\gll \textbf{He} \textbf{tiare} ko au \textbf{he} \textbf{raŋi} \textbf{he} \textbf{hetu{\ꞌ}u} \\
\textsc{pred} flower \textsc{prom} \textsc{1sg} \textsc{pred} sky \textsc{pred} star \\

\glt 
‘The flowers, me, the sky and the stars’ \textstyleExampleref{[R222 title]}
\z

\subparagraph{In lists} Noun phrases in enumerations or lists may also be syntactically isolated, in which case they are also marked by \textit{he}:

\ea\label{ex:5.47}
\gll Te aŋa nō {\ꞌ}a Reŋa he tunu i te kai: \textbf{he} moa, \textbf{he} tarake,  \textbf{he} kūmā, ika {\ꞌ}e tētahi atu.\\
\textsc{art} work just of\textsc{.a} Renga \textsc{pred} cook \textsc{acc} \textsc{art} food \textsc{pred} chicken \textsc{pred} corn  \textsc{pred} sweet\_potato fish and other away\\

\glt
‘What Renga used to do was cooking food: chicken, corn, sweet potato, fish and other things.’ \textstyleExampleref{[R363.119]} 
\z

(Proper nouns\is{Noun!proper} and pronouns in isolation and in lists are marked with \textit{ko}; common nouns\is{Noun!common} are marked with \textit{ko} when uniquely identifiable\is{Identifiability}; see \sectref{sec:4.7.11.1}.)

\subparagraph{After \textit{pē} ‘like’} After the preposition \textit{pē}\is{pe ‘like’@pē ‘like’} ‘like’ (\sectref{sec:4.7.8}), \textit{he} is obligatory.

\subparagraph{After the negator \textit{{\ꞌ}ina}} \textit{He} is used in noun phrases immediately following the negator \textit{{\ꞌ}ina}\is{ina (negator)@{\ꞌ}ina (negator)}, whether the noun phrase is referential or not (\sectref{sec:10.5.1}):

\ea\label{ex:5.48}
\gll \textbf{{\ꞌ}Ina} \textbf{he} \textbf{rū{\ꞌ}au} \textbf{nei} he turu mai ki Haŋa Roa. \\
\textsc{neg} \textsc{pred} old\_woman \textsc{prox} \textsc{ntr} go\_down hither to Hanga Roa \\

\glt 
‘This old woman did not go down to Hanga Roa.’ \textstyleExampleref{[R380.006]} 
\z

\subparagraph{In topicalisation} \textit{He} occasionally marks topicalised\is{Topicalisation} subjects in a verbal clause\\ (\sectref{sec:8.6.2.2}):

\ea\label{ex:5.49}
\gll \textbf{He} \textbf{taŋata} he oho he ruku i te ika mo te hora kai. \\
\textsc{pred} man \textsc{ntr} go \textsc{ntr} dive \textsc{acc} \textsc{art} fish for \textsc{art} time eat \\

\glt 
‘The men went diving for fish for lunch.’ \textstyleExampleref{[R183.019]} 
\z

\subsubsection{\textit{He} is a determiner}\label{sec:5.3.4.2}

The discussion so far has shown that the distribution of \textit{he} is quite different from that of \textit{t}{}-determiners: it usually does not mark verbal arguments, does not occur after most prepositions, but instead is mainly used when the noun phrase functions as predicate or is in another non-argument position. Even so, \textit{he} is most plausibly analysed as a determiner, i.e. an element occurring in the same structural position as the \textit{t-}determiners. There are different reasons for doing so.\footnote{\label{fn:257}\citet{Cook1999} proposes the same analysis for \ili{Hawaiian} \textit{he}, based on the fact that it can be preceded by the preposition \textit{me} ‘like’, cannot be followed by another determiner, and does not occur before pronouns.} 

\begin{enumerate}
\item
\textit{He} excludes other determiners. \textit{He} and \textit{te}\is{te (article)} never occur together, and neither can \textit{he} co-occur with a \textit{t-}possessive\is{Pronoun!possessive!t-class} pronoun; if a \textit{he}{}-marked noun phrase has a possessive pronoun, the latter must be postnominal:
\end{enumerate}

\ea\label{ex:5.50}
\gll Te me{\ꞌ}e nei \textbf{he} \textbf{toto} \textbf{ō{\ꞌ}oku}. \\
\textsc{art} thing \textsc{prox} \textsc{pred} blood \textsc{poss.1sg.o} \\

\glt 
‘This is my blood.’ \textstyleExampleref{[Luke 22:20]}
\z

\ea\label{ex:5.51}
\textit{*Te me{\ꞌ}e nei \textbf{he} \textbf{ō{\ꞌ}oku}\textbf{/tō{\ꞌ}oku} toto.}
\z

\begin{enumerate}
\item[]
Likewise, \textit{he} it is excluded when the noun phrase contains a prenominal numeral. Here is an example with a noun phrase following the negation \textit{{\ꞌ}ina}, a context in which normally \textit{he} would be used (see \ref{ex:5.48} above):
\end{enumerate}

\ea\label{ex:5.52}
\gll {\ꞌ}Ina e tahi me{\ꞌ}e o mātou mo kai.\\
\textsc{neg} \textsc{num} one thing of \textsc{1pl.excl} for eat\\

\glt 
‘We didn’t have anything to eat (lit. there was not one thing of ours to eat).’ \textstyleExampleref{[R130.002]}  
\z

\begin{enumerate}
\setcounter{enumi}{1}
\item
Although \textit{he} is precluded after almost all prepositions, there is one exception: \textit{he} does occur – and is even obligatory – after the preposition \textit{pē}\is{pe ‘like’@pē ‘like’} ‘like’ (\sectref{sec:4.7.8}), as the following little riddle shows:
\end{enumerate}

\ea\label{ex:5.53}
\gll {\ꞌ}Iti{\ꞌ}iti \textbf{pē} \textbf{he} kio{\ꞌ}e, hāpa{\ꞌ}o i te hare \textbf{pē} \textbf{he} paiheŋa haka {\ꞌ}āriŋa. \\
small:\textsc{red} like \textsc{pred} rat care\_for \textsc{acc} \textsc{art} house like \textsc{pred} dog \textsc{caus} face\\

\glt 
‘Small like a mouse, guarding the house like an insolent dog.’ \textstyleExampleref{[R144.007]} 
\z

\begin{enumerate}
\setcounter{enumi}{2}
\item
Just like \textit{t-}determiners, \textit{he} is placed before quantifiers\is{Quantifier} like \textit{\mbox{ta{\ꞌ}ato{\ꞌ}a}}, \textit{rauhuru} and \textit{tētahi}:
\end{enumerate}

\ea\label{ex:5.54}
\gll Te me{\ꞌ}e nō e noho era \textbf{he} \textbf{rauhuru} nō atu o te taro. \\
\textsc{art} thing just \textsc{ipfv} stay \textsc{dist} \textsc{pred} diverse just away of \textsc{art} taro \\

\glt 
‘The only thing that was still there, was many kinds of taro.’ \textstyleExampleref{[R363.004]} 
\z

\ea\label{ex:5.55}
\gll Te vaka o Paka{\ꞌ}a pē \textbf{he} \textbf{tētahi} vaka era {\ꞌ}ā. \\
\textsc{art} boat of Paka’a like \textsc{pred} other boat \textsc{dist} \textsc{ident} \\

\glt 
‘Paka’a’s boat was just like other boats.’ \textstyleExampleref{[R344.040]} 
\z

\begin{enumerate}
\setcounter{enumi}{3}
\item
Like other determiners, \textit{he} does not occur before pronouns or proper names. 
Pronouns and proper names are rather preceded by \textit{ko}\is{ko (prominence marker)} or the proper article\is{a (proper article)} \textit{a} (see \sectref{sec:5.13.2}).

\item
Like other determiners, \textit{he} can signal nominalisation of a verb; see \REF{ex:3.23} on p.~\pageref{ex:3.23} for an example.
\end{enumerate}

This evidence clearly shows that \textit{he} is a determiner, even if its distribution is different from other determiners. 
\is{he (nominal predicate marker)|)}
\subsection{Numeral phrases in determiner position}\label{sec:5.3.5}
\is{Numeral!in noun phrase|(}
As indicated in \sectref{sec:5.3.1} above, numeral phrases may occur in determiner position, thereby excluding other determiners. Interestingly, they occur both in referential noun phrases (which would otherwise contain a \textit{t-}determiner) and in non-referential noun phrases (which would otherwise contain the predicate marker \textit{he}).

Here are two examples of prenominal numerals in referential noun phrases, as subject \REF{ex:5.56} and direct object \REF{ex:5.57}, respectively:

\ea\label{ex:5.56}
\gll He poreko \textbf{e} \textbf{toru} poki. \\
\textsc{ntr} born \textsc{num} three child \\

\glt 
‘Three children were born.’ \textstyleExampleref{[R352.010]} 
\z

\ea\label{ex:5.57}
\gll He aŋa \textbf{e} \textbf{tahi} paepae. \\
\textsc{ntr} make \textsc{num} one shack \\

\glt 
‘He built a shelter.’ \textstyleExampleref{[Blx-2-1.015]}
\z

Prenominal numerals also occur in various non-referential (or at least less referential) constructions, in which the determiner \textit{he} would be used otherwise (cf. the uses listed in \sectref{sec:5.3.4.1} above): 

After \textit{{\ꞌ}ina}:

\ea\label{ex:5.58}
\gll {\ꞌ}Ina \textbf{e} \textbf{tahi} kope i {\ꞌ}ite ko ai te me{\ꞌ}e i rē. \\
\textsc{neg} \textsc{num} one person \textsc{pfv} know \textsc{prom} who \textsc{art} thing \textsc{pfv} won \\

\glt
‘No one (lit. not one person) knew who had won.’ \textstyleExampleref{[R448.018]} 
\z

In an existential clause:

\ea\label{ex:5.59}
\gll \textbf{E} \textbf{tahi} poki te {\ꞌ}īŋoa ko Eva ka ho{\ꞌ}e {\ꞌ}ahuru matahiti. \\
\textsc{num} one child \textsc{art} name \textsc{prom} Eva \textsc{cntg} one ten year \\

\glt
‘There was a child called Eva, ten years old.’ \textstyleExampleref{[R210.001]} 
\z

In a list:

\ea\label{ex:5.60}
\gll E ono kope o ruŋa: \textbf{e} \textbf{hā} taŋata, \textbf{e} \textbf{tahi} vi{\ꞌ}e, {\ꞌ}e he poki e tahi.\\
\textsc{num} six person of above \textsc{num} four man \textsc{num} one woman and \textsc{pred} child \textsc{num} one\\

\glt 
‘There were six people on board: four men, one woman, and one child.’ \textstyleExampleref{[R231.085]} 
\z

In the first version of this grammar \citep{Kieviet2016}, prenominal numerals were analysed as being in a post-determiner position. However, on a closer analysis it makes more sense to analyse them as determiners: they are in complementary distribution with other determiners and they occur in contexts where otherwise either a \textit{t}{}-determiner or \textit{he} is required. Under the present analysis, noun phrases with a prenominal numeral are no exception to the rule that in these contexts a determiner is obligatory.

Finally, if prenominal numerals are determiners, this also explains why they are excluded after the instrumental preposition \textit{hai}, which precludes the use of a determiner: in order to use a numeral after \textit{hai}, the numeral must occur after the noun (see \REF{ex:5.65} in \sectref{sec:5.4.2} below).
\is{Determiner|)}
\section{Numerals in the noun phrase}\label{sec:5.4}

\sectref{sec:3.5} discusses numerals in general; in the present section, their occurrence in the noun phrase is discussed. Numerals can appear either before or after the noun; both positions will be discussed in turn.

\subsection{Numerals before the noun}\label{sec:5.4.1}

Numeral phrases very often appear before the noun. In \sectref{sec:5.3.5} I argued that these numerals are in determiner position, as they exclude other determiners and have a distribution typical of determiners.

As discussed in \sectref{sec:5.3.2.1}, prenominal numerals cannot be preceded by a preposition. As a consequence, they occur most commonly in noun phrases functioning as subject or direct object\footnote{\label{fn:258}The same constraint applies in \ili{Tahitian}: with prenominal numerals, the object marker is omitted (\citealt[184]{LazardPeltzer2000}).} (see (\ref{ex:5.56}–\ref{ex:5.57}) in \sectref{sec:5.3.5} above); however, the noun phrase may also be an oblique argument\is{Oblique} \REF{ex:5.61} or adjunct \REF{ex:5.62}. Without a prenominal marker, the noun phrase in \REF{ex:5.61} would be preceded by the preposition \textit{ki} ‘to’, while the adjunct noun phrase in \REF{ex:5.62} would be preceded by \textit{{\ꞌ}i} ‘in, at’.

\ea\label{ex:5.61}
\gll E ko {\ꞌ}avai e au \textbf{e} \textbf{tahi} taŋata i tā{\ꞌ}aku poki. \\
\textsc{ipfv} \textsc{neg.ipfv} give \textsc{ag} \textsc{1sg} \textsc{num} one person \textsc{acc} \textsc{poss.1sg.a} child \\

\glt 
‘I won’t give my child to anybody\textit{.}’ \textstyleExampleref{[R229.069]} 
\z

\ea\label{ex:5.62}
\gll He noho \textbf{e} \textbf{toru} marama {\ꞌ}i Aro Huri. \\
\textsc{ntr} stay \textsc{num} three month at Aro Huri \\

\glt 
‘He stayed three months in Aro Huri\textit{.}’ \textstyleExampleref{[MsE-109.013]}
\z

Constructions like \REF{ex:5.61} above are somewhat rare, though; it is unusual for the preposition \textit{ki} to be omitted. 

\subsection{Numerals after the noun}\label{sec:5.4.2}
When the noun phrase is preceded by a preposition requiring a determiner (\sectref{sec:5.3.2.1}), numerals must be placed after the noun. Here are examples with possessive \textit{o} \REF{ex:5.63}, and the preposition \textit{i} after a locational\is{Locational} \REF{ex:5.64}:\footnote{\label{fn:259}The only case in which a numeral does occur after a locative expression, is when the noun phrase is headless\is{Noun phrase!headless}. In the following example, the noun phrase \textit{e tahi o kōrua} consists of a numeral phrase and a possessive; there is no head noun.
\ea
\gll 
...{\ꞌ}o topa tā{\ꞌ}ue rō atu te {\ꞌ}ati a ruŋa \textbf{e} \textbf{tahi} o kōrua.\\
  ~~~lest happen by\_chance \textsc{emph} away \textsc{art} problem by above \textsc{num} one of \textsc{2pl} \\
  \glt 
  ‘lest a disaster may fall on one of you’ (R313.010) \z
In this case, there is no postnominal position available (alternatively, one could assume that the numeral is in postnominal position, which cannot be distinguished from the prenominal position anyhow).}

\ea\label{ex:5.63}
\gll te hare \textbf{o} \textbf{te} taŋata \textbf{e} \textbf{tahi} \\
\textsc{art} house of \textsc{art} man \textsc{num} one \\

\glt 
‘the house of one man’ \textstyleExampleref{[Notes]}
\z

\ea\label{ex:5.64}
\gll He eke ki ruŋa \textbf{i} \textbf{te} mā{\ꞌ}ea \textbf{e} \textbf{tahi}.\\
\textsc{ntr} go\_up to above at \textsc{art} stone \textsc{num} one\\

\glt 
‘He climbed on a stone.’ \textstyleExampleref{[R229.347]} 
\z

The instrumental preposition \textit{hai}\is{hai (instrumental prep.)} ‘with’ excludes a determiner (\sectref{sec:4.7.9}); here as well, numerals must be postnominal.

\ea\label{ex:5.65}
\gll E {\ꞌ}auhau era {\ꞌ}i te {\ꞌ}āva{\ꞌ}e \textbf{hai} māmoe \textbf{e} \textbf{hā}. \\
\textsc{ipfv} pay \textsc{dist} at \textsc{art} month \textsc{ins} sheep \textsc{num} four \\

\glt
‘He was paid four sheep (lit. with four sheep) per month.’ \textstyleExampleref{[R250.053]} 
\z

Likewise, numerals must be postnominal after the preposition \textit{pē}\is{pe ‘like’@pē ‘like’} ‘like’, which is obligatorily followed by the predicate marker \textit{he} (\sectref{sec:4.7.8}):

\ea\label{ex:5.66}
\gll Ta{\ꞌ}ato{\ꞌ}a mata e ai rō {\ꞌ}ana te rāua taŋata pū{\ꞌ}oko... \textbf{pē} he suerekao  \textbf{e} \textbf{tahi} te haka aura{\ꞌ}a.\\
all tribe \textsc{ipfv} exist \textsc{emph} \textsc{cont} \textsc{art} \textsc{3pl} man head like \textsc{pred} governor  \textsc{num} one \textsc{art} \textsc{caus} meaning\\

\glt 
‘All tribes had a leader... someone like a governor (lit. like one governor the meaning).’ \textstyleExampleref{[R371.006]} 
\z

In other situations, a noun phrase must contain a \textit{t-}determiner for discourse reasons. When a participant is definite, this is indicated by a \textit{t-}determiner + a postnominal demonstrative\is{Demonstrative!postnominal} (\sectref{sec:4.6.3}). In such cases, numerals must come after the noun. Consider the following example: 

\ea\label{ex:5.67}
\gll A Makemake i hakaroŋo mai era ki te ani atu o \textbf{tou} ŋāŋata \textbf{era}  \textbf{e} \textbf{rua}.\\
\textsc{prop} Makemake \textsc{pfv} listen hither \textsc{dist} to \textsc{art} request away of \textsc{dem} men \textsc{dist}  \textsc{num} two\\

\glt
‘Makemake listened to the request of those two men.’ \textstyleExampleref{[Fel-40.044]}
\z

The two men have been mentioned before and are therefore known; this is signalled by anaphoric\is{Anaphora} \textit{tou N era}. The numeral \textit{e rua} necessarily occurs after the noun.

\subsection{Optional numeral placement; \textit{e tahi} ‘one’}\label{sec:5.4.3}
\is{tahi ‘one’|(}
\sectref{sec:5.4.1} above describes contexts in which the numeral can be prenominal. This does not mean that the numeral must be prenominal in these contexts. Syntactically, in most of these cases the numeral can be placed after the noun as well. Here are examples with a postnominal numeral in a subject noun phrase \REF{ex:5.68} and a direct object noun phrase \REF{ex:5.69}:

\ea\label{ex:5.68}
\gll He oho mai te Miru \textbf{e} \textbf{rua}, ko Tema ko Pau {\ꞌ}a Vaka. \\
\textsc{ntr} go hither \textsc{art} Miru \textsc{num} two \textsc{prom} Tema \textsc{prom} Pau a Vaka \\

\glt 
‘Two Miru men came, Tema and Pau a Vaka.’ \textstyleExampleref{[Mtx-3-06.024]}
\z

\ea\label{ex:5.69}
\gll He tute mai i te moa \textbf{e} \textbf{tahi}. \\
\textsc{ntr} chase hither \textsc{acc} \textsc{art} chicken \textsc{num} one \\

\glt
‘He chased a (lit. one) chicken.’ \textstyleExampleref{[Mtx-7-03.033]}
\z

In cases like (\ref{ex:5.68}–\ref{ex:5.69}) the choice between pre- and postnominal numerals is syntactically free; however, there may be a subtle difference in meaning. This is especially the case with \textit{e tahi} ‘one’. In prenominal position, \textit{e tahi} tends to have a \is{Specific reference!non-specific reference}non-specific sense. This sense is especially clear after negations, when \textit{tahi} can often be translated as ‘any’:

\ea\label{ex:5.70}
\gll He hāhaki mai, pero \textbf{{\ꞌ}ina} \textbf{kai} rava{\ꞌ}a \textbf{e} \textbf{tahi} me{\ꞌ}e. \\
\textsc{ntr} gather\_shellfish hither but \textsc{neg} \textsc{neg.pfv} obtain \textsc{num} one thing \\

\glt
‘She went to gather shellfish, but she didn’t get anything.’ \textstyleExampleref{[R178.026]} 
\z

Similarly, in existential clauses\is{Clause!existential}, \textit{{\ꞌ}ina e tahi} is used in the sense ‘not any, none at all’:

\ea\label{ex:5.71}
\gll \textbf{{\ꞌ}Ina} \textbf{e} \textbf{tahi} kona toe mo moe. \\
\textsc{neg} \textsc{num} one place remain for sleep \\

\glt
‘There was no place left to sleep.’ \textstyleExampleref{[R339.027]} 
\z

When the numeral is placed after the noun, its sense is often \is{Specific reference}specific, ‘one, a certain’: 

\ea\label{ex:5.72}
\gll He moe ki te uka \textbf{e} \textbf{tahi}... \\
\textsc{ntr} lie\_down to \textsc{art} girl \textsc{num} one \\

\glt
‘He married a (certain) girl...’ \textstyleExampleref{[Blx-3.002]}
\z

There is no absolute distinction between the two, though. For example, in narrative texts, both prenominal and postnominal \textit{e tahi} are common to introduce participants at the beginning of stories:

\ea\label{ex:5.73}
\gll \textbf{E} \textbf{tahi} taŋata hōnui, te {\ꞌ}īŋoa o tū taŋata era ko {\ꞌ}Ohovehi. \\
\textsc{num} one man respected \textsc{art} name of \textsc{dem} man \textsc{dist} \textsc{prom} Ohovehi \\

\glt 
‘There was a respected man, the name of that man was Ohovehi.’ \textstyleExampleref{[R310.001]} 
\z

\ea\label{ex:5.74}
\gll He taŋata \textbf{e} \textbf{tahi} ko Marupua te {\ꞌ}īŋoa. \\
\textsc{pred} man \textsc{num} one \textsc{prom} Marapua \textsc{art} name \\

\glt 
‘There was a man called Marupua.’ \textstyleExampleref{[R481.001]}\textstyleExampleref{} 
\z
\is{tahi ‘one’|)}
\is{Numeral!in noun phrase|)}
\section{Plural markers}\label{sec:5.5}
\is{Plural!noun|(}\subsection{The plural marker \textit{ŋā}}\label{sec:5.5.1}
\is{nza (plural marker)@ŋā (plural marker)|(}\subsubsection[The position of ŋā]{The position of \textit{ŋā}}\label{sec:5.5.1.1}

The plural marker \textit{ŋā} always occurs immediately before the noun: 

\ea\label{ex:5.75}
\gll He tu{\ꞌ}u mai tou \textbf{ŋā} \textbf{uka} era. \\
\textsc{ntr} arrive hither \textsc{dem} \textsc{pl} girl \textsc{dist} \\

\glt
‘Those girls arrived’. \textstyleExampleref{[Blx-3.053]}
\z

As this example shows, \textit{ŋā} is not an article.\footnote{\label{fn:260}\textit{Pace} \citet[474]{DuFeu1987}.} It occurs in a different position than the article \textit{te} and often co-occurs with it. This is different from its cognates in most other \is{Eastern Polynesian}Eastern Polynesian languages, which are usually plural articles.\footnote{\label{fn:261}\ili{Hawaiian} \textit{naa}, \ili{Māori} \textit{nga}, \ili{Marquesan} \textit{na} and the possible cognate \ili{Tahitian} \textit{nā} are all determiners, which do not co-occur with \textit{te}. In \ili{Hawaiian} and \ili{Māori} this article denotes plurality, in \ili{Marquesan} and \ili{Tahitian} it is used for a dual or limited plural (see \citealt[19]{Elbert1976}; \citealt[20]{Biggs1973}; \citealt[11]{Zewen1987}; \citealt[16]{AcadémieTahitienne1986}). In \ili{Tahitian}, according to \citet[18]{AcadémieTahitienne1986}, \textit{nā} is incompatible with the article \textit{te} and composite determiners containing \textit{te} (despite Tryon’s example \textit{tā{\ꞌ}u nā {\ꞌ}urī}, \citet[17]{Tryon1970}), but it may co-occur with the demonstrative \textit{taua}: \textit{taua nā tamari{\ꞌ}i a Noa}... ‘those children of Noah’ (Gen. 9:18).

In \ili{Rarotongan}, on the other hand, the particle \textit{ŋā} – which is most commonly used for pairs – is commonly preceded by the article or another determiner: \textit{tōku ngā metua} ‘my parents’ (\citealt[405–406]{Buse1963Nominal}); \textit{te ngā pēre pūtē} ‘the two bales of sacks’ (Sally Nicholas, p.c.).}  

The fact that the plural is always contiguous to the noun, is an indication of its close syntactic association to the noun. Other indications are:

\begin{itemize}
\item 
The plural of \textit{taŋata} ‘man’ coalesced from *\textit{ŋā taŋata} into \textit{ŋāŋata}. 

\item 
Unlike any other prenominal element, \textit{ŋā} can precede a noun which modifies another noun:

\end{itemize}

\ea\label{ex:5.76}
\gll Ta{\ꞌ}e he aŋa \textbf{ŋā} vi{\ꞌ}e rā. \\
\textsc{conneg} \textsc{pred} work \textsc{pl} woman \textsc{dist} \\

\glt 
‘That’s not women’s work.’ \textstyleExampleref{[R347.103]} 
\z

\subsubsection[Use and non{}-use of ŋā ]{Use and non-use of \textit{ŋā}} \label{sec:5.5.1.2}

\textit{Ŋā} is not obligatory. When it is clear that the noun phrase has plural reference, \textit{ŋā} can be left out; this happens in the following situations:

Firstly, when the noun phrase contains a numeral:

\ea\label{ex:5.77}
\gll Vi{\ꞌ}e nei e ai rō {\ꞌ}ā \textbf{e} \textbf{rua} \textbf{poki} vahine. \\
woman \textsc{prox} \textsc{ipfv} exist \textsc{emph} \textsc{cont} \textsc{num} two child female \\

\glt 
‘This woman had two daughters’ \textstyleExampleref{[R491.008]} 
\z

Secondly, when the noun is subject of a verb which has a plural form as in \REF{ex:5.78}, or is modified by a plural adjective.

\ea\label{ex:5.78}
\gll He ŋaro tū pere{\ꞌ}oa era, he \textbf{ŋāŋaro} \textbf{te} \textbf{va{\ꞌ}ehau}. \\
\textsc{ntr} disappear \textsc{dem} car \textsc{dist} \textsc{ntr} \textsc{pl}:disappear \textsc{art} soldier \\

\glt 
‘The carriage disappeared and the soldiers disappeared.’ \textstyleExampleref{[R491.039]} 
\z

Thirdly, when the noun phrase contains the collective marker \textit{kuā} (\sectref{sec:5.2}).

Finally, when there are other indications in the context that the noun phrase has plural reference. The following example occurs in a story about a party. No plural marker is needed to indicate that a party involves more than one man and more than one woman:

\ea\label{ex:5.79}
\gll He hoki \textbf{te} \textbf{taŋata}, \textbf{te} \textbf{vi{\ꞌ}e}, te ŋā poki... ki to rātou hare. \\
\textsc{ntr} return \textsc{art} man \textsc{art} woman \textsc{art} \textsc{pl} child to \textsc{art}:of \textsc{3pl} house \\

\glt
‘(When the party is finished,) men, women and children go home.’ \textstyleExampleref{[Mtx-7-30.037]}
\z

In conclusion, \textit{ŋā} can be omitted when it is clear that reference is plural. However, this does not mean that \textit{ŋā} is only used when there is no other clue for plurality. It may co-occur with a numeral or other quantifier\is{Quantifier} as in \REF{ex:5.80} or a plural verb form as in \REF{ex:5.81}:

\ea\label{ex:5.80}
\gll He e{\ꞌ}a mai tou \textbf{ŋāŋata} \textbf{e} \textbf{ono} mai roto mai te hare ki haho. \\
\textsc{ntr} go\_out hither \textsc{dem} men \textsc{num} six from inside from \textsc{art} house to outside \\

\glt 
‘Those six men came out of the house.’ \textstyleExampleref{[Ley-4-01.023]}
\z

\ea\label{ex:5.81}
\gll Ko \textbf{{\ꞌ}a{\ꞌ}ara} {\ꞌ}ana tū ŋā vārua era. \\
\textsc{prf} \textsc{pl}:wake\_up \textsc{cont} \textsc{dem} \textsc{pl} spirit \textsc{dist} \\

\glt
‘Those spirits woke up.’ \textstyleExampleref{[R233.026]} 
\z

The only case in which \textit{ŋā} is obligatory, is with the noun \textit{io} ‘young man’, which (almost) only occurs as a plural \textit{ŋā io}. \textit{Ŋā io} is especially common in older stories, but is still in use. It is so much a unit that \citet{Englert1978,Englert1980} writes it as one word. 

\subsubsection[Semantics of ŋā]{Semantics of \textit{ŋā}}\label{sec:5.5.1.3}

In older texts, \textit{ŋā} is almost exclusively used with nouns referring to humans: \textit{taŋata} ‘man’, \textit{vi{\ꞌ}e} ‘woman’, \textit{poki} ‘child’, \textit{matu{\ꞌ}a} ‘father’, \textit{taina} ‘brother’, et cetera.\footnote{\label{fn:262}\citet[26]{Englert1978} states that \textit{ŋā} is only used for persons.} 

Nowadays, \textit{ŋā} is frequently used with inanimate nouns as well, including abstract nouns:\footnote{\label{fn:263}According to \citet[170]{Schuhmacher1993}, this development occurred under influence of \ili{Tahitian} \textit{nā}; more likely, it was influenced by \ili{Spanish} – where plurality is consistently marked – or a (quite natural) language-internal development.}

\ea\label{ex:5.82}
\gll Te \textbf{ŋā} \textbf{vaka} ra{\ꞌ}e tu{\ꞌ}u mai era e ueue nō {\ꞌ}ana {\ꞌ}i rote vai  {\ꞌ}i te reherehe.\\
\textsc{art} \textsc{pl} boat first arrive hither \textsc{dist} \textsc{ipfv} sway:\textsc{red} just \textsc{cont} at inside\_\textsc{art} water  at \textsc{art} weak\\

\glt 
‘The first boats that arrived rocked in the water because they were so flimsy.’ \textstyleExampleref{[R539-1.550]}
\z

\ea\label{ex:5.83}
\gll E tai{\ꞌ}o {\ꞌ}i ra{\ꞌ}e {\ꞌ}e {\ꞌ}ai ka pāhono iho te \textbf{ŋā} \textbf{{\ꞌ}ui} ena. \\
\textsc{exh} read at first and there \textsc{cntg} answer just\_then \textsc{art} \textsc{pl} ask \textsc{med} \\

\glt
‘First read, then answer these questions.’ \textstyleExampleref{[R534.013]} 
\z

The sense of \textit{ŋā} is very general. It can be used for small and large numbers alike:

\ea\label{ex:5.84}
\gll tā{\ꞌ}ana ŋā poki \textbf{e} \textbf{rua} \\
\textsc{poss.3sg.a} \textsc{pl} child \textsc{num} two \\

\glt 
‘his two children’ \textstyleExampleref{[R376.033]} 
\z

\ea\label{ex:5.85}
\gll He pōrekoreko \textbf{me{\ꞌ}e} \textbf{rahi} \textbf{nō} \textbf{atu} rāua ŋā poki. \\
\textsc{ntr} born:\textsc{red} thing many just away \textsc{3pl} \textsc{pl} child \\

\glt
‘They had many children (lit. many their children were born).’ \textstyleExampleref{[R438.049]} 
\z

It can be used for items forming a group as in \REF{ex:5.86}, or for a plurality of separate items as in \REF{ex:5.87}:\footnote{\label{fn:264}\textit{Pace} \citet[135]{DuFeu1996}; \citet[485]{DuFeu1987}.}

\ea\label{ex:5.86}
\gll I taŋi era te oe, he tāhuti tahi \textbf{te} \textbf{ŋā} \textbf{poki} he haka kāuŋa. \\
\textsc{pfv} cry \textsc{dist} \textsc{art} bell \textsc{ntr} \textsc{pl}:run all \textsc{art} \textsc{pl} child \textsc{ntr} \textsc{caus} line\_up \\

\glt 
‘When the bell sounded, all the children ran and stood in line.’ \textstyleExampleref{[R334.012]} 
\z

\ea\label{ex:5.87}
\gll Te nua rakerake mo te \textbf{ŋā} \textbf{me{\ꞌ}e} ha{\ꞌ}ere tahaŋa nō a te ara. \\
\textsc{art} cloth\_cape bad:\textsc{red} for \textsc{art} \textsc{pl} thing walk aimlessly just by \textsc{art} road \\

\glt
‘The ordinary capes were for the ones (=people) who just walked along the road.’ \textstyleExampleref{[Ley-5-04.012]}
\z

In conclusion, \textit{ŋā} may indicate any kind of plurality with any noun.

\subsection{Co-occurrence of \textit{ŋā} and the determiner}\label{sec:5.5.2}

As shown above, the occurrence of \textit{ŋā} is independent of the occurrence of the article. However, there are some noun phrases containing \textit{ŋā} which do not have any determiner, even though there should be a determiner according to the conditions listed in \sectref{sec:5.3.2}. Here are some examples:

\ea\label{ex:5.88}
\gll ¿Ki hē kōrua ko \textbf{ŋā} \textbf{kope}? \\
~to \textsc{cq} \textsc{2pl} \textsc{prom} \textsc{pl} person \\

\glt 
‘Where are you going, guys?’ \textstyleExampleref{[Ley-4-05.066]}
\z

\ea\label{ex:5.89}
\gll Ka raŋi \textbf{ŋā} \textbf{kope} ka oho mai. \\
\textsc{imp} call \textsc{pl} person \textsc{imp} go hither \\

\glt 
‘Call the guys to come.’ \textstyleExampleref{[R232.058]} 
\z

\ea\label{ex:5.90}
\gll E tahi pihi o te ta{\ꞌ}u o \textbf{ŋā} \textbf{poki} o Miru, he ora haka{\ꞌ}ou  \textbf{ŋā} \textbf{poki} o Tūpāhotu.\\
\textsc{num} one decade of \textsc{art} year of \textsc{pl} child of Miru \textsc{ntr} live again  \textsc{pl} child of Tupahotu\\

\glt 
‘After ten years of (the reign of) the children of Miru, the children of Tupahotu revived.’ \textstyleExampleref{[Mtx-3-07.016]}
\z

\ea\label{ex:5.91}
\gll He oho au ki \textbf{ŋā} \textbf{hare} he no{\ꞌ}ino{\ꞌ}i hai kona mahute. \\
\textsc{ntr} go \textsc{1sg} to \textsc{pl} house \textsc{ntr} request:\textsc{red} \textsc{ins} place mulberry \\

\glt
‘I’m going to the houses to ask for mulberry fibres.’ \textstyleExampleref{[R352.025]} 
\z

Although these examples are unusual, they are grammatical and can be explained in one of several ways:

\begin{itemize}
\item 
In \REF{ex:5.88} and \REF{ex:5.89} the noun is \textit{kope}. \textit{(Kōrua) ko ŋā kope} is more or less a frozen expression, though \REF{ex:5.89} shows that it also occurs without \textit{ko}. It expresses endearment: ‘those dear boys’.

\item 
The noun phrase in \REF{ex:5.90} can be regarded as similar to a name: \textit{ŋā poki o Miru} ‘the Miru people’. Rapa Nui has more cases where names are introduced by \textit{ŋā}:

\end{itemize}

\ea\label{ex:5.92}
\gll E ono \textbf{Ŋā} \textbf{Ruti} \textbf{Matakeva}... He oho e tahi \textbf{Ŋā} \textbf{Ruti,} \\
\textsc{num} six Nga Ruti Matakeva \textsc{ntr} go \textsc{num} one Nga Ruti \\

\glt
‘There were six (men called) Nga Ruti Matakeva... One Nga Ruti went...’ \textstyleExampleref{[Mtx-3-11.001,005]}
\z

\begin{itemize}
\item 
	\REF{ex:5.91} may be an example of non-referential\is{Referentiality} use. In such expressions the noun phrase does not refer to any house, but to houses in general. (See (\ref{ex:5.20}–\ref{ex:5.21}) on p.~\pageref{ex:5.20}.)
\is{nza (plural marker)@ŋā (plural marker)|)}
\end{itemize}
\subsection{Other words used as plural markers}\label{sec:5.5.3}

Sometimes plurality is expressed by other words than \textit{ŋā}. 

\subparagraph{\textit{Mau}} Some speakers use the \ili{Tahitian} plural marker \textit{mau}\is{mau (plural marker)} (not to be confused with the emphatic marker, \sectref{sec:5.8}). \ili{Tahitian} \textit{mau}, like Rapa Nui \textit{ŋā}, is a marker which occurs after the article. For speakers familiar with \ili{Tahitian}, the similarity in syntax would facilitate using the \ili{Tahitian} form.

\ea\label{ex:5.93}
\gll te \textbf{mau} \textbf{matahiti} i noho era {\ꞌ}i Rapa Nui \\
\textsc{art} \textsc{pl} year \textsc{pfv} stay \textsc{dist} at Rapa Nui \\

\glt 
‘the years when he lived on Rapa Nui’ \textstyleExampleref{[R231.306]} 
\z	

\ea\label{ex:5.94}
\gll mo te \textbf{mau} \textbf{mā{\ꞌ}ohi} o Rapa Nui\\
for \textsc{art} \textsc{pl} indigenous of Rapa Nui\\

\glt
‘for the indigenous people of Rapa Nui’ \textstyleExampleref{[billboard in the street]}
\z

Like most \ili{Tahitian} borrowings\is{Tahitian influence}, this is a relatively recent phenomenon. 

\subparagraph{\textit{Nu{\ꞌ}u}} \textit{Nu{\ꞌ}u}\is{nuu ‘people’@nu{\ꞌ}u ‘people’} ‘people’ (an inherently plural noun, borrowed from \ili{Tahitian} \textit{\mbox{nu{\ꞌ}u}} ‘army, collectivity’) can be placed in front of another noun and play the same role as a plural marker. It implies a group of people being and/or acting together.

\ea\label{ex:5.95}
\gll Mai ira ia māua i oho ai ki te hare o tō{\ꞌ}oku \textbf{nu{\ꞌ}u} \textbf{huŋavai}. \\
from \textsc{ana} then \textsc{1du.excl} \textsc{pfv} go \textsc{pvp} to \textsc{art} house of \textsc{poss.1sg.o} people parent\nobreakdash-in\nobreakdash-law \\

\glt 
‘From there we went to the house of my parents-in-law.’ \textstyleExampleref{[R107.018]} 
\z

\ea\label{ex:5.96}
\gll Tā{\ꞌ}ana me{\ꞌ}e haŋa... he reka ananake ko tō{\ꞌ}ona \textbf{nu{\ꞌ}u} \textbf{repahoa}.\\
\textsc{poss.3sg.a} thing want \textsc{pred} entertaining together \textsc{prom} \textsc{poss.3sg.o} people friend\\

\glt
‘What he likes is... having a good time with his friends.’ \textstyleExampleref{[R489.003]} 
\z

This does not mean that \textit{nu{\ꞌ}u} is a plural marker like \textit{ŋā}. Syntactically \textit{nu{\ꞌ}u} is a head noun modified by another noun. It can even be preceded by \textit{ŋā} (\textit{tū ŋā \mbox{nu{\ꞌ}u} era} ‘those people’).
\is{Plural!noun|)}
\section{The noun: headless noun phrases}\label{sec:5.6}
\is{Noun phrase!headless}\is{Noun phrase!headless}
In most contexts, the noun is obligatory; headless\is{Noun phrase!headless} noun phrases are uncommon in Rapa Nui. They do occur, but only in certain specific contexts.

\subparagraph{\ref{sec:5.6}.1} With numerals, and with quantifiers\is{Quantifier} like \textit{tētahi} (\sectref{sec:4.4.6.2}) and \textit{ta{\ꞌ}ato{\ꞌ}a} (\sectref{sec:4.4.2}).

\ea\label{ex:5.97}
\gll \textbf{E} \textbf{tahi} i va{\ꞌ}ai ki a tō{\ꞌ}ona māmātia. \textbf{Tētahi} \textbf{atu} i va{\ꞌ}ai  ki tētahi nu{\ꞌ}u.\\
\textsc{num} one \textsc{pfv} give to \textsc{prop} \textsc{poss.3sg.o} aunt other away \textsc{pfv} give  to other people\\

\glt 
‘One she gave to her aunt. The others she gave to other people.’ \textstyleExampleref{[R168.006–007]}
\z

\ea\label{ex:5.98}
\gll \textbf{Ta{\ꞌ}ato{\ꞌ}a} e tahuti era, e tari mai era i te kai. \\
all \textsc{ipfv} run \textsc{dist} \textsc{ipfv} carry hither \textsc{dist} \textsc{acc} \textsc{art} food \\

\glt 
‘All ran, carrying the food.’ \textstyleExampleref{[R210.155]} 
\z

\subparagraph{\ref{sec:5.6}.2} After a \textit{t}{}-possessive\is{Pronoun!possessive!t-class} pronoun, in the partitive\is{Partitive} construction “possessive \textit{o te} noun” (\sectref{sec:6.2.2}). In this construction, the noun phrase does not have a head noun; instead, the main concept is expressed by a genitive phrase:

\ea\label{ex:5.99}
\gll Kai toe \textbf{tā{\ꞌ}ana} \textbf{o} \textbf{te} \textbf{ika}, o te {\ꞌ}ura, o te kō{\ꞌ}iro. \\
\textsc{neg.pfv} remain \textsc{poss.3sg.a} of \textsc{art} fish of \textsc{art} lobster of \textsc{art} conger\_eel \\

\glt
‘There was no fish, lobster or conger eel left for her.’ \textstyleExampleref{[Mtx-4-04.003]}
\z

In other cases it is also possible to leave out the noun after a \textit{t}{}-possessive\is{Pronoun!possessive!t-class} pronoun. The implied head noun may be expressed in a preceding clause as in \REF{ex:5.100}, or not at all as in \REF{ex:5.101}.

\ea\label{ex:5.100}
\gll Ka rova{\ꞌ}a e Hete i te māmari ka puā tako{\ꞌ}a a Kikio ki \textbf{tā{\ꞌ}ana}. \\
\textsc{cntg} obtain \textsc{ag} Hete \textsc{acc} \textsc{art} egg \textsc{cntg} touch also \textsc{prop} Kikio to \textsc{poss.3sg.a} \\

\glt 
‘When Hete got an egg, Kikio also touched his (one).’ \textstyleExampleref{[R438.042]} 
\z

\ea\label{ex:5.101}
\gll He va{\ꞌ}ai \textbf{tā{\ꞌ}ana}, \textbf{tā{\ꞌ}ana}, he tuha{\ꞌ}a. \\
\textsc{ntr} give \textsc{poss.3sg.a} \textsc{poss.3sg.a} \textsc{ntr} distribute \\

\glt 
‘He gave everyone his share, distributing it.’ \textstyleExampleref{[R372.123]} 
\z

\subparagraph{\ref{sec:5.6}.3} Similarly, a possessive phrase may occur without head noun; the head noun is understood from the context. The noun phrase starts with \textit{to}, which is a contraction of the article \textit{te} and the possessive marker \textit{o} (\sectref{sec:6.2.3}):

\ea\label{ex:5.102}
\gll Ko Koka te {\ꞌ}īŋoa o tō{\ꞌ}ona hoi... ko Parasa \textbf{to} \textbf{te} \textbf{rū{\ꞌ}au} era  {\ꞌ}ā{\ꞌ}ana.\\
\textsc{prom} Koka \textsc{art} name of \textsc{poss.3sg.o} horse \textsc{prom} Parasa \textsc{art}:of \textsc{art} old\_woman \textsc{dist}  \textsc{poss.3sg.a}\\

\glt 
‘Koka was the name of his horse, Parasa the (name) of his old wife.’ \textstyleExampleref{[R539-1.420]}
\z

\ea\label{ex:5.103}
\gll ¿Ko ai te {\ꞌ}īŋoa o te taŋata? ¿\textbf{To} \textbf{te} \textbf{vi{\ꞌ}e}? \\
~\textsc{prom} who \textsc{art} name of \textsc{art} name \textsc{art}:of \textsc{art} woman \\

\glt 
‘What is the man’s name? And the woman’s?’ \textstyleExampleref{[Notes]}
\z

\subparagraph{\ref{sec:5.6}.4} Headless noun phrases\is{Noun phrase!headless} are marginally possible in noun phrases containing adjectives. Adjectives in the noun phrase usually need a noun. If need be, a generic noun like \textit{kope} ‘person’ or \textit{me{\ꞌ}e}\is{mee ‘thing’@me{\ꞌ}e ‘thing’} ‘thing’ is used.

\ea\label{ex:5.104}
\gll He kī te poki (kope, me{\ꞌ}e) nuinui ki te poki (kope, me{\ꞌ}e) {\ꞌ}iti{\ꞌ}iti... \\
\textsc{ntr} say \textsc{art} child ~person thing big:\textsc{red} to \textsc{art} child ~person thing small:\textsc{red} \\

\glt
‘(There were two children.) The big one said to the small one...’ \textstyleExampleref{[Notes]}
\z

But with a few adjectives in a specific idiomatic sense, the noun can be left out:

\ea\label{ex:5.105}
\gll Te pepe nei mo \textbf{te} \textbf{hōnui}. Te pepe era mo \textbf{te} \textbf{rikiriki}. \\
\textsc{art} chair \textsc{prox} for \textsc{art} respected \textsc{art} chair \textsc{dist} for \textsc{art} small:\textsc{pl}:\textsc{red} \\

\glt
‘These chairs are for the authorities. Those chairs are for the small people.’ \textstyleExampleref{[Notes]}
\z

In the (infrequent) cases above, the nounless construction refers to someone or something possessing a quality. These should be distinguished from nominally used adjectives which refer to the quality as such. The former can be considered as ellipsis of a noun, the latter as conversion of an adjective to a noun \citep[96]{Bhat1994}. In the following examples, \textit{nuinui} ‘big’ is used as a noun in the sense ‘bigness, size, greatness’. It cannot be used in the sense ‘big one’. 

\ea\label{ex:5.106}
\gll Te \textbf{nuinui} o Tahiti e {\ꞌ}āmui atu tāua e ono \textbf{nuinui} nei o Rapa Nui. \\
\textsc{art} big:\textsc{red} of Tahiti \textsc{ipfv} add away \textsc{1du.incl} \textsc{num} six big \textsc{prox} of Rapa Nui \\

\glt 
‘The size of Tahiti altogether is six times the size of Rapa Nui.’ \textstyleExampleref{[R348.003]} 
\z

\subparagraph{\ref{sec:5.6}.5} Relative clauses (\sectref{sec:11.4}) can never be headless\is{Clause!relative!headless}, but need to be preceded by a noun. When no other noun is available, the dummy noun \textit{me{\ꞌ}e}\is{mee ‘thing’@me{\ꞌ}e ‘thing’} is used. This happens for example in clefts (\sectref{sec:9.2.6}).

To summarise: headless noun phrases occur occasionally, on the condition that the noun phrase contains either a possessor, a numeral phrase, a quantifier, or one of a small set of adjectives. They cannot occur with just any adjective; neither is the presence of a relative clause or a demonstrative sufficient to omit the head noun.

\section{Modifiers in the noun phrase}\label{sec:5.7}
\is{Noun!as modifier|(}\subsection{Introduction: types of modifiers}\label{sec:5.7.1}

The noun may be modified by various elements: nouns, adjectives or – less commonly – verbs.\footnote{\label{fn:265}Cf. \citet[325]{Vernaudon2011}, who gives examples of an adjective, noun and verb modifying a noun in \ili{Tahitian}.} A modifying noun may in turn be modified by another noun, verb or adjective, and so on. Modifying verbs may be followed by a direct object; modifying adjectives may be modified by various elements, such as degree markers.

At first sight, a modifying noun or verb seems to have the same status as a modifying adjective, but there are important differences between the two. Syntactically, a modifying noun or verb is incorporated into the head noun; it is a bare noun or verb, not followed by verb phrase particles. Modifying adjectives, on the other hand, form an adjective phrase, which may contain elements like degree modifiers, negators\is{Negation} and adverbs (\sectref{sec:5.7.3.2}). This correlates with a difference in position: when a noun phrase contains both a modifying noun or verb and an adjective, the former is usually closer to the head noun.

Semantically, a modifying noun or verb tends to express a single concept together with the head noun. In other words, the combination is a compound\is{Compound}, a single lexical item. Adjectives, on the other hand, express some additional property of the concept expressed by the head. For example, in the following noun phrase, \textit{pū{\ꞌ}oko haka tere} ‘head \textsc{caus} run’ is a compound\is{Compound} noun with the sense ‘leader, head’, consisting of a noun and a modifying verb. The adjective phrase \textit{ta{\ꞌ}e tano} ‘not correct’ = ‘unrighteous’ modifies this compound\is{Compound} noun.

\ea\label{ex:5.107}
\gll te {\ob}\textbf{pū{\ꞌ}oko} \textbf{haka} \textbf{tere}\,{\cb} {\ob}\textbf{ta{\ꞌ}e} \textbf{tano}\,{\cb} era o te hare ture \\
\textsc{art} {\db}head \textsc{caus} run {\db}\textsc{conneg} correct \textsc{dist} of \textsc{art} house judgment \\

\glt
‘the unrighteous head of the courthouse’ \textstyleExampleref{[Luke 18:6]}
\z

We may therefore assume a distinction between modifiers as part of a compound\is{Compound} and modifiers in a post-nominal modifier position; in other words: noun adjuncts versus noun phrase adjuncts. This coincides with another syntactic difference: the order of elements within a compound\is{Compound} is fixed, while the relative order of adjectives is sometimes free ((\ref{ex:5.135}–\ref{ex:5.136}) on p.~\pageref{ex:5.135}).

Now the distinction between nominal and adjectival modifiers is not absolute. In the following example, the adjective \textit{{\ꞌ}āpī} and the proper noun \textit{rapa nui} both modify the noun \textit{poki}; there is no functional difference between the two modifiers.

\ea\label{ex:5.108}
\gll \textbf{Poki} \textbf{{\ꞌ}āpī} te me{\ꞌ}e era, \textbf{poki} \textbf{rapa} \textbf{nui} te me{\ꞌ}e ena. \\
child new \textsc{art} thing \textsc{dist} child Rapa Nui \textsc{art} thing \textsc{med} \\

\glt
‘That one is a young child, that one is a Rapa Nui child.’ \textstyleExampleref{[R416.238]} 
\z

In the following example, the noun+adjective combination \textit{tuki tōumāmari} is further removed from the noun than the adjective \textit{teatea}; here the modifying noun \textit{tuki} is obviously not incorporated into the head noun \textit{pokopoko}, but is a modifier on the same level as adjectives.

\ea\label{ex:5.109}
\gll Ka ma{\ꞌ}u mai ta{\ꞌ}a pokopoko teatea \textbf{tuki} \textbf{tōuamāmari}. \\
\textsc{imp} carry hither \textsc{poss.2sg.a} container white:\textsc{red} dot yellow \\

\glt
‘Bring your white, yellow-dotted bowl.’ \textstyleExampleref{[Notes]}
\z

Also, the fact that a certain noun+noun combination is a semantic unit does not imply that it is necessarily a syntactic unit as well, occupying the head position as a whole. Syntactic structure does not always mirror semantic structure. The underlined expressions in the following examples have an idiomatic sense, but they are not a syntactic unit. While there is a compound \textit{vare/ŋao} ‘slimy’ + ‘neck’ = ‘to crave’, here the same two elements are used in a verb + subject construction:\footnote{\label{fn:266}However, idioms like this do have a tendency to become syntactically united. In newer texts the expression \textit{mate te manava} is not found; instead, the compound verb\is{Compound} \textit{manava mate} is used.}

\ea\label{ex:5.110}
\gll He \textbf{vare} \textbf{te} \textbf{ŋao} ki te kai hāhaki {\ꞌ}i tai.\\
\textsc{ntr} slimy \textsc{art} neck to \textsc{art} food gather\_shellfish at sea\\

\glt 
‘They were craving to get shellfish on the seashore.’ \textstyleExampleref{[Mtx-7-30.043]}
\z

\ea\label{ex:5.111}
\gll He \textbf{mate} \textbf{te} \textbf{manava} o tau ŋā uka era ki tau ŋā io era.\\
\textsc{ntr} die \textsc{art} stomach of \textsc{dem} \textsc{pl} girl \textsc{dist} to \textsc{dem} \textsc{pl} lad \textsc{dist}\\

\glt
‘The girls fell in love with those boys.’ \textstyleExampleref{[Mtx-6-03.079]}
\z

Thus, the fact that a collocation is a semantic unit does not imply that its parts are in a single position in the noun phrase. Moreover, some noun-adjective combinations also express a single concept, just like noun + noun compounds.

\ea\label{ex:5.112}
\gll parau{\rmfnm} {\ꞌ}āpī ~ ~ ~ haraoa mata \\
word new ‘news’ ~ ~ bread raw ‘flour’\\
\z
\footnotetext{This compound\is{Compound} was borrowed from \ili{Tahitian}\is{Tahitian influence} as a whole. ‘Word’ is the \ili{Tahitian} sense of \textit{parau}; in Rapa Nui, \textit{parau} on its own does not mean ‘word’, but ‘paper’, ‘document’ or ‘authority’.}

In conclusion, there is no absolute distinction between modifying nouns and modifying adjectives. However, the following things are clear:

%\setcounter{listWWviiiNumlxxvileveli}{0}
\begin{enumerate}
\item 
the noun phrase may contain various modifiers;

\item 
modifiers closer to the noun are semantically closer to it as well. This is illustrated in \REF{ex:5.107} above; see also (\ref{ex:5.135}–\ref{ex:5.136}) on p.~\pageref{ex:5.135};

\item 
modifying nouns and verbs are usually incorporated into the head noun, occurring as bare modifiers immediately after the head noun. They tend to express a single concept together with the head noun;

\item 
modifying adjectives are not incorporated into the head noun. They may be further removed from the head noun and form an adjective phrase; they tend to express an additional property of the concept expressed by the head. 

\end{enumerate}

Because of the distinction between 3 and 4, the noun phrase chart in \sectref{sec:5.1} places compounds as a whole in the head position, while modifying adjectives are placed in a separate slot.

In the following sections, the different types of modifiers will be discussed: \sectref{sec:5.7.2} deals with compounds, \sectref{sec:5.7.3} with modifying adjectives. Even though this section is part of the chapter on noun phrases, verb compounds (i.e. compounds with a verb as head and occurring in a verbal context) will be discussed in \sectref{sec:5.7.2.4}.

\subsection{Compounds}\label{sec:5.7.2}
\is{Compound|(}
As shown in the previous section, compounds in Rapa Nui are formed by simply juxtaposing two words. The head word comes first, then the modifier. The structure may be recursive: the modifier may itself be the head to a second modifier. The modifying element may be a noun or verb. Most compounds are nouns (i.e. they have a noun as their head), but the discussion in these sections includes examples of compound\is{Compound} verbs and adjectives as well.

A distinction can be made between lexical and syntactic compounds (see \citealt[175]{Dryer2007Noun}). Lexical compounds have a meaning which is not predictable from the meaning of their parts, while syntactic compounds are productive constructions with a predictable meaning. Both are found in Rapa Nui and are discussed separately below. There is, however, no sharp distinction between the two. Certain compounds have a somewhat specialised, not quite predictable sense, yet it is easy to see how this sense could have arisen from the sense of their components. In fact, the distinction between lexical and syntactic compounds can be thought of as a continuum. At one end are completely predictable and productive compounds, at the other end are compounds with a completely unpredictable (e.g. figurative) sense. \tabref{tab:39} gives examples illustrating different points along this continuum.

\begin{table}
\begin{tabularx}{\textwidth}{L{39mm}L{21mm}L{26mm}L{27mm}}
\lsptoprule
{semantic relation} & {example} & {sense of parts} & {sense of whole}\\
\midrule
predictable & \textit{ivi ika} & bone + fish & fish bone\\
specialised, transparent & \textit{hare pure}  & house + prayer & church \\
less transparent & \textit{manu/pātia}\footnotemark{} & insect + sting & wasp\\
metaphorical & \textit{manu/rere} & bird + to fly & airplane\\
idiomatic, opaque & \textit{manu/uru} & bird + to enter & guest \\
idiomatic, opaque & \textit{vare/ŋao}& slimy + neck & to crave \\
\lspbottomrule
\end{tabularx}
\caption{Syntactic and lexical compounds}
\label{tab:39}
\end{table}

\footnotetext{ For compounds written as one word in the standard orthography\is{Orthography}, the parts are separated by a slash.} 
\subsubsection[Lexical compounds]{Lexical compounds}\label{sec:5.7.2.1}

As discussed in the previous section, there are various degrees of lexical compounding. Some compounds are specialised in meaning (i.e. the sense of the compound\is{Compound} cannot be predicted from the sense of the parts), but it is still clear how their meaning is derived from the meaning of the parts:\footnote{\label{fn:269}In the tables in this section, the second column gives the meaning of the component parts, the third column the meaning of the whole compound\is{Compound}.} 
\ea\label{ex:5.113}
\begin{tabbing}
xxxxxxxxxxxx \= xxxxxxxxxxxxxxxxxxxxx \= xxxxxxxxxxxxxxx  \kill
\textup{a.} \textit{manu/meri}  \> \textup{insect + honey}   \> \textup{bee}\\
  b. \textit{manu/pātia}  \>  {insect + sting}  \>  wasp\\
  c. \textit{manu/rere }  \>  {bird + to fly} \>   airplane\\
  d. \textit{kiri/va{\ꞌ}e} \>   {skin + foot} \>   shoe\\
  e. \textit{tuke/ŋao}  \>  {leaf vein + neck} \>   {nape of the neck}\\
  f. \textit{mata/vai}  \>  {eye + water}  \>  tear\\
  g. \textit{repa/hoa}  \>  {friend + friend} \>   friend
\end{tabbing}
\z
In the last two examples above, the relation between the two words is not that between head and modifier. In \textit{matavai}, the second noun \textit{vai} is semantically the head.\footnote{\label{fn:270}Another example is \textit{motore vaka} ‘motor boat’, noted by \citet[322]{Fischer2001Hispan}; this is probably a calque\is{Borrowing!calque} from \ili{English}.}\textstyleFootnoteSymbol{} In \textit{repahoa}, both components are synonyms which together yield a third synonym.

The compound\is{Compound} may also be a verb or adjective:

\ea\label{ex:5.114}
\begin{tabbing}
xxxxxxxxxxxx \= xxxxxxxxxxxxxxxxxxxxx \= xxxxxxxxxxxxxxx  \kill
  a. \textit{ma{\ꞌ}u/rima} \>  take hold + hand  \> catch in the act, surprise\\
  b. \textit{tunu/ahi}  \> cook + fire \>  to roast on a fire\\
  c. \textit{aŋa/rahi}  \> work + much \>  difficult
\end{tabbing}
\z
%\todo[inline]{General issue: space after tabbing is too large}
Some compounds are more than specialised in meaning: their sense is to a greater or lesser degree opaque.

\ea\label{ex:5.115}
\begin{tabbing}
xxxxxxxxxxxx \= xxxxxxxxxxxxxxxxxxxxx \= xxxxxxxxxxxxxxx  \kill
  a. \textit{hua/tahi} \>  fruit + one \>  only child\\
  b. \textit{manu/piri} \>  bird + join \>  friend\\
  c. \textit{vare/ŋao} \>  slimy + neck \>     to crave, desire
\end{tabbing}
\z

Opacity goes even further in compounds where one or both components do not occur at all in Rapa Nui (at least, not in the sense underlying the compound\is{Compound}); the origin of these components may or may not be reconstructible. 

\ea\label{ex:5.116}
\begin{tabbing}
xxxxxxxxxxxx \= xxxxxxxxxxxxxxxxxxxxxxx \= xxxxxxxxxxxxxxx  \kill
  a. \textit{hata/uma} \> \is{Proto-Polynesian}PPN \textit{*fatafata} ‘chest’ + \> sternum\\
\> ~~RN \textit{uma} ‘chest’  \\
  b. \textit{hatu/kai} \> RN \textit{hatu} ‘clod’ + ?  \> coagulated blood\\
  c. \textit{hānau/tama} \> PPN \textit{*fānau} ‘give birth’ + \> pregnant; pregnant woman\\
\> ~~\is{Proto-Polynesian}PPN *\textit{tama} ‘child’ 
\end{tabbing}
\z

These compounds either developed at a stage when both components were still in use in the sense they had in the protolanguage, or else they were inherited from the protolanguage as a whole. Opaque compounds may also have a more recent origin, being borrowed as a whole. One such word is \textit{hare toa} ‘store’, borrowed from \ili{Tahitian}\is{Tahitian influence}. The first part means ‘house’ (Rapa Nui \textit{hare}, \ili{Tahitian} \textit{fare}), the second part means ‘store’ in \ili{Tahitian} (from \ili{English}) but is not used in other contexts in Rapa Nui.\footnote{\label{fn:271}\textit{hare toa} is written as two words, because (at least some) speakers know the origin and meaning of the second part.}

In other cases, both components are known as Rapa Nui words, but one of them is no longer in use, or at least archaic. 

\ea\label{ex:5.117}
\begin{tabbing}
xxxxxxxxxxxx \= xxxxxxxxxxxxxxxxxxxxxxx \= xxxxxxxxxxxxxxx  \kill
  a. \textit{rau/huru} \>  hundred (archaic) + sort \>  manifold, diverse\\
  b. \textit{hiri/toe}  \> braid + hairlock (archaic) \>  hairband\\
  c. \textit{koro/haŋa} \>  when (archaic) + want \>  maybe
\end{tabbing}
\z

Such compounds function practically as single words: the original sense of their parts no longer plays a role.

Near the other end of the spectrum, i.e. similar to syntactic compounds, are compounds which are quite transparent in meaning, but which are still lexicalised to a certain degree; that is, they may be a single unit in the mental lexicon of speakers of the language. Though it is impossible to say exactly whether a compound\is{Compound} is or is not lexicalised, two indications for lexicalisation of a compound\is{Compound} are:

\begin{itemize}
\item 
it is used frequently;

\item 
it expresses a single concept, and is often a single word in other languages.

\end{itemize}

Some examples are:

\ea\label{ex:5.118}
\begin{tabbing}
xxxxxxxxxxxx \= xxxxxxxxxxxxxxxxx \= xxxxxxxxxxxxxxx \kill
a. \textit{ma{\ꞌ}ori hāpī} \> expert + learn \> teacher\\
b. \textit{hare hāpī} \> house + learn \> school\\
c. \textit{hare pure} \> house + pray \> church\\
d. \textit{hi{\ꞌ}o mata} \> glass + eye \> eyeglasses, spectacles\\
e. \textit{kona hare} \> place + house \> home
\end{tabbing}
\z

\subsubsection[Syntactic compounds]{Syntactic compounds}\label{sec:5.7.2.2}

Syntactic compounds are transparent in sense: their meaning can be predicted from the meaning of the parts. Syntactic compounds are productive and may express a wide variety of semantic relations. Here are some examples:

\ea\label{ex:5.119}
%Was: \small{
\normalsize{
\begin{tabbing}
xxxxxxxxxxxxxx \= xxxxxxxxxxxxxxxxx \= xxxxxxxxxxxxxxx \= xxxxxxxxxxxxx \kill
 \textit{\textbf{\textup{compound}}}\is{Compound} \> \textbf{sense of parts} \> \textbf{sense of whole} \> \textbf{semantic relation}\\
 a. \textit{kete kai} \> basket + food \> basket of food \> A containing B \\
 b. \textit{hare oru} \> house + pig \> pigsty \> A destined for B\\
 c. \textit{kūpeŋa ika} \> net + fish \> fishnet \> A destined for B\\
 d. \textit{karone pure} \> necklace + shell \> shell necklace \> A made of B\\
 e. \textit{tumu {\ꞌ}ānani} \> tree + orange \> orange tree \> A of the kind B, or:\\
\> \> \> A producing B\\
 f. \textit{{\ꞌ}ā{\ꞌ}ati vaka} \> contest + boat \> rowing contest \> A of the kind B\\
 g. \textit{{\ꞌ}au {\ꞌ}umu} \> smoke~+~earth~oven \> earth~oven smoke \> A originating from B\\
 h. \textit{pū{\ꞌ}oko ika} \> head + fish \> fish head \> A part of B
\end{tabbing}
}
%\todo[inline]{Text just doesn’ t fit, so I used small text}
\z
In syntactic compounds, the plural marker \textit{ŋā}\is{nza (plural marker)@ŋā (plural marker)} may intervene between the two nouns:

\ea\label{ex:5.120}
\gll Ta{\ꞌ}e he aŋa \textbf{ŋā} vi{\ꞌ}e rā. \\
\textsc{conneg} \textsc{pred} work \textsc{pl} woman \textsc{dist} \\

\glt 
‘It’s not women’s work.’ \textstyleExampleref{[R347.103]}  
\z

As illustrated in the previous section, the second element of a lexical compound\is{Compound} may also be a verb. This also happens with syntactic compounds. The noun may refer to a location where the event expressed by the verb takes place (as in a and b below), or an instrument used to perform the action expressed by the verb (as in c).

\ea\label{ex:5.121}
\begin{tabbing}
xxxxxxxxxxxxxxxxxx \= xxxxxxxxxxxxxxxxxxxxx \= xxxxxxxxxxxxxxx \kill
  a. \textit{{\ꞌ}ana ha{\ꞌ}uru} \> cave + to sleep  \> cave for sleeping\\
  b. \textit{henua poreko}   \> country + be born   \> country of birth\\
  c. \textit{hau hī}   \> line + to fish   \> fishing line
\end{tabbing}
\z

Compounds may also consist of three members. The third word is a noun \REF{ex:5.122}, verb \REF{ex:5.123} or adjective \REF{ex:5.124} modifying the second noun; together they modify the head noun. (On modifying verbs, see \sectref{sec:5.7.2.3} below.)

\ea\label{ex:5.122}
\begin{tabbing}
xxxxxxxxxxxxxxxxxx \= xxxxxxxxxxxxxxxxxxxxx \= xxxxxxxxxxxxxxx \kill
  a. \textit{kona nūna{\ꞌ}a hare}   \> place [group + house]  \> village\\
  b. \textit{kona tumu pika}   \> place [tree + fig]  \> figtree grove
\end{tabbing}
\z

\ea\label{ex:5.123}
\begin{tabbing}
xxxxxxxxxxxxxxxxxx \= xxxxxxxxxxxxxxxxxxxxx \= xxxxxxxxxxxxxxx \kill
  a. \textit{pūtē hare hāpī}   \> bag [house + learn] \>  schoolbag\\
  b. \textit{hāipoipo hare pure}  \>  wedding [house + pray] \>  church wedding
\end{tabbing}
\z

\ea\label{ex:5.124}
\begin{tabbing}
xxxxxxxxxxxxxxxxxx \= xxxxxxxxxxxxxxxxxxxxx \= xxxxxxxxxxxxxxx \kill
  a. \textit{nu{\ꞌ}u kiri teatea} \> people  [skin + white] \>   light-skinned people\\
  b. \textit{kona {\ꞌ}ō{\ꞌ}one rivariva} \>  place [soil + good] \>  place of good soil
\end{tabbing}
\z

\subsubsection[Incorporation of objects and verbs]{Incorporation of objects and verbs}\label{sec:5.7.2.3}
\is{Object!incorporation|(}
A verb as modifier may in turn be followed by its object. Like any modifying noun, the object is a bare noun, not marked with a determiner and/or object marker. This is a case of object incorporation: the object loses its object marking and its status as a noun phrase, and is directly adjoined to the verb.

\ea\label{ex:5.125}
\begin{tabbing}
xxxxxxxxxxxxxxxxxxxx \= xxxxxxxxxxxxxxxxxxxxxxx \= xxxxxxxxxxxxxxx \kill
  a. \textit{kona ha{\ꞌ}amuri {\ꞌ}Atua} \> place [to worship + God]  \> temple\\
  b. \textit{hi{\ꞌ}o u{\ꞌ}i {\ꞌ}āriŋa} \> glass [to watch + face]  \> mirror\\
  c. \textit{taŋata keukeu henua} \> man [to labour/till + land]  \> farmer\\
  d. \textit{{\ꞌ}āua {\ꞌ}oka kai} \> garden [to plant + food]  \> plantation, field
\end{tabbing}
\z

A combination of noun and verb modifiers and object incorporation may lead to even longer compounds, as the following examples show:

\ea\label{ex:5.126}
\gll {\ꞌ}i te mahana ta{\ꞌ}e noho {\ꞌ}i te \textbf{kona} \textbf{{\ꞌ}āua} {\ob}\textbf{{\ꞌ}oka} \textbf{kai}\,{\cb} nei {\ꞌ}ā{\ꞌ}ana \\
at \textsc{art} day \textsc{conneg} stay at \textsc{art} place enclosure {\db}to\_plant food \textsc{prox} \textsc{poss.3sg.a} \\

\glt 
‘on a day when he did not stay in his garden plot’ \textstyleExampleref{[R381.004]} 
\z

\ea\label{ex:5.127}
\gll Hai \textbf{me{\ꞌ}e} \textbf{potupotu} \textbf{niuniu} \textbf{taratara} {\ob}\textbf{haro} \textbf{{\ꞌ}āua}\,{\cb} ena e aŋa era  te me{\ꞌ}e vivi rikiriki.\\
\textsc{ins} thing piece:\textsc{red} wire:\textsc{red} spine:\textsc{red} {\db}pull enclosure \textsc{med} \textsc{ipfv} make \textsc{dist}  \textsc{art} thing chain small:\textsc{pl}:\textsc{red}\\

\glt 
‘With pieces of barbed fence wire they made little chains.’ \textstyleExampleref{[R364.005]} 
\z

It is also possible to incorporate the verb into the noun which is semantically its object. These compounds are unusual in that the noun is syntactically the head of the compound (it retains its status as a regular noun, i.e. head of a noun phrase), even though it is semantically an argument of the verb.\footnote{\label{fn:272}For a somewhat similar mismatch between syntax and semantics, cf. the nominal purpose construction discussed in \sectref{sec:11.6.3}. There as well as here, an event is expressed by a nominal construction, with one of the arguments of the verb in question as syntactic head.
Both of these are among the many instances in Rapa Nui where a nominal construction serves to express an event (\sectref{sec:3.2.5}).} These compounds may appear in any nominal context, just like any noun or noun compound\is{Compound}. (In (\ref{ex:5.128}–\ref{ex:5.129}), the compound\is{Compound} is the predicate of a nominal clause\is{Clause!nominal}.)

\ea\label{ex:5.128}
\gll {\ꞌ}I tō{\ꞌ}ona mahana he ai mai te aŋa he \textbf{{\ꞌ}āua} \textbf{titi},  {\ꞌ}o he \textbf{rau} \textbf{kato}.\\
at \textsc{poss.3sg.o} day \textsc{ntr} exist hither \textsc{art} work \textsc{pred} enclosure build  or \textsc{pred} leaf pick\\

\glt 
‘On certain days there were jobs like making fences or picking leaves.’ \textstyleExampleref{[R380.084]} 
\z

\ea\label{ex:5.129}
\gll He \textbf{kai} \textbf{toke} nō mai o te taŋata te aŋa. \\
\textsc{pred} food steal just hither of \textsc{art} man \textsc{art} do \\

\glt
‘Stealing the people’s food was what she did all the time.’{\rmfnm} \textstyleExampleref{[R368.017]}  
\z
\footnotetext{The noun \textit{kai} has a genitive modifier \textit{o te taŋata}; this is leapfrogged over by the incorporated verb. The construction is similar to nominal purpose constructions (\sectref{sec:11.6.3}).}

Noun + verb compounds are similar to bare relative clauses\is{Clause!relative!bare} (\sectref{sec:11.4.5}): in the latter, the verb – which is always initial in relative clauses\is{Clause!relative} – is not preceded by an aspectual; just as in a compounds, it follows immediately after the head noun. There are two important differences, however. 

In the first place, a bare relative clause\is{Clause!relative!bare} is still a clause: the verb is part of a verb phrase which may contain postverbal particles, such as \textit{iho} in \REF{ex:5.130}. Moreover, arguments of the verb may be expressed by independent case-marked noun phrases, such as the subject \textit{e ia} (with agentive marking) in \REF{ex:5.131}. 

\ea\label{ex:5.130}
\gll He aŋa i te paepae e tahi {\ꞌ}i tu{\ꞌ}a o tō{\ꞌ}ona hare {\ꞌ}āpī \textbf{aŋa} \textbf{iho}.\\
\textsc{ntr} make \textsc{acc} \textsc{art} shack \textsc{num} one at back of \textsc{poss.3sg.o} house new do just\_then\\

\glt 
‘He built a shelter behind his new house he had just built.’ \textstyleExampleref{[R250.131]} 
\z

\ea\label{ex:5.131}
\gll He {\ꞌ}amo tahi mai ia i tū ŋā kai \textbf{haka} \textbf{rito} \textbf{era} \textbf{e} \textbf{ia}.\\
\textsc{ntr} carry all hither then \textsc{acc} \textsc{dem} \textsc{pl} food \textsc{caus} ready \textsc{dist} \textsc{ag} \textsc{3sg}\\

\glt
‘He carried all that food he had prepared.’ \textstyleExampleref{[R304.078]} 
\z

By contrast, a modifying verb in a compound\is{Compound} does not form a clause. No other VP elements can be included. 

Secondly, a bare relative clause\is{Clause!relative!bare} expresses an event which happens or happened at a specific time, whether once or repeatedly. By contrast, an incorporated verb denotes something which characterises the noun, irrespective of whether the event has really taken place or not. For example, a plot of land may be \textit{{\ꞌ}āua {\ꞌ}oka kai} (garden for planting food, (125)d), even when nothing has been planted yet. 
\is{Object!incorporation|)}

\subsubsection[Compound verbs]{Compound verbs}\label{sec:5.7.2.4}

Though the vast majority of compounds in Rapa Nui function as nouns, compound verbs are also found. Some of these were mentioned in \sectref{sec:5.7.2.1}, e.g. the lexical compound\is{Compound} \textit{tunuahi} (cook + fire) ‘to roast on a fire’. 

Most compound verbs consist of a verb + noun. The noun may have various semantic roles in relation to the verb; interestingly, it is usually not the direct object, but often the instrument with which the action is performed:

\ea\label{ex:5.132}
\gll He to{\ꞌ}o mai era he \textbf{tunu} \textbf{pani}, he \textbf{tunuahi}. \\
\textsc{ntr} take hither \textsc{dist} \textsc{ntr} cook pan \textsc{ntr} cook.fire \\

\glt 
‘They took the food and cooked it in the pan, roasted it on a fire.’ \textstyleExampleref{[R107.049]} 
\z

\ea\label{ex:5.133}
\gll He \textbf{tunu} \textbf{mā{\ꞌ}ea} \textbf{vera}, haka hopu i te poki hai vai vera. \\
\textsc{ntr} cook stone hot \textsc{caus} bathe \textsc{acc} \textsc{art} child \textsc{ins} water hot \\

\glt
‘He cooked (the water) with hot rocks, and bathed the child with hot water.’ \textstyleExampleref{[Mtx-1-07.016]}
\z

In the following example, the modifier \textit{rapa nui} can also be considered as an instrument in a loose sense.

\ea\label{ex:5.134}
\gll {\ꞌ}E nu{\ꞌ}u ta{\ꞌ}e rahi {\ꞌ}i te ra{\ꞌ}ā nei e \textbf{{\ꞌ}aroha} \textbf{rapa} \textbf{nui} nei. \\
and people \textsc{conneg} many at \textsc{art} day \textsc{prox} \textsc{ipfv} greet Rapa Nui \textsc{prox} \\

\glt
‘Few people today greet each other in Rapa Nui (with this Rapa Nui greeting).’ \textstyleExampleref{[R530.038]} 
\z

That these combinations are compounds is clear from the fact that the noun is not preceded by a determiner, nor by a preposition indicating its semantic role. (For example, the instrumental role would normally be indicated by \textit{hai}.) Also, postverbal particles follow the noun (\textit{nei} in \REF{ex:5.134} above), showing that the noun has been incorporated into the verb phrase.
\is{Compound|)}
\subsection{Modifying adjectives}\label{sec:5.7.3}
\is{Adjective!in noun phrase|(}
As discussed in \sectref{sec:5.7.1}, modifying adjectives are usually semantically different from modifying nouns. This section discusses a few issues concerning adjectives in the noun phrase.

Several elements occurring in the adjective position are discussed elsewhere:

\begin{itemize}
\item 
the ordinal\is{Numeral!ordinal} numeral \textit{ra{\ꞌ}e} ‘first’ (\sectref{sec:4.3.3})

\item 
the interrogative adjective \textit{h}\textit{ē}\is{he (content question marker)@hē (content question marker)} ‘which’ (\sectref{sec:10.3.2.3})

\item 
the quantifying\is{Quantifier} element \textit{rahi}\is{rahi ‘much/many’} ‘much, many’ (\sectref{sec:4.4.7})

\item 
the noun negator \textit{kore} ‘without; lack of’ (\sectref{sec:10.5.7})

\end{itemize}
\subsubsection[Multiple adjectives]{Multiple adjectives}\label{sec:5.7.3.1}

As \REF{ex:5.109} shows, the noun phrase may contain more than one adjective. The order of the adjectives is not fixed:

\ea\label{ex:5.135}
\gll He aŋa i te hare \textbf{teatea} \textbf{nuinui}. \\
\textsc{ntr} make \textsc{acc} \textsc{art} house white:\textsc{red} big:\textsc{red} \\

\glt 
‘He built a big white house.’ \textstyleExampleref{[Notes]}
\z

\ea\label{ex:5.136}
\gll He aŋa i te hare \textbf{nuinui} \textbf{teatea}. \\
\textsc{ntr} make \textsc{acc} \textsc{art} house big:\textsc{red} white:\textsc{red} \\

\glt
‘He built a big house, which was white.’ \textstyleExampleref{[Notes]}
\z

As the translation shows, there is a subtle difference between the two examples above. The adjective closest to the noun denotes the quality that is most fundamental in the context; this noun + adjective combination is in turn modified by the second adjective. This is in line with the general principle noted in \sectref{sec:5.7.1}: elements closest to the noun are semantically closer to it as well; they form a unit with the noun which may in turn be modified by other modifiers.

Cases of multiple adjectives are uncommon, though. The contrasting examples above were given during a discussion session. An example from the text corpus is the following:

\ea\label{ex:5.137}
\gll He u{\ꞌ}i mai i te {\ꞌ}ohe \textbf{tītika} \textbf{rivariva}. \\
\textsc{ntr} look hither \textsc{acc} \textsc{art} bamboo straight good:\textsc{red} \\

\glt
‘You look (= one looks) for a straight, good bamboo stick.’ \textstyleExampleref{[R360.015]} 
\z

More commonly, multiple adjectives are separated by a pause or the conjunction \textit{{\ꞌ}e} ‘and’; for an example, see R215, sentence 02 in Appendix A (p.~\pageref{sec:a.1}).

\subsubsection[The adjective phrase]{The adjective phrase}\label{sec:5.7.3.2}

The adjective constituent which modifies the noun is not always a bare adjective, but can be a phrase containing other elements: adverbs and/or particles.

The adjective may be preceded by a modifier of degree: \textit{{\ꞌ}apa}\is{apa ‘part’@{\ꞌ}apa ‘part’} ‘to a moderate degree, somewhat, sort of’ or \textit{{\ꞌ}ata}\is{ata ‘more’@{\ꞌ}ata ‘more’} ‘to a higher degree, more’. \textit{{\ꞌ}Ata} is discussed in \sectref{sec:3.5.1.1}; here are examples of \textit{{\ꞌ}apa}:

\ea\label{ex:5.138}
\gll he vere \textbf{{\ꞌ}apa} \textbf{meamea} \\
\textsc{pred} beard part red:\textsc{red} \\

\glt 
‘a reddish beard’ \textstyleExampleref{(\citealt[77]{WeberR2003})} 
\z

\ea\label{ex:5.139}
\gll He oti {\ꞌ}ā te henua mā{\ꞌ}ohi \textbf{{\ꞌ}apa} \textbf{hāhine} ki te ŋā henua ena  o te America\_del\_Sur.\\
\textsc{ntr} finish \textsc{cont} \textsc{art} land indigenous part near to \textsc{art} \textsc{pl} land \textsc{med}  of \textsc{art} South\_America\\

\glt 
‘This is the only Polynesian island sort of close to the countries of South America.’ \textstyleExampleref{[R350.003]} 
\z

The adjective may be followed by an intensifying adverb\is{Adverb} \textit{rahi} ‘much’, \textit{\mbox{ri{\ꞌ}ari{\ꞌ}a}} ‘very, terribly’, \textit{taparahi-ta{\ꞌ}ata} ‘terribly’ (a \ili{Tahitian} phrase which literally means ‘killing people’), \textit{tano} ‘in a moderate degree’, \textit{mau} (in \REF{ex:5.142}, emphasised by the identity particle \textit{{\ꞌ}ā}):

\ea\label{ex:5.140}
\gll Me{\ꞌ}e \textbf{rivariva} \textbf{rahi} te me{\ꞌ}e nei mo te orara{\ꞌ}a o te mahiŋo o Rapa Nui. \\
thing good:\textsc{red} much \textsc{art} thing \textsc{prox} for \textsc{art} life of \textsc{art} people of Rapa Nui \\

\glt 
‘This was something very good for the life of the people of Rapa Nui.’ \textstyleExampleref{[R231.314]} 
\z

\ea\label{ex:5.141}
\gll Pē he {\ꞌ}ariki te ha{\ꞌ}aaura{\ꞌ}a, me{\ꞌ}e \textbf{mo{\ꞌ}a} \textbf{ri{\ꞌ}ari{\ꞌ}a}. \\
like \textsc{pred} king \textsc{art} example thing respect very \\

\glt 
‘The king, for example, was very sacred.’ \textstyleExampleref{[R371.009]} 
\z

\ea\label{ex:5.142}
\gll {\ꞌ}I nei i nonoho i u{\ꞌ}i ai kāiŋa \textbf{rivariva} \textbf{mau} \textbf{{\ꞌ}ā}. \\
at \textsc{prox} \textsc{pfv} \textsc{pl}:stay \textsc{pfv} look \textsc{pvp} homeland good really \textsc{ident} \\

\glt
‘Here they stayed and saw that it was a really good country.’ \textstyleExampleref{[R420.054]} 
\z

The adjective may also be followed by a prepositional phrase as in \REF{ex:5.143}:

\ea\label{ex:5.143}
\gll {\ꞌ}I Haŋa Roa te nonoho haŋa, {\ꞌ}i te kona {\ob}\textbf{hāhine}   {\ob}\textbf{ki} \textbf{te} \textbf{{\ꞌ}ōpītara} \textbf{tuai} \textbf{era}\,{\cb}\,{\cb}.\\
at Hanga Roa \textsc{art} \textsc{pl}:stay \textsc{nmlz} at \textsc{art} place {\db}near {\db}to \textsc{art} hospital ancient \textsc{dist}\\

\glt 
‘They lived in Hanga Roa, in a place close to the old hospital.’ \textstyleExampleref{[R380.003]} 
\z\is{Adjective!in noun phrase|)}\is{Noun!as modifier|)}
\section{Adverbs and \textit{nō} in the noun phrase}\label{sec:5.8}
\subsection{Adverbs}\label{sec:5.8.1}
\is{Adverb!in noun phrase}
As the position chart in \sectref{sec:5.1} shows, after the quantifier\is{Quantifier} phrase there is a position for adverbs. The only adverbs found here are \textit{haka{\ꞌ}ou}\is{haka{\ꞌ}ou ‘again’} ‘again’, \textit{tako{\ꞌ}a}\is{tako{\ꞌ}a ‘also’} ‘also’ and \textit{mau}\is{mau ‘really’} ‘really’. \textit{Haka{\ꞌ}ou} and \textit{tako{\ꞌ}a} are more common in verb phrases, but do appear in noun phrases occasionally; they are discussed in sections \sectref{sec:4.5.3.4} and \sectref{sec:4.5.3.2}, respectively. \textit{Mau} may co-occur with another adverb\is{Adverb} (just as in the verb phrase, \sectref{sec:4.5.1}), hence its separate slot in the noun phrase chart in \sectref{sec:5.1}.

\subsection{The limitative marker \textit{nō}}\label{sec:5.8.2}
\is{no ‘just’@nō ‘just’!in noun phrase|(}
\textit{Nō} is a marker of limitation, which is also common in the verb phrase (\sectref{sec:7.4.1}). In the noun phrase, \textit{nō} has a number of uses. In several constructions it serves to restrict the reference of a noun phrase, though – as will be illustrated below – not necessarily the noun phrase it occurs in. In other cases it is used in the sense ‘just, simply’ in much the same way as in verb phrases.

\subsubsection{‘The only one’}\label{sec:5.8.2.1}

In initial subject NPs, \textit{nō} indicates that the set referred to by the noun phrase has only one entity, viz. the one described in the rest of the sentence. The sentence can be paraphrased as: ‘There is only one [NP], and that is [rest of sentence]’, or more simply: ‘[rest of sentence] is the only [NP].’ For example in \REF{ex:5.144}: ‘There was only one thing on board, and that was a piece of pumpkin’, or ‘A piece of pumpkin was the only thing on board.’ 

\ea\label{ex:5.144}
\gll \textbf{Te} \textbf{me{\ꞌ}e} \textbf{nō} o ruŋa, he parehe mautini, he oti mau nō. \\
the thing just of above \textsc{pred} piece pumpkin \textsc{ntr} finish really just \\

\glt 
‘The only thing (they had with them) on board was a piece of pumpkin, that was all.’ \textstyleExampleref{[R303.054]} 
\z

\ea\label{ex:5.145}
\gll {\ꞌ}E \textbf{tō{\ꞌ}ona} \textbf{{\ꞌ}īŋoa} \textbf{nō} pa{\ꞌ}i i {\ꞌ}ite era e tātou ko Sebastián Englert. \\
and \textsc{poss.3sg.o} name just in\_fact \textsc{pfv} know \textsc{dist} \textsc{ag} \textsc{1pl.incl} \textsc{prom} Sebastián Englert \\

\glt 
‘And the only name we knew him by, was Sebastián Englert.’ \textstyleExampleref{[R375.005]} 
\z

\subsubsection{‘Only that one’}\label{sec:5.8.2.2}

With noun phrases in other positions, \textit{nō} signals that the rest of the sentence applies only to the entities described by the noun phrase with \textit{nō}. The sentence can be paraphrased as: ‘only for [NP] is it true that [rest of sentence]’. For example in \REF{ex:5.146}: ‘Only for lobster and crab is it true that they fished with it’; in other words: ‘Lobster and crabs were the only (bait) they fished with.’

\ea\label{ex:5.146}
\gll Te taŋata o nei e hī era \textbf{hai} \textbf{{\ꞌ}ura} \textbf{nō} rāua ko te pīkea. \\
\textsc{art} man of \textsc{prox} \textsc{ipfv} to\_fish \textsc{dist} \textsc{ins} lobster just \textsc{3pl} \textsc{prom} \textsc{art} crab \\

\glt 
‘The people here used to fish only with lobster and crab.’ \textstyleExampleref{[R354.029]} 
\z

\ea\label{ex:5.147}
\gll \textbf{{\ꞌ}I} \textbf{te} \textbf{pō} \textbf{nō} te ika nei ana hī. \\
at \textsc{art} night just \textsc{art} fish \textsc{prox} \textsc{irr} to\_fish \\

\glt
‘Only at night this fish can be fished.’ \textstyleExampleref{[R364.007]} 
\z

This is also common with \textit{nō} in predicate noun phrases. \textit{Nō} indicates that there is only one entity to which the subject applies, viz. the one referred to in the noun phrase containing \textit{nō}. The sentence can be paraphrased as: ‘Only [predicate] is [subject]’, or more naturally: ‘[predicate] is the only [subject].’ This happens for example in the identifying clause\is{Clause!identifying} (\sectref{sec:9.2.2}) in \REF{ex:5.148} below: ‘Only she was the new child inside’ = ‘She was the only new child inside.’ 

\ea\label{ex:5.148}
\gll \textbf{Ko} \textbf{ia} \textbf{nō} te poki {\ꞌ}āpī o roto. \\
\textsc{prom} \textsc{3sg} just \textsc{art} child new of inside \\

\glt 
‘She was the only new child inside (the class).’ \textstyleExampleref{[R151.020]} 
\z

\subsubsection{‘Just’}\label{sec:5.8.2.3}

In all cases above, \textit{nō} limits the reference of a noun phrase. It may also have a weaker sense: ‘just, simply, no more than’:

\ea\label{ex:5.149}
\gll \textbf{He} \textbf{tāvini} \textbf{nō} māua ō{\ꞌ}ou. \\
\textsc{pred} servant just \textsc{1du.excl} \textsc{poss.2sg.o} \\

\glt 
‘We are just your slaves.’ \textstyleExampleref{[R214.015]} 
\z

\ea\label{ex:5.150}
\gll \textbf{He} \textbf{repahoa} \textbf{nō} au ō{\ꞌ}ou. \\
\textsc{pred} friend just \textsc{1sg} \textsc{poss.2sg.o} \\

\glt 
‘I am just your friend.’ \textstyleExampleref{[R308.032]} 
\z

\subsubsection[Contrastive use]{Contrastive use}\label{sec:5.8.2.4}

\textit{Nō} is used in a number of expressions indicating a contrast. The noun phrase \textit{te N nō}, placed initially in the clause, functions as a connective which signals that the following clause is an exception to what has been stated before. An appropriate translation is ‘however’. The noun may express how this contrast is to be evaluated, whether negatively as in \REF{ex:5.151}, positively as in \REF{ex:5.152}, or neutral as in \REF{ex:5.153}. In \REF{ex:5.151}, the contrast is reinforced with the \ili{Spanish} conjunction\is{Conjunction} \textit{pero}.

\ea\label{ex:5.151}
\gll Pero \textbf{te} \textbf{{\ꞌ}ino} \textbf{nō}, {\ꞌ}ina e tahi materiare mo aŋa.\\
but \textsc{art} bad just \textsc{neg} \textsc{num} one material for make\\

\glt 
‘(He wanted to build a house.) But unfortunately (=the problem was), there were no building materials.’ \textstyleExampleref{[R231.156]} 
\z

\ea\label{ex:5.152}
\gll \textbf{Te} \textbf{riva} \textbf{nō}, e ta{\ꞌ}ero era, {\ꞌ}ina he tiŋa{\ꞌ}i i tā{\ꞌ}ana hua{\ꞌ}ai. \\
\textsc{art} good just \textsc{ipfv} drunk \textsc{dist} \textsc{neg} \textsc{ntr} strike \textsc{acc} \textsc{poss.3sg.a} family \\

\glt 
‘(He used to drink.) Fortunately (=the good thing was), when he was drunk, he did not beat his family.’ \textstyleExampleref{[R309.056]} 
\z

\ea\label{ex:5.153}
\gll \textbf{Te} \textbf{me{\ꞌ}e} \textbf{nō,} {\ꞌ}i ruŋa i tū vaka era ō{\ꞌ}ona e ai rō {\ꞌ}ā e tahi pē{\ꞌ}ue, e rua miro {\ꞌ}i te kaokao o te vaka. \\
\textsc{art} thing just at above at \textsc{dem} boat \textsc{dist} \textsc{poss.3sg.o} \textsc{ipfv} exist \textsc{emph} \textsc{cont} \textsc{num} one mat \textsc{num} two wood at \textsc{art} side:\textsc{red} of \textsc{art} boat \\

\glt 
‘(His boat was like the other ones;) however, in his boat there was a rug, and two poles on the sides of the boat.’ \textstyleExampleref{[R344.040]} 
\z
\is{no ‘just’@nō ‘just’!in noun phrase|)}

\section{The identity marker \textit{{\ꞌ}ā}/\textit{{\ꞌ}ana}}\label{sec:5.9}
\is{a (identity)@{\ꞌ}ā (identity)|(}
\textit{{\ꞌ}Ā} and \textit{{\ꞌ}ana}\is{a (identity)@{\ꞌ}ā (identity)} are variant forms of the same particle.\footnote{\label{fn:274}In other Eastern Polynesian languages, cognates of \textit{{\ꞌ}ana} are used in the verb phrase, but not in the noun phrase (see Footnote \ref{fn:328} on p.~\pageref{fn:328}).} This particle functions as a continuous marker in the verb phrase and as an identity marker in the noun phrase. This section deals with its use in the noun phrase; its use in the verb phrase is discussed in \sectref{sec:7.2.5.5}.

The choice between \textit{{\ꞌ}ā} and \textit{{\ꞌ}ana} is partly a stylistic one. \textit{{\ꞌ}Ā} is somewhat more informal (and therefore more common in oral language), while \textit{{\ꞌ}ana} is more formal. Rhythm may also play a role: in some contexts a one-syllable\is{Syllable} particle may yield a better rhythm than a two-syllable\is{Syllable} one, or the opposite. 

Other euphonic effects may play a role as well. For example, after the particle \textit{ena}, one usually finds \textit{{\ꞌ}ā}, not \textit{{\ꞌ}ana}: the alliterating \textit{ena {\ꞌ}ana} is avoided.\footnote{\label{fn:275}By contrast, after \textit{era} both \textit{{\ꞌ}ā} and \textit{{\ꞌ}ana} are commonly used.}

Part of the difference is ideolectical, as shown by the fact that some (groups of) texts show a strong preference for one variant. For example, in Ley \textit{{\ꞌ}ā} is about six times as common as \textit{{\ꞌ}ana} (296 against 58 occurrences), while in MsE \textit{{\ꞌ}ana} is predominant (121 against 23 occurrences). One recent text (R539) shows an extraordinary preference for \textit{{\ꞌ}ana} (557x \textit{{\ꞌ}ana} against 30x \textit{{\ꞌ}ā}), while some oral texts use \textit{{\ꞌ}ā} almost exclusively. In most texts, however, the two occur in more equal proportions, though \textit{{\ꞌ}ā} is more common overall.

Concerning the use of \textit{{\ꞌ}ā}/\textit{{\ꞌ}ana}: with a pronoun it may be used when the pronoun has a \textsc{reflexive}\is{Reflexive} sense, i.e. is coreferential with the subject of the clause. The pronoun may be, for example, the direct object or an oblique argument:

\ea\label{ex:5.154}
\gll Ko ri{\ꞌ}ari{\ꞌ}a {\ꞌ}ana {\ꞌ}i tū māuiui era ō{\ꞌ}ona e ma{\ꞌ}u era {\ꞌ}i roto  \textbf{i} \textbf{a} \textbf{ia} \textbf{{\ꞌ}ā}.\\
\textsc{prf} afraid \textsc{cont} \textsc{acc} \textsc{dem} sick \textsc{dist} \textsc{poss.3sg.o} \textsc{ipfv} carry \textsc{dist} at inside  at \textsc{prop} \textsc{3sg} \textsc{ident}\\

\glt 
‘She was afraid of the sickness she carried inside herself.’ \textstyleExampleref{[R301.091]} 
\z

\ea\label{ex:5.155}
\gll He noho {\ꞌ}i ruŋa i te mā{\ꞌ}ea e tahi, he kī \textbf{ki} \textbf{a} \textbf{ia} \textbf{{\ꞌ}ā}... \\
\textsc{ntr} sit at above at \textsc{art} stone \textsc{num} one \textsc{ntr} say to \textsc{prop} \textsc{3sg} \textsc{ident} \\

\glt
‘He sat down on a stone and said to himself...’ \textstyleExampleref{[R229.365]} 
\z

However, \textit{{\ꞌ}ā}/\textit{{\ꞌ}ana} as such is not a reflexive\is{Reflexive} marker: a noun phrase containing \textit{{\ꞌ}ā} does not need to be in the same clause as its antecedent. In the following example, \textit{{\ꞌ}ā} appears with a subject pronoun, coreferential with the subject of the preceding sentence: 

\ea\label{ex:5.156}
\gll He kī atu ia e tō{\ꞌ}oku koro era ki a au... {\ꞌ}Ai ka kī  haka{\ꞌ}ou atu \textbf{e} \textbf{ia} \textbf{{\ꞌ}ā}...\\
\textsc{ntr} say away then \textsc{ag} \textsc{poss.1sg.o} Dad \textsc{dist} to \textsc{prop} \textsc{1sg} there \textsc{cntg} say  again away \textsc{ag} \textsc{3sg} \textsc{ident}\\

\glt
‘Then my uncle (lit. father) said to me.... Then he himself said again...’ \textstyleExampleref{[R230.254-6]}
\z

It is more accurate to analyse \textit{{\ꞌ}ā/{\ꞌ}ana} in broader terms: it serves as a marker of \textsc{identity}. As such, it can be used in different ways. Sometimes it indicates that the referent of the noun phrase is identical to another referent in the same clause, as in the reflexive\is{Reflexive} examples (\ref{ex:5.154}–\ref{ex:5.155}) above. In other cases it indicates that the referent of the noun phrase is identical to another referent mentioned earlier in the text, as in \REF{ex:5.156}. It may also underline that the referent is identical to an entity known in some other way (‘the same’). Some examples:

\ea\label{ex:5.157}
\gll I oti era te kai, he ha{\ꞌ}uru rō {\ꞌ}ai a Taparahi {\ꞌ}i tū kona era \textbf{{\ꞌ}ā}. \\
\textsc{pfv} finish \textsc{dist} \textsc{art} eat \textsc{ntr} sleep \textsc{emph} \textsc{subs} \textsc{prop} Taparahi at \textsc{dem} place \textsc{dist} \textsc{ident} \\

\glt 
‘When he had finished eating, Taparahi slept at that same place.’ \textstyleExampleref{[R250.032]} 
\z

\ea\label{ex:5.158}
\gll I poreko ai a ia {\ꞌ}i te motu mau nei \textbf{{\ꞌ}ā} {\ꞌ}i te matahiti 1922. \\
\textsc{pfv} born \textsc{pvp} \textsc{prop} \textsc{3sg} at \textsc{art} islet really \textsc{prox} \textsc{ident} at \textsc{art} year 1922 \\

\glt
‘He was born on this very same island here in the year 1922.’ \textstyleExampleref{[R487.041]} 
\z

In \REF{ex:5.157}, the place where Taparahi sleeps is the same place where he has just eaten. In \REF{ex:5.158}, the island where the person in question is born is the same island where the story is being told. 

These examples also illustrate the syntax of \textit{{\ꞌ}ā}/\textit{{\ꞌ}ana}: when \textit{{\ꞌ}ā}/\textit{{\ꞌ}ana} follows a noun, the noun phrase also has a demonstrative: usually prenominal (\textit{tū} in \REF{ex:5.157}), occasionally postnominal (\textit{nei} in \REF{ex:5.158}). When \textit{{\ꞌ}ana} follows a pronoun, no demonstrative is used. 

After a possessive pronoun, \textit{{\ꞌ}ā} (often preceded by \textit{mau}) stresses the identity of the possessor\is{Possession}: ‘one’s own’.

\ea\label{ex:5.159}
\gll ¿E ai rō {\ꞌ}ā tu{\ꞌ}u vaka \textbf{ō{\ꞌ}ou} \textbf{mau} \textbf{{\ꞌ}ā}? \\
~\textsc{ipfv} exist \textsc{emph} \textsc{ident} \textsc{poss.2sg.o} boat \textsc{poss.2sg.o} really \textsc{ident} \\

\glt 
‘Do you have your own boat?’ \textstyleExampleref{[Notes]}
\z

One more nominal construction in which \textit{{\ꞌ}ā/{\ꞌ}ana} is used, is \textit{ko te V iŋa {\ꞌ}ā/{\ꞌ}ana} (\sectref{sec:3.2.3.1.1}). 
\is{a (identity)@{\ꞌ}ā (identity)|)}

\section{The deictic particle \textit{ai}}\label{sec:5.10}
\is{ai (deictic)|(}
The deictic particle \textit{ai} is used when pointing at something; it can only be used when the entity referred to is visible.

\ea\label{ex:5.160}
\gll ¿O hua{\ꞌ}ai hē te rū{\ꞌ}au era \textbf{ai}? \\
~of family \textsc{cq} \textsc{art} old\_woman \textsc{dist} there \\

\glt 
‘Of which family is that old woman over there?’ \textstyleExampleref{[R413.305]} 
\z

\ea\label{ex:5.161}
\gll {\ꞌ}Ai {\ꞌ}ō te me{\ꞌ}e pē Mariana {\ꞌ}ā \textbf{ai}. \\
there really \textsc{art} thing like Mariana \textsc{ident} there \\

\glt
‘There is someone (who looks) like Mariana.’ \textstyleExampleref{[R415.423]} 
\z

As these examples show, \textit{ai} is usually preceded by a postnominal demonstrative\is{Demonstrative!postnominal} (\textit{era}, \textit{nei} or \textit{ena}) or an identity marker (\textit{{\ꞌ}ā} or \textit{{\ꞌ}ana}). 

This particle is similar in function to the sentence-initial particle \textit{{\ꞌ}ai}\is{ai (deictic)@{\ꞌ}ai (deictic)} ‘there is’. Hhe particles are phonetically different, however: NP{}-final \textit{ai} has no glottal\is{Glottal plosive}, while initial \textit{{\ꞌ}ai} does. Even so, the two could be etymologically related (\sectref{sec:2.2.5} on glottals\is{Glottal plosive} in particles). Another possibility is that final \textit{ai} has developed from the existential verb \textit{ai}. This verb is used postnominally to construct certain types of relative clauses\is{Clause!relative} (\sectref{sec:11.4.3}):

\ea\label{ex:5.162}
\gll te nu{\ꞌ}u \textbf{ai} o te vaka\\
\textsc{art} people exist of \textsc{art} boat\\

\glt
‘the people who had a boat’ \textstyleExampleref{[R200.086]} 
\z
	
It is conceivable that the deictic particle \textit{ai} developed from a relative clause\is{Clause!relative} which was truncated, and of which only the verb was left.
\is{ai (deictic)|)}

\section{Heavy shift}\label{sec:5.11}
\is{Heavy shift|(}
Sometimes longer subphrases are placed at the end of the noun phrase. This is in accordance with a universal tendency to move long constituents to the end of the phrase or clause, a phenomenon known as heavy shift\is{Heavy shift} \citep[326]{Payne1997}.

In \REF{ex:5.163} below, the noun is modified by a complex adjective phrase ‘smaller than it’. The adjective itself is in its normal position, but its complement \textit{ki a ia} ‘than it’, which expresses the standard of comparison, is placed after the postnominal demonstrative\is{Demonstrative!postnominal} \textit{era}. In \REF{ex:5.164}, the whole adjective phrase is placed at the end of the noun phrase, even after the relative clause\is{Clause!relative}:

\ea\label{ex:5.163}
\gll He take{\ꞌ}a ta{\ꞌ}ato{\ꞌ}a mai e tāua te ta{\ꞌ}ato{\ꞌ}a ma{\ꞌ}uŋa \textbf{{\ꞌ}ata} \textbf{rikiriki} era  \textbf{ki} \textbf{a} \textbf{ia}.\\
\textsc{ntr} see all hither \textsc{ag} \textsc{1du.incl} \textsc{art} all hill more small:\textsc{pl}:\textsc{red} \textsc{dist}  to \textsc{prop} \textsc{3sg}\\

\glt 
‘We will also see all the mountains smaller than it (=Terevaka).’ \textstyleExampleref{[R314.002]} 
\z

\ea\label{ex:5.164}
\gll {\ꞌ}I tū hora era ia i u{\ꞌ}i atu ai a Kālia ko te \textbf{me{\ꞌ}e} \textbf{teatea} \textbf{e} \textbf{tahi} {\ob}e take{\ꞌ}a mai era mai ruŋa i tū pahī era\,{\cb} \textbf{{\ꞌ}ata} \textbf{nuinui} \textbf{ki} \textbf{te} \textbf{taŋata} \textbf{e} \textbf{tahi}.\\
at \textsc{dem} time \textsc{dist} then \textsc{pfv} look away \textsc{pvp} \textsc{prop} Kalia \textsc{prom} \textsc{art} thing white:\textsc{red}  \textsc{num} one {\db}\textsc{ipfv} see hither \textsc{dist} from above at \textsc{dem} ship \textsc{dist} more big:\textsc{red} to \textsc{art} man \textsc{num} one\\

\glt 
‘At that moment Kalia saw something white, which was seen on the ship, bigger than a man.’ \textstyleExampleref{[R345.061]}\textstyleExampleref{} 
\z
\is{Heavy shift|)}
\section{Appositions}\label{sec:5.12}
\is{Apposition|(}\subsection{Common nouns in apposition}\label{sec:5.12.1}
\is{Noun!common}\is{Apposition}
Common noun phrases\is{Noun!common} in apposition\is{Apposition} are never preceded by a \textit{t}{}-determiner. They may be marked in several ways: without any marker (bare appositions\is{Apposition}), with the predicate marker \textit{he}, or with the prominence marker \textit{ko}.

\subparagraph{Bare appositions} Bare appositions\is{Apposition} may have generic reference, indicating that the head noun belongs to a certain class of referents. In \REF{ex:5.165}, the apposition tells that Renga Roiti belongs to the class of female children. 

\ea\label{ex:5.165}
\gll He poreko ko Reŋa Roiti, \textbf{poki} tamahahine. \\
\textsc{ntr} born \textsc{prom} Renga Roiti child female \\

\glt 
‘Renga Roiti, a girl, was born.’ \textstyleExampleref{[Mtx-7-15.002]}
\z

\ea\label{ex:5.166}
\gll {\ꞌ}I ira e noho era tū taŋata era, \textbf{taŋata} keukeu henua {\ꞌ}oka kai. \\
at \textsc{ana} \textsc{ipfv} stay \textsc{dist} \textsc{dem} man \textsc{dist} man labour:\textsc{red} land plant food \\

\glt
‘There that man lived, a farmer who planted crops.’ \textstyleExampleref{[R372.036]} 
\z

They may also have \is{Specific reference}specific reference, identifying the head noun with a certain referent. For example, the apposition\is{Apposition} in \REF{ex:5.167} tells that Papeete is the same place as the capital of Tahiti.

\ea\label{ex:5.167}
\gll te kona ko Pape{\ꞌ}ete, \textbf{kona} \textbf{rarahi} o Tahiti \\
\textsc{art} place \textsc{prom} Papeete place important of Tahiti \\

\glt 
‘the city of Papeete, the capital of Tahiti’ \textstyleExampleref{[R231.045]} 
\z

\subparagraph{\textit{He}-marked appositions} Appositions introduced by \textit{he}\is{he (nominal predicate marker)} may also be either specific\is{Specific reference} as in \REF{ex:5.168} or generic as in \REF{ex:5.169}. In the Bible translation, appositions tend to be marked with \textit{he}, possibly because the translation employs a relatively polished/formal style.

\ea\label{ex:5.168}
\gll He oho ki Vērene {\ꞌ}i Hūrea, \textbf{he} \textbf{kona} poreko o tō{\ꞌ}ona hakaara  ko Tāvita.\\
\textsc{ntr} go to Bethlehem at Judea \textsc{pred} place born of \textsc{poss.3sg.o} ancestor  \textsc{prom} David\\

\glt 
‘They went to Bethlehem in Judea, the birth place of his ancestor David.’ \textstyleExampleref{[Luk. 2:4]}
\z

\ea\label{ex:5.169}
\gll He haka hāhine rā nu{\ꞌ}u ki a Feripe, \textbf{he} \textbf{kope} o Vetetaira. \\
\textsc{ntr} \textsc{caus} near \textsc{dist} people to \textsc{prop} Philip \textsc{pred} person of Bethsaida \\

\glt 
‘Those people approached Philip, a man from Bethsaida.’ \textstyleExampleref{[John 12:21]}
\z

Bare and \textit{he}{}-marked appositions\is{Apposition} are used as the equivalent of nonrestrictive relative clauses\is{Clause!relative}, clauses which provide information about a noun phrase without limiting its reference.\footnote{\label{fn:276}\citet[207]{Andrews2007Relative} does not consider nonrestrictive clauses as relative clauses\is{Clause!relative}, as relative clauses\is{Clause!relative} (in his definition) delimit the reference of the noun phrase.} In Rapa Nui, relative clauses\is{Clause!relative} must be restrictive, and therefore they cannot be attached to nouns which already have a unique reference, like proper names. To add a clause\is{Clause!relative} providing more information to such a noun, a noun with generic meaning (e.g. \textit{me{\ꞌ}e} ‘thing’, \textit{kope} ‘person’) is placed in apposition\is{Apposition}; a relative clause\is{Clause!relative} is attached to this apposition\is{Apposition}, limiting the reference of the generic noun:

\ea\label{ex:5.170}
\gll He oho ia a Vakaiaheva ki Rano Raraku, \textbf{kona} [{\ꞌ}i ira te kape  e noho era].\\
\textsc{ntr} go then \textsc{prop} Vakaiaheva to Rano Rarako place ~at \textsc{ana} \textsc{art} boss  \textsc{ipfv} stay \textsc{dist}\\

\glt 
‘Vakaiaheva went to Rano Raraku, the place where the boss lived.’ \textstyleExampleref{[R440.028]} 
\z

\ea\label{ex:5.171}
\gll He turu a Rovi, \textbf{he} \textbf{taŋata} [hāpa{\ꞌ}o i te poki {\ꞌ}a Hotu {\ꞌ}ariki]. \\
\textsc{ntr} go\_down \textsc{prop} Rovi \textsc{ntr} person ~care\_for \textsc{acc} \textsc{art} child of\textsc{.a} Hotu king \\

\glt 
‘Rovi came down, the man who took care of the child of king Hotu.’ \textstyleExampleref{[R422.002]} 
\z

\subparagraph{\textit{Ko}-marked appositions} Sometimes a common noun apposition\is{Apposition} is marked by the prominence marker \textit{ko}\is{ko (prominence marker)} (\sectref{sec:4.7.11}), followed by a determiner.\footnote{\label{fn:277}\textit{Ko} in appositions\is{Apposition} is common in Polynesian languages, see \citet[45]{Clark1976}.} This happens when the apposition\is{Apposition} refers to an entity uniquely identifiable\is{Identifiability} by the hearer (cf. \sectref{sec:9.2.1} on the distinction between \textit{ko-}marked and \textit{he-}marked noun phraes).

\ea\label{ex:5.172}
\gll He tu{\ꞌ}u mai te {\ꞌ}avione ra{\ꞌ}e \textbf{ko} \textbf{te} \textbf{{\ꞌ}avione} ena e kī ena he DC 10. \\
\textsc{ntr} arrive hither \textsc{art} airplane first \textsc{prom} \textsc{art} airplane \textsc{med} \textsc{ipfv} say \textsc{med} \textsc{pred} DC 10 \\

\glt 
‘The first airplane, the airplane called DC 10, arrived.’ \textstyleExampleref{[R203.062]} 
\z

\ea\label{ex:5.173}
\gll te mahiŋo i haka maraŋa ena {\ꞌ}i ruŋa i te henua nei \textbf{ko} \textbf{te} \textbf{kāiŋa} \\
\textsc{art} people \textsc{pfv} \textsc{caus} scattered \textsc{med} at above at \textsc{art} land \textsc{prox} \textsc{prom} \textsc{art} homeland \\

\glt 
‘the people who spread over the land, over the homeland’ \textstyleExampleref{[R350.016]} 
\z

\subsection{Proper nouns in apposition}\label{sec:5.12.2}
\is{Noun!proper}\is{Apposition}
If the apposition\is{Apposition} is a proper noun, it is introduced by \textit{ko}\is{ko (prominence marker)}. This is to be expected, as proper nouns\is{Noun!proper} are inherently uniquely identifiable\is{Identifiability} in a given context (\sectref{sec:9.2.1}).

\ea\label{ex:5.174}
\gll He oho mai era te {\ꞌ}ariki \textbf{ko} \textbf{Hotu} \textbf{Matu{\ꞌ}a}, he rarama era. \\
\textsc{ntr} go hither \textsc{dist} \textsc{art} king \textsc{prom} Hotu Matu’a \textsc{ntr} inspect \textsc{dist} \\

\glt 
‘King Hotu Matu’a came and examined it.’ \textstyleExampleref{[Mtx-2-02.043]}
\z

\ea\label{ex:5.175}
\gll te kona \textbf{ko} \textbf{Pape{\ꞌ}ete}, kona rarahi o Tahiti \\
\textsc{art} place \textsc{prom} Papeete place important of Tahiti \\

\glt
‘the city of Papeete, the capital of Tahiti’ \textstyleExampleref{[R231.045]} 
\z

The examples above illustrate the most common way to express a combination of a common noun and a name: the common noun is the head noun; the name follows as apposition\is{Apposition}, introduced by \textit{ko}. There are exceptions though: sometimes \textit{ko} is not used as in \REF{ex:5.176}; sometimes the name precedes the common noun as in \REF{ex:5.177}: 

\ea\label{ex:5.176}
\gll Te kona noho o te {\ꞌ}ariki \textbf{Hotu} \textbf{Matu{\ꞌ}a} {\ꞌ}i Hiva Mara{\ꞌ}e Reŋa. \\
\textsc{art} place stay of \textsc{art} king Hotu Matu’a at Hiva Mara’e Renga \\

\glt 
‘The place where king Hotu Matu’a lived in Hiva was Mara’e Renga.’ \textstyleExampleref{[Ley-2-01.002]}
\z

\ea\label{ex:5.177}
\gll He turu a Rovi, he taŋata hāpa{\ꞌ}o i te poki {\ꞌ}a \textbf{Hotu} \textbf{{\ꞌ}ariki}. \\
\textsc{ntr} go\_down \textsc{prop} Rovi \textsc{pred} man care\_for \textsc{acc} \textsc{art} child of\textsc{.a} Hotu king \\

\glt 
‘Rovi came down, the man who took care of the child of king Hotu.’ \textstyleExampleref{[R422.002]} 
\z
\is{Apposition|)}
\section{The proper noun phrase}\label{sec:5.13}
\is{Noun phrase!proper|(}
\is{Proper noun}Proper noun phrases are those headed by proper nouns\is{Noun!proper}. As discussed in \sectref{sec:3.3.2}, the class of proper nouns\is{Noun!proper} in Rapa Nui not only includes names of persons, but a number of kinship terms\is{Kinship term} and other nouns as well, as well as pronouns. These items are grouped together on syntactic grounds: they do not take the determiner \textit{te}, but the proper article\is{a (proper article)} \textit{a}. 

What proper nouns\is{Noun!proper} have in common semantically, is that they refer to a unique entity. Unlike common nouns\is{Noun!common}, which denote a property or class, they do not need a determiner to be referential. \citet[456]{Anderson2004} argues that proper names and pronouns belong to the same category as determiners and deictics like \textit{this}: while determiners turn a common noun into a referential\is{Referentiality} expression, proper nouns\is{Noun!proper} are inherently referential. While common nouns\is{Noun!common} can function as predicates, proper nouns\is{Noun!proper} cannot. In Rapa Nui this means that they cannot take the predicate marker \textit{he}\is{he (nominal predicate marker)}. And as they do not need a determiner to acquire referentiality\is{Referentiality}, they do not take the common noun article \textit{te}. 

In \sectref{sec:5.13.1}, the structure of the proper noun phrase is discussed. \sectref{sec:5.13.2} examines the distribution and structural position of the proper article\is{a (proper article)} \textit{a}.

\subsection{Structure of the proper noun phrase}\label{sec:5.13.1}

As \citet[108]{Dixon2010-1} points out, proper nouns\is{Noun!proper} usually have fewer syntactic possibilities than common nouns\is{Noun!common}. In Rapa Nui, most proper noun phrases consist only of a proper noun preceded – if syntactically appropriate – by the proper article\is{a (proper article)} \textit{a}. Even so, the proper noun phrase may contain a range of other elements as well. The full structure of the proper noun phrase, including the preceding preposition, is shown in \tabref{tab:40} and \tabref{tab:41}.

\begin{table}
%was: small{
\normalsize{
\begin{tabularx}{\textwidth}{p{3mm}p{18mm}p{20mm}p{20mm}p{20mm}p{20mm}} 
\lsptoprule
& 0& 1& 2& 3& 4\\
& preposition& proper article\is{a (proper article)}& coll. marker& (determiner)& nucleus\\
\midrule
& \textit{{\ꞌ}i, ki, \newline mai} etc.& \textit{a}& \textit{kuā}\is{kua (collective)@kuā (collective)}& \textit{t}\textit{\textup{{}-possessive}}\is{Pronoun!possessive!t-class}& \textit{proper noun}\\
\midrule
 §&  & \ref{sec:5.13.2}& \ref{sec:5.2}& \ref{sec:6.2.1}& \ref{sec:3.3.2}\\
\lspbottomrule
\end{tabularx}
}
\caption{The proper noun phrase: prenominal elements}
\label{tab:40}
\end{table}


\begin{table}
%was: small{
\normalsize{
\begin{tabularx}{\textwidth}{p{3mm}p{10mm}p{15mm}p{20mm}p{15mm}p{15mm}p{15mm}} 
\lsptoprule
& 5& 6& \textit{\textup{7}}& 8& 9& 10\\
& adverb \is{Adverb}&   emphatic marker&  limitative marker &postnom. demonstr.\is{Demonstrative!postnominal}&  identity marker&   possessor\\
\midrule
& \textit{tako{\ꞌ}a}& \textit{mau}\is{mau ‘really’}& \textit{nō}\is{no ‘just’@nō ‘just’}& \textit{nei;} \textit{ena;} \newline  \textit{era}& \textit{{\ꞌ}ā;} \textit{{\ꞌ}ana}\is{a (identity)@{\ꞌ}ā (identity)}& \textit{\textup{possessive phrase}}\\
\midrule
 §& \ref{sec:5.8}& \ref{sec:5.8}& \ref{sec:5.8}& \ref{sec:4.6.3}& \ref{sec:5.9}& \ref{sec:6.2.1}\\
\lspbottomrule
\end{tabularx}
}\caption{The proper noun phrase: postnominal elements}
\label{tab:41}
\end{table}

The head is obligatory, and so are the preposition and the proper article\is{a (proper article)}, if required by the syntactic context. All other elements are optional. 

With the exception of the proper article\is{a (proper article)}, all items occur in the common noun phrase as well. They have been discussed in the preceding sections (see the paragraph references in the tables). 

The following examples illustrate different possibilities; each word or phrase is numbered according to the numbering in \tabref{tab:40} and \tabref{tab:41}.

\ea\label{ex:5.178}
\gll ki\textsubscript{0} a\textsubscript{1} tō{\ꞌ}oku\textsubscript{3} {\ob}matu{\ꞌ}a vahine\,{\cb}\textsubscript{4} \\
to \textsc{prop} \textsc{poss.2sg.o} {\db}parent female \\

\glt 
‘(I said) to my mother’ \textstyleExampleref{[R334.287]} 
\z

\ea\label{ex:5.179}
\gll a\textsubscript{1} kuā\textsubscript{2} koro\textsubscript{4} \\
\textsc{prop} \textsc{coll} father \\

\glt 
‘Father and the others’ \textstyleExampleref{[R184.032]} 
\z

\ea\label{ex:5.180}
\gll ki\textsubscript{0} a\textsubscript{1} Rātaro\textsubscript{4} tako{\ꞌ}a\textsubscript{5} \\
to \textsc{prop} Lazarus also \\

\glt 
‘(they wanted to see) Lazarus as well’ \textstyleExampleref{[John 12:9]}
\z

\ea\label{ex:5.181}
\gll ko\textsubscript{0} {\ꞌ}Anakena\textsubscript{4} mau\textsubscript{6} nō\textsubscript{7} \\
\textsc{prom} Anakena really just \\

\glt 
‘only Anakena (was the place where the people were not ill)’ \textstyleExampleref{[R231.098]} 
\z

\ea\label{ex:5.182}
\gll Pē\textsubscript{0} {\ob}Māria Gonzales\,{\cb}\textsubscript{4} mau\textsubscript{6} {\ꞌ}ā\textsubscript{9} \\
like {\db}Maria Gonzales really \textsc{ident} \\

\glt 
‘(That woman looks) like Maria Gonzales herself’ \textstyleExampleref{[R416.360]} 
\z

Most of these elements (except for the \textit{kuā}, determiners, and genitive phrases) may occur with pronouns as well. (See \sectref{sec:3.3.2}: pronouns belong syntactically to the class of proper nouns.). A few examples:

\ea\label{ex:5.183}
\gll ko ia tako{\ꞌ}a \\
\textsc{prom} \textsc{3sg} also \\

\glt 
‘he (knew it) as well’ \textstyleExampleref{[R620.037]} 
\z

\ea\label{ex:5.184}
\gll ko au mau nō \\
\textsc{prom} \textsc{1sg} really just \\

\glt 
‘really just I’ \textstyleExampleref{[R404.048]} 
\z

\ea\label{ex:5.185}
\gll ko au nei\\
\textsc{prom} \textsc{1sg} \textsc{prox}\\

\glt 
‘I here (am Huri a Vai)’ \textstyleExampleref{[R304.086]} 
\z

The determiner position plays a marginal role in personal noun phrases. It can only be filled by possessive pronouns\is{Pronoun!possessive}, and only when the head noun is a kinship term; see \REF{ex:5.198} on p.~\pageref{ex:5.198}. The post-nominal elements are uncommon as well. 

\subsection{The proper article \textit{a}}\label{sec:5.13.2}
\is{a (proper article)}\is{a (proper article)|(}
This section discusses the proper article\is{a (proper article)} \textit{a}.\footnote{\label{fn:278}In Polynesian linguistics, \textit{a} is more commonly called “personal article”; I use “proper article\is{a (proper article)}”, a term suggested by \citet[108]{Dixon2010-1}, as \textit{a} is exclusively used with the class of proper nouns. The term “proper” seems more appropriate than “personal”: this class is not defined by ‘personal’ (i.e. human) reference, but by its ‘proper’, name-like character.} In \sectref{sec:5.13.2.1} the contexts are listed in which this article occurs. In \sectref{sec:5.13.2.2} the question is raised whether \textit{a} is a determiner.

According to \citet[58]{Clark1976}, \textit{a} occurs in almost all Polynesian languages preceding a personal noun or pronoun after certain prepositions; in a number of Nuclear Polynesian languages it is also used in the nominative case. Both are true for Rapa Nui as well, see below.\footnote{\label{fn:279}In languages where \textit{a} is only used after prepositions, it tends to be considered (and written) as one word together with the preceding \textit{i} or \textit{ki}: \textit{ia}, \textit{kia}. See e.g. \citet[107]{ElbertPukui1979} for \ili{Hawaiian}, \citet[186]{LazardPeltzer2000} for \ili{Tahitian}.} The nominative case marker \textit{{\ꞌ}a} in \ili{Tongan} reflects the same \is{Proto-Polynesian}PPN particle.\footnote{\label{fn:280}\citet[429]{Fischer1994} presumes that the Old Rapa Nui form was \textit{{\ꞌ}a}, which was replaced by \ili{Tahitian} \textit{a} in Modern RN. This is based on the fact that the form reconstructed for \is{Proto-Polynesian}PPN is \textit{*{\ꞌ}a}; the latter is based on the \ili{Tongan} nominative marker \textit{{\ꞌ}a}. Notice, however, that the form does not have a glottal\is{Glottal plosive} in other languages which normally preserve the \is{Proto-Polynesian}PPN glottal (\ili{Rennell}, \ili{East Uvean} and \ili{East Futunan})\is{Glottal plosive}. It is thus well possible that \textit{a} had lost the glottal\is{Glottal plosive} by PNP. In any case, the glottal\is{Glottal plosive} is unstable in particles in Polynesian languages, especially in initial particles, and may disappear and (re)appear unpredictably (\sectref{sec:2.2.5}; \citealt[20]{Clark1976}). Notice also that in \ili{Tahitian} \textit{a} has a more limited distribution than in Rapa Nui: it is only used after prepositions.} 

\subsubsection[Contexts in which a is used]{Contexts in which \textit{a} is used}\label{sec:5.13.2.1}

The proper article\is{a (proper article)} \textit{a} is not the proper noun equivalent of the common noun article \textit{te}\is{te (article)}: it is not used in the same contexts where a common noun would have the article \textit{te}. The use of the proper article\is{a (proper article)} is limited to the following contexts:

\subparagraph{Subject} The proper article is used when the noun phrase or pronoun is subject of a verbal or nonverbal clause.

\ea\label{ex:5.186}
\gll He tutu \textbf{a} \textbf{nua} i te ahi. \\
\textsc{ntr} set\_fire \textsc{prop} Mum \textsc{acc} \textsc{art} fire \\

\glt 
‘Mum lighted the fire.’ \textstyleExampleref{[R232.047]} 
\z

\ea\label{ex:5.187}
\gll {\ꞌ}I te ahiahi he oho \textbf{a} \textbf{au} he tatau i te ū.\\
at \textsc{art} afternoon \textsc{ntr} go \textsc{prop} \textsc{1sg} \textsc{ntr} squeeze \textsc{acc} \textsc{art} milk\\

\glt
‘In the late afternoon I go and milk the cows.’ \textstyleExampleref{[R334.277]} 
\z

With personal pronouns\is{Pronoun!personal} used as subject, the proper article\is{a (proper article)} is sometimes left out. This happens especially with the plural pronouns and \textit{koe}, less commonly with \textit{au}, never with \textit{ia}. 

\ea\label{ex:5.188}
\gll Ka e{\ꞌ}a mai \textbf{rāua} mai te hāpī... \\
\textsc{cntg} go\_out hither \textsc{3pl} from \textsc{art} learn \\

\glt 
‘When they came from school...’ \textstyleExampleref{[R381.012]} 
\z

\ea\label{ex:5.189}
\gll ¿He aha \textbf{koe} e taŋi ena? \\
~\textsc{ntr} what \textsc{2sg} \textsc{ipfv} cry \textsc{med} \\

\glt 
‘Why are you crying?’ \textstyleExampleref{[R229.185]} 
\z

Usually, the proper article\is{a (proper article)} is omitted before the subject pronoun of an imperative\is{Imperative} clause, as in \REF{ex:5.190}. 

\ea\label{ex:5.190}
\gll Ka oho koe. \\
\textsc{imp} go \textsc{2sg} \\

\glt 
‘Go.’ \textstyleExampleref{[Notes]}
\z

\subparagraph{After prepositions ending in \textit{-i}}\footnote{\label{fn:281} In almost all Polynesian languages \textit{a} occurs after \textit{i}, \textit{ki} and \textit{mai}, but not after other prepositions. \citet[58]{Clark1976} suggests this can be explained by a morphophonemic rule which deleted \textit{a} after prepositions ending in a non-high vowel. This rule must have been operative at a stage prior to \is{Proto-Polynesian}Proto-Polynesian, as it affected all Polynesian languages. The fact that \textit{a} in Rapa Nui does not occur after \textit{hai}\is{hai (instrumental prep.)} ‘with’ shows that the rule is no longer productive.} When a proper noun is preceded by \textit{{\ꞌ}i/i}\is{i ‘in, at’@{\ꞌ}i ‘in, at’} ‘in, at’, the accusative marker \textit{i}, \textit{mai}\is{i (preposition)} ‘from’ or \textit{ki}\is{ki (preposition)} ‘to’, the proper article is used. When the preposition is \textit{mai}, the preposition \textit{i} is added between \textit{mai} and the proper article (see \REF{ex:4.272} on p.~\pageref{ex:4.272}).

\ea\label{ex:5.191}
\gll I e{\ꞌ}a era au e kimi {\ꞌ}ā \textbf{i} \textbf{a} kōrua. \\
\textsc{pfv} go\_out \textsc{dist} \textsc{1sg} \textsc{ipfv} search \textsc{cont} \textsc{acc} \textsc{prop} \textsc{2pl} \\

\glt 
‘I went out and looked for you all.’ \textstyleExampleref{[R182.012]} 
\z

\ea\label{ex:5.192}
\gll He kī a Kaiŋa \textbf{ki} \textbf{a} Makita \textbf{ki} \textbf{a} Roke{\ꞌ}aua... \\
\textsc{ntr} say \textsc{prop} Kainga to \textsc{prop} Makita to \textsc{prop} Roke’aua \\

\glt 
‘Kainga said to Makita and Roke’aua...’ \textstyleExampleref{[R243.063]} 
\z

The proper article\is{a (proper article)} is \textbf{not} used after any other preposition: agentive \textit{e}, vocative\is{Vocative} \textit{e}, genitive \textit{o}, the prominence marker \textit{ko}, and the prepositions \textit{mo/mā} ‘for’, \textit{a} ‘by’, \textit{{\ꞌ}o} ‘because of’, \textit{pe} ‘toward’, \textit{pē} ‘like’, \textit{hai} ‘with’. The proper noun or pronoun follows immediately after these markers:

\ea\label{ex:5.193}
\gll Ka oho mai, e (*a) Tiare ē. \\
\textsc{imp} go hither \textsc{voc} \textsc{prop} Tiare \textsc{voc} \\

\glt 
‘Come, Tiare.’ \textstyleExampleref{[R152.035]} 
\z

\ea\label{ex:5.194}
\gll ko (*a) koe, ko (*a) Alberto, ko (*a) Carlo \\
\textsc{prom} \textsc{prop} \textsc{2sg} \textsc{prom} \textsc{prop} Alberto \textsc{prom} \textsc{prop} Carlo \\

\glt 
‘you, Alberto, and Carlo’ \textstyleExampleref{[R103.026]} 
\z

When the noun phrase or pronoun is used in an elliptic construction, \textit{a} is used if a context is implied where it would normally be used. In the following example, the reply \textit{a au} implies the clause ‘I climbed the crater’, in which the pronoun is subject of a verbal clause, a context in which \textit{a} would be used. 

\ea\label{ex:5.195}
\gll —¿Pē hē koe i iri ai ki te rano? —¿\textbf{A} \textbf{au}? A raro {\ꞌ}ā, a pie. \\
~~~~~like \textsc{cq} \textsc{2sg} \textsc{pfv} ascend \textsc{pvp} to \textsc{art} crater\_lake ~~~~\textsc{prop} \textsc{1sg} by below \textsc{ident} by foot \\

\glt
‘—How did you climb the crater? —Me? On foot.’ \textstyleExampleref{[R623.015–017]}
\z

In other contexts, isolated proper nouns\is{Noun!proper} are marked by the prominence marker \textit{ko} (\sectref{sec:4.7.11.1}).

\subsubsection[Is a a determiner?]{Is \textit{a} a determiner?}\label{sec:5.13.2.2}

In a number of respects, the proper article\is{a (proper article)} shows complementary distribution with the common noun article \textit{te}:

\begin{itemize}
\item 
It never co-occurs with the article \textit{te}\is{te (article)}.

\item 
It occurs mostly with those elements that do not take \textit{te}: names and personal pronouns\is{Pronoun!personal}. (Only a few nouns may function both like proper nouns\is{Noun!proper} and common nouns\is{Noun!common}, see \sectref{sec:3.3.2}.)

\item 
In contexts where the proper article\is{a (proper article)} is used, it is obligatory, just like \textit{te} is obligatory.

\end{itemize}

It seems a logical step to analyse \textit{a} as an article, and indeed, in Polynesian linguistics \textit{a} is often labelled as “personal article” (see e.g. \citealt[58]{Clark1976}, \citealt[109]{Bauer1993}, \citealt[62]{Cablitz2006}). There are, however, important differences in distribution between \textit{te} and \textit{a} in Rapa Nui. For one thing, after the prepositions \textit{e}, \textit{o}, \textit{ko}, \textit{a}, \textit{o} and \textit{pe}, the article \textit{te} is obligatory (\sectref{sec:5.3.2.1}), but \textit{a} is not used.

It is even questionable whether \textit{a} is a determiner at all. For one thing, the collective marker \textit{kuā} precedes the determiner \textit{te}, but occurs after the proper article\is{a (proper article)}:

\ea\label{ex:5.196}
\gll a kuā Tiare \\
\textsc{prop} \textsc{coll} Tiare \\

\glt 
‘Tiare and the others’ \textstyleExampleref{[R315.227]} 
\z

\ea\label{ex:5.197}
\gll kuā te kape \\
\textsc{coll} \textsc{art} captain \\

\glt
‘the captain and company’ \textstyleExampleref{[R416.864]} 
\z

Secondly, while \textit{a} does not co-occur with the article \textit{te}, it does co-occur occasionally with possessive pronouns\is{Pronoun!possessive} which are in the determiner position:

\ea\label{ex:5.198}
\gll He {\ꞌ}ui iho ia ō{\ꞌ}oku ki \textbf{a} \textbf{tō{\ꞌ}oku} koro era... \\
\textsc{ntr} ask just\_then then \textsc{poss.1sg.o} to \textsc{prop} \textsc{poss.1sg.o} Dad \textsc{dist} \\

\glt
‘Then I asked my uncle (lit. father) again...’ \textstyleExampleref{[R230.121]} 
\z

These data show that \textit{a} is not in the determiner position, but in an earlier position in the noun phrase. It can thus only be called “proper article\is{a (proper article)}” in a loose way, without implying that it occupies the same position as other articles.

\textit{A} is not a preposition or case marker, either, as it occurs both with subject nouns/pro\-nouns and after several prepositions, such as the accusative marker\is{i (accusative marker)} \textit{i}.
\is{Noun phrase!proper|)}\is{a (proper article)|)}
\section{Conclusions}\label{sec:5.14}

The preceding sections have shown that the structure of the noun phrase in Rapa Nui is complex, with no less than seventeen different slots. Apart from the head, the only element which is obligatory in most contexts is the determiner. In the determiner position, two fundamentally different elements occur: \textit{t}{}-demonstratives and the predicate marker \textit{he}. The former mark referentiality\is{Referentiality} (not specificity or definiteness; the latter is indicated by demonstratives), while \textit{he} marks a noun phrase as non-referential. Indefiniteness is sometimes indicated by the numeral \textit{e tahi} ‘one’.

In subject position and after most prepositions, the determiner is obligatory. On the other hand, the determiner cannot co-occur with prenominal numerals and certain quantifiers; this means that the latter are excluded when a determiner is needed. 

Two elements which do not occur in the determiner position are the collective marker \textit{kuā/koā} and plural markers. The proper article \textit{a}, which precedes proper nouns and pronouns, is not a determiner either: it occurs in a different position in the noun phrase. Also, it occurs in less contexts than determiners; in many contexts, proper nouns are not marked with \textit{a}. This means that \textit{a} is not the proper noun counterpart of the article \textit{te}.

The head noun is usually obligatory. There are a few constructions in which a noun phrase is headless, but all of these are relatively rare.

The noun may be modified by either a noun, verb or adjective, but these do not have the same status. Modifying nouns and verbs are incorporated into the head noun, forming a compound: they are bare words and express a single concept together with the head noun. Modifying adjectives, on the other hand, express an additional concept and may form an adjective phrase.

Modifying verbs are superficially similar to bare relative clauses; in both of these, the verb is not preceded by any aspect or mood marker. However, unlike modifying verbs, bare relative clauses are full clauses which may contain arguments and modifiers. Also, they do not express a single concept together with the head noun, but express a specific event.

Finally, the noun may be modified by certain adverbs, the limitative marker \textit{nō}, the identity marker \textit{{\ꞌ}ā/{\ꞌ}ana} and the deictic particle \textit{ai}.
\is{Noun phrase|)}
