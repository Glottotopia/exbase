\chapter[Mood and negation]{Mood and negation}\label{ch:10}
\section{Introduction}\label{sec:10.1}

Mood\is{Mood|(} concerns the pragmatic status of a sentence, the speech act performed by uttering the sentence: a sentence can either be a statement (declarative mood), command (imperative\is{Imperative} mood) or question (interrogative mood) (\citealt[95]{Dixon2010-1}; \citealt[294]{Payne1997}). A fourth (minor) speech act is the exclamative\is{Exclamative}, in which the speaker gives an affective response to a fact presumed to be known by the hearer (\citealt[316]{KönigSiemund2007}).

This chapter deals with mood; sections \sectref{sec:10.2}–\ref{sec:10.4} discuss imperative\is{Imperative}, interrogative and exclamative\is{Exclamative} constructions, respectively. Furthermore, this chapter discusses negation (\sectref{sec:10.5}).

\section{Imperative mood}\label{sec:10.2}
\is{Imperative}\subsection{The imperative} \label{sec:10.2.1}
\is{Imperative|(}
Imperatives are expressed by two preverbal markers, which also have an aspectual value: the contiguity marker \is{ka (imperative marker)|(}\is{e (exhortative)|(}\textit{ka} (\sectref{sec:7.2.6}) and the imperfective marker \textit{e} (\sectref{sec:7.2.5}). \textit{Ka} is used for actions which are to be performed immediately; \textit{ka} with imperative\is{Imperative} function is glossed \textsc{imp}(erative). \textit{E} is used for actions which are to be performed in the future or which are to be performed repeatedly or habitually, as well as for general instructions; \textit{e} with imperative\is{Imperative} function is glossed \textsc{exh}(ortative). \textit{Ka} and \textit{e} can be characterised as marking \textsc{direct} and \textsc{indirect} injunctions, respectively. A few examples of both markers:

\ea\label{ex:10.1}
\gll \textbf{Ka} \textbf{e{\ꞌ}a} ki haho \textbf{ka} \textbf{to{\ꞌ}o} mai hai vai mā{\ꞌ}aku mo unu. \\
\textsc{imp} go\_out to outside \textsc{imp} take hither \textsc{ins} water \textsc{ben.1sg.a} for drink \\

\glt 
‘Go outside and bring water for me to drink.’ \textstyleExampleref{[R229.231]} 
\z

\ea\label{ex:10.2}
\gll \textbf{Ka} \textbf{uru} mai kōrua ki roto. \\
\textsc{imp} enter hither \textsc{2pl} to inside \\

\glt 
‘Come in (said to two people).’ \textstyleExampleref{[R229.261]} 
\z

\ea\label{ex:10.3}
\gll \textbf{Ka} \textbf{{\ꞌ}ara} mai koe, e nua ē. \\
\textsc{imp} wake\_up hither \textsc{2sg} \textsc{voc} Mum \textsc{voc} \\

\glt 
‘Wake up, Mum.’ \textstyleExampleref{[R229.315]} 
\z

\ea\label{ex:10.4}
\gll Ana tomo kōrua ki {\ꞌ}uta, \textbf{e} \textbf{u{\ꞌ}i} atu kōrua ki te motu.\\
\textsc{irr} go\_ashore \textsc{2pl} to inland \textsc{exh} look away \textsc{2pl} to \textsc{art} islet\\

\glt 
‘When you go ashore, watch towards the islet.’ \textstyleExampleref{[Ley-2-02.005]}
\z

\ea\label{ex:10.5}
\gll \textbf{E} \textbf{hāpa{\ꞌ}o} kōrua i a Puakiva. \\
\textsc{exh} care\_for \textsc{2pl} \textsc{acc} \textsc{prop} Puakiva \\

\glt
‘Take care of Puakiva.’ \textstyleExampleref{[R229.420–421]}
\z

As these examples show, the subject can be either omitted \REF{ex:10.1} or expressed (\ref{ex:10.2}–\ref{ex:10.5}). If expressed, it is a 2\textsuperscript{nd} person pronoun placed after the verb. Unlike other subject pronouns, it is not preceded by the proper article\is{a (proper article)} \textit{a} (\sectref{sec:5.13.2.1}). 

In a series of commands, only the first imperative\is{Imperative} tends to have an expressed subject:

\ea\label{ex:10.6}
\gll Ka {\ꞌ}ara mai \textbf{koe}, ka kai tā{\ꞌ}au o te kai. \\
\textsc{imp} wake\_up hither \textsc{2sg} \textsc{imp} eat \textsc{poss.2sg.a} of \textsc{art} food \\

\glt
‘Wake up, eat some food (lit. your [part] of the food).’ \textstyleExampleref{[R310.104]} 
\z

As discussed in \sectref{sec:8.4.1}, the direct object has the accusative marker \textit{i} when the subject is expressed (as in \ref{ex:10.5}); when the subject is not expressed, the accusative marker\is{i (accusative marker)} is omitted.

There are clear functional similarities between the imperative\is{Imperative} use of the markers \textit{ka} and \textit{e} and their aspectual uses. 

\begin{itemize}
\item 
Imperative \textit{ka} indicates immediate commands, which are temporally and situationally close to the moment of speech; similarly, \textit{ka} in non-imperative\is{Imperative} clauses indicates temporal contiguity\is{Contiguity, temporal} (\sectref{sec:7.2.6}). The main difference is that, while \textit{ka} in general expresses temporal contiguity\is{Contiguity, temporal} to another event in the discourse, imperative\is{Imperative} \textit{ka} is linked to the extratextual context, i.e. the speech situation. 

\item 
\textit{E} in imperative\is{Imperative} clauses marks future\is{Future} and habitual\is{Aspect!habitual} events, something to be expected of an imperfective marker (\sectref{sec:7.2.5}).\footnote{\label{fn:484}\textit{Ka} is used as an imperative\is{Imperative} marker in various EP languages. In most descriptions, all uses of \textit{ka} are subsumed under a single particle. Imperative or subjunctive \textit{e} is found for example in \ili{Māori} (\citealt[403]{Waite1990}; \citealt[30]{Bauer1993}), \ili{Tahitian} (\citealt[28]{LazardPeltzer2000}) and \ili{Hawaiian} (\citealt[61]{ElbertPukui1979}). \citet{WeberR2003} describes imperative\is{Imperative} \textit{ka} and aspectual \textit{ka} as distinctive particles; he also distinguishes exhortative\is{Exhortative} \textit{e} from imperfective \textit{e}.} 

\end{itemize}

While \textit{ka} and \textit{e} can be followed by any postverbal particle (depending on the function of the clause), in imperatives\is{Imperative} the range of postverbal particles with both aspectuals is limited. As the following example shows, the verb can be followed by evaluatives (\textit{nō}\is{no ‘just’@nō ‘just’} and \textit{rō}\is{ro (emphatic marker)@rō (emphatic marker)}) and directionals\is{Directional} (\textit{mai} and \textit{atu}); postverbal demonstratives\is{Demonstrative!postverbal} and the continuity marker \textit{{\ꞌ}ā/{\ꞌ}ana}\is{a (postverbal)@{\ꞌ}ā (postverbal)} are excluded.

\ea\label{ex:10.7}
\gll Ka haka noho \textbf{nō} \textbf{atu} koe i a au {\ꞌ}i nei. \\
\textsc{imp} \textsc{caus} stay just away \textsc{2sg} \textsc{acc} \textsc{prop} \textsc{1sg} at \textsc{prox} \\

\glt 
‘Let me just stay here.’ \textstyleExampleref{[R229.013]} 
\z

The imperative\is{Imperative} can be used with any verb. It is rarely used with adjectives, but this may have pragmatic rather than syntactic reasons: there are simply not many situations in which it is appropriate to order someone to have a certain property. For an example of an imperative-marked adjective, see \REF{ex:3.90} on p.~\pageref{ex:3.90}.

As the examples above show, the imperative\is{Imperative} has a wide range of pragmatic usages, including commands, requests, invitations and permissions. It is used between persons of equal or of different status; it is not inappropriate to address a higher-status person with an imperative\is{Imperative}. In the Bible translation, the imperative\is{Imperative} is commonly used in prayer; in the following example from the corpus, a chief is addressed in the imperative\is{Imperative}:

\ea\label{ex:10.8}
\gll E te {\ꞌ}ariki ē, e Tu{\ꞌ}u Kōihu ē, ka va{\ꞌ}ai mai koe  i to mātou mōai.\\
\textsc{voc} \textsc{art} king \textsc{voc} \textsc{voc} Tu’u Koihu \textsc{voc} \textsc{imp} give hither \textsc{2sg}  \textsc{acc} \textsc{art}:of \textsc{1pl.excl} statue\\

\glt 
‘O king Tu’u Koihu, give us a statue (lit. our statue).’ \textstyleExampleref{[Mtx-4-01.048]}
\z

Very occasionally, the imperative\is{Imperative} marker is omitted; this happens especially before the causative\is{Causative} marker \textit{haka}, possibly for euphonic reasons, to prevent the sequence \textit{ka~haka}.

\ea\label{ex:10.9}
\gll \textbf{Haka} \textbf{rito} koe, e nua ē, mo kā i to tātou {\ꞌ}umu āpō. \\
\textsc{caus} ready \textsc{2sg} \textsc{voc} Mum \textsc{voc} for kindle \textsc{acc} \textsc{art}:of \textsc{1pl.incl} earth\_oven tomorrow \\

\glt 
‘Get ready, Mum, to light our earth oven tomorrow.’ \textstyleExampleref{[R352.041]} 
\z

\subsection{Third-person injunctions (jussives)}\label{sec:10.2.2}

\textit{Ka} or \textit{e} are also used to express instructions or advice to be carried out by a third-person Agent. This happens for example in procedural texts, which describe how something is done or should be done.

As the following examples show, the subject may occur either before or after the verb, as in declarative clauses.

\ea\label{ex:10.10}
\gll Te taŋata ta{\ꞌ}ato{\ꞌ}a \textbf{ka} \textbf{oho} tahi \textbf{ka} \textbf{uruuru} i te kahu {\ꞌ}uri{\ꞌ}uri. \\
\textsc{art} person all \textsc{imp} go all \textsc{imp} dress:\textsc{red} \textsc{acc} \textsc{art} clothes black:\textsc{red} \\

\glt 
‘All the people must go and put on black clothes.’ \textstyleExampleref{[R210.164]} 
\z

\ea\label{ex:10.11}
\gll Te me{\ꞌ}e nei he ruku \textbf{e} \textbf{ai} te ŋā me{\ꞌ}e nei: he pātia, he hi{\ꞌ}o, he raperape...\\
\textsc{art} thing \textsc{prox} \textsc{pred} dive \textsc{ipfv} exist \textsc{art} \textsc{pl} thing \textsc{prox} \textsc{pred} harpoon \textsc{pred} glass \textsc{pred} swim\_fin\\

\glt 
‘For underwater fishing, you need (lit. there should be) the following things: a harpoon, glasses, fins...’ \textstyleExampleref{[R360.001]} 
\z
\is{ka (imperative marker)|)}\is{e (exhortative)|)}
\subsection{First-person injunctions (hortatives)}\label{sec:10.2.3}
\is{Hortative|(}
First-person injunctions (hortatives)\footnote{\label{fn:485}On the term ‘hortative’, see \citet[305, 313]{KönigSiemund2007}) and \citet[207]{Andrews2007Noun}.} are marked with \textit{ki}, the marker also used in certain purpose clauses\is{Clause!temporal} (\sectref{sec:11.5.3}). As with imperatives\is{Imperative}, the subject is optional; if expressed, it is a pronoun which occurs after the verb and which is not preceded by the proper article\is{a (proper article)} \textit{a}. 

\ea\label{ex:10.12}
\gll Ki noho tātou ki mana{\ꞌ}u pē hē te huru o te vaikava. \\
\textsc{hort} sit \textsc{1pl.incl} \textsc{hort} think like \textsc{cq} \textsc{art} manner of \textsc{art} sea \\

\glt 
‘Let’s sit down and think about what the sea is like.’ \textstyleExampleref{[R334.173]} 
\z

\ea\label{ex:10.13}
\gll Ki iri, e nua ē, ki ruŋa ki te vaka. \\
\textsc{hort} go\_up \textsc{voc} Mum \textsc{voc} to above to \textsc{art} boat \\

\glt 
‘Let’s go out (to sea), Mum, by boat.’ \textstyleExampleref{[R368.024]} 
\z

\ea\label{ex:10.14}
\gll ¿Ki aŋa te {\ꞌ}āriŋa ora mo to mātou korohu{\ꞌ}a? \\
~\textsc{hort} make \textsc{art} face live for \textsc{art}:of \textsc{1pl.excl} old\_man \\

\glt
‘Shall we make a memento (lit. living face) for our father?’ \textstyleExampleref{[Ley-4-06.004]}
\z

As \REF{ex:10.14} shows, \textit{ki} is also used to mark proposals in question form.

The hortative may be introduced by \textit{matu}\is{matu ‘come on’} ‘come on, let’s do it’, an interjection which also occurs in isolation. It can also be introduced by the directional\is{Directional} \textit{mai}. Note that this is an atypical use of the directional\is{Directional}, which normally occurs postverbally (\sectref{sec:7.5}).

\ea\label{ex:10.15}
\gll Matu, e koro ē, ki e{\ꞌ}a ki haho. \\
come\_on \textsc{voc} Dad \textsc{voc} \textsc{hort} go\_out to outside \\

\glt 
‘Come on, Dad, let’s go outside.’ \textstyleExampleref{[R229.107]} 
\z

\ea\label{ex:10.16}
\gll Mai ki turu rō tāua ki tai. \\
hither \textsc{hort} go\_down \textsc{emph} \textsc{1du.incl} to sea \\

\glt 
‘Come, let’s go to the seaside.’ \textstyleExampleref{[R245.112]}\textstyleExampleref{} 
\z
\is{Imperative|)}\is{Hortative|)}

\section{Interrogatives}\label{sec:10.3}
\subsection{Polar questions}\label{sec:10.3.1}
\is{Question!polar|(}\is{Question|(}
Polar questions (also known as yes/no questions) usually do not have a special marker, though the particle \textit{hoki} may be used (see below); nor do they differ from statements in word order. The only difference between polar questions\is{Question!polar} and statements is intonational\is{Intonation}: whereas in statements the final phrase of the sentence is normally pronounced in a low tone, polar questions\is{Question!polar} have a high rise on the final stressed syllable\is{Syllable} (\sectref{sec:2.4.2}; cf. \citealt[27]{DuFeu1995}). Here are a few examples of polar questions\is{Question!polar}:

\ea\label{ex:10.17}
\gll ¿Ko {\ꞌ}ite {\ꞌ}ā koe i te hī? \\
~\textsc{prf} know \textsc{cont} \textsc{2sg} \textsc{acc} \textsc{art} to\_fish \\

\glt 
‘Do you know how to fish?’ \textstyleExampleref{[R245.101]} 
\z

\ea\label{ex:10.18}
\gll ¿{\ꞌ}Ina {\ꞌ}ō koe he oho ki te hāpī? \\
~\textsc{neg} really \textsc{2sg} \textsc{ntr} go to \textsc{art} learn \\

\glt 
‘Don’t you go to school?’ \textstyleExampleref{[R245.086]} 
\z

\ea\label{ex:10.19}
\gll ¿E tano rō hō te me{\ꞌ}e mana{\ꞌ}u era e Tuki mo aŋa? \\
~\textsc{ipfv} correct \textsc{emph} \textsc{dub} \textsc{art} thing think \textsc{dist} \textsc{ag} Tuki for do \\

\glt 
‘Is it correct what Tuki plans to do?’ \textstyleExampleref{[R535.211]} 
\z

\ea\label{ex:10.20}
\gll ¿Hai kai piropiro {\ꞌ}ō ana va{\ꞌ}ai mai ki a māua? \\
~\textsc{ins} food rotten:\textsc{red} really \textsc{irr} give hither to \textsc{prop} \textsc{1du.excl} \\

\glt
‘Are you giving us rotten food?’ \textstyleExampleref{[R310.260]} 
\z

As these examples show, various particles can be added after the first constituent:

\begin{itemize}
\item 
\textit{{\ꞌ}Ō}\is{o (asseverative)@{\ꞌ}ō (asseverative)} in \REF{ex:10.18} and \REF{ex:10.20} indicates counterexpectation\is{Counterexpectation} (\sectref{sec:4.5.4.5}); it is used in rhetorical questions to which a negative answer is expected, or in negative rhetorical questions to which a positive answer is expected.

\item 
\textit{Hō}\is{ho (dubitative)@hō (dubitative)} in \REF{ex:10.19} indicates doubt (\sectref{sec:4.5.4.6}).

\end{itemize}

When a constituent within the clause is questioned, it is in focus\is{Focus} position: it is fronted as in \REF{ex:10.20}.

\is{hoki (polar questions)|(}Polar questions may be marked with \textit{hoki} (glossed \textsc{pq} = polar question), which is placed at the start of the sentence. \textit{Hoki} is less common in modern Rapa Nui than in older texts, but it does occur. It is used especially when the speaker expects a certain answer to the question, whether affirmative as in (\ref{ex:10.21}–\ref{ex:10.22}) or negative as in (\ref{ex:10.23}–\ref{ex:10.24}). For example, in \REF{ex:10.22} the context makes clear that the speaker assumes that the hearer has indeed heard the dream; on the other hand, in \REF{ex:10.23}, the speaker does not believe that the hearer has ever seen a devil.

\ea\label{ex:10.21}
\gll ¿Hoki e ai rō {\ꞌ}ā te famiria? \\
~~\textsc{pq} \textsc{ipfv} exist \textsc{emph} \textsc{cont} \textsc{art} family \\

\glt 
‘You have a family (don’t you)?’ \textstyleExampleref{[R103.093]} 
\z

\ea\label{ex:10.22}
\gll ¿Hoki ko ŋaro{\ꞌ}a {\ꞌ}ā e koe te vārua nei {\ꞌ}a Hina? \\
~~\textsc{pq} \textsc{prf} perceive \textsc{cont} \textsc{ag} \textsc{2sg} \textsc{art} spirit \textsc{prox} of\textsc{.a} Hina \\

\glt 
‘Did you hear Hina’s dream?’ \textstyleExampleref{[R313.087]} 
\z

\ea\label{ex:10.23}
\gll ¿Hoki ko tike{\ꞌ}a {\ꞌ}ā e koe te tātane ra{\ꞌ}e? \\
~~\textsc{pq} \textsc{prf} see \textsc{cont} \textsc{ag} \textsc{2sg} \textsc{art} devil first \\

\glt 
‘Have you ever seen a devil?’ \textstyleExampleref{[R215.029]} 
\z

\ea\label{ex:10.24}
\gll ¿Hoki e ketu rō koe i te hare o te taŋata ki raro? \\
~~\textsc{pq} \textsc{ipfv} raise \textsc{emph} \textsc{2sg} \textsc{acc} \textsc{art} house of \textsc{art} man to below \\

\glt 
‘(one wind to another:) Could you destroy someone’s house (lit. raise down a house of a man)?!’ \textstyleExampleref{[R314.121]}\textstyleExampleref{} 
\z

When a question contains a negation, it depends on the underlying presupposition which answering strategy (‘yes’ or ‘no’) is appropriate. In the following examples, the person asking the question presupposes that the underlying proposition is true; in \REF{ex:10.25} for example, the speaker expects that the person pointed out is indeed Vivika. The positive reply ‘yes’ confirms this expectation. In \REF{ex:10.26}, the asker expects the addressee to want to have him as father; negative response ‘no’ refutes this expectation.

\ea\label{ex:10.25}
\gll —¿Ta{\ꞌ}e ko Vivika? —{\ꞌ}Ēē. Ko ia. \\
~~~~~\textsc{conneg} \textsc{prom} Vivika ~~~yes \textsc{prom} \textsc{3sg} \\

\glt 
‘—Isn’t that Vivika? —Yes, it’s her.’ \textstyleExampleref{[R415.947]} 
\z

\ea\label{ex:10.26}
\gll —¿Kai haŋa {\ꞌ}ō koe ko au {\ꞌ}ā tō{\ꞌ}ou matu{\ꞌ}a?  —{\ꞌ}Ina, tō{\ꞌ}oku mau {\ꞌ}ā.\\
~~~~~\textsc{neg.pfv} want really \textsc{2sg} \textsc{prom} \textsc{1sg} \textsc{ident} \textsc{poss.2sg.o} parent   ~~~\textsc{neg}~ \textsc{poss.1sg.o} really \textsc{ident}\\

\glt
‘—Don’t you want me to be your father? —No, I want my own (father).’ \textstyleExampleref{[Mtx-7-26.036–037]}
\z

On the other hand, when the speaker presupposes that the underlying proposition is not true, this negative expectation can be confirmed with a positive answer:

\ea\label{ex:10.27}
\gll —¿{\ꞌ}Ina he pepe? —{\ꞌ}Ēē. E nohonoho nō {\ꞌ}ā {\ꞌ}i raro. \\
~~~~~\textsc{neg~} \textsc{pred} chair ~~~~yes \textsc{ipfv} sit:\textsc{red} just \textsc{cont} at below \\

\glt 
‘—There were no chairs? —Indeed. They sat on the floor.’ \textstyleExampleref{[R413.635]} 
\is{hoki (polar questions)|)}\z

\is{Question!polar|)}
\subsection{Content questions}\label{sec:10.3.2}
\is{Question!content|(}
Content questions are formed with one of the following question words: \textit{ai} ‘who’, \textit{aha} ‘what’, \textit{hē} ‘where, when, which’, or \textit{hia} ‘how many, how much’. These are always the nucleus of the first constituent of the clause.
Each question word belongs to a different word class, as can be seen from the elements preceding them. For example, \textit{ai} is a pronoun, while \textit{aha} is best categorised as a common noun. In the following sections, these question words will be discussed in turn.

\subsubsection{\textit{Ai/{\ꞌ}ai} ‘who’}\label{sec:10.3.2.1}
\is{ai ‘who’|(}
The question word ‘who’ has two forms: \textit{ai} and \textit{{\ꞌ}ai}.\footnote{\label{fn:486}Reflexes of \textit{ai} occurs in most or all Polynesian languages. In Tongic the form is \textit{hai}, which suggests that the \is{Proto-Polynesian}PPN form was \textit{*hai}. In some \is{Eastern Polynesian}EP languages (\ili{Tahitian}, \ili{Māori}, \ili{Hawaiian}), the form is \textit{vai/wai}. No other language has a form \textit{{\ꞌ}ai} except \ili{Rarotongan}, where the glottal\is{Glottal plosive} is the regular reflex of PEP \textit{*h}, \textit{*f} or \textit{*s} (\sectref{sec:2.5.2}).} \textit{Ai} occurs after prepositions and after the proper article\is{a (proper article)} \textit{a}, while \textit{{\ꞌ}ai} occurs in possessive and benefactive forms. Syntactically, \textit{ai/{\ꞌ}ai} is a pronoun: like personal pronouns\is{Pronoun!personal}, it is preceded by the proper article\is{a (proper article)} \textit{a} after the prepositions \textit{{\ꞌ}i/i} and \textit{ki} (\textit{ki a ai}), it follows immediately after other prepositions (\textit{ko ai}), and is never preceded by a determiner. 

\textit{Ai} is always in focus\is{Focus}. In \textsc{nominal} clauses\is{Clause!nominal}, this means that \textit{ai} is preposed and receives the main clause stress\is{Stress}. It is marked with \textit{ko}\is{ko (prominence marker)}, just like all pronouns used as identifying predicate (\sectref{sec:9.2.2}). Two examples:

\ea\label{ex:10.28}
\gll ¿Ko ai koe? \\
~\textsc{prom} who \textsc{2sg} \\

\glt 
‘Who are you?’ \textstyleExampleref{[R304.097]} 
\z

\ea\label{ex:10.29}
\gll ¿Ko ai te rū{\ꞌ}au era o tu{\ꞌ}a {\ꞌ}ai? \\
~\textsc{prom} who \textsc{art} old\_woman \textsc{dist} of back there \\

\glt 
‘Who is the old woman there in the back?’ \textstyleExampleref{[R416.1092]}
\z

In a verbal clause, when \textit{ai} is a \textsc{core argument} (S, A or O), it is not only preposed and stressed, but the clause takes a focus\is{Focus} construction. Just as in declarative clauses, two constructions are possible: the actor-emphatic or a cleft.

When \textit{ai} is Agent, an actor-emphatic\is{Actor-emphatic construction} construction can be used (\sectref{sec:8.6.3}). In this construction, the Agent is marked as possessive (if the clause is perfective) or benefactive (if the clause is imperfective); this means that the interrogative is \textit{{\ꞌ}a~{\ꞌ}ai} or \textit{mā~{\ꞌ}ai}, respectively. The object is often placed before the verb and tends to be unmarked.

\ea\label{ex:10.30}
\gll ¿{\ꞌ}A {\ꞌ}ai i aŋa te korone nei? \\
~of\textsc{.a} who \textsc{pfv} make \textsc{art} necklace \textsc{prox} \\

\glt 
‘Who made this necklace?’ \textstyleExampleref{[R208.263]} 
\z

\ea\label{ex:10.31}
\gll ¿Mā {\ꞌ}ai koe e hāpa{\ꞌ}o? \\
~for\textsc{.a} who \textsc{2sg} \textsc{ipfv} care\_for \\

\glt 
‘Who will take care of you?’ \textstyleExampleref{[R438.011]} 
\z

\ea\label{ex:10.32}
\gll ¿{\ꞌ}A {\ꞌ}ai kōrua te tautoru atu hai moni...? \\
~of\textsc{.a} who \textsc{2pl} \textsc{art} help away \textsc{ins} money \\

\glt 
‘Who helped you with money...?’ \textstyleExampleref{[R621.024]} 
\z

When \textit{ai} is any core argument (regardless of its semantic role), a cleft\is{Cleft} construction can be used (\sectref{sec:9.2.6}).\footnote{\label{fn:487}\citet{PotsdamPolinsky2011} distinguish three questioning strategies in Polynesian: displacement (= preposing the Wh-constituent), clefts, and pseudo-clefts (=clefts in which the relative clause\is{Clause!relative} has a head noun; in Rapa Nui, this is the only cleft\is{Cleft} strategy possible, see sec. \sectref{sec:9.2.6} and \sectref{sec:8.6.2.1}). They tentatively analyse Rapa Nui as using the displacement strategy, but admit that data are scarce. One example is given of a construction as in \REF{ex:10.36}, as well as a number of oblique examples (which indeed have a displacement structure), and one example of \textit{he aha} in the sense of ‘why’ (which is also an oblique with displacement). However, in Rapa Nui texts, pseudo-clefts abound in questions, both with \textit{ai} ‘who’ (such as in (\ref{ex:10.33}–\ref{ex:10.35})) and with \textit{aha} ‘what’ (such as \REF{ex:10.44} in the next section).} In this construction, \textit{ko ai} is a nominal predicate, followed by a subject containing a relative clause\is{Clause!relative}. The subject noun is usually the generic \is{mee ‘thing’@me{\ꞌ}e ‘thing’}\textit{me{\ꞌ}e}, though other nouns are also used. A few examples:

S/A questioned:

\ea\label{ex:10.33}
\gll ¿Ko ai te me{\ꞌ}e ŋau era i te kiko {\ꞌ}ai? \\
~\textsc{prom} who \textsc{art} thing bite \textsc{dist} \textsc{acc} \textsc{art} meat there \\

\glt 
‘Who is the one biting the meat there?’ \textstyleExampleref{[R416.1310]}
\z

\ea\label{ex:10.34}
\gll ¿Ko ai te nu{\ꞌ}u ra{\ꞌ}e i tu{\ꞌ}u ki ira...? \\
~\textsc{prom} who \textsc{art} people first \textsc{pfv} arrive to \textsc{ana} \\

\glt
‘Who were the first people who arrived there...?’ \textstyleExampleref{[R616.390]} 
\z

O questioned:

\ea\label{ex:10.35}
\gll ¿Ko ai te me{\ꞌ}e ena e kōrua ka haka tere ena?\\
~\textsc{prom} who \textsc{art} thing \textsc{med} \textsc{ag} \textsc{2pl} \textsc{cntg} \textsc{caus} run \textsc{med}\\

\glt
‘(If everybody wants to govern the island,) whom will you govern?’ \textstyleExampleref{[R647.370]} 
\z

Only very occasionally is \textit{ko ai} immediately followed by a verb; this happens especially in older texts. It is impossible to tell whether this is a simple clause, or a cleft\is{Cleft} with headless relative clause (a construction not attested otherwise, \sectref{sec:8.6.2.1}).

\ea\label{ex:10.36}
\gll ¿Ko ai i mate? \\
~\textsc{prom} who \textsc{pfv} die \\

\glt 
‘Who died?’ \textstyleExampleref{[MsE-046.009]}
\z

When a \textsc{possessor}\is{Possession} is questioned, the form \textit{{\ꞌ}ai} is used, preceded by \textit{o} or \textit{{\ꞌ}a}: like all singular pronouns, \textit{{\ꞌ}ai} is subject to the \textit{o}/\textit{a} distinction\is{Possession!o/a distinction} (\sectref{sec:6.3.2}). The clause is a proprietary clause\is{Clause!proprietary} (\sectref{sec:9.4.2}) with fronted predicate. Two examples:

\ea\label{ex:10.37}
\gll ¿O {\ꞌ}ai te hare nei? \\
~of who \textsc{art} house \textsc{prox} \\

\glt 
‘Whose house is this (lit. whose is this house)?’ \textstyleExampleref{[R208.194]} 
\z

\ea\label{ex:10.38}
\gll ¿{\ꞌ}A {\ꞌ}ai te vi{\ꞌ}e era e kī era ko Campana? \\
~of\textsc{.a} who \textsc{art} woman \textsc{dist} \textsc{ipfv} say \textsc{dist} \textsc{prom} Campana \\

\glt 
‘Whose (wife) is the woman called Campana?’ \textstyleExampleref{[R416.1164]}
\z

When \textit{ai} questions an \textsc{oblique} constituent, this constituent is simply fronted.\footnote{\label{fn:488}As (\ref{ex:10.39}–\ref{ex:10.40}) show, if the clause also has a subject, it is usually preverbal. This is usual after most preverbal constituents (\sectref{sec:8.6.1.1}).} 

\ea\label{ex:10.39}
\gll ¿\textbf{Ki} \textbf{a} \textbf{ai} a Omoaŋa i māhani ai {\ꞌ}i {\ꞌ}Ōroŋo? \\
~to \textsc{prop} who \textsc{prop} Omoanga \textsc{pfv} accustomed \textsc{pvp} at Orongo \\

\glt 
‘Who did Omoanga get to know in Orongo?’ \textstyleExampleref{[R616.017]} 
\z

\ea\label{ex:10.40}
\gll ¿\textbf{{\ꞌ}I} \textbf{muri} \textbf{i} \textbf{a} \textbf{ai} a Eva ka noho era {\ꞌ}i a Tire? \\
~at near at \textsc{prop} who \textsc{prop} Eva \textsc{cntg} stay \textsc{dist} at \textsc{prop} Chile \\

\glt 
‘With whom would Eva stay in Chile?’ \textstyleExampleref{[R615.660]} 
\z

\textit{Ai} asks about persons, while \textit{aha} ‘what’ asks about things. To ask about names, \textit{ai} is used. This applies even when the name asked for is the name of an inanimate entity:

\ea\label{ex:10.41}
\gll ¿Ko ai te {\ꞌ}īŋoa o rā kona? \\
~\textsc{prom} who \textsc{art} name of \textsc{dist} place \\

\glt 
‘What is the name of that place?’ \textstyleExampleref{[R124.014]}\textstyleExampleref{} 
\z
\is{ai ‘who’|)}

\subsubsection{\textit{Aha} ‘what, why’}\label{sec:10.3.2.2}
\is{aha ‘what’|(}
Unlike \textit{ai} ‘who’, \textit{aha} ‘what’\footnote{\label{fn:489}Cognates of \textit{aha} ({\textless} \is{Proto-Polynesian}PPN \textit{*hafa}, going back to PAN) occur throughout Polynesian languages, but especially in Eastern Polynesian (outside \is{Eastern Polynesian}EP e.g. in \ili{Kapingamarangi} and \ili{Nukuoro}). Most Tongic and Samoic languages have a reflex of \is{Proto-Polynesian}PPN \textit{*haa} instead (Pollex, see \citealt{GreenhillClark2011}).} is a common noun: it is preceded by the predicate marker \textit{he} or the article \textit{te}, never by the proper article\is{a (proper article)} \textit{a}. Apart from this, it is also used as noun modifier. Like \textit{ai}, \textit{aha} is in focus\is{Focus}: it always occurs initially and is stressed. 

As a \textsc{nominal} predicate, \textit{aha} is preceded by \textit{he}\is{he (nominal predicate marker)}; the construction is a simple classifying clause\is{Clause!classifying} (\sectref{sec:9.2.1}).

\ea\label{ex:10.42}
\gll ¿He aha te me{\ꞌ}e era pē he tiare {\ꞌ}ā? \\
~\textsc{pred} what \textsc{art} thing \textsc{dist} like \textsc{pred} flower \textsc{ident} \\

\glt 
‘What are those things (that look) like flowers?’ \textstyleExampleref{[R210.195]} 
\z

\ea\label{ex:10.43}
\gll ¿He aha kōrua?\\
~\textsc{pred} what \textsc{2pl}\\

\glt 
‘What (kind of people) are you?’ \textstyleExampleref{[Egt-02.137]}
\z

When questioning arguments in a \textsc{verbal} clause, a cleft\is{Cleft} construction is used: \textit{aha} is marked as nominal predicate, while the subject noun phrase consists of an anchor noun (usually \is{mee ‘thing’@me{\ꞌ}e ‘thing’}\textit{me{\ꞌ}e}) followed by a relative clause\is{Clause!relative}. 

Sometimes the S/A argument is questioned; as \textit{aha} questions non-human entites, this is not very common:

\ea\label{ex:10.44}
\gll ¿He aha te me{\ꞌ}e i topa ki a koro? \\
~\textsc{pred} who \textsc{art} thing \textsc{pfv} happen to \textsc{prop} Dad \\

\glt
‘What happened to Dad?’ \textstyleExampleref{[R615.594]} 
\z

More commonly, the O argument is questioned. As in all object relative clauses\is{Clause!relative} (\sectref{sec:11.4.2}), the subject is either marked with \textit{e} as in \REF{ex:10.45}, or the possessive-relative\is{Clause!possessive-relative} construction is used (\sectref{sec:11.4.4}) as in \REF{ex:10.46}.

\ea\label{ex:10.45}
\gll ¿He aha te me{\ꞌ}e i kī atu e Kihi? \\
~\textsc{pred} what \textsc{art} thing \textsc{pfv} say away \textsc{ag} Kihi \\

\glt 
‘What did Kihi say?’ \textstyleExampleref{[R615.738]} 
\z

\ea\label{ex:10.46}
\gll ¿He aha te kōrua me{\ꞌ}e i aŋa i {\ꞌ}Apina? \\
~\textsc{pred} what \textsc{art} \textsc{2pl} thing \textsc{pfv} do at Apina \\

\glt 
‘What did you do in Apina?’ \textstyleExampleref{[R301.197]} 
\z

When \textit{aha} has an \textsc{oblique} role, it is simply preposed as constituent of the verbal clause. As with \textit{ai} ‘who’, the subject is usually preverbal. After prepositions, \textit{aha} is preceded by the article \textit{te} (like all common nouns), with the exception of the instrumental preposition \textit{hai} (which is never followed by a determiner) and benefactive \textit{mo}. \textit{Mo aha} is used to ask about the purpose of an event.

\ea\label{ex:10.47}
\gll ¿{\ꞌ}O te aha a Mako{\ꞌ}i i oho mai ai ki a Paepae? \\
~because\_of \textsc{art} what \textsc{prop} Mako’i \textsc{pfv} go hither \textsc{pvp} to \textsc{prop} Paepae \\

\glt 
‘Why did Mako’i go to Paepae?’ \textstyleExampleref{[R615.699]} 
\z

\ea\label{ex:10.48}
\gll ¿A ruŋa i te aha koe i oho mai ai? \\
~by above at \textsc{art} what \textsc{2sg} \textsc{pfv} go hither \textsc{pvp} \\

\glt 
‘By/on what (means of transport) did you come?’ \textstyleExampleref{[R245.178]} 
\z

\ea\label{ex:10.49}
\gll ¿Hai aha a au ka rēkaro nei ki a Ravira? \\
~\textsc{ins} what \textsc{prop} \textsc{1sg} \textsc{cntg} gift \textsc{prox} to \textsc{prop} Ravira \\

\glt 
‘What (lit. with what) will I give as a present to Ravira?’ \textstyleExampleref{[R175.002]} 
\z

\ea\label{ex:10.50}
\gll E haŋu ē, ¿mo aha koe e {\ꞌ}ui mai ena? \\
\textsc{voc} dear\_child \textsc{voc} ~for what \textsc{2sg} \textsc{ipfv} ask hither \textsc{med} \\

\glt 
‘Dear child, for what (purpose) are you asking (this)?’ \textstyleExampleref{[R250.114]} 
\z

\textit{He aha} is also used in the sense ‘why’. In this case, it is an oblique, which is part of a simple verbal clause with preverbal subject: \textit{he aha S V}. The structure of the sentence is thus different from \textit{he aha} as subject or object, which have a cleft structure \textit{he aha} [NP Rel]; compare \REF{ex:10.51} with (\ref{ex:10.45}–\ref{ex:10.46}) above: 

\ea\label{ex:10.51}
\gll ¿He aha koe e taŋi ena? \\
~\textsc{pred} what \textsc{2sg} \textsc{ipfv} cry \textsc{med} \\

\glt 
‘Why are you crying?’ \textstyleExampleref{[Ley-9-55.064]}
\z

\textit{Aha} is used as an adjective ‘what, which’, especially after time nouns. The noun phrase containing \textit{aha} is clause-initial:

\ea\label{ex:10.52}
\gll ¿Hora aha te manurere ka tu{\ꞌ}u mai? \\
~~time what \textsc{art} airplane \textsc{cntg} arrive hither \\

\glt 
‘What time does the plane arrive?’ \textstyleExampleref{[R208.214]} 
\z

\ea\label{ex:10.53}
\gll ¿Mahana aha a koe ka oho ki Santiago? \\
~day what \textsc{prop} \textsc{2sg} \textsc{cntg} go to Santiago \\

\glt
‘What day are you going to Santiago?’ \textstyleExampleref{[R208.226]} 
\z

As these examples show, the noun is not preceded by a determiner. Cf. the use of \textit{hē} as an adjective (\sectref{sec:10.3.2.3}). 
\is{aha ‘what’|)}

\subsubsection{\textit{Hē} ‘where, when, how, which’}\label{sec:10.3.2.3}
\is{he (content question marker)@hē (content question marker)|(}
\textit{Hē} is used to ask about places, times and situations.\footnote{\label{fn:490}According to Pollex (\citealt{GreenhillClark2011}), \textit{hē} is a reflex of \is{Proto-Polynesian}PPN \textit{*fē} ‘where’, which occurs in a number of Samoic and Tongic languages. However, it is more plausible that \textit{hē} reflects \is{Proto-Polynesian}PNP \textit{*fea} ‘where’, which is widespread both in Samoic and \is{Eastern Polynesian}EP languages (e.g. \ili{Tahitian}, \ili{Hawaiian}, \ili{Marquesan} \textit{hea}, \ili{Māori} \textit{whea}, \ili{Rarotongan}, \ili{Mangarevan} \textit{{\ꞌ}ea}). Cf. \sectref{sec:2.5.2} on monophthongisation of particles.} Because of its wide range of functions, it is glossed ‘\textsc{cq}’ (content question). Syntactically, it is a locational\is{Locational} (\sectref{sec:3.6}): it is immediately preceded by prepositions, without any determiner. Like all question words, it is in focus and always occurs as the first constituent of the clause.

\subparagraph{Location} When preceded by a locative preposition (\textit{{\ꞌ}i} ‘at’ \textit{ki} ‘to’, \textit{mai} ‘from’\textit{, a} ‘by, towards’) or \textit{o} ‘of’, \textit{hē} has a locative sense ‘where’. As the examples show, \textit{hē} may be the predicate of a verbless clause as in (\ref{ex:10.54}–\ref{ex:10.55}), or an oblique in a verbal clause as in (\ref{ex:10.56}–\ref{ex:10.58}). In verbal clauses, the subject is usually preverbal.

\ea\label{ex:10.54}
\gll ¿{\ꞌ}I hē koe {\ꞌ}i te ŋā tiempo nei {\ꞌ}ī a{\ꞌ}a? \\
~at \textsc{cq} \textsc{2sg} at \textsc{art} \textsc{pl} time \textsc{prox} \textsc{imm} \textsc{deic} \\

\glt 
‘Where were you in these times?’ \textstyleExampleref{[R415.349]} 
\z

\ea\label{ex:10.55}
\gll ¿O hē te taŋata era? \\
~of \textsc{cq} \textsc{art} man \textsc{dist} \\

\glt 
‘Where is that man from?’ \textstyleExampleref{[Ley-3-06.003]}
\z

\ea\label{ex:10.56}
\gll ¿{\ꞌ}I hē a koe e noho ena? \\
~at \textsc{cq} \textsc{prop} \textsc{2sg} \textsc{ipfv} stay \textsc{med} \\

\glt 
‘Where do you live?’ \textstyleExampleref{[R399.052]} 
\z

\ea\label{ex:10.57}
\gll ¿A hē nei rā i ŋaro ai?\\
~by \textsc{cq} \textsc{prox} \textsc{intens} \textsc{pfv} disappear \textsc{pvp}\\

\glt 
‘In what direction did (the fish) disappear?’ \textstyleExampleref{[R301.179]} 
\z

\ea\label{ex:10.58}
\gll ¿Ki hē kōrua ko te poki i iri mai ena? \\
~to \textsc{cq} \textsc{2pl} \textsc{prom} \textsc{art} child \textsc{pfv} ascend hither \textsc{med} \\

\glt
‘Where did you and the child go up to?’ \textstyleExampleref{[R229.205]} 
\z

In nominal clauses\is{Clause!nominal}, \textit{hē} is also used without a preceding preposition. Its sense is similar to \textit{{\ꞌ}i hē} ‘where’, but it is only used to ask about things that are situationally close; often, the addressee is directly involved. Compare (\ref{ex:10.59}–\ref{ex:10.60}) with \REF{ex:10.54} and \REF{ex:10.56} above:

\ea\label{ex:10.59}
\gll ¿Hē koe, e vovo ē? \\
~\textsc{cq} \textsc{2sg} \textsc{voc} dear\_girl \textsc{voc} \\

\glt 
‘Where are you, my girl?’ \textstyleExampleref{[R372.030]} 
\z

\ea\label{ex:10.60}
\gll ¿Hē te kona mamae atu?\\
~\textsc{cq} \textsc{art} place pain away\\

\glt 
‘Where is the place (=body part) that hurts?’ \textstyleExampleref{[R481.100]} 
\z

\subparagraph{Situation} \textit{Pē}\is{pe ‘like’@pē ‘like’} \textit{hē} ‘like what, how’ asks about a situation; it is the interrogative counterpart of \textit{pē ira} ‘like that’ (\sectref{sec:4.6.5.2}). It occurs for example in the common greeting \textit{Pē hē koe} ‘how are you’. As with other prepositions, in a verbal clause the subject is usually preverbal. 

\ea\label{ex:10.61}
\gll ¿Pē hē koe, e hoa ē? \\
~like \textsc{cq} \textsc{2sg} \textsc{voc} friend \textsc{voc} \\

\glt 
‘How are you, my friend?’ \textstyleExampleref{[R237.116]} 
\z

\ea\label{ex:10.62}
\gll ¿Pē hē koe i {\ꞌ}ite ai mo tarai i te mōai? \\
~like \textsc{cq} \textsc{2sg} \textsc{pfv} know \textsc{pvp} for carve \textsc{acc} \textsc{art} statue \\

\glt 
‘How did you know how to carve statues?’ \textstyleExampleref{[R647.063]} 
\z

\subparagraph{Time} To ask about time, \textit{hē} is preceded by \textit{{\ꞌ}aŋa-}\is{anza- ‘recent past’@{\ꞌ}aŋa- ‘recent past’} (past) or \textit{a} (future)\is{a- ‘future’}. \textit{{\ꞌ}Aŋahē} is written as one word; \textit{a hē} is written as two words and is homophonic to \textit{a hē} ‘by what place’ (see \REF{ex:10.57} above). The particles \textit{{\ꞌ}aŋa-} and \textit{a} also occur with other roots (\sectref{sec:3.6.4}). As \REF{ex:10.64} shows, \textit{{\ꞌ}aŋahe} is preceded by locative prepositions.

\ea\label{ex:10.63}
\gll ¿\textbf{A} \textbf{hē} tātou ka iri haka{\ꞌ}ou mai mo piroto? \\
~\textsc{fut} \textsc{cq} \textsc{1pl.incl} \textsc{cntg} ascend again hither for soccer \\

\glt 
‘When will we go up again to play soccer?’ \textstyleExampleref{[R155.007]} 
\z

\ea\label{ex:10.64}
\gll ¿Mai {\ꞌ}aŋahē {\ꞌ}ā a Rapa Nui i topa rō ai ki te tire? \\
~from when.\textsc{past} \textsc{ident} \textsc{prop} Rapa Nui \textsc{pfv} happen \textsc{emph} \textsc{pvp} to \textsc{art} Chile \\

\glt 
‘From when did Rapa Nui go over to the Chileans?’ \textstyleExampleref{[R616.673]} 
\z

\subparagraph{Adjectival use} Finally, \textit{hē} is used as an adjective ‘which’. As the examples below show, the questioned noun is preceded by the appropriate preposition marking its semantic role, but does not have a determiner. For example, the questioned NP in \REF{ex:10.65} is \textit{o hua{\ꞌ}ai hē}, not *\textit{o te hua{\ꞌ}ai hē}, even though the preposition \textit{o} must normally be followed by a determiner (\sectref{sec:5.3.2.1}).

\ea\label{ex:10.65}
\gll ¿O hua{\ꞌ}ai \textbf{hē} te rū{\ꞌ}au era {\ꞌ}ai? \\
~~of family \textsc{cq} \textsc{art} old\_woman \textsc{dist} \textsc{deic} \\

\glt 
‘From which family is that woman over there?’ \textstyleExampleref{[R413.305]} 
\z

\ea\label{ex:10.66}
\gll —¿Ko poki \textbf{hē} rā poki hiko era i ta{\ꞌ}a me{\ꞌ}e? —Poki tane. \\
~~~~\textsc{prom} child \textsc{cq} \textsc{dist} child snatch \textsc{dist} \textsc{acc} \textsc{poss.2sg.a} thing ~~~child male \\

\glt
‘—Which child [was the child who] snatched your things? —A boy.’ \textstyleExampleref{[R172.012–014]}
\z

There is no sharp difference in meaning between \textit{hē} and \textit{aha} used as adjective (see (\ref{ex:10.52}–\ref{ex:10.53}) above), except that the latter only occurs with time nouns, while \textit{hē} occurs with any type of noun. Possibly \textit{hē} implies a choice from a closed range, though \REF{ex:10.66} above appears to be a counterexample.
\is{he (content question marker)@hē (content question marker)|)}

\subsubsection{\textit{Hia} ‘how much, how many’}\label{sec:10.3.2.4}
\is{hia ‘how many’|(}
\textit{Hia} ‘how much, how many’ ({\textless} \is{Proto-Polynesian}PPN \textit{*fiha,} with reflexes throughout Polynesia) is a numeral: it is always preceded by one of the numeral particles \textit{e}, \textit{ka} and \textit{hoko} (\sectref{sec:4.3.2}). \textit{Hia} may occur in a noun phrase as in (\ref{ex:10.67}–\ref{ex:10.68}), or as a separate constituent as in (\ref{ex:10.69}–\ref{ex:10.70}). In either case, it is placed at the start of the sentence.

\ea\label{ex:10.67}
\gll ¿E hia māmari o roto te hakapupa? \\
~\textsc{num} how\_many egg of inside \textsc{art} nest \\

\glt 
‘How many eggs are there inside the nest?’ \textstyleExampleref{[R173.019]} 
\z

\ea\label{ex:10.68}
\gll ¿Ka hia matahiti ō{\ꞌ}ou, e pāpātio ē?\\
~\textsc{cntg} how\_many year \textsc{poss.2sg.o} \textsc{voc} uncle \textsc{voc}\\

\glt 
‘How old are you (lit. how many years are yours), uncle?’ \textstyleExampleref{[R416.843]} 
\z

\ea\label{ex:10.69}
\gll ¿E hia tō{\ꞌ}oku tārahu mō{\ꞌ}ou? \\
~\textsc{num} how\_much \textsc{poss.1sg.o} debt \textsc{ben.2sg.o} \\

\glt 
‘How much do I owe you (lit. how much [is] my debt for you)?’ \textstyleExampleref{[R208.200]} 
\z

\ea\label{ex:10.70}
\gll ¿Hoko hia kōrua i oho ai?\\
~\textsc{num.pers} how\_many \textsc{2pl} \textsc{pvp} go \textsc{pvp}\\

\glt 
‘(With) how many did you go?’ \textstyleExampleref{[R124.008]}\textstyleExampleref{} 
\z
\is{hia ‘how many’|)}

\subsection{Dependent questions}\label{sec:10.3.3}
\is{Question!dependent|(}
Dependent questions, i.e. questions in subordinate clauses, occur mainly after speech verbs\is{Verb!speech} and cognitive verbs. 

\subparagraph{Polar questions} Dependent polar questions\is{Question!polar} are optionally introduced by \textit{hoki} as in \REF{ex:10.72}. In \REF{ex:10.71}, \textit{hoki} is not used, but here the question has a tag \textit{{\ꞌ}o {\ꞌ}ina}.

\ea\label{ex:10.72}
\gll He kī ki te {\ꞌ}auario o tū kona era {\ob}\textbf{hoki} e puē rō  mo tari rō {\ꞌ}ai i tō{\ꞌ}ona me{\ꞌ}e\,{\cb}.\\
\textsc{ntr} say to \textsc{art} guard of \textsc{dem} place \textsc{dist} {\db}\textsc{pq} \textsc{ipfv} can \textsc{emph}  for carry \textsc{emph} \textsc{subs} \textsc{acc} \textsc{poss.3sg.o} thing\\

\glt
‘She asked the guard of the place if he could carry her stuff.’ \textstyleExampleref{[R210.205]} 
\z

\ea\label{ex:10.71}
\gll Kai {\ꞌ}ite mai au {\ob}e take{\ꞌ}a haka{\ꞌ}ou rō mai koe {\ꞌ}o {\ꞌ}ina\,{\cb}. \\
\textsc{neg.pfv} know hither \textsc{1sg} {\db}\textsc{ipfv} see again \textsc{emph} hither \textsc{2sg} or \textsc{neg} \\

\glt 
‘I don’t know if you will see me again or not.’ \textstyleExampleref{[R210.072]} 
\z

Alternatively, the question is marked with the irrealis\is{Irrealis} marker \textit{ana} (\sectref{sec:11.5.2.2}):

\ea\label{ex:10.73}
\gll {\ꞌ}O ira a au i {\ꞌ}ui atu ena {\ob}\textbf{ana} haŋa koe mo turu mai  ki nei...\,{\cb}\\
because\_of \textsc{ana} \textsc{prop} \textsc{1sg} \textsc{pfv} ask away \textsc{med} {\db}\textsc{irr} want \textsc{2sg} for go\_down hither  to \textsc{prox}\\

\glt 
‘Therefore I asked you if you wanted to come here...’ \textstyleExampleref{[R315.269]} 
\z

\subparagraph{Content questions} Dependent content questions\is{Question!content} are marked with one of the question words discussed in the previous sections. Just as in main clause questions, the questioned constituent is placed at the start of the clause.

\ea\label{ex:10.74}
\gll Kai {\ꞌ}ite a au {\ob}\textbf{ko} \textbf{ai} a ia\,{\cb}. \\
\textsc{neg.pfv} know \textsc{prop} \textsc{1sg} {\db}\textsc{prom} who \textsc{prop} \textsc{3sg} \\

\glt 
‘I don’t know who she is.’ \textstyleExampleref{[R413.356]} 
\z

\ea\label{ex:10.75}
\gll Ka u{\ꞌ}i a Haŋa Roa {\ob}\textbf{he} aha e ta{\ꞌ}e tu{\ꞌ}u mai nei\,{\cb}. \\
\textsc{imp} look by Hanga Roa {\db}\textsc{pred} what \textsc{ipfv} \textsc{conneg} arrive hither \textsc{prox} \\

\glt 
‘Look towards Hanga Roa why he is not coming.’ \textstyleExampleref{[R229.137]} 
\z

\ea\label{ex:10.76}
\gll ...{\ꞌ}i te ta{\ꞌ}e {\ꞌ}ite {\ob}\textbf{{\ꞌ}i} \textbf{hē} a ia ka noho era\,{\cb}. \\
~~~at \textsc{art} \textsc{conneg} know {\db}at \textsc{cq} \textsc{prop} \textsc{3sg} \textsc{cntg} stay \textsc{dist} \\

\glt 
‘(He was afraid) because he didn’t know where he would stay.’ \textstyleExampleref{[R314.016]} 
\z
\is{Question!dependent|)}

\is{Question|)}\is{Question!content|)}
\section{Exclamatives}\label{sec:10.4}
\is{Exclamative|(}
There are three constructions in Rapa Nui specifically used for exclamations. They are marked with the aspectual \textit{ka}, the preposition \textit{ko} and the deictic particle \textit{{\ꞌ}ai}, respectively. These constructions will be discussed in turn in the next subsections.

\subsection{\textit{Ka} in exclamations}\label{sec:10.4.1}
\is{ka (aspect marker)}
With certain adjectives the continguity marker \textit{ka} (\sectref{sec:7.2.6}) is used in an emphatic sense, often in exclamations. In this construction, the quality expressed by the adjective is emphasised. This construction is only used with a limited number of adjectives, all of which express a positive evaluation: \textit{riva} ‘good’, \textit{reka} ‘pleasant’, \textit{tau} ‘beautiful, handsome’, in older texts also \textit{ma{\ꞌ}itaki} ‘clean; beautiful’. A few examples:

\ea\label{ex:10.77}
\gll ¡\textbf{Ka} \textbf{riva} {\ꞌ}ō! \\
\textsc{cntg} good really \\

\glt 
‘Very good!’ \textstyleExampleref{[R334.319]} 
\z

\ea\label{ex:10.78}
\gll ¡\textbf{Ka} \textbf{tau} te mahana nei {\ꞌ}i te ra{\ꞌ}ā! \\
\textsc{cntg} pretty \textsc{art} day \textsc{prox} at \textsc{art} sun \\

\glt 
‘What a nice sunny day!’ \textstyleExampleref{[Notes]}
\z

\ea\label{ex:10.79}
\gll {\ꞌ}Ai te nuinui o te pū{\ꞌ}oko ko tetu, ¡\textbf{ka} \textbf{ma{\ꞌ}itaki} te pū{\ꞌ}oko! \\
there \textsc{art} big:\textsc{red} of \textsc{art} head \textsc{prom} enormous \textsc{cntg} handsome \textsc{art} head \\

\glt
‘The skull was this big, it was enormous, and how beautiful it was!’ \textstyleExampleref{[Ley-2-10.010]}
\z

This construction is similar in function to \textit{{\ꞌ}ai te} preceding an adjective (\sectref{sec:10.4.3} below); in fact, in \REF{ex:10.79} above the two constructions are used side by side. The choice between the two is lexically determined: while \textit{ka} is only used with adjectives denoting a positive evaluation, \textit{{\ꞌ}ai te} is used with adjectives of size. 

The origin of this use of \textit{ka} may lay in the tendency of \textit{ka} to denote an extent, a use which is for example seen in the construction \textit{ka V rō}\is{ka (aspect marker)!ka V rō} ‘until’ (\sectref{sec:11.6.2.5}) and in the use of \textit{ka} with numerals (\sectref{sec:4.3.2.2}).

\subsection{\textit{Ko} in exclamations}\label{sec:10.4.2}

In modern Rapa Nui, \is{ko (prominence marker)!in exclamations}\textit{ko te} X is used in exclamations to convey a strong emotion about something.\footnote{\label{fn:491}\citet[149]{Moyse-Faurie2011} points out, that predicate (i.e. \textit{ko}{}-marked) noun phrases in Polynesian languages often have an exclamative\is{Exclamative} function.} This usage does not occur in older texts. Sometimes it involves a noun as in \REF{ex:10.80}, but more commonly, exclamative\is{Exclamative} \textit{ko te} is followed by an adjective as in \REF{ex:10.81}. The speaker expresses his or her emotion about the quality expressed, implying that the quality is true to a high degree: ‘How beautiful!’.

\ea\label{ex:10.80}
\gll ¡Ko te manu hope{\ꞌ}a o te tau! \\
~\textsc{prom} \textsc{art} animal last of \textsc{art} pretty \\

\glt 
‘What an extremely pretty animal!’ \textstyleExampleref{[R345.072]} 
\z

\ea\label{ex:10.81}
\gll ¡Ko te tau! \\
~\textsc{prom} \textsc{art} pretty \\

\glt
‘How beautiful!’ \textstyleExampleref{[R412.384]} 
\z

The person or thing possessing the quality in question is marked with the preposition \textit{i} ‘corresponding to’ (\sectref{sec:4.7.2}):

\ea\label{ex:10.82}
\gll ¡Ko te nene \textbf{i} \textbf{te} \textbf{kiko}, \textbf{i} \textbf{te} \textbf{tātou} \textbf{kai}! \\
~\textsc{prom} \textsc{art} sweet at \textsc{art} meat at \textsc{art} \textsc{1pl.incl} food \\

\glt 
‘How tasty is the meat, our food!’ \textstyleExampleref{[R333.543]} 
\z

\ea\label{ex:10.83}
\gll ¡Ko te {\ꞌ}aroha \textbf{i} \textbf{te} \textbf{rū{\ꞌ}au} \textbf{era}! \\
~\textsc{prom} \textsc{art} pity at \textsc{art} old\_woman \textsc{dist} \\

\glt
‘Poor old woman!’ \textstyleExampleref{[R413.103]} 
\z

A similar construction is \textit{ko te aha}\is{aha ‘what’} ‘what’, followed by a noun phrase:

\ea\label{ex:10.84}
\gll ¡Ko te aha te pōhāhā! ¡Ko te aha te {\ꞌ}ua! \\
~\textsc{prom} \textsc{art} what \textsc{art} dark \textsc{~prom} \textsc{art} what \textsc{art} rain \\

\glt 
‘What a darkness! What a rain!’ \textstyleExampleref{[R241.035–036]}
\z

\ea\label{ex:10.85}
\gll ¡Ko te aha te haka {\ꞌ}āriŋa! \\
~\textsc{prom} \textsc{art} what \textsc{art} \textsc{caus} face \\

\glt 
‘What an insolence!’ \textstyleExampleref{[R208.083]} 
\z

\subsection{\textit{{\ꞌ}Ai} in exclamations}\label{sec:10.4.3}

Adjectives of size, such as \textit{nuinui} ‘big’ and \textit{kumi} ‘big, long’, occur in a nominal construction in which they are preceded by the deictic particle \textit{{\ꞌ}ai}\is{ai (deictic)@{\ꞌ}ai (deictic)} (\sectref{sec:4.5.4.1.2}).

\ea\label{ex:10.86}
\gll E ai rō {\ꞌ}ā e rua hare toa, \textbf{{\ꞌ}ai} \textbf{te} \textbf{nuinui} tetu. \\
\textsc{ipfv} exist \textsc{emph} \textsc{cont} \textsc{num} two house store there \textsc{art} big:\textsc{red} enormous \\

\glt 
‘There were two stores, they were enormous.’ \textstyleExampleref{[R239.072]} 
\z

\ea\label{ex:10.87}
\gll {\ꞌ}I roto te hare manupātia. ¡\textbf{{\ꞌ}Ai} \textbf{te} \textbf{kumi}! \\
at inside \textsc{art} house wasp ~there \textsc{art} big \\

\glt 
‘Inside was a wasps’ nest. It was so big!’ \textstyleExampleref{[R133.004]}\textstyleExampleref{} 
\z
\is{Exclamative|)}
\is{Mood|)}
\section{Negation}\label{sec:10.5}
\is{Negation|(}
\is{ina (negator)@{\ꞌ}ina (negator)|(}Rapa Nui has three clausal negators\is{Negation}:
\begin{tabbing}
xxxx \= xxxxxxxx \= xxxxxxxxxxx\kill
\> \textit{{\ꞌ}ina} \>    neutral (discussed in \sectref{sec:10.5.1}–\ref{sec:10.5.2})\\
\> \textit{kai}   \>  perfective (\sectref{sec:10.5.3})\\
\> \textit{(e) ko}   \>  imperfective (\sectref{sec:10.5.4}–\ref{sec:10.5.5})
\end{tabbing}

The neutral character of \textit{{\ꞌ}ina} is shown by the fact that it occurs in a variety of contexts, is always followed by the neutral aspectual \textit{he}\is{he (aspect marker)}, and can be combined in a single clause with one of the other negators\is{Negation}.

While \textit{{\ꞌ}ina} is a phrase head, \textit{(e) ko} and \textit{kai} are preverbal particles which occur in the same position as – and thus replace – the aspectual marker\is{Aspect marker} (\sectref{sec:7.1}). This means that there are fewer aspectual distinctions in negative clauses than in positive ones (cf. \citealt[129]{Dixon2012}).

Apart from the three clausal negators\is{Negation}, Rapa Nui has a constituent negator \textit{\mbox{ta{\ꞌ}e}} (\sectref{sec:10.5.6}) and an existential/noun negator \textit{kore} (\sectref{sec:10.5.7}). 

The verb phrase particle \textit{hia/ia} ‘not yet’, which occurs in combination with different negators\is{Negation}, is discussed in \sectref{sec:10.5.8}.

\subsection{The neutral negator \textit{{\ꞌ}ina}}\label{sec:10.5.1}

\textit{{\ꞌ}Ina} is the most neutral negator; of all the negators\is{Negation}, it has the widest range of use. 

\subsubsection{Verbal clauses}\label{sec:10.5.1.1}

\textit{{\ꞌ}ina} is a common negator in verbal clauses, as the following examples show:

\ea\label{ex:10.88}
\gll \textbf{{\ꞌ}Ina} a Heru he u{\ꞌ}i rō mai hai mata.\\
\textsc{neg} \textsc{prop} Heru \textsc{ntr} watch \textsc{emph} hither \textsc{ins} eye\\

\glt 
‘Heru did not watch (her) with his eyes.’ \textstyleExampleref{[R313.165]} 
\z

\ea\label{ex:10.89}
\gll \textbf{{\ꞌ}Ina} a au he ha{\ꞌ}amā haka{\ꞌ}ou {\ꞌ}i te hora nei. \\
\textsc{neg} \textsc{prop} \textsc{1sg} \textsc{ntr} ashamed again at \textsc{art} time \textsc{prox} \\

\glt 
‘Now I am not ashamed any more.’ \textstyleExampleref{[R334.069]} 
\z

\ea\label{ex:10.90}
\gll \textbf{{\ꞌ}Ina} mau {\ꞌ}ā koe he haŋa mai ki a au.\\
\textsc{neg} really \textsc{ident} \textsc{2sg} \textsc{ntr} love hither to \textsc{prop} \textsc{1sg}\\

\glt 
‘You really don’t love me.’ \textstyleExampleref{[R229.468]} 
\z

\ea\label{ex:10.91}
\gll \textbf{{\ꞌ}Ina}, ho{\ꞌ}i, he ho{\ꞌ}o mau ena, te me{\ꞌ}e nō, ko ai {\ꞌ}ana mo kai. \\
\textsc{neg} indeed \textsc{ntr} sell really \textsc{med} \textsc{art} thing just \textsc{prf} exist \textsc{cont} for eat \\

\glt 
‘They did not sell (the fish); but it was there to eat.’ \textstyleExampleref{[R539-1.365]}
\z

\ea\label{ex:10.92}
\gll He ha{\ꞌ}amata he riri, {\ꞌ}e \textbf{{\ꞌ}ina} he hakaroŋo ki tū vānaŋa era  o tū hoa era ō{\ꞌ}ona.\\
\textsc{ntr} begin \textsc{ntr} angry and \textsc{neg} \textsc{ntr} listen to \textsc{dem} word \textsc{dist}  of \textsc{dem} friend \textsc{dist} \textsc{poss.3sg.o}\\

\glt
‘He began to get angry, and did not listen to the words of his friend.’ \textstyleExampleref{[R237.152]} 
\z

These examples illustrate a number of characteristics of \textit{{\ꞌ}ina}:

\begin{itemize}
\item 
\textit{{\ꞌ}Ina} is almost always clause-initial.

\item 
\textit{{\ꞌ}Ina} is neutral with respect to aspect; the verb is always marked with the neutral aspectual \textit{he}\is{he (aspect marker)}. It occurs in narrative contexts and habitual\is{Aspect!habitual} clauses, and it is used both for actions and states. However, it is used mostly in imperfective contexts; negations of one-time events tend to be expressed with other negators\is{Negation}, though \REF{ex:10.92} shows that this is not a strict rule. 

\item 
The subject of the clause occurs immediately after \textit{{\ꞌ}ina}, before the verb; in other words, the constituent order is SV/AVO. 

\end{itemize}

For the sake of comparison: the unmarked positive counterpart of \REF{ex:10.88} would be:
\ea\label{ex:10.88a}
\gll He u{\ꞌ}i rō mai a Heru hai mata.\\
\textsc{ntr} watch \textsc{emph} hither \textsc{prop} Heru \textsc{ins} eye\\

\glt 
‘Heru watched (her) with his eyes.’ \textstyleExampleref{[R313.165]} 
\z



Only occasionally is the subject in postverbal position. Usually a postverbal subject is marked with the agentive marker \textit{e}\is{e (agent marker)}. In general, preverbal subjects\is{Subject!preverbal} are not \textit{e}{}-marked (\sectref{sec:8.3.1.1}), which could be the reason why the \textit{e}{}-marked subject is placed after the verb.

\ea\label{ex:10.93}
\gll {\ꞌ}Ina he aŋiaŋi \textbf{e} \textbf{tū} \textbf{ŋā} \textbf{{\ꞌ}aku{\ꞌ}aku} \textbf{era} e aha {\ꞌ}ā te {\ꞌ}ariki. \\
\textsc{neg} \textsc{ntr} certain:\textsc{red} \textsc{ag} \textsc{dem} \textsc{pl} spirit \textsc{dist} \textsc{ipfv} what \textsc{cont} \textsc{art} king \\

\glt
‘Those spirits did not know what the king was doing.’ \textstyleExampleref{[R532-06.018]}
\z

In (\ref{ex:10.88}–\ref{ex:10.90}) above, the subject is a proper noun or pronoun. When the subject is a common noun and preverbal, it is usually not preceded by the article \textit{te}, but by the predicate marker \textit{he}\is{he (nominal predicate marker)}. This happens despite the fact that it refers to a definite entity, while \textit{he} normally marks nonreferential noun phrases (\sectref{sec:5.3.4.1}).

\ea\label{ex:10.94}
\gll {\ꞌ}Ina \textbf{he} \textbf{rū{\ꞌ}au} \textbf{nei} he turu mai ki Haŋa Roa. \\
\textsc{neg} \textsc{pred} old\_woman \textsc{prox} \textsc{ntr} go\_down hither to Hanga Roa \\

\glt 
‘This old women did not go down to Hanga Roa.’ \textstyleExampleref{[R380.006]} 
\z

\ea\label{ex:10.95}
\gll Te probrema ho{\ꞌ}i, {\ꞌ}ina \textbf{he} \textbf{māmā} o nā poki o nei. \\
\textsc{art} problem indeed \textsc{neg} \textsc{pred} mother of \textsc{med} child of \textsc{prox} \\

\glt 
‘The problem is, the mother of the child is not here.’ \textstyleExampleref{[R403.051]} 
\z

\ea\label{ex:10.96}
\gll ¿{\ꞌ}Ina {\ꞌ}ō \textbf{he} \textbf{mata} o Hotu {\ꞌ}Iti he taŋitaŋi ki te Tūpāhotu?\\
~~\textsc{neg} really \textsc{pred} tribe of Hotu Iti \textsc{pred} cry:\textsc{red} to \textsc{art} Tupahotu\\

\glt
‘The tribe of Hotu Iti doesn’t mourn for the Tupahotu, does it?’ \textstyleExampleref{[R304.070]} 
\z

\textit{{\ꞌ}Ina} may be followed by the article or another \textit{t}{}-deteminer, but this happens only occasionally:

\ea\label{ex:10.97}
\gll Te {\ꞌ}ati nō {\ꞌ}ina \textbf{te} \textbf{ŋā} \textbf{poki} he haŋa mo {\ꞌ}ite. \\
\textsc{art} problem just \textsc{neg} \textsc{art} \textsc{pl} child \textsc{ntr} want for know \\

\glt 
‘The problem is that the children don’t want to know.’ \textstyleExampleref{[R647.094]} 
\z

In (\ref{ex:10.94}–\ref{ex:10.96}) above, the construction \textit{{\ꞌ}ina he N VP} is a verbal clause in which \textit{he N} is the preverbal subject. However, the same sequence of elements may also be an existential clause\is{Clause!existential}, in which the verb phrase is part of a relative clause\is{Clause!relative} (see (\ref{ex:10.107}–\ref{ex:10.109}) below on the negation of existential clauses\is{Clause!existential}). 

\ea\label{ex:10.98}
\gll {\ꞌ}Ina he tētahi kona o te hakari [i {\ꞌ}ati]. \\
\textsc{neg} \textsc{pred} other place of \textsc{art} body ~\textsc{pfv} problem \\

\glt 
‘(There is) no other part of the body (which) is in trouble.’ \textstyleExampleref{[R481.091]} 
\z

\ea\label{ex:10.99}
\gll {\ꞌ}Ina he hua{\ꞌ}ai rahi [vānaŋa ki te ŋā poki i te re{\ꞌ}o henua].\\
\textsc{neg} \textsc{pred} family many ~speak to \textsc{art} \textsc{pl} child \textsc{acc} \textsc{art} voice land\\

\glt
‘(There are) not many families (who) speak the language of the island to the children.’ \textstyleExampleref{[R533.006]} 
\z

Constructions like (\ref{ex:10.98}–\ref{ex:10.99}) are quite distinct from (\ref{ex:10.94}–\ref{ex:10.96}) above. Firstly, the noun phrase does not refer to a specific entity, but predicates the existence of the category as a whole: ‘there is not...’ In the second place, the verb is marked in ways typical of relative clauses\is{Clause!relative}. While the verb in (\ref{ex:10.94}–\ref{ex:10.96}) has the neutral marker \textit{he}, verbs in relative clauses\is{Clause!relative} are typically marked with the aspectuals \textit{i} or \textit{e} or with zero marking, but not by \textit{he} (\sectref{sec:11.4.3}; \sectref{sec:11.4.5}).\footnote{\label{fn:492}That these two constructions are distinct is confirmed by the fact that \textit{i}, \textit{e} and zero marking never occur after \textit{{\ꞌ}ina} + proper noun or pronoun; they are limited to constructions with a common noun, which are open to an existential analysis.}

A third difference between verbal \textit{{\ꞌ}ina} clauses and existential constructions is, that in the latter the noun phrase after \textit{{\ꞌ}ina} is not always the S/A argument of the verb. This is illustrated in (\ref{ex:10.100}–\ref{ex:10.101}), where the noun phrase following \textit{{\ꞌ}ina} is the Patient. As \REF{ex:10.101} shows, the Agent may be expressed as a possessive, a construction common in relative clauses\is{Clause!possessive-relative} (\sectref{sec:11.4.4}).

\ea\label{ex:10.100}
\gll {\ꞌ}Ina he me{\ꞌ}e i rovā o tū pō era. \\
\textsc{neg} \textsc{pred} thing \textsc{pfv} obtain of \textsc{art} night \textsc{dist} \\

\glt 
‘They did not catch anything (lit. there was no thing obtained) that night.’ \textstyleExampleref{[R359.005]} 
\z

\ea\label{ex:10.101}
\gll ¿{\ꞌ}Ina {\ꞌ}ō he {\ꞌ}a{\ꞌ}amu {\ꞌ}ā{\ꞌ}au i ma{\ꞌ}u mai mai Haŋa Roa?\\
~\textsc{neg} really \textsc{pred} story \textsc{poss.2sg.a} \textsc{pfv} carry hither from Hanga Roa\\

\glt
‘Haven’t you brought any news (lit. are there no stories you brought) from Hanga Roa?’ \textstyleExampleref{[R380.039]} 
\z

Constructions as in (\ref{ex:10.98}–\ref{ex:10.101}) are relatively unusual. More commonly, the noun phrase in negative existential constructions is preceded by the numeral \textit{e tahi} ‘one’. \textit{{\ꞌ}Ina e tahi N} has become the usual way to express ‘not one, no one, nobody’:

\ea\label{ex:10.102}
\gll I oti era te {\ꞌ}ā{\ꞌ}ati, {\ꞌ}ina e tahi kope i {\ꞌ}ite  ko ai te me{\ꞌ}e i rē.\\
\textsc{pfv} finish \textsc{dist} \textsc{art} contest \textsc{neg} \textsc{num} one person \textsc{pfv} know  \textsc{prom} who \textsc{art} thing \textsc{pfv} won\\

\glt 
‘When the contest was finished, no one (lit. not one person) knew who had won.’ \textstyleExampleref{[R448.018]} 
\z

\ea\label{ex:10.103}
\gll {\ꞌ}Ina e tahi taŋata tere o ira; hoko rua mau nō. \\
\textsc{neg} \textsc{num} one person run of \textsc{ana} \textsc{num.pers} two really just \\

\glt 
‘Nobody was sailing there; just the two (of us).’ \textstyleExampleref{[R230.410]} 
\z

All examples so far involve \textit{{\ꞌ}ina} as sole negator in the clause. However, more often than not, \textit{{\ꞌ}ina} as verbal clause negator co-occurs with a second clausal negator, either perfective \textit{kai}\is{kai (negator)} or imperfective \textit{(e) ko}\is{e ko (negator)}. \tabref{tab:64} gives the number of occurrences of \textit{{\ꞌ}ina} in verbal clauses in the text corpus with and without a second negator.

\begin{table}
\begin{tabularx}{.66\textwidth}{Xrrr}
\lsptoprule

&  \textit{{\ꞌ}ina} ... \textit{kai V} &  29.5\%&  (366) \\
&  \textit{{\ꞌ}ina} ... \textit{e ko V} &  19.5\%&  (242)\\
&  \textit{{\ꞌ}ina} ... \textit{ko V} &  21.4\%&  (265)\\
\midrule
\multicolumn{2}{l}{total with other negators\is{Negation}:} &  70.5\%&  (873)\\
\tablevspace
no other negator:  &\textit{{\ꞌ}ina} ... \textit{he V} &  29.5\%&  (366)\\
\lspbottomrule
\end{tabularx}
\caption{Frequencies of single and double negators}
\label{tab:64}
\end{table}

A few examples of double negation:

\ea\label{ex:10.104}
\gll {\ꞌ}Ina a au kai maruaki. \\
\textsc{neg} \textsc{prop} \textsc{1sg} \textsc{neg.pfv} hungry \\

\glt 
‘I am not hungry.’ \textstyleExampleref{[R208.250]} 
\z

\ea\label{ex:10.105}
\gll ¡{\ꞌ}Ina mātou e ko hoa i a koe! \\
~\textsc{neg} \textsc{1pl.excl} \textsc{ipfv} \textsc{neg.ipfv} abandon at \textsc{prop} \textsc{2sg} \\

\glt 
‘We will not leave you alone!’ \textstyleExampleref{[MsE-028.012]}
\z

\ea\label{ex:10.106}
\gll {\ꞌ}Ina e ko kai i te kahi o tō{\ꞌ}ona vaka. \\
\textsc{neg} \textsc{ipfv} \textsc{neg.ipfv} eat \textsc{acc} \textsc{art} tuna of \textsc{poss.3sg.o} boat \\

\glt
‘(The fisherman) would not eat the tuna (caught with) his boat.’ \textstyleExampleref{[Ley-5-27.013]}
\z

\textit{{\ꞌ}Ina} ... \textit{e ko}\is{e ko (negator)} and \textit{{\ꞌ}ina} ... \textit{kai}\is{kai (negator)} are multiple markings of a single negation. The effect of multiple marking may be a slight reinforcement or emphasis; notice however that multiple marking is so common, that it cannot be a highly marked form.\footnote{\label{fn:493}See \citet[91]{Dixon2012} on multiple marking. According to \citet[224]{Payne1985}, there is a strong crosslinguistic tendency for negatives to be reinforced by other elements in the clause.}  As the examples illustrate, the subject is usually preverbal, just like constructions where \textit{{\ꞌ}ina} is the only negator in the clause.

In one situation the use of the double negation is almost exceptionless: the imperative\is{Imperative}. This is discussed in \sectref{sec:10.5.5}.

\subsubsection{Nonverbal clauses}\label{sec:10.5.1.2}

Several types of nonverbal clauses are negated by \textit{{\ꞌ}ina}.

\textsc{Existential} clauses\is{Clause!existential} (\sectref{sec:9.3}) are negated by placing \textit{{\ꞌ}ina} in front of the nominal predicate as in \REF{ex:10.107}. The same is true for subtypes of existential clauses: existential-locative clauses as in \REF{ex:10.108}, possessive clauses as in \REF{ex:10.109}. 

\ea\label{ex:10.107}
\gll Matahiti nei \textbf{{\ꞌ}ina} he taŋata mo hāpī i te ŋā aŋa nei. \\
year \textsc{prox} \textsc{neg} \textsc{pred} person for teach \textsc{acc} \textsc{art} \textsc{pl} work \textsc{prox} \\

\glt 
‘This year there is no one (lit. there is no man) to teach these matters.’ \textstyleExampleref{[R640.016]} 
\z

\ea\label{ex:10.108}
\gll He tike{\ꞌ}a mātou e tahi kāiŋa {\ꞌ}iti{\ꞌ}iti, \textbf{{\ꞌ}ina} he taŋata o ruŋa. \\
\textsc{ntr} see \textsc{1pl.excl} \textsc{num} one homeland small:\textsc{red} \textsc{neg} \textsc{pred} person of above \\

\glt 
‘We saw a small island, there was nobody there.’ \textstyleExampleref{[Egt-02.409]}
\z

\ea\label{ex:10.109}
\gll {\ꞌ}\textbf{Ina} pa{\ꞌ}i o māua kona mo noho. \\
\textsc{neg} in\_fact of \textsc{1du.excl} place for stay \\

\glt
‘For we do not have a place to live.’ \textstyleExampleref{[R229.210]} 
\z

Notice that positive existential clauses are nowadays usually constructed with the existential verb \textit{ai} (\sectref{sec:9.3.1}); negative clauses, however, are constructed without a verb, as these examples show.

\textsc{Locative} clauses (\sectref{sec:9.4.1}) can be negated with \textit{{\ꞌ}ina} in front of the subject. As in verbal clauses, the subject has the predicate marker \textit{he}\is{he (nominal predicate marker)}, even when it has definite reference\is{Referentiality} (see (\ref{ex:10.94}–\ref{ex:10.96}) above). 

\ea\label{ex:10.110}
\gll {\ꞌ}Ina he māmā o nā poki o nei. \\
\textsc{neg} \textsc{pred} mother of \textsc{med} child of \textsc{prox} \\

\glt
‘The mother of that boy is not here.’ \textstyleExampleref{[R403.051]} 
\z

Alternatively, the locative phrase is negated by the constituent negator \textit{ta{\ꞌ}e} (see \REF{ex:10.143} on p.~\pageref{ex:10.143})\is{tae (negator)@ta{\ꞌ}e (negator)}.

\subsubsection{Independent polarity item}\label{sec:10.5.1.3}

Besides negating verbal and nominal clauses\is{Clause!nominal}, \textit{{\ꞌ}ina} also functions as independent polarity item ‘no’:

\ea\label{ex:10.111}
\gll —E Reŋa, ka e{\ꞌ}a mai ki haka hopu atu. —{\ꞌ}Ina, ko hopu {\ꞌ}ā au. \\
~~~\textsc{voc} Renga \textsc{imp} go\_out hither to \textsc{caus} wash away ~~~\textsc{neg}~ \textsc{prf} wash \textsc{cont} \textsc{1sg} \\

\glt 
‘—Renga, come out so I can wash you. —No, I have washed (already).’ \textstyleExampleref{[Mtx-7-15.046]}
\z

\ea\label{ex:10.112}
\gll —I eke rō koe {\ꞌ}i ruŋa i te pahī era? —{\ꞌ}Ina.\\
~~~\textsc{pfv} go\_up \textsc{emph} \textsc{2sg} at above at \textsc{art} ship \textsc{dist} ~~~\textsc{neg}~~\\

\glt 
‘—Did you go on board that ship? —No.’ \textstyleExampleref{[R413.811]}\textstyleExampleref{} 
\z

\subsection{Status and origin of \textit{{\ꞌ}ina}}\label{sec:10.5.2}

In many Polynesian languages, some negators\is{Negation} are verbs, or at least have important characteristics in common with verbs: they occur in the position of the predicate and they are preceded and/or followed by VP elements such as aspectuals (see \citealt[209–211]{Payne1985}; \citealt{Broschart1999} on \ili{Tongan}). The rest of the sentence may be constructed as a subordinate clause, as evidenced by the constituent order (subject\is{Subject!raising} raising) and by the fact that the choice of aspectuals on the main verb is limited in the same way as in other subordinate clauses. The latter happens for example in \ili{Tahitian} (\citealt{LazardPeltzer1999}; \citealt[49]{LazardPeltzer2000}) and \ili{Māori} (\citealt{Hohepa1969Not}; \citealt[139–141]{Bauer1993}).

The question is whether Rapa Nui \textit{{\ꞌ}ina} can be analysed as a matrix verb followed by a subordinate clause.\footnote{\label{fn:494}Note that \textit{{\ꞌ}ina} is not related to verb-like negators\is{Negation} in other Polynesian languages (but see the discussion on \ili{Mangarevan} \textit{inau} below). The latter either do not have a cognate in Rapa Nui or a cognate with a different status; for example, the negative verb \textit{{\ꞌ}ikai} in \ili{Tongan} is related to the negative particle \textit{kai} in Rapa Nui.} \citet[57]{WeberN2003} assumes a biclausal structure, when she analyses subject placement in \textit{{\ꞌ}ina} constructions by a raising rule, in which the subject\is{Subject!raising} is moved to the subject position of the higher clause. \citet[159–160]{Stenson1981} gives several arguments to treat \textit{{\ꞌ}ina} as a matrix verb: it may be separated from the negated verb by the subject (while the otherwise common VSO order is marginal in \textit{{\ꞌ}ina}-clauses); it may co-occur with the negators\is{Negation} \textit{kai} and \textit{e ko}, and unlike the latter, it co-occurs with an aspect marker\is{Aspect marker}. It should be noted, however, that the last two points only show that \textit{{\ꞌ}ina} has a different status from \textit{kai} and \textit{e ko}, without demonstrating its verbal character. After all, the aspect marker\is{Aspect marker} does not occur in front of \textit{{\ꞌ}ina} itself, but in front of the following verb.

Another possible indication for the verbal character of \textit{{\ꞌ}ina} is, that it can be followed by a wide range of verb phrase particles: certain adverbs (\textit{mau} ‘really’, \textit{\mbox{tako{\ꞌ}a}} ‘also’), the emphatic marker \textit{rō}, the directional\is{Directional} \textit{atu}, postverbal demonstratives\is{Demonstrative!postverbal} and the identity marker \textit{{\ꞌ}ā}. This is illustrated in \REF{ex:10.90} above and in the following example:

\ea\label{ex:10.113}
\gll {\ꞌ}Ina \textbf{rō} \textbf{atu} he noho i a au. \\
\textsc{neg} \textsc{emph} away \textsc{ntr} stay at \textsc{prop} \textsc{1sg} \\

\glt 
‘I couldn’t keep (my fishing line) steady (lit. It didn’t stay at all to me).’ \textstyleExampleref{[R230.162]} 
\z

Despite these arguments, there are good reasons not to analyse \textit{{\ꞌ}ina} as a verb followed by a subordinate clause. 

%\setcounter{listWWviiiNumlxivleveli}{0}
\begin{enumerate}
\item 
The most obvious difference between \textit{{\ꞌ}ina} and verbs is, that \textit{{\ꞌ}ina} is never preceded by an aspectual. Verbs are always preceded by aspectuals (with a few well-defined exceptions, see \sectref{sec:7.2.2}).

\item 
In \ili{Māori} and \ili{Tahitian}, one argument for a biclausal analysis of negative constructions is, that the choice of aspectuals with the main verb is limited to precisely those aspectuals occurring in subordinate clauses. In Rapa Nui however, the reverse is true: the main verb after \textit{{\ꞌ}ina} is obligatorily marked with neutral \textit{he}\is{he (aspect marker)}, while those markers typical of subordinate clauses (\textit{i}, \textit{e} and Ø) do not occur. 

\item 
As shown above, \textit{{\ꞌ}ina} can be combined with the negators\is{Negation} \textit{kai}\is{kai (negator)} and \textit{e ko}\is{e ko (negator)}. Both of these are main clause negators\is{Negation}; subordinate clauses are mostly negated with the constituent negator \textit{ta{\ꞌ}e}. \textit{{\ꞌ}Ina} is never combined with the negator \textit{ta{\ꞌ}e}, which suggests that the clause following \textit{{\ꞌ}ina} is a main clause.

\item 
The fact that \textit{{\ꞌ}ina} is almost invariably clause-initial can also be considered as an argument against its verbal status. No verb is as consistently initial as \textit{{\ꞌ}ina}; even auxiliary verbs like \textit{ha{\ꞌ}amata} ‘begin’ may be preceded by subjects and other constituents. Rather, its obligatory initial position places \textit{{\ꞌ}ina} on a par with focus elements like interrogatives (\sectref{sec:10.3.2}) and deictic particles (\sectref{sec:4.5.4.1}).

\end{enumerate}

The main argument for analysing \textit{{\ꞌ}ina} as a matrix verb in a biclausal construction, is that it attracts the subject: after \textit{{\ꞌ}ina}, the subject\is{Subject!preverbal} is usually preverbal. In this respect, \textit{{\ꞌ}ina} constructions are similar to constructions with auxiliary verbs such as \textit{ha{\ꞌ}amata} ‘begin’ (\sectref{sec:11.3.2.1}), and it may be tempting to analyse both along the same lines. However, auxiliary verbs in Rapa Nui are not the only elements that trigger preverbal subject placement. Subjects tend to be preverbal after a wide range of initial elements, including adjuncts and deictic particles (\sectref{sec:8.6.1.1}; cf. Footnote \ref{fn:420} on p.~\pageref{fn:420}). 

We may conclude that \textit{{\ꞌ}ina} is not a verb and that \textit{{\ꞌ}ina} constructions are monoclausal. Even so, it should be noted that \textit{{\ꞌ}ina} is significantly different from other negators\is{Negation}: \textit{{\ꞌ}ina} is a phrase nucleus, while other negators\is{Negation} are prenuclear particles\is{Particle!prenuclear}. \textit{{\ꞌ}Ina} forms a constituent on its own, which may contain various postnuclear particles\is{Particle!postnuclear}. This is confirmed by the fact that second-position particles (which are placed after the first constituent) occur immediately after \textit{{\ꞌ}ina}. Here is an example with \textit{pa{\ꞌ}i} (\sectref{sec:4.5.4.2}):

\ea\label{ex:10.114}
\gll {\ꞌ}Ina, \textbf{pa{\ꞌ}i}, a mātou kai māuiui {\ꞌ}i te rōviro. \\
\textsc{neg} in\_fact \textsc{prop} \textsc{1pl.excl} \textsc{neg.pfv} sick at \textsc{art} smallpox \\

\glt
‘In fact, we were not sick with smallpox.’ \textstyleExampleref{[R539-1.680]}
\z

The fact that \textit{{\ꞌ}ina} is consistently initial, conforms to a general crosslinguistic tendency for negative particles to come first (\citealt[560]{Miestamo2007} and refs. there). It may also be explained by the possible origin of \textit{{\ꞌ}ina}. Unlike other negators\is{Negation} in Rapa Nui, \textit{{\ꞌ}ina} is not widely found in other Polynesian languages. The only plausible cognate I have found is Mangareven \textit{inau}.\footnote{\label{fn:495}\textit{Inau} may in turn be related to \textit{kinau}, found in some languages in West-Polynesia in the sense ‘to persist against something’ (Pollex, \citealt{GreenhillClark2011}). In \ili{East Futunan} and \ili{East Uvean}, this verb has ‘to deny’ as one of its senses.} The latter is used both as independent negator ‘no’ and as verb ‘to deny a proposition; to refuse’ (\citealt[24]{Tregear2009}; \citealt[83]{Rensch1991}).

If \textit{{\ꞌ}ina} is indeed related to \ili{Mangarevan} \textit{inau}, this suggests that it originated as an independent polarity item.\footnote{\label{fn:496}The verbal use in \ili{Mangarevan} may be a secondary development, one which is not unexpected given the great freedom of cross-categorial use in Polynesian languages.} This would confirm Clark’s suggestion (\citealt[104]{Clark1976}) that \textit{{\ꞌ}ina} started out as reinforcement of another negator (‘no, we will not go’) and developed into a clausal negator, a cross-linguistically common process which is known as \isi{Jespersen’s Cycle} (\citealt[566]{Miestamo2007}). This analysis would provide a historical explanation for the fact that \textit{{\ꞌ}ina} is always clause-initial, and the fact that it is often accompanied by another negator. 
\is{ina (negator)@{\ꞌ}ina (negator)|)}
\subsection{The perfective negator \textit{kai}}\label{sec:10.5.3}

\textit{Kai}\is{kai (negator)|(} negates clauses in the perfective\is{Perfective aspect} aspect.\footnote{\label{fn:497}The negator \textit{kai} occurs in a few other Polynesian languages (\ili{Māori}, \ili{Pukapuka}, \ili{Tikopian}) but only as a negative imperative\is{Imperative} marker and/or in the sense ‘lest’ (Pollex, see \citealt{GreenhillClark2011}). More widespread are reflexes of \is{Proto-Polynesian}PPN \textit{*{\ꞌ}ikai}, which has various negative senses in all branches of Polynesian.} It precedes the verb and occurs in the same position as aspectuals. As discussed in \sectref{sec:10.5.1} above, it is often combined with \textit{{\ꞌ}ina}, in which case the subject usually precedes the verb. 

\textit{Kai} is used to negate events in narrative as in (\ref{ex:10.115}–\ref{ex:10.116}), and any past\is{Past} events as in (\ref{ex:10.117}–\ref{ex:10.119}).\footnote{\label{fn:498}The latter point is illustrated somewhat more extensively, to show that \textit{kai} does indeed negate past tense clauses, the positive counterpart of which would have perfective \textit{i}. In this respect my analysis is different from \citet[79]{Englert1978}, who claims that \textit{i}{}-clauses are negated by \textit{ta{\ꞌ}e} (an analysis followed by \citealt[158]{Chapin1978} and \citealt[157]{Stenson1981}). In fact, \textit{ta{\ꞌ}e} is not the default negator of \textit{i}, but is used to negate certain constructions with \textit{i} and \textit{e} (\sectref{sec:10.5.6} below).} If these clauses were positive, the former would be marked with \textit{he}, the latter with perfective \textit{i} or – if the speaker wishes to emphasise their present relevance – perfect \textit{ko V {\ꞌ}ā}\is{ko V {\ꞌ}ā (perfect aspect)}.

\ea\label{ex:10.115}
\gll He hoki mai ki {\ꞌ}uta, \textbf{kai} iri ki te hakanonoŋa. \\
\textsc{ntr} return hither to inland \textsc{neg.pfv} ascend to \textsc{art} fishing\_zone \\

\glt 
‘They returned inland, they did not go out to the \textit{hakanononga} fishing zones.’ \textstyleExampleref{[Ley-6-43.031]}
\z

\ea\label{ex:10.116}
\gll \textbf{Kai} pāhono e Hotu i tū vānaŋa era {\ꞌ}a Tahoŋa. \\
\textsc{neg.pfv} answer \textsc{ag} Hotu \textsc{acc} \textsc{dem} word \textsc{dist} of\textsc{.a} Tahonga \\

\glt 
‘Hotu did not reply to those words of Tahonga.’ \textstyleExampleref{[R301.273]} 
\z

\ea\label{ex:10.117}
\gll ¿He aha rā ia {\ꞌ}ina ho{\ꞌ}i koe \textbf{kai} kī mai? \\
~\textsc{pred} what \textsc{intens} then \textsc{neg} indeed \textsc{2sg} \textsc{neg.pfv} say hither \\

\glt 
‘Why then didn’t you tell me?’ \textstyleExampleref{[R372.050]} 
\z

\ea\label{ex:10.118}
\gll ¿{\ꞌ}Ina koe \textbf{kai} {\ꞌ}ā{\ꞌ}ati i te {\ꞌ}ā{\ꞌ}ati era? \\
~\textsc{neg} \textsc{2sg} \textsc{neg.pfv} compete \textsc{acc} \textsc{art} contest \textsc{dist} \\

\glt 
‘(talking about an event in the past:) Didn’t you compete in that contest?’ \textstyleExampleref{[R415.738]} 
\z

\ea\label{ex:10.119}
\gll E nua, \textbf{kai} kī mai ho{\ꞌ}i koe pē hē te tunu haŋa o te kai era \\
\textsc{voc} Mum \textsc{neg.pfv} say hither indeed \textsc{2sg} like \textsc{cq} \textsc{art} cook \textsc{nmlz} of \textsc{art} food \textsc{dist} \\

\glt 
‘Mum, you didn’t tell me how to cook that food.’ \textstyleExampleref{[R236.091]} 
\z

\textit{Kai} is also used to negate stative verbs\is{Verb!stative}. In positive clauses, these verbs are commonly marked with perfect aspect\is{Aspect!perfect} \textit{ko V {\ꞌ}ā}\is{ko V {\ꞌ}ā (perfect aspect)} (\sectref{sec:7.2.7.2}).

\ea\label{ex:10.120}
\gll {\ꞌ}Ina a au \textbf{kai} maruaki. \\
\textsc{neg} \textsc{prop} \textsc{1sg} \textsc{neg.pfv} hungry \\

\glt 
‘I am not hungry.’ \textstyleExampleref{[R208.250]} 
\z

\ea\label{ex:10.121}
\gll ¡Ko haŋa {\ꞌ}ā a au mo topa atu! ¡\textbf{Kai} haŋa a au mo oho!  \\
~\textsc{prf} want \textsc{cont} \textsc{prop} \textsc{1sg} for descend away ~\textsc{neg.pfv} want \textsc{prop} \textsc{1sg} for go  \\

\glt 
‘I want to get off (the ship)! I don’t want to go!’ \textstyleExampleref{[R210.106–107]}
\z

\ea\label{ex:10.122}
\gll {\ꞌ}Ina a au \textbf{kai} haŋa mo iri atu. \\
\textsc{neg} \textsc{prop} \textsc{1sg} \textsc{neg.pfv} want for ascend hither \\

\glt 
‘I don’t want to go up (to the hospital).’ \textstyleExampleref{[R162.023]} 
\z

In \REF{ex:10.117}, \REF{ex:10.120} and \REF{ex:10.122}, \textit{kai} co-occurs with the neutral negator \textit{{\ꞌ}ina}\is{ina (negator)@{\ꞌ}ina (negator)}. There is little or no semantic or pragmatic difference between clauses with and without \textit{{\ꞌ}ina}, though he examples with \textit{{\ꞌ}ina} may be slightly more emphatic than constructions with \textit{kai} alone.

Just like any verb phrase, a verb phrase marked with \textit{kai} may contain various kinds of postverbal particles, such as directionals\is{Directional} (\textit{mai} in \REF{ex:10.117} and \REF{ex:10.119} above). When the clause has perfect aspect\is{Aspect!perfect}, the continuity marker \textit{{\ꞌ}ā/{\ꞌ}ana}\is{a (postverbal)@{\ꞌ}ā (postverbal)} may be added. This marker is obligatory with the perfect marker \textit{ko} and indicates continuity of a state (\sectref{sec:7.2.5.5}); in combination with \textit{kai} it indicates that the negative state still continues, i.e. that a positive action has not yet taken place, or that a positive state has not yet been reached.

\ea\label{ex:10.123}
\gll E {\ꞌ}iti{\ꞌ}iti nō {\ꞌ}ā a koe; \textbf{kai} {\ꞌ}ite \textbf{{\ꞌ}ana} e tahi me{\ꞌ}e  o te via taŋata.\\
\textsc{ipfv} small:\textsc{red} just \textsc{cont} \textsc{prop} \textsc{2sg} \textsc{neg.pfv} know \textsc{cont} \textsc{num} one thing  of \textsc{art} life person\\

\glt 
‘You are (still) little; you don’t know anything about human life (yet).’ \textstyleExampleref{[R210.052]} 
\z

\ea\label{ex:10.124}
\gll ¿\textbf{Kai} {\ꞌ}ara \textbf{{\ꞌ}ana} {\ꞌ}ō a nua era ko Kava, e ta{\ꞌ}e tu{\ꞌ}u mai nei?\\
~\textsc{neg.pfv} wake\_up \textsc{cont} really \textsc{prop} Mum \textsc{dist} \textsc{prom} Kava \textsc{ipfv} \textsc{conneg} arrive hither \textsc{prox}\\

\glt 
‘Hasn’t mother Kava not woken up (yet), that she doesn’t come?’ \textstyleExampleref{[R229.359]}\textstyleExampleref{} 
\z
\is{kai (negator)|)}
\subsection{The imperfective negator \textit{(e) ko}}\label{sec:10.5.4}

\textit{(E) ko}\is{e ko (negator)|(}\footnote{\label{fn:499}The origin of \textit{ko} is unclear. Pollex (\citealt{GreenhillClark2011}) mentions a negative imperative\is{Imperative} form *\textit{kaua} in Fijian and Polynesian, which could have assimilated {\textgreater} \textit{*kō} {\textgreater} \textit{ko}. The semantic correspondence is tempting, but the evidence for \textit{*kaua} is not very strong; more common is \textit{{\ꞌ}aua}, which occurs throughout Polynesian and which could be at the root of Rapa Nui \textit{{\ꞌ}o} ‘lest’ (\sectref{sec:11.5.4}). Alternatively, \textit{ko} could be a shortening of \textit{kore}, which is the main negator in verbal clauses in Central-Eastern Polynesian languages \citep[100]{Clark1976}. This would explain the fact that \textit{e} is a fixed part of the negation in most contexts: in \is{Central-Eastern Polynesian}CE languages, \textit{kore} fused with preceding aspectuals (esp. \textit{ka} and \textit{e}). NB \textit{kore} itself also occurs in Rapa Nui as a lexical negator (\sectref{sec:10.5.7}).} is the imperfective negator. Like \textit{kai}, it replaces the aspectual in front of the verb. The first element \textit{e} (tentatively glossed as imperfective\is{e (imperfective)}) is almost always included, except in the imperative\is{Imperative}. Like \textit{kai}, \textit{e ko} it can be reinforced with \textit{{\ꞌ}ina}, which triggers preverbal verb placement; compare \REF{ex:10.126} and \REF{ex:10.127} below.

\textit{E ko} has the same range of use as imperfective \textit{e}. Is is used in sentences expressing a future event or intention:

\ea\label{ex:10.125}
\gll A koe, e Vai Ora ē, \textbf{e} \textbf{ko} ai ta{\ꞌ}a rua poki. \\
\textsc{prop} \textsc{2sg} \textsc{voc} Vai Ora \textsc{voc} \textsc{ipfv} \textsc{neg.ipfv} exist \textsc{poss.2sg.a} two child \\

\glt 
‘You, Vai Ora, won’t have another child.’ \textstyleExampleref{[R301.077]} 
\z

\ea\label{ex:10.126}
\gll \textbf{E} \textbf{ko} {\ꞌ}avai e au e tahi taŋata i tā{\ꞌ}aku poki. \\
\textsc{ipfv} \textsc{neg.ipfv} give \textsc{ag} \textsc{1sg} \textsc{num} one person \textsc{acc} \textsc{poss.1sg.a} child \\

\glt 
‘I won’t give my child to anybody.’ \textstyleExampleref{[R229.069]} 
\z

\ea\label{ex:10.127}
\gll {\ꞌ}Ina a au \textbf{e} \textbf{ko} {\ꞌ}avai atu ki a koe i a Puakiva. \\
\textsc{neg} \textsc{prop} \textsc{1sg} \textsc{ipfv} \textsc{neg.ipfv} give away to \textsc{prop} \textsc{2sg} \textsc{acc} \textsc{prop} Puakiva \\

\glt
‘I won’t give Puakiva to you.’ \textstyleExampleref{[R229.010]} 
\z

It also negates habitual\is{Aspect!habitual} actions and general facts.

\ea\label{ex:10.128}
\gll {\ꞌ}Ina a {\ꞌ}Orohe \textbf{e} \textbf{ko} hoa i tō{\ꞌ}ona taina {\ꞌ}iti{\ꞌ}iti. \\
\textsc{neg} \textsc{prop} Orohe \textsc{ipfv} \textsc{neg.ipfv} abandon \textsc{acc} \textsc{poss.3sg.o} sibling small:\textsc{red} \\

\glt 
‘(When they walk to school,) Orohe does not leave his little sister alone.’ \textstyleExampleref{[R166.005]} 
\z

\ea\label{ex:10.129}
\gll Mo ta{\ꞌ}e e{\ꞌ}a o te nu{\ꞌ}u hī ika, \textbf{e} \textbf{ko} ai te ika mo kai. \\
if \textsc{conneg} go\_out of \textsc{art} people to\_fish fish \textsc{ipfv} \textsc{neg.ipfv} exist \textsc{art} fish for eat \\

\glt
‘If the fishermen don’t go out, there is no fish to eat.’ \textstyleExampleref{[R334.261]} 
\z

Finally, \textit{e ko} negates stative verbs\is{Verb!stative}. This includes auxiliaries like \textit{puē}\is{pue ‘can’@puē ‘can’}, as in \REF{ex:10.132}.

\ea\label{ex:10.130}
\gll \textbf{E} \textbf{ko} rivariva te kāiŋa, e ko nahonaho te noho oŋa. \\
\textsc{ipfv} \textsc{neg.ipfv} good:\textsc{red} \textsc{art} homeland \textsc{ipfv} \textsc{neg.ipfv} comfortable \textsc{art} stay \textsc{nmlz} \\

\glt 
‘The land wasn’t good, life was not comfortable (up until now).’ \textstyleExampleref{[R368.103]} 
\z

\ea\label{ex:10.131}
\gll ¿\textbf{E} ko haŋa {\ꞌ}ō koe mo {\ꞌ}ori o tāua? \\
~\textsc{ipfv} \textsc{neg.ipfv} want really \textsc{2sg} for dance of \textsc{1du.incl} \\

\glt 
‘Don’t you want to dance with me (lit. us to dance)?’ \textstyleExampleref{[R315.115]} 
\z

\ea\label{ex:10.132}
\gll \textbf{E} \textbf{ko} puē ho{\ꞌ}i tāua mo hī {\ꞌ}i te kona nei. \\
\textsc{ipfv} \textsc{neg.ipfv} can indeed \textsc{1du.incl} for to\_fish at \textsc{art} place \textsc{prox} \\

\glt 
‘We cannot fish in this place.’ \textstyleExampleref{[R237.149]} 
\z

\subsection{Negation of the imperative}\label{sec:10.5.5}
\is{Imperative}
Negative commands are marked by the imperfective negator \textit{(e) ko}, usually in combination with \textit{{\ꞌ}ina}. While \textit{e} is obligatory in other uses of the imperfective negator, in imperatives\is{Imperative} it is often left out, as in \REF{ex:10.133} and \REF{ex:10.135} below. However, when \textit{{\ꞌ}ina} is not included, as in \REF{ex:10.136}, \textit{e} is obligatory.

As with other uses of \textit{{\ꞌ}ina}\is{ina (negator)@{\ꞌ}ina (negator)}, the subject – if expressed at all – tends to be placed before the verb. 

The following examples show, that \textit{({\ꞌ}ina) (e) ko} negates both immediate commands (marked with \textit{ka} when positive, \sectref{sec:10.2.1}) and non-immediate commands (marked with \textit{e} when positive).

\ea\label{ex:10.133}
\gll Ka mou, \textbf{{\ꞌ}ina} koe \textbf{ko} taŋi haka{\ꞌ}ou. \\
\textsc{imp} quiet \textsc{neg} \textsc{2sg} \textsc{neg.ipfv} cry again \\

\glt 
‘Be quiet, don’t cry anymore.’ \textstyleExampleref{[R229.343]} 
\z

\ea\label{ex:10.134}
\gll E hāpa{\ꞌ}o kōrua i a Puakiva. \textbf{{\ꞌ}Ina} kōrua \textbf{e} \textbf{ko} tiŋa{\ꞌ}i i a ia. \\
\textsc{exh} care\_for \textsc{2pl} \textsc{acc} \textsc{prop} Puakiva \textsc{neg} \textsc{2pl} \textsc{ipfv} \textsc{neg.ipfv} strike \textsc{acc} \textsc{prop} \textsc{3sg} \\

\glt 
‘You two take care of Puakiva. Don’t beat him.’ \textstyleExampleref{[R229.420]} 
\z

\ea\label{ex:10.135}
\gll \textbf{{\ꞌ}Ina} \textbf{ko} pōŋeha \textbf{ko} makenu rahi tako{\ꞌ}a. \\
\textsc{neg} \textsc{neg.ipfv} noise \textsc{neg.ipfv} move much also \\

\glt 
‘Don’t make noise or move a lot.’ \textstyleExampleref{[R210.171]} 
\z

\ea\label{ex:10.136}
\gll \textbf{E} \textbf{ko} oho koe ki te rua hare. \\
\textsc{ipfv} \textsc{neg.ipfv} go \textsc{2sg} to \textsc{art} other house \\

\glt
‘Don’t go to another house.’ \textstyleExampleref{[R310.016]} 
\z

\is{Imperative}First and third person injunctions are negated in the same way. Notice that in \REF{ex:10.138} below, the subject remains in postverbal position.

\ea\label{ex:10.137}
\gll \textbf{{\ꞌ}Ina} a tātou \textbf{ko} eke {\ꞌ}i ruŋa i te tumu era. \\
\textsc{neg} \textsc{prop} \textsc{1pl.incl} \textsc{neg.ipfv} go\_up at above at \textsc{art} tree \textsc{dist} \\

\glt 
‘Let’s not climb that tree.’ \textstyleExampleref{[R481.044]} 
\z

\ea\label{ex:10.138}
\gll \textbf{{\ꞌ}Ina} \textbf{ko} tu{\ꞌ}u haka{\ꞌ}ou {\ꞌ}i te hora era e tahi taŋata. \\
\textsc{neg} \textsc{neg.ipfv} arrive again at \textsc{art} time \textsc{dist} \textsc{num} one person \\

\glt 
‘(When he was in mourning), at that time nobody could go to his house anymore.’ \textstyleExampleref{[R310.160]}\textstyleExampleref{} 
\z
\is{e ko (negator)|)}
\subsection{The constituent negator \textit{ta{\ꞌ}e}}\label{sec:10.5.6}

\textit{Ta{\ꞌ}e}\is{tae (negator)@ta{\ꞌ}e (negator)|(} has a wide range of uses, all of which can be characterised as constituent negation: \textit{ta{\ꞌ}e} is used whenever something other than a main clause is negated, i.e. a subordinate clause or a constituent of a clause.\footnote{\label{fn:500}Cognates of \textit{ta{\ꞌ}e} are widespread; they occur in most Samoic-Outlier languages, as well as in \ili{Tongan} and a number of \is{Central-Eastern Polynesian}CE languages (\ili{Māori}, \ili{Marquesan}, \ili{Mangarevan}). The glottal\is{Glottal plosive} only occurs in those languages that preserved the \is{Proto-Polynesian}PPN glottal\is{Glottal plosive}, such as \ili{Tongan} and Rapa Nui. The initial vowel was assimilated to \textit{e} in all languages except \ili{Tongan} and Rapa Nui, and in most Samoic-Outlier the initial consonant changed to \textit{s-} (or a reflex of \textit{*s-}) or \textit{l-}. As a result, the current form is \textit{see}, \textit{hee} or \textit{lee} in most SO languages, and \textit{tee} in \is{Central-Eastern Polynesian}CE languages. \citet[85–87]{Clark1976} argues for \textit{*ta{\ꞌ}e} as the \is{Proto-Polynesian}PPN form. This had probably assimilated to \textit{*te{\ꞌ}e} in PNP (see also \citealt{Hamp1977}); the question remains whether Rapa Nui \textit{ta{\ꞌ}e} should be explained as subsequent dissimilation, or whether \textit{*ta{\ꞌ}e} survived alongside \textit{*te{\ꞌ}e} in PNP \citep[87]{Clark1976}.

% \todo[inline]{In the next paragraph a reference to Pupu-takao1908, is there a way to have just the abbreviation "Pupu-takao" quoted? If not, it's OK.}
In SO languages, reflexes of \textit{*ta{\ꞌ}e} are the unmarked negator. In \ili{Mangarevan} as well, \textit{tē} seems to be a main clause negator (\citealt[78]{Janeau1908}; examples are found in \citet{Pupu-takao1908}, e.g. Mark 4:40 \textit{Tē kereto ana noti ra kotou?} ‘Do you still not believe?’). In \ili{Marquesan}, on the other hand, \textit{tē} is a preverbal modifier (\citealt[52]{MutuTeìkitutoua2002}.} Besides, \textit{ta{\ꞌ}e} is used to negate the predicate of certain types of nonverbal clauses.

\subparagraph{\ref{sec:10.5.6}.1} \textit{Ta{\ꞌ}e} negates noun phrases which are the \textsc{predicate} of a nonverbal clause. This may be a classifying clause with a \textit{he}{}-marked predicate (\sectref{sec:9.2.1}) as in \REF{ex:10.139},\footnote{\label{fn:501}There is a difference between:
\ea
\textit{Ta{\ꞌ}e he taŋata} \textup{‘It is not a man’ (classifying)}
\z
\ea
\textit{{\ꞌ}Ina he taŋata} \textup{‘There is no man’ (existential, \REF{ex:10.102} in \sectref{sec:10.5.1})}\z} or an identifying clause with a \textit{ko}{}-marked predicate (\sectref{sec:9.2.2}) as in \REF{ex:10.140}.

\ea\label{ex:10.139}
\gll \textbf{Ta{\ꞌ}e} he mōrore te poki nei, {\ꞌ}ā{\ꞌ}au mau te poki nei. \\
\textsc{conneg} \textsc{pred} bastard \textsc{art} child \textsc{prox} \textsc{poss.2sg.a} really \textsc{art} child \textsc{prox} \\

\glt 
‘This child is not a bastard, the child is your own.’ \textstyleExampleref{[Ley-2-07.027]}
\z

\ea\label{ex:10.140}
\gll \textbf{Ta{\ꞌ}e} ko Reŋa Roiti ta{\ꞌ}a me{\ꞌ}e ena. \\
\textsc{conneg} \textsc{prom} Renga Roiti \textsc{poss.2sg.a} thing \textsc{med} \\

\glt
‘That one (lit. ‘your thing’) is not Renga Roiti.’ \textstyleExampleref{[Ley-9-56.092]}
\z

\textit{Ta{\ꞌ}e} does not negate nouns as such: nouns are negated with \textit{kore} (\sectref{sec:10.5.7}). 

\subparagraph{\ref{sec:10.5.6}.2} \textit{Ta{\ꞌ}e} negates \textsc{other phrases}: prepositional phrases serving as arguments in a verbal clause as in (\ref{ex:10.141}–\ref{ex:10.142}), prepositional predicates as in \REF{ex:10.143}, possessive predicates as in \REF{ex:10.144}:

\ea\label{ex:10.141}
\gll ¡\textbf{Ta{\ꞌ}e} ho{\ꞌ}i ki a koe a au i vānaŋa atu ai! \\
~\textsc{conneg} indeed to \textsc{prop} 2S \textsc{prop} \textsc{1sg} \textsc{pfv} talk away \textsc{pvp} \\

\glt 
‘It wasn’t to you I was talking!’ \textstyleExampleref{[R315.135]} 
\z

\ea\label{ex:10.142}
\gll ...mahana va{\ꞌ}ai era i te mauku, \textbf{ta{\ꞌ}e} i te henua \\
~~~day give \textsc{dist} \textsc{acc} \textsc{art} grass \textsc{conneg} \textsc{acc} \textsc{art} land \\

\glt 
‘the day when (king Atamu Tekena) gave the vegetation (to the Chileans), (but) not the land’ \textstyleExampleref{[R649.172]} 
\z

\ea\label{ex:10.143}
\gll Tō{\ꞌ}oku hare \textbf{ta{\ꞌ}e} a te ara ko Tu{\ꞌ}u Kōihu. \\
\textsc{poss.1sg.o} house \textsc{conneg} by \textsc{art} road \textsc{prom} Tu’u Koihu \\

\glt 
‘My house is not by the road Tu’u Koihu.’ \textstyleExampleref{[Notes]}
\z

\ea\label{ex:10.144}
\gll Te hare nei, \textbf{ta{\ꞌ}e} ō{\ꞌ}oku. \\
\textsc{art} house \textsc{prox} \textsc{conneg} \textsc{poss.1sg.o} \\

\glt 
‘This house is not mine.’ \textstyleExampleref{[R229.268]} 
\z

\subparagraph{\ref{sec:10.5.6}.3} \textit{Ta{\ꞌ}e} negates \textsc{nominalised verbs}\is{Verb!nominalised}:

\ea\label{ex:10.145}
\gll Kai puē tako{\ꞌ}a a ia mo hāpī {\ꞌ}o te \textbf{ta{\ꞌ}e} rava o te moni.\\
\textsc{neg.pfv} can also \textsc{prop} \textsc{3sg} for learn because\_of \textsc{art} \textsc{conneg} sufficient of \textsc{art} money\\

\glt 
‘He could not study as well (like his brother), because there was not enough money (lit. because of the not sufficient of the money).’ \textstyleExampleref{[R231.006]} 
\z

\ea\label{ex:10.146}
\gll ¿Ko take{\ꞌ}a {\ꞌ}ā e koe tu{\ꞌ}u \textbf{ta{\ꞌ}e} hakaroŋo ena? \\
~\textsc{prf} see \textsc{cont} \textsc{ag} \textsc{2sg} \textsc{poss.2sg.o} \textsc{conneg} listen \textsc{med} \\

\glt 
‘Do you see how disobedient you were (lit. your not listening)?’ \textstyleExampleref{[R481.117]} 
\z

\subparagraph{\ref{sec:10.5.6}.4} \textit{Ta{\ꞌ}e} negates \textsc{subconstituents}, such as adjectives \REF{ex:10.147} and quantifiers\is{Quantifier} \REF{ex:10.148} in the noun phrase.

\ea\label{ex:10.147}
\gll A Hiero poki \textbf{ta{\ꞌ}e} porio ni \textbf{ta{\ꞌ}e} pāpaku. \\
\textsc{prop} Hiero child \textsc{conneg} fat nor \textsc{conneg} thin \\

\glt 
‘Hiero was neither a fat nor a skinny child.’ \textstyleExampleref{[R315.020]} 
\z

\ea\label{ex:10.148}
\gll Hora nei \textbf{ta{\ꞌ}e} ta{\ꞌ}ato{\ꞌ}a taŋata {\ꞌ}ite o ruŋa. \\
time \textsc{prox} \textsc{conneg} all person know of above \\

\glt 
‘Nowadays, not all people know about it.’ \textstyleExampleref{[R647.206]} 
\z

\subparagraph{\ref{sec:10.5.6}.5} \textit{Ta{\ꞌ}e} also occurs in the verb phrase. It negates \textsc{subordinate clauses} introduced by a subordinating marker. These markers are in the same position as aspectuals (\sectref{sec:11.5}); \textit{\mbox{ta{\ꞌ}e}} occurs between the marker and the verb. Below are examples with \textit{mo} ‘to, in order to’ and \textit{ana} ‘irrealis\is{Irrealis}’:

\ea\label{ex:10.149}
\gll {\ꞌ}E {\ꞌ}ina he puē \textbf{mo} \textbf{ta{\ꞌ}e} u{\ꞌ}i atu. \\
and \textsc{neg} \textsc{ntr} can for \textsc{conneg} look away \\

\glt 
‘And I’m not able not to look at you.’ \textstyleExampleref{[R308.023]} 
\z

\ea\label{ex:10.150}
\gll \textbf{Ana} \textbf{ta{\ꞌ}e} hā{\ꞌ}aki mai koe, he tiŋa{\ꞌ}i mātou i a koe. \\
\textsc{irr} \textsc{conneg} inform hither \textsc{2sg} \textsc{ntr} kill \textsc{1pl.excl} \textsc{acc} \textsc{prop} \textsc{2sg} \\

\glt 
‘If you don’t tell us, we will kill you.’ \textstyleExampleref{[Mtx-7-21.030]}
\z

Subordinate clauses without subordinating marker are also negated by \textit{ta{\ꞌ}e}. In these cases, \textit{ta{\ꞌ}e} co-occurs with an aspect marker\is{Aspect marker}, usually \textit{i} or \textit{e}. As in the examples above, \textit{ta{\ꞌ}e} occurs between the marker and the verb. Below are examples of relative clauses\is{Clause!relative} (\ref{ex:10.151}–\ref{ex:10.152}) (the second without aspectual), a temporal clause\is{Clause!temporal} \REF{ex:10.153}, and the conjunction\is{Conjunction} \textit{{\ꞌ}āhani} \REF{ex:10.154}.

\ea\label{ex:10.151}
\gll Te vānaŋa rapa nui ta{\ꞌ}e he me{\ꞌ}e {\ob}e \textbf{ta{\ꞌ}e} haŋa rō {\ꞌ}ā e au\,{\cb}. \\
\textsc{art} word Rapa Nui \textsc{conneg} \textsc{pred} thing {\db}\textsc{ipfv} \textsc{conneg} like \textsc{emph} \textsc{cont} \textsc{ag} \textsc{1sg} \\

\glt 
‘The Rapa Nui language is not something I don’t like.’ \textstyleExampleref{[R648.251]} 
\z

\ea\label{ex:10.152}
\gll A Julio taŋata {\ob}\textbf{ta{\ꞌ}e} {\ꞌ}ite i te haka tere i te vaka\,{\cb}. \\
\textsc{prop} Julio person {\db}\textsc{conneg} know \textsc{acc} \textsc{art} \textsc{caus} run \textsc{acc} \textsc{art} boat \\

\glt 
‘Julio is a man who does not know how to navigate a boat.’ \textstyleExampleref{[R303.151]} 
\z

\ea\label{ex:10.153}
\gll I \textbf{ta{\ꞌ}e} kore era tu{\ꞌ}u tokerau era he mana{\ꞌ}u mo haka tītika  i te vaka ki Tahiti.\\
\textsc{pfv} \textsc{conneg} lack \textsc{dist} \textsc{poss.2sg.o} wind \textsc{dist} \textsc{ntr} think for \textsc{caus} straight  \textsc{acc} \textsc{art} boat to Tahiti\\

\glt 
‘When the wind did not die down, they decided to steer the boat to Tahiti.’ \textstyleExampleref{[R303.064]} 
\z

\ea\label{ex:10.154}
\gll {\ꞌ}Āhani {\ꞌ}ō tō{\ꞌ}oku nua era i \textbf{ta{\ꞌ}e} mate, {\ꞌ}ī au  {\ꞌ}i muri i a ia {\ꞌ}i te hora nei.\\
if\_only really \textsc{poss.1sg.o} Mum \textsc{dist} \textsc{pfv} \textsc{conneg} die \textsc{imm} \textsc{1sg}  at near in \textsc{prop} \textsc{3sg} in \textsc{art} time \textsc{prox}\\

\glt 
‘If my mother had not died, I would be near her at this time.’ \textstyleExampleref{[R245.007]} 
\z

\subparagraph{\ref{sec:10.5.6}.6} Interestingly, \textit{ta{\ꞌ}e} also occurs in the verb phrase in \textsc{main clauses}, mainly with the aspect markers \textit{i}\is{i (perfective)} and \textit{e}\is{e (imperfective)}. This happens when the verb phrase is preceded by an oblique constituent. As suggested in Footnote \ref{fn:420} on p.~\pageref{fn:420}, this preposed constituent acts somewhat like a subordinating predicate.

\ea\label{ex:10.155}
\gll {\ob}Hai {\ꞌ}arero\,{\cb}, pa{\ꞌ}i, e \textbf{ta{\ꞌ}e} ŋaro ena te haka tere iŋa  o te motu nei.\\
{\db}\textsc{ins} tongue in\_fact \textsc{ipfv} \textsc{conneg} lost \textsc{med} \textsc{art} \textsc{caus} run \textsc{nmlz}  of \textsc{art} island \textsc{prox}\\

\glt 
‘By means of the language, the culture of this island will not be lost.’ \textstyleExampleref{[R647.155]} 
\z

\ea\label{ex:10.156}
\gll {\ob}{\ꞌ}O ira\,{\cb} pa{\ꞌ}i i \textbf{ta{\ꞌ}e} ma{\ꞌ}u ai hai me{\ꞌ}e mo kai. \\
{\db}because\_of \textsc{ana} in\_fact \textsc{pfv} \textsc{conneg} carry \textsc{pvp} \textsc{ins} thing for eat \\

\glt 
‘Therefore they didn’t take anything to eat.’ \textstyleExampleref{[R303.053]} 
\z

\ea\label{ex:10.157}
\gll {\ob}Mai rā mahana\,{\cb} i \textbf{ta{\ꞌ}e} aŋa haka{\ꞌ}ou ai.\\
{\db}from \textsc{dist} day \textsc{pfv} \textsc{conneg} work again \textsc{pvp}\\

\glt 
‘From that day on, she did not work any more.’ \textstyleExampleref{[R441.005]} 
\z

\ea\label{ex:10.158}
\gll ¿{\ob}He aha\,{\cb} e \textbf{ta{\ꞌ}e} aŋa rivariva ena i te rāua aŋa? \\
~~\textsc{pred} what \textsc{ipfv} \textsc{conneg} do good \textsc{med} \textsc{acc} \textsc{art} \textsc{3pl} work \\

\glt
‘Why don’t they do their work well?’ \textstyleExampleref{[R648.249]} 
\z

However, in such cases, main clause negators\is{Negation} are also used. This is illustrated in \REF{ex:10.117} above and in the following example:

\ea\label{ex:10.159}
\gll {\ꞌ}O ira, \textbf{{\ꞌ}ina} \textbf{e} \textbf{ko} ŋaro te kaikai. \\
because\_of \textsc{ana} \textsc{neg} \textsc{ipfv} \textsc{neg.ipfv} lost \textsc{art} string\_figure \\

\glt 
‘Therefore, the (art of making) string figures will not be lost.’ \textstyleExampleref{[R648.133]} 
\z

\subparagraph{10.5.6.7.} Finally, \textit{ta{\ꞌ}e} is used in combination with the other negators\is{Negation} to express \textsc{double negation}; \textit{ta{\ꞌ}e} and the other negator cancel each other out, resulting in a strong affirmation. The other negator may be \textit{kai}\is{kai (negator)} or \textit{e ko}\is{e ko (negator)}; as \REF{ex:10.161} shows, it may be reinforced by \textit{{\ꞌ}ina}\is{ina (negator)@{\ꞌ}ina (negator)}.

\ea\label{ex:10.160}
\gll \textbf{Kai} \textbf{ta{\ꞌ}e} haka {\ꞌ}ite ko ai a ia hai me{\ꞌ}e rivariva aŋa. \\
\textsc{neg.pfv} \textsc{conneg} \textsc{caus} know \textsc{prom} who \textsc{prop} \textsc{3sg} \textsc{ins} thing good:\textsc{red} do \\

\glt 
‘(God) did not fail to make known who he is, by the good things he did.’ (\textstyleExampleref{Acts 14:17]}\textstyleExampleref{} 
\z

\ea\label{ex:10.161}
\gll ...\textbf{{\ꞌ}ina} \textbf{e} \textbf{ko} \textbf{ta{\ꞌ}e} rava{\ꞌ}a te ika.\\
~~~\textsc{neg} \textsc{ipfv} \textsc{neg.ipfv} \textsc{conneg} obtain \textsc{art} fish\\

\glt 
‘(If the mother does not eat the fish caught by her firstborn son,) he will not fail to catch fish.’ \textstyleExampleref{[Ley-5-27.008]}
\z
\is{tae (negator)@ta{\ꞌ}e (negator)|)}
\subsection{The negator \textit{kore}}\label{sec:10.5.7}
\is{kore (negator)|(}
\textit{Kore}\footnote{\label{fn:502}\textit{Kore} is common in \is{Eastern Polynesian}EP languages; in all languages but Rapa Nui, it is either an existential negation (‘there is not’), or negates certain types of verbal clauses. In Rapa Nui, existential clauses\is{Clause!existential} are negated with \textit{{\ꞌ}ina} (\sectref{sec:10.5.1}). \textit{Kore} has the more specific sense ‘to be lacking’. It does not occur in non-EP languages; outside Polynesian, \citet[98]{Clark1976} mentions a verb \textit{ore} in \ili{Sa’a} (Solomon Islands) ‘to remain behind’ and \ili{Lau} (Fiji) ‘to fail, lack’. This may suggest that \textit{kore} originated as a verb meaning ‘to lack’ and developed into something more like a negator in \is{Central-Eastern Polynesian}PCE (\citealt[101–102]{Clark1976}).} is a verb, meaning ‘to lack, be absent, be gone’:

\ea\label{ex:10.162}
\gll E ko \textbf{kore} te {\ꞌ}ura era mā nīrā. \\
\textsc{ipfv} \textsc{neg.ipfv} lack \textsc{art} lobster \textsc{dist} for today.\textsc{fut} \\

\glt 
‘The lobster won’t be lacking (=we will have plenty of lobster) for today.’ \textstyleExampleref{[R230.033]} 
\z

\ea\label{ex:10.163}
\gll He u{\ꞌ}i, ku \textbf{kore} {\ꞌ}ā te taŋi. \\
\textsc{ntr} look \textsc{prf} lack \textsc{cont} \textsc{art} cry \\

\glt 
‘He looked (at his wife); the crying was over.’ \textstyleExampleref{[Ley-9-55.076]}
\z

Besides, \textit{kore} is used to negate nouns, indicating that the entity expressed by the noun does not exist in the given context; as a noun negator it immediately follows the noun in adjective position. When the noun is a modifier as in \REF{ex:10.164}, \textit{kore} can be translated as ‘without N’; in other cases as in (\ref{ex:10.165}–\ref{ex:10.166}), it can be translated as ‘lack of N’:

\ea\label{ex:10.164}
\gll Te ŋā poki \textbf{matu{\ꞌ}a} \textbf{kore} era o koā Eugenio te hāpa{\ꞌ}o.\\
\textsc{art} \textsc{pl} child parent lack \textsc{dist} of \textsc{coll} Eugenio \textsc{art} care\_for\\

\glt 
‘Children without parents, Eugenio and the others took care of them.’ \textstyleExampleref{[R231.308]} 
\z

\ea\label{ex:10.165}
\gll Te {\ꞌ}ati he \textbf{matariki} \textbf{kore} mo oro o rā hora. \\
\textsc{art} problem \textsc{pred} file lack for grate of \textsc{dist} time \\

\glt 
‘The problem was the lack of files to sharpen (the fishhooks) at the time.’ \textstyleExampleref{[R539-1.335]}
\z

\ea\label{ex:10.166}
\gll Ko pakiroki {\ꞌ}ā te taŋata {\ꞌ}i te \textbf{kai} \textbf{kore}.\\
\textsc{prf} thin \textsc{cont} \textsc{art} person at \textsc{art} food lack\\

\glt 
‘The people were skinny from lack of food.’ \textstyleExampleref{[R372.025]} 
\z
\is{kore (negator)|)}

\subsection{\textit{Hia}/\textit{ia} ‘not yet’}\label{sec:10.5.8}
\is{hia ‘not yet’|(}
\textit{Hia} (etymology unknown) is used after negated verbs; the sense of the negator + \textit{hia} is ‘not yet’. \textit{Hia} occurs immediately after the verb, before directionals\is{Directional}:

\ea\label{ex:10.167}
\gll ¡Kai topa \textbf{hia} atu {\ꞌ}ō tā{\ꞌ}aku vānaŋa koe i pāhono rō mai ai! \\
~\textsc{neg.pfv} descend yet away really \textsc{poss.1sg.a} word \textsc{2sg} \textsc{pfv} answer \textsc{emph} hither \textsc{pvp} \\

\glt
‘I hadn’t finished speaking yet when you answered!’ \textstyleExampleref{[R314.139]} 
\z

\textit{Hia} may occur with any negator: \textit{kai} as in \REF{ex:10.167} above, \textit{e ko} \REF{ex:10.168} or \textit{ta{\ꞌ}e} \REF{ex:10.169}.

\ea\label{ex:10.168}
\gll \textbf{E} \textbf{ko} \textbf{{\ꞌ}o{\ꞌ}oa} \textbf{hia} te moa ka kī ena e koe e toru kī iŋa  kai {\ꞌ}ite koe ko ai a au.\\
\textsc{ipfv} \textsc{neg.ipfv} crow yet \textsc{art} chicken \textsc{cntg} say \textsc{med} \textsc{ag} \textsc{2sg} \textsc{num} three say \textsc{nmlz}  \textsc{neg.pfv} know \textsc{2sg} \textsc{prom} who \textsc{prop} \textsc{1sg}\\

\glt 
‘Before the rooster crows, you will say three times that you don’t know who I am.’ \textstyleExampleref{[Jhn. 13:38]}
\z

\ea\label{ex:10.169}
\gll He {\ꞌ}a{\ꞌ}amu, mata \textbf{ta{\ꞌ}e} \textbf{{\ꞌ}ite} \textbf{hia} pē nei ē: he tahutahu. \\
\textsc{ntr} tell while \textsc{conneg} know yet like \textsc{prox} thus \textsc{pred} witch \\

\glt
‘She told it, without knowing yet that (the other person) was a witch.’ \textstyleExampleref{[R532-07.044]}
\z

As (\ref{ex:10.167}–\ref{ex:10.168}) show, \textit{hia} is often used when an action or event is interrupted by another event. In these cases, the function of the negator + \textit{hia} is similar to a temporal marker ‘before’.

Sometimes the variant \textit{ia}\is{ia ‘not yet’} is found. This should not be confused with the sentential particle \textit{ia} ‘then’ (\sectref{sec:4.5.4.1}): while the latter occurs after the verb phrase, \textit{ia} ‘yet’ occurs before other postverbal particles:

\ea\label{ex:10.170}
\gll \textbf{Kai} \textbf{tomo} \textbf{ia} mai {\ꞌ}ā {\ꞌ}i te ahiahi i {\ꞌ}ite tahi rō ai te {\ꞌ}uta i tū parau {\ꞌ}āpī era.\\
\textsc{neg.pfv} go\_ashore yet hither \textsc{cont}\textsc{} at \textsc{art} afternoon \textsc{pfv} know all \textsc{emph} \textsc{pvp} \textsc{art} inland \textsc{acc} \textsc{dem} word new \textsc{dist}\\

\glt 
‘They had not arrived yet in the afternoon when all people ashore knew the news.’ \textstyleExampleref{[R345.015]}\textstyleExampleref{} 
\z
\is{Negation|)}\is{hia ‘not yet’|)}

\newpage 
\section{Conclusions}\label{sec:10.6}

In this chapter, non-declarative moods have been discussed, as well as negation.

Two aspect markers serve to mark imperatives: the contiguity marker \textit{ka} is used for direct commands, imperfective \textit{e} for indirect commands. Though imperatives usually occur in the second person (often with explicit subject), they may occur in the third person as well. For first-person injunctions (e.g. exhortations), the purpose marker \textit{ki} is used. 

Polar questions usually do not have any special marking. Sometimes they are marked with the question marker \textit{hoki}; in addition, the particles \textit{{\ꞌ}ō} and \textit{hō} may be used to add a note of counterexpectation or doubt, respectively.

Content questions are marked by four question words, each of which belongs to a different word class:

\begin{itemize}
\item 
\textit{Ai} ‘who’ is a proper noun; it is often used in an identifying cleft construction, preceded by the default preposition \textit{ko};

\item 
\textit{Aha} ‘what’ is a common noun; it is often used in a classifying cleft construction, preceded by the predicate marker \textit{he};

\item 
\textit{Hē} ‘where, when, how, which’ is a locational; it is preceded by a preposition, without a determiner;

\item 
\textit{Hia} ‘how many’ is a numeral; it is preceded by a numeral particle.

\end{itemize}

Rapa Nui has three main clause negators: neutral \textit{{\ꞌ}ina}, perfective \textit{kai} and imperfective \textit{(e) ko}. \textit{{\ꞌ}Ina} is a phrase head; it may seem to have some properties of a predicate (e.g. triggering subject raising), but the same is true for a number of other clause-initial elements, such as deictic particles, while \textit{{\ꞌ}ina} lacks crucial features of a predicate.

The other two negators are preverbal markers; they are often combined with \textit{{\ꞌ}ina}.

All units other than main clauses are negated by \textit{ta{\ꞌ}e}: noun phrases, nominalised verbs, subconstituents and subordinate clauses. \textit{Ta{\ꞌ}e} is also used to negate certain types of main clauses: those which have an \textit{e}{}- or \textit{i}{}-marked verb, preceded by an initial oblique constituent. This suggests that these clauses have some features of subordinate clauses: the initial oblique functions as a kind of matrix predicate (see Footnote \ref{fn:420} on p.~\pageref{fn:420}).
