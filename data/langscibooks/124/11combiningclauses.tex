\chapter[Combining clauses]{Combining clauses}\label{ch:11}
\section{Introduction}\label{sec:11.1}

Clauses can be combined in several ways. Two or more main clauses can be linked by juxtaposition or by using a coordinating conjunction\is{Conjunction} (\sectref{sec:11.2}). Alternatively, one clause may contain another as subordinate clause: various categories of verbs take a clausal complement (\sectref{sec:11.3}); nouns may be modified by a relative clause\is{Clause!relative} (\sectref{sec:11.4}); adverbial clauses serve as an adjunct in a main clause (\sectref{sec:11.6}).

In Rapa Nui, different strategies are used to combine clauses, depending on the type of clause. Some constructions have a conjunction, others have a preverbal subordinating \is{Conjunction}marker, others yet are unmarked.\footnote{\label{fn:503}The latter can be recognised as subordinate clauses by the use of the negator \textit{ta{\ꞌ}e} (see \REF{ex:11.210} on p.~\pageref{ex:11.210}), and by the use of aspectual marker (only \textit{i}, \textit{e} or \textit{ka}; the verb is always followed by a postverbal demonstrative).} Conjunctions only occur in certain types of adverbial clauses and will be discussed in the appropriate subsections of \sectref{sec:11.6}. Preverbal markers cut through the distinction between types of subordinate clauses, therefore they are discussed separately in \sectref{sec:11.5}.

\section{Coordination} \label{sec:11.2}
\is{Coordination|(}\subsection{Asyndetic and syndetic coordination}\label{sec:11.2.1}

Old Rapa Nui did not have a coordinating conjunction\is{Conjunction}. Both phrases and clauses were linked by simple juxtaposition (i.e. asyndetic coordination, see \citealt[7]{Haspelmath2007}). \REF{ex:11.1} shows juxtaposed clauses, while \REF{ex:11.2} contains a string of juxtaposed noun phrases.

\ea\label{ex:11.1}
\gll He oho a te ara, he tike{\ꞌ}a te kohe; he rei hai va{\ꞌ}e, he hati te kohe... \\
\textsc{ntr} go by \textsc{art} road \textsc{ntr} see \textsc{art} \textit{kohe} \textsc{ntr} step \textsc{ins} foot \textsc{ntr} break \textsc{art} \textit{kohe} \\

\glt 
‘He went along the road, he saw a \textit{kohe} plant; he stepped on it, the \textit{kohe} broke...’ \textstyleExampleref{[Ley-2-01.018]}
\z

\ea\label{ex:11.2}
\gll He māmate te taŋata, te vi{\ꞌ}e, te poki, te korohu{\ꞌ}a. \\
\textsc{ntr} \textsc{pl}:die \textsc{art} man \textsc{art} woman \textsc{art} child \textsc{art} old\_man \\

\glt
‘Men, women, children, old people died.’ \textstyleExampleref{[Ley-2-01.010]}
\z

Sometimes the adverbs \textit{tako{\ꞌ}a}\is{tako{\ꞌ}a ‘also’} and \textit{hoki}\is{hoki ‘also’} ‘also’ (\sectref{sec:4.5.3.2}–\ref{sec:4.5.3.3}) are used to link clauses or phrases. In \REF{ex:11.3}, two clauses with similar information about different participants are linked with \textit{tako{\ꞌ}a}. In \REF{ex:11.4}, the last item in a list of noun phrases is marked with \textit{hoki}. The latter happens only in older texts.

\ea\label{ex:11.3}
\gll He to{\ꞌ}o Hereveri i tō{\ꞌ}ona o te tīta{\ꞌ}a henua, he to{\ꞌ}o \textbf{tako{\ꞌ}a}  Te Roŋo i tō{\ꞌ}ona o te tīta{\ꞌ}a.\\
\textsc{ntr} take Hereveri \textsc{acc} \textsc{poss.3sg.o} of \textsc{art} terrain land \textsc{ntr} take also  Te Rongo \textsc{acc} \textsc{poss.3sg.o} of \textsc{art} terrain\\

\glt 
‘Hereveri took his piece of land; Te Rongo took his piece of land as well.’ \textstyleExampleref{[Egt-02.045]}
\z

\ea\label{ex:11.4}
\gll {\ꞌ}I te tapa te matu{\ꞌ}a, a koro, a nua, te uka riva,  te repa riva \textbf{hoki}.\\
at \textsc{art} side \textsc{art} parents \textsc{prop} Dad \textsc{prop} Mum \textsc{prop} girl good  \textsc{prop} young\_man good also\\

\glt 
‘To the side are the parents, the fathers, the mothers, the pretty girls, also the handsome boys.’ \textstyleExampleref{[Ley-5-24.013]}
\z

In modern Rapa Nui, the conjunction\is{Conjunction} \textit{{\ꞌ}e}\is{e ‘and’@{\ꞌ}e ‘and’|(} ‘and’ (probably a \ili{Tahitian} borrowing\is{Tahitian influence}\footnote{\label{fn:504}Concerning the origin of \textit{{\ꞌ}e} in \ili{Tahitian}: although it is phonologically identical to \ili{French} ‘et’, the fact that \textit{{\ꞌ}e} is already common in the 1838 \ili{Tahitian} Bible translation (\citealt{TeBibilia1996}) indicates that it predates \ili{French} influence. A similar conjunction\is{Conjunction} (spelled \textit{e}, \textit{{\ꞌ}e} or \textit{ē}) occurs in \ili{Pa’umotu}, \ili{Rarotongan} and \ili{Mangarevan}, but not in \ili{Marquesan} and \ili{Māori}.}) is used to link clauses and phrases; it occurs in clause- and phrase-initial position.
% \todo[inline]{In fn. 2, "TeBibilia1996" is quoted. It would be nice to have a short reference such as "Te Bibilia 1996", but the system includes the whole title with translation. Is there some option for quoting an abbreviated version?}

In old texts \textit{{\ꞌ}e} is found a few times in Mtx, but not in other corpora (Ley and MsE); this suggests that \textit{{\ꞌ}e} was emerging in the 1930s. In newer texts, it occurs over 3,000 times; this can be (partially) explained by changing speaking and writing styles under the influence of \ili{Spanish} and other languages.

Even though \textit{{\ꞌ}e} is very common nowadays, the most common strategy for linking clauses is still juxtaposition. Juxtaposition is especially used to link clauses referring to successive events in discourse. For example, in narrative, sequences such as the following are common:

\ea\label{ex:11.5}
\gll He tahuti a Eva ki haho, he oŋa ki te vaka, he take{\ꞌ}a tō{\ꞌ}ona koro.\\
\textsc{ntr} run \textsc{prop} Eva to outside \textsc{ntr} look to \textsc{art} boat \textsc{ntr} see \textsc{poss.3sg.o} Dad\\

\glt
‘Eva ran outside, stared at the boat, saw her Dad.’ \textstyleExampleref{[R210.095]} 
\z

In other situations, the conjunction\is{Conjunction} \textit{{\ꞌ}e} tends to be used. \textit{{\ꞌ}E} is common in the following situations (the list is not exhaustive, and neither are these categories mutually exclusive):

%\setcounter{listWWviiiNumcxxvileveli}{0}
\begin{enumerate}
\item 
To mark the final event in a series of three or more events:
\end{enumerate}

\ea\label{ex:11.6}
\gll Ka oho nō koe ka kai \textbf{{\ꞌ}e} ka ha{\ꞌ}uru. \\
\textsc{imp} go just \textsc{2sg} \textsc{imp} eat and \textsc{imp} sleep \\

\glt 
‘Just go, eat and sleep.’ \textstyleExampleref{[R304.013]} 
\z

\ea\label{ex:11.7}
\gll He e{\ꞌ}a haka{\ꞌ}ou a Manutara mai tou hare era he oho \textbf{{\ꞌ}e} he tu{\ꞌ}u  ki te hare o tō{\ꞌ}ona rua taina ko {\ꞌ}Antonio.\\
\textsc{ntr} go\_out again \textsc{prop} Manutara from \textsc{dem} house \textsc{dist} \textsc{ntr} go and \textsc{ntr} arrive  to \textsc{art} house of \textsc{poss.3sg.o} two brother \textsc{prom} Antonio\\

\glt
‘Manutara went out again from the house, he went and arrived at the house of his other brother Antonio.’ \textstyleExampleref{[R309.083]} 
\z

\begin{enumerate}
\setcounter{enumi}{1}
\item 
To link a pair of clauses not referring to successive events; these clauses are often parallel in some way and may involve a contrast between two items:
\end{enumerate}

\ea\label{ex:11.8}
\gll Te {\ꞌ}āriŋa he taŋata mau ena, \textbf{{\ꞌ}e} te hakari he kavakava. \\
\textsc{art} face \textsc{pred} person really \textsc{med} and \textsc{art} body \textsc{ntr} rib \\

\glt 
‘Their faces were like (normal) people, but their bodies were ribs.’ \textstyleExampleref{[R233.021]} 
\z

\ea\label{ex:11.9}
\gll Hora maha nei, \textbf{{\ꞌ}e} hora hitu tātou ka tu{\ꞌ}u iho. \\
hour four \textsc{prox} and hour seven \textsc{1pl.incl} \textsc{cntg} arrive just\_then \\

\glt
‘It is now four o’clock, and seven o’clock we will arrive.’ \textstyleExampleref{[R210.198]} 
\z

\begin{enumerate}
\setcounter{enumi}{2}
\item 
To link subordinate clauses:
\end{enumerate}


\ea\label{ex:11.10}
\gll He hoki koe mo haka mao i tu{\ꞌ}u hāpī \textbf{{\ꞌ}e} mo haka tītika  te aŋa o te misione.\\
\textsc{ntr} return \textsc{1sg} for \textsc{caus} finish \textsc{acc} \textsc{poss.2sg.o} learn and for \textsc{caus} straight  \textsc{art} work of \textsc{art} mission\\

\glt
‘You will return to finish your studies and to direct the mission work.’ \textstyleExampleref{[R231.244]} 
\z

\begin{enumerate}
\setcounter{enumi}{3}
\item 
To indicate a larger break in a sentence. This often involves a shift to a different type of information (indicated by a different aspect marker\is{Aspect marker}) or a shift in subject:
\end{enumerate}

\ea\label{ex:11.11}
\gll E ma{\ꞌ}u mai {\ꞌ}ā a mātou i te rēkaro nei mā{\ꞌ}au,  \textbf{{\ꞌ}e} {\ꞌ}i te hora nei he oho tātou he koa.\\
\textsc{ipfv} carry hither \textsc{cont} \textsc{prop} \textsc{1pl.excl} \textsc{acc} \textsc{art} present \textsc{prox} \textsc{ben.2sg.a}  and at \textsc{art} time \textsc{prox} \textsc{ntr} go \textsc{1pl.incl} \textsc{ntr} happy\\

\glt 
‘We (excl.) bring this present for you, and now we (incl.) will go and have fun.’ \textstyleExampleref{[R210.127]} 
\z

\ea\label{ex:11.12}
\gll He noho rō {\ꞌ}ai a nua he u{\ꞌ}i i te ŋā poki,  \textbf{{\ꞌ}e} hoko tahi nō a koro e iri era ki {\ꞌ}uta.\\
\textsc{ntr} stay \textsc{emph} \textsc{subs} \textsc{prop} Mum \textsc{ntr} look \textsc{acc} \textsc{art} \textsc{pl} child  and \textsc{num.pers} one just \textsc{prop} Dad \textsc{ipfv} ascend \textsc{dist} to inland\\

\glt
‘Mum stayed (and) looked after the children, and Dad went up to the field on his own.’ \textstyleExampleref{[R235.080]} 
\z


When two clauses are both under the scope of a single initial constituent, they are usually juxtaposed without conjunction\is{Conjunction} and without repetition of the initial constituent. Examples are \textit{{\ꞌ}o ira} ‘therefore’ in \REF{ex:11.13} and the interrogative phrase in \REF{ex:11.14}. As \REF{ex:11.14} also shows, verb phrase particles – both the aspectual and the negator \textit{ta{\ꞌ}e} – are repeated in the second clause.

\ea\label{ex:11.13}
\gll {\ꞌ}E \textbf{{\ꞌ}o} \textbf{ira} a mātou i tu{\ꞌ}u mai nei i {\ꞌ}auario nei  i te mōai nei.\\
and because\_of \textsc{ana} \textsc{prop} \textsc{1pl.excl} \textsc{pfv} arrive hither \textsc{prox} \textsc{pfv} guard \textsc{prox}  \textsc{acc} \textsc{art} statue \textsc{prox}\\

\glt 
‘And therefore we have come and put this statue under guard.’ \textstyleExampleref{[R650.034]} 
\z

\ea\label{ex:11.14}
\gll ¿\textbf{He} \textbf{aha} rā ia kōrua \textbf{i} \textbf{ta{\ꞌ}e} oho mai ai  \textbf{i} \textbf{ta{\ꞌ}e} hā{\ꞌ}aki mai ai...?\\
\textsc{pred} what \textsc{intens} then \textsc{2pl} \textsc{pfv} \textsc{conneg} go hither \textsc{pvp}  \textsc{pfv} \textsc{conneg} inform hither \textsc{pvp}\\

\glt 
‘Why didn’t you come and tell me...?’ \textstyleExampleref{[R313.106]} 
\z

When two noun phrases are coordinated in modern Rapa Nui, they are usually linked with \textit{{\ꞌ}e}. When the list is longer than two as in \REF{ex:11.16}, \textit{{\ꞌ}e} occurs only before the last item; the other items are juxtaposed:

\ea\label{ex:11.15}
\gll {\ꞌ}E tako{\ꞌ}a e ai rō {\ꞌ}ana te tenito \textbf{{\ꞌ}e} te europeo noho {\ꞌ}i Tahiti. \\
and also \textsc{ipfv} exist \textsc{emph} \textsc{cont} \textsc{art} Chinese and \textsc{art} European stay at Tahiti \\

\glt 
‘And there are also Chinese and Europeans living on Tahiti.’ \textstyleExampleref{[R348.011]} 
\z

\ea\label{ex:11.16}
\gll He marere he oho rō {\ꞌ}ai te pipihoreko, te manavai  \textbf{{\ꞌ}e} te hare moa.\\
\textsc{ntr} scatter \textsc{ntr} go \textsc{emph} \textsc{subs} \textsc{art} cairn \textsc{art} rock\_garden  and \textsc{art} house chicken\\

\glt
‘The rock piles, the rock gardens and the chicken houses gradually fell apart.’ \textstyleExampleref{[R621.018]} 
\z

When noun phrases marked with prepositions are coordinated, the preposition is repeated, including the accusative marker\is{i (accusative marker)} \textit{i}; the last item may be preceded by \textit{{\ꞌ}e} as in (\ref{ex:11.17}–\ref{ex:11.18}), but juxtaposition is also common, as in (\ref{ex:11.19}–\ref{ex:11.20}):

\ea\label{ex:11.17}
\gll {\ꞌ}I roto i te piha nei a kōrua ka hāpī ena \textbf{i} te tai{\ꞌ}o  \textbf{{\ꞌ}e} \textbf{i} te pāpa{\ꞌ}i.\\
at inside at \textsc{art} room \textsc{prox} \textsc{prop} \textsc{2pl} \textsc{cntg} learn \textsc{med} \textsc{acc} \textsc{art} read  and \textsc{acc} \textsc{art} write\\

\glt 
‘In this (class)room you will learn to read and to write.’ \textstyleExampleref{[R334.043]} 
\z

\ea\label{ex:11.18}
\gll {\ꞌ}I roto i te māhatu \textbf{o} tā{\ꞌ}ana vi{\ꞌ}e \textbf{{\ꞌ}e} \textbf{o} tā{\ꞌ}ana ŋā poki... \\
at inside at \textsc{art} heart of \textsc{poss.3sg.a} woman and of \textsc{poss.3sg.a} \textsc{pl} child \\

\glt 
‘In the heart of his wife and of his children...’ \textstyleExampleref{[R649.087]} 
\z

\ea\label{ex:11.19}
\gll Kā ŋā poki he ma{\ꞌ}u \textbf{i} te keke, \textbf{i} te haraoa, \textbf{i} te me{\ꞌ}e.\\
every \textsc{pl} child \textsc{ntr} carry \textsc{acc} \textsc{art} cake \textsc{acc} \textsc{art} bread \textsc{acc} \textsc{art} thing\\

\glt 
‘All the children carried cakes, bread and (other) things.’ \textstyleExampleref{[R165.001]} 
\z

\ea\label{ex:11.20}
\gll He hiro i te hau mo hī \textbf{o} te kahi \textbf{o} te ika. \\
\textsc{ntr} braid \textsc{acc} \textsc{art} line for to\_fish of \textsc{art} tuna of \textsc{art} fish \\

\glt 
‘He braided lines for fishing tuna (and) (other) fish.’ \textstyleExampleref{[R310.020]}\textstyleExampleref{} 
\z
\is{e ‘and’@{\ꞌ}e ‘and’|)}
In modern Rapa Nui, \ili{Spanish} \textit{pero}\is{pero ‘but’} ‘but’ is often used as adversative conjunction: 

\ea\label{ex:11.21}
\gll He ma{\ꞌ}u mai he tunu i te māmoe \textbf{pero} kai mākona tū nu{\ꞌ}u era.\\
\textsc{ntr} carry hither \textsc{ntr} cook \textsc{acc} \textsc{art} sheep but \textsc{neg.pfv} satiated \textsc{dem} people \textsc{dist}\\

\glt
‘He carried the sheep and cooked it, but the people were not satiated.’ \textstyleExampleref{[R183.033]} 
\z

Despite its frequent use in everyday speech, \textit{pero} is perceived as an intrusion, as witnessed by the fact that it is little used in the written texts in the corpus. In the Bible translation, it is not used at all. As \REF{ex:11.8} above shows, \textit{{\ꞌ}e} is also used in situations where other languages would have an adversative conjunction.

\subsection{Disjunction}\label{sec:11.2.2}

In old texts, disjunction is expressed by juxtaposition:

\ea\label{ex:11.22}
\gll He tia i te nua hai ivi manu, ivi moa, ivi taŋata.\\
\textsc{ntr} sew \textsc{acc} \textsc{art} cape with bone bird bone chicken bone man\\

\glt
‘(The women of old) sewed capes with (needles made of) bird bones, chicken bones (or) human bones.’ \textstyleExampleref{[Ley-5-04.013]}
\z

In modern Rapa Nui, disjunction is expressed by \textit{{\ꞌ}o} ‘or’\is{o ‘or’@{\ꞌ}o ‘or’|(},\footnote{\label{fn:505}This particle should not be confused with preverbal \textit{{\ꞌ}o} ‘lest’ (\sectref{sec:11.5.4}), or with the preposition \textit{{\ꞌ}o} ‘because of’ (\sectref{sec:4.7.2.2}).} a conjunction borrowed from \ili{Spanish} \textit{o}. \textit{{\ꞌ}O} may connect clauses as in (\ref{ex:11.23}–\ref{ex:11.24}) or phrases as in \REF{ex:11.25}:

\ea\label{ex:11.23}
\gll Te ŋā kai {\ꞌ}āpī ra{\ꞌ}e era ana momore, ana pa{\ꞌ}o, \textbf{{\ꞌ}o} ana keri, e ma{\ꞌ}u  to te hare pure {\ꞌ}i ra{\ꞌ}e.\\
\textsc{ntr} \textsc{pl} food new first \textsc{dist} \textsc{irr} \textsc{red}:cut \textsc{irr} chop or \textsc{irr} dig \textsc{ipfv} carry  \textsc{art}:of \textsc{art} house pray at first\\

\glt 
‘The first new food which would be picked, cut or dug up, had to be taken to the church first.’ \textstyleExampleref{[R539-3.150]}
\z

\ea\label{ex:11.24}
\gll ...he oho \textbf{{\ꞌ}o} he hāpī \textbf{{\ꞌ}o} he e{\ꞌ}a he ha{\ꞌ}ere \textbf{{\ꞌ}o} he oho ki kampō.\\
~~~\textsc{ntr} go or \textsc{ntr} learn or \textsc{ntr} go\_out \textsc{ntr} walk or \textsc{ntr} go to countryside\\

\glt 
‘(When his work was finished,) he would go or study or go out for a walk or go to the countryside.’ \textstyleExampleref{[R302.051]} 
\z

\ea\label{ex:11.25}
\gll He {\ꞌ}aiua i te aŋa ki a nua \textbf{{\ꞌ}o} ki a koro. \\
\textsc{ntr} help \textsc{acc} \textsc{art} work to \textsc{prop} Mum or to \textsc{prop} Dad \\

\glt
‘They help Mum or Dad with the work.’ \textstyleExampleref{[R157.001]} 
\z

Unlike \textit{{\ꞌ}e} ‘and’, \textit{{\ꞌ}o} may also connect nouns; in that case, the parts on either side of \textit{{\ꞌ}o} are not complete noun phrases. In the following examples, \textit{{\ꞌ}o} is directly followed by the second noun; prenominal elements, such as determiners and the plural marker \textit{ŋā} in \REF{ex:11.26}, precede the first noun, while the postnominal demonstrative \textit{era} follows the second noun:

\ea\label{ex:11.26}
\gll ...{\ꞌ}i {\ob}te ŋā tāpati \textbf{{\ꞌ}o} {\ꞌ}āva{\ꞌ}e era\,{\cb} e noho era {\ꞌ}i tahatai. \\
~~~at {\db}\textsc{art} \textsc{pl} week or month \textsc{dist} \textsc{ipfv} stay \textsc{dist} at coast \\

\glt 
‘...in the weeks or months they stay on the coast.’ \textstyleExampleref{[R200.047]} 
\z

\ea\label{ex:11.27}
\gll E ma{\ꞌ}u tako{\ꞌ}a koe i {\ob}te me{\ꞌ}e pūtē \textbf{{\ꞌ}o} {\ꞌ}avahata\,{\cb} mo ha{\ꞌ}a{\ꞌ}ī  o ta{\ꞌ}a siera.\\
\textsc{exh} carry also \textsc{2sg} \textsc{acc} {\db}\textsc{art} thing bag or box for fill  of \textsc{poss.2sg.a} sawfish\\

\glt 
‘Also bring a bag or box to put your sawfish in.’ \textstyleExampleref{[R364.031]}\textstyleExampleref{} 
\z
\is{o ‘or’@{\ꞌ}o ‘or’|)}
\is{Coordination|)}
\section{Clausal arguments}\label{sec:11.3}
\is{Clause!complement|(}
This section deals with verbs which take a clausal argument, i.e. an argument containing a predicate. This includes a number of different types of verbs: perception verbs such as \textit{tike{\ꞌ}a} ‘to see’; aspectual verbs such as \textit{ha{\ꞌ}amata} ‘to begin’; cognitive verbs such as \textit{{\ꞌ}ite} ‘to know’; speech verbs such as \textit{kī} ‘to say’, attitude verbs such as \textit{haŋa} ‘to want’; modal verbs such as \textit{puē} ‘can’. These verbs occur in a variety of multiclausal constructions:

%\setcounter{listWWviiiNumxixleveli}{0}
\begin{enumerate}
\item 
Complement clauses introduced by a subordinating marker (usually \textit{mo}):
\end{enumerate}

\ea\label{ex:11.28}
\gll He oho ia a Kihi {\ob}mo taŋi\,{\cb}.\\
\textsc{ntr} go then \textsc{prop} Kihi {\db}for cry\\

\glt
‘Kihi was about to cry.’ \textstyleExampleref{[R215.024]} 
\z

\begin{enumerate}
\setcounter{enumi}{1}
\item 
Nominalised complement clauses\is{Clause!nominalised}, in which the verb is introduced by the article \textit{te}; it may be preceded by the \textsc{acc} marker \textit{i}, as in the following example:
\end{enumerate}

\ea\label{ex:11.29}
\gll {\ꞌ}O ira i ta{\ꞌ}e hōrou ai {\ob}i te vara{\ꞌ}a i te taŋata o ruŋa\,{\cb}. \\
because\_of \textsc{ana} \textsc{pfv} \textsc{conneg} quick \textsc{pvp} {\db}\textsc{acc} \textsc{art} obtain \textsc{acc} \textsc{art} person of above \\

\glt
‘Therefore, they didn’t catch the people on top (of the islet) quickly.’ \textstyleExampleref{[R304.048]} 
\z

\begin{enumerate}
\setcounter{enumi}{2}
\item 
Asyndetic coordination:
\end{enumerate}

\ea\label{ex:11.30}
\gll He ha{\ꞌ}amata te perete{\ꞌ}i {\ob}he hīmene\,{\cb}. \\
\textsc{ntr} begin \textsc{art} cricket {\db}\textsc{ntr} sing \\

\glt
‘The cricket started to sing.’ \textstyleExampleref{[R212.052]} 
\z

\begin{enumerate}
\setcounter{enumi}{3}
\item 
Independent clauses:
\end{enumerate}


\ea\label{ex:11.31}
\gll He u{\ꞌ}i atu, {\ob}ka pū te manu taiko\,{\cb}. \\
\textsc{ntr} look away {\db}\textsc{cntg} approach \textsc{art} bird \textit{taiko} \\

\glt
‘She saw a \textit{taiko} bird come by.’ \textstyleExampleref{[Ley-9-55.078]}
\z


It depends on the matrix verb which type of construction is used. Only types 1 and 2 involve a proper complement, that is, a constituent which is syntactically an argument of the matrix verb. For lack of a better term, constructions of types 3 and 4 will sometimes be referred to as “complement” or “complement clause” in the following sections, but one should bear in mind that this does not imply that they are syntactically a complement of the verb.

Types 3 and 4 are quite similar; in fact, 3 is a subset of 4, with the following two restrictions:

\begin{itemize}
\item 
Asyndetically coordinated clauses generally have identical aspect marking\is{Aspect marker}; in type 4, the aspect marking of the complement clause is independent from that of the main clause.

\item 
While independent clauses may be separated from the matrix clause by markers such as \textit{pē nei ē} ‘like this, as follows’ (see e.g. \REF{ex:11.65} below), this is not possible in asyndetically coordinated clauses.

\end{itemize}

Despite their similarities, types 3 and 4 should be distinguished, as they occur with different verbs.

In addition to these four types of constructions, the same matrix verbs may also have a involve monoclausal constructions: nominal arguments and serial verbs. An example of a serial verb construction with complementation function is the following:

\ea\label{ex:11.32}
\gll {\ꞌ}O ira i hōrou \textbf{i} \textbf{oho} mai era {\ꞌ}i tū mahana era. \\
because\_of \textsc{ana} \textsc{pfv} quick \textsc{pfv} go hither \textsc{dist} at \textsc{dem} day \textsc{dist} \\

\glt
‘Therefore he went quickly that day.’ \textstyleExampleref{[R105.108]} 
\z

In the following subsections, the different categories of verbs mentioned above will be discussed in turn. In \sectref{sec:11.3.7}, the use of these different constructions will be summarised.

\subsection{Perception verbs}\label{sec:11.3.1}
\is{Verb!perception}
Perception verbs like \textit{u{\ꞌ}i} ‘to see, watch’, \textit{hakaroŋo} ‘to listen’ and \textit{ŋaro{\ꞌ}a} ‘to hear’ can be followed by a nominal complement (\sectref{sec:8.6.4.2}), or by a clause which is syntactically independent of the perception verb (type 4 above). The latter will be discussed in the following subsections.

\subsubsection[Use of aspectuals ]{Use of aspectuals} \label{sec:11.3.1.1}

When a perception verb is followed by a clause describing the perceived event, the range of aspect markers in this clause is limited: \textit{ka}, \textit{ko V {\ꞌ}ā}\is{ko V {\ꞌ}ā (perfect aspect)} and \textit{e} are used, while \textit{i} and \textit{he} do not occur. The absence of perfective \textit{i} is not surprising: events which are over and done with are usually not the object of perception. The absence of neutral \textit{he} is not unexpected either: \textit{he} is not able to provide the necessary temporal/aspectual link between the two clauses.

\subparagraph{Contiguity marker \textit{ka}} When the clause expresses an activity or event which is perceived while it is happening, it is often marked with the contiguity marker \textit{ka}. \textit{Ka}\is{ka (aspect marker)} (\sectref{sec:7.2.6}) expresses simultaneity\is{Simultaneity} between the event of perception and the event which is perceived: both take place at the same time. 

\ea\label{ex:11.33}
\gll He u{\ꞌ}i atu, \textbf{ka} \textbf{pū} te manu taiko. \\
\textsc{ntr} look away \textsc{cntg} approach \textsc{art} bird \textit{taiko} \\

\glt 
‘She saw a \textit{taiko} bird come by.’ \textstyleExampleref{[Ley-9-55.078]}
\z

\ea\label{ex:11.34}
\gll He hakaroŋo mai Kaiŋa, \textbf{ka} \textbf{{\ꞌ}ui} \textbf{Vaha}: ‘¿Ko ai koe?’\\
\textsc{ntr} listen hither Kainga \textsc{cntg} ask Vaha ~\textsc{prom} who \textsc{2sg}\\

\glt 
‘Kainga heard Vaha asking: ‘Who are you?’’ \textstyleExampleref{[Mtx-3-01.127]}
\z

\subparagraph{Perfect aspect \textit{ko V {\ꞌ}ā}} When the clause expresses a state of affairs which is perceived, it is marked with the perfect aspect\is{Aspect!perfect} \textit{ko V {\ꞌ}ā}\is{ko V {\ꞌ}ā (perfect aspect)} (\sectref{sec:7.2.7}). This state of affairs may be the result of an event which has taken place before; what is seen is not the event itself but a situation from which the event can be inferred. 

\is{Verb!perception}The \textit{ko}{}-marked complement is often a stative verb or a temporal noun like \textit{pō} ‘night’; the perfect aspect\is{Aspect!perfect} expresses that this state has come about in some way, without specifying how. In \REF{ex:11.35} it is night because it has \textit{become} night, and the ship is far from Rapa Nui because it has been \textit{moving} further and further away. 

\ea\label{ex:11.35}
\gll He u{\ꞌ}i atu \textbf{ko} \textbf{pō} \textbf{{\ꞌ}ā}, {\ꞌ}e \textbf{ko} \textbf{roaroa} \textbf{{\ꞌ}ana} te pahī mai Rapa Nui.\\
\textsc{ntr} look away \textsc{prf} night \textsc{cont} and \textsc{prf} distant:\textsc{red} \textsc{cont} \textsc{art} ship from Rapa Nui\\

\glt 
‘She saw that it was night, and that the ship was far from Rapa Nui.’ \textstyleExampleref{[R210.116]} 
\z

\subparagraph{Imperfective \textit{e}} The third aspectual used after verbs of perception is imperfective \textit{e}, usually followed by the continuity marker \textit{{\ꞌ}ā/{\ꞌ}ana}. While \textit{ko V {\ꞌ}ā}\is{ko V {\ꞌ}ā (perfect aspect)} indicates a state which has come about, \textit{e V {\ꞌ}ā}\is{e (imperfective)!e V {\ꞌ}ā} underlines the continuous nature of a situation, without implying the process by which it has come about (\sectref{sec:7.2.5.4} on \textit{e V {\ꞌ}ā}\is{e (imperfective)!e V {\ꞌ}ā}).

\ea\label{ex:11.36}
\gll {\ꞌ}Ī ka u{\ꞌ}i atu ena ko te repa {\ꞌ}i roto \textbf{e} \textbf{moe} \textbf{rō} \textbf{{\ꞌ}ā}. \\
\textsc{imm} \textsc{cntg} look away \textsc{med} \textsc{prom} \textsc{art} young\_man at inside \textsc{ipfv} lie \textsc{emph} \textsc{cont} \\

\glt 
‘Right then she saw a young man inside, lying down.’ \textstyleExampleref{[R310.045]} 
\z

\ea\label{ex:11.37}
\gll He u{\ꞌ}i atu \textbf{e} \textbf{huri} \textbf{rō} \textbf{{\ꞌ}ā} te {\ꞌ}āriŋa o Heru a ruŋa. \\
\textsc{ntr} look away \textsc{ipfv} turn \textsc{emph} \textsc{cont} \textsc{art} face of Heru by above \\

\glt 
‘They saw that Heru was lying face up.’ \textstyleExampleref{[R313.043]} 
\z
\is{Verb!perception}
\subsubsection[NP + clause]{NP + clause}\label{sec:11.3.1.2}

Often a perception verb is followed first by an object NP expressing the person or thing which is perceived, then a clause specifying what happens to this referent (cf. \ili{English} ‘he saw someone coming’). The object NP in this construction may be marked in several ways: with the accusative marker\is{i (accusative marker)} \textit{i} as in (\ref{ex:11.38}–\ref{ex:11.39}), but also with the topic marker \textit{ko}\is{ko (prominence marker)} as in (\ref{ex:11.40}–\ref{ex:11.41}) (\sectref{sec:8.6.4.5}). The verb in the complement clause is often marked with \textit{ka}\is{ka (aspect marker)}.

\ea\label{ex:11.38}
\gll He u{\ꞌ}i \textbf{i} tū kahu era ō{\ꞌ}ona ko momore tahi {\ꞌ}ā.\textup{} \\
\textsc{ntr} look \textsc{acc} \textsc{dem} clothes \textsc{dist} \textsc{poss.3sg.o} \textsc{prf} \textsc{red}:cut all \textsc{cont} \\

\glt 
‘He saw that those clothes of his were all torn.’ \textstyleExampleref{[R250.017]} 
\z

\ea\label{ex:11.39}
\gll He take{\ꞌ}a \textbf{i} a Hoto Vari ka pū mai. \\
\textsc{ntr} see \textsc{acc} \textsc{prop} Hoto Vari \textsc{cntg} approach hither \\

\glt 
‘He saw Hoto Vari approaching.’ \textstyleExampleref{[R304.004]} 
\z

\ea\label{ex:11.40}
\gll E ha{\ꞌ}uru nō {\ꞌ}ā a Eva he hakaroŋo atu \textbf{ko} te re{\ꞌ}o ka raŋi... \\
\textsc{ipfv} sleep just \textsc{cont} \textsc{prop} Eva \textsc{ntr} listen away \textsc{prom} \textsc{art} voice \textsc{cntg} call \\

\glt 
‘When Eva was sleeping, she heard a voice calling...’ \textstyleExampleref{[R210.180]} 
\z
\is{Verb!perception}
\ea\label{ex:11.41}
\gll {\ꞌ}Ī ka u{\ꞌ}i mai nei \textbf{ko} te kio{\ꞌ}e e rua ka o{\ꞌ}o ka oho atu. \\
\textsc{imm} \textsc{cntg} see hither \textsc{prox} \textsc{prom} \textsc{art} rat \textsc{num} two \textsc{cntg} enter \textsc{cntg} go away \\

\glt
‘There he sees two rats making their way in (lit. entering going).’ \textstyleExampleref{[R310.459]} 
\z

How should these constructions be analysed? At first sight, the complement clause in (\ref{ex:11.38}–\ref{ex:11.41}) can be considered as a relative clause\is{Clause!relative} to the object. One argument against this is the function of the aspect marker\is{Aspect marker}: whereas relative clauses\is{Clause!relative} marked with \textit{ka} usually express an event posterior\is{Posteriority} to that in the surrounding clause(s) (\sectref{sec:11.4.3}), in these examples the \textit{ka}{}-marked clauses express an event \is{Simultaneity}simultaneous to the perception event of the matrix clause. Moreover, as \REF{ex:11.39} shows, the clause may follow a proper noun, even though proper nouns\is{Noun!proper} normally do not take relative clauses\is{Clause!relative}.

A second possibility would be to regard the object NP and the complement clause as two complements of the perception verb. This would mean that perception verbs\is{Verb!perception}, which normally take one complement, take two complements in this construction. Such an analysis would only be plausible if these arguments fulfilled different semantic roles. However, the noun phrase and the clause do not express different semantic roles connected to the action; neither do they express two instances of the same semantic role (*‘I saw him \textit{and} coming’); rather, they are two aspects of a single semantic role: the nominal complement refers to the perceived referent, while the clause expresses an action which is not only performed by that entity, but also part of the same perceived situation. 

Therefore it seems more plausible to consider the nominal complement and the complement clause as a single constituent. The fact that the noun phrase can be marked with \textit{ko}\is{ko (prominence marker)} (which is the default case marker in the absence of other markers) is an argument for this analysis. Constructions (\ref{ex:11.38}–\ref{ex:11.39}) suggest that this noun phrase can be raised to the object position of the matrix verb.
\is{Verb!perception}
\subsection{Aspectual and manner verbs}\label{sec:11.3.2}
\is{Verb!aspectual verb}\subsubsection{\textit{Ha{\ꞌ}amata} ‘begin’}\label{sec:11.3.2.1}
\is{haamata ‘to begin’@ha{\ꞌ}amata ‘to begin’}
\textit{Ha{\ꞌ}amata} ‘begin’ is usually followed by a clause expressing the event which begins. In most cases this clause is juxtaposed, with the same verb marking as \textit{\mbox{ha{\ꞌ}amata}}. Thus, both verbs may be marked with neutral \textit{he} as in \REF{ex:11.42}, perfective \textit{i} as in \REF{ex:11.43}, or perfect \textit{ko V {\ꞌ}ā}\is{ko V {\ꞌ}ā (perfect aspect)} as in \REF{ex:11.44}:

\ea\label{ex:11.42}
\gll \textbf{He} ha{\ꞌ}amata te perete{\ꞌ}i \textbf{he} hīmene. \\
\textsc{ntr} begin \textsc{art} cricket \textsc{ntr} sing \\

\glt 
‘The cricket started to sing.’ \textstyleExampleref{[R212.052]} 
\z

\ea\label{ex:11.43}
\gll Pē ira \textbf{i} ha{\ꞌ}amata ai te tūrita \textbf{i} tu{\ꞌ}u mai ai ki nei.\\
like \textsc{ana} \textsc{pfv} begin \textsc{pvp} \textsc{art} tourist \textsc{pfv} arrive hither \textsc{pvp} to \textsc{prox}\\

\glt 
‘In that way, the tourists started to arrive here.’ \textstyleExampleref{[R376.076]} 
\z

\ea\label{ex:11.44}
\gll {\ꞌ}I tū hora era \textbf{ko} ha{\ꞌ}amata atu \textbf{{\ꞌ}ana} tū {\ꞌ}ua era \textbf{ko} hoa \textbf{{\ꞌ}ā}.\\
at \textsc{dem} time \textsc{dist} \textsc{prf} begin away \textsc{cont} \textsc{dem} rain \textsc{dist} \textsc{prf} throw \textsc{cont}\\

\glt
‘At that time the rain had started to fall.’ \textstyleExampleref{[R536.042]} 
\z

\is{Verb!aspectual verb}This identical marking is not limited to aspect markers\is{Aspect marker}. In \REF{ex:11.45}, both verbs are marked with the negator \textit{kai}. In \REF{ex:11.46}, \textit{ha{\ꞌ}amata} is the verb of a bare relative clause\is{Clause!relative!bare} (\sectref{sec:11.4.5}), which is characterised by the absence of an aspect marker; the complement verb \textit{tu{\ꞌ}u} is likewise unmarked.
\is{haamata ‘to begin’@ha{\ꞌ}amata ‘to begin’}

\ea\label{ex:11.45}
\gll \textbf{Kai} ha{\ꞌ}amata a au \textbf{kai} pa{\ꞌ}o {\ꞌ}ā e tahi miro. \\
\textsc{neg.pfv} begin \textsc{prop} \textsc{1sg} \textsc{neg.pfv} chop \textsc{cont} \textsc{num} one tree \\

\glt 
‘I haven’t yet started to chop down a tree.’ \textstyleExampleref{[R363.091]} 
\z

\ea\label{ex:11.46}
\gll Hora \textbf{ha{\ꞌ}amata} \textbf{tu{\ꞌ}u} mai era o te pere{\ꞌ}oa {\ꞌ}i nei {\ꞌ}ana... \\
time begin arrive hither \textsc{dist} of \textsc{art} car at \textsc{prox} \textsc{ident} \\

\glt
‘When cars started to arrive here...’ \textstyleExampleref{[R539-2.145]}
\z

As (\ref{ex:11.42}–\ref{ex:11.45}) show, the S/A of the second verb is often placed in the subject\is{Subject!raising} position of the matrix clause. However, it may also be placed after the complement verb:

\ea\label{ex:11.47}
\gll He ha{\ꞌ}amata he taŋi \textbf{a} \textbf{Puakiva} ki a Vaha. \\
\textsc{ntr} begin \textsc{ntr} cry \textsc{prop} Puakiva to \textsc{prop} Vaha \\

\glt 
‘Puakiva began to cry for Vaha.’ \textstyleExampleref{[R229.149]} 
\z

\is{Verb!aspectual verb}A second construction is that in which the complement is expressed as a nominalised\is{Verb!nominalised} verb (i.e. preceded by the determiner \textit{te}). This complement may have the accusative marker\is{i (accusative marker)} \textit{i}\is{i (accusative marker)} as in \REF{ex:11.48}, but usually this marker is omitted, as in \REF{ex:11.49}:

\ea\label{ex:11.48}
\gll ...i ha{\ꞌ}amata ai \textbf{i} \textbf{te} \textbf{amo} i te mā{\ꞌ}ea era o te kona ena. \\
~~~\textsc{pfv} begin \textsc{pvp} \textsc{acc} \textsc{art} clean \textsc{acc} \textsc{art} stone \textsc{dist} of \textsc{art} place \textsc{med} \\

\glt 
‘...they started to clear away the stones in that place.’ \textstyleExampleref{[R539-2.213]}
\z

\ea\label{ex:11.49}
\gll He ha{\ꞌ}amata a Kava \textbf{te} \textbf{māuiui}. \\
\textsc{ntr} begin \textsc{prop} Kava \textsc{art} sick \\

\glt
‘Kava started to get ill.’ \textstyleExampleref{[R229.224]} 
\z

Despite the nominalised character of the complement, it still has verbal characteristics: its Patient (\textit{i te mā{\ꞌ}ea era} in \REF{ex:11.48}) is marked with \textit{i}.
\is{haamata ‘to begin’@ha{\ꞌ}amata ‘to begin’}
\is{Verb!aspectual verb}
\subsubsection{\textit{Oti} ‘finish’}\label{sec:11.3.2.2}
\is{oti ‘to finish’}
The verb \textit{oti} has several senses: ‘to be finished, done, over’ (e.g. a story), ‘to run out’, ‘to be the only one’. One common use is ‘to finish doing something’, where \textit{oti} is followed by a complement clause.

The complement verb is nominalised\is{Verb!nominalised}, i.e. marked with the article \textit{te}. Sometimes it is preceded by the accusative marker\is{i (accusative marker)} \textit{i}, in other cases \textit{i} is omitted.\footnote{\label{fn:506}In \ili{Māori}, the complement verb never has the accusative marker\is{i (accusative marker)}; \citet{Hooper1984Unusual} argues that the complement verb is the subject of \textit{oti}.}  As the examples show, the subject of the second verb may be placed in the subject position of \textit{oti}\is{Subject!raising} as in \REF{ex:11.50} and \REF{ex:11.52}, or follow the complement verb as in \REF{ex:11.51} and \REF{ex:11.53}.

\ea\label{ex:11.50}
\gll I oti era tū taŋata era \textbf{i} \textbf{te} \textbf{vānaŋa}...\\
\textsc{pfv} finish \textsc{dist} \textsc{dem} person \textsc{dist} \textsc{acc} \textsc{art} speak\\

\glt 
‘When the man had finished speaking...’ \textstyleExampleref{[R315.377]} 
\z

\ea\label{ex:11.51}
\gll Ko oti {\ꞌ}ā \textbf{i} \textbf{te} \textbf{hopu} Kaiŋa i tō{\ꞌ}ona rima. \\
\textsc{prf} finish \textsc{cont} \textsc{acc} \textsc{art} wash Kainga \textsc{acc} \textsc{poss.3sg.o} hand \\

\glt 
‘Kainga had finished washing his hands.’ \textstyleExampleref{[R243.078]} 
\z

\ea\label{ex:11.52}
\gll ...{\ꞌ}o ira kai hini i oti tahi rō ai tū hare era \textbf{te} \textbf{vera}. \\
~~~because\_of \textsc{ana} \textsc{neg.pfv} delay \textsc{pfv} finish all \textsc{emph} \textsc{pvp} \textsc{dem} house \textsc{dist} \textsc{art} burn \\

\glt 
‘...therefore it wasn’t long before the house was completely burned.’ \textstyleExampleref{[R250.120]} 
\z

\ea\label{ex:11.53}
\gll I oti era \textbf{te} \textbf{kī} au, he turu ko au ko te vi{\ꞌ}e. \\
\textsc{pfv} finish \textsc{dist} \textsc{art} say \textsc{1sg} \textsc{ntr} go\_down \textsc{prom} \textsc{1sg} \textsc{prom} \textsc{art} woman \\

\glt
‘When I had finished saying this, I went down (to the coast) with my wife.’ \textstyleExampleref{[Egt-02.066]}
\z

When the complement verb is transitive, the Patient may be raised to the subject position of \textit{oti}, showing that the complement clause is passivised\is{Passive}: 

\ea\label{ex:11.54}
\gll Ki oti \textbf{te} \textbf{kōrua} \textbf{parau} te tuha{\ꞌ}a {\ꞌ}i te pō{\ꞌ}ā...  \\
when finish \textsc{art} \textsc{2pl} document \textsc{art} distribute at \textsc{art} morning  \\

\glt 
‘When your certificates have been handed out in the morning...’ \textstyleExampleref{[R315.368]} 
\z

\textit{Oti} as a matrix verb with a complement may also be expressed in a serial verb\is{Serial verb} construction. For examples, see (\ref{ex:7.181}–\ref{ex:7.182}) in \sectref{sec:7.7.3}.
\is{oti ‘to finish’}

\subsubsection{\textit{Hōrou} ‘hurry’}\label{sec:11.3.2.3}
\is{horou ‘to hurry’@hōrou ‘to hurry’}
\textit{Hōrou} ‘to hurry, (be) quick’ is used as an adjective or adverb\is{Adverb}, but more commonly it is a main verb taking a clausal argument. This argument can be expressed in a variety of ways: 

\begin{itemize}
\item 
in juxtaposition as in \REF{ex:11.55}, with identical marking of both verbs; 

\item 
as a serial verb\is{Serial verb} as in \REF{ex:11.56}, with repetition of the aspect marker\is{Aspect marker}, but nothing else between the two verbs (\sectref{sec:7.7} on serial verbs); 

\item 
as a nominalised verb\is{Verb!nominalised}, either with accusative marker\is{i (accusative marker)} as in \REF{ex:11.57}, or without as in \REF{ex:11.58}.

\end{itemize}

\ea\label{ex:11.55}
\gll E hōrou koe \textbf{e} \textbf{turu}. \\
\textsc{exh} hurry \textsc{2sg} \textsc{exh} go\_down \\

\glt 
‘Go down quickly.’ \textstyleExampleref{[R231.143]} 
\z

\ea\label{ex:11.56}
\gll {\ꞌ}O ira i hōrou \textbf{i} \textbf{oho} mai era {\ꞌ}i tū mahana era. \\
because\_of \textsc{ana} \textsc{pfv} quick \textsc{pfv} go hither \textsc{dist} at \textsc{dem} day \textsc{dist} \\

\glt 
‘Therefore he went quickly that day.’ \textstyleExampleref{[R105.108]} 
\z

\ea\label{ex:11.57}
\gll {\ꞌ}O ira i ta{\ꞌ}e hōrou ai \textbf{i} \textbf{te} \textbf{vara{\ꞌ}a} i te taŋata o ruŋa. \\
because\_of \textsc{ana} \textsc{pfv} \textsc{conneg} quick \textsc{pvp} \textsc{acc} \textsc{art} obtain \textsc{acc} \textsc{art} person of above \\

\glt 
‘Therefore, they didn’t catch the people on top (of the islet) quickly.’ \textstyleExampleref{[R304.048]} 
\z

\ea\label{ex:11.58}
\gll E ko hōrou te ika \textbf{te} \textbf{pū} mo pātia hai pātia ku hape {\ꞌ}ā  te haha{\ꞌ}u iŋa.\\
\textsc{ipfv} \textsc{neg.ipfv} quick \textsc{art} fish \textsc{art} approach for spear \textsc{ins} spear \textsc{prf} fault \textsc{cont}  \textsc{art} tie \textsc{nmlz}\\

\glt
‘The fish would not come quickly to be speared with a harpoon that had not been tied properly.’ \textstyleExampleref{[R360.019]} 
\z

As \REF{ex:11.58} shows, the subject of the second verb may be raised to the subject\is{Subject!raising} position of \textit{hōrou} (in this case, the Patient is raised, showing that the complement clause is passivised\is{Passive}).

\subsubsection{\textit{Oho} ‘go, about to’}\label{sec:11.3.2.4}
\is{oho ‘to go’}
\textit{Oho} ‘go’ usually refers to physical movement; in this sense, it is the most unmarked motion verb\is{Verb!motion}. \textit{Oho} is also used as an aspectual verb, indicating that an event is about to happen (possibly under influence of \ili{Spanish} \textit{ir}, cf. \citealt[392]{Fischer2007}). In this sense, \textit{oho} is followed by a complement clause introduced by \textit{mo}\is{mo (preverbal)}.

\ea\label{ex:11.59}
\gll He \textbf{oho} ia a Kihi mo taŋi.\\
\textsc{ntr} go then \textsc{prop} Kihi for cry\\

\glt 
‘Kihi was about to cry.’ \textstyleExampleref{[R215.024]} 
\z

\ea\label{ex:11.60}
\gll I \textbf{oho} era a Kekoa mo rere mai... he {\ꞌ}aka he hoki a tu{\ꞌ}a. \\
\textsc{pfv} go \textsc{dist} \textsc{prop} Kekoa for jump hither \textsc{ntr} hesitate \textsc{ntr} return by back \\

\glt 
‘When Kekoa was about to jump... he hesitated and turned back.’ \textstyleExampleref{[R108.010]}\textstyleExampleref{}\is{horou ‘to hurry’@hōrou ‘to hurry’} 
\z

\subsection{Cognitive verbs}\label{sec:11.3.3}
\is{Verb!cognitive}\is{Verb!cognitive}
Cognitive verbs\is{Verb!cognitive} include \textit{{\ꞌ}ite} ‘to know’,\footnote{\label{fn:507}For \textit{{\ꞌ}ite} expressing possibility or ability, see \sectref{sec:11.3.6} below.} \textit{aŋiaŋi} ‘to know, be certain’, \textit{mana{\ꞌ}u} ‘to think’ and the obsolete \textit{ma{\ꞌ}a} ‘know’. They may take a nominal object, which – depending on the verb – is marked with \textit{i} or \textit{ki} (\sectref{sec:8.6.4.2}).

The content of knowledge or thought may also be an event. This is expressed by an independent clause, which can be nominal as in \REF{ex:11.61} or verbal as in (\ref{ex:11.62}–\ref{ex:11.63}). As \REF{ex:11.64} shows, the clause may also be a dependent question. In each example, the bracketed part could function as a clause by itself.

\ea\label{ex:11.61}
\gll Ko \textbf{{\ꞌ}ite} {\ꞌ}ana ho{\ꞌ}i kōrua {\ob}te vārua me{\ꞌ}e mana\,{\cb}. \\
\textsc{prf} know \textsc{cont} indeed \textsc{2pl} {\db}\textsc{art} spirit thing power \\

\glt 
‘You know that spirits are powerful.’ \textstyleExampleref{[R310.023]} 
\z

\ea\label{ex:11.62}
\gll He \textbf{aŋiaŋi} e Ataraŋa {\ob}e ko hoki haka{\ꞌ}ou tū vi{\ꞌ}e era {\ꞌ}ā{\ꞌ}ana\,{\cb}.\\
\textsc{ntr} certain:\textsc{red} \textsc{ag} Ataranga {\db}\textsc{ipfv} \textsc{neg.ipfv} return again \textsc{dem} women \textsc{dist} \textsc{poss.3sg.a}\\

\glt 
‘Ataranga knew for sure that his wife would not return.’ \textstyleExampleref{[R532-01.019]}
\z

\ea\label{ex:11.63}
\gll He \textbf{mana{\ꞌ}u} rō {\ꞌ}ai te taŋata o nei {\ob}ko māmate {\ꞌ}ā a koā Taparahi\,{\cb}.\\
\textsc{ntr} think \textsc{emph} \textsc{subs} \textsc{art} person of \textsc{prox} {\db}\textsc{prf} \textsc{pl}:die \textsc{cont} \textsc{prop} \textsc{coll} Taparahi\\

\glt 
‘The people here thought that Taparahi and the others had died.’ \textstyleExampleref{[R250.243]} 
\z

\ea\label{ex:11.64}
\gll Ko \textbf{{\ꞌ}ite} {\ꞌ}ana ho{\ꞌ}i kōrua {\ob}{\ꞌ}i hē a ia\,{\cb}.\\
\textsc{prf} know \textsc{cont} indeed \textsc{2pl} {\db}at \textsc{cq} \textsc{prop} \textsc{3sg}\\

\glt
‘For you know where she is.’ \textstyleExampleref{[R229.277]} 
\z

The content clause may be introduced by the phrase \textit{pē nei}\is{pe ‘like’@pē ‘like’!pē nei} \textit{ē} ‘like this’ (\sectref{sec:4.6.5.1}).

\ea\label{ex:11.65}
\gll Ko {\ꞌ}ite rivariva {\ꞌ}ā e koe pē nei ē: ko haŋa {\ꞌ}ā a ia  mo oho mo hāpī.\\
\textsc{prf} know good:\textsc{red} \textsc{cont} \textsc{ag} \textsc{2sg} like \textsc{prox} thus \textsc{prf} want \textsc{cont} \textsc{prop} \textsc{3sg}  for go for study\\

\glt 
‘You know very well that she wants to go and study.’ \textstyleExampleref{[R210.066]} 
\z
\is{Verb!cognitive}
\subsection{Speech verbs}\label{sec:11.3.4}
\is{Verb!speech}
As discussed in \sectref{sec:8.6.4.2}, there are two types of speech verbs\is{Verb!speech} in Rapa Nui, ‘say’-type and ‘talk’-type verbs. Only the former, which include e.g. \textit{kī} ‘say’ and \textit{\mbox{{\ꞌ}a{\ꞌ}amu}} ‘tell’, can be followed by a clause (or longer discourse) expressing the content of speech. This can be a direct speech, which usually follows without a specific marker:

\ea\label{ex:11.66}
\gll He kī: ‘¡Ka moe ki raro!’\\
\textsc{ntr} say ~~\textsc{imp} lie to below\\

\glt
‘He said: “Lie down!”’ \textstyleExampleref{[Ley-5-28a.003]}
\z

When the speech verb is followed by an indirect speech, it is often introduced by \textit{pē nei (ē)} ‘like this’ (\sectref{sec:4.6.5.1}):

\ea\label{ex:11.67}
\gll Kai kī atu e te nu{\ꞌ}u hāpa{\ꞌ}o i a koe \textbf{pē} \textbf{nei} \textbf{ē}: a koe he poki {\ꞌ}a Hakahonu.\\
\textsc{neg.pfv} say away \textsc{ag} \textsc{art} people care\_for \textsc{acc} \textsc{prop} \textsc{2sg} like \textsc{prox} thus \textsc{prop} \textsc{2sg} \textsc{ntr} child of\textsc{.a} Hakahonu\\

\glt 
‘The people who took care of you haven’t told you that you are the child of Hakahonu.’ \textstyleExampleref{[R427.016]} 
\z

\textit{Kī} ‘say’ may also be followed by a complement clause introduced by the purpose marker \textit{mo}\is{mo (preverbal)} (\sectref{sec:11.5.1}); usually with a different subject, in the sense ‘tell/ask someone to...’, occasionally with the same subject, in the sense ‘to tell one’s intention’. The identity of the subject can only be known from the context.

\ea\label{ex:11.68}
\gll He kī haka{\ꞌ}ou e rā poki {\ob}mo haka hoki i tā{\ꞌ}ana kōreha\,{\cb}. \\
\textsc{ntr} say again \textsc{ag} \textsc{dist} child {\db}for \textsc{caus} return \textsc{acc} \textsc{poss.3sg.a} eel \\
\is{Verb!speech}
\glt 
‘The child told/asked (them) again to give his eel back.’ \textstyleExampleref{[R532-10.014]}
\z

\ea\label{ex:11.69}
\gll He uru atu he kī {\ob}mo {\ꞌ}aruke i tō{\ꞌ}ona kutu\,{\cb}. \\
\textsc{ntr} enter away \textsc{ntr} say {\db}for delouse \textsc{acc} \textsc{poss.3sg.o} louse \\

\glt 
‘They entered and told (him) they would delouse him.’ \textstyleExampleref{[R310.030]} 
\z
\is{Verb!speech}
\subsection{Attitude verbs}\label{sec:11.3.5}

Under this heading a varied group of verbs is included which involve emotion, mental state, \is{Verb!volition}volition and desire. These include \textit{haŋa} ‘to want’, \textit{pohe} ‘to desire’, \textit{\mbox{ri{\ꞌ}ari{\ꞌ}a}} ‘to fear’, \textit{ha{\ꞌ}amā} ‘to be ashamed’, \textit{mana{\ꞌ}u} ‘to consider, intend, decide’ (for \textit{mana{\ꞌ}u} as cognitive verb, see \sectref{sec:11.3.3}).

These verbs may take a nominal complement introduced by \textit{i} or \textit{ki} (\sectref{sec:8.6.4.2}). They may also take a clausal complement introduced by \textit{mo}\is{mo (preverbal)} (\sectref{sec:11.5.1}):

\ea\label{ex:11.70}
\gll E haŋa rō {\ꞌ}ā a au {\ob}mo kī atu e tahi vānaŋa\,{\cb}. \\
\textsc{ipfv} want \textsc{emph} \textsc{cont} \textsc{prop} \textsc{1sg} {\db}for say away \textsc{num} one thing \\

\glt 
‘I want to say one thing.’ \textstyleExampleref{[R447.025]} 
\z

\ea\label{ex:11.71}
\gll He ha{\ꞌ}amā a Tiare {\ob}mo uru ki roto i te piha hāpī\,{\cb}. \\
\textsc{ntr} ashamed \textsc{prop} Tiare {\db}for enter to inside at \textsc{art} room learn \\

\glt 
‘Tiare was ashamed to enter the classroom.’ \textstyleExampleref{[R334.032]} 
\z

\ea\label{ex:11.72}
\gll He mana{\ꞌ}u ia a ia {\ob}mo oho ki te kona hare o tō{\ꞌ}ona  māmātia era ko Keke\,{\cb}.\\
\textsc{ntr} think then \textsc{prop} \textsc{3sg} {\db}for go to \textsc{art} place house of \textsc{poss.3sg.o}  aunt \textsc{dist} \textsc{prom} Keke\\

\glt
‘She decided to go to the house of her aunt Keke.’ \textstyleExampleref{[R345.090]} 
\z

\is{Verb!volition}As these examples show, the complement clause usually has the same subject as the matrix clause and is unexpressed. A different subject\is{Subject!possessive} is possible, though; this subject is expressed in the same way as in all \textit{mo}\is{mo (preverbal)}{}-clauses (\sectref{sec:11.5.1.2}): usually as possessive, but sometimes with the agent marker \textit{e}\is{e (agent marker)}:

\ea\label{ex:11.73}
\gll {\ꞌ}Ina kai haŋa {\ob}mo oho \textbf{ō{\ꞌ}ou} ki te kona roaroa\,{\cb}. \\
\textsc{neg} \textsc{neg.pfv} want {\db}for go \textsc{poss.2sg.o} to \textsc{art} place distant:\textsc{red} \\

\glt 
‘I don’t want you to go to a distant place.’ \textstyleExampleref{[R210.018]} 
\z

\ea\label{ex:11.74}
\gll Ko haŋa {\ꞌ}ā a au {\ob}mo haka hopu mai \textbf{e} \textbf{koe} i a au  paurō te mahana\,{\cb}.\\
\textsc{prf} want \textsc{cont} \textsc{prop} \textsc{1sg} {\db}for \textsc{caus} bathe hither \textsc{ag} \textsc{2sg} \textsc{acc} \textsc{prop} \textsc{1sg}  every \textsc{art} day\\

\glt 
‘I want you to wash me every day.’ \textstyleExampleref{[R313.178]} 
\z

Negative complements can be introduced by \textit{{\ꞌ}o}\is{o ‘lest’@{\ꞌ}o ‘lest’} ‘lest’ (\sectref{sec:11.5.4}), which expresses an adverse effect to be avoided.

\ea\label{ex:11.75}
\gll {\ꞌ}Ai a Vai Ora ka ri{\ꞌ}ari{\ꞌ}a nō {\ob}\textbf{{\ꞌ}o} māuiui rō {\ꞌ}i te rari\,{\cb}. \\
then \textsc{prop} Vai Ora \textsc{cntg} afraid just {\db}lest sick \textsc{emph} at \textsc{art} wet \\

\glt 
‘Then Vai Ora was afraid (her child) would get ill from being wet.’ \textstyleExampleref{[R301.151]} 
\z

\ea\label{ex:11.76}
\gll Ana haŋa koe {\ob}\textbf{{\ꞌ}o} mana{\ꞌ}u rahi koe ki te poki\,{\cb}, mo tāua {\ꞌ}ana  e hāpa{\ꞌ}o i a rāua ko Kava.\\
\textsc{irr} want \textsc{2sg} {\db}lest think much \textsc{2sg} to \textsc{art} child for \textsc{1du.incl} \textsc{ident}  \textsc{ipfv} care\_for \textsc{acc} \textsc{prop} \textsc{3pl} \textsc{prom} Kava\\

\glt 
‘If you don’t want to worry (lit. if you want lest you think much) about the boy, we will care for him and Kava.’ \textstyleExampleref{[R229.028]} 
\z
\is{Verb!volition}
\subsection{Modal verbs}\label{sec:11.3.6}
\is{Verb!modal}
Various verbs can be used to express modal concepts such as ability, possibility and obligation. These verbs are followed by a complement clause, which is in most cases introduced by \textit{mo}. Most of these verbs are also used in other constructions, e.g. with a nominal complement. If the subject is expressed, it occurs in the main clause (except with \textit{tiene que}, see below).

\textit{Riva} and \textit{rivariva} ‘good’, followed by \textit{mo}\is{mo (preverbal)} \textit{V,} express ability, possibility or permission:

\ea\label{ex:11.77}
\gll {\ꞌ}Ina pa{\ꞌ}i a ia e ko rivariva mo hāpa{\ꞌ}o i a Puakiva. \\
\textsc{neg} in\_fact \textsc{prop} \textsc{3sg} \textsc{ipfv} \textsc{neg.ipfv} good:\textsc{red} for care\_for \textsc{acc} \textsc{prop} Puakiva \\

\glt 
‘She was not able to take care of Puakiva.’ \textstyleExampleref{[R229.003]} 
\z

\ea\label{ex:11.78}
\gll —¿Te ŋā poki e ko riva mo o{\ꞌ}o ki te kona aŋa vaka? —E riva nō.\\
~~~~~\textsc{art} \textsc{pl} child \textsc{ipfv} \textsc{neg.ipfv} good for enter to \textsc{art} place make canoe ~~~\textsc{ipfv} good just\\

\glt 
‘—Can’t the children enter the canoe building site? —They can.’ \textstyleExampleref{[R363.137–138]}
\z

When \textit{{\ꞌ}ite}\is{ite ‘to know’@{\ꞌ}ite ‘to know’} ‘to know’ is followed by \textit{i te V} (i.e. a nominalised verb\is{Verb!nominalised} marked as direct object), it often expresses ability, often a particular skill. Alternatively, it may express a habit or inclination, as in \REF{ex:11.80}.
\is{Verb!modal}

\ea\label{ex:11.79}
\gll Ko {\ꞌ}ite {\ꞌ}ā i te pāpa{\ꞌ}i, i te tai{\ꞌ}o, i te vānaŋa i tētahi {\ꞌ}arero... \\
\textsc{prf} know \textsc{cont} \textsc{acc} \textsc{art} write \textsc{acc} \textsc{art} read \textsc{acc} \textsc{art} speak \textsc{acc} other tongue \\

\glt 
‘He could write, read, speak other languages...’ \textstyleExampleref{[R539-1.052]}
\z

\ea\label{ex:11.80}
\gll {\ꞌ}Ina a au kai {\ꞌ}ite i te kai i te {\ꞌ}ate. \\
\textsc{neg} \textsc{prop} \textsc{1sg} \textsc{neg.pfv} know \textsc{acc} \textsc{art} eat \textsc{acc} \textsc{art} liver \\

\glt 
‘I don’t eat liver, I’m not used to eating liver.’ \textstyleExampleref{[R245.238]} 
\z

\textit{Rova{\ꞌ}a/rava{\ꞌ}a}\is{rova{\ꞌ}a ‘to obtain’} ‘to obtain’, followed by \textit{mo}\is{mo (preverbal)} \textit{V}, is used in the sense ‘to be able, to succeed’:

\ea\label{ex:11.81}
\gll Kai rava{\ꞌ}a e roto mo haka ra{\ꞌ}u mai i te kūpeŋa.\\
\textsc{neg.pfv} obtain \textsc{ag} inside for \textsc{caus} hook hither \textsc{acc} \textsc{art} net\\

\glt 
‘Those inside (the net) did not succeed to hook the net.’ \textstyleExampleref{[R304.128]} 
\z

Possibility is often expressed by \textit{puē}\is{pue ‘can’@puē ‘can’}. This word is borrowed from \is{Spanish influence}\ili{Spanish} \textit{puede}, the third person sg. present tense of \textit{poder} ‘can, be able’,\footnote{\label{fn:508}It is not uncommon for \ili{Spanish} words to be borrowed in the 3\textsuperscript{rd} person sg. present \citep[197]{Makihara2001Adaptation}. The weak pronunciation of intervocalic \textit{d} in Chilean \ili{Spanish} facilitates its elision\is{Elision} (\sectref{sec:2.5.3.1}); the resulting VV sequence coalesces into a single long vowel.} but is used in all persons and numbers. It is followed by \textit{mo V}.
\is{Verb!modal}

\ea\label{ex:11.82}
\gll {\ꞌ}Ina e ko puē mātou mo ho{\ꞌ}o atu i te puka pē ira.  \\
\textsc{neg} \textsc{ipfv} \textsc{neg.ipfv} can \textsc{1pl.excl} for trade away \textsc{acc} \textsc{art} book like \textsc{ana}  \\

\glt 
‘We cannot sell the books like that.’ \textstyleExampleref{[R206.021]} 
\z

\ea\label{ex:11.83}
\gll I puē iho ai ananake mo e{\ꞌ}a mo aŋa i te rāua aŋa misione. \\
\textsc{pfv} can just\_then \textsc{pvp} together for go\_out for do \textsc{acc} \textsc{art} \textsc{3pl} work mission \\

\glt 
‘From then on they could go out together to do their mission work.’ \textstyleExampleref{[R231.281]} 
\z

\ea\label{ex:11.84}
\gll {\ꞌ}I te hora nei ka puē iho nei au mo hāpī rivariva i te pure  ki te taŋata.\\
at \textsc{art} time \textsc{prox} \textsc{cntg} can just\_now \textsc{prox} \textsc{1sg} for teach good:\textsc{red} \textsc{acc} \textsc{art} pray  to \textsc{art} person\\

\glt 
‘Now I can teach the people well how to pray.’ \textstyleExampleref{[R231.195]} 
\z

\textit{Tiene que}\is{tiene que ‘must’}, which expresses both obligation (‘have to’) and necessity (‘must’), is \is{Spanish influence}borrowed from \ili{Spanish} \textit{tiene}, the third person sg. present from \textit{tener}. Just like \textit{puē}, it is used for all persons and numbers. The complementiser \textit{que} was borrowed along with the verb;\footnote{\label{fn:509}In this respect \textit{tiene} is less integrated into the language than \textit{puē}, which takes the Rapa Nui complementiser \textit{mo}. \textit{Puē} is much more common in the text corpus (176x \textit{puē}, 20x \textit{tiene}). The difference in complementiser can also be explained from \ili{Spanish} itself: the auxiliary \textit{poder} (3sg. \textit{puede}) is followed by a bare verb, a construction which would be highly unusual in Rapa Nui, hence the insertion of \textit{mo}.} \textit{que} is followed by a clausal complement, as in \ili{Spanish}.\footnote{\label{fn:510}See \citet[207–210]{Makihara2001Adaptation} for more examples and discussion.}
\is{Verb!modal}

The subject usually comes after the main verb as in \REF{ex:11.86}; in this respect \textit{tiene que} is different from other modal verbs\is{Verb!modal}, where the subject follows the modal verb immediately. However, \REF{ex:11.87} shows that the subject can be raised to the subject position of \textit{tiene}.

\ea\label{ex:11.85}
\gll Tiene\_que ai te hare pure tuai era.\\
must be \textsc{art} house pray old \textsc{dist}\\

\glt 
‘This must be the old church.’ \textstyleExampleref{[R416.060]} 
\z

\ea\label{ex:11.86}
\gll Tiene\_que vānaŋa tāua i te vānaŋa rapa nui. \\
must speak \textsc{1du.incl} \textsc{acc} \textsc{art} talk Rapa Nui \\

\glt 
‘We must speak the Rapa Nui language.’ \textstyleExampleref{\citep[208]{Makihara2001Adaptation}} 
\z

\ea\label{ex:11.87}
\gll Tiene tātou que mana{\ꞌ}u hai forma positiva pē mu{\ꞌ}a. \\
must \textsc{1pl.incl} \textit{que} think \textsc{ins} form positive toward front \\

\glt 
‘From now on, we must think positively.’ \textstyleExampleref{(\citealt[208]{Makihara2001Adaptation})} 
\z
\is{Verb!modal}

\subsection{Summary}\label{sec:11.3.7}

As stated in the introduction to this section, while certain verbs are followed by a complement clause marked with a subordinating marker, other verbs are followed by a juxtaposed clause which is interpreted as semantic complement; yet others are followed by an independent clause. \tabref{tab:65} summarises the use of these strategies for different types of verbs.

\begin{table}

\begin{tabularx}{\textwidth}{Xcccccl}
\lsptoprule 
& 
\parbox{1.2cm}{subord. marker}&
\parbox{1.1cm}{serial verb}& 
\parbox{1.1cm}{nomin. verb}& 
\parbox{2.1cm}{juxtaposition}& 
\parbox{1.0cm}{indep. clause}& 
{§}\\
\midrule
perception verbs &  &  &  &  & x& \ref{sec:11.3.1}\\
\tablevspace
{\textit{ha{\ꞌ}amata}\newline’begin’} &  &  &  & x&  & \ref{sec:11.3.2.1}\\
\tablevspace
{\textit{oti}\newline ‘finish’} &  & x& x&  &  & \ref{sec:11.3.2.2}\\
\tablevspace
{\textit{hōrou}\newline’hurry’} &  & x& x& x&  & \ref{sec:11.3.2.3}\\
\tablevspace
{\textit{oho}\newline ‘go’} & \textit{mo}&  &  &  &  & \ref{sec:11.3.2.4}\\
\tablevspace
cognitive verbs &  &  &  &  & x& \ref{sec:11.3.3}\\
\tablevspace
speech verbs & \textit{mo}&  &  &  & x& \ref{sec:11.3.4}\\
\tablevspace
attitude verbs & \textit{mo, {\ꞌ}o}&  &  &  &  & \ref{sec:11.3.5}\\
\tablevspace
\textit{riva(riva)}\newline ‘able’ & \textit{mo}&  &  &  &  & \ref{sec:11.3.6}\\
\tablevspace
\textit{{\ꞌ}ite} ‘know how to’ &  &  & x&  &  & \ref{sec:11.3.6}\\
\tablevspace
\textit{rova{\ꞌ}a} \newline‘able’ & \textit{mo}&  &  &  &  & \ref{sec:11.3.6}\\
\tablevspace
\textit{puē}\newline ‘can’ & \textit{mo}&  &  &  &  & \ref{sec:11.3.6}\\
\tablevspace
{\textit{tiene} \textit{que} \newline ‘must’} & \textit{(que)}&  &  &  &  & \ref{sec:11.3.6}\\
\lspbottomrule
\end{tabularx}
\caption{Complementation strategies}
% \todo[inline]{Using midrules would make this table easier to read. I suggest we add a dotted (or dashed) line after every other row. How to do that?}
\label{tab:65}
\end{table}
\is{Clause!complement|)}

\section{Relative clauses}\label{sec:11.4}
\is{Clause!relative|(}\subsection{Introduction}\label{sec:11.4.1}

Relative clauses modify the head noun in a noun phrase. In Rapa Nui, as in most languages, the head noun is external to the relative clause\is{Clause!relative} itself; it is a constituent of a higher clause. As this noun has a semantic role both in the higher clause and in the matrix clause, \citet[317]{Dixon2010-2} uses the term \textit{common argument} (CA).

In Rapa Nui, relative clauses\is{Clause!relative} are not marked by special markers or relative pronouns. They have the following syntactic features:

\begin{itemize}
\item 
They follow the head noun and usually occur at the end of the noun phrase.

\item 
They are almost always predicate-initial.

\item 
Most types of relative clauses exhibit a gapping strategy: the common argument is not expressed in the relative clause\is{Clause!relative}.

\item 
The aspectual \textit{he}\is{he (aspect marker)} is rare; the most common aspectuals are \textit{e} and \textit{i}.

\item 
The aspectual is often left out.

\item 
The S/A argument of the relative clause\is{Clause!relative} may be expressed by a pre- or postnominal possessor\is{Possession} modifying the head noun.

\item 
When the common argument is direct object in the relative clause\is{Clause!relative}, the subject is often \textit{e}{}-marked.

\item 
The verb in the relative clause\is{Clause!relative} may be raised to a position adjacent to the head noun.

\end{itemize}

All these features will be discussed and illustrated below. First, a number of preliminary remarks.

\begin{itemize}
\item 
Relative clauses always modify an overt head noun. Headless relative clauses\is{Clause!relative!headless} do not occur in Rapa Nui.

\item 
Relative clauses are always restrictive, i.e. they restrict the reference of the noun phrase. Rapa Nui does not have nonrestrictive relative clauses\is{Clause!relative}, clauses which add information without limiting the reference. If such a clause is called for, a generic noun is placed in apposition\is{Apposition} to the head noun to serve as an anchor for the relative clause\is{Clause!relative} (see (\ref{ex:5.170}–\ref{ex:5.171}) on p.~\pageref{ex:5.170}).

\item 
Relative clauses are used in cleft\is{Cleft} constructions, which serve to put a noun in focus (\sectref{sec:9.2.6}). Clefts are also used to construct a verbal clause after the interrogatives \textit{ai} ‘who’ (\sectref{sec:10.3.2.1}) and \textit{aha} ‘what’ (\sectref{sec:10.3.2.2}).

\end{itemize}
\subsection{Relativised constituents}\label{sec:11.4.2}

In many languages, there are restrictions on the types of constituents that can be relativised. \citet{KeenanComrie1977,KeenanComrie1979} account for this by proposing a “noun phrase accessibility hierarchy”: 

\ea\label{ex:11.88a}
\label{hierarchy}  subject {\textgreater} direct object {\textgreater} indirect object {\textgreater} oblique {\textgreater} possessor\is{Possession}
\z

All languages allow subject relativisation; not all languages allow relativisation of other constituents. A language may have one or more relativisation strategies; according to Keenan \& Comrie, a given strategy will always apply to a continuous segment of this hierarchy. 

This principle holds in many languages, though exceptions have turned up; among Polynesian languages, the hierarchy does not hold in \ili{Māori} (see \citealt{Harlow2007Maori}).

In this section, relativisation of different constituents in Rapa Nui will be discussed and illustrated. At the end of the section, the issue of the noun phrase hierarchy will be revisited.

\subparagraph{\ref{sec:11.4.2}.1~ Subject} Subject relativisation is common. The subject is not expressed in the relative clause\is{Clause!relative}.

\ea\label{ex:11.88}
\gll A Taparahi he poki e tahi {\ob}i poreko ai {\ꞌ}i {\ꞌ}uta\,{\cb}. \\
\textsc{prop} Taparahi \textsc{ntr} child \textsc{num} one {\db}\textsc{pfv} born \textsc{pvp} at inland \\

\glt 
‘Taparahi was a child who was born in the countryside.’ \textstyleExampleref{[R250.001]} 
\z

\ea\label{ex:11.89}
\gll He {\ꞌ}aroha mai ki te nu{\ꞌ}u varavara {\ob}tu{\ꞌ}u ki te kona hoa pahī nei\,{\cb}. \\
\textsc{ntr} greet hither to \textsc{art} people scarce {\db}arrive to \textsc{art} place throw ship \textsc{prox} \\

\glt 
‘They greeted the few people who had come to the place where the ship was launched.’ \textstyleExampleref{[R250.235]} 
\z

\ea\label{ex:11.90}
\gll Ka rahi atu te nu{\ꞌ}u {\ob}e {\ꞌ}aroha mai era hai tāvana teatea\,{\cb}. \\
\textsc{cntg} many away \textsc{art} people {\db}\textsc{ipfv} greet hither \textsc{dist} \textsc{ins} sheet white:\textsc{red} \\

\glt 
‘Numerous were the people who greeted them with white bedsheets.’ \textstyleExampleref{[R210.087]} 
\z

\subparagraph{\ref{sec:11.4.2}.2~ Object} When the object is relativised, it is not expressed in the relative clause\is{Clause!relative}. In object relative clauses\is{Clause!relative}, the subject is often \textit{e}\is{e (agent marker)}\textit{{}-}marked. This conforms to a general pattern: \textit{e}{}-marking of the subject is the rule in transitive clauses without an expressed object (\sectref{sec:8.3.1.1}).

\ea\label{ex:11.91}
\gll Me{\ꞌ}e rahi te me{\ꞌ}e rivariva {\ob}i aŋa e te {\ꞌ}ariki nei ko Hotu Matu{\ꞌ}a  mo tō{\ꞌ}ona nu{\ꞌ}u\,{\cb}.\\
\textsc{ntr} many \textsc{art} thing good:\textsc{red} {\db}\textsc{pfv} do \textsc{ag} \textsc{art} king \textsc{prox} \textsc{prom} Hotu Matu’a  for \textsc{poss.3sg.o} people\\

\glt 
‘Many were the good things king Hotu Matu’a did for his people.’ \textstyleExampleref{[R369.024]} 
\z

\ea\label{ex:11.92}
\gll He take{\ꞌ}a i tū aŋa erc {\ob}e aŋa mai era e Huri {\ꞌ}a Vai\,{\cb}. \\
\textsc{ntr} see \textsc{acc} \textsc{dem} work \textsc{dist} {\db}\textsc{ipfv} do hither \textsc{dist} \textsc{ag} Huri a Vai \\

\glt
‘He saw the thing which Huri a Vai did.’ \textstyleExampleref{[R304.004]} 
\z

Interestingly, the \textit{e}{}-marked subject may precede the verb if it is pronominal, even though preverbal subjects\is{Subject!preverbal} in general are not \textit{e}{}-marked (\sectref{sec:8.3.1.1}), and even though preverbal constituents in relative clauses are rare.

\ea\label{ex:11.93}
\gll He va{\ꞌ}ai tahi e {\ꞌ}Oho Takatore i tū ŋā me{\ꞌ}e ta{\ꞌ}ato{\ꞌ}a era   {\ob}\textbf{e} \textbf{ia} i ma{\ꞌ}u era\,{\cb}.\\
\textsc{ntr} give all \textsc{ag} Oho Takatore \textsc{acc} \textsc{dem} \textsc{pl} thing all \textsc{dist}  {\db}\textsc{ag} \textsc{3sg} \textsc{pfv} carry \textsc{dist}\\

\glt
‘Oho Takatore gave (him) all the things he had brought.’ \textstyleExampleref{[R304.115]} 
\z

Pronominal subjects are not always \textit{e}{}-marked; in the following example, the subject pronoun is marked with the proper article\is{a (proper article)} \textit{a}:

\ea\label{ex:11.94}
\gll Ka hakaroŋo rivariva mai tā{\ꞌ}aku vānaŋa nei {\ob}\textbf{a} \textbf{au}  ka kī atu nei ki a koe\,{\cb}.\\
\textsc{imp} listen good:\textsc{red} hither \textsc{poss.1sg.a} word \textsc{prox} {\db}\textsc{prop} \textsc{1sg}  \textsc{cntg} say away \textsc{prox} to \textsc{prop} \textsc{2sg}\\

\glt 
‘Listen well to my words I am going to say to you.’ \textstyleExampleref{[R229.243]} 
\z

\subparagraph{\ref{sec:11.4.2}.3~ Oblique} When oblique arguments\footnote{\label{fn:511}This includes “indirect objects” (\sectref{sec:8.8.1}).} are relativised, the common argument is expressed pronominally in the relative clause\is{Clause!relative}.\footnote{\label{fn:512}\citet[1]{Silva-Corvalán1978} gives an example of an oblique argument relativised with gapping, but such a construction does not occur anywhere in my the text corpus.} As examples in native texts are scarce, two example from the Bible translation are given.

\ea\label{ex:11.95}
\gll ...nu{\ꞌ}u {\ob}\textbf{ki} \textbf{a} \textbf{rāua} a au i va{\ꞌ}ai ai i te māramarama  mo te rāua aŋa\,{\cb}\\
~~~people {\db}to \textsc{prop} \textsc{3pl} \textsc{prop} \textsc{1sg} \textsc{pfv} give \textsc{pvp} \textsc{acc} \textsc{art} intelligent  for \textsc{art} \textsc{3pl} work\\

\glt 
‘people to whom I have given intelligence for their task’ \textstyleExampleref{[Exo. 28:3]}
\z

\ea\label{ex:11.96}
\gll A au he {\ꞌ}Atua, kope {\ob}\textbf{ki} \textbf{a} \textbf{ia} e ha{\ꞌ}amuri ena e te kōrua tupuna\,{\cb}. \\
\textsc{prop} \textsc{1sg} \textsc{pred} God person {\db}to \textsc{prop} \textsc{3sg} \textsc{ipfv} worship \textsc{med} \textsc{ag} \textsc{art} \textsc{2pl} ancestor \\

\glt 
‘I am God, the one whom your ancestors worshipped.’ \textstyleExampleref{[Mat. 22:32]}
\z

\subparagraph{\ref{sec:11.4.2}.4~ Adjunct} Adjuncts are relativised without being expressed in the relative clause\is{Clause!relative}. These usually express place as in \REF{ex:11.97} or time as in \REF{ex:11.98}, but other adjuncts are possible as in \REF{ex:11.99}.

\ea\label{ex:11.97}
\gll Ki te kona ta{\ꞌ}ato{\ꞌ}a {\ob}e oho era a Hēmi\,{\cb}...\\
to \textsc{art} place all {\db}\textsc{ipfv} go \textsc{dist} \textsc{prop} Hemi\\

\glt 
‘To all the places (where) Hemi went...’ \textstyleExampleref{[R476.004]} 
\z

\ea\label{ex:11.98}
\gll {\ꞌ}i te hora era {\ob}e paka rō {\ꞌ}ā te kōpū\,{\cb} \\
at \textsc{art} time \textsc{dist} {\db}\textsc{ipfv} conspicuous \textsc{emph} \textsc{cont} \textsc{art} belly \\

\glt 
‘at the time (when) the belly was showing (=in a late stage of pregnancy)’ \textstyleExampleref{[R301.004]} 
\z

\ea\label{ex:11.99}
\gll ¿Ko {\ꞌ}ite {\ꞌ}ana ho{\ꞌ}i e koe he aha te ha{\ꞌ}aaura{\ꞌ}a {\ob}au i ta{\ꞌ}e pāhono ai\,{\cb}?\\
~\textsc{prf} know \textsc{cont} indeed \textsc{ag} \textsc{2sg} \textsc{pred} what \textsc{art} meaning  {\db}\textsc{1sg} \textsc{pfv} \textsc{conneg} answer \textsc{pvp}\\

\glt
‘Do you know what the reason was (why) I didn’t answer?’ \textstyleExampleref{[R363.109]} 
\z

In two situations a relativised locative constituent is represented by the pro-form \textit{ira} (\sectref{sec:4.6.5.2}): 

%\setcounter{listWWviiiNumxciiileveli}{0}
\begin{enumerate}
\item 
When the relative clause\is{Clause!relative} is a locative clause, i.e. the relativised phrase is predicate\is{Clause!locative}:
\end{enumerate}

\ea\label{ex:11.100}
\gll He tu{\ꞌ}u ki te kona {\ob}\textbf{{\ꞌ}i} \textbf{ira} te honu\,{\cb}. \\
\textsc{ntr} arrive to \textsc{art} place {\db}at \textsc{ana} \textsc{art} turtle \\

\glt
‘They arrived at the place where the turtle was.’ \textstyleExampleref{[R532-03.008]}
\z

\begin{enumerate}
\setcounter{enumi}{1}
\item 
When a preposition is needed to specify the nature of the locative relationship, for example, when a movement is involved from (\textit{mai}) the referent:
\setcounter{enumi}{1}
\end{enumerate}

\ea\label{ex:11.101}
\gll ...mo oho ō{\ꞌ}ona ki Hiva ki te henua era {\ob}\textbf{mai} \textbf{ira} tō{\ꞌ}ona nu{\ꞌ}u matu{\ꞌ}a era i oho mai ai\,{\cb}. \\
~~~for go \textsc{poss.3sg.o} to Hiva to \textsc{art} land \textsc{dist} {\db}from \textsc{ana} \textsc{poss.3sg.o} people parent \textsc{dist} \textsc{pfv} go hither \textsc{pvp} \\

\glt
‘...to go to Hiva, the country from which his parents had come.’ \textstyleExampleref{[R370.002]} 
\z


As these examples show, the \textit{ira} constituent is in clause-initial position in the relative clause\is{Clause!relative}.

\subparagraph{\ref{sec:11.4.2}.5~ Possessor} Relative clauses with possessor\is{Possession} relativisation are rare, but they do occur. The possessor\is{Possession} is expressed pronominally in the relative clause\is{Clause!relative}, in the same position where it would be in a main clause. In the following example, \textit{te rāua} is coreferential to the head noun \textit{nu{\ꞌ}u}.

\ea\label{ex:11.102}
\gll ...tētahi atu \textbf{nu{\ꞌ}u} tu{\ꞌ}u atu, {\ob}haru tako{\ꞌ}a i \textbf{te} \textbf{rāua} henua e te fiko\,{\cb}.\\
~~~other away people arrive away {\db}grab also \textsc{acc} \textsc{art} \textsc{3pl} land \textsc{ag} \textsc{art} government\\

\glt 
‘...other people who had arrived, whose land the government had also grabbed.’ \textstyleExampleref{[R649.055]} 
\z

\subparagraph{\ref{sec:11.4.2}.6~ Identifying predicates} Predicates\is{Clause!identifying} of identifying clauses (\sectref{sec:9.2.2}) may also be relativised. In this case, the predicate is expressed in the relative clause\is{Clause!relative} as a pronoun preceded by \textit{ko}.\footnote{\label{fn:513}See \sectref{sec:9.2.2} for arguments to consider the \textit{ko}{}-marked pronoun as predicate.} 

\ea\label{ex:11.103}
\gll te kope era {\ob}\textbf{ko} \textbf{ia} te pū{\ꞌ}oko haka tere o te intitucione\,{\cb} \\
\textsc{art} person \textsc{dist} {\db}\textsc{prom} \textsc{3sg} \textsc{art} head \textsc{caus} run of \textsc{art} institute \\

\glt 
‘the person who is the head of the institute’ \textstyleExampleref{[R647.143]} 
\z

\ea\label{ex:11.104}
\gll He {\ꞌ}ui mātou ki te nu{\ꞌ}u {\ob}\textbf{ko} \textbf{rāua} te me{\ꞌ}e i aŋiaŋi o ruŋa i te aŋa nei\,{\cb}.\\
\textsc{ntr} ask \textsc{1pl.excl} to \textsc{art} people {\db}\textsc{prom} \textsc{3pl} \textsc{art} thing \textsc{pfv} certain:\textsc{red} of above at \textsc{art} work \textsc{prox}\\

\glt 
‘We’ll ask the people who are the ones who know about this work.’ \textstyleExampleref{[R535.193]} 
\z

\subparagraph{\ref{sec:11.4.2}.7~ Existential clauses} To relativise existential clauses\is{Clause!existential}, the verb \textit{ai}\is{ai ‘to exist’} ‘to exist’ is used. As discussed in \sectref{sec:9.3}, there are two subtypes of existential clauses\is{Clause!existential}: existential-locative\is{Clause!existential-locative} (‘there is a house in the field’, see \sectref{sec:9.3.2}) and possessive (‘there is his house’ = ‘he has a house’, see \sectref{sec:9.3.3}). An example of a relativised existential-locative is the following:

\ea\label{ex:11.105}
\gll Kona {\ob}ai o te miro o rā hora\,{\cb} ko te hare pure. \\
place {\db}exist of \textsc{art} tree of \textsc{dist} time \textsc{prom} \textsc{art} house prayer \\

\glt
‘The place where there were trees at the time was the church.’ \textstyleExampleref{[R539-1.524]}
\z

In this example the location (\textit{kona}) is relativised, while the Existee (the entity that exists in a given place) is expressed in the relative clause\is{Clause!relative}, marked with the possessive preposition \textit{o}. 

The Existee can also be relativised, with the location expressed in the relative clause\is{Clause!relative}:

\ea\label{ex:11.106}
\gll ...he aha te me{\ꞌ}e {\ob}i ai {\ꞌ}i Rapa Nui\,{\cb} \\
~~~\textsc{pred} what \textsc{art} thing {\db}\textsc{pfv} exist at Rapa Nui \\

\glt
‘(they want to know) what are the things that exist on Rapa Nui’ \textstyleExampleref{[R470.006]} 
\z

In possessive clauses\is{Clause!possessive}, the possessor\is{Possession} can be relativised as in \REF{ex:11.107}; in this case, the possessee is expressed in the relative clause\is{Clause!relative}. The possessee can also be relativised as in (\ref{ex:11.108}–\ref{ex:11.109}), in which case the possessor\is{Possession} is expressed in the relative clause\is{Clause!relative}.

\ea\label{ex:11.107}
\gll Ko {\ꞌ}ata rahi {\ꞌ}ana te ŋā poki {\ob}ai o te veka\,{\cb}. \\
\textsc{prf} more many \textsc{cont} \textsc{art} \textsc{pl} child {\db}exist of \textsc{art} scholarship \\

\glt 
‘The number of children who have a scholarship has increased.’ \textstyleExampleref{[R648.213]} 
\z

\ea\label{ex:11.108}
\gll ¿He aha te {\ꞌ}ati {\ob}ai o te vi{\ꞌ}e nei o ruŋa i te {\ꞌ}a{\ꞌ}amu nei\,{\cb}?\\
{\db}\textsc{pred} what \textsc{art} problem ~exist of \textsc{art} woman \textsc{prox} of above at \textsc{art} story \textsc{prox}\\

\glt 
‘What was the problem that the woman in this story had (lit. that existed of this woman)?’ \textstyleExampleref{[R616.603]} 
\z

\ea\label{ex:11.109}
\gll He haŋu pūai {\ob}ta{\ꞌ}e ai i te ta{\ꞌ}ato{\ꞌ}a taŋata\,{\cb}. \\
\textsc{pred} strength strong {\db}\textsc{conneg} exist at \textsc{art} all person \\

\glt
‘(\textit{Mana}) was a strong force that not everyone had.’ \textstyleExampleref{[R634.002]} 
\z

As these examples show, possessees in the relative clause\is{Clause!relative} are marked with genitive \textit{o} as in \REF{ex:11.107}; possessors are marked either with \textit{o} as in \REF{ex:11.108} or the general-purpose preposition \textit{i} as in \REF{ex:11.109}. 

To summarise: there are two relativising strategies in Rapa Nui: one involving a gap (non-expressed constituent), one involving a resumptive element. It depends on the role of the relativised constituent which strategy is used; this is shown in \tabref{tab:66}.\footnote{\label{fn:514}Existential clauses are not included separately in this table. When the Existee/possessee is relativised, it is the subject of the clause; when the possessor\is{Possession} or location is relativised, it can be considered as an adjunct.}

\begin{table}
\begin{tabularx}{.66\textwidth}{L{39mm}Z{15mm}Z{15mm}} 
\lsptoprule
& {gapping}& {pronoun}\\
\midrule
subject & x& \\
direct object & x& \\
oblique argument &  & x\\
adjunct & x& (x)\\
{possessor\is{Possession}} &  & x\\
identifying predicate &  & x\\
\lspbottomrule
\end{tabularx}
\caption{Relativisation strategies}
\label{tab:66}
\end{table}

We can now look once again at the noun phrase hierarchy given in \REF{ex:11.88a} on p.~\pageref{ex:11.88a} above. As \tabref{tab:66} shows, whether or not the situation in Rapa Nui conforms to Keenan \& Comrie’s generalisation that every relativising strategy involves a continuous segment of the hierarchy, depends on how the syntactic categories of Rapa Nui are mapped to this hierarchy. If oblique arguments (a category including arguments such as Recipients) are taken as a rough equivalent of their category of “indirect object”, the gapping strategy in Rapa Nui does not apply to a continuous segment of the hierarchy: it applies to subjects, direct objects and adjuncts (with the latter, the pronoun strategy also occurs, but marginally), but not to “indirect objects”.

\subsection{Aspect marking in relative clauses}\label{sec:11.4.3}
\is{Clause!relative}
The most common aspect markers\is{Aspect marker} in relative clauses\is{Clause!relative} are perfective \textit{i} and imperfective \textit{e}. \textit{Ka} and \textit{ko} are not unusual either, but \textit{he} is rare. All of these will be briefly discussed in turn.

Perfective \textit{i}\is{i (perfective)} is the most general aspectual in relative clauses\is{Clause!relative}. It may mark events performed at the same time as the events in the main clause as in \REF{ex:11.110}, or completed prior to the events in the main clause as in \REF{ex:11.111}; it may also mark states as in \REF{ex:11.112}.

The verb may be followed by a postverbal demonstrative\is{Demonstrative!postverbal} (including \textit{ai}), but this is optional.

\ea\label{ex:11.110}
\gll {\ꞌ}I tū hora era {\ob}Eva i ŋaro{\ꞌ}a era i tū vānaŋa era {\ꞌ}a koro\,{\cb},  he hakaroŋo atu...\\
at \textsc{dem} time \textsc{dist} {\db}Eva \textsc{pfv} perceive \textsc{dist} \textsc{acc} \textsc{dem} word \textsc{dist} of\textsc{.a} Dad  \textsc{ntr} feel away\\

\glt 
‘At the moment Eva heard those words Dad (spoke), she felt...’ \textstyleExampleref{[R210.075]} 
\z

\ea\label{ex:11.111}
\gll He taŋi ki tū poki era {\ꞌ}ā{\ꞌ}ana {\ob}i to{\ꞌ}o era e Kava\,{\cb}. \\
\textsc{ntr} cry to \textsc{dem} child \textsc{dist} \textsc{poss.3sg.a} {\db}\textsc{pfv} take \textsc{dist} \textsc{ag} Kava \\

\glt 
‘She cried for her child, which had been taken by Kava.’ \textstyleExampleref{[R229.095]} 
\z

\ea\label{ex:11.112}
\gll He oti mau {\ꞌ}ā te taŋata {\ob}i ta{\ꞌ}e māuiui o te kona hare era\,{\cb}.\\
\textsc{ntr} finish really \textsc{cont} \textsc{art} person {\db}\textsc{pfv} \textsc{conneg} sick of \textsc{art} place house \textsc{dist}\\

\glt 
‘He was the only person in the house who wasn’t sick.’ \textstyleExampleref{[R250.091]} 
\z

Imperfective \textit{e}\is{e (imperfective)} in relative clauses\is{Clause!relative} often refers to events which are going on at the time of reference, as in \REF{ex:11.113}. Alternatively, it may indicate events which happen repeatedly or habitually, as in \REF{ex:11.114}. The verb is usually followed by a postverbal demonstrative\is{Demonstrative!postverbal} (\sectref{sec:7.2.5.4}).

\ea\label{ex:11.113}
\gll ...{\ꞌ}i te reka o te rāua ara {\ob}e oho era\,{\cb}. \\
~~~at \textsc{art} entertaining of \textsc{art} \textsc{3pl} way {\db}\textsc{ipfv} go \textsc{dist} \\

\glt 
‘(Eva stopped crying,) because of the enjoyment of the trip they were making.’ \textstyleExampleref{[R210.137]} 
\z

\ea\label{ex:11.114}
\gll Te aŋa {\ꞌ}a Puakiva {\ob}e {\ꞌ}avai era e Pipi\,{\cb}, he apaapa hukahuka... \\
\textsc{art} work of\textsc{.a} Puakiva {\db}\textsc{ipfv} give \textsc{dist} \textsc{ag} Pipi \textsc{pred} gather firewood:\textsc{red} \\

\glt 
‘The work Puakiva got assigned by Pipi, was gathering firewood...’ \textstyleExampleref{[R229.396]} 
\z

Perfect \textit{ko/ku V {\ꞌ}ā}\is{ko V {\ꞌ}ā (perfect aspect)} indicates a state which has come about in some way: with event verbs as in \REF{ex:11.115}, the state is the result of the event described by the verb; with statives as in \REF{ex:11.116}, the situation has resulted from some unspecified process.

\ea\label{ex:11.115}
\gll ...{\ꞌ}e he mataroa repahoa o koro {\ob}ko ma{\ꞌ}u mai {\ꞌ}ā ka rahi atu  te pahu peti\,{\cb}.\\
~~~and \textsc{pred} sailor friend of Dad {\db}\textsc{prf} carry hither \textsc{cont} \textsc{cntg} many away  \textsc{art} can peach\\

\glt 
‘...and some sailors, friends of Dad, who had brought many cans of peaches.’ \textstyleExampleref{[R210.125]} 
\z

\ea\label{ex:11.116}
\gll Ta{\ꞌ}e he tiare; he henua {\ob}ko hāhine {\ꞌ}ā a tātou mo tu{\ꞌ}u\,{\cb}. \\
\textsc{conneg} \textsc{pred} flower \textsc{pred} land {\db}\textsc{prf} near \textsc{cont} \textsc{prop} \textsc{1pl.incl} for arrive \\

\glt 
‘These are not flowers; it is the land which we are close to arriving at.’ \textstyleExampleref{[R210.197]} 
\z

When the contiguity marker \textit{ka}\is{ka (aspect marker)} is used in a relative clause\is{Clause!relative}, the clause expresses an event posterior\is{Posteriority} to the events in the context. In direct speech this means the clause refers to the future\is{Future}, as in \REF{ex:11.117}; in narrative texts the \textit{ka}{}-marked relative clause\is{Clause!relative} is posterior\is{Posteriority} with respect to the time of the main action, as in \REF{ex:11.118}. The verb is always followed by a postverbal demonstrative\is{Demonstrative!postverbal}.

\ea\label{ex:11.117}
\gll Te {\ꞌ}īŋoa o te kai era {\ob}ka ma{\ꞌ}u mai era ki a koe\,{\cb} he ioioraŋi. \\
\textsc{art} name of \textsc{art} food \textsc{dist} {\db}\textsc{cntg} carry hither \textsc{dist} to \textsc{prop} \textsc{2sg} \textsc{pred} \textit{ioioraŋi} \\

\glt 
‘The name of the food they will bring you is \textit{ioioraŋi}.’ \textstyleExampleref{[R310.060]} 
\z

\ea\label{ex:11.118}
\gll He turu ia te taŋata ta{\ꞌ}e ko {\ꞌ}iti ki tū kona era o te pahī {\ob}ka hoa era ki haho i te tai\,{\cb}.\\
\textsc{ntr} go\_down then \textsc{art} man \textsc{conneg} \textsc{prom} little to \textsc{dem} place \textsc{dist} of \textsc{art} ship  {\db}\textsc{cntg} throw \textsc{dist} to outside at \textsc{art} sea\\

\glt 
‘Many (lit. not a few) people went down to the place where the ship would be launched.’ \textstyleExampleref{[R250.211]} 
\z

Neutral \textit{he}\is{he (aspect marker)} is rarely used in relative clauses\is{Clause!relative}. In the few examples I found, its function seems to be similar to \textit{ka}:

\ea\label{ex:11.119}
\gll {\ꞌ}I te mahana era {\ob}he oho\,{\cb}, ko {\ꞌ}ara {\ꞌ}ā a Eva {\ꞌ}i te hora ono  o te pō{\ꞌ}ā.\\
at \textsc{art} day \textsc{dist} {\db}\textsc{ntr} go \textsc{prf} wake\_up \textsc{cont} \textsc{prop} Eva at \textsc{art} time six  of \textsc{art} morning\\

\glt 
‘On the day she was going to leave, Eva woke up at six in the morning.’ \textstyleExampleref{[R210.028]} 
\z

Finally, relative clauses\is{Clause!relative} may be marked with the purpose marker \textit{mo}\is{mo (preverbal)} (\sectref{sec:11.5.1}), in which case they express an event destined to happen:

\ea\label{ex:11.120}
\gll He haka take{\ꞌ}a e Kava i te kona {\ob}\textbf{mo} aŋa o te hare\,{\cb}. \\
\textsc{ntr} \textsc{caus} see \textsc{ag} Kava \textsc{acc} \textsc{art} place {\db}for make of \textsc{art} house \\

\glt 
‘Kava showed (him) the place to build the house.’ \textstyleExampleref{[R229.217]} 
\z

\ea\label{ex:11.121}
\gll E tupa nō {\ꞌ}ana hai taŋata i te uka era {\ob}\textbf{mo} hāipoipo\,{\cb}  {\ꞌ}i ruŋa i tū pahī era.\\
\textsc{ipfv} carry just \textsc{cont} \textsc{ins} person \textsc{acc} \textsc{art} girl \textsc{dist} {\db}for marry  at above at \textsc{dem} ship \textsc{dist}\\

\glt 
‘With (several) people, they carried the girl who was to be married in the boat.’ \textstyleExampleref{[R539-3.034]}
\z

\subsection{Possessive-relative constructions}\label{sec:11.4.4}

In possessive-relative\is{Clause!possessive-relative|(} constructions, the head noun is preceded or followed by a possessor\is{Possession}, which is coreferential to the subject of the relative clause\is{Clause!relative}; the latter is not expressed in the relative clause\is{Clause!relative} itself. (These constructions occur in Rapa Nui as well as in various other Polynesian languages.) Possessive-relative constructions only occur when a constituent other than the subject is relativised; they are found with both object and adjunct relativisation. An example is the following:

\ea\label{ex:11.122}
\gll ¿He aha \textbf{te} \textbf{kōrua} me{\ꞌ}e {\ob}i aŋa {\ꞌ}i {\ꞌ}Apina\,{\cb}? \\
~\textsc{ntr} what \textsc{art} \textsc{2pl} thing {\db}\textsc{pfv} do at Apina \\

\glt
‘What did you do (lit. what [is] your thing did) in Apina?’ \textstyleExampleref{[R301.197]} 
\z

Syntactically, \textit{te kōrua} is a possessive pronoun modifying \textit{me{\ꞌ}e} ‘thing’; it is coreferential to the implied subject of the relative clause\is{Clause!relative}. 

When the possessor\is{Possession} is pronominal, it may either precede the noun as in (\ref{ex:11.123}–\ref{ex:11.124}), or follow it as in \REF{ex:11.125} (\sectref{sec:6.2.1}):

\ea\label{ex:11.123}
\gll ...mo haka oho ki \textbf{tā}\textbf{{\ꞌ}}\textbf{ana} kona era {\ob}i pohe\,{\cb}. \\
~~~for \textsc{caus} go to \textsc{poss.3sg.a} place \textsc{dist} {\db}\textsc{pfv} desire \\

\glt 
‘...to make (the horse) go to the place he wanted (it to go).’ \textstyleExampleref{[R345.087]} 
\z

% \todo[inline]{The following example has extra spacing below, probably because of the footnote. Can this be fixed?}	
\ea\label{ex:11.124}
\gll ¿Pē hē te vai i kōnā ai {\ꞌ}i \textbf{tō{\ꞌ}oku} hora {\ob}rere mai nei\,{\cb}? \\
~like \textsc{cq} \textsc{art} water \textsc{pfv} splash \textsc{pvp} in \textsc{poss.1sg.o} time {\db}jump hither \textsc{prox} \\

\glt 
‘How did the water splash at the time when I jumped?’ \textstyleExampleref{[R108.125]}\footnote{That this is a relative clause\is{Clause!relative}, not just a modifying verb, is shown by the verb phrase particle \textit{mai}.}
\z



\ea\label{ex:11.125}
\gll Te aŋa ra{\ꞌ}e \textbf{{\ꞌ}ā{\ꞌ}ana} {\ob}i aŋa\,{\cb} he hāpa{\ꞌ}o māmoe. \\
\textsc{art} work first \textsc{poss.3sg.a} {\db}\textsc{pfv} do \textsc{pred} care\_for sheep \\

\glt
‘The first work he did, was looking after sheep.’ \textstyleExampleref{[R487.015]} 
\z

When the possessor\is{Possession} is a full noun phrase, it must occur after the noun:

\ea\label{ex:11.126}
\gll {\ꞌ}I tū hora era \textbf{o} \textbf{Kekoa} {\ob}e rere mai era\,{\cb}...\\
at \textsc{dem} time \textsc{dist} of Kekoa {\db}\textsc{ipfv} jump hither \textsc{dist}\\

\glt 
‘At the moment when Kekoa jumped...’ \textstyleExampleref{[R408.024]} 
\z

\ea\label{ex:11.127}
\gll Te kenu \textbf{{\ꞌ}a} \textbf{Hetu{\ꞌ}u} {\ob}i rova{\ꞌ}a ai\,{\cb}, kenu rivariva. \\
\textsc{art} spouse of\textsc{.a} Hetu’u {\db}\textsc{pfv} obtain \textsc{pvp} husband good:\textsc{red} \\

\glt
‘The husband which Hetu’u obtained, was a good husband.’ \textstyleExampleref{[R441.021]}{\rmfnm} 
\z
\footnotetext{Examples such as \REF{ex:11.127} are potentially ambiguous. As discussed above, in object relative clauses\is{Clause!relative} the subject is sometimes preverbal and preceded by the proper article\is{a (proper article)} \textit{a} (see \REF{ex:11.94} in \sectref{sec:11.4.2} above). Now the proper article\is{a (proper article)} \textit{a} is homophonous to the possessive preposition \textit{{\ꞌ}a}, and both may be followed by proper nouns; therefore, in examples such as \REF{ex:11.127}, the subject could also be analysed as a nominative subject marked with the proper article\is{a (proper article)} \textit{a.} However, an analysis as genitive (i.e. \textit{{\ꞌ}a} rather than \textit{a}) is more plausible, as only pronouns occur unambiguously as preverbal subjects\is{Subject!preverbal} in the relative clause\is{Clause!relative}; noun phrase subjects in relative clauses\is{Clause!relative} are always postverbal (see e.g. (\ref{ex:11.91}–\ref{ex:11.92}) above).}

Possessive-relative constructions occur in other Polynesian languages as well. There has been some discussion on the question whether the possessor\is{Possession} is raised from the subject\is{Subject!raising} position of the relative clause\is{Clause!relative} (e.g. \citealt[367]{Harlow2000}; \citealt[185]{Harlow2007Maori}), or whether it is a genuine noun phrase possessor\is{Possession} which happens to be coreferential to the relative clause subject \citep[116]{Clark1976}. In Rapa Nui the second option is more plausible. First, the possessor\is{Possession} can be in the same positions as in any other noun phrase, which suggests that it is no different from other possessors in the noun phrase. Second, as the examples above show, the form of the possessive construction varies between \is{Possession!o/a distinction}\textit{a-} and \textit{o-}possession: \textit{a}{}-possession in \REF{ex:11.123}, \REF{ex:11.125} and \REF{ex:11.127}, \textit{o}{}-possession in \REF{ex:11.124} and \REF{ex:11.126}. As \textit{a-} and \textit{o-}possession express different semantic relationships between possessor\is{Possession} and possessee (\sectref{sec:6.3.2}), this suggests that there is a direct relation between the possessor\is{Possession} and the head noun, even though the primary function of the possessor\is{Possession} seems to be the expression of the relative clause\is{Clause!relative} subject.\footnote{\label{fn:517}\citet{HerdMacdonald2011} make a similar observation for other Polynesian languages. They propose a structure where there is a relation between the possessor\is{Possession} and the relative construction as a whole. This involves a control relation (not raising) between possessor\is{Possession} and relative clause\is{Clause!relative} subject.} And indeed, in most of these cases the choice between \textit{{\ꞌ}a} and \textit{o} is governed by the same principles guiding this choice in possessive constructions in general. In \REF{ex:11.127}, where the relation between possessor\is{Possession} and head noun is one between husband and wife, \textit{{\ꞌ}a} is used (\sectref{sec:6.3.3.1.1}). In \REF{ex:11.125}, the use of \textit{{\ꞌ}a} is possibly motivated by the active relationship of the possessee to the head noun ‘work’ (\sectref{sec:6.3.3.2} item 2). In \REF{ex:11.126}, \textit{o} is used with a temporal noun, again conforming to a general pattern (\sectref{sec:6.3.3.3} item 11). In fact, given the wide range of relationships expressed by possessive constructions in Rapa Nui, all possessive-relatives seem to exhibit some kind of possessive relationship also attested in simple possessive constructions.

If this analysis is correct, the possessor\is{Possession} is not the result of raising, but is a normal noun phrase possessor\is{Possession} which happens to be coreferential to the relative clause\is{Clause!relative} subject. Under coreferentiality\is{Referentiality}, the latter is left unexpressed. 

This analysis is confirmed by the fact that there are also possessive-relative constructions where the possessor\is{Possession} is not the subject of the relative clause\is{Clause!relative}, but an oblique/embed\-ded constituent as in \REF{ex:11.128}:

\largerpage

\ea\label{ex:11.128}
\gll He pura mata \textbf{te} \textbf{kōrua} me{\ꞌ}e {\ob}take{\ꞌ}a mai\,{\cb}. \\
\textsc{pred} only eye \textsc{art} \textsc{2pl} thing {\db}see hither \\

\glt 
‘Your eyes are the only thing that can be seen (lit. mere eyes are your thing seen).’ \textstyleExampleref{[R245.217]}\textstyleExampleref{} 
\z
\is{Clause!possessive-relative|)}
\subsection{Bare relative clauses; verb raising}\label{sec:11.4.5}
\is{Clause!relative!bare}\is{Clause!relative!bare|(}
Bare relative clauses\is{Clause!relative} are relative clauses\is{Clause!relative} in which the verb is not preceded by an aspectual. In Rapa Nui in general, the aspectual is obligatory, except in a few well-defined contexts (\sectref{sec:7.2.2}), one of which is when the verb is adjectival (i.e. functions as noun modifier). Even so, in \sectref{sec:5.7.2.3} I argued that bare relatives are different from adjectival modifiers: unlike the latter, they are truly verbal in that they indicate an event taking place at a specific time; moreover, they can be followed by verb phrase particles and verb arguments. 

Here are a number of examples of bare relative clauses\is{Clause!relative}.

\ea\label{ex:11.129}
\gll He {\ꞌ}aroha mai ki te nu{\ꞌ}u varavara {\ob}tu{\ꞌ}u ki te kona hoa pahī nei\,{\cb}. \\
\textsc{ntr} greet hither to \textsc{art} people scarce {\db}arrive to \textsc{art} place throw ship \textsc{prox} \\

\glt 
‘They greeted the few people who had come to the place where the ship was launched.’ \textstyleExampleref{[R250.235]} 
\z

\ea\label{ex:11.130}
\gll {\ꞌ}Ina he vece {\ob}haka hoki mai i te tarake\,{\cb}. \\
\textsc{neg} \textsc{pred} time {\db}\textsc{caus} return hither \textsc{acc} \textsc{art} corn \\

\glt 
‘At no time (lit. there was not a time) (the buyers) refused the corn (which he offered for sale).’ \textstyleExampleref{[R250.080]} 
\z

\ea\label{ex:11.131}
\gll Ko mātou nō te me{\ꞌ}e {\ob}noho o nei\,{\cb}. \\
\textsc{prom} \textsc{1pl.excl} just \textsc{art} thing {\db}stay of \textsc{prox} \\

\glt 
‘We are the only ones living here.’ \textstyleExampleref{[R404.050]} 
\z

\ea\label{ex:11.132}
\gll Te me{\ꞌ}e nei he hī siera, me{\ꞌ}e {\ob}ai mai mu{\ꞌ}a {\ꞌ}ana  {\ꞌ}ātā ki te hora nei\,{\cb}.\\
\textsc{art} thing \textsc{prox} \textsc{pred} to\_fish sawfish thing {\db}exist from before \textsc{ident}  until to \textsc{art} time \textsc{prox}\\

\glt
‘This thing, fishing for sawfish, is something that has existed from the past until now.’ \textstyleExampleref{[R364.001]} 
\z

These examples show that bare relative clauses\is{Clause!relative} are not limited to one single aspect. In most cases they express a one-time event which has been completed as in \REF{ex:11.129}, i.e. the clause has perfective aspect; however, they may also be habitual\is{Aspect!habitual} as in \REF{ex:11.130}, durative\is{Aspect!durative} as in \REF{ex:11.131}, or stative as in \REF{ex:11.132}.

As these examples also show, the verb tends to come straight after the head noun. Only in \REF{ex:11.129} are noun and verb separated by the adjective \textit{varavara}. Other elements occasionally occurring between noun and verb are quantifiers\is{Quantifier} as in \REF{ex:11.133} and postnominal demonstratives\is{Demonstrative} as in \REF{ex:11.134}:

\ea\label{ex:11.133}
\gll He turu tahi tū nu{\ꞌ}u \textbf{ta{\ꞌ}ato{\ꞌ}a} ha{\ꞌ}aau era. \\
\textsc{ntr} go\_down all \textsc{dem} people all agree \textsc{dist} \\

\glt 
‘All the people who had agreed (on the plan) went down (to the coast).’ \textstyleExampleref{[R250.233]} 
\z

\ea\label{ex:11.134}
\gll ...mo ha{\ꞌ}ateitei i te nu{\ꞌ}u \textbf{era} oho era ki Tahiti. \\
~~~for honour \textsc{acc} \textsc{art} people \textsc{dist} go \textsc{dist} to Tahiti \\

\glt 
‘...to honour the people who went to Tahiti.’ \textstyleExampleref{[R202.003]} 
\z

Even though noun and verb can be separated by these noun phrase elements, there is a strong tendency to place the verb adjacent to the noun. Often \is{Verb!raising}the verb is raised to a position straight after the noun, before other noun phrase elements. In \REF{ex:11.135}, the verb \textit{hatu} is raised to a position before the quantifier\is{Quantifier} \textit{ta{\ꞌ}ato{\ꞌ}a}, while the subject of the relative clause\is{Clause!relative} is stranded after \textit{ta{\ꞌ}ato{\ꞌ}a}. (The status of \textit{era} is discussed below.)

\ea\label{ex:11.135}
\gll He oho tū poki era pē tū me{\ꞌ}e {\ob}hatu\,{\cb} \textbf{ta{\ꞌ}ato{\ꞌ}a} era {\ob}e tū rū{\ꞌ}au era\,{\cb}.\\
\textsc{ntr} go \textsc{dem} child \textsc{dist} like \textsc{dem} thing {\db}advise all \textsc{dist} {\db}\textsc{ag} \textsc{dem} old\_woman \textsc{dist}\\

\glt
‘The boy went (and did) like all the things advised by the old woman.’ \textstyleExampleref{[R310.105]} 
\z

Similarly, in \REF{ex:11.136}, the verb \textit{tu{\ꞌ}u} is raised over the postnominal possessor\is{Possession} \textit{\mbox{{\ꞌ}ā{\ꞌ}ana}}. Notice that even though the relative clause\is{Clause!relative} only consists of a verb, it is still a true relative clause\is{Clause!relative}, not an “adjectival” verb: \textit{tu{\ꞌ}u} refers to a specific event, it is not a time-stable property of the child (\sectref{sec:5.7.2.3}).

\ea\label{ex:11.136}
\gll He vānaŋa ararua ko tū poki {\ob}tu{\ꞌ}u\,{\cb} era {\ꞌ}ā{\ꞌ}ana. \\
\textsc{ntr} talk the\_two \textsc{prom} \textsc{dem} child {\db}arrive \textsc{dist} \textsc{poss.3sg.a} \\

\glt
‘She spoke with her child who had arrived.’ \textstyleExampleref{[R532-01.007]}
\z
\is{Verb!raising}
In \REF{ex:11.137} the verb \textit{hiŋa} is raised both over the particle \textit{{\ꞌ}ā} and the possessor\is{Possession} \textit{o te poki}.\footnote{\label{fn:518}\textit{{\ꞌ}Ā} occurs both in the noun phrase (expressing identity) and in the verb phrase (expressing continuity); here it is a noun phrase particle, modifying the noun: ‘the very same day’.} The same happens in \REF{ex:11.138}, where the possessor\is{Possession} ‘of the morning’ modifies the head noun, while the next phrase ‘to school’ is the part of the relative clause left stranded\is{Clause!relative}.

\ea\label{ex:11.137}
\gll ...{\ꞌ}i te mahana {\ob}hiŋa\,{\cb} era {\ꞌ}ā o te poki? \\
~~~at \textsc{art} day {\db}fall \textsc{dist} \textsc{ident} of \textsc{art} child \\

\glt 
‘(Why didn’t you come and tell me) on the same day the child fell?’ \textstyleExampleref{[R313.106]} 
\z

\ea\label{ex:11.138}
\gll ...mai te hora {\ob}turu\,{\cb} era {\ꞌ}ā o te pō{\ꞌ}ā {\ob}ki te hāpī\,{\cb} ki tū hora era \\
~~~from \textsc{art} time {\db}go\_down \textsc{dist} \textsc{ident} of \textsc{art} morning {\db}to \textsc{art} learn to \textsc{dem} time \textsc{dist} \\

\glt
‘...from the morning time, when he went down to school, until then’ \textstyleExampleref{[R245.009]} 
\z

\is{era (distal)!postnominal}Examples (\ref{ex:11.135}–\ref{ex:11.138}) all involve a demonstrative \textit{era}. Now this demonstrative (as well as \textit{nei} and \textit{ena}) is common both in the noun phrase and in the verb phrase, so \textit{a priori} it may be either a postnominal particle over which the verb has been raised, or a verb phrase particle belonging to the relative clause\is{Clause!relative}. The position of \textit{era} in the examples suggest that the former is the case, as indicated by the brackets. \textit{Era} occurs after the quantifier\is{Quantifier} in \REF{ex:11.135}, but before the possessor\is{Possession} in \REF{ex:11.136} and before the particle \textit{{\ꞌ}ā} in (\ref{ex:11.137}–\ref{ex:11.138}); in other words, \textit{era} occurs in its usual noun phrase position (see the chart in \sectref{sec:5.1}). If \textit{era} were a verb phrase particle, it would be unclear why it is raised with the verb in (\ref{ex:11.136}–\ref{ex:11.138}), but left stranded in \REF{ex:11.135}.

Another reason to consider \textit{era} as postnominal rather than postverbal, is that it co-occurs with the demonstrative \textit{tū}, which is always accompanied by a postnominal de\-mon\-strative\is{Demonstrative!postnominal} (\sectref{sec:4.6.2.1}). When \textit{tū} co-occurs with \textit{era} after the verb, this suggests that the verb has been raised.\footnote{\label{fn:519}Relative clause verbs may have a postverbal demonstrative\is{Demonstrative!postverbal}, even when the head noun also has a demonstrative; see \textit{nu{\ꞌ}u era {\ob}oho era\,{\cb}} in \REF{ex:11.134}. Nevertheless, raised verbs never have a demonstrative of their own: two consecutive demonstratives\is{Demonstrative} never occur (\textit{*nu{\ꞌ}u {\ob}oho\,{\cb} era {\ob}era\,{\cb}}). This can be accounted for by a rule deleting one of two consecutive demonstratives\is{Demonstrative}.} This is illustrated in \REF{ex:11.135} above; the same analysis can be extended to examples such as the following:

\ea\label{ex:11.139}
\gll He kī ki a Kava i \textbf{tū} vānaŋa {\ob}kī\,{\cb} \textbf{era} {\ob}e Pea e tā{\ꞌ}ana kenu\,{\cb}. \\
\textsc{ntr} say to \textsc{prop} Kava \textsc{acc} \textsc{dem} word {\db}say \textsc{dist} {\db}\textsc{ag} Pea \textsc{ag} \textsc{poss.3sg.a} spouse \\
\is{Verb!raising}
\glt 
‘She told Kava the words spoken by her husband Pea.’ \textstyleExampleref{[R229.075]} 
\z

\ea\label{ex:11.140}
\gll E u{\ꞌ}i mai era a \textbf{tū} kona {\ob}kī\,{\cb} \textbf{era} {\ob}e nua\,{\cb}. \\
\textsc{ipfv} look hither \textsc{dist} by \textsc{dem} place ~say \textsc{dist} {\db}\textsc{ag} Mum \\

\glt
‘She looked towards the place Mum had told.’ \textstyleExampleref{[R210.083]} 
\z

In other words, even though \textit{kī era e nua} in \REF{ex:11.140} seems to be a relative clause\is{Clause!relative}, the presence of \textit{tū} suggests that \textit{era}\is{era (distal)!postnominal} is not part of the relative clause, but is a noun phrase particle which has been leapfrogged over by the verb.

Examples such as (\ref{ex:11.139}–\ref{ex:11.140}) are quite common. In fact, the tendency to leave out the aspectual and (if needed) to raise the verb is strongest with definite/anaphoric\is{Anaphora} noun phrases like the ones illustrated here. Leaving out the aspectual has the effect of downplaying the action/event character of the relative clause\is{Clause!relative}: what the relative clause\is{Clause!relative} denotes is not so much an event but rather a fact; this fact is part of the referential description in the noun phrase.
\is{Clause!relative|)}\is{Clause!relative!bare|)}\is{Verb!raising}
\section{Subordinating markers}\label{sec:11.5}

The preverbal markers \textit{mo}, \textit{ana}, \textit{ki}, \textit{{\ꞌ}o} and \textit{mai} are used to mark certain types of clauses. As these markers occur in the same position as aspectuals (\sectref{sec:7.1}), they do not co-occur with the latter, which means that a clause containing one of these particles is not marked for aspect.

In subordinate clauses, these markers are always clause-initial; no constituents are placed before the verb phrase. \textit{Ana}, \textit{ki} and – somewhat marginally – \textit{mo} also occur in main clauses. As their functions in main and subordinate clauses are clearly similar, all their uses will be discussed together in the following sections, with two exceptions:

\begin{itemize}
\item 
The hortative\is{Hortative} use of \textit{ki} is treated in the section on imperatives\is{Imperative} (\sectref{sec:10.2.3}).

\item 
The use of \textit{mo} in complement clauses is discussed in the section on complement clauses (\sectref{sec:11.3}).

\end{itemize}
\subsection{The purpose/conditional marker \textit{mo}}\label{sec:11.5.1}
\is{mo (preverbal)|(}
\textit{Mo} is by far the most common subordinating marker.\footnote{\label{fn:520}Preverbal \textit{mo} probably developed from (or is an extended use of) the benefactive preposition (\sectref{sec:4.7.7}). To my knowledge, Rapa Nui is the only language in which \textit{mo} developed into a preverbal marker. The fact that the subject is often expressed as a possessor (\sectref{sec:11.5.1.2}) may be a trace of the prepositional character of \textit{mo}.} It is used to mark complements of cognitive verbs (\sectref{sec:11.3.3}), speech verbs\is{Verb!speech} (\sectref{sec:11.3.4}), attitude verbs (\sectref{sec:11.3.5}) and modal verbs\is{Verb!modal} (\sectref{sec:11.3.6}). In addition, it marks both purpose clauses\is{Clause!purpose} and conditional clauses\is{Clause!conditional}; these will be discussed in \sectref{sec:11.5.1.1}. \sectref{sec:11.5.1.2} discusses the expression of arguments in \textit{mo-}clauses. Occasionally \textit{mo} occurs in main clauses; this is discussed in \sectref{sec:11.5.1.3}.

\subsubsection{\textit{Mo} in adverbial clauses}\label{sec:11.5.1.1}

\textit{Mo} marks \textsc{purpose} clauses\is{Clause!purpose}. As the examples below show, the \textit{mo}{}-clause usually follows the main clause. 

\ea\label{ex:11.141}
\gll He tahuti a Eva \textbf{mo} eke ki ruŋa i te vaka. \\
\textsc{ntr} run \textsc{prop} Eva for go\_up to above at \textsc{art} boat \\

\glt 
‘Eva ran to get on the boat.’ \textstyleExampleref{[R210.060]} 
\z

\ea\label{ex:11.142}
\gll He haka {\ꞌ}ara i a Tahoŋa \textbf{mo} haka unu hai rā{\ꞌ}au. \\
\textsc{ntr} \textsc{caus} wake\_up \textsc{acc} \textsc{prop} Tahonga for \textsc{caus} drink \textsc{ins} medicine \\

\glt 
‘She woke Tahonga up to give her medicine to drink.’ \textstyleExampleref{[R301.159]} 
\z

\ea\label{ex:11.143}
\gll He moko tētahi taŋata he rutu \textbf{mo} {\ꞌ}a{\ꞌ}aru mai; {\ꞌ}ina kai rava{\ꞌ}a.\\
\textsc{ntr} rush some man \textsc{ntr} gather for grab hither \textsc{neg} \textsc{neg.pfv} obtain\\

\glt
‘Some people rushed together to grab (her); they did not catch her.’ \textstyleExampleref{[Ley-9-55.149]}
\z

\textit{Mo} also marks \textsc{conditional} clauses\is{Clause!conditional}.

\ea\label{ex:11.144}
\gll \textbf{Mo} mate tā{\ꞌ}ue, {\ꞌ}o \textbf{mo} ŋaro, he rahi tu{\ꞌ}u māuiui {\ꞌ}i te {\ꞌ}aroha. \\
if die by\_chance or if lost \textsc{ntr} much \textsc{poss.2sg.o} sick at \textsc{art} pity \\

\glt 
‘If (the bird) dies accidentally, or if it gets lost, you would suffer much from feeling sorry.’ \textstyleExampleref{[R213.027]} 
\z

\ea\label{ex:11.145}
\gll Me{\ꞌ}e ihu pi{\ꞌ}ipi{\ꞌ}i; \textbf{mo} vānaŋa mai, me{\ꞌ}e re{\ꞌ}o huru kē. \\
thing nose crushed:\textsc{red} if talk hither thing voice manner different \\

\glt
‘They are snub-nosed; if they talk, they have a strange voice.’ \textstyleExampleref{[R310.252]} 
\z

Conditional clauses\is{Clause!conditional} usually precede the main clause as in these examples, though this is not a rigid rule. 

As \textit{mo} is a preverbal marker, it is always immediately followed by a verb. This means that \textit{mo} constructions would be impossible with nominal clauses\is{Clause!nominal}, or in contexts where a different preverbal marker is called for (e.g. the negation \textit{kai}). In such cases, the existential verb \textit{ai}\is{ai ‘to exist’} can be employed as an auxiliary verb. \textit{Ai} in turn is followed by a clause which is structured as a main clause (this construction is further discussed in \sectref{sec:9.6.1}).

\ea\label{ex:11.146}
\gll {\ꞌ}E \textbf{mo} \textbf{ai} he tire koe, bueno ka mana{\ꞌ}u pa he tire. \\
and if exist \textsc{ntr} Chilean \textsc{2sg} good \textsc{imp} think like \textsc{pred} Chilean \\

\glt 
‘And if you are a Chilean, OK, think like a Chilean.’ \textstyleExampleref{[R625.098]} 
\z

\ea\label{ex:11.147}
\gll \textbf{Mo} \textbf{ai} kai {\ꞌ}ite i nei hīmene, ka {\ꞌ}ui haka{\ꞌ}ou ia koe.\\
if exist \textsc{neg.pfv} know \textsc{acc} \textsc{prox} song \textsc{imp} ask again then \textsc{2sg}\\

\glt 
‘If you don’t know this song, then ask again.’ \textstyleExampleref{[R615.139]} 
\z

\subsubsection[Arguments in the mo{}-clause]{Arguments in the \textit{mo}{}-clause}\label{sec:11.5.1.2}

\subparagraph{The S/A argument} The S or A argument of the \textit{mo}{}-clause is often coreferential to the subject of the main clause, in which case it is usually not expressed. See e.g. \REF{ex:11.142} above.

When the S/A argument is expressed, it is either as a possessive\is{Subject!possessive} (with preposition \textit{o} or a possessive pronoun of the \textit{o}{}-class) as in (\ref{ex:11.148}–\ref{ex:11.150}), or with the agent marker \textit{e} as in (\ref{ex:11.151}–\ref{ex:11.154})\is{e (agent marker)}. The latter occurs more or less in the same contexts as in main clauses (\sectref{sec:8.3}): in transitive\is{Verb!transitive} VA-clauses without explicit object as in \REF{ex:11.151}; in VOA-clauses as in \REF{ex:11.152}; with verbs like \textit{ŋaro{\ꞌ}a} as in \REF{ex:11.153}; when it is contrasted with other referents as in \REF{ex:11.154}.

\ea\label{ex:11.148}
\gll He oho tātou ki {\ꞌ}Anakena {\ob}mo māta{\ꞌ}ita{\ꞌ}i \textbf{ō{\ꞌ}ou}\,{\cb}. \\
\textsc{ntr} go \textsc{1pl.incl} to Anakena {\db}for observe \textsc{poss.2sg.o} \\

\glt 
‘We’ll go to Anakena for you to watch.’ \textstyleExampleref{[R301.259]} 
\z

\ea\label{ex:11.149}
\gll {\ob}Mo haŋa \textbf{ō{\ꞌ}ou} mo {\ꞌ}ite a hē a au e ŋaro nei\,{\cb}...\\
{\db}if want \textsc{poss.2sg.o} for know by \textsc{cq} \textsc{prop} \textsc{1sg} \textsc{ipfv} disappear \textsc{prox}\\

\glt 
‘If you want to know where I disappear (then come with me).’ \textstyleExampleref{[R212.010]} 
\z

\ea\label{ex:11.150}
\gll Ka hoa hai haraoa, {\ob}mo oho mai \textbf{o} \textbf{te} \textbf{ika} ena mo kai\,{\cb}. \\
\textsc{ipfv} throw \textsc{ins} bread {\db}for go hither of \textsc{art} fish \textsc{med} for eat \\

\glt 
‘Throw bread, so that fish will come to eat.’ \textstyleExampleref{[R301.215]} 
\z

\ea\label{ex:11.151}
\gll {\ꞌ}O ira i {\ꞌ}avai ai i a Puakiva {\ob}mo hāpa{\ꞌ}o \textbf{e} \textbf{te} \textbf{vi{\ꞌ}e} \textbf{nei} \textbf{ko} \textbf{Kava}\,{\cb}.\\
because\_of \textsc{ana} \textsc{pfv} give \textsc{pvp} \textsc{acc} \textsc{prop} Puakiva {\db}for care\_for \textsc{ag} \textsc{art} woman \textsc{prox}  \textsc{prom} Kava\\

\glt 
‘Therefore they gave Puakiva to this woman Kava to take care of. (lit. gave Puakiva to take care by this woman Kava).’ \textstyleExampleref{[R229.006]} 
\z

\ea\label{ex:11.152}
\gll Ka haka noho nō atu koe i a au {\ꞌ}i nei {\ob}mo take{\ꞌ}a nō mai  o Puakiva \textbf{e} \textbf{au}\,{\cb}.\\
\textsc{imp} \textsc{caus} stay just away \textsc{2sg} \textsc{acc} \textsc{prop} \textsc{2sg} at \textsc{prox} {\db}for see just hither  of Puakiva \textsc{ag} \textsc{1sg}\\

\glt 
‘Let me stay here, so I can see Puakiva.’ \textstyleExampleref{[R229.013]} 
\z

\ea\label{ex:11.153}
\gll ¡{\ꞌ}Ī a au ka oho rō hai kona {\ob}mo ŋaro{\ꞌ}a \textbf{e} \textbf{au} te ora\,{\cb}! \\
~\textsc{imm} \textsc{prop} \textsc{1sg} \textsc{cntg} go \textsc{emph} \textsc{ins} place {\db}for perceive \textsc{ag} \textsc{1sg} \textsc{art} life \\

\glt 
‘Now I will go to a place to find (lit. feel) rest!’ \textstyleExampleref{[R214.042]} 
\z

\ea\label{ex:11.154}
\gll Ko haŋa {\ꞌ}ā a au {\ob}mo haka hopu mai \textbf{e} \textbf{koe} i a au  paurō te mahana\,{\cb}.\\
\textsc{prf} want \textsc{cont} \textsc{prop} \textsc{1sg} {\db}for \textsc{caus} bathe hither \textsc{ag} \textsc{2sg} \textsc{acc} \textsc{prop} \textsc{1sg}  every \textsc{art} day\\

\glt
‘I want you (not mother) to wash me every day.’ \textstyleExampleref{[R313.178]} 
\z

The fact that the S/A argument is often expressed as a possessive\is{Subject!possessive}, does not mean that the \textit{mo}{}-clause is nominal. Apart from the possessive constituent, the clause is wholly verbal: the verb is not preceded by a determiner, it may be followed by VP particles such as \textit{mai} in \REF{ex:11.154}, and as the same example also shows, the object may have the accusative marker\is{i (accusative marker)} \textit{i}.

\subparagraph{The O argument} The O argument of a \textit{mo-}clause is either expressed as a direct object – preceded by the accusative marker\is{i (accusative marker)} \textit{i} – or as a possessive. \REF{ex:11.154} above and \REF{ex:11.155} below show \textit{i}{}-marked direct objects; in (\ref{ex:11.156}–\ref{ex:11.157}), the O is expressed as a possessive.

\ea\label{ex:11.155}
\gll ¡Ka haka hāhine mai koe mo u{\ꞌ}i atu \textbf{i} \textbf{tu{\ꞌ}u} \textbf{tau} \textbf{ena}  pē he ra{\ꞌ}ā {\ꞌ}ā!\\
~\textsc{imp} \textsc{caus} near hither \textsc{2sg} for look away \textsc{acc} \textsc{poss.2sg.o} pretty \textsc{med}  like \textsc{pred} sun \textsc{ident}\\

\glt 
‘Come near, so I can see your beauty like the sun!’ \textstyleExampleref{[R301.212]} 
\z

\ea\label{ex:11.156}
\gll Ka oho mai koe, mo u{\ꞌ}i {\ꞌ}iti{\ꞌ}iti \textbf{o} \textbf{te} \textbf{poki} {\ꞌ}ī e ha{\ꞌ}uru {\ꞌ}ana. \\
\textsc{imp} come hither \textsc{2sg} for look little:\textsc{red} of \textsc{art} child \textsc{imm} \textsc{ipfv} sleep \textsc{cont} \\

\glt 
‘Come, to have a look at the child that is sleeping.’ \textstyleExampleref{[R235.047]} 
\z

\ea\label{ex:11.157}
\gll ...he vahivahi mo tatau \textbf{o} \textbf{te} \textbf{pua{\ꞌ}a}, mo hāŋai \textbf{o} \textbf{te} \textbf{oru} {\ꞌ}e mo puru \textbf{o} \textbf{te} \textbf{hoi}.\\
~~~\textsc{ntr} divide:\textsc{red} for to\_milk of \textsc{art} cow for feed of \textsc{art} pig and for close of \textsc{art} horse\\

\glt
‘...he divided (the piece of land) to milk cows, to raise pigs and to enclose horses.’ \textstyleExampleref{[R250.047]} 
\z

I have not noticed any difference between the two constructions. There may be a distinction in prominence, with less significant objects marked as possessive. However this may be, object marking in \textit{mo-}clauses is significantly different from object marking in main clauses: contexts where the object is possessive are not the same contexts where the object would be zero-marked in main clauses (\sectref{sec:8.4.1}).

\subsubsection{\textit{Mo} in main clauses}\label{sec:11.5.1.3}

Occasionally preverbal \textit{mo} is used in main clauses. In these clauses, the subject\is{Subject!preverbal} is always expressed; the constituent order is almost always SV/AVO. When the subject is a pronoun or proper noun, it is marked with \textit{ko}. This structure reminds of clauses with \textit{ko}{}-marked topicalised\is{Topicalisation} subjects (\sectref{sec:8.6.2.1}).

The general sense is that of a subject being destined in some way to perform the action described by the verb. Depending on the context, the clause may express a plan or intention as in \REF{ex:11.158}, an instruction as in \REF{ex:11.159}, or permission as in \REF{ex:11.160}.

\ea\label{ex:11.158}
\gll Ko au \textbf{mo} \textbf{noho} mo tiaki i te tātou hare. \\
\textsc{prom} \textsc{1sg} for stay for guard \textsc{acc} \textsc{art} \textsc{1pl.incl} house \\

\glt 
‘(If you like, you go there.) I will stay and guard our house.’ \textstyleExampleref{[R399.130]} 
\z

\ea\label{ex:11.159}
\gll Ko Teke \textbf{mo} \textbf{teki} atu ki ruŋa ki to{\ꞌ}u miro ena... Ko au \textbf{mo} \textbf{oho}  a te rara mata{\ꞌ}u.\\
\textsc{prom} Teke for jump away to above to \textsc{poss.2sg.o} ship \textsc{med} \textsc{prom} \textsc{1sg} for go  by \textsc{art} side right\\

\glt 
‘Teke is to jump onto your ship... I will go (with my ship) by the righthand side.’ \textstyleExampleref{[MsE\nobreakdash-077.010]}
\z

\ea\label{ex:11.160}
\gll Nu{\ꞌ}u era ka tu{\ꞌ}u ra{\ꞌ}e era ko rāua \textbf{mo} \textbf{o{\ꞌ}o} ra{\ꞌ}e. \\
people \textsc{dist} \textsc{cntg} arrive first \textsc{dist} \textsc{prom} \textsc{3pl} for enter first \\

\glt 
‘The people who arrived first, they could enter first.’ \textstyleExampleref{[R250.071]} 
\z

With a negation, \textit{mo}{}-clauses may express a prohibition or dissuasion. Several negative constructions occur. The constituent negator \textit{ta{\ꞌ}e}\is{tae (negator)@ta{\ꞌ}e (negator)} can be used to negate the subject as in \REF{ex:11.161} or the predicate as in \REF{ex:11.162}. A construction with the clause negator \textit{{\ꞌ}ina}\is{ina (negator)@{\ꞌ}ina (negator)} is also possible, as in \REF{ex:11.163}.

\ea\label{ex:11.161}
\gll \textbf{Ta{\ꞌ}e} māua mo moto haka{\ꞌ}ou. \\
\textsc{conneg} \textsc{1du.excl} for fight again \\

\glt 
‘We should not fight any more.’ \textstyleExampleref{[R211.014]} 
\z

\ea\label{ex:11.162}
\gll ...mo {\ꞌ}ite rō {\ꞌ}ai e te ta{\ꞌ}ato{\ꞌ}a \textbf{ta{\ꞌ}e} mo hopu e tahi {\ꞌ}i ira.\\
~~~for know \textsc{emph} \textsc{cont} \textsc{ag} \textsc{art} all \textsc{conneg} for bathe \textsc{num} one at \textsc{ana}\\

\glt 
‘(Malo put up the stick) so all would know that nobody (lit. not one) could swim there.’ \textstyleExampleref{[R108.030]} 
\z

\ea\label{ex:11.163}
\gll \textbf{{\ꞌ}Ina} e tahi taŋata mo tu{\ꞌ}u haka{\ꞌ}ou ki tū kona era. \\
\textsc{neg} \textsc{num} one person for arrive again to \textsc{dem} place \textsc{dist} \\

\glt
‘Nobody could enter that place any more.’ \textstyleExampleref{[R310.158]}\textstyleExampleref{} 
\z

More work is needed to find out the exact function of \textit{mo} in main clauses, and the syntactic constraints that apply in this construction.
\is{mo (preverbal)|)}

\subsection{The irrealis marker \textit{ana}}\label{sec:11.5.2}
\is{Irrealis}\is{Irrealis|(}
\textit{Ana}\is{ana ‘irrealis’|(} is an irrealis\is{Irrealis} marker.\footnote{\label{fn:521}This particle does not occur in any other language, with the exception of \ili{Māori} \textit{ana} ‘if and when’ \citep[130]{Biggs1973}, which corresponds to the use of Rapa Nui \textit{ana} in conditional/temporal clauses\is{Clause!temporal}.} The irrealis\is{Irrealis} mode, as defined by \citet[244]{Payne1997}, does not assert that the event has happened or will happen. Neither does it assert that the event did \textit{not} happen or will not happen: the irrealis\is{Irrealis} refrains from any claim about the truth of the proposition expressed by the clause. 

\textit{Ana} is mostly used to mark events which may or may not happen, for example intentions, possibilities and obligations; this will be amply illustrated in the following subsections.

In some cases the event has actually happened; this is not inconsistent with the irrealis\is{Irrealis} as defined above. In the following example, the speaker refers back to a question her interlocutor has just asked:

\ea\label{ex:11.164}
\gll ¿Mo aha {\ꞌ}ana koe \textbf{ana} {\ꞌ}ui rō mai?\\
~for what \textsc{ident} \textsc{2sg} \textsc{irr} ask \textsc{emph} hither\\

\glt
‘Why would you ask this?’ \textstyleExampleref{[R315.028]} 
\z

Even though the asking is a real event, the speaker refers to it as something unrealised, perhaps conceived as a more general truth (‘why would anybody ask something like this?’), or as something which is inherently improbable.

\textit{Ana} occurs in the same structural position as aspect markers\is{Aspect marker}; \textit{ana} and aspect markers\is{Aspect marker} are mutually exclusive. Clauses marked by \textit{ana} are therefore not differentiated for aspect (but see \REF{ex:11.184} below).

As \REF{ex:11.164} shows, \textit{ana} can be followed by evaluative markers (\textit{rō}\is{ro (emphatic marker)@rō (emphatic marker)}) and directionals\is{Directional} (\textit{mai}). It cannot be followed by postverbal demonstratives\is{Demonstrative!postverbal} or the VP{}-final particles \textit{{\ꞌ}ā} and \textit{{\ꞌ}ai}.

The following subsections will deal with uses of \textit{ana} in main clauses (\sectref{sec:11.5.2.1}) and subordinate clauses (\sectref{sec:11.5.2.2}), respectively.

\subsubsection{\textit{Ana} in main clauses}\label{sec:11.5.2.1}

\paragraph{} \textit{Ana} is used to express \textsc{intentions}. While the outcome of the intended event is inherently uncertain, the intention itself may be quite firm: \REF{ex:11.165} occurs in a context where two parents have just agreed to call their baby Tahonga; in the quoted sentence, this decision is confirmed.

\ea\label{ex:11.165}
\gll Ko Tahoŋa te {\ꞌ}īŋoa o te tāua poki \textbf{ana} nape. \\
\textsc{prom} Tahonga \textsc{art} name of \textsc{dem} \textsc{1du.incl} child \textsc{irr} call \\

\glt 
‘Tahonga is the name we will call our child.’ \textstyleExampleref{[R301.146]} 
\z

\ea\label{ex:11.166}
\gll Āpō nō tāua \textbf{ana} vānaŋa. \\
tomorrow just \textsc{1du.incl} \textsc{irr} speak \\

\glt 
‘Tomorrow we will talk\textstyleExampleref{.’ [R304.014]} 
\z

\paragraph{} \textit{Ana} may express \textsc{potential} events, events which may or may not happen.

\ea\label{ex:11.167}
\gll A {\ꞌ}uta hō a Vaha \textbf{ana} oho rō. \\
by inland \textsc{dub} \textsc{prop} Vaha \textsc{irr} go \textsc{emph} \\

\glt
‘Vaha might go by the inland way.’ \textstyleExampleref{[Mtx-3-01.142]}
\z

Whether the event will happen or not, may depend on a condition which is stated explicitly. Thus, \textit{ana} may occur in the apodosis, the clause expressing the consequence of a conditional\is{Clause!conditional} or temporal clause\is{Clause!temporal}. 

\ea\label{ex:11.168}
\gll Ki hāhine nō tāua mo tu{\ꞌ}u \textbf{ana} ma{\ꞌ}u iho e au te kai.\\
when close just \textsc{1du.incl} for arrive \textsc{irr} carry just\_then \textsc{ag} \textsc{1sg} \textsc{art} food\\

\glt
‘When we are close to arrival, then I will take the food.’ \textstyleExampleref{[R215.026]} 
\z

Even without a conditional clause\is{Clause!conditional} construction, the occurrence of the event marked by \textit{ana} may be contingent on another event: it is the result of, or at least follows upon, an event expressed in an earlier clause: ‘X, only then Y’. In this case – as in \REF{ex:11.93} above – the verb is usually followed by \textit{iho} ‘just then’. 

\ea\label{ex:11.169}
\gll He me{\ꞌ}e {\ꞌ}o kai vave, e hoki au, \textbf{ana} kai iho. \\
\textsc{pred} thing lest eat yet \textsc{ipfv} return \textsc{1sg} \textsc{irr} eat just\_then \\

\glt 
‘Don’t{\rmfnm} eat yet; I will return, then you can eat.’ \textstyleExampleref{[Mtx-3-01.194]}
\z
\footnotetext{\textit{He me{\ꞌ}e {\ꞌ}o} is a now obsolete construction expressing prohibitions.}

\ea\label{ex:11.170}
\gll A mātou e iri ki te rano {\ꞌ}o ki Ro{\ꞌ}iho \textbf{ana} rova{\ꞌ}a iho te vai. \\
\textsc{prop} \textsc{1pl.excl} \textsc{ipfv} ascend to \textsc{art} crater or to Ro’iho \textsc{irr} obtain just\_then \textsc{art} water \\

\glt
‘We will go up to the crater or to Ro’iho, where we (will/may) find water.’ \textstyleExampleref{[R487.035]} 
\z

As a marker of potentiality, \textit{ana} is also used in content questions. The question may be a real one to which an answer is expected as in \REF{ex:11.171}, or a rhetorical one as in \REF{ex:11.172}:

\ea\label{ex:11.171}
\gll ¿{\ꞌ}I hē māua \textbf{ana} aŋa i nā kai? \\
~at \textsc{cq} \textsc{1du.excl} \textsc{irr} make \textsc{acc} \textsc{med} food \\

\glt 
‘Where will we prepare the meal?’ \textstyleExampleref{[Luke 22:9]}
\z

\ea\label{ex:11.172}
\gll ¿A hē \textbf{ana} tētere te hānau {\ꞌ}e{\ꞌ}epe {\ꞌ}i te ura o te ahi,  {\ꞌ}ina he ara mo tētere?\\
~by \textsc{cq} \textsc{irr} \textsc{pl}:run \textsc{art} race corpulent at \textsc{art} flame of \textsc{art} fire  \textsc{neg} \textsc{pred} way for \textsc{pl}:run\\

\glt 
‘Where could the ‘corpulent race’ flee from the flame of fire, as there was nowhere to flee?’ \textstyleExampleref{[Mtx-3-02.034]}
\z

\paragraph{} \textit{Ana} also has a \textsc{deontic}\is{Deontic mode} use: it is used to express instructions, obligations or norms, as well as permission.

\ea\label{ex:11.173}
\gll \textbf{Ana} tu{\ꞌ}u kōrua ki ira hora pae o te popohaŋa. \\
\textsc{irr} arrive 1p to \textsc{ana} hour five of \textsc{art} dawn \\

\glt 
‘You must arrive there five o’clock in the morning.’ \textstyleExampleref{[R310.272]} 
\z

\ea\label{ex:11.174}
\gll E tahi nō ika mata rāua ko te {\ꞌ}āuke \textbf{ana} kai {\ꞌ}i te mahana. \\
\textsc{num} one just fish raw \textsc{3pl} \textsc{prom} \textsc{art} seaweed \textsc{irr} eat at \textsc{art} day \\

\glt
‘He was allowed to eat just one raw fish with seaweed per day.’ \textstyleExampleref{[Fel-40.11]}
\z

In the second person, deontic\is{Deontic mode} \textit{ana} is similar in function to imperative\is{Imperative} \textit{ka} and exhortative\is{Exhortative} \textit{e}. While the latter two are only used with clause-initial verbs, \textit{ana} is especially used when the verb phrase is non-initial. In \REF{ex:11.176}, initial \textit{e} alternates with non-initial \textit{ana}:

\ea\label{ex:11.175}
\gll Ki tā{\ꞌ}aku vānaŋa \textbf{ana} hakaroŋo mai. \\
to \textsc{poss.1sg.a} word \textsc{irr} listen hither \\

\glt 
‘You must listen to my words.’ \textstyleExampleref{[R229.280]} 
\z

\ea\label{ex:11.176}
\gll E ha{\ꞌ}amuri koe ki a Iehoha ki tu{\ꞌ}u {\ꞌ}Atua  {\ꞌ}e ki a ia mau nō koe \textbf{ana} tāvini.\\
\textsc{exh} worship \textsc{2sg} to \textsc{prop} Jehovah to \textsc{poss.2sg.o} God  and to \textsc{prop} \textsc{3sg} really just \textsc{2sg} \textsc{irr} serve\\

\glt 
‘Worship Jehovah your God, and serve only him.’ \textstyleExampleref{[Mat. 4:10]}
\z

\paragraph{} \textit{Ana} may also mark clauses which express a \textsc{general practice}, something which \textbf{is} normally/usually done in a given situation. This use is found especially in procedural contexts, where the speaker describes how certain things are normally done or should be done. In Rapa Nui, procedures are generally expressed by strings of \textit{he}{}-clauses, with occasional imperatives\is{Imperative} (see \REF{ex:7.5} on p.~\pageref{ex:7.5}). But \textit{ana} may be used as well, especially when the verb is non-initial. 

\ea\label{ex:11.177}
\gll {\ꞌ}I te pō nō te ika nei \textbf{ana} hī. \\
at \textsc{art} night just \textsc{art} fish \textsc{prox} \textsc{irr} to\_fish \\

\glt 
‘This (type of) fish is only fished at night.’ \textstyleExampleref{[R364.007]} 
\z

\ea\label{ex:11.178}
\gll Hai me{\ꞌ}e he raŋaria \textbf{ana} tari mai i te mā{\ꞌ}ea. \\
\textsc{ins} thing \textsc{pred} sled \textsc{irr} transport hither \textsc{acc} \textsc{art} stone \\

\glt
‘(This is what I saw in my youth:) With a sled they would transport the stones.’ \textstyleExampleref{[R107.044]} 
\z

Examples like \REF{ex:11.177} could be considered as deontic\is{Deontic mode}, prescribing how something should be done. However, \REF{ex:11.178} shows that \textit{ana} is used even when the procedure is not an instruction to the present-day hearer, but a description of how something was done in the past. Such contexts can be considered irrealis\is{Irrealis}, as they do not describe events which happened at a specific occasion.\footnote{\label{fn:523}\citet[245]{Payne1997} points out that habitual\is{Aspect!habitual} aspect is less realis\is{Realis} than perfective aspect.} 

\subsubsection{\textit{Ana} in subordinate clauses}\label{sec:11.5.2.2}

\paragraph{} In subordinate clauses, \textit{ana} is used to express a \textsc{condition}: the event may or may not happen, but only if it happens will the event in the main clause take place. The conditional clause\is{Clause!conditional} tends to precede the main clause.

\ea\label{ex:11.179}
\gll \textbf{Ana} haŋa koe {\ꞌ}o mana{\ꞌ}u rahi koe ki te poki, mo tāua {\ꞌ}ana  e hāpa{\ꞌ}o i a rāua ko Kava.\\
\textsc{irr} want \textsc{2sg} lest think much \textsc{2sg} to \textsc{art} child for \textsc{1du.incl} \textsc{ident}  \textsc{ipfv} care\_for \textsc{acc} \textsc{prop} \textsc{3pl} \textsc{prom} Kava\\

\glt 
‘If you don’t want to worry about the child, we will take care of her and Kava.’ \textstyleExampleref{[R229.028]} 
\z

\ea\label{ex:11.180}
\gll E u{\ꞌ}i atu te mata ki a au; \textbf{ana} noho mai au, \textbf{ana} raraŋa mai au i te kete, ku ha{\ꞌ}uru {\ꞌ}ā te hānau {\ꞌ}e{\ꞌ}epe.\\
\textsc{exh} look away \textsc{art} eye to \textsc{prop} \textsc{1sg} \textsc{irr} sit hither \textsc{1sg} \textsc{irr} weave hither \textsc{1sg} \textsc{acc} \textsc{art} basket \textsc{prf} sleep \textsc{cont} \textsc{art} race corpulent\\

\glt
‘Look at me; if I sit down, if I am weaving a basket, (that means that) the ‘corpulent race’ are asleep.’ \textstyleExampleref{[Ley-3-06.025]}
\z

As these examples show, the apodosis is usually marked with an aspectual, i.e. in the realis\is{Realis} mood. Alternatively, the apodosis may also be marked with \textit{ana} (cf. \REF{ex:11.168} above). This can lead to a situation in which both the conditional clause\is{Clause!conditional} and the apodosis are marked with \textit{ana}:

\ea\label{ex:11.181}
\gll \textbf{Ana} haŋa mo hakarere nō {\ꞌ}i Orohie, {\ꞌ}i rā {\ꞌ}ana \textbf{ana} hakarere... \\
\textsc{irr} want for leave just at Orohie at \textsc{dist} \textsc{ident} \textsc{irr} leave \\

\glt
‘If they want to leave (the statue) in Orohie, there they leave it...’ \textstyleExampleref{[Ley-4-06.015]}
\z

In other cases, the question is not \textit{whether} the event in the subordinate clause happens, but \textit{when}: the event is expected to happen or has already happened, and the same is true for the main clause event dependent on it. However, \textit{ana} signals that the clause is still irrealis\is{Irrealis} in some way. It may indicate an event which takes or took place habitually (see the discussion about \REF{ex:11.178} above), or an event which is expected (with more or less certainty) to take place in the future. \textit{Ana} is not used with events which have taken place at a definite moment in the past. 

\ea\label{ex:11.182}
\gll \textbf{Ana} mate te taŋata, te matu{\ꞌ}a, he hohora te moeŋa... \\
\textsc{irr} die \textsc{art} man \textsc{art} parent \textsc{ntr} spread \textsc{art} mat \\

\glt 
‘When a man – a father – dies, they spread out a mat...’ \textstyleExampleref{[Ley-4-08.001]}
\z

\ea\label{ex:11.183}
\gll \textbf{Ana} pō, he tutu hai ahi. \\
\textsc{irr} night \textsc{ntr} kindle \textsc{ins} fire \\

\glt 
‘When it is dark, we will light a fire.’ \textstyleExampleref{[R210.085]} 
\z

\paragraph{} \textit{Ana} also occurs in \textsc{dependent polar questions}\is{Question!polar} (‘whether’):

\ea\label{ex:11.184}
\gll {\ꞌ}Ī {\ꞌ}ō a au he oho he u{\ꞌ}i \textbf{ana} ai ko {\ꞌ}ara {\ꞌ}ana. \\
\textsc{imm} really \textsc{prop} \textsc{1sg} \textsc{ntr} go \textsc{ntr} look \textsc{irr} exist \textsc{prf} wake\_up \textsc{cont} \\

\glt 
‘I’m going straightaway and look whether she has woken up.’ \textstyleExampleref{[R229.366]} 
\z

\is{Interrogative}While \textit{ana} is usually followed by the main verb of the clause, sometimes it is followed by the existential verb \textit{ai}\is{ai ‘to exist’} ‘exist’ (just like \textit{mo}, see \REF{ex:11.146} on p.~\pageref{ex:11.146}); the rest of the clause follows as a complement to this verb. This allows the speaker to use \textit{ana} with a nonverbal clause as in \REF{ex:11.185}, or to express aspect in addition to irrealis\is{Irrealis}, as in \REF{ex:11.184} above, where the main verb is marked with perfect aspect\is{Aspect!perfect} \textit{ko}. 

\ea\label{ex:11.185}
\gll He {\ꞌ}ui e Aio ki tū korohu{\ꞌ}a era \textbf{ana} ai {\ob}pē ira mau te parauti{\ꞌ}a\,{\cb}. \\
\textsc{ntr} ask \textsc{ag} Aio to \textsc{dem} old\_man \textsc{dist} \textsc{irr} exist {\db}like \textsc{ana} really \textsc{art} truth \\

\glt 
‘Aio asked the old man if those things were true (lit. if it was: like that [was] the truth).’ \textstyleExampleref{[R532-14.016]}
\z
\is{Irrealis|)}\is{ana ‘irrealis’|)}
\subsection{The purpose/temporal marker \textit{ki}}\label{sec:11.5.3}
\is{ki (preverbal)|(}
The preverbal marker \textit{ki} is used in subordinate clauses expressing time (‘when’) and purpose (‘in order to, so that’). In main clauses it marks hortatives, i.e. first-person injunctions. In this section, its use in subordinate clauses is discussed; hortatives are discussed in \sectref{sec:10.2.3}. 

Even though \textit{ki} is homophonous to the preposition \textit{ki}, the two are probably etymologically distinct. The verbal marker \textit{ki} is probably derived from \is{Proto-Polynesian}PPN \textit{*kia}, which occurs in many languages with an optative and/or purposive sense.\footnote{\label{fn:524}\textit{Kia} was shortened to \textit{ki} in various languages. \citet[30]{Clark1976} mentions \ili{Kapingamarangi}, \ili{Nukumanu}, \ili{Sikaiana} and \ili{Luangiua}. \ili{Hawaiian} \textit{i} (\citealt[61]{ElbertPukui1979}) seems to represent the same particle. As the particle is \textit{kia}/\textit{{\ꞌ}ia} in most \is{Central-Eastern Polynesian}CE languages, the shortening to \textit{ki} in Rapa Nui must have been an independent development which took place after Rapa Nui broke off from \is{Eastern Polynesian}PEP (\sectref{sec:2.5.2} on the monophthongisation of particles). This process may have taken place relatively recently: there are a few occurrences of \textit{kia} in older texts, mostly in fossilised phrases such as \textit{ka oho,} \textit{kia tika} ‘go straight’ (Mtx-2-03.018; Mtx-6-07.014); see discussion in \citet[429]{Fischer1994}. Nowadays \textit{kia} survives in \textit{kiahio} ‘keep courage, be strong’ (cf. \textit{hiohio} ‘strong’).}  If this is correct, the preposition and the verbal marker \textit{ki} were distinct lexemes in the protolanguage. However, because of the goal-oriented character of preverbal \textit{ki}, it is glossed ‘to’, just like the preposition.

\subparagraph{\ref{sec:11.5.3}.1} For \textsc{purpose clauses}\is{Clause!purpose}, the default marker is \textit{mo} (\sectref{sec:11.5.1.1}). \textit{Ki} is used especially in the following circumstances:

In the first place, after an imperative\is{Imperative} or hortative. 

\ea\label{ex:11.186}
\gll Ka uru mai koe ki roto \textbf{ki} {\ꞌ}avai atu a au i tā{\ꞌ}au o te kai. \\
\textsc{imp} entr hither \textsc{2sg} to inside to give away \textsc{prop} \textsc{1sg} \textsc{acc} \textsc{poss.2sg.a} of \textsc{art} food \\

\glt 
‘Come inside, so I will/can give you your food.’ \textstyleExampleref{[R229.417]} 
\z

\ea\label{ex:11.187}
\gll Ka hōrou mai koe \textbf{ki} oho rō tāua. \\
\textsc{ipfv} hurry hither \textsc{2sg} to go \textsc{emph} \textsc{1du.incl} \\

\glt
‘Hurry up, so we can go.’ \textstyleExampleref{[R313.109]} 
\z

When the \textit{ki}{}-clause has a first person plural subject as in \REF{ex:11.187}, the clause may have hortative overtones: ‘so we (can) go’ {\textgreater} ‘let’s go’.

Secondly, when \textit{mo}\is{mo (preverbal)} would be potentially ambiguous. In \REF{ex:11.188}, the main verb \textit{pohe} is followed by a complement clause marked with \textit{mo}. If the next clause were also marked with \textit{mo}, it could be read as a second complement of \textit{pohe}; to ensure a reading as purpose clause, \textit{ki} is used. The same happens in \REF{ex:11.189}: while the \textit{mo}{}-clause expresses the purpose of the preceding main clause, the \textit{ki}{}-clause after that expresses the ultimate purpose, the higher-order goal of the preceding clauses as a whole.

\ea\label{ex:11.188}
\gll {\ꞌ}Ī e pohe atu ena mo {\ꞌ}ata noho mai \textbf{ki} {\ꞌ}ata keukeu ai tētahi aŋa.\\
\textsc{imm} \textsc{ipfv} desire away \textsc{med} for more stay hither to more labour:\textsc{red} \textsc{pvp} other work\\

\glt 
‘I would like him to stay here a bit more, in order to get other projects done.’ \textstyleExampleref{[R204.005]} 
\z

\largerpage

\ea\label{ex:11.189}
\gll O te hānau {\ꞌ}e{\ꞌ}epe i keri ai i te rua...  mo pae o te hānau momoko, \textbf{ki} noho e hānau {\ꞌ}e{\ꞌ}epe nō.\\
of \textsc{art} race corpulent \textsc{pfv} dig \textsc{pvp} \textsc{acc} \textsc{art} hole  for finished of \textsc{art} race slender to stay \textsc{ag} race corpulent just\\

\glt 
‘The ‘corpulent race’ dug a hole... to exterminate the ‘slender race’, so the ‘corpulent race’ would be the only ones (left).’ \textstyleExampleref{[Ley-3-06.019]}
\z

Thirdly, to express a result\is{Clause!result} not intended by the main-clause subject. This is illustrated in the following two examples. The \textit{ki}{}-clause does not express a purpose which the main-clause subject had in mind; rather, it is a purpose external to the intentions of the subject.

\ea\label{ex:11.190}
\gll ¿He aha te me{\ꞌ}e i me{\ꞌ}e e ia \textbf{ki} aŋiaŋi ai e tātou ko koa {\ꞌ}ā?\\
~\textsc{pred} thing \textsc{art} thing \textsc{pfv} thing \textsc{ag} \textsc{3sg} to certain:\textsc{red} \textsc{pvp} \textsc{ag} \textsc{1pl.incl} \textsc{prf} happy \textsc{cont}\\

\glt 
‘What things did she do so that we (the readers of the story) know that she was happy?’ \textstyleExampleref{[R615.658]} 
\z

\ea\label{ex:11.191}
\gll Māuruuru haka{\ꞌ}ou ki te mau mahiŋo era i {\ꞌ}ui mai era: hē te mātou ra{\ꞌ}atira, \textbf{ki} hakaroŋo atu tā{\ꞌ}ana vānaŋa.\\
thank again to \textsc{art} \textsc{pl} people \textsc{dist} \textsc{pfv} ask hither \textsc{dist} \textsc{cq} \textsc{art} \textsc{1pl.excl} chief to listen away \textsc{poss.3sg.a} word\\

\glt
‘Thanks again to the people who asked: where is our chief, so we can hear his words.’ \textstyleExampleref{[R205.044]} 
\z

As these examples show, the subject of the \textit{ki}{}-clause is expressed in the same way as in main clauses: either unmarked as in (\ref{ex:11.186}–\ref{ex:11.187}) or with the agent marker \textit{e} as in \REF{ex:11.189}. In this respect, \textit{ki}{}-clauses are different from \textit{mo-}clauses, which usually have a possessive subject.

A peculiarity of \textit{ki}{}-clauses with purpose sense, is that the verb is often followed by \textit{ai}, the postverbal demonstrative\is{Demonstrative!postverbal} which otherwise only occurs after \textit{i} (\sectref{sec:7.6.5}). This is illustrated in \REF{ex:11.188} and \REF{ex:11.190} above.

\subparagraph{\ref{sec:11.5.3}.2} \textit{Ki} also marks \textsc{temporal} clauses\is{Clause!temporal}.\footnote{\label{fn:525}The double function of reflexes of PPN \textit{*kia} as both optative/purposive and temporal markers is also found with \ili{Māori} \textit{kia} (\citealt[62, 459]{Bauer1993}) and \ili{Tahitian} \textit{{\ꞌ}ia} (\citealt[138–139]{LazardPeltzer2000}); unlike Rapa Nui, in these languages the particle is not used in temporal clauses referring to the past. In Rapa Nui, the purposive sense of \textit{*kia} has to a large degree been taken over by \textit{mo}, as discussed above.} As the examples below show, these occur in various contexts: with past\is{Past} reference, with future\is{Future} reference, or habitual\is{Aspect!habitual}. \textit{Ki}{}-clauses usually occur before the main clause, but as \REF{ex:11.195} shows, they may also be placed after the main clause.

\ea\label{ex:11.192}
\gll \textbf{Ki} oti a Puakiva te vānaŋa i kī ai e koro... \\
when finish \textsc{prop} Puakiva \textsc{art} talk \textsc{pfv} say \textsc{pvp} \textsc{ag} Dad \\

\glt 
‘When Puakiva had finished speaking, Dad said...’ \textstyleExampleref{[R229.490]} 
\z

\ea\label{ex:11.193}
\gll He haka hū au i te {\ꞌ}umu, \textbf{ki} oti he oho a koe... \\
\textsc{ntr} \textsc{caus} burn \textsc{1sg} \textsc{acc} \textsc{art} earth\_oven when finish \textsc{ntr} go \textsc{prop} \textsc{2sg} \\

\glt 
‘I will light the earth oven, when finished you will go...’ \textstyleExampleref{[R184.007]} 
\z

\ea\label{ex:11.194}
\gll \textbf{Ki} oho ararua e ma{\ꞌ}u te rima. \\
when go the\_two \textsc{exh} hold \textsc{art} hand \\

\glt 
‘When the two of you walk together, hold hands.’ \textstyleExampleref{[R166.004]} 
\z

\ea\label{ex:11.195}
\gll He aŋa tātou he haka hōrou mo turu o tātou ki tai  \textbf{ki} tu{\ꞌ}u mai a nua.\\
\textsc{ntr} work \textsc{1pl.incl} \textsc{ntr} \textsc{caus} quick for go\_down of \textsc{1pl.incl} to sea  when arrive hither \textsc{prop} Mum\\

\glt 
‘We will work quickly, so we can go to the sea when Mum comes.’ \textstyleExampleref{[R229.456]} 
\z

\textit{Ki}{}-marked clauses may indicate a goal or temporal boundary: ‘until’. This occurs for example after the verb \textit{tiaki} ‘wait’.\footnote{\label{fn:526}In other contexts, ‘until’ is more commonly expressed by \textit{ka V rō}\is{ka (aspect marker)!ka V rō}, and/or using \textit{{\ꞌ}ātā} (\sectref{sec:11.6.2.5}).}

\ea\label{ex:11.196}
\gll He tiaki \textbf{ki} hū tahi te hukahuka. \\
\textsc{ntr} wait to burn all \textsc{art} firewood:\textsc{red} \\

\glt 
‘They wait until all the firewood is burned.’ \textstyleExampleref{[R333.460]} 
\z

\ea\label{ex:11.197}
\gll He noho rō atu {\ꞌ}ai o tū nu{\ꞌ}u era {\ꞌ}i ira \textbf{ki} ora riva o te mata o Māhina Tea.\\
\textsc{ntr} stay \textsc{emph} away \textsc{subs} of \textsc{dem} people \textsc{dist} at there to live good of \textsc{art} eye of Mahina Tea\\

\glt
‘The people stayed there until Mahina Tea’s eyes had healed well.’ \textstyleExampleref{[R399.235]} 
\z

The preposition \textit{ki} has the same use, see \REF{ex:4.266} on p.~\pageref{ex:4.266}. This shows that the two particles \textit{ki}, though etymologically distinct, are closely related.

In fact, there is not an absolute distinction between the senses ‘when’ and ‘until’. Whether \textit{ki} is translated as one or the other, mainly depends on whether it is connected to the preceding clause (‘X until Y’) or to the following clause (‘when Y, then Z’). When connected to both, the \textit{ki}{}-clause marks a boundary point or “hinge” between two events:

\ea\label{ex:11.198}
\gll {\ꞌ}I roto e hāpa{\ꞌ}o era \textbf{ki} takataka tahi te tarake  he to{\ꞌ}o mai he huhu.\\
at inside \textsc{ipfv} care\_for \textsc{dist} to/when gather:\textsc{red} all \textsc{art} corn  \textsc{ntr} take hither \textsc{ntr} strip\\

\glt 
‘Inside they stored (the corn) until all the corn was gathered, (then) they would take it and strip it.’ \textstyleExampleref{[R250.068]} 
\z

\ea\label{ex:11.199}
\gll He uru ki raro i te ro{\ꞌ}i he piko, \textbf{ki} roa te hora he e{\ꞌ}a  he tere mai.\\
\textsc{ntr} enter to below at \textsc{art} bed \textsc{ntr} hide to/when long \textsc{art} time \textsc{ntr} go\_out  \textsc{ntr} run hither\\

\glt 
‘He would go under the bed and hide, when/until a long time (had passed), then he would come out and run away.’ \textstyleExampleref{[R250.185]} 
\z
\is{ki (preverbal)|)}

\subsection{\textit{{\ꞌ}O} ‘lest’}\label{sec:11.5.4}
\is{o ‘lest’@{\ꞌ}o ‘lest’|(}
The preverbal marker \textit{{\ꞌ}o}\footnote{\label{fn:527}The origin of \textit{{\ꞌ}o} is unclear. It may be a reflex of \is{Proto-Polynesian}PPN \textit{*{\ꞌ}aua} ‘negative imperative’, which occurs throughout Polynesia (Tongic, Samoic-Outlier and \is{Eastern Polynesian}EP). Cf. also Footnote \ref{fn:499} on p.~\pageref{fn:499} on the origin of the negator \textit{(e) ko}.

Another possible cognate is \ili{Tahitian} \textit{{\ꞌ}o}, which introduces clauses after “des verbes exprimant la crainte, la méfiance, et parfois l’eventualité” (verbs expressing fear, mistrust, and sometimes contingency), and which is followed by a nominalised verb\is{Verb!nominalised} (\citealt[197]{AcadémieTahitienne1986}). However, given the fact that Rapa Nui \textit{{\ꞌ}o} occurs in old texts already, it is relatively unlikely that it is a borrowing from \ili{Tahitian}\is{Tahitian influence}.} indicates a consequence which is to be avoided. It can be translated as ‘lest’ or ‘so that ... not’.

\textit{{\ꞌ}O}{}-marked clauses usually occur after the main clause and are always verb-initial. The subject is expressed in the same way as in main clauses: unmarked as in \REF{ex:11.200}, or with the agent marker \textit{e} as in \REF{ex:11.201}. 

\ea\label{ex:11.200}
\gll He oho a Eva he piko \textbf{{\ꞌ}o} kī rō a koro mo ta{\ꞌ}e oho ki hiva.\\
\textsc{ntr} go \textsc{prop} Eva \textsc{ntr} hide lest say \textsc{emph} \textsc{prop} Dad for \textsc{conneg} go to mainland\\

\glt 
‘Eva went and hid lest Dad would tell her not to go to the mainland.’ \textstyleExampleref{[R210.026]} 
\z

\ea\label{ex:11.201}
\gll He tētere he pipiko tahi \textbf{{\ꞌ}o} vara{\ꞌ}a rō e te Miru i a rāua mo tiaŋi. \\
\textsc{ntr} \textsc{pl}:run \textsc{ntr} \textsc{pl}:hide all lest catch \textsc{emph} \textsc{ag} \textsc{art} Miru \textsc{acc} \textsc{prop} \textsc{3pl} for kill \\

\glt 
‘All of them fled and hid, lest the Miru would catch them to kill them.’ \textstyleExampleref{[R304.039]} 
\z

\ea\label{ex:11.202}
\gll ¿He aha te kōrua me{\ꞌ}e ka aŋa ena \textbf{{\ꞌ}o} ai pē ira? \\
~\textsc{pred} what \textsc{art} \textsc{2pl} thing \textsc{cntg} do \textsc{med} lest exist like \textsc{ana} \\

\glt 
‘What will you do so that it won’t happen?’ \textstyleExampleref{[R648.239]} 
\z

\ea\label{ex:11.203}
\gll E tiaki {\ꞌ}ana hoki Kaiŋa i a Vaha \textbf{{\ꞌ}o} iri atu Vaha  ki ruŋa ki te motu.\\
\textsc{ipfv} wait \textsc{cont} also Kainga \textsc{acc} \textsc{prop} Vaha lest ascend away Vaha  to above to \textsc{art} islet\\

\glt
‘Kainga waited for Vaha, so Vaha wouldn’t climb on the islet.’ \textstyleExampleref{[Mtx-3-01.124]}
\z

In modern Rapa Nui, a verb marked with \textit{{\ꞌ}o} is usually followed by the asseverative particle \textit{rō}\is{ro (emphatic marker)@rō (emphatic marker)} (\sectref{sec:7.4.2}), as illustrated in (\ref{ex:11.200}–\ref{ex:11.201}) above.

Occasionally \textit{{\ꞌ}o} is found in complement clauses expressing a negative complement: \textit{\mbox{ri{\ꞌ}ari{\ꞌ}a} {\ꞌ}o} ‘to fear lest’, \textit{haŋa {\ꞌ}o} ‘to want that not...’ \is{o ‘lest’@{\ꞌ}o ‘lest’|)}(see (\ref{ex:11.75}–\ref{ex:11.76}) in \sectref{sec:11.3.5}). 

\subsection{\textit{Mai} ‘before; while’}\label{sec:11.5.5}
\is{mai ‘while, before’|(}
\textit{Mai}, which is common as a preposition ‘from’ (\sectref{sec:4.7.4}) and as a directional\is{Directional} ‘movement towards deictic centre’ (\sectref{sec:7.5}), also occurs occasionally as a preverbal marker. It indicates an event prior to the event in the main clause: ‘before’.

\ea\label{ex:11.204}
\gll He tunu atu au i to tāua kai \textbf{mai} pō. \\
\textsc{ntr} cook away \textsc{1sg} \textsc{acc} \textsc{art}:of \textsc{1du.incl} food from night \\

\glt
‘I will cook our food, before it gets dark.’ \textstyleExampleref{[R229.140]} 
\z

\textit{Mai} is often reinforced by the constituent negator \textit{ta{\ꞌ}e}\is{tae (negator)@ta{\ꞌ}e (negator)}, which in this construction does not invert the polarity of the clause.

\ea\label{ex:11.205}
\gll ¡Ka hōrou mai, \textbf{mai} \textbf{ta{\ꞌ}e} taŋi te oe!\\
~\textsc{imp} hurry hither from \textsc{conneg} cry \textsc{art} bell\\

\glt
‘Hurry up, before the bell strikes!’ \textstyleExampleref{[R334.077]} 
\z

As these examples show, the event in the \textit{mai}{}-clause indicates the end point of a time frame, which limits the time available to accomplish the action in the main clause. Event A should be done before (\textit{mai}) event B happens.\footnote{\label{fn:528}Interestingly, in \ili{Hawaiian} \textit{mai} marks events to be avoided; it marks both negative imperatives and events (always unpleasant ones) which almost happen, but not quite: \textit{Mai hā{\ꞌ}ule ke keike} ‘The child almost fell’ (\citealt[61–63]{ElbertPukui1979}). This is somewhat similar to temporal \textit{mai} in Rapa Nui, though the latter is limited to subordinate clauses.} 

The event in the \textit{mai}{}-clause may also be something which is to be avoided altogether: A should be done before B happens, so that B will not happen at all.

\ea\label{ex:11.206}
\gll Ka horohorou koe \textbf{mai} \textbf{ta{\ꞌ}e} {\ꞌ}atrasao. \\
\textsc{imp} \textsc{red}:hurry \textsc{2sg} from \textsc{conneg} tardy \\

\glt 
‘Hurry up or you will be late.’ \textstyleExampleref{[R245.019]} 
\z

\ea\label{ex:11.207}
\gll {\ꞌ}Ī au he oho rō {\ꞌ}ai \textbf{mai} \textbf{ta{\ꞌ}e} ma{\ꞌ}urima i a au. \\
\textsc{imm} \textsc{1sg} \textsc{ntr} go \textsc{emph} \textsc{subs} from \textsc{conneg} surprise \textsc{acc} \textsc{prop} \textsc{1sg} \\

\glt 
‘I’m going now, before (=or else) they will catch me.’ \textstyleExampleref{[R304.117]} 
\z

Occasionally, the \textit{mai}{}-clause marks not the boundary of a time frame, but the time frame as such during which the action in the main clause is to be performed: ‘while, as long as’. In this case, the verb is followed by the continuity marker \textit{{\ꞌ}ā/{\ꞌ}ana}\is{a (postverbal)@{\ꞌ}ā (postverbal)} (\sectref{sec:7.2.5.5}):

\ea\label{ex:11.208}
\gll ¿{\ꞌ}O te aha koe i ta{\ꞌ}e hā{\ꞌ}aki mai ai \textbf{mai} noho \textbf{{\ꞌ}ana}  {\ꞌ}i Hiva, {\ꞌ}i te kāiŋa?\\
~because\_of \textsc{art} what \textsc{2sg} \textsc{pfv} \textsc{conneg} inform hither \textsc{pvp} from stay \textsc{cont}  at Hiva at \textsc{art} homeland\\

\glt 
‘Why didn’t you tell me when we still lived in Hiva, in the homeland?’ \textstyleExampleref{[Ley-2-07.028]}
\z

\ea\label{ex:11.209}
\gll {\ꞌ}O ira ka hā{\ꞌ}ere {\ꞌ}i roto i te mā{\ꞌ}eha, \textbf{mai} ai atu \textbf{{\ꞌ}ana} te mōrī. \\
because\_of \textsc{ana} \textsc{imp} walk:\textsc{pl} at inside at \textsc{art} light from exist away \textsc{cont} \textsc{art} light \\

\glt 
‘Therefore walk in the light, while there is still light.’ \textstyleExampleref{[John 12:35]}\textstyleExampleref{} 
\z
\is{mai ‘while, before’|)}

\subsection{Summary}\label{sec:11.5.6}

In the preceding sections, five preverbal markers have been discussed which introduce subordinate clauses; two of these also introduce certain types of main clauses. \tabref{tab:67} summarises the different functions of these markers.

\begin{table}
%\small{
\begin{tabularx}{\textwidth}{L{8mm}L{50mm}L{36mm}L{12mm}}
\lsptoprule
& {clause type} & {gloss} & {§}\\
\midrule
\textit{mo} & complement & that & \ref{sec:11.3}\\
& purpose & to, in order to & \ref{sec:11.5.1.1}\\
& conditional & if & \ref{sec:11.5.1.1}\\
& main clause: destined & be to, be destined to & \ref{sec:11.5.1.3}\\
\tablevspace
%\midrule
\textit{ana} & main clause: intention & will, let’s & \ref{sec:11.5.2.1}\\
& main clause: potential & may, might, could & \ref{sec:11.5.2.1}\\
& main clause: apodosis & (if,) then & \ref{sec:11.5.2.1}\\
& main clause: deontic & must; be allowed to & \ref{sec:11.5.2.1}\\
& main clause: general practice & always, usually & \ref{sec:11.5.2.1}\\
& conditional & if & \ref{sec:11.5.2.2}\\
& temporal & when & \ref{sec:11.5.2.2}\\
& dependent question & whether & \ref{sec:11.5.2.2}\\
\tablevspace
%\midrule
\textit{ki} & hortative & let’s & \ref{sec:10.2.3}\\
& purpose & so, in order to & \ref{sec:11.5.3}\\
& temporal & when; until & \ref{sec:11.5.3}\\
\tablevspace
%\midrule
{\textit{{\ꞌ}o}} & negative purpose & lest & \ref{sec:11.5.4}\\
& negative complement & that not & \ref{sec:11.3.5}\\
\tablevspace
%\midrule
{\textit{mai}} & temporal & before; while & \ref{sec:11.5.5}\\
\lspbottomrule
\end{tabularx}
%}
\caption{Functions of preverbal markers}
% \todo[inline]{Replaced midrules by vertical spacce.}
\label{tab:67}
\end{table}

The sections above also show, that case marking in subordinate clauses follows the same rules as in main clauses: the S/A argument is marked with Ø or \textit{e}, the O argument with \textit{i} or Ø, depending on the factors described in \sectref{sec:8.3}–\ref{sec:8.4}. The only exception is \textit{mo}, where both arguments are often marked as possessive.

\section{Adverbial clauses}\label{sec:11.6}
\is{Clause!adverbial|(}\subsection{Adverbial clause strategies}\label{sec:11.6.1}

Adverbial clauses provide an adverbial modification of the main clause. They can be constructed in various ways:

%\setcounter{listWWviiiNumxxivleveli}{0}
\begin{enumerate}
\item 
using one of the preverbal markers discussed in \sectref{sec:11.5} above;

\item 
using a conjunction\is{Conjunction} (where \textit{conjunction}\is{Conjunction} is defined as a clause-initial word which indicates the function of the clause and which is not part of the verb phrase);

\item 
without any special marking. In this case, the relationship between the subordinate clause and the main clause is indicated by the aspectual marker\is{Aspect marker}, possibly in combination with certain postverbal particles;

\item 
using a nominal construction. Properly speaking, such a construction is not an adverbial clause, but as it fulfils similar functions, it will be mentioned in this section as well.

\end{enumerate}

Type 3 clauses are subordinate, even though they lack a conjunction or subordinating marker; this is indicated by the fact that they are negated with the constituent negator \textit{ta{\ꞌ}e} (\sectref{sec:10.5.6}.5), not by a main clause negator. Here is an example of a negated temporal clause. Cf. also \REF{ex:11.257} on p.~\pageref{ex:11.257} (a reason clause marked with \textit{he}).

\ea\label{ex:11.210}
\gll {\ob}I \textbf{ta{\ꞌ}e} kore era tu{\ꞌ}u tokerau era\,{\cb} he mana{\ꞌ}u mo haka tītika  i te vaka ki Tahiti.\\
{\db}\textsc{pfv} \textsc{conneg} lack \textsc{dist} \textsc{poss.2sg.o} wind \textsc{dist} \textsc{ntr} think for \textsc{caus} straight  \textsc{acc} \textsc{art} boat to Tahiti\\

\glt
‘When the wind did not die down, they decided to steer the boat to Tahiti.’ \textstyleExampleref{[R303.064]} 
\z

In the following subsections, adverbial clauses are discussed, grouped by function: time (\sectref{sec:11.6.2}), purpose (\sectref{sec:11.6.3}), reason/result (\sectref{sec:11.6.4}), condition (\sectref{sec:11.6.6}), concession (\sectref{sec:11.6.7}) and circumstance (\sectref{sec:11.6.8}). This is followed by an overview (\sectref{sec:11.6.9}) summarising the different strategies used.

\subsection{Time}\label{sec:11.6.2}
\is{Clause!temporal|(}
A temporal clause\is{Clause!temporal} is a subordinate clause which provides a temporal framework for the event in the main clause. Rapa Nui has a variety of temporal clause\is{Clause!temporal} constructions. Some of these involve a conjunction\is{Conjunction} or a nominal construction; in others, the temporal relation is expressed by an aspectual marker\is{Aspect marker}.

\subsubsection[Cohesive clauses]{Cohesive clauses}\label{sec:11.6.2.1}
\is{Clause!cohesive|(}
In Rapa Nui discourse – especially in narrative – it is common to find an unmarked subordinate clause at the beginning of a sentence, which provides a temporal framework for the main clause. \citet[116]{WeberR2003} labels these \textit{cohesive}: they connect the events to the preceding context and provide a setting for the events that follow. Two examples: 

\ea\label{ex:11.211}
\gll \textbf{I} \textbf{pō} \textbf{era}, he e{\ꞌ}a mai roto mai te vai te hānau {\ꞌ}e{\ꞌ}epe. \\
\textsc{pfv} night \textsc{dist} \textsc{ntr} go\_out from inside from \textsc{art} water \textsc{art} race corpulent \\

\glt 
‘When it had become night, the ‘corpulent race’ came out of the water.’ \textstyleExampleref{[Ley-3-06.046]}
\z

\ea\label{ex:11.212}
\gll \textbf{I} \textbf{poreko} \textbf{era} a Puakiva, he māuiui a Kuha. \\
\textsc{pfv} born \textsc{dist} \textsc{prop} Puakiva \textsc{ntr} sick \textsc{prop} Kuha \\

\glt
‘After Puakiva was born, (his mother) Kuha got sick.’ \textstyleExampleref{[R229.001]} 
\z

Cohesive clauses are characterised by the following features:

\begin{itemize}
\item 
They precede the main clause.

\item 
They do not have a conjunction\is{Conjunction} or subordinating marker.

\item 
They are always predicate-initial, i.e. nothing precedes the verb phrase.

\item 
The aspectual is usually \textit{i}, though \textit{e} and \textit{ka} are also found.

\item 
The verb is almost always followed by a postverbal demonstrative\is{Demonstrative!postverbal}, usually \textit{era}.\footnote{\label{fn:529}In a representative corpus containing 304 \textit{i}{}-marked cohesive clauses\is{Clause!cohesive}, 281 (92.4\%) have \textit{era}; \textit{ai} occurs in 13 clauses (4.3\%), while the remaining clauses have \textit{nei} (7x), \textit{ena} (1x) or no PVD\is{Demonstrative!postverbal} at all (2x).} 

\end{itemize}

As the examples above show, cohesive clauses\is{Clause!cohesive} marked with perfective \textit{i}\is{i (perfective)} express an event anterior to the event in the main clause (\sectref{sec:7.2.4.2}), which provides the setting for the event in the main clause.

Cohesive clauses marked with imperfective \textit{e}\is{e (imperfective)} indicate events simultaneous to the event in the main clause. They may be continuous\is{Aspect!continuous} as in \REF{ex:11.213} or habitual\is{Aspect!habitual} as in \REF{ex:11.214}:

\ea\label{ex:11.213}
\gll \textbf{E} \textbf{ha{\ꞌ}uru} \textbf{nō} \textbf{{\ꞌ}ā} a Eva he hakaroŋo atu ko te re{\ꞌ}o ka raŋi... \\
\textsc{ipfv} sleep just \textsc{cont} \textsc{prop} Eva \textsc{ntr} listen away \textsc{prom} \textsc{art} voice \textsc{cntg} call \\

\glt 
‘When Eva was still asleep, she heard a voice calling...’ \textstyleExampleref{[R210.080]} 
\z

\ea\label{ex:11.214}
\gll \textbf{E} \textbf{kā} \textbf{era} i tou {\ꞌ}umu era paurō te mahana,  {\ꞌ}ina he {\ꞌ}ō{\ꞌ}otu te {\ꞌ}umu e pō rō era.\\
\textsc{ipfv} kindle \textsc{dist} \textsc{acc} \textsc{dem} earth\_oven \textsc{dist} every \textsc{art} day  \textsc{neg} \textsc{pred} cooked \textsc{art} earth\_oven \textsc{ipfv} night \textsc{emph} \textsc{dist}\\

\glt 
‘When they lighted the earth oven every day, the food was not cooked until night.’ \textstyleExampleref{[R352.013]} 
\z

The contiguity marker \textit{ka}\is{ka (aspect marker)} in cohesive clauses\is{Clause!cohesive} expresses temporal contiguity\is{Simultaneity}: the event in the subordinate clause marks the starting point of the event in the main clause.

\ea\label{ex:11.215}
\gll \textbf{Ka} \textbf{tu{\ꞌ}u} \textbf{mai} \textbf{era} a koro ki te kai he oho a Eva he piko. \\
\textsc{cntg} arrive hither \textsc{dist} \textsc{prop} Dad to \textsc{art} eat \textsc{ntr} go \textsc{prop} Eva \textsc{ntr} hide \\

\glt 
‘When Dad came to eat, Eva would go and hide.’ \textstyleExampleref{[R210.026]} 
\z

\ea\label{ex:11.216}
\gll \textbf{Ka} \textbf{hakame{\ꞌ}eme{\ꞌ}e} \textbf{era} he riri a Taparahi. \\
\textsc{cntg} mock \textsc{dist} \textsc{ntr} angry \textsc{prop} Taparahi \\

\glt 
‘When they mocked, Taparahi would get angry.’ \textstyleExampleref{[R250.151]} 
\z

Perfect aspect \textit{ko V {\ꞌ}ā}\is{ko V {\ꞌ}ā (perfect aspect)} in cohesive clauses\is{Clause!cohesive} (as in main clauses) expresses a state resulting from a process. In cohesive clauses\is{Clause!cohesive}, \textit{ko V {\ꞌ}ā} only occurs with stative verbs\is{Verb!stative}.

\ea\label{ex:11.217}
\gll \textbf{Ko} \textbf{nuinui} \textbf{{\ꞌ}ā} a Te Manu he hāipoipo ki tā{\ꞌ}ana vi{\ꞌ}e.\\
\textsc{prf} big:\textsc{red} \textsc{cont} \textsc{prop} Te Manu \textsc{ntr} marry to \textsc{poss.3sg.a} woman\\

\glt 
‘When Te Manu had grown up, he married a (lit. his) woman.’ \textstyleExampleref{[R245.256]} 
\z

\ea\label{ex:11.218}
\gll \textbf{Ko} \textbf{{\ꞌ}ō{\ꞌ}otu}{\rmfnm} \textbf{mai} \textbf{{\ꞌ}ā} te {\ꞌ}umu he ma{\ꞌ}oa.\\
\textsc{prf} cooked hither \textsc{cont} \textsc{art} earth\_oven \textsc{ntr} open\_earth\_oven\\

\glt 
‘When the (food in the) earth oven is cooked, they open it.’ \textstyleExampleref{[R372.075]} 
\z
\footnotetext{\textit{{\ꞌ}Ō{\ꞌ}otu} is a stative verb meaning ‘to be cooked, done’, not an active verb\is{Verb!active} ‘to cook’.}

Concerning the function of cohesive clauses\is{Clause!cohesive} in discourse: in many cases the preposed clause expresses an event which is predictable from the situation or from the preceding events. The event is just to be expected, and therefore it is backgrounded to a subordinate clause. In the following example, the person in question is on his way to Hanga Oteo. Puna Marengo is a place that lies on the way to Hanga Oteo, so it is only natural that he passes it on the way.

\ea\label{ex:11.219}
\gll He e{\ꞌ}a he oho ki Haŋa {\ꞌ}Ōteo. \textbf{I} \textbf{haka} \textbf{noi} \textbf{atu} \textbf{era} {\ꞌ}i ruŋa o te nihinihi era o Puna Māreŋo, he u{\ꞌ}i atu ko te {\ꞌ}au... \\
\textsc{ntr} go\_out \textsc{ntr} go to Hanga Oteo \textsc{pfv} \textsc{caus} incline away \textsc{dist} at above of \textsc{art} curve:\textsc{red} \textsc{dist} of Puna Marengo \textsc{ntr} look away \textsc{prom} \textsc{art} smoke \\

\glt
‘He went out to Hanga Oteo. When he had come down the slope of Puna Marengo, he saw smoke...’ \textstyleExampleref{[R313.091]} 
\z

The preposed clause is not always closely connected to the preceding context, however. It may also have a transitional function, marking the start of a new scene or episode in the story. Such transitional clauses may express a lapse of time between the previous and the next event, or indicate the point in time at which the next events take place:

\ea\label{ex:11.220}
\gll \textbf{I} \textbf{hinihini} \textbf{era} he oho mai he ha{\ꞌ}i i tū poki era {\ꞌ}ā{\ꞌ}ana. \\
\textsc{pfv} delay:\textsc{red} \textsc{dist} \textsc{ntr} go hither \textsc{ntr} embrace \textsc{acc} \textsc{dem} child \textsc{dist} \textsc{poss.3sg.a} \\

\glt 
‘After that, he went to embrace his child.’ \textstyleExampleref{[R210.068]} 
\z

\ea\label{ex:11.221}
\gll \textbf{I} \textbf{tu{\ꞌ}u} \textbf{nei} \textbf{ki} \textbf{te} \textbf{mahana} \textbf{e} \textbf{tahi} he e{\ꞌ}a haka{\ꞌ}ou te taŋata nei... \\
\textsc{pfv} arrive \textsc{prox} to \textsc{art} day \textsc{num} one \textsc{ntr} go\_out again \textsc{art} man \textsc{prox} \\

\glt 
‘When a certain day came, this man went out again...’ \textstyleExampleref{[R310.025]}\textstyleExampleref{} 
\z
\is{Clause!cohesive|)}

\subsubsection[Other unmarked temporal clauses ]{Other unmarked temporal clauses} \label{sec:11.6.2.2}
\is{Clause!temporal}
Apart from cohesive clauses\is{Clause!cohesive}, there are other temporal clauses\is{Clause!temporal} without a conjunction\is{Conjunction} or subordinator. The only way in which these clauses are marked, is by an aspectual which is different from the aspectual in the main clause. They may be marked with \textit{i}, \textit{e} or \textit{ka}.

\subparagraph{Perfective \textit{i}} In \sectref{sec:7.2.4.2} on perfective \textit{i}\is{i (perfective)}, it was shown that \textit{i}{}-marked clauses may express a restatement, conclusion or clarification of the preceding clause. Subordinate \textit{i}{}-marked clauses are somewhat similar in function; they express an event which is \is{Simultaneity}simultaneous to the event expressed in the preceding clause. 

\ea\label{ex:11.222}
\gll Kai take{\ꞌ}a mai \textbf{i} \textbf{u{\ꞌ}i} \textbf{ai} e māua ko Vai Ora. \\
\textsc{neg.pfv} see hither \textsc{pfv} look \textsc{pvp} \textsc{ag} \textsc{1du.excl} \textsc{prom} Vai Ora \\

\glt 
‘We didn’t see (the fish) when Vai Ora and I looked.’ \textstyleExampleref{[R301.292]} 
\z

\ea\label{ex:11.223}
\gll Me{\ꞌ}e koa atu a Tahoŋa \textbf{i} \textbf{e{\ꞌ}a} \textbf{mai} \textbf{ai} mai {\ꞌ}Ōroŋo. \\
thing happy away \textsc{prop} Tahonga \textsc{pfv} go\_out hither \textsc{pfv} from Orongo \\

\glt 
‘Tahonga was happy when he came back from Orongo.’ \textstyleExampleref{[R301.316]} 
\z

\subparagraph{Imperfective \textit{e}} Temporal clauses may also be marked with imperfective \textit{e}\is{e (imperfective)}. These clauses express a continuous event \is{Simultaneity}simultaneous to the one in the main clause. As discussed in \sectref{sec:7.2.5.4}, \textit{e-}marked verbs in main clauses are followed either by a postverbal demonstrative\is{Demonstrative!postverbal} (PVD) or the continuity marker \textit{{\ꞌ}ā/{\ꞌ}ana}. The same is true in temporal clauses\is{Clause!temporal}: the verb is either followed by a PVD\is{Demonstrative!postverbal} \is{e (imperfective)!e V PVD}as in (\ref{ex:11.224}–\ref{ex:11.225}), or by \textit{{\ꞌ}ā/{\ꞌ}ana}\is{e (imperfective)!e V {\ꞌ}ā} as in (\ref{ex:11.226}–\ref{ex:11.227}).

\ea\label{ex:11.224}
\gll He me{\ꞌ}e mai mai roto mai tau {\ꞌ}ana era \textbf{e} \textbf{vero} \textbf{atu} \textbf{era} hai akaue... \\
\textsc{ntr} thing hither from inside from \textsc{dem} cave \textsc{dist} \textsc{ipfv} throw away \textsc{dist} \textsc{ins} stake \\

\glt 
‘They said from inside the cave, while (the enemy) threw sticks at them...’ \textstyleExampleref{[Mtx-3-02.042]}
\z

\ea\label{ex:11.225}
\gll He oho haka{\ꞌ}ou \textbf{e} \textbf{{\ꞌ}ui} \textbf{era} ki te hare, ki te hare era. \\
\textsc{ntr} go again \textsc{ipfv} ask \textsc{dist} to \textsc{art} house to \textsc{art} house \textsc{dist} \\

\glt 
‘He went again, asking from house to house.’ \textstyleExampleref{[R310.152]} 
\z

\ea\label{ex:11.226}
\gll Ko tu{\ꞌ}u haka{\ꞌ}ou mai {\ꞌ}ā tū ŋā poki era \textbf{e} \textbf{ma{\ꞌ}u} \textbf{rō} \textbf{{\ꞌ}ā}  i te ra{\ꞌ}akau.\\
\textsc{prf} arrive again hither \textsc{cont} \textsc{dem} \textsc{pl} child \textsc{dist} \textsc{ipfv} carry \textsc{emph} \textsc{cont}  \textsc{acc} \textsc{art} castor\_oil\_plant\\

\glt 
‘The children had come back, carrying castor oil leaves.’ \textstyleExampleref{[R313.053]} 
\z

\ea\label{ex:11.227}
\gll Terā ka pāhono mai e Vaha \textbf{e} \textbf{koa} \textbf{rō} \textbf{{\ꞌ}ā}... \\
then \textsc{cntg} answer hither \textsc{ag} Vaha \textsc{ipfv} happy \textsc{emph} \textsc{cont} \\

\glt
‘Then Vaha answered happily...’ \textstyleExampleref{[R304.098]} 
\z

Though all these clauses are similar in function, there is a difference between clauses marked with \textit{e V PVD}\is{Demonstrative!postverbal} and the ones marked with \textit{e V {\ꞌ}ā}\is{e (imperfective)!e V {\ꞌ}ā}. The constructions with a PVD\is{Demonstrative!postverbal} can be characterised as true temporal clauses\is{Clause!temporal}, indicating an event which takes place at the same time as the main event. The clauses with \textit{{\ꞌ}ā} are more like circumstantial\is{Clause!circumstantial} or manner clauses, further defining the nature of the event in the main clause or the manner in which it takes place. They have less the character of an independent event and can often be translated with a participle.

There are two indications for the more participial character of the \textit{{\ꞌ}ā} constructions:

%\setcounter{listWWviiiNumlxxiileveli}{0}
\begin{enumerate}
\item 
With \textit{{\ꞌ}ā}, the subject is always the same as in the main clause; in the PVD\is{Demonstrative!postverbal} construction, the subject can be different, as in \REF{ex:11.224}.

\item 
With \textit{{\ꞌ}ā}, the predicate can be an adjective, as in \REF{ex:11.227}; in the PVD\is{Demonstrative!postverbal} construction, this is rare, unless the adjective indicates a process. (The same is true in main clauses, \sectref{sec:7.2.5.4}).

\end{enumerate}

\subparagraph{Contiguity marker \textit{ka}} Subordinate clauses marked with the contiguity marker \textit{ka} indicate an event which is simultaneous\is{Simultaneity} with the event expressed in the main clause: \is{ka (aspect marker)}

\ea\label{ex:11.228}
\gll He ruku te {\ꞌ}atariki, \textbf{ka} \textbf{noho} \textbf{nō} \textbf{atu} te haŋupotu. \\
\textsc{ntr} dive \textsc{art} firstborn \textsc{cntg} stay just away \textsc{art} last\_child \\

\glt 
‘The eldest dived, while the youngest stayed (ashore).’ \textstyleExampleref{[Mtx-7-30.012]}
\z

\ea\label{ex:11.229}
\gll \textbf{Ka} \textbf{turu} \textbf{nei} tāua, he tu{\ꞌ}u mai a koro era ko Vaha ki nei. \\
\textsc{cntg} go\_down \textsc{prox} \textsc{1du.incl} \textsc{ntr} arrive hither \textsc{prop} Dad \textsc{dist} \textsc{prom} Vaha to \textsc{prox} \\

\glt
‘When we go down, father Vaha will come here.’ \textstyleExampleref{[R229.187]} 
\z

As these examples show, the subordinate clause may precede or follow the main clause. As in \REF{ex:11.229}, the verb is often followed by a postverbal demonstrative\is{Demonstrative!postverbal}.

\subsubsection[Development of hora ‘time’ into a pseudo{}-conjunction]{Development of \textit{hora} ‘time’ into a pseudo-conjunction}\label{sec:11.6.2.3}
\is{Conjunction}
Temporal adjuncts can be expressed by a temporal noun preceded by a preposition; the most general temporal noun is \textit{hora} ‘time’. The adjunct can be further specified by a modifier, e.g. a genitive as in \REF{ex:11.230} or a relative clause\is{Clause!relative} as in \REF{ex:11.231}:

\ea\label{ex:11.230}
\gll \textbf{{\ꞌ}I} \textbf{te} \textbf{hora} \textbf{era} {\ꞌ}ana o tō{\ꞌ}oku māmārū{\ꞌ}au era i oti rō ai  rā oho iŋa.\\
at \textsc{art} time \textsc{dist} \textsc{ident} of \textsc{poss.1sg.o} grandmother \textsc{dist} \textsc{pfv} finish \textsc{emph} \textsc{pvp}  \textsc{dist} go \textsc{nmlz}\\

\glt 
‘In my grandmother’s time this custom (lit. going) finished.’ \textstyleExampleref{[R648.137]} 
\z

\ea\label{ex:11.231}
\gll \textbf{{\ꞌ}I} \textbf{te} \textbf{hora} \textbf{era} e ora nō {\ꞌ}ā tā{\ꞌ}ana kenu era, {\ꞌ}ā{\ꞌ}ana te oho  ki te kona aŋa...\\
at \textsc{art} time \textsc{dist} \textsc{ipfv} live just \textsc{cont} \textsc{poss.3sg.a} husband \textsc{dist} \textsc{poss.3sg.a} \textsc{art} go  to \textsc{art} place work\\

\glt
‘At the time when her husband was still alive, she was the one who would go to work...’ \textstyleExampleref{[R349.005]} 
\z

Now as discussed in \sectref{sec:5.3.2.3}, the article can be omitted before clause-initial nouns followed by a demonstrative like \textit{era}. At the same time, the preposition \textit{{\ꞌ}i} can be omitted as well. This results in constructions like the following:

\ea\label{ex:11.232}
\gll \textbf{Hora} \textbf{ena} e vānaŋa {\ꞌ}ā ki te rua, ¿e u{\ꞌ}i rō {\ꞌ}ā koe  a roto i te mata?\\
time \textsc{med} \textsc{ipfv} talk \textsc{cont} to \textsc{art} other ~\textsc{ipfv} look \textsc{emph} \textsc{cont} \textsc{2sg}  by inside at \textsc{art} eye\\

\glt 
‘When you talk to someone else, do you look (them) in the eyes?’ \textstyleExampleref{[R209.027]} 
\z

\ea\label{ex:11.233}
\gll \textbf{Hora} take{\ꞌ}a \textbf{era} e au, ¡{\ꞌ}ai te nehenehe!\\
time see \textsc{dist} \textsc{ag} \textsc{1sg} ~there \textsc{art} beautiful\\

\glt
‘When I saw her, she was so beautiful!’ \textstyleExampleref{[R413.099]} 
\z

In the constructions above, \textit{hora ena/era} resembles a temporal conjunction\is{Conjunction}; semantic bleaching is taking place, where \textit{hora ena/era} comes to mean little more than ‘when’. Notice however, that the construction is syntactically still a nominal phrase with relative clause\is{Clause!relative}: as \REF{ex:11.233} shows, the aspectual can be omitted, something which is only possible in relative clauses\is{Clause!relative} (\sectref{sec:11.4.5}). (Also, the verb \textit{take{\ꞌ}a} has been raised from the relative clause\is{Clause!relative}.)

\subsubsection{Anteriority: ‘before’}\label{sec:11.6.2.4}

Rapa Nui has a variety of devices to express that the event in the subordinate clause takes place prior to the event in the main clause. One of these is preverbal \textit{mai}, discussed in \sectref{sec:11.5.5}. The following strategies are also used:

\subparagraph{\textit{{\ꞌ}I ra{\ꞌ}e}}\is{rae ‘first’@ra{\ꞌ}e ‘first’} \textit{Ra{\ꞌ}e} is a locational\is{Locational} meaning ‘first’ (\sectref{sec:3.6.4.1}). \textit{{\ꞌ}I ra{\ꞌ}e ki}, followed by a nominalised verb\is{Verb!nominalised}, means ‘before’:

\ea\label{ex:11.234}
\gll Paurō te mahana e {\ꞌ}ara era {\ꞌ}i te pō era {\ꞌ}ā, \textbf{{\ꞌ}i} \textbf{ra{\ꞌ}e} \textbf{ki} te e{\ꞌ}a  o te ra{\ꞌ}ā.\\
every \textsc{art} day \textsc{ipfv} wake\_up \textsc{dist} at \textsc{art} night \textsc{dist} \textsc{ident} at first to \textsc{art} go\_out  of \textsc{art} sun\\

\glt 
‘Every day he woke up early in the morning, before the sun came up.’ \textstyleExampleref{[R448.003]} 
\z

\subparagraph{\textit{Ante}}\is{ante ‘before’} \textit{Ante} ({\textless} Sp. \textit{antes}) is used as an adverb\is{Adverb} meaning ‘before, earlier, previously’. It is also used as a conjunction\is{Conjunction}, followed by \textit{ki} + nominalised verb\is{Verb!nominalised}:

\ea\label{ex:11.235}
\gll Pero \textbf{ante} \textbf{ki} te uru, he oho tahi te ŋā poki he fira ra{\ꞌ}e. \\
but before to \textsc{art} enter \textsc{ntr} go all \textsc{art} \textsc{pl} child \textsc{ntr} line first \\

\glt 
‘But before going in, the children first go and stand in line.’ \textstyleExampleref{[R151.012]} 
\z

\subparagraph{\textit{{\ꞌ}Ō ira}}\is{o ira ‘before’@{\ꞌ}ō ira ‘before’} \textit{{\ꞌ}ō ira} ‘before’\footnote{\label{fn:531}Not to be confused with \textit{{\ꞌ}o ira} ‘therefore’ (\sectref{sec:11.6.4}).} consists of the otherwise unknown particle \textit{{\ꞌ}ō}, followed by the pro-form \textit{ira} (\sectref{sec:4.6.5.2}). It is always followed by a \textit{ka}{}-marked verb. As \REF{ex:11.236} shows, the subject after \textit{{\ꞌ}ō ira} is usually preverbal.

\ea\label{ex:11.236}
\gll Te rāua henua ra{\ꞌ}e i noho ko Perú, \textbf{{\ꞌ}ō\_ira} te Inca ka tu{\ꞌ}u. \\
\textsc{art} \textsc{3pl} land first \textsc{pfv} stay \textsc{prom} Peru  before \textsc{art} Inca \textsc{cntg} arrive \\

\glt 
‘The first land where they lived was Peru, before the Incas arrived.’ \textstyleExampleref{[R376.011]} 
\z

\subparagraph{\textit{Hia}}\is{hia ‘not yet’} The postverbal marker \textit{hia}, combined with a negation, means ‘not yet’; in a multiclause construction it indicates that an event has not happened before another occurs (\sectref{sec:10.5.8}).

\subsubsection{Temporal limit: ‘until’}\label{sec:11.6.2.5}

\is{ka (aspect marker)}‘Until’ is often expressed by the aspectual \textit{ka} (\sectref{sec:7.2.6}) in combination with the emphatic marker \textit{rō} (\sectref{sec:7.4.2}). This is in line with the function of \textit{ka} as a contiguity marker: the event or state expressed in the \textit{ka-}clause marks the temporal boundary of another event, often indicating the natural or expected outcome of an action performed to completion. These \textit{ka-}clauses usually occur sentence-finally.

\ea\label{ex:11.237}
\gll He kai a Te Manu \textbf{ka} \textbf{mākona} \textbf{rō}. \\
\textsc{ntr} eat \textsc{prop} Te Manu \textsc{cntg} satiated \textsc{emph} \\

\glt 
‘Te Manu ate until he was satiated.’ \textstyleExampleref{[R245.067]} 
\z

\ea\label{ex:11.238}
\gll I noho ai a Te Manu {\ꞌ}i muri tū pāpārū{\ꞌ}au era \textbf{ka} \textbf{rova{\ꞌ}a} \textbf{rō}  ho{\ꞌ}e {\ꞌ}ahuru tūma{\ꞌ}a matahiti.\\
\textsc{pfv} stay \textsc{pvp} \textsc{prop} Te Manu at near \textsc{dem} grandfather \textsc{dist} \textsc{cntg} obtain \textsc{emph}  one ten more\_or\_less year\\

\glt
‘Te Manu stayed with his grandfather until he was about ten years old.’ \textstyleExampleref{[R245.159]} 
\z

In the examples above, the subject of the main clause reaches a certain state or end point; for example, in \REF{ex:11.238}, Te Manu reaches a state of satiation after having eaten. The stative verb in the \textit{ka V rō}\is{ka (aspect marker)!ka V rō} clause may also specify the action of the main clause, which is performed – or is to be performed – to a certain extent or in a certain way. (Cf. the use of \textit{ka} before numerals to mark an extent, \sectref{sec:4.3.2.2}).

\ea\label{ex:11.239}
\gll E hatu era ki a {\ꞌ}Ohovehi \textbf{ka} \textbf{rivariva} \textbf{rō}. \\
\textsc{exh} advise \textsc{dist} to \textsc{prop} Ohovehi \textsc{cntg} good:\textsc{red} \textsc{emph} \\

\glt 
‘Advise Ohovehi well!’ \textstyleExampleref{[R310.277]} 
\z

\ea\label{ex:11.240}
\gll He uru atu ararua he here i te kūpeŋa \textbf{ka} \textbf{hiohio} \textbf{rō}. \\
\textsc{ntr} enter away the\_two \textsc{ntr} tie \textsc{acc} \textsc{art} net \textsc{cntg} strong:\textsc{red} \textsc{emph} \\

\glt 
‘The two went in and tied the net firmly.’ \textstyleExampleref{[R310.397]}\textstyleExampleref{} 
\z

A second way to express ‘until’ is by means of \textit{{\ꞌ}ātā}\is{ata ‘until’@{\ꞌ}ātā ‘until’} ({\textless} Sp. \textit{hasta}). \textit{{\ꞌ}Ātā} is used in nominal constructions before the preposition \textit{ki} (see \REF{ex:4.267} on p.~\pageref{ex:4.267}), but also in verbal constructions, followed by \textit{ka V rō}\is{ka (aspect marker)!ka V rō}. As \REF{ex:11.242} shows, \textit{{\ꞌ}ātā} may be shortened to \textit{{\ꞌ}ā}:

\ea\label{ex:11.241}
\gll Mai ki hāpa{\ꞌ}o nō tātou i a ia \textbf{{\ꞌ}ātā} \textbf{ka} nuinui \textbf{rō}.\\
hither \textsc{hort} care\_for just \textsc{1pl.incl} \textsc{acc} \textsc{prop} \textsc{3sg} until \textsc{cntg} big:\textsc{red} \textsc{emph}\\

\glt 
‘Let us take care of him until he is big.’ \textstyleExampleref{[R211.063]} 
\z
%\todo[inline]{In previous version, there were two empty lines between glosses and free translation. I removed spaces in front of \\, see if empty lines are gone.}

\ea\label{ex:11.242}
\gll ...{\ꞌ}ai ka haka teka ka oho ki Haŋa Piko  \textbf{{\ꞌ}ā} tāua \textbf{ka} tomo \textbf{rō} nei.\\
~~~~there \textsc{cntg} \textsc{caus} turn \textsc{cntg} go to Hanga Piko  until \textsc{1du.incl} \textsc{cntg} go\_ashore \textsc{emph} \textsc{prox}\\

\glt 
‘...then we will turn and go to Hanga Piko, until we come ashore.’ \textstyleExampleref{[R230.401]} 
\z

In the third place: less commonly, the conjunction\is{Conjunction} \textit{{\ꞌ}ahara}\is{ahara ‘until’@{\ꞌ}ahara ‘until’} is used, followed by \textit{ka}:

\ea\label{ex:11.243}
\gll He noho rō {\ꞌ}ai tāua \textbf{{\ꞌ}ahara} \textbf{ka} haka hoki rō koe i a au  ki mu{\ꞌ}a ki tō{\ꞌ}oku nu{\ꞌ}u.\\
\textsc{ntr} stay \textsc{emph} \textsc{subs} \textsc{1du.incl} until \textsc{cntg} \textsc{caus} return \textsc{emph} \textsc{2sg} \textsc{acc} \textsc{prop} \textsc{1sg}  to before to \textsc{poss.1sg.o} people\\

\glt 
‘We will stay, until you make me return to my people.’ \textstyleExampleref{[Fel-1978.115]}
\z

Finally, ‘until’ may be expressed by the subordinator \textit{ki}, especially after verbs like \textit{tiaki} ‘wait’ (\sectref{sec:11.5.3}).

\is{Clause!temporal|)}
\subsection{Purpose: bare purpose clauses}\label{sec:11.6.3}
\is{Clause!purpose!bare|(}
Purpose clauses\is{Clause!purpose} are often marked with preverbal \textit{mo} (\sectref{sec:11.5.1.1}) or \textit{ki} (\sectref{sec:11.5.3}). Purpose may also be expressed by a bare verb, i.e. a verb without aspect marker\is{Aspect marker}.\footnote{\label{fn:532}Clauses with a bare verb cannot be analysed as juxtaposed main clauses, as main clause verbs always have an aspect marker, except occasionally when the verb is followed by certain postverbal particles (\sectref{sec:7.2.2}).} This verb is always initial in the clause. Bare purpose clauses\is{Clause!purpose} are found especially after motion verbs\is{Verb!motion}. A few examples:

\ea\label{ex:11.244}
\gll Paurō te mahana e e{\ꞌ}a era te poki ki haho \textbf{māta{\ꞌ}ita{\ꞌ}i} i te raŋi  {\ꞌ}e i te vaikava.\\
every \textsc{art} day \textsc{ipfv} go\_out \textsc{dist} \textsc{art} child to outside observe \textsc{acc} \textsc{art} sky  and \textsc{acc} \textsc{art} ocean\\

\glt 
‘Every morning the child went outside to watch the sky and the sea’ \textstyleExampleref{[R532-07.004]}
\z

\ea\label{ex:11.245}
\gll He oti te kai, he moe te {\ꞌ}ariki ki raro \textbf{haka} \textbf{ora}. \\
\textsc{ntr} finish \textsc{art} eat \textsc{ntr} lie \textsc{art} king to below \textsc{caus} live \\

\glt
‘When the meal was finished, the king lay down to rest.’ \textstyleExampleref{[Ley-2-10.017]}
\z

More commonly, the purpose of an action is expressed by a noun phrase introduced by the preposition \textit{ki}, followed by a bare verb. Here are a few examples:

\ea\label{ex:11.246}
\gll Te poki nei i iri atu ai \textbf{ki} \textbf{te} \textbf{tarake} \textbf{toke}. \\
\textsc{art} child \textsc{prox} \textsc{pfv} ascend away \textsc{pvp} to \textsc{art} corn steal \\

\glt 
‘This boy went (to the field) to steal corn.’ \textstyleExampleref{[R132.003]} 
\z

\ea\label{ex:11.247}
\gll He iri ararua \textbf{ki} \textbf{te} \textbf{rāua} \textbf{hoi} \textbf{{\ꞌ}a{\ꞌ}aru} \textbf{mai}.\\
\textsc{ntr} ascend the\_two to \textsc{art} \textsc{3pl} horse grab hither\\

\glt 
‘Both of them went to grab their horse.’ \textstyleExampleref{[R170.002]} 
\z

\ea\label{ex:11.248}
\gll He turu tahi mātou \textbf{ki} \textbf{te} \textbf{pērīkura} \textbf{māta{\ꞌ}ita{\ꞌ}i} {\ꞌ}i te hare hāpī era. \\
\textsc{ntr} go\_down all \textsc{1pl.excl} to \textsc{art} movie watch at \textsc{art} house learn \textsc{dist} \\

\glt
‘We all went down to watch a movie at school.’ \textstyleExampleref{[R410.010]} 
\z

In these examples, the main verb is a motion verb\is{Verb!motion}; the \textit{ki}{}-marked noun phrase is the Goal of movement. This noun phrase is followed by a bare verb, of which the preceding noun is the Patient.\footnote{\label{fn:533}\citet[424]{Clark1983Review} points out that the same construction occurs in \ili{Marquesan} and \ili{Mangarevan}. Different from what Clark suggests, in Rapa Nui this construction is not limited to generic objects, as \REF{ex:11.247} shows.}  

The noun in this construction is not an incorporated object\is{Object!incorporation} of the following verb: it is the head of a regular noun phrase, marked with the article \textit{te} and preceded by a preposition. A somewhat more plausible analysis would be to consider the verb as incorporated into the noun; however, the directional\is{Directional} \textit{mai} in \REF{ex:11.247} shows that the verb is the head of a true verb phrase. It is best to analyse these constructions simply as a combination of a noun phrase and a bare purpose clause, rather than assuming that the noun phrase + verb are a single constituent. An additional reason to do so is that this construction is not an isolated phenomenon, but an instance (admittedly, the most common instance) of a group of constructions in which a locative noun phrase and a purpose clause occur together. Related constructions include:

\begin{itemize}
\item
a \textit{ki te N V} construction where the noun is not the verb’s Patient:
\end{itemize}

\ea\label{ex:11.249}
\gll I oti era he turu ki raro \textbf{ki} \textbf{te} \textbf{teata} \textbf{māta{\ꞌ}ita{\ꞌ}i}. \\
\textsc{pfv} finish \textsc{dist} \textsc{ntr} go\_down to below to \textsc{art} cinema watch \\

\glt
‘After that, they went down to the cinema to watch (a movie).’ \textstyleExampleref{[R210.145]} 
\z

\begin{itemize}
\item 
a Source noun phrase (with preposition \textit{mai} ‘from’) followed by a bare verb:
\end{itemize}

\ea\label{ex:11.250}
\gll He tu{\ꞌ}u mai tau vi{\ꞌ}e matu{\ꞌ}a era \textbf{mai} \textbf{te} \textbf{kūmara} \textbf{keri}. \\
\textsc{ntr} arrive hither \textsc{dem} woman parent \textsc{dist} from \textsc{art} sweet\_potato dig \\

\glt
‘The mother came (back) from harvesting sweet potatoes.’ \textstyleExampleref{[MsE-094.006]}
\z

\begin{itemize}
\item 
a \textit{ki}{}-marked Goal noun phrase followed by a \textit{mo}{}-marked purpose clause:
\end{itemize}

\ea\label{ex:11.251}
\gll {\ꞌ}I te ahiahi he e{\ꞌ}a a {\ꞌ}Orohe ki ruŋa i te vaka  \textbf{ki} \textbf{te} \textbf{ika} \textbf{mo} \textbf{ma{\ꞌ}u} \textbf{mai}.\\
at \textsc{art} afternoon \textsc{ntr} go\_out \textsc{prop} Orohe to above at \textsc{art} boat  to \textsc{art} fish for carry hither\\

\glt
‘In the afternoon Orohe went out by boat to bring fish.’ \textstyleExampleref{[R160.005]} 
\z

\begin{itemize}
\item 
a \textit{ki}{}-marked Goal noun phrase, with the associated action left implicit:
\end{itemize}

\ea\label{ex:11.252}
\gll —¿Ki hē a kuā {\ꞌ}Orohe i iri ai {\ꞌ}i ruŋa i te vaka?  —\textbf{Ki} \textbf{te} \textbf{rāua} \textbf{ika} {\ꞌ}i ruŋa i te toka.\\
~~~~~to~ \textsc{cq} \textsc{prop} \textsc{coll} Orohe \textsc{pfv} ascend \textsc{pfv} at above at \textsc{art} boat  ~~~to~ \textsc{art} \textsc{3pl} fish at above at \textsc{art} rock\\

\glt
‘—Where did Orohe and the others go by boat? —To their fish (i.e. to catch fish) on the rocks.’ \textstyleExampleref{[R154.038]}\textstyleExampleref{} 
\z

\begin{itemize}
\item 
a \textit{mo}{}-marked Goal noun phrase followed by a purpose clause (the latter may be either bare or marked with \textit{mo}):
\end{itemize}

\ea\label{ex:11.253}
\gll ...{\ꞌ}ai ka ma{\ꞌ}u atu ki hiva \textbf{mo} \textbf{te} \textbf{purumu} \textbf{mo} \textbf{aŋa}. \\
~~~there \textsc{cntg} carry away to mainland for \textsc{art} broom for make \\

\glt
‘...then they transported (the horsehair) to the mainland to make brooms.’ \textstyleExampleref{[R539-02.091]}
\z

These examples suggest that \textit{ki te N V} in (\ref{ex:11.246}–\ref{ex:11.248}) should not be analysed as a special construction involving a single NP+V constituent. Rather, it is a combination of two constituents, a nominal Goal phrase followed by a bare purpose clause, either of which may also occur on its own. 
\is{Clause!purpose!bare|)}

\subsection{Reason}\label{sec:11.6.4}
\is{Clause!reason|(}
Reason clauses can be constructed in several ways. In the first place, reason is often expressed by nominalised\is{Verb!nominalised} clauses marked with the prepositions \textit{{\ꞌ}i} and \textit{{\ꞌ}o} (\sectref{sec:4.7.2.2}). 

Secondly, in modern Rapa Nui, the phrase \is{mee ‘thing’@me{\ꞌ}e ‘thing’!{\ꞌ}i te me{\ꞌ}e ‘because’}\textit{{\ꞌ}i te me{\ꞌ}e (era)} (lit. ‘in the thing’ or ‘because of the thing’) is used as a conjunction\is{Conjunction} introducing a reason clause. As the examples show, the reason clause either precedes or follows the main clause.

\ea\label{ex:11.254}
\gll He ri{\ꞌ}ari{\ꞌ}a \textbf{{\ꞌ}i} \textbf{te} \textbf{me{\ꞌ}e} \textbf{era} ko piri {\ꞌ}ā ki a rāua te ta{\ꞌ}oraha. \\
\textsc{ntr} afraid at \textsc{art} thing \textsc{dist} \textsc{prf} join \textsc{cont} to \textsc{prop} \textsc{3pl} \textsc{art} whale \\

\glt 
‘They are afraid because whales approach them.’ \textstyleExampleref{[R364.038–039]} 
\z

\ea\label{ex:11.255}
\gll Bueno, \textbf{{\ꞌ}i} \textbf{te} \textbf{me{\ꞌ}e} \textbf{era} e {\ꞌ}iti{\ꞌ}iti nō {\ꞌ}ā au {\ꞌ}ina he ha{\ꞌ}ati{\ꞌ}a mai  e tō{\ꞌ}oku pāpā era mo eke ki ruŋa te hoi.\\
good at \textsc{art} thing \textsc{dist} \textsc{ipfv} small:\textsc{red} just \textsc{cont} \textsc{1sg} \textsc{neg} \textsc{ntr} permit hither  \textsc{ag} \textsc{poss.1sg.o} father \textsc{dist} for go\_up to above \textsc{art} horse\\

\glt 
‘OK, because I was little, my father didn’t allow me to mount a horse.’ \textstyleExampleref{[R101.004]} 
\z

Thirdly, the reason clause may also be a subordinate clause marked with the aspectual \textit{he}\is{he (nominal predicate marker)}. That this is a subordinate clause, is shown by the fact that it is negated with the constituent negator \textit{ta{\ꞌ}e}\is{tae (negator)@ta{\ꞌ}e (negator)} (\sectref{sec:10.5.6}); main clauses would have a different negator. 

\ea\label{ex:11.256}
\gll I tu{\ꞌ}u mai ai ki Rapa Nui mai Marite \textbf{he} ai o te aŋa  o tō{\ꞌ}ona matu{\ꞌ}a tane {\ꞌ}i nei.\\
\textsc{pfv} arrive hither \textsc{pvp} to Rapa Nui from America \textsc{ntr}/\textsc{pred} exist of \textsc{art} work  of \textsc{poss.3sg.a} parent male at \textsc{prox}\\

\glt 
‘He came to Rapa Nui from America because his father had work here.’ \textstyleExampleref{[R461.002]} 
\z

\ea\label{ex:11.257}
\gll Te nu{\ꞌ}u nei i tētere ai \textbf{he} \textbf{ta{\ꞌ}e} ha{\ꞌ}ati{\ꞌ}a e te hua{\ꞌ}ai  mo hāipoipo ararua.\\
\textsc{art} people \textsc{prox} \textsc{pfv} \textsc{pl}:run \textsc{pvp} \textsc{ntr}/\textsc{pred} \textsc{conneg} permit \textsc{ag} \textsc{art} family  for marry the\_two\\

\glt
‘These people fled because their family did not allow them to marry.’ \textstyleExampleref{[R303.144]} 
\z

In these constructions, \textit{he} can also be considered as a nominal predicate marker followed by a nominalised verb (hence the double gloss in the examples above)\is{Verb!nominalised}. One reason to do so, is that other nominal constructions are also used to express reasons: in \REF{ex:11.258} a nominalised verb\is{Verb!nominalised} preceded by a possessive pronoun, in \REF{ex:11.259} a subordinate existential construction (an existential main clause would be \textit{{\ꞌ}Ina he me{\ꞌ}e mo kai}):

\ea\label{ex:11.258}
\gll \textbf{Tu{\ꞌ}u} \textbf{ta{\ꞌ}e} \textbf{hakaroŋo} ena ki te vānaŋa o te taote; {\ꞌ}o ira  koe i māuiui haka{\ꞌ}ou ena.\\
\textsc{poss.2sg.o} \textsc{conneg} listen \textsc{med} to \textsc{art} word of \textsc{art} doctor because\_of \textsc{ana}  \textsc{2sg} \textsc{pfv} sick again \textsc{med}\\

\glt 
‘You didn’t listen to the words of the doctor, therefore you got sick again.’ \textstyleExampleref{[R237.087]} 
\z

\ea\label{ex:11.259}
\gll \textbf{He} \textbf{me{\ꞌ}e} \textbf{kore} mo kai, {\ꞌ}o ira au e taŋi nei. \\
\textsc{pred} thing lack for eat because\_of \textsc{ana} \textsc{1sg} \textsc{ipfv} cry \textsc{prox} \\

\glt 
‘There is nothing (lit. the lack of things) to eat, therefore I am crying.’ \textstyleExampleref{[R349.013]} 
\z
\is{Clause!reason|)}

\subsection{Result}\label{sec:11.6.5}

\is{Clause!result|(}Results may be marked by the adverbial connector \textit{{\ꞌ}o}\is{o ‘because of’@{\ꞌ}o ‘because of’} \textit{ira}\is{ira (anaphor)} ‘because of that; therefore’ (the reason preposition \textit{{\ꞌ}o} followed by the pro-form \textit{ira}). As \REF{ex:11.261} shows, it is possible to mark both the reason clause (in this case, a nominal construction) and the result clause.

\ea\label{ex:11.260}
\gll {\ꞌ}Ina pa{\ꞌ}i o māua kona mo noho. \textbf{{\ꞌ}O} \textbf{ira} au i iri mai nei  ki a koe...\\
\textsc{neg} in\_fact of \textsc{1du.excl} place for stay because\_of \textsc{ana} \textsc{1sg} \textsc{pfv} ascend hither \textsc{prox}  to \textsc{prop} \textsc{2sg}\\

\glt 
‘We don’t have a place to live. Therefore I have come up to you...’ \textstyleExampleref{[R229.210–211]}
\z

\ea\label{ex:11.261}
\gll {\ꞌ}I te ta{\ꞌ}e hakaroŋo ō{\ꞌ}ou \textbf{{\ꞌ}o} \textbf{ira} koe i hiŋa ena.\\
at \textsc{art} \textsc{conneg} listen \textsc{poss.2sg.o} because\_of \textsc{ana} \textsc{2sg} \textsc{pfv} fall \textsc{med}\\

\glt
‘Because you didn’t listen, therefore you fell.’ \textstyleExampleref{[R481.136]} 
\z

As these examples show, the subject tends to be placed straight after \textit{{\ꞌ}o ira}. This conforms to a general preference for preverbal subjects\is{Subject!preverbal} after initial oblique constituents (\sectref{sec:8.6.1.1}).
\is{Clause!result|)}
\subsection{Condition} \label{sec:11.6.6}
\is{Clause!conditional|(}\is{Clause!conditional}
Conditional clauses\is{Clause!conditional} can be marked by one of the subordinators \textit{mo} (\sectref{sec:11.5.1.1}) and \textit{ana} (\sectref{sec:11.5.2.2}). 

Condition is not always marked, however: clauses with a conditional sense may also occur without special marking. The verb is marked with one of the aspectuals \textit{i}, \textit{e} or \textit{ka} and followed by a postverbal demonstrative. Two examples:

\ea\label{ex:11.262}
\gll {\ꞌ}E \textbf{i} \textbf{haŋa} \textbf{era} koe mo rere ki ta{\ꞌ}a kona i mana{\ꞌ}u,  he rere rō {\ꞌ}ai koe....\\
and \textsc{pfv} want \textsc{dist} \textsc{2sg} for fly to \textsc{poss.2sg.a} place \textsc{pfv} think  \textsc{ntr} fly \textsc{emph} \textsc{subs} \textsc{2sg}\\

\glt 
‘And if you want to fly to the place you think of, you (can) fly...’ \textstyleExampleref{[R378.006]} 
\z

\ea\label{ex:11.263}
\gll \textbf{Ka} \textbf{hāŋai} \textbf{atu} \textbf{ena} ki a koe, he mate koe. \\
\textsc{cntg} feed away \textsc{med} to \textsc{prop} \textsc{2sg} \textsc{ntr} die \textsc{2sg} \\

\glt
‘If (the two spirits) feed you, you will die.’ \textstyleExampleref{[R310.061]} 
\z

The contiguity marker \textit{ka}\is{ka (aspect marker)} is relatively common in clauses expressing a condition\is{Clause!conditional}. It seems natural that a marker which indicates temporal contiguity\is{Contiguity, temporal} (simultaneous or sequential events) also marks logical contiguity, i.e. contingency of one event on another.

To mark irreal conditions, the conjunction\is{Conjunction} \textit{{\ꞌ}āhani}\is{ahani ‘if only’@{\ꞌ}āhani ‘if only’} (var. \textit{{\ꞌ}ani, {\ꞌ}ahari}) is used.\footnote{\label{fn:534}{\ꞌ}Āhani {\textless} Tah. \textit{{\ꞌ}ahani}, a var. of \textit{{\ꞌ}ahari/{\ꞌ}ahiri/{\ꞌ}ahini}, which is likewise a conjunction introducing an irreal condition clause.}

\ea\label{ex:11.264}
\gll \textbf{{\ꞌ}Āhani} {\ꞌ}ō au he {\ꞌ}ono, ko ho{\ꞌ}o mai {\ꞌ}ā au i te hare e tahi... \\
if\_only really \textsc{1sg} \textsc{pred} rich \textsc{prf} buy hither \textsc{cont} \textsc{1sg} \textsc{acc} \textsc{art} house \textsc{num} one \\

\glt 
‘If I were rich, I would buy a house...’ \textstyleExampleref{[R399.182]} 
\z

\ea\label{ex:11.265}
\gll \textbf{{\ꞌ}Āhani} {\ꞌ}ō tō{\ꞌ}oku nua era i ta{\ꞌ}e mate,  {\ꞌ}ī au {\ꞌ}i muri i a ia {\ꞌ}i te hora nei.\\
if\_only really \textsc{poss.1sg.o} Mum \textsc{dist} \textsc{pfv} \textsc{conneg} die  \textsc{imm} \textsc{1sg} at near at \textsc{prop} \textsc{3sg} at \textsc{art} time \textsc{prox}\\

\glt
‘If my mother hadn’t died, I would be with her now.’ \textstyleExampleref{[R245.007]} 
\z

As these examples show, the subject\is{Subject!preverbal} after \textit{{\ꞌ}āhani} is usually \is{Clause!conditional|)}preverbal (\sectref{sec:8.6.1.1}). 

\subsection{Concession}\label{sec:11.6.7}
\is{Clause!concessive|(}
The aspectual marker\is{Aspect marker} \textit{ka}\is{ka (aspect marker)}, in combination with the directional \textit{atu}\is{atu ‘away’}, can be used in a concessive sense, indicating a circumstance which might be expected to prevent – but actually does not prevent – the event in the main clause.\footnote{\label{fn:535}This does not mean that all \textit{ka V atu} constructions have a concessive sense, see e.g. example \REF{ex:11.263} above.}

\ea\label{ex:11.266}
\gll \textbf{Ka} \textbf{rahi} \textbf{atu} tā{\ꞌ}aku poki, e hāpa{\ꞌ}o nō e au {\ꞌ}ā. \\
\textsc{cntg} many away \textsc{poss.1sg.a} child \textsc{ipfv} care\_for just \textsc{ag} \textsc{1sg} \textsc{ident} \\

\glt
‘Even if I have many children, I will care for them myself.’ \textstyleExampleref{[R229.023]} 
\z

As discussed in \sectref{sec:7.2.6}, \textit{ka} expresses temporal contiguity\is{Contiguity, temporal}; the concessive sense follows in a way from this basic sense. By explicitly juxtaposing two events or situations which are temporally contiguous or simultaneous, the contrast between the two is highlighted.\footnote{\label{fn:536}The same use can be observed for constructions expressing simultaneity in other languages. \ili{English} ‘while’ can be used in the sense ‘even though’ (‘While he had a good job, he did not earn enough to support his expensive tastes.’). The \ili{French} \textit{gérondif}, preceded by ‘tout en’, has a concessive sense (‘La police a des soupçons tout en ignorant l’identité du coupable’ = ‘The police has suspicions, but does not know the identity of the culprit.’).} 

The \textit{ka V atu} construction with concessive sense is especially common with the existential verb \textit{ai}, in the expressions \textit{ka ai atu} ‘even’ and \textit{ka ai atu pē ira/nei} ‘even though; even so’:

\ea\label{ex:11.267}
\gll \textbf{Ka} \textbf{ai} \textbf{atu} te me{\ꞌ}e {\ꞌ}iti{\ꞌ}iti hope{\ꞌ}a, he tau nō ki a au. \\
\textsc{cntg} exist away \textsc{art} thing small:\textsc{red} last \textsc{ntr} pretty just to \textsc{prop} \textsc{1sg} \\

\glt 
‘Even the smallest things are beautiful to me.’ \textstyleExampleref{[R224.037–038]}
\z

\ea\label{ex:11.268}
\gll E haka topa rō mai {\ꞌ}ā mai roto tētahi nūna{\ꞌ}a henua  \textbf{ka} \textbf{ai} \textbf{atu} \textbf{pē} \textbf{nei} ē: {\ꞌ}i te Pacífico {\ꞌ}ā.\\
\textsc{ipfv} \textsc{caus} happen \textsc{emph} hither \textsc{cont} from inside some group land  \textsc{cntg} exist away like \textsc{prox} thus at \textsc{art} Pacific \textsc{ident}\\

\glt 
‘Some groups of islands are excluded (from Oceania), even though they are in the Pacific.’ \textstyleExampleref{[R342.005]} 
\z

A second way to express concession is by means of the preposition \textit{nōatu}\is{noatu ‘no matter’@nōatu ‘no matter’},\footnote{\label{fn:537}\textit{Nōatu} {\textless} \textit{nō} ‘just’ + \textit{atu} ‘away’, but probably borrowed as a whole from \ili{Tahitian} \textit{noātu} (\citealt[310]{AcadémieTahitienne1986}).} followed by a nominalised verb\is{Verb!nominalised}:

\ea\label{ex:11.269}
\gll \textbf{Nōatu} te paŋaha{\ꞌ}a, te mahana te mahana e hāpī ena {\ꞌ}i ira.\\
no\_matter \textsc{art} heavy \textsc{art} day \textsc{art} day \textsc{ipfv} teach \textsc{med} at \textsc{ana}\\

\glt 
‘Even though it's heavy, they teach there day after day.’ \textstyleExampleref{[R537.023]} 
\z

Finally, concession is expressed by the adverbial expression \is{mee ‘thing’@me{\ꞌ}e ‘thing’!te me{\ꞌ}e nō ‘however’}\textit{te me{\ꞌ}e nō} ‘however, even so’, which functions as a coordinating conjunction (\sectref{sec:5.8.2.4}):

\ea\label{ex:11.270}
\gll {\ꞌ}Apa te toe a au he mate; \textbf{te} \textbf{me{\ꞌ}e} \textbf{nō}, {\ꞌ}ī a au e ora nō {\ꞌ}ā.\\
half \textsc{art} remain \textsc{prop} \textsc{1sg} \textsc{ntr} die \textsc{art} thing just \textsc{imm} \textsc{prop} \textsc{1sg} \textsc{ipfv} live just \textsc{cont}\\

\glt 
‘I almost died; even so, I am alive.’ \textstyleExampleref{[R437.050]}\textstyleExampleref{} 
\z
\is{Clause!concessive|)}

\subsection{Circumstance}\label{sec:11.6.8}
\is{Clause!circumstantial|(}
Circumstantial clauses may be expressed by \textit{koia ko}\is{koia ko ‘with’} ‘with’ preceding the verb (\sectref{sec:8.10.4.2}):

%The following has a wrong number in Word-version 5 Oct. From here on, numbering in this file is 1 higher.

\ea\label{ex:11.271}
\gll He hoki mai a Kāiŋa \textbf{koia} \textbf{ko} taŋi.\\
\textsc{ntr} return hither \textsc{prop} Kainga \textsc{com} \textsc{prom} cry\\

\glt
‘Kainga returned crying.’ \textstyleExampleref{[R243.173]} 
\z

Alternatively, \textit{mā}\is{ma ‘with’@mā ‘with’} ‘and, with’ may be used, followed by a nominalised verb.\footnote{\label{fn:538}\textit{Mā} has a limited distribution in Rapa Nui: it is only used in the construction under discussion and in numerals. Both uses are also found in (and were probably borrowed from) \ili{Tahitian} (see Footnote \ref{fn:167} on p.~\pageref{fn:167}).}\is{Verb!nominalised} As \REF{ex:11.273} shows, \textit{mā te} may be assimilated to \textit{mata}.

\ea\label{ex:11.272}
\gll E noho nō {\ꞌ}ā \textbf{mā} \textbf{te} aŋa kore, \textbf{mā} \textbf{te} hupehupe. \\
\textsc{ipfv} stay just \textsc{cont} with \textsc{art} do lack with \textsc{art} lazy \\

\glt 
‘She lived doing nothing, being lazy.’ \textstyleExampleref{[R368.016]} 
\z

\ea\label{ex:11.273}
\gll He {\ꞌ}a{\ꞌ}amu, \textbf{mata} ta{\ꞌ}e {\ꞌ}ite hia pē nei ē: he tahutahu. \\
\textsc{ntr} tell with\_the \textsc{conneg} know yet like \textsc{prox} thus \textsc{ntr} witch \\

\glt 
‘She told (the other woman), without knowing that she was a witch.’ \textstyleExampleref{[R532-07.044]}
\z

When circumstances are states rather than events, they tend to be expressed in a clause in the perfect aspect (\textit{ko V {\ꞌ}ā}), without a special marker. 

\ea\label{ex:11.274}
\gll He taŋi \textbf{ko} \textbf{{\ꞌ}ū} \textbf{{\ꞌ}ā} era pē he pua{\ꞌ}a. \\
\textsc{ntr} cry \textsc{prf} bellow \textsc{cont} \textsc{dist} like \textsc{pred} cow \\

\glt 
‘He cried, howling like a cow.’ \textstyleExampleref{[R210.016]} 
\z

\ea\label{ex:11.275}
\gll He raŋi mai \textbf{ko} \textbf{riri} \textbf{rivariva} \textbf{{\ꞌ}ā}... \\
\textsc{ntr} call hither \textsc{prf} angry good:\textsc{red} \textsc{cont} \\

\glt
‘Very angry, she shouted...’ \textstyleExampleref{[R245.214]} 
\z

I have not found this construction in older texts, so it may be a modern development.

Perfect aspect\is{Aspect!perfect} clauses expressing circumstances are especially common in the construction \textit{ko V {\ꞌ}ā}\is{ko V {\ꞌ}ā (perfect aspect)} \textit{e V era}\is{e (imperfective)!e V PVD}. In this construction, the second clause is marked with \textit{e V era}\is{e (imperfective)!e V PVD} and expresses an action, while the preceding \textit{ko V {\ꞌ}ā} clause expresses a quality (e.g. a feeling or attitude) possessed by the subject performing the action. Even though \textit{e V era}\is{e (imperfective)!e V PVD} in general expresses durative\is{Aspect!durative} actions, in this construction it is not necessarily durative\is{Aspect!durative}. 

\ea\label{ex:11.276}
\gll \textbf{Ko} \textbf{riri} \textbf{{\ꞌ}ā} e kī era ki a nua... \\
\textsc{prf} angry \textsc{cont} \textsc{ipfv} say \textsc{dist} to \textsc{prop} Mum \\

\glt 
‘Angrily she said to Mum...’ \textstyleExampleref{[R210.062]} 
\z

\ea\label{ex:11.277}
\gll He māroa ki ruŋa, \textbf{ko} \textbf{nene} \textbf{{\ꞌ}ā} e u{\ꞌ}i era pe tū haŋa era. \\
\textsc{ntr} stand to above \textsc{prf} tremble \textsc{cont} \textsc{ipfv} look \textsc{dist} toward \textsc{dem} bay \textsc{dist} \\

\glt
‘He stood up and looked trembling towards that bay.’ \textstyleExampleref{[R408.128]} 
\z

Notice that \textit{e V era}\is{e (imperfective)!e V PVD} is obligatory when the circumstantial \textit{ko V {\ꞌ}ā}\is{ko V {\ꞌ}ā (perfect aspect)} clause comes first; when the circumstantial clause follows the main clause, the main clause may be \textit{he}{}-marked, as in (\ref{ex:11.274}–\ref{ex:11.275}) above.
\is{Clause!circumstantial|)}

\subsection{Summary}\label{sec:11.6.9}

Events which modify the event in the main clause, can be expressed in several ways. Certain interclausal relationships are expressed using a subordinating marker or conjunction. In other cases no special marker is used; even so, the modifying clause is subordinate, as is shown by the fact that these clauses are negated by the subordinate negator \textit{ta{\ꞌ}e} rather than a main clause negator. The various strategies are summarised in \tabref{tab:68}.

\begin{table}[t]
\resizebox{\textwidth}{!}{
\small
\begin{tabular}{p{15mm}p{10mm}p{14mm}p{17mm}p{15mm}p{16mm}p{14mm}}
\lsptoprule
& § & {subord. marker}& {conjunction}& no subord.\newline marking& {nominal}& {adverbial connector}\\
\midrule
temporal & \ref{sec:11.6.2.1}–\ref{sec:11.6.2.3} &  & \textit{hora}\newline ‘time’& A/M V PVD&  & \\
\tablevspace
‘before’ & \ref{sec:11.6.2.4} &  & \textit{{\ꞌ}ō ira}\newline ‘before’&  & \textit{{\ꞌ}i ra{\ꞌ}e}\newline ‘first’;\newline  \textit{ante}\newline ‘before’& \\
\tablevspace
‘until’ & \ref{sec:11.6.2.5} & \textit{ki} ‘to’& \textit{{\ꞌ}ātā}\newline ‘until’;\newline \textit{{\ꞌ}ahara}\newline ‘until’& \textit{ka V rō}&  & \\
\tablevspace
purpose & \ref{sec:11.6.3} & \textit{mo} ‘for’;\newline \textit{ki} ‘to’&  & bare verb&  & \\
\tablevspace
reason & \ref{sec:11.6.4} &  & \textit{{\ꞌ}i te me{\ꞌ}e}\newline ‘because’&  & \textit{{\ꞌ}i} ‘at’;\textit{{\ꞌ}o}\newline ‘because of’;\newline \textit{he} ‘\textsc{pred}’& \\
\tablevspace
result & \ref{sec:11.6.5} &  &  &  &  & \textit{{\ꞌ}o ira}\newline ‘therefore’\\
\tablevspace
condition & \ref{sec:11.6.6} & \textit{mo} ‘if’;\newline \textit{ana} ‘\textsc{irr}’&  & A/M V PVD&  & \\
\tablevspace
irreal condition & \ref{sec:11.6.6} &  & \textit{{\ꞌ}āhani}\newline ‘if~only’&  &  & \\
\tablevspace
concession & \ref{sec:11.6.7} &  &  & \textit{ka V atu}& \textit{nōatu}\newline ‘no matter’& \textit{te me{\ꞌ}e nō}\newline ‘however’\\
\lspbottomrule
\end{tabular}
}
\caption{Overview of adverbial clauses}
% \todo[inline]{top-align}
\label{tab:68}
\end{table}
\is{Clause!adverbial|)}
\section{Conclusions}\label{sec:11.7}

This chapter has explored the ways in which clauses are combined. A common way to combine clauses is simple juxtaposition. In fact, older Rapa Nui did not have any coordinating conjunction. In modern Rapa Nui \textit{{\ꞌ}e} ‘and’ is used, but juxtaposition is still the default strategy for coordinating clauses. Juxtaposition is not only used to express sequential events, but also to express semantic complements of the verbs \textit{ha{\ꞌ}amata} ‘begin’ and \textit{hōrou} ‘hurry’. 

Rapa Nui has various strategies to express the argument of a matrix verb. Only some of these involve a proper complement clause, i.e. a clause which is syntactically dependent on the main verb; they may involve the subordinating marker \textit{mo} ‘for, in order to’, or a nominalised complement. Other verbs are followed by a juxtaposed clause or an independent clause.

The subordinator \textit{mo} marks both complement clauses and adverbial clauses; interestingly, it marks both purpose and condition. The marker \textit{ana} has an even wider range of functions, all of which can be characterised as irrealis: an \textit{ana-}marked clause refrains from claiming the truth of the proposition expressed. \textit{Ana}{}-marked clauses express intentions, potential events and obligations, but also general truths. In subordinate clauses, \textit{ana} marks conditional clauses and dependent questions.

Relative clauses in Rapa Nui are not marked by a conjunction or preverbal marker, but they have various distinctive properties: they are invariably verb-initial and the choice of aspectuals is limited. A peculiar feature is, that the aspect marker may be left out (in most other clause types, unmarked verbs are rare or nonexistent). In these “bare relative clauses”, the verb is often raised to a position immediately after the head noun, before any postnominal markers.

A wide range of constituents can be relativised; most of these are not expressed in the relative clause, others are expressed as a pronoun. The distribution of these two constructions does not entirely conform to the noun phrase hierarchy proposed by \citet{KeenanComrie1977}: while subjects, objects and adjuncts are left unexpressed, oblique arguments (which are higher in the hierarchy than adjuncts) are expressed pronominally, just like constituents low in the hierarchy like possessors.

There is a tendency to express the entity which is subject of the relative clause as a possessor before or after the head noun: ‘your thing [did yesterday]’ = ‘the thing you did yesterday’. Syntactically there is nothing special about these constructions: the possessor is no different from other possessors in the noun phrase; the relative clause is no different from other relative clauses, apart from the fact that the subject is not expressed.

