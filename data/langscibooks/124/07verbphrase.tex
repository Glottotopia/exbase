\chapter[The verb phrase]{The verb phrase}\label{ch:7}
\section{The structure of the verb phrase}\label{sec:7.1}
\is{Verb phrase|(}
In Rapa Nui, the verb phrase consists of a verb, usually preceded by a preverbal marker, and often followed by one or more particles which contribute aspectual, spatial or other nuances. 

The structure of the verb phrase is shown in \tabref{tab:44} and \tabref{tab:45}.\footnote{\label{fn:308}Adapted and expanded from \citet[26]{WeberR2003}.}

\begin{table}
%\footnotesize{
\fittable{
\begin{tabularx}{136mm}{p{2mm}p{56mm}p{14mm}p{13mm}p{11mm}p{14mm}} 
\lsptoprule
& \centering{1}\arraybackslash& 2& 3& 4 & 5\\
& \centering{A/M}\arraybackslash& constit. negator& habitual\is{Aspect!habitual}& degree& causative\is{Causative}\\
\midrule
& aspect/mood: \newline \textit{he, i, e, ka, ku/ko} (\sectref{sec:7.2}); \newline 
subordinators/modality: \newline \textit{mo, ki, ana}, \textit{{\ꞌ}o, mai} (\sectref{sec:11.5}); \newline 
clausal~negators\is{Negation}: \textit{kai, (e) ko} (\sectref{sec:10.5}) & \textit{ta{\ꞌ}e}\is{tae (negator)@ta{\ꞌ}e (negator)}& \textit{rava}\is{rava ‘usually’}\textit{/ vara}& \textit{{\ꞌ}ata}\is{ata ‘more’@{\ꞌ}ata ‘more’}; \textit{{\ꞌ}apa}\is{apa ‘part’@{\ꞌ}apa ‘part’}& \textit{haka}\\
\midrule
§ &  & \ref{sec:10.5.6}& \ref{sec:7.3.1}& \ref{sec:7.3.2}& \ref{sec:8.12}\\
\lspbottomrule
\end{tabularx}
}
\caption{The verb phrase: preverbal elements}
\label{tab:44}
\end{table}

\begin{table}
%\footnotesize{
\fittable{
\begin{tabularx}{135mm}{p{2mm}p{13mm}p{18mm}p{9mm}p{16mm}p{17mm}p{17mm}p{12mm}} 
\lsptoprule
& 6& 7& \textit{\textup{8}}& 9& 10& 11& 12\\
& nucleus & adverb & 'yet' & evaluative & directional & postverbal dem. & final \\
\midrule
& verb& \textit{tahi, iho,} \newline \textit{tako{\ꞌ}a} etc. & \textit{hia}\is{hia ‘not yet’}& \textit{rō}\is{ro (emphatic marker)@rō (emphatic marker)}; \textit{nō}\is{no ‘just’@nō ‘just’}& \textit{mai}; \textit{atu}& \textit{nei}\is{nei (proximal)!postverbal}; \textit{ena}\is{ena (medial distance)!postverbal}; \newline \textit{era}\is{era (distal)!postverbal}; \textit{ai}\is{ai (postverbal)}& \textit{{\ꞌ}ā/{\ꞌ}ana}\is{a (postverbal)@{\ꞌ}ā (postverbal)}; \textit{{\ꞌ}ai}\\
\midrule
§ &  & \ref{sec:4.5.1}& \ref{sec:10.5.8}& \ref{sec:7.4}& \ref{sec:7.5}& \ref{sec:7.6}& \ref{sec:7.2.5.5}; \ref{sec:7.2.3.3}\\
\lspbottomrule
\end{tabularx}
}
\caption{The verb phrase: postverbal elements}
\label{tab:45}
\end{table}
% \todo[inline]{Somehow the em-dashes are not quite aligned: the first line in each table row starts slightly further to the right than the following ones.}
The preverbal constituents 2–5 may occur in different orders, depending on their relative scope. For examples, see (\ref{ex:7.96}–\ref{ex:7.99}) on p.~\pageref{ex:7.99}. 

In this chapter, the various elements occurring in the verb phrase are discussed. The largest section (\sectref{sec:7.2}) is devoted to the set of five aspect markers. Aspect markers can largely be described in terms of well-known categories such as perfectivity and imperfectivity; one marker, however (\textit{ka}) is more elusive. 

Another major topic is directional marking (\sectref{sec:7.5}). Two directional markers are used in various ways to orient events with respect to a deictic centre, pointing either towards or away from this deictic centre.

Shorter sections deal with preverbal particles (\sectref{sec:7.3}), evaluative markers (\sectref{sec:7.4}) and postverbal demonstratives (\sectref{sec:7.6}). Finally, \sectref{sec:7.7} deals with serial verb\is{Serial verb} constructions, a construction in which two or more verbs occur in a single verb phrase.

As the tables show, the first slot (labelled A/M, aspect/mood) contains not only aspect markers but a variety of other particles as well. Two aspect markers\is{Aspect marker} also mark imperative\is{Imperative} mood; this is discussed in \sectref{sec:10.2}. The preverbal slot is also home to a group of subordinating particles; these are discussed in \sectref{sec:11.5}. Two negators occur in the same position; these are discussed in \sectref{sec:10.5}. 

Finally, verb phrase adverbs are discussed with other minor parts of speech in \sectref{sec:4.5.1}. The particle \textit{hia} ‘yet’, which occurs in combination with negators\is{Negation}, is discussed in \sectref{sec:10.5.8}.

\section{Aspect marking}\label{sec:7.2}
\is{Aspect marker|(}\subsection{Introduction}\label{sec:7.2.1}

As the chart in the previous paragraph shows, the first slot in the verb phrase may be occupied by particles of various nature: aspect markers\is{Aspect marker}, subordinators and negators\is{Negation}. This means that a verb is either marked for aspect, introduced by a subordinator, or negated by \textit{kai} or \textit{(e) ko}. Combinations of these are impossible. This means, for example, that purpose clauses\is{Clause!purpose} introduced by \textit{mo} and clauses negated with \textit{kai} are not marked for aspect.\footnote{\label{fn:309}Neither is aspect marked when the verb is nominalised (\sectref{sec:3.2.3}).} 

In this section, the use of the aspectual markers is discussed.\footnote{\label{fn:310}This section is largely based on the analysis of all clauses in a subcorpus of 29 texts: 15 old texts, containing 2597 clauses; 14 new texts, containing 5834 clauses.}  This discussion will make clear that all markers have indeed an aspectual value and do not mark tense. In other words, they do not specify how the event is located in time, whether it happens before, at, or after the time of utterance. Rather, they are concerned with the internal temporal structure of the event and how the event is temporally related to other events in the context. The aspect markers\is{Aspect marker} are listed in \tabref{tab:46}.

\begin{table}
\begin{tabularx}{.5\textwidth}{p{20mm}p{20mm}p{10mm}}
\lsptoprule

{\textit{he}} & neutral & \sectref{sec:7.2.3}\\
{\textit{i}} & perfective & \sectref{sec:7.2.4}\\
{\textit{e}} & imperfective & \sectref{sec:7.2.5}\\
{\textit{ka}} & contiguity & \sectref{sec:7.2.6}\\
{\textit{ku/ko V {\ꞌ}ā}\is{ko V {\ꞌ}ā (perfect aspect)}} & perfect & \sectref{sec:7.2.7}\\
\lspbottomrule
\end{tabularx}
\caption{Overview of aspect markers}
\label{tab:46}
\end{table}

Certain aspectual functions are expressed by a combination of an aspectual marker\is{Aspect marker} and one or two postverbal particles; these particle combinations will be discussed as a whole. 

The discussion in this section is largely restricted to main clauses. The use of aspectuals with subordinate clauses (complement clauses, relative clauses\is{Clause!relative} and adverbial clauses) is discussed in Chapter 11. As certain subordinate clause types are strongly linked to – and highly illustrative of – certain aspectuals, reference will be made to Chapter 11 where appropriate. 

\subsection{The obligatoriness of aspectuals}\label{sec:7.2.2}

In most contexts, the use of aspectuals is obligatory. Verbs without aspectuals occur in the following situations:

\begin{itemize}
\item 
Verbs which are incorporated, i.e. part of a compound\is{Compound} noun (\sectref{sec:5.7.2.3}).

\item 
{\sloppy
Bare relative clauses\is{Clause!relative!bare} (\sectref{sec:11.4.5}); this includes the nominal purpose construction\\ (\sectref{sec:11.6.3}).
}

\item 
Bare purpose clauses (\sectref{sec:11.6.3}).

\item 
Occasionally in imperatives\is{Imperative} (\sectref{sec:10.2.1}).

\end{itemize}

Otherwise, aspectuals are occasionally omitted clause-initially (and especially sen\-tence-initially) in an informal style, if the verb is followed by one or more postverbal particles. As the following examples show, different aspectuals can be omitted. In \REF{ex:7.1}, the perfect aspect marker \textit{ko} is implied; the usual perfect aspect construction is \textit{ko V {\ꞌ}ā} (\sectref{sec:7.2.7}). In \REF{ex:7.2}, neutral \textit{he} is implied: the verb is followed by \textit{rō {\ꞌ}ai}, which points towards the construction \textit{he V rō {\ꞌ}ai}\is{he (aspect marker)!he V rō {\ꞌ}ai} (\sectref{sec:7.2.3.3}); the second clause in \REF{ex:7.2} shows the same construction in full, with aspect marker. 

\ea\label{ex:7.1}
\gll Pae tahi \textbf{{\ꞌ}ā} te taŋata mo māta{\ꞌ}ita{\ꞌ}i. \\
go\_all all \textsc{cont} \textsc{art} person for observe \\

\glt 
‘All the people went to watch.’ \textstyleExampleref{[R415.785]} 
\z

\ea\label{ex:7.2}
\gll Noho \textbf{rō} \textbf{{\ꞌ}ai} te tau{\ꞌ}a, he rakerake rō {\ꞌ}ai ararua {\ꞌ}aro. \\
stay \textsc{emph} \textsc{subs} \textsc{art} battle \textsc{ntr} bad:\textsc{red} \textsc{emph} \textsc{subs} the\_two side \\

\glt 
‘The battle went on, it got bad on both sides.’ \textstyleExampleref{[R104.074]} 
\z

\subsection{Neutral \textit{he}}\label{sec:7.2.3}
\is{he (aspect marker)|(}\subsubsection{Introduction}\label{sec:7.2.3.1}

\textit{He} is the most common aspect marker\is{Aspect marker}. It probably developed from the nominal predicate marker \textit{he}\is{he (nominal predicate marker)} (\sectref{sec:5.3.4}). This development took place only in Rapa Nui – no other Polynesian language has an aspect marker\is{Aspect marker} cognate to \textit{he} – so we may tentatively conclude that it took place after the language split off from PEP.\footnote{\label{fn:311}Interestingly, \citet[57]{Cook1999} gives an example of \textit{he} in \ili{Hawaiian} preceding a verb.} 

While it may go too far to consider nominal and verbal \textit{he} as one particle synchronically, the two are very similar in function. The nominal predicate marker \textit{he} marks noun phrases as predicates, without attributing any aspectual value to them. Aspect concerns the internal temporal structure of an event; as entities (expressed in a noun phrase) do not have an internal temporal structure, they cannot be marked for any specific aspect. In the same way, the aspectual \textit{he} is the least specific of all aspect markers\is{Aspect marker}.\footnote{\label{fn:312}Another phenomenon linking the predicate marker \textit{he} and the aspectual \textit{he}, is that the negation \textit{{\ꞌ}ina}\is{ina (negator)@{\ꞌ}ina (negator)} is either followed by \textit{he} + noun (never by the article \textit{te}), or by \textit{he} + verb (never by a different aspectual). See section \sectref{sec:10.5.1}.} \citet[64]{Englert1978} calls it “the most general, most used and least precise tense” (my trl.). \citet[153]{Chapin1978} labels it as a “neutral marker”, a term I adopt in this grammar (gloss \textsc{ntr}). The range of use of \textit{he} will be discussed in the next section; the examples will make clear that \textit{he} is used in a wide variety of clauses; these clauses may be punctual, durative\is{Aspect!durative} or habitual\is{Aspect!habitual}; they may convey events in a narrative, future events or instructions. This confirms the idea that \textit{he} itself expresses none of these functions, but is a neutral marker. The aspectual value of the clause is not expressed by \textit{he} as such, but can be deduced from the nature of the verb and/or the context. In other words, \textit{he} is functionally unmarked.\footnote{\label{fn:313}\citet{Chapin1978} suggests a different unified account for nominal and verbal \textit{he}: the “noun” after \textit{he} could be a verb, i.e. in \textit{he taŋata koe} ‘you are a man’, \textit{taŋata} could be analysed as a verb, an analysis also proposed (though in a more cautious wording) by \citet[22]{FinneyAlexander1998}. This analysis is syntactically implausible, however, as \textit{he} is followed by a true noun phrase. As the examples in section \sectref{sec:5.3.4} show, the noun following \textit{he} may be preceded and followed by noun phrase elements like adjectives and numerals, while verb-phrase particles like \textit{rō}, \textit{atu} and \textit{{\ꞌ}ai} are excluded.}  

In many cases, a \textit{he}{}-marked clause depends on other clauses in the context for its aspectual value. In narrative, a perfective clause may set the scene, after which a series of \textit{he}{}-marked clauses follow (see \REF{ex:7.4} below). Another example: \textit{he} may mark a series of instructions, but only when the first of these is explicitly marked as imperative\is{Imperative} (see \REF{ex:7.5} below).

\textit{He} is rare in subordinate clauses, which may also be due to its neutral character. Subordinate clauses typically stand in some temporal or aspectual relation to their main clause, whether simultaneous, overlapping, contiguous, anterior or posterior. \textit{He} is not able to supply this temporal link, hence it is not suitable in these contexts.

\subsubsection[Range of use]{Range of use}\label{sec:7.2.3.2}
\largerpage
As indicated above, \textit{he} does not express any specific aspect; rather, it depends on the context for its aspectual value. In this section, this will be illustrated through examples of different contexts in which \textit{he} is used.

\paragraph{}\label{sec:7.2.3.2.1} In narrative, \textit{he}{}-marked clauses express the \textsc{theme line} of a story. Strings of \textit{he-}clauses constitute the “back bone” of a story, describing the sequence of narrative events.\footnote{\label{fn:314}The following examples are translated more literally than usual in this grammar, to convey the idea of the concatenation of \textit{he}{}-clauses.} The following is a typical example:

\ea\label{ex:7.3}
\gll \textbf{He} e{\ꞌ}a mai a nua \textbf{he} haka rivariva \textbf{he} oti \textbf{he} e{\ꞌ}a  \textbf{he} turu ararua ko Eva \textbf{he} tu{\ꞌ}u ki Haŋa Piko.\\
\textsc{ntr} go\_out hither \textsc{prop} Mum \textsc{ntr} \textsc{caus} good:\textsc{red} \textsc{ntr} finish \textsc{ntr} go\_out  \textsc{ntr} go\_down the\_two \textsc{prom} Eva \textsc{ntr} arrive to Hanga Piko\\

\glt
‘Mum went out, she made preparations, she finished, she went out, she went down with Eva, they arrived in Hanga Piko.’ \textstyleExampleref{[R210.038]} 
\z

Other aspectuals may interrupt the stream of \textit{he-}clauses to indicate non-theme line elements of the narrative; they serve for example to provide background information, to express events anterior\is{Anteriority} to the theme line, and to mark events which are highlighted in some way. (See the discussion on perfective \textit{i} in \sectref{sec:7.2.4.2} below.)\footnote{\label{fn:315}The interplay of \textit{he} and other aspectuals and their respective functions in narrative have been analysed by \citet{WeberR2003}.} 

The string of \textit{he-}clauses providing the theme line of the story is usually preceded by one or more clauses which establish a time frame in which the events take place. The following example is the beginning of a story. The story starts with a cohesive clause, a temporal clause\is{Clause!temporal} providing a time frame for what follows, marked with perfective \textit{i} (\sectref{sec:11.6.2.1}\is{Clause!cohesive}). After that, the story continues with \textit{he}{}-marked clauses.

\ea\label{ex:7.4}
\gll \textbf{I} poreko \textbf{era} a Puakiva, \textbf{he} māuiui a Kuha, tō{\ꞌ}ona matu{\ꞌ}a vahine. \\
\textsc{pfv} born \textsc{dist} \textsc{prop} Puakiva \textsc{ntr} sick \textsc{prop} Kuha \textsc{poss.3sg.o} parent female \\

\glt
‘When Puakiva was born, his mother Kuha got sick.’ \textstyleExampleref{[R229.001]} 
\z

The fact that the time reference is established beforehand, confirms the idea that \textit{he} is a neutral aspect marker\is{Aspect marker}: \textit{he} has no temporal or aspectual value of its own, but continues within a previously established framework.\footnote{\label{fn:316}Notice that \textit{he} itself does not explicitly indicate either that the events happen sequentially; there is no ‘and then’ sense implied in \textit{he} as such. This feature is understood in the narrative context (cf. \citealt[127]{Hooper1998} on zero-marked narrative events in \ili{Tokelauan}).}

In other text types, theme-line clauses are also marked with \textit{he}. For example, in procedural texts the theme line consists of a series of steps which are taken to perform a certain procedure: building a boat, performing a burial, making a traditional cape. In the following example, the speaker describes how to prepare a certain medicine. The first step of the procedure is indicated by the imperative\is{Imperative} \textit{e}, conveying a general instruction; this is followed by a series of \textit{he}{}-marked verbs.

\ea\label{ex:7.5}
\gll \textbf{E} \textbf{haka} \textbf{piha{\ꞌ}a} i te vai. Ki oti \textbf{he} \textbf{to{\ꞌ}o} mai \textbf{he} \textbf{huri}  \textbf{he} \textbf{haka} \textbf{tano} te matu{\ꞌ}a~pua{\ꞌ}a ki roto o te vai piha{\ꞌ}a.\\
\textsc{exh} \textsc{caus} boil \textsc{acc} \textsc{art} water when finish \textsc{ntr} take hither \textsc{ntr} turn  \textsc{ntr} \textsc{caus} correct \textsc{art} \textit{matu{\ꞌ}a pua{\ꞌ}a} to inside of \textsc{art} water boil\\

\glt 
‘Boil water. When done, take it and pour the right amount of \textit{matu{\ꞌ}a pua{\ꞌ}a} (a medicinal plant) into the boiling water.’ \textstyleExampleref{[R313.160–161]}
\z

\paragraph{}\label{sec:7.2.3.2.2} \textit{He}{}-marked clauses may refer to durative\is{Aspect!durative} or habitual\is{Aspect!habitual} actions as in \REF{ex:7.6},\footnote{\label{fn:317}To mark durativity or habituality explicitly, \textit{e V era}\is{e (imperfective)!e V PVD} or \textit{e V {\ꞌ}ā/{\ꞌ}ana} is used (\sectref{sec:7.2.5.4}).} or general truths as in \REF{ex:7.7}.

\ea\label{ex:7.6}
\gll Paurō te mahana \textbf{he} turu au ki te hāpī. \\
every \textsc{art} day \textsc{ntr} go\_down \textsc{1sg} to \textsc{art} learn \\

\glt 
‘Every day I go to school.’ \textstyleExampleref{[R151.059]} 
\z

\ea\label{ex:7.7}
\gll \textbf{He} hīmene te perete{\ꞌ}i. \textbf{He} kirukiru te manu. \textbf{He} {\ꞌ}ūmō te pua{\ꞌ}a. \\
\textsc{ntr} sing \textsc{art} cricket \textsc{ntr} chirp \textsc{art} bird \textsc{ntr} moo \textsc{art} cow \\

\glt
‘Crickets sing. Birds chirp. Cows bellow.’ \textstyleExampleref{[Notes]}
\z

\paragraph{}\label{sec:7.2.3.2.3} \textit{He} is used with stative verbs\is{Verb!stative} (including adjectives) to express a state of affairs which holds at the time of reference. This may be the time of speech as in \REF{ex:7.8}, or the time of the narrative as in \REF{ex:7.9}.

\ea\label{ex:7.8}
\gll \textbf{He} \textbf{nene} nō ta{\ꞌ}a ika mata, e nua ē. \\
\textsc{ntr} sweet just \textsc{poss.2sg.a} fish raw \textsc{voc} Mum \textsc{voc} \\

\glt 
‘Your raw fish is really nice, Mum.’ \textstyleExampleref{[R535.095]} 
\z

\ea\label{ex:7.9}
\gll He topa te poki tamahahine... He hāŋai i a Uho, \textbf{he} \textbf{nuinui}. \\
\textsc{ntr} descend \textsc{art} child female \textsc{ntr} feed \textsc{acc} \textsc{prop} Uho \textsc{ntr} big:\textsc{red} \\

\glt 
‘A girl was born... They raised Uho and she grew up.’ \textstyleExampleref{[Ley-9-55.026–027]}
\z

\paragraph{}\label{sec:7.2.3.2.4} \textit{He}-marked clauses may express events that are about to happen or foreseen in the (near) future\is{Future}. The time frame is established in the context (‘next year’ in \REF{ex:7.10}).

\ea\label{ex:7.10}
\gll Matahiti ena \textbf{he} \textbf{hoki} a au ki te hāpī. \\
year \textsc{med} \textsc{ntr} return \textsc{prop} \textsc{1sg} to \textsc{art} learn \\

\glt
‘Next year I will return to school.’ \textstyleExampleref{[R210.003]} 
\z

To express the future character of the action explicitly, \textit{e V rō}\is{e (imperfective)!e V rō} is used (\sectref{sec:7.2.5.3}). Clauses expressing plans or intentions may also be marked with \textit{ka} (\sectref{sec:7.2.6.3}).

\subsubsection{\textit{He} and postverbal particles}\label{sec:7.2.3.3}

\paragraph{Demonstratives}\label{sec:7.2.3.3.1} Unlike the aspectuals \textit{i}, \textit{e} and \textit{ka}, \textit{he} is rarely followed by one of the postverbal demonstratives\is{Demonstrative!postverbal} \textit{nei}, \textit{ena} or \textit{era} (\sectref{sec:7.6}).\footnote{\label{fn:318}\textit{He V era} occurs relatively often in the stories recorded by Métraux\ia{Métraux, Alfred}; however, this probably represents the much more common construction \textit{e V era}\is{e (imperfective)!e V PVD}: Métraux, whose first language was \ili{French}, sometimes took initial glottals\is{Glottal plosive} for \textit{h} and vice versa.} Occasional examples are found:

\ea\label{ex:7.11}
\gll \textbf{He} \textbf{moe} \textbf{era} koe e Hina ē {\ꞌ}i te kata. \\
\textsc{ntr} lie \textsc{dist} \textsc{2sg} \textsc{voc} Hina \textsc{voc} at \textsc{art} laugh \\

\glt 
‘Hina laughed her head off (lit. you, Hina, lied down laughing).’ \textstyleExampleref{[R313.025]} 
\z

\paragraph{\textit{He V rō {\ꞌ}ai}}\label{sec:7.2.3.3.2} More common is the construction \textit{he V rō {\ꞌ}ai}\is{ai (postverbal)@{\ꞌ}ai (postverbal)}: a \textit{he}{}-marked verb followed by the asseverative particle \textit{rō} (\sectref{sec:7.4.2}) and the sequential particle \textit{{\ꞌ}ai}. \citet[125]{WeberR2003} shows that \textit{he V rō {\ꞌ}ai} in narrative texts indicates notable, important events on the theme line: significant developments or culminating points in the story. My analysis of several narrative texts confirms this. \textit{he V rō {\ꞌ}ai} clauses indicate events which are either climactic in a story, final in a sequence, or both.

In other cases, \textit{he V rō {\ꞌ}ai}\is{he (aspect marker)!he V rō {\ꞌ}ai} marks an event which is not only final in a series, but which constitutes a climax in the story. In the following example, a sequence of events is concluded with \textit{he V rō {\ꞌ}ai}\is{he (aspect marker)!he V rō {\ꞌ}ai}: the woman tries to catch her child, which has turned into a fish, but in vain: the child disappears. The last event, the climax of the sequence, is marked with \textit{rō {\ꞌ}ai}.

\ea\label{ex:7.12}
\gll He tute he oho e te vi{\ꞌ}e nei... {\ꞌ}e \textbf{he} \textbf{ŋaro} \textbf{rō} \textbf{atu} \textbf{{\ꞌ}ai}.\\
\textsc{ntr} chase \textsc{ntr} go \textsc{ag} \textsc{art} woman \textsc{prox} and \textsc{ntr} disapeaar \textsc{emph} away \textsc{subs}\\

\glt
‘The woman chased the fish.... but it disappeared.’ \textstyleExampleref{[R338.009]} 
\z

The use of \textit{rō} in this construction conforms to the general sense of \textit{rō}, asserting the reality of the event (\sectref{sec:7.4.2}).

\textit{He V rō {\ꞌ}ai}\is{he (aspect marker)!he V rō {\ꞌ}ai} is also used at points of emotional intensity; in the following example (from the same story as \REF{ex:7.12}), the mother is grieved because her child has disappeared.

\ea\label{ex:7.13}
\gll Te matu{\ꞌ}a vahine o te poki nei \textbf{he} \textbf{taŋi} \textbf{rō} \textbf{atu} \textbf{{\ꞌ}ai}.\\
\textsc{art} parent female of \textsc{art} child \textsc{prox} \textsc{ntr} cry \textsc{emph} away \textsc{subs}\\

\glt 
‘The mother of the child cried.’ \textstyleExampleref{[R338.008]} 
\z

\subsubsection[Summary]{Summary}\label{sec:7.2.3.4}

The discussion above has shown that \textit{he} does not express one single aspect. It is used in punctual, durative, habitual and stative clauses; the verb may refer to a timeless truth, a narrative event or a future event. This wide range indicates that \textit{he} is a neutral aspect marker, which in itself does not express any aspect. The aspectual value of the clause is contributed by the context, for example a time phrase, a temporal clause or a preceding imperative.

\textit{He} is especially common in sequences of clauses expressing successive events; this happens both in narrative and procedural discourse.
\is{he (aspect marker)|)}
\subsection{Perfective \textit{i}}\label{sec:7.2.4}
\is{i (perfective)|(}\subsubsection[Introduction]{Introduction}\label{sec:7.2.4.1}

\textit{I} is the perfective marker.\footnote{\label{fn:319}Perfective \textit{i} is common in Eastern Polynesian languages; non-EP languages have \textit{na}, \textit{ne} or \textit{ni}. \citet[314]{Wilson2012} suggests a development PNP \textit{*ne} {\textgreater} Central Northern Outliers \textit{*ni} {\textgreater} \is{Eastern Polynesian}PEP *\textit{i}.} The perfective aspect presents an event as a single, unanalysable whole (\citealt[3]{Comrie1976}; \citealt[35]{Dixon2012}), without considering its internal structure (e.g. its duration). In other words, the perfective regards the event from the outside, while the imperfective considers its temporal make-up from the inside.

Perfective aspect is naturally correlated with past tense \citep[72]{Comrie1976}, and in fact, \textit{i} usually marks events in the past\is{Past}. \textit{I} has been characterised as a past tense marker in several descriptions of Rapa Nui and other Polynesian languages.\footnote{\label{fn:320}See \citet[156]{DuFeu1996} for Rapa Nui, \citet{MutuTeìkitutoua2002} for \ili{Marquesan}, \citet[172]{AcadémieTahitienne1986} for \ili{Tahitian}, \citet[34]{Biggs1973} for \ili{Māori}. Note that \citet[153]{Chapin1978} labels Rapa Nui \textit{i} as perfective.} In non-narrative contexts \textit{i} is the common aspectual for past events, as the following examples show. As \REF{ex:7.17} shows, it may also express general facts about the past.

\ea\label{ex:7.14}
\gll A au \textbf{i} \textbf{oho} \textbf{mai} \textbf{nei} ki a koe mo noho ō{\ꞌ}oku {\ꞌ}i nei. \\
\textsc{prop} \textsc{1sg} \textsc{pfv} go hither \textsc{prox} to \textsc{prop} \textsc{2sg} for stay \textsc{poss.1sg.o} at \textsc{prox} \\

\glt 
‘I have come to you to live here.’ \textstyleExampleref{[R245.072]} 
\z

\ea\label{ex:7.15}
\gll Ko koe \textbf{i} \textbf{rē}. \\
\textsc{prom} \textsc{2sg} \textsc{pfv} win \\

\glt 
‘You have won.’ \textstyleExampleref{[R210.071]} 
\z

\ea\label{ex:7.16}
\gll ¡E Nuahine Pīkea {\ꞌ}Uri ē, {\ꞌ}ā{\ꞌ}au rō ta{\ꞌ}a moeŋa nei o māua \textbf{i} \textbf{toke}! \\
~\textsc{voc} Nuahine Pikea Uri \textsc{voc} \textsc{poss.2sg.a} \textsc{emph} \textsc{poss.2sg.a} mat \textsc{prox} of \textsc{1du.excl} \textsc{pfv} steal \\

\glt 
‘Nuahine Pikea Uri, it was you who stole that mat of ours!’ \textstyleExampleref{[R310.428]} 
\z

\ea\label{ex:7.17}
\gll Te me{\ꞌ}e o te mātāmu{\ꞌ}a me{\ꞌ}e ta{\ꞌ}e vānaŋa, \textbf{i} \textbf{mou} \textbf{nō}. \\
\textsc{art} thing of \textsc{art} past thing \textsc{conneg} talk \textsc{pfv} quiet just \\

\glt
‘The people of old used not to speak, they kept silent.’ \textstyleExampleref{[R310.216]} 
\z

There are cases, however, where \textit{i} conveys a non-past event. For example, in \REF{ex:7.18} \textit{i} is used with reference to the future:

\ea\label{ex:7.18}
\gll \textbf{I} \textbf{o{\ꞌ}o} \textbf{era} koe ki roto i tu{\ꞌ}u hare era e noho koe. \\
\textsc{pfv} enter \textsc{dist} \textsc{2sg} to inside at \textsc{poss.2sg.o} house \textsc{dist} \textsc{ipfv} stay \textsc{2sg} \\

\glt
‘When you have entered into your house, stay there.’ \textstyleExampleref{[R310.297]} 
\z

Conversely, other aspectuals are used besides \textit{i} in clauses referring to past events: narrative \textit{he}\is{he (aspect marker)} (\sectref{sec:7.2.3}), imperfective \textit{e V {\ꞌ}ā}\is{e (imperfective)!e V {\ꞌ}ā} (\sectref{sec:7.2.5.4}). This means that \textit{i} is not a past tense marker; rather, it expresses that an action is temporally closed. This may in turn mean that the event is in the past, or anterior to other events, or finished at a certain point, but neither of these is a necessary condition for the use of \textit{i}.

\citet[17–18]{Comrie1976} stresses that perfective is not the same as punctual. This is true in Rapa Nui as well: while \textit{i} often marks punctual events, it is equally used to mark events that have a certain duration. This is clear in examples like the following, where the perfective is used for events that take place over many years:

\ea\label{ex:7.19}
\gll A Te Manu \textbf{i} \textbf{noho} \textbf{ai} {\ꞌ}i muri i tū māmātia era ō{\ꞌ}ona  {\ꞌ}ātā ki te nuinui iŋa.\\
\textsc{prop} Te Manu \textsc{pfv} stay \textsc{pvp} at near at \textsc{dem} grandmother \textsc{dist} \textsc{poss.3sg.o}  until to \textsc{art} big:\textsc{red} \textsc{nmlz}\\

\glt 
‘Te Manu stayed with his aunt until he had grown up.’ \textstyleExampleref{[R245.246]} 
\z

In main clauses, the \textit{i}{}-marked verb is usually followed by a postverbal demonstrative (PVD). The use of PVDs after \textit{i}{}-marked verbs will be discussed in more detail in \sectref{sec:7.6.5}. 

\subsubsection[Neutral he versus perfective i]{Neutral \textit{he} versus perfective \textit{i}}\label{sec:7.2.4.2}

As discussed in \sectref{sec:7.2.3.2} above, \textit{he} is used to mark the theme line of discourse. This means that the relation between \textit{he} and \textit{i} calls for an explanation. As \citet[293]{Timberlake2007} points out, the perfective is typically the aspect of narrative texts: a perfective event leads to a new state, which is the input for the next event; a string of such events constitutes a narrative. In Rapa Nui, however, sequential events in a narrative are marked with \textit{he}, not \textit{i} (\sectref{sec:7.2.3.2} above). 

As shown in this section, \textit{i} is used when the event is not sequential to the event in the preceding clause, for example in clauses providing background information. Moreover, \textit{i} is used to highlight events, setting them of from the theme line of \textit{he}{}-marked clauses. This means that \textit{i} is used in narrative discourse to mark clauses not belonging to the theme line for some reason. 

\subparagraph{Anteriority} \textit{I} is used when the event is anterior\is{Anteriority} with respect to the theme line of the story (i.e. where the \ili{English} equivalent is the pluperfect).

\ea\label{ex:7.20}
\gll Māuiui nei \textbf{i} \textbf{tu{\ꞌ}u} \textbf{mai} \textbf{ai} ki Rapa Nui o te nu{\ꞌ}u empereao  o te Compañía \textbf{i} \textbf{ma{\ꞌ}u} \textbf{mai}.\\
sick \textsc{prox} \textsc{pfv} arrive hither \textsc{pvp} to Rapa Nui of \textsc{art} people employee  of \textsc{art} company \textsc{pfv} carry hither\\

\glt
‘This disease had arrived on Rapa Nui, introduced by the employees of the Company.’ \textstyleExampleref{[R250.084]} 
\z

\textit{I} with anterior events is especially common in cohesive clauses\is{Clause!cohesive}, temporal clauses\is{Clause!temporal} preceding a main clause (\sectref{sec:11.6.2.1}).

\subparagraph{Highlighting} \textit{I} marks events which the speaker wants to highlight in the stream of \textit{he}{}-clauses. In the following example, Kalia, the protagonist of the story, has been swimming all night to get to Ao Tea Roa to get help for the people of Kapiti. The moment in which she finally arrives and is able to warn the people of Ao Tea Roa, is marked with \textit{i V ai}. As this example shows, the significance of the event may be underlined by the asseverative particle \textit{rō} (\sectref{sec:7.4.2}).

\ea\label{ex:7.21}
\gll Kai puhi rivariva ia te haŋu {\ꞌ}i te poto o te aho  \textbf{i} \textbf{ohu} \textbf{rō} \textbf{atu} \textbf{ai} mo haka {\ꞌ}ite i tū {\ꞌ}ati era.\\
\textsc{neg.pfv} blow good:\textsc{red} then \textsc{art} breath at \textsc{art} short of \textsc{art} breath  \textsc{pfv} shout \textsc{emph} away \textsc{pvp} for \textsc{caus} know \textsc{acc} \textsc{dem} problem \textsc{dist}\\

\glt 
‘Short of breath, she shouted to make the trouble known.’ \textstyleExampleref{[R347.128]} 
\z

\subparagraph{Intervening events} More specifically, \textit{i} is used when the clause expresses what may be called an intervening event. As \citet[3]{Comrie1976} indicates, the perfective sees the action as an unanalysable whole, without an internal temporal structure. Therefore, the perfective is used in many languages to express punctual events. In Rapa Nui, the perfective is often used with punctual events which take place while something else is happening. The punctual event interrupts another event which has been going on for some time: it intervenes into an existing situation.

This is common after the imperfective \textit{e V nō {\ꞌ}ā}\is{e (imperfective)!e V nō {\ꞌ}ana} (\sectref{sec:7.2.5.4}):

\ea\label{ex:7.22}
\gll E noho nō {\ꞌ}ā a Te Manu \textbf{i} \textbf{vari} \textbf{atu} \textbf{ai} a Nune... \\
\textsc{ipfv} sit just \textsc{cont} \textsc{prop} Te Manu \textsc{pfv} pass away \textsc{pvp} \textsc{prop} Nune \\

\glt 
‘When Te Manu was sitting, Nune came by...’ \textstyleExampleref{[R245.174]} 
\z

\ea\label{ex:7.23}
\gll E iri nō atu {\ꞌ}ā \textbf{i} \textbf{take{\ꞌ}a} \textbf{rō} \textbf{ai} e te vi{\ꞌ}e o tū pāpā era o Te Manu.\\
\textsc{ipfv} ascend just away \textsc{cont} \textsc{pfv} see \textsc{emph} \textsc{pvp} \textsc{ag} \textsc{art} woman of \textsc{dem} father \textsc{dist} of Te Manu\\

\glt 
‘When he was going up, the wife of Te Manu’s father saw him.’ \textstyleExampleref{[R245.214]} 
\z

\subparagraph{Background} \textit{I}{}-marked clauses may express background information. For example, in the introduction of a story, \textit{i}{}-clauses may serve to set the stage by telling what happened before the beginning of the story, as in \REF{ex:7.24}. \textit{I-}marked clauses may also express restatements or clarifications, as in \REF{ex:7.25}.\footnote{\label{fn:321}Similarly, \textit{i}{}-marked clauses may express background events in subordinate clauses (\sectref{sec:11.6.2.2}).}

\ea\label{ex:7.24}
\gll Te ara nei o te nu{\ꞌ}u nei, \textbf{i} \textbf{e{\ꞌ}a} \textbf{ai} mai Haŋa Roa o Tai  {\ꞌ}i ruŋa o te vaka nei.\\
\textsc{art} way \textsc{prox} of \textsc{art} people \textsc{prox} \textsc{pfv} go\_out \textsc{pvp} from Hanga Roa o Tai  at above of \textsc{art} boat \textsc{prox}\\

\glt 
‘As to these people’s trip, they had left Hanga Roa o Tai by boat.’ \textstyleExampleref{[R361.004]} 
\z

\ea\label{ex:7.25}
\gll ...he iri he oho ki te kona hare era. \textbf{I} \textbf{tahuti} a Tiare  \textbf{i} \textbf{iri} \textbf{ai} ki te kona hare era.\\
~~~\textsc{ntr} ascend \textsc{ntr} go to \textsc{art} place house \textsc{dist} \textsc{pfv} run \textsc{prop} Tiare  \textsc{pfv} ascend \textsc{pvp} to \textsc{art} place house \textsc{dist}\\

\glt 
‘...she went up to her home. Running, Tiare went up to her home.’ \textstyleExampleref{[R151.053]} 
\z

\subsubsection[Summary]{Summary}\label{sec:7.2.4.3}

\textit{I} is the perfective marker: it marks events which are viewed as a whole, without internal temporal structure. The event is usually, but not always, in the past. 

In narrative, \textit{i} is used for events which stand out in some way from the thematic backbone of events marked with \textit{he}: \textit{i} marks background events, restatements and conclusions, flashbacks, but also events which are highlighted.
\is{i (perfective)|)}
\subsection{Imperfective \textit{e}}\label{sec:7.2.5}
\is{e (imperfective)|(}\is{Aspect!imperfective}\subsubsection[Introduction]{Introduction}\label{sec:7.2.5.1}

\textit{E} is the imperfective marker. It is common throughout Polynesian languages; Pollex (see \citealt{GreenhillClark2011}) glosses it as ‘non-past’. According to \citet[24]{Comrie1976}, the imperfective makes “explicit reference to the internal temporal structure of a situation, viewing a situation from within” (see also \citealt[35]{Dixon2012}). Languages may grammaticalise certain subcategories of the imperfective; Comrie divides the imperfective into two subcategories: \textsc{continuous}\is{Aspect!continuous} (an event or situation goes on for some time) and \textsc{habitual}\is{Aspect!habitual} (“a situation which is characteristic of an extended period of time”, 27–28). The continuous can be further divided into progressive\is{Progressive} and nonprogressive: in various European languages, stative verbs\is{Verb!stative} may be used in the imperfective (with a continuous interpretation), but not in a progressive\is{Progressive} form. The \textsc{progressive}\is{Progressive} is thus a combination of a continuous meaning and non-stativity (35–36).\footnote{\label{fn:322}Others consider “continuous” and “progressive\is{Progressive}” as synonymous, see e.g. \citet[34]{Dixon2012}.} 

As it will turn out, the categories mentioned here are relevant in Rapa Nui as well. While \textit{e} as such expresses imperfectivity, finer distinctions are expressed by \textit{e} in combination with certain postverbal particles. Thus the aspectual value of the clause is defined not by \textit{e} alone, but by a combination of \textit{e} and postverbal particles. The following particles contribute to the aspect of the clause: the evaluative markers \textit{rō} and \textit{nō}, the continuity marker \textit{{\ꞌ}ā/{\ꞌ}ana},\footnote{\label{fn:323}There is no difference in function between \textit{{\ꞌ}ā} and \textit{{\ꞌ}ana}; \textit{{\ꞌ}ā} is more common (\sectref{sec:5.9}). In this section, \textit{{\ꞌ}ā} will be used as a shorthand for \textit{{\ꞌ}ā/{\ꞌ}ana}.}  and the postverbal demonstratives\is{Demonstrative!postverbal} (PVDs\is{Demonstrative!postverbal}) \textit{nei/ena/era}. With \textit{e}, these particles show the following cooccurrence restrictions:

\ea\label{ex:7.26a}
\gll
\textit{e} ~ \textup{V} ~ \textup{(adverb)}\is{Adverb} ~ \textup{(}rō/nō\textup{)} ~ \textup{(}\textit{mai/atu}\textup{)} ~ ~ ~  \textup{(}\textit{{\ꞌ}ā/{\ꞌ}ana}\textup{)}\\
\textit{e} ~ V ~ \textup{(adverb)}\is{Adverb} ~ ~ ~  (\textit{mai/atu}) ~ \textit{nei/ena/era}\\
\z
In other words, PVDs\is{Demonstrative!postverbal} after \textit{e} do not cooccur with either the evaluative markers \textit{rō} and \textit{nō} or the continuity marker \textit{{\ꞌ}ā/{\ꞌ}ana},\footnote{\label{fn:324}This restriction is specific to imperfective \textit{e}; after other aspectuals, postverbal demonstratives\is{Demonstrative} do co-occur with \textit{rō}, \textit{nō} and \textit{{\ꞌ}ā/{\ꞌ}ana}.} but the latter two categories do occur together.\footnote{\label{fn:325}In fact, these two categories co-occur far more often than one would statistically expect: while \textit{e V {\ꞌ}ā}\is{e (imperfective)!e V {\ꞌ}ā} occurs 35 times and \textit{e V nō}/\textit{rō} occurs 34 times, \textit{e V nō/rō {\ꞌ}ā} occurs no less than 153 times.} 

In the following sections, different constructions with \textit{e} will be discussed: bare \textit{e} (i.e. without any postverbal particle) is briefly discussed in \sectref{sec:7.2.5.2}, \textit{e V rō}\is{e (imperfective)!e V rō} in \sectref{sec:7.2.5.3}. \textit{E V era}\is{e (imperfective)!e V PVD} and \textit{e V {\ꞌ}ā}\is{e (imperfective)!e V {\ꞌ}ā} (which largely occur in the same contexts and have similar functions) are treated together in \sectref{sec:7.2.5.4}. Finally, in \sectref{sec:7.2.5.5}, the distinction between \textit{e V era}\is{e (imperfective)!e V PVD} and \textit{e V {\ꞌ}ā}\is{e (imperfective)!e V {\ꞌ}ā} is explored.

\subsubsection[Bare e]{Bare \textit{e}}\label{sec:7.2.5.2}

The aspect marker \textit{e} without any postverbal particle occurs in two contexts only: 

\begin{enumerate}
\item
as an exhortative\is{Exhortative} marker, used for non-immediate commands (\sectref{sec:10.2.1}):
\end{enumerate}

\ea\label{ex:7.26}
\gll \textbf{E} \textbf{hāpa{\ꞌ}o} kōrua i a Puakiva. \\
\textsc{exh} care\_for \textsc{2pl} \textsc{acc} \textsc{prop} Puakiva \\

\glt
‘Take care of Puakiva.’ \textstyleExampleref{[R229.420–421]}
\z

\begin{enumerate}
\setcounter{enumi}{1} 
\item
in the imperfective actor-emphatic\is{Actor-emphatic construction} construction (\sectref{sec:8.6.3}):
\end{enumerate}
\ea\label{ex:7.27}
\gll Mā{\ꞌ}aku {\ꞌ}ā \textbf{e} \textbf{e{\ꞌ}a} ki te manu. \\
\textsc{ben.1sg.a} \textsc{ident} \textsc{ipfv} go\_out to \textsc{art} bird \\

\glt
‘I myself will go up to the birds.’ \textstyleExampleref{[Egt-01.014]}
\z

In all other contexts, the \textit{e}{}-marked verb is followed by one or more postverbal particles.

\subsubsection{\textit{e V rō}: future}\label{sec:7.2.5.3}
\is{e (imperfective)!e V rō}
The combination of imperfective \textit{e} and the emphatic particle \textit{rō} expresses future\is{Future} events. It is used to express intentions or plans:

\ea\label{ex:7.28}
\gll Ka noho kōrua ko koro, \textbf{e} hoki \textbf{rō} mai mātou ka muraki tā{\ꞌ}au pāpaku.\\
\textsc{imp} stay \textsc{2sg} \textsc{prom} Dad \textsc{ipfv} return \textsc{emph} hither \textsc{1pl.excl} \textsc{cntg} bury \textsc{poss.2sg.a} corpse\\

\glt 
‘You and Dad should stay, we will return and bury the body.’ \textstyleExampleref{[Ley-4-08.017]}
\z

\ea\label{ex:7.29}
\gll \textbf{E} hāpa{\ꞌ}o \textbf{rō} e au i tā{\ꞌ}ana poki. \\
\textsc{ipfv} care\_for \textsc{emph} \textsc{ag} \textsc{1sg} \textsc{acc} \textsc{poss.3sg.a} child \\

\glt 
‘I will look after her child.’ \textstyleExampleref{[R229.081]} 
\z

\subsubsection{\textit{E} with postverbal demonstratives and with \textit{{\ꞌ}ā/{\ꞌ}ana}}\label{sec:7.2.5.4}
\is{e (imperfective)!e V PVD|(}\is{Demonstrative!postverbal|(}\is{e (imperfective)!e V {\ꞌ}ā|(}
As pointed out in \sectref{sec:7.2.5.1} above, \textit{e} is used in combination with both PVDs\is{Demonstrative!postverbal} and \textit{{\ꞌ}ā}.\footnote{\label{fn:326}In this section, \textit{{\ꞌ}ā} is a shorthand for \textit{{\ꞌ}ā/{\ꞌ}ana}.} With either of these, the clause has either a \textsc{habitual}\is{Aspect!habitual} or a \textsc{continuous} sense, both of which are subcategories of the imperfective. The question is, whether there is any difference between \textit{e V PVD}\is{Demonstrative!postverbal} and \textit{e V {\ꞌ}ā}\is{e (imperfective)!e V {\ꞌ}ā}. In this section the use of \textit{e} with these markers is discussed. This discussion will show that there is a great deal of overlap between both constructions, but that there are differences in use as well.

\textit{E V PVD}\is{Demonstrative!postverbal} and \textit{e V {\ꞌ}ā} occur in main clauses and in temporal subordinate clauses. The former are discussed here, the latter will be discussed in \sectref{sec:11.6.2.2} (see (\ref{ex:11.224}–\ref{ex:11.227})).

In main clauses, \textit{e V PVD}\is{Demonstrative!postverbal} expresses either a continuous\is{Aspect!continuous} action as in (\ref{ex:7.30}–\ref{ex:7.31}), or a habitual\is{Aspect!habitual} action as in (\ref{ex:7.32}–\ref{ex:7.33}):

\ea\label{ex:7.30}
\gll \textbf{E} \textbf{piko} \textbf{era} a Kaiŋa. \\
\textsc{ipfv} hide \textsc{dist} \textsc{prop} Kainga \\

\glt 
‘Kainga was hiding.’ \textstyleExampleref{[R304.093]} 
\z

\ea\label{ex:7.31}
\gll Te {\ꞌ}ori, te hīmene rapa nui te reka \textbf{e} \textbf{u{\ꞌ}i} \textbf{era} e Eva. \\
\textsc{art} dance \textsc{art} song Rapa Nui \textsc{art} entertaining \textsc{ipfv} look \textsc{dist} \textsc{ag} Eva \\

\glt 
‘Eva looked at the dancing, Rapa Nui singing and the entertainment.’ \textstyleExampleref{[R210.133]} 
\z

\ea\label{ex:7.32}
\gll Paurō te mahana a Huri {\ꞌ}a Vai \textbf{e} \textbf{iri} \textbf{era} mai Haŋa Tu{\ꞌ}u Hata ki Kauhaŋa o Varu.\\
every \textsc{art} day \textsc{prop} Huri a Vai \textsc{ipfv} ascend \textsc{dist} from Hanga Tu’u Hata to Kauhanga o Varu\\

\glt 
‘Every day, Huri a Vai went up from Hanga Tu’u Hata to Kauhanga o Varu.’ \textstyleExampleref{[R304.001]} 
\z

\ea\label{ex:7.33}
\gll Ta{\ꞌ}ato{\ꞌ}a me{\ꞌ}e rakerake \textbf{e} \textbf{haka} \textbf{aŋa} \textbf{era} ki a Puakiva. \\
all thing bad:\textsc{red} \textsc{ipfv} \textsc{caus} do \textsc{dist} to \textsc{prop} Puakiva \\

\glt 
‘He made Puakiva do all bad/dirty jobs.’ \textstyleExampleref{[R229.397]} 
\z

\textit{E V {\ꞌ}ā}\is{e (imperfective)!e V {\ꞌ}ā} also expresses either continuous\is{Aspect!continuous} actions as in (\ref{ex:7.34}–\ref{ex:7.35}) or habitual\is{Aspect!habitual} actions as in \REF{ex:7.36}; the latter is not very common, though.

\ea\label{ex:7.34}
\gll \textbf{E} \textbf{{\ꞌ}oka} \textbf{{\ꞌ}ana} a Tama te Rano Kao i te maika {\ꞌ}i raro i te rano. \\
\textsc{ipfv} plant \textsc{cont} \textsc{prop} Tama te Rano Kao \textsc{acc} \textsc{art} banana at below at \textsc{art} crater \\

\glt 
‘Tama te Rano Kao was planting bananas below in the crater.’ \textstyleExampleref{[Mtx-3-11.053]}
\z

\ea\label{ex:7.35}
\gll A koro \textbf{e} \textbf{aŋa} \textbf{{\ꞌ}ā} {\ꞌ}i te {\ꞌ}uahu. \\
\textsc{prop} Dad \textsc{ipfv} work \textsc{cont} at \textsc{art} wharf \\

\glt 
‘Dad was working on the wharf.’ \textstyleExampleref{[R210.041]} 
\z

\ea\label{ex:7.36}
\gll Te hi{\ꞌ}o ho{\ꞌ}i \textbf{e} \textbf{aŋa} \textbf{nō} \textbf{{\ꞌ}ā} {\ꞌ}i rā hora e te nu{\ꞌ}u pa{\ꞌ}ari era.\\
\textsc{art} glass indeed \textsc{ipfv} make just \textsc{cont} at \textsc{dist} time \textsc{ag} \textsc{art} people adult \textsc{dist}\\

\glt
‘The (diving) glasses were made at that time by the older people.’ \textstyleExampleref{[R360.027]} 
\z

\textit{e V {\ꞌ}ā}\is{e (imperfective)!e V {\ꞌ}ā} is also used with adjectives, expressing an enduring state:\footnote{\label{fn:327}The most frequent stative use of \textit{e V {\ꞌ}ā}\is{e (imperfective)!e V {\ꞌ}ā} is with the existential verb \textit{ai}\is{ai ‘to exist’}: the fossilised expression \textit{e ai rō {\ꞌ}ā} ‘there is’ is a very common existential construction (\sectref{sec:9.3.1}).}

\ea\label{ex:7.37}
\gll \textbf{E} \textbf{mata} \textbf{nō} \textbf{{\ꞌ}ana} ho{\ꞌ}i te miro era i hore mai era. \\
\textsc{ipfv} unripe just \textsc{cont} indeed \textsc{art} wood \textsc{dist} \textsc{pfv} cut hither \textsc{dist} \\

\glt 
‘The wood that has been cut is still green.’ \textstyleExampleref{[R200.063]} 
\z

\ea\label{ex:7.38}
\gll \textbf{E} \textbf{{\ꞌ}iti{\ꞌ}iti} \textbf{nō} \textbf{{\ꞌ}ā} a koe. \\
\textsc{ipfv} small:\textsc{red} just \textsc{cont} \textsc{prop} \textsc{2sg} \\

\glt
‘You are still small.’ \textstyleExampleref{[R210.052]} 
\z

By contrast, \textit{e V PVD}\is{Demonstrative!postverbal} is rarely used with statives. It never occurs with adjectives of dimension, value of colour (the prototypical\is{Prototype} adjectives, see \sectref{sec:3.5.1.3}), only with adjectives from other categories:

\ea\label{ex:7.39}
\gll ¿He aha \textbf{e} \textbf{aŋarahi} \textbf{ena} mo haka rehu ō{\ꞌ}oku i a koe? \\
~\textsc{pred} what \textsc{ipfv} difficult \textsc{med} for \textsc{caus} forgotten \textsc{poss.1sg.o} \textsc{acc} \textsc{prop} \textsc{2sg} \\

\glt 
‘Why is it difficult to forget you?’ \textstyleExampleref{[R452.025–026]}
\z

\tabref{tab:47} summarises these findings. Plain x indicates that the category in question is common; (x) indicates uncommon or restricted occurrence.

\begin{table}
\begin{tabularx}{.66\textwidth}{L{35mm}Z{15mm}Z{15mm}} 
\lsptoprule
& \textit{e V PVD}\is{Demonstrative!postverbal}& \textit{e V {\ꞌ}ā}\is{e (imperfective)!e V {\ꞌ}ā}\\
\midrule
continuous event & x& x\\
{habitual\is{Aspect!habitual} event} & x& (x)\\
state & (x)& x\\
\lspbottomrule
\end{tabularx}
\caption{Functions of \textit{e V PVD} and \textit{e V {\ꞌ}ā}}
\label{tab:47}
\end{table}

\newpage 
\subsubsection[Postverbal demonstratives versus {\ꞌ}ā; the function of {\ꞌ}ā]{Postverbal demonstratives versus \textit{{\ꞌ}ā}; the function of \textit{{\ꞌ}ā}}\label{sec:7.2.5.5}

As the discussion in \sectref{sec:7.2.5.4} shows, there is a great deal of overlap between \textit{e V PVD}\is{Demonstrative!postverbal} and \textit{e V {\ꞌ}ā}\is{e (imperfective)!e V {\ꞌ}ā}. Both are used in a habitual\is{Aspect!habitual} and a progressive\is{Progressive} sense; both are found in main and subordinate clauses. Even so, the two cannot always be used interchangeably. One difference lies in the possibility to express additional meaning elements: as shown in \sectref{sec:7.2.5.1}, PVDs\is{Demonstrative!postverbal} do not co-occur with the evaluative particles \textit{nō} or \textit{rō} (\sectref{sec:7.4}); in order to use one of these markers in an imperfective clause, \textit{{\ꞌ}ā} must be used instead (see (\ref{ex:7.36}–\ref{ex:7.38}) above).

While \textit{{\ꞌ}ā} can be used together with \textit{nō}\is{no ‘just’@nō ‘just’} and \textit{rō}, the PVD\is{Demonstrative!postverbal} also has some possibilities of its own: different PVDs\is{Demonstrative!postverbal} indicate different degrees of distance. The default – and by far the most frequent – PVD\is{Demonstrative!postverbal} is \textit{era}, as in (\ref{ex:7.30}–\ref{ex:7.33}) above; \textit{nei} can be used to indicate proximity to the speaker as in \REF{ex:7.40}, \textit{ena} to indicate proximity to the hearer as in \REF{ex:7.41}.

\ea\label{ex:7.40}
\gll Pē nei \textbf{e} \textbf{kī} \textbf{nei} e te nu{\ꞌ}u nei: ko mate {\ꞌ}ana koe. \\
like \textsc{prox} \textsc{ipfv} say \textsc{prox} \textsc{ag} \textsc{art} people \textsc{prox} \textsc{prf} die \textsc{cont} \textsc{2sg} \\

\glt 
‘This is what these people are saying: you have died.’ \textstyleExampleref{[R229.316]} 
\z

\ea\label{ex:7.41}
\gll ¡{\ꞌ}Ī mau {\ꞌ}ā a au \textbf{e} \textbf{taŋi} \textbf{atu} \textbf{ena} ki a kōrua ko te ŋā poki! \\
~\textsc{imm} really \textsc{ident} \textsc{prop} \textsc{1sg} \textsc{ipfv} cry away \textsc{med} to \textsc{prop} \textsc{2sg} \textsc{prom} \textsc{art} \textsc{pl} child \\

\glt 
‘I was just missing you, children!’ \textstyleExampleref{[R313.097]} 
\z

Apart from these possibilities to express additional meaning elements, there is a more general difference between \textit{e V era}\is{e (imperfective)!e V PVD} and \textit{e V {\ꞌ}ā}\is{e (imperfective)!e V {\ꞌ}ā}. This is suggested by two facts:

%\setcounter{listWWviiiNumxcvleveli}{0}
\begin{enumerate}
\item 
As discussed above, \textit{e V {\ꞌ}ā}\is{e (imperfective)!e V {\ꞌ}ā} can be used with adjectives to indicate a state (see (\ref{ex:7.37}–\ref{ex:7.38}) above). On the other hand, adjectives rarely enter into the \textit{e V PVD}\is{Demonstrative!postverbal} construction. A similar difference can be observed in temporal clauses\is{Clause!temporal} (discussed in \sectref{sec:11.6.2.2}): \textit{e V {\ꞌ}ā}\is{e (imperfective)!e V {\ꞌ}ā} is more stative-like, while \textit{e V PVD}\is{Demonstrative!postverbal} is more dynamic.

\item 
In main clauses, \textit{e V {\ꞌ}ā}\is{e (imperfective)!e V {\ꞌ}ā} constructions only rarely have habitual\is{Aspect!habitual} sense; habituality is usually expressed by \textit{e V PVD}\is{Demonstrative!postverbal}. Similarly, in cohesive clauses\is{Clause!cohesive} (\sectref{sec:11.6.2.1}), I have not found any example of habitual\is{Aspect!habitual} \textit{e V {\ꞌ}ā}\is{e (imperfective)!e V {\ꞌ}ā}, while habitual\is{Aspect!habitual} \textit{e V PVD}\is{Demonstrative!postverbal} is quite common. 

\end{enumerate}

This raises the question of the function of the marker \textit{{\ꞌ}ā}\is{a (postverbal)@{\ꞌ}ā (postverbal)}. According to \citet[52]{WeberR2003}, \textit{{\ꞌ}ā} is a progressive\is{Progressive} marker. This would fit many of its occurrences; however, it should be noted that progressive\is{Progressive} events may also be expressed by \textit{e V PVD}\is{Demonstrative!postverbal}. Moreover, \textit{e V {\ꞌ}ā}\is{e (imperfective)!e V {\ꞌ}ā} can be used with stative verbs\is{Verb!stative}, whereas the progressive\is{Progressive} (in Comrie’s definition, see \sectref{sec:7.2.5.1} above) is limited to nonstative verbs. 

Another fact which should be taken into consideration, is that \textit{{\ꞌ}ā} occurs after the perfect marker \textit{ko/ku} as well; in fact, after the perfect marker \textit{{\ꞌ}ā} is obligatory. Now perfect aspect\is{Aspect!perfect} is incompatible with the progressive\is{Progressive}; rather, it indicates the continuing relevance of a situation which has come about in the past. As will be discussed in \sectref{sec:7.2.7.1} and \sectref{sec:7.2.7.4} below, \textit{ko V {\ꞌ}ā}\is{ko V {\ꞌ}ā (perfect aspect)} expresses a state of affairs resulting from an earlier event, rather than the event itself. We may conclude that \textit{{\ꞌ}ā} marks \textsc{continuity}\is{Aspect!continuous} or stability over time: \textit{e V {\ꞌ}ā}\is{e (imperfective)!e V {\ꞌ}ā} expresses that an event or a state continues; \textit{ko V {\ꞌ}ā}\is{ko V {\ꞌ}ā (perfect aspect)} indicates the continuing relevance of a state which has started in the past.\footnote{\label{fn:328}\textit{Ana} occurs in other Eastern Polynesian languages (which have not retained the \is{Proto-Polynesian}Proto-Polynesian glottal\is{Glottal plosive} plosive\is{Glottal plosive}) as a post-verbal particle marking a continuing action or state, usually after imperfective \textit{e}, e.g. \ili{Hawaiian} (\citealt[57–60]{ElbertPukui1979}), \ili{Marquesan} (\citealt[67]{MutuTeìkitutoua2002}), \ili{Mangarevan} \citep[32]{Janeau1908}, \ili{Māori} (\citealt[416–419]{Bauer1993}). Interestingly, in \ili{Marquesan} the variant \textit{aa} is used as well; given the fact that other languages only have the longer form, it is not unlikely that Rapa Nui \textit{{\ꞌ}ā} and \ili{Marquesan} \textit{aa} are independent developments. In \ili{Hawaiian}, \textit{ana} alternates with postverbal demonstratives as in Rapa Nui.
The use of \textit{{\ꞌ}ana/{\ꞌ}ā} in the noun phrase (\sectref{sec:5.9}) is unique to Rapa Nui.} Hence the gloss \textsc{cont}(inuity).\footnote{\label{fn:329}In addition, \textit{{\ꞌ}ā} is used in negated perfect aspect clauses, marked with preverbal \textit{kai} (see (\ref{ex:10.123}–\ref{ex:10.124}) on p.~\pageref{ex:10.123}).} 

Notice that this does not mean that \textit{{\ꞌ}ā} as such is a marker of continuous aspect. Continuous aspect (expressing events which continue for some time, whether stative or nonstative) is a subcategory of the imperfective, which is expressed by either \textit{e V PVD}\is{Demonstrative!postverbal} or \textit{e V {\ꞌ}ā}\is{e (imperfective)!e V {\ꞌ}ā}. \textit{{\ꞌ}Ā} itself simply emphasises the continuity or stability of a situation, whether in combination with imperfective \textit{e} or perfect \textit{ko}. 

This is confirmed by the occasional use of \textit{{\ꞌ}ā} after the preverbal marker \textit{mai} (\sectref{sec:11.5.5}): \textit{mai} as such indicates a temporal boundary (‘before, until’); in combination with \textit{{\ꞌ}ana} it expresses the continuation of a state up to a certain point: ‘while, as long as’.

The meaning of postverbal \textit{{\ꞌ}ā} is clearly related to the meaning of postnominal \textit{{\ꞌ}ā} (\sectref{sec:5.9}); while postverbal \textit{{\ꞌ}ā} indicates stability of an event over time, postnominal \textit{{\ꞌ}ā} underlines the identity of a referent, i.e. stability in reference: ‘the same, himself’.\is{e (imperfective)!e V PVD|)}\is{Demonstrative!postverbal|)}\is{e (imperfective)!e V {\ꞌ}ā|)}
\subsubsection{Summary}\label{sec:7.2.5.6}

\textit{E} is the imperfective marker. Its temporal/aspectual value is further defined by certain postverbal particles, as indicated in \tabref{tab:48}.

\begin{table}
\begin{tabularx}{\textwidth}{p{55mm}X}
\lsptoprule
\textit{e} &— imperative;\newline — imperfective actor-emphatic\\
\tablevspace
{\textit{e V rō} (emphatic marker)} &— future\\
\tablevspace
{\textit{e V nei}/\textit{ena}/\textit{era}} \newline (postverbal demonstratives) &— continuous; \newline — habitual; \newline — (stative – rarely) \\
\tablevspace
{\textit{e V {\ꞌ}ā/{\ꞌ}ana} (continuity marker)} &— continuous; \newline — stative; \newline — (habitual – rarely) \\
\lspbottomrule
\end{tabularx}
\caption{Functions of imperfective e}
\label{tab:48}
\end{table}


In clauses where the verb is non-initial, \textit{e} tends to be used whenever the clause has nonpast reference; this will be briefly discussed in \sectref{sec:7.2.8} below.
\is{e (imperfective)|)}
\subsection{The contiguity marker \textit{ka}}\label{sec:7.2.6}
\is{ka (aspect marker)|(}\subsubsection[Introduction: ka in Polynesian and in Rapa Nui]{Introduction: \textit{ka} in Polynesian and in Rapa Nui}\label{sec:7.2.6.1}

\textit{Ka} occurs in most Polynesian languages. It tends to be a somewhat elusive marker. \citet[347–348]{Pawley1970} glosses \is{Proto-Polynesian}PPN/PNP \textit{*kaa} as “anticipatory, future” and \is{Central-Eastern Polynesian}PCE \textit{*kaa} as “inceptive”; Pollex (\citealt{GreenhillClark2011}) has \is{Proto-Polynesian}PPN *\textit{ka} as an inceptive marker. In most grammars of Polynesian languages, it is explained as inceptive and/or future and/or imperative\is{Imperative}; the latter function occurs only in \is{Eastern Polynesian}EP languages.\footnote{\label{fn:330}\is{Proto-Polynesian}PPN \textit{*ka} reflects a Proto-Oceanic coordinating conjunction\is{Conjunction} \textit{*ka} ‘and then’ (\citealt[85]{LynchRoss2002}; \citealt{Lichtenberk2014}), which developed into a marker of sequentiality, future tense, irrealis\is{Irrealis}, imperative\is{Imperative} and/or inceptive in various (groups of) languages. Evidently, the use of \textit{*ka} is not narrowed down to a single function in Polynesian.} 

For Rapa Nui, the existing grammars offer little analysis on \textit{ka}. \citet[63, 72]{Englert1978} does not list or discuss \textit{ka} among the “tenses”, but only gives examples of its use in the imperative\is{Imperative}. According to \citet[37]{DuFeu1996} \textit{ka} and \textit{ki} are momentary particles indicating temporal relationships between actions; she gives examples of the use of \textit{ka} in the imperative\is{Imperative} \REF{ex:7.37}, \textit{ka V rō}\is{ka (aspect marker)!ka V rō} in the sense ‘until’ \REF{ex:7.51} and \textit{ka} in temporal clauses\is{Clause!temporal} referring to the future. \citet[154]{Chapin1978} indicates that there are various other uses of \textit{ka} besides the imperative\is{Imperative}, but that on the basis of his data, it is not possible to reach any satisfactory generalisation regarding these uses.

\citet[33]{WeberR2003}, on the contrary, offers a thorough analysis of \textit{ka}. On the basis of a number of newer narrative texts he concludes that \textit{ka} does not give information about the aspectual value of the verb itself, but about its temporal relation to a following or preceding proposition. He postulates that \textit{ka} indicates temporal \textsc{contiguity}\is{Contiguity, temporal} between two events, in that the two events are temporally adjacent or overlapping. 

My analysis, as outlined below, largely confirms and refines Weber’s findings. In many of its uses, \textit{ka} represents a \textsc{boundary}, setting off one event from another; this happens for example when one event represents a temporal limit for another, ongoing event. In other cases \textit{ka} indicates \is{Simultaneity}\textsc{simultaneity} with respect to the event expressed in a preceding or following clause. This simultaneity can be either total or partial (i.e. overlapping). Both situations can be subsumed under the label “contiguity” (\textsc{cntg}), proposed by \citet{WeberR2003}. 

This section discusses the contiguity marker \textit{ka}; the use of \textit{ka} as imperative\is{Imperative} marker (which occurs more frequently in discourse) is discussed in \sectref{sec:10.2.1}. \citet{WeberR2003} treats the contiguity marker and the imperative\is{Imperative} marker as different particles; in \sectref{sec:10.2.1} I will argue that the two are best considered as a single particle.

Another use of \textit{ka} not discussed in the present section, is \textit{ka}\is{ka (numeral particle)} preceding numerals (\sectref{sec:4.3.2.2}). The discussion and examples there show, that \textit{ka} indicates a quantity which has been reached, a use which corresponds neatly to \textit{ka} as a boundary marker. 

In the following subsections, different contexts in which \textit{ka} occurs, are discussed in turn. First a number of uses in subordinate clauses are briefly listed (\sectref{sec:7.2.6.2}), then its use in main clauses is discussed (\sectref{sec:7.2.6.3}). In \sectref{sec:7.2.6.4}, some minor uses of \textit{ka} are listed.

\subsubsection{\textit{Ka} in subordinate clauses}\label{sec:7.2.6.2}

\textit{Ka} is used in a wide range of subordinate clauses. In this section, these constructions are listed with a single example; they are discussed in more detail in Chapter 11. 

\textit{Ka} occurs in complements of perception verbs\is{Verb!perception} (\sectref{sec:11.3.1.1}):

\ea\label{ex:7.42}
\gll He u{\ꞌ}i atu, \textbf{ka} pū te manu taiko. \\
\textsc{ntr} look away \textsc{cntg} approach \textsc{art} bird \textit{taiko} \\

\glt
‘She saw a \textit{taiko} bird come by.’ \textstyleExampleref{[Ley-9-55.078]}
\z

In relative clauses\is{Clause!relative} (\sectref{sec:11.4.3}), \textit{ka} indicates events posterior\is{Posteriority} to the time of reference:

\ea\label{ex:7.43}
\gll Te {\ꞌ}īŋoa o te kai era [\textbf{ka} ma{\ꞌ}u mai era ki a koe] he ioioraŋi. \\
\textsc{art} name of \textsc{art} food \textsc{dist} ~\textsc{cntg} carry hither \textsc{dist} to \textsc{prop} \textsc{2sg} \textsc{pred} \textit{ioioraŋi} \\

\glt
‘The name of the food they will bring you is \textit{ioioraŋi}.’ \textstyleExampleref{[R310.060]} 
\z

In temporal clauses\is{Clause!temporal} (\sectref{sec:11.6.2.1}, \sectref{sec:11.6.2.2}), \textit{ka} indicates temporal contiguity with the event in the main clause\is{Simultaneity}:

\ea\label{ex:7.44}
\gll \textbf{Ka} hakame{\ꞌ}eme{\ꞌ}e era he riri a Taparahi. \\
\textsc{cntg} mock \textsc{dist} \textsc{ntr} angry \textsc{prop} Taparahi \\

\glt
‘When they mocked, Taparahi would get angry.’ \textstyleExampleref{[R250.151]} 
\z

\textit{Ka} marks conditional clauses\is{Clause!conditional} (\sectref{sec:11.6.6}):

\ea\label{ex:7.45}
\gll \textbf{Ka} hāŋai atu ena ki a koe, he mate koe. \\
\textsc{cntg} feed away \textsc{med} to \textsc{prop} \textsc{2sg} \textsc{ntr} die \textsc{2sg} \\

\glt
‘If (the two spirits) feed you, you will die.’ \textstyleExampleref{[R310.061]} 
\z

\textit{Ka} occurs after certain temporal conjunctions: \textit{{\ꞌ}ō ira}\is{o ira ‘before’@{\ꞌ}ō ira ‘before’} ‘before’ (\sectref{sec:11.6.2.4}); \textit{{\ꞌ}ātā}\is{ata ‘until’@{\ꞌ}ātā ‘until’}/\textit{{\ꞌ}ā} ‘until’, \textit{{\ꞌ}ahara}\is{ahara ‘until’@{\ꞌ}ahara ‘until’} ‘until’ (\sectref{sec:11.6.2.5}): 

\ea\label{ex:7.46}
\gll Mai ki hāpa{\ꞌ}o nō tātou i a ia \textbf{{\ꞌ}ātā} \textbf{ka} nuinui \textbf{rō}. \\
hither \textsc{hort} care\_for just \textsc{1pl.incl} \textsc{acc} \textsc{prop} \textsc{3sg} until \textsc{cntg} big:\textsc{red} \textsc{emph} \\

\glt
‘Let us take care of him until he is big.’ \textstyleExampleref{[R211.063]} 
\z

Without a conjunction\is{Conjunction}, \textit{ka V rō}\is{ka (aspect marker)!ka V rō} marks a temporal boundary, ‘until’ (\sectref{sec:11.6.2.5}):

\ea\label{ex:7.47}
\gll He kai a Te Manu \textbf{ka} mākona \textbf{rō}. \\
\textsc{ntr} eat \textsc{prop} Te Manu \textsc{cntg} satiated \textsc{emph} \\

\glt
‘Te Manu ate until he was satiated.’ \textstyleExampleref{[R245.067]} 
\z

\textit{Ka V atu} marks a concessive clause (\sectref{sec:11.6.7}):

\ea\label{ex:7.48}
\gll \textbf{Ka} rahi \textbf{atu} tā{\ꞌ}aku poki, e hāpa{\ꞌ}o nō e au {\ꞌ}ā. \\
\textsc{cntg} many away \textsc{poss.1sg.a} child \textsc{ipfv} care\_for just \textsc{ag} \textsc{1sg} \textsc{ident} \\

\glt
‘Even if I have many children, I will care for them myself.’ \textstyleExampleref{[R229.023]} 
\z

In most of these examples, \textit{ka} expresses temporal contiguity\is{Contiguity, temporal}. The event in the \textit{ka-}clause is temporally contiguous to the event in the main clause; often it indicates a boundary to the event in the main clause as in \REF{ex:7.44}, \REF{ex:7.46} and \REF{ex:7.47}; sometimes the event overlaps with or is simultaneous to the main clause event as in \REF{ex:7.42}. 

\subsubsection{\textit{Ka} in main clauses}\label{sec:7.2.6.3}

When \textit{ka} occurs in main clauses, the clause often refers to an event \textsc{posterior}\is{Posteriority} to the time of reference, something which happens later than other events in the context. As in subordinate clauses, the verb is often followed by a postverbal demonstrative\is{Demonstrative!postverbal}.

In direct speech, the time of reference is the moment of speech; the \textit{ka}{}-clause refers to the future, but always the immediate or very near future\is{Future}:

\ea\label{ex:7.49}
\gll {\ꞌ}I {\ꞌ}Ohovehi mātou \textbf{ka} \textbf{noho} \textbf{nei} {\ꞌ}ātā ki te ŋaro haŋa o te pahī. \\
at Ohovehi \textsc{1pl.excl} \textsc{cntg} stay \textsc{prox} until to \textsc{art} disappear \textsc{nmlz} of \textsc{art} ship \\

\glt 
‘We will stay in Ohovehi until the ship disappears (behind the horizon).’ \textstyleExampleref{[R210.083]} 
\z

\ea\label{ex:7.50}
\gll ¿{\ꞌ}I hē tāua \textbf{ka} \textbf{kimi} \textbf{nei} i te tāua māmari? \\
~at \textsc{cq} \textsc{1du.incl} \textsc{cntg} search \textsc{prox} \textsc{acc} \textsc{art} \textsc{1du.incl} egg \\

\glt
‘Where will we search for eggs?’ \textstyleExampleref{[R245.199]} 
\z

In these cases – different from the subordinate clauses in the previous section – the temporal/aspectual reference of the clause is not determined by its relation to surrounding clauses, but independently anchored in the non-linguistic context. For example, \REF{ex:7.50} forms a complete speech, so the sentence has no direct linguistic context. The contiguity mark\is{Contiguity, temporal}er indicates that the event is contiguous to the time of reference, in this case, the moment of speech.

Posterior \textit{ka}{}-clauses also occur in narrative contexts. These clauses describe events which happen later than the main line of events. As in the examples above, the \textit{ka}{}-event is posterior\is{Posteriority} to the time of reference (in this case, the main line of the story). An example:

\ea\label{ex:7.51}
\gll ...he oho ararua ki Santiago ki te hare era o Ma{\ꞌ}atea. {\ꞌ}I ira ho{\ꞌ}i  \textbf{ka} \textbf{noho} \textbf{era}.\\
~~~\textsc{ntr} go the\_two to Santiago to \textsc{art} house \textsc{dist} of Ma’atea at \textsc{ana} indeed  \textsc{cntg} stay \textsc{dist}\\

\glt 
‘The two went to Santiago to the house of Ma’atea. There she would stay.’ \textstyleExampleref{[R210.221]} 
\z

Sometimes two successive clauses are both marked with \textit{ka}, indicating \textsc{temporal contiguity} between the two events: one event marks the boundary of the other. In this construction, the first clause is a temporal clause\is{Clause!temporal} providing a time frame for the second. The second clause is the main clause, but this can only be concluded on semantic grounds; the clauses do not differ syntactically, except in their respective order. 

\ea\label{ex:7.52}
\gll \textbf{Ka} \textbf{haka} \textbf{mao} tū vānaŋa era a Moe, \textbf{ka} \textbf{taŋi} \textbf{mai} te oe  mo o{\ꞌ}o ananake ki te rāua hāpī.\\
\textsc{cntg} \textsc{caus} finish \textsc{dem} speak \textsc{dist} \textsc{prop} Moe \textsc{cntg} cry hither \textsc{art} bell  for enter together to \textsc{art} \textsc{3pl} learn\\

\glt 
‘When Moe had finished speaking, the bell sounded for all to enter their class.’ \textstyleExampleref{[R315.075]} 
\z

\ea\label{ex:7.53}
\gll \textbf{Ka} \textbf{tu{\ꞌ}u} \textbf{mai} \textbf{nei}, e rāua mau {\ꞌ}ana \textbf{ka} \textbf{{\ꞌ}a{\ꞌ}amu} \textbf{nei}  i te rāua {\ꞌ}ati.\\
\textsc{cntg} arrive hither \textsc{prox} \textsc{ag} \textsc{3pl} really \textsc{ident} \textsc{cntg} tell \textsc{prox}  \textsc{acc} \textsc{art} \textsc{3pl} problem\\

\glt 
‘When they arrived, they themselves told about their trouble.’ \textstyleExampleref{[R361.035]} 
\z

Finally, \textit{ka} in main clauses is common after certain clause-initial \textsc{particles}, especially deictic particles: \textit{{\ꞌ}ī}\is{i (deictic)@{\ꞌ}ī (deictic)} ‘here; right now’; \textit{{\ꞌ}ai}\is{ai (deictic)@{\ꞌ}ai (deictic)} ‘there (\sectref{sec:4.5.4.1}); then’.

\subsubsection[Other uses of ka]{Other uses of \textit{ka}}\label{sec:7.2.6.4}

Firstly, \textit{ka} is used in an exclamative\is{Exclamative} construction preceding adjectives (\sectref{sec:10.4.1}).

Secondly, as discussed in the previous sections, \textit{ka} is commonly used to indicate temporally contiguous events. A natural derivative from this is its use to indicate alternatives. When there are two alternative events or states, either of which can be true, they can be expressed by two \textit{ka-}clauses. An appropriate translation is ‘whether ... or’.

\ea\label{ex:7.54}
\gll \textbf{Ka} \textbf{{\ꞌ}uri{\ꞌ}uri} \textbf{ka} \textbf{teatea} te huruhuru, ko tū māhatu {\ꞌ}ā. \\
\textsc{cntg} black:\textsc{red} \textsc{cntg} white:\textsc{red} \textsc{art} hair \textsc{prom} \textsc{dem} heart \textsc{ident} \\

\glt 
‘Whether your hair is black or white, it’s the same heart.’ \textstyleExampleref{[R211.078]} 
\z

\ea\label{ex:7.55}
\gll O te ta{\ꞌ}ato{\ꞌ}a mahana te aŋa nei e aŋa era \textbf{ka} \textbf{rohirohi},  \textbf{ka} \textbf{ta{\ꞌ}e} \textbf{rohirohi}.\\
of \textsc{art} all day \textsc{art} work \textsc{prox} \textsc{ipfv} do \textsc{dist} \textsc{cntg} tired:\textsc{red}  \textsc{cntg} \textsc{conneg} tired:\textsc{red}\\

\glt 
‘The work was done every day, whether (you were) tired or not.’ \textstyleExampleref{[R539-2.026]}
\z

\subsubsection{Summary}\label{sec:7.2.6.5}

\textit{Ka} is best characterised as a contiguity marker: it marks events which are temporally contiguous to events in a neighbouring clause. This means that the temporal value of a \textit{ka-}marked clause often depends on a preceding or following clause; not surprising, \textit{ka} often occurs in a subordinating clause, relating it temporally to the main clause.

The \textit{ka}{}-clause may also be related to an (implied) time of reference; it is usually posterior to this reference time. 
\is{ka (aspect marker)|)}
\subsection{Perfect aspect \textit{ko V {\ꞌ}ā}}\label{sec:7.2.7}
\is{ko V {\ꞌ}ā (perfect aspect)|(}\is{ko V {\ꞌ}ā (perfect aspect)}
Per\is{Aspect!perfect}fect aspect is marked by the aspect marker\is{Aspect marker} \textit{ku/ko}, in combination with the continuous marker \textit{{\ꞌ}ana} or \textit{{\ꞌ}ā} (\sectref{sec:7.2.5.5}). 

First an etymological note. The aspectual particle \textit{ko/ku} reflects \is{Proto-Polynesian}PPN \textit{*kua}, which serves as a perfect aspect\is{Aspect!perfect} marker\is{Aspect marker} in almost every Polynesian language \citep[30]{Clark1976}.\footnote{\label{fn:331}\citet[15]{MassamLee2006} mistakenly assume that preverbal \textit{ko} in Rapa Nui is the same particle as the prominence marker \textit{ko}. The historical data show that this cannot be the case.} It has the form \textit{kua} in most languages; apart from Rapa Nui, only a few other languages have dropped the final \textit{-a}.\footnote{\label{fn:332}All of the latter are outliers (e.g. \ili{Takuu}, \ili{Kapingamarangi}), except \ili{Marquesan} \citep[34]{Zewen1987} and \ili{Mangarevan} \citep[61]{Janeau1908}, in which the \textit{-a} is dropped before verbs having more than two syllables\is{Syllable}.} 

In Rapa Nui both \textit{ku} and \textit{ko} are used as perfect aspect\is{Aspect!perfect} marker\is{Aspect marker}. On etymological grounds, \textit{ku} must be the original form, and indeed, in older texts only \textit{ku} is found. Today \textit{ko} is prevalent, while the use of \textit{ku} is limited to certain speakers.
\is{a (postverbal)@{\ꞌ}ā (postverbal)|(}

\textit{{\ꞌ}Ā} is a reduced form of \textit{{\ꞌ}ana}; the choice between both variants is free (\sectref{sec:5.9}). A verb marked with \textit{ku/ko} is always followed by \textit{{\ꞌ}ana/{\ꞌ}ā}. 

According to \citet{Comrie1976}, the perfect aspect\is{Aspect!perfect} relates a state to a preceding situation: the perfect\is{Aspect!perfect} signals that a situation in the past has a continuing relevance in the present.

In Rapa Nui, the perfect \textit{ko V {\ꞌ}ā}\is{ko V {\ꞌ}ā (perfect aspect)}\footnote{\label{fn:333}Henceforth, \textit{ko V {\ꞌ}ā}\is{ko V {\ꞌ}ā (perfect aspect)} is used as a shorthand for \textit{ko/ku V {\ꞌ}ana/{\ꞌ}ā}.} emphasises a current state of affairs. With active verbs\is{Verb!active}, it refers to an event anterior\is{Anteriority} to the time of reference, which has resulted in a current situation. With stative verbs\is{Verb!stative}, it refers to the state of affairs itself, which has started at some moment in the past. (In fact, with some verbs it is questionable whether \textit{ko V {\ꞌ}ā} refers to the anterior event or to a resulting state, an ambiguity which is inherent in the character of the perfect.) The time of reference may be in the present, in the past, or in the future; in other words, \textit{ko V {\ꞌ}ā} has no temporal value. 

In the following sections, different uses of the perfect aspect\is{Aspect!perfect} will be discussed. 

\subsubsection[Anterior events leading to a present situation]{Anterior events leading to a present situation}\label{sec:7.2.7.1}
\is{Anteriority}
With active verbs\is{Verb!active}, \textit{ko V {\ꞌ}ā}\is{ko V {\ꞌ}ā (perfect aspect)} indicates that the action has taken place and has led to a certain state of affairs which still holds at the time of reference. The time of reference may be the present, in which case the action took place in the past. A few examples:

\ea\label{ex:7.56}
\gll Ko hiko {\ꞌ}ā tā{\ꞌ}aku haraoa e Te Manu. \\
\textsc{prf} snatch \textsc{cont} \textsc{poss.1sg.a} bread \textsc{ag} Te Manu \\

\glt 
‘My bread has been snatched by Te Manu.’ \textstyleExampleref{[R245.039]} 
\z

\ea\label{ex:7.57}
\gll ¿Ko kai {\ꞌ}ā koe? \\
~\textsc{prf} eat \textsc{cont} \textsc{2sg} \\

\glt 
‘Have you eaten?’ \textstyleExampleref{[R245.058]} 
\z

\ea\label{ex:7.58}
\gll Ko haka moe {\ꞌ}ana {\ꞌ}i rote {\ꞌ}ōpītara. \\
\textsc{prf} \textsc{caus} lie \textsc{cont} at inside\_\textsc{art} hospital \\

\glt 
‘They have put him into hospital.’ \textstyleExampleref{[R210.122]} 
\z

The time of reference may also be in the past\is{Past}. This happens especially in narrative, where \textit{ko V {\ꞌ}ā}\is{ko V {\ꞌ}ā (perfect aspect)} clauses relate events which have taken place anterior\is{Anteriority} to theme line events. The \ili{English} equivalent is the pluperfect. The following example shows the alternation between theme-line events (\textit{he}) and anterior events (\textit{ko V {\ꞌ}ā}).

\ea\label{ex:7.59}
\gll \textbf{He} e{\ꞌ}a tau poki era, \textbf{he} {\ꞌ}a{\ꞌ}aru mai. \textbf{Ku} e{\ꞌ}a \textbf{{\ꞌ}ā} Kaiŋa,  \textbf{ku} kā \textbf{{\ꞌ}ā} i te {\ꞌ}umu, \textbf{he} unu i tau moa era...\\
\textsc{ntr} go\_out \textsc{dem} child \textsc{dist} \textsc{ntr} grab hither \textsc{prf} go\_out \textsc{cont} Kainga  \textsc{prf} kindle \textsc{cont} \textsc{acc} \textsc{art} earth\_oven \textsc{ntr} pluck \textsc{acc} \textsc{dem} chicken \textsc{dist}\\

\glt 
‘The child went out and grabbed (the chickens). Kainga had already gone out and lighted the fire for the earth oven; he plucked those chickens...’ \textstyleExampleref{[Ley-8-53.004]}
\z

The time of reference may be in the future\is{Future}: at a certain point in time something will have happened. 

\ea\label{ex:7.60}
\gll \textbf{Ko} \textbf{e{\ꞌ}a} \textbf{{\ꞌ}ā} te ŋā vārua era ana tu{\ꞌ}u kōrua. \\
\textsc{prf} go\_out \textsc{cont} \textsc{art} \textsc{pl} spirit \textsc{dist} \textsc{irr} arrive \textsc{2pl} \\

\glt 
‘The spirits will have left when you arrive.’ \textstyleExampleref{[R310.273]} 
\z

Sometimes \textit{ko V {\ꞌ}ā}\is{ko V {\ꞌ}ā (perfect aspect)} is used with action verbs without an anterior sense. The event takes place not before, but at the time of reference; for example, it takes place at the same time as events in the immediate context which are marked with \textit{he}. In these cases \textit{ko V {\ꞌ}ā} emphasises the completed character of the event: the event is done as soon as it is started. An example is the following.

\ea\label{ex:7.61}
\gll Hora hitu \textbf{ko} \textbf{o{\ꞌ}o} \textbf{{\ꞌ}ā} ki rote hare pure ki te pure. \\
hour seven \textsc{prf} enter \textsc{cont} at inside\_\textsc{art} house pray to \textsc{art} prayer \\

\glt
‘Seven o’clock they entered into the chapel for prayer.’ \textstyleExampleref{[R210.140]} 
\z

The perfect emphasises that at seven o’clock the action of entering was over and done with; in other words, it took place at exactly seven o’clock. 

\subsubsection[Present states]{Present states}\label{sec:7.2.7.2}

With stative verbs\is{Verb!stative}, \textit{ko V {\ꞌ}ā}\is{ko V {\ꞌ}ā (perfect aspect)} is frequently used to indicate that a state of affairs has been reached. Use of the perfect aspect\is{Aspect!perfect} suggests that some change has taken place, leading to the situation at the time of reference; in other words, the situation has not always been there, but is the result of some unspecified prior process.\footnote{\label{fn:334}Cf. \citet[57]{Comrie1976}: in many languages, present states are expressed using the perfect, whereas in \ili{English}, the present is used in such cases: \ili{Greek} \textit{tethnēkenai}, \ili{English} ‘be dead’.} 

Here are a number of examples of \textit{ko V {\ꞌ}ā}\is{ko V {\ꞌ}ā (perfect aspect)} with stative verbs\is{Verb!stative}.

\ea\label{ex:7.62}
\gll Ko ve{\ꞌ}ave{\ꞌ}a {\ꞌ}ā {\ꞌ}i te rahi o te māuiui. \\
\textsc{prf} hot:\textsc{red} \textsc{cont} at \textsc{art} much of \textsc{art} sick \\

\glt 
‘She was hot because of her grave illness.’ \textstyleExampleref{[R229.229]} 
\z

\ea\label{ex:7.63}
\gll Ku pakapaka {\ꞌ}ā te henua. Ku oŋe {\ꞌ}ā tātou. \\
\textsc{prf} dry:\textsc{red} \textsc{cont} \textsc{art} land \textsc{prf} shortage \textsc{cont} \textsc{1pl.incl} \\

\glt 
‘The land is dry. We are in need.’ \textstyleExampleref{[R352.116]} 
\z

\ea\label{ex:7.64}
\gll Hora nei pa{\ꞌ}i ko veve {\ꞌ}ā te taŋata. \\
time \textsc{prox} in\_fact \textsc{prf} poor \textsc{cont} \textsc{art} person \\

\glt
‘Now the people are poor.’ \textstyleExampleref{[R250.128]} 
\z

In all these cases \textit{ko V {\ꞌ}ā}\is{ko V {\ꞌ}ā (perfect aspect)} retains its character as a perfect aspect\is{Aspect!perfect} marker\is{Aspect marker}: the present situation is one which has not always existed, but which has come about at some point, often quite recently. 

The range of verbs which commonly take \textit{ko V {\ꞌ}ā}\is{ko V {\ꞌ}ā (perfect aspect)} is wide. Roughly speaking, three categories can be distinguished:

In the first place: physical and mental states, including for example pain, sickness, anger, happiness. Also included in this category are \textit{ha{\ꞌ}uru} ‘to sleep’, \textit{ora} ‘to live’ (\textsc{prf} ‘to recover’), and \textit{mate} ‘to die’ (\textsc{prf} ‘to be dead’), as well as verbs with a more active sense like \textit{kata} ‘to laugh’, \textit{taŋi} ‘to cry’, \textit{{\ꞌ}eki{\ꞌ}eki} ‘to sob’.

\ea\label{ex:7.65}
\gll Ko \textbf{mamae} {\ꞌ}ā tō{\ꞌ}oku niho. \\
\textsc{prf} pain \textsc{cont} \textsc{poss.1sg.o} tooth \\

\glt 
‘My tooth hurts.’ \textstyleExampleref{[R208.275]} 
\z

\ea\label{ex:7.66}
\gll Kai e{\ꞌ}a tū nu{\ꞌ}u era ki haho; ko \textbf{tataŋi} {\ꞌ}ana. \\
\textsc{neg.pfv} go\_out \textsc{dem} people \textsc{dist} to outside \textsc{prf} \textsc{pl}:cry \textsc{cont} \\

\glt 
‘Those people did not go outside; they cried.’ \textstyleExampleref{[R229.329]} 
\z

\ea\label{ex:7.67}
\gll He mana{\ꞌ}u e Puakiva ko \textbf{ha{\ꞌ}uru} {\ꞌ}ana. \\
\textsc{ntr} think \textsc{ag} Puakiva \textsc{prf} sleep \textsc{cont} \\

\glt 
‘Puakiva thought that (Kava) was asleep.’ \textstyleExampleref{[R229.292]} 
\z

Secondly: verbs of volition\is{Verb!volition}.

\ea\label{ex:7.68}
\gll A au ko \textbf{pohe} rivariva {\ꞌ}ana mo ha{\ꞌ}uru. \\
\textsc{prop} \textsc{1sg} \textsc{prf} desire good:\textsc{red} \textsc{cont} for sleep \\

\glt 
‘I really want to sleep.’ \textstyleExampleref{[R229.246]} 
\z

\ea\label{ex:7.69}
\gll Ko \textbf{haŋa} {\ꞌ}ā a ia mo oho mo hāpī.  \\
\textsc{prf} want \textsc{cont} \textsc{prop} \textsc{3sg} for go for learn  \\

\glt 
‘She wants to go to study.’ \textstyleExampleref{[R210.066]} 
\z

Thirdly: verbs of perception\is{Verb!perception} (esp. \textit{ŋaro{\ꞌ}a} ‘to hear/perceive’) and cognition\is{Verb!cognitive}.

\ea\label{ex:7.70}
\gll Ko \textbf{ŋaro{\ꞌ}a} mai {\ꞌ}ana {\ꞌ}ō e au te hau{\ꞌ}a huru kē  o te kai nei.\\
\textsc{prf} perceive hither \textsc{cont} really \textsc{ag} \textsc{1sg} \textsc{art} smell manner different  of \textsc{art} food \textsc{prox}\\

\glt 
‘I smell a strange smell of this food.’ \textstyleExampleref{[R236.026]} 
\z

\ea\label{ex:7.71}
\gll E nua, {\ꞌ}i te hora nei ko \textbf{{\ꞌ}ite} {\ꞌ}ana a au i te parauti{\ꞌ}a. \\
\textsc{voc} Mum at \textsc{art} time \textsc{prox} \textsc{prf} know \textsc{cont} \textsc{prop} \textsc{1sg} \textsc{acc} \textsc{art} truth \\

\glt 
‘Mum, now I know the truth.’ \textstyleExampleref{[R229.495]} 
\z

Examples (\ref{ex:7.69}–\ref{ex:7.71}) show, that the use of the perfect aspect\is{Aspect!perfect} with a present sense is not limited to prototypical\is{Prototype} stative verbs\is{Verb!stative}. Verbs like \textit{{\ꞌ}ite}, \textit{ŋaro{\ꞌ}a} and \textit{haŋa} are clearly active: they are transitive\is{Verb!transitive} verbs, the subject of which can be marked with the agent marker \textit{e} (\sectref{sec:8.3.1.2}), yet they tend to have the perfect aspect\is{Aspect!perfect} marker\is{Aspect marker}.

\subsubsection{\textit{Ko V era {\ꞌ}ā}: ‘well and truly finished’}\label{sec:7.2.7.3}
\is{ko V {\ꞌ}ā (perfect aspect)}
The verb phrase marked by \textit{ko V {\ꞌ}ā}\is{ko V {\ꞌ}ā (perfect aspect)} may contain the demonstrative particle \textit{era}. As discussed in \sectref{sec:7.6.4}, this particle indicates spatial or temporal distance. When used in a perfect aspect\is{Aspect!perfect} clause, \textit{era} underlines the temporal and conceptual distance between the time of reference and the time at which the event took place: the action is well and truly finished, possibly a considerably time ago.\footnote{\label{fn:335}This does not mean that \textit{ko V era {\ꞌ}ā}\is{ko V {\ꞌ}ā (perfect aspect)} indicates a pluperfect\is{Pluperfect}, though it can be used in pluperfect sense (\sectref{sec:7.2.7.1} above gives examples where \textit{ko V {\ꞌ}ā} marks pluperfect events).} Often ‘already’ is an appropriate translation.

\ea\label{ex:7.72}
\gll ...he haro mai te kahi, he e{\ꞌ}a ki ruŋa, \textbf{ku} \textbf{mate} \textbf{era} \textbf{{\ꞌ}ā}. \\
~~~\textsc{ntr} pull hither \textsc{art} tuna \textsc{ntr} go\_out to above \textsc{prf} die \textsc{dist} \textsc{cont} \\

\glt 
‘...he pulled up the tuna, it came up, it had already died.’ \textstyleExampleref{[Ley-6-44.041]}
\z

\ea\label{ex:7.73}
\gll {\ꞌ}I Colombia e ai rō {\ꞌ}ana e tahi motu \textbf{ko} \textbf{e{\ꞌ}a} \textbf{era} \textbf{{\ꞌ}ā}  {\ꞌ}i te matahiti 1991 te rāua rei o ruŋa i te inmigración.\\
at Colombia \textsc{ipfv} exist \textsc{emph} \textsc{cont} \textsc{num} one island \textsc{prf} go\_out \textsc{dist} \textsc{cont}  at \textsc{art} year 1991 \textsc{art} \textsc{3pl} law of above at \textsc{art} immigration\\

\glt 
‘In Colombia there is an island where a law on immigration came out in 1991 already.’ \textstyleExampleref{[R649.231]} 
\z

\subsubsection[Perfect ko V {\ꞌ}ā versus perfective i]{Perfect \textit{ko V {\ꞌ}ā} versus perfective \textit{i}}\label{sec:7.2.7.4}

As discussed in \sectref{sec:7.2.7.1} above, \textit{ko V {\ꞌ}ā} marks anterior events leading to a present state. Now it is worthwile to compare the use of perfect \textit{ko V {\ꞌ}ā} and perfective \textit{i}. Both are used to mark events in the (recent) past; to repeat two examples with \textit{i} from \sectref{sec:7.2.4.1}:

%74-75 below were previously numbered 14-15. This means that from here on till the end of this chapter, the numbering is two higher than in the Word version (and also two higher than in the hard references!).

\ea\label{ex:7.74}
\gll A au \textbf{i} \textbf{oho} \textbf{mai} \textbf{nei} ki a koe mo noho ō{\ꞌ}oku {\ꞌ}i nei. \\
\textsc{prop} \textsc{1sg} \textsc{pfv} go hither \textsc{prox} to \textsc{prop} \textsc{2sg} for stay \textsc{poss.1sg.o} at \textsc{prox} \\

\glt 
‘I have come to you to live here.’ \textstyleExampleref{[R245.072]} 
\z

\ea\label{ex:7.75}
\gll Ko koe \textbf{i} \textbf{rē}. \\
\textsc{prom} \textsc{2sg} \textsc{pfv} win \\

\glt
‘You have won.’ \textstyleExampleref{[R210.071]} 
\z

These examples illustrate a typical use of \textit{i}: in many cases, \textit{i}{}-marked clauses express not just a past event, but an event which has a bearing on the present: the event has led to a state which is relevant right now. For example, in \REF{ex:7.74}, the subject has just arrived, leading to a situation in the present; ‘I came’ results in ‘I am here now’. And in \REF{ex:7.75}, ‘you won’ means as much as ‘OK, I give in, you win’, i.e. it describes a current situation, not just something which happened in the past. In other words, \textit{i} is used in situations which seem to be similar to (\ref{ex:7.56}–\ref{ex:7.58}) in the previous section, where perfect aspect \textit{ko V {\ꞌ}ā} is used. 

Now there is considerable variation between languages in the extent to which the perfect is used (\citealt[52–56]{Comrie1976}). Examples such as the ones above suggest that in Rapa Nui the perfect aspect is not used in all cases where a past event has resulted in a current state of affairs. A tentative explanation is, that \textit{ko V {\ꞌ}ā}\is{ko V {\ꞌ}ā (perfect aspect)} is used when the emphasis is on the current \textsc{state} resulting from the event, while \textit{i} is used whenever the emphasis is on the \textsc{event} itself. 

In this respect it is telling that the \textit{i-}marked verb is often preceded by a subject\is{Subject!preverbal} (as in (\ref{ex:7.74}–\ref{ex:7.75})), while \textit{ko V {\ꞌ}ā}\is{ko V {\ꞌ}ā (perfect aspect)} with event/action verbs\footnote{\label{fn:336}With stative verbs\is{Verb!stative}, \textit{ko V {\ꞌ}ā}\is{ko V {\ꞌ}ā (perfect aspect)} does occur with preposed subjects. Using \textit{i} with these verbs would rule out a stative interpretation. \textit{Ko V {\ꞌ}ā} also occurs with preverbal subjects\is{Subject!preverbal} after the deictic particle \textit{{\ꞌ}ī} (\sectref{sec:4.5.4.1.1}).} either has a subject after the verb or no subject at all; only very rarely is \textit{ko V {\ꞌ}ā}\is{ko V {\ꞌ}ā (perfect aspect)} preceded by a subject. As the default constituent order in Rapa Nui is verb—subject, initial subjects are more prominent than subjects following the verb. If \textit{ko V {\ꞌ}ā}\is{ko V {\ꞌ}ā (perfect aspect)} is more state-oriented while \textit{i} is more event-oriented, it is not unexpected that the agent of an \textit{i}{}-marked verb tends to be more prominent than the agent of a \textit{ko V {\ꞌ}ā}\is{ko V {\ꞌ}ā (perfect aspect)} marked verb.

\subsubsection{Summary}\label{sec:7.2.7.5}

\textit{Ko} (var. \textit{ku}) is always accompanied by postverbal \textit{{\ꞌ}ā} (var. \textit{{\ꞌ}ana}). \textit{Ko V {\ꞌ}ā} marks perfect aspect: it indicates a situation holding at the time of reference, which has come about in some way. A comparison with \textit{i}{}-marked verbs shows, that \textit{ko V {\ꞌ}ā} is state-oriented, while \textit{i} is event-oriented. 

This is confirmed by the fact that \textit{ko V {\ꞌ}ā} is used with a wide range of verbs which can be characterised as stative.
\is{ko V {\ꞌ}ā (perfect aspect)|)} \is{a (postverbal)@{\ꞌ}ā (postverbal)|)}
\subsection{Aspectuals and constituent order}\label{sec:7.2.8}

There is a correlation between the use of aspectuals and constituent order. As a general rule, when the clause contains a preverbal constituent, the range of aspectuals tends to be limited to perfective \textit{i} and imperfective \textit{e}: \textit{i} is used with past reference; \textit{e} (followed by a postverbal demonstrative (PVD) after the verb) is used when the reference is non-past. The other aspectuals (\textit{he}, \textit{ka} and \textit{ko V {\ꞌ}ā}) are uncommon.\footnote{\label{fn:337}See Footnote \ref{fn:418} on p.~\pageref{fn:418} about other phenomena affecting clauses with preverbal constituents.} 

This tendency is very strong with preverbal non-subjects (1–5 below); in some constructions (such as content questions and the actor-emphatic) it is even an absolute rule. It is less strong with preverbal subjects (6 below).

%\setcounter{listWWviiiNumxlviileveli}{0}
\begin{enumerate}
\item 
Initial locative phrases; even when the clause expresses an event which is part of the main story line, \textit{i} is used rather than \textit{he}.\footnote{\label{fn:338}Temporal phrases, on the other hand, are commonly followed by \textit{he}:
\ea
\gll 
{\ꞌ}I te rua ra{\ꞌ}ā \textbf{he} u{\ꞌ}i atu te hānau momoko...\\
   at \textsc{art} two day \textsc{ntr} look away \textsc{art} race slender \\
   \glt 
  ‘The next day, the ‘slender race’ saw...’ \textstyleExampleref{[Ley-3-06.028]}  \z }
\end{enumerate}

\ea\label{ex:7.76}
\gll {\ob}Mai Haŋa Roa\,{\cb} i iri ai ki {\ꞌ}Ōroŋo. \\
{\db}from Hanga Roa \textsc{pfv} ascend \textsc{pfv} to Orongo \\

\glt
‘From Hanga Roa he went up to Orongo.’ \textstyleExampleref{[Ley-2-02.054]}
\z

\begin{enumerate}
\setcounter{enumi}{1} 
\item 
Noun phrases containing a numeral have a strong tendency to be sentence-initial, regardless their semantic relation to the verb; for example, they may be subject as in \REF{ex:7.77}, or adjunct as in \REF{ex:7.78}. After such a preverbal constituent the verb tends to be marked with \textit{i} or \textit{e}. 

\end{enumerate}

\ea\label{ex:7.77}
\gll {\ob}E rua nō hānau {\ꞌ}e{\ꞌ}epe taŋata\,{\cb} i rere mai. \\
{\db}\textsc{num} two just race corpulent man \textsc{pfv} fly hither \\

\glt 
‘Only two men from the ‘corpulent race’ jumped.’ \textstyleExampleref{[Mtx-3-02.038]}
\z

\ea\label{ex:7.78}
\gll {\ob}E tahi mahana ta{\ꞌ}ato{\ꞌ}a\,{\cb} i aŋa ai mo {\ꞌ}auhau o tū ūtu{\ꞌ}a  era ō{\ꞌ}ona.\\
{\db}\textsc{num} one day all \textsc{pfv} work \textsc{pvp} for pay of \textsc{dem} punishment  \textsc{dist} \textsc{poss.3sg.o}\\

\glt
‘One whole day he worked to pay his punishment.’ \textstyleExampleref{[R250.026]} 
\z

\begin{enumerate}
\setcounter{enumi}{2} 
\item 
After adverbial clause connectors like \textit{{\ꞌ}o ira} ‘therefore’,\footnote{\label{fn:339}\textit{{\ꞌ}o ira} is sometimes followed by \textit{he}, but other clause connectors are not.} \textit{pē nei} ‘like this’ and \textit{pē ira} ‘like that’: 

\end{enumerate}

\ea\label{ex:7.79}
\gll {\ob}{\ꞌ}O ira\,{\cb} i kī ai ko Ŋā Ihu More {\ꞌ}a Pua Katike.  \\
{\db}because\_of \textsc{ana} \textsc{pfv} say \textsc{pvp} \textsc{prom} Ŋā Ihu More {\ꞌ}a Pua Katike  \\

\glt
‘Therefore they were called Ŋā Ihu More {\ꞌ}a Pua Katike’ \textstyleExampleref{[R310.253]} 
\z

\begin{enumerate}
\setcounter{enumi}{3} 
\item 
After question words\is{Question} like \textit{{\ꞌ}a {\ꞌ}ai} ‘who’ and \textit{he aha} ‘what, why’: 

\end{enumerate}

\ea\label{ex:7.80}
\gll ¿{\ob}{\ꞌ}A {\ꞌ}ai\,{\cb} rā ia i u{\ꞌ}i haka{\ꞌ}ou rō atu? \\
~~~of\textsc{.a} who \textsc{intens} then \textsc{pfv} look again \textsc{emph} away \\

\glt 
‘Who would have seen them again?’ \textstyleExampleref{[R361.019]} 
\z

\ea\label{ex:7.81}
\gll ¿{\ob}He aha\,{\cb} koe e taŋi ena?\\
~~~\textsc{pred} what \textsc{2sg} \textsc{ipfv} cry \textsc{med}\\

\glt
‘Why are you crying?’ \textstyleExampleref{[Mtx-7-12.024]}
\z

\begin{enumerate}
\setcounter{enumi}{4} 
\item 
In the actor-emphatic construction, in which the verb is preceded by a possessive expressing the Agent (\sectref{sec:8.6.3}):

\end{enumerate}

\ea\label{ex:7.82}
\gll {\ob}O tō{\ꞌ}ona matu{\ꞌ}a\,{\cb} i aŋa i te hare nei mo Puakiva. \\
{\db}of \textsc{poss.3sg.o} parent \textsc{pfv} make \textsc{acc} \textsc{art} house \textsc{prox} for Puakiva \\

\glt 
‘It was her father who made this house for Puakiva.’ \textstyleExampleref{[R229.269]} 
\z

\ea\label{ex:7.83}
\gll {\ob}Mā{\ꞌ}au\,{\cb} e māuruuru ki a Pea hai {\ꞌ}īŋoa ō{\ꞌ}oku. \\
{\db}\textsc{ben.1sg.a} \textsc{ipfv} thank to \textsc{prop} Pea with name \textsc{poss.1sg.o} \\

\glt
‘You will thank Pea in my name.’ \textstyleExampleref{[R229.086]} 
\z

\begin{enumerate}
\setcounter{enumi}{5} 
\item 
Preverbal subjects\is{Subject!preverbal} show a certain tendency to be followed by \textit{i} or \textit{e}:

\end{enumerate}

\ea\label{ex:7.84}
\gll {\ob}A au\,{\cb} i oho mai nei ki a koe mo noho ō{\ꞌ}oku {\ꞌ}i nei. \\
{\db}\textsc{prop} \textsc{1sg} \textsc{pfv} go hither \textsc{prox} to \textsc{prop} \textsc{2sg} for stay \textsc{poss.1sg.o} at \textsc{prox} \\

\glt
‘I have come to you to live here.’ \textstyleExampleref{[R245.072]} 
\z

However, preverbal subjects followed by \textit{he} are by no means uncommon. For examples, see \REF{ex:8.69} and \REF{ex:7.71} on p.~\pageref{ex:8.69}.

The only preverbal constituent which does not show a correlation with \textit{i} and \textit{e}, is the negator \textit{{\ꞌ}ina}; as shown in \sectref{sec:10.5.1}, the verb in a clause negated with \textit{{\ꞌ}ina} is usually marked with the neutral aspectual \textit{he}.
\is{Aspect marker|)}
\section{Preverbal particles}\label{sec:7.3}
\subsection{\textit{Rava} ‘given to’}\label{sec:7.3.1}
\is{rava ‘usually’|(}
\textit{Rava}\footnote{\label{fn:340}{\textless} \is{Proto-Polynesian}PPN \textit{*lawa} ‘sufficient, abundant, completed’; cognates in other languages are used as predicate, not as premodifier. Some languages have a postmodifier {\textless} PNP \textit{*lawa}, which has an intensifying sense ‘very, completely’.} always precedes the verb. It indicates either that the action is performed on a regular basis, or that the subject is inclined to perform the action. \textit{Rava} has a variant \textit{vara}; there is little – if any – difference between the two.

\textit{Rava} may occur in a verb phrase which serves as clause predicate:

\ea\label{ex:7.85}
\gll ¿{\ꞌ}Ina {\ꞌ}ō te hoko toru era e \textbf{rava} \textbf{e{\ꞌ}a} era ananake? \\
~~\textsc{neg} really \textsc{art} \textsc{num.pers} three \textsc{dist} \textsc{ipfv} given\_to go\_out \textsc{dist} together \\

\glt
‘Don’t those three always go out together?’ \textstyleExampleref{[R366.044]} 
\z

However, this is not very common: usually \textit{rava} + verb occurs after a noun, in a bare relative clause\is{Clause!relative!bare}. In these constructions, \textit{rava} + V indicates an action which is not performed at a certain point in time, but which characterises the preceding noun. The expression has therefore a relatively time-stable character. A few examples:\footnote{\label{fn:341}In \REF{ex:7.88}, the noun is implied: ‘(the ones) given to sleeping’.}

\ea\label{ex:7.86}
\gll He tu{\ꞌ}u mai te pahī \textbf{rava} ma{\ꞌ}u mai i te me{\ꞌ}e mo roto  i te hare toa.\\
\textsc{ntr} arrive hither \textsc{art} ship given\_to carry hither \textsc{acc} \textsc{art} thing for inside  at \textsc{art} hourse store\\

\glt 
‘The ship arrived which used to bring things for the store.’ \textstyleExampleref{[R250.094]} 
\z

\ea\label{ex:7.87}
\gll Te me{\ꞌ}e \textbf{rava} \textbf{aŋa} o tātou {\ꞌ}i rā mahana he porotē. \\
\textsc{art} thing given\_to do of \textsc{1pl.incl} at \textsc{dist} day \textsc{pred} parade \\

\glt 
‘What we always do on that day (=18 September, the national holiday) is parading.’ \textstyleExampleref{[R334.309]} 
\z

\ea\label{ex:7.88}
\gll ¡Ka {\ꞌ}ara, \textbf{rava} \textbf{ha{\ꞌ}uru} kē, kōrua! \\
~\textsc{imp} wake\_up given\_to sleep different \textsc{2pl} \\

\glt 
‘Wake up, you sleepyheads!’ \textstyleExampleref{[Ley-4-05.008]}
\z
\is{rava ‘usually’|)}

\subsection{Degree modifiers} \label{sec:7.3.2}

\textit{{\ꞌ}Apa} and \textit{{\ꞌ}ata} are degree modifiers, which precede the verb root. 

\subparagraph{\ref{sec:7.3.2}.1~ \textit{{\ꞌ}Apa}} \textit{{\ꞌ}Apa}\is{apa ‘part’@{\ꞌ}apa ‘part’} (which is also a noun meaning ‘part, portion, piece’) indicates a moderate degree: ‘somewhat, kind of’.\footnote{\label{fn:342}\textit{{\ꞌ}Apa} may be borrowed from \ili{Tahitian}\is{Tahitian influence} \textit{{\ꞌ}apa} ‘half of a fish or animal, cut lengthwise’ (\ili{Pa’umotu} \textit{kapa}). \citet[315]{Fischer2001Hispan} suggests it was borrowed from \ili{Tahitian}\is{Tahitian influence} \textit{{\ꞌ}afa} ‘half’ (which was itself borrowed from \ili{English}).} It is often used with stative predicates, but found with actions as well. 

\ea\label{ex:7.89}
\gll Ko \textbf{{\ꞌ}apa} \textbf{ora} {\ꞌ}iti {\ꞌ}ā a au. \\
\textsc{prf} part live little \textsc{cont} \textsc{prop} \textsc{1sg} \\

\glt 
‘I am somewhat recovered.’ \textstyleExampleref{[R231.325]} 
\z

\ea\label{ex:7.90}
\gll Te ti{\ꞌ}ara{\ꞌ}a nei he ‘r’ e \textbf{{\ꞌ}apa} \textbf{huru} \textbf{kē} rō {\ꞌ}ā te kī iŋa  {\ꞌ}i te ŋā {\ꞌ}arero nei ararua.\\
\textsc{art} letter \textsc{prox} \textsc{pred} r \textsc{ipfv} part manner different \textsc{emph} \textsc{cont} \textsc{art} say \textsc{nmlz}  at \textsc{art} \textsc{pl} tongue \textsc{prox} the\_two\\

\glt
‘This letter ‘r’, its pronunciation is a little different in these two languages.’ \textstyleExampleref{[R616.145]} 
\z

In \REF{ex:7.91}, \textit{{\ꞌ}apa} semantically quantifies the object: ‘we somewhat obtained X’ = ‘we obtained a few X’ (cf. (\ref{ex:7.95}–\ref{ex:7.96}) for a similar use of \textit{{\ꞌ}ata}).

\ea\label{ex:7.91}
\gll Ko \textbf{{\ꞌ}apa} rova{\ꞌ}a mai {\ꞌ}ā te me{\ꞌ}e pāreherehe matā. \\
\textsc{prf} part obtain hither \textsc{cont} \textsc{art} thing piece:\textsc{red} obsidian \\

\glt 
‘We obtained a few pieces of obsidian.’ \textstyleExampleref{[R629.030]} 
\z

\subparagraph{\ref{sec:7.3.2}.2~ \textit{{\ꞌ}Ata}} \textit{{\ꞌ}Ata} indicates a high degree, either comparative\is{Comparative} (‘more’), superlative\is{Superlative} (‘most’) or absolute (‘very; thoroughly’).\footnote{\label{fn:343}Cognates occur in several \is{Eastern Polynesian}EP languages. These are preverbal as in Rapa Nui, but only have an absolute sense: ‘carefully, slowly’ (Pollex, see \citealt{GreenhillClark2011}; \citealt[74]{ElbertPukui1979} for \ili{Hawaiian} \textit{aka}, \citealt[92]{Bauer1993} for \ili{Māori} \textit{aata}). Possibly it occurs in SO languages as well: \citet[188]{Besnier2000} mentions a preverbal particle \textit{ata} ‘properly, in moderation’ in \ili{Tuvaluan}, though only one example is provided, where it is part of an idiom.} It is used in comparative constructions with adjectives (\sectref{sec:3.5.1.1}); with event verbs it is also used in a comparative sense, comparing the intensity of the event to a previous situation: ‘more than before’.

\ea\label{ex:7.92}
\gll He \textbf{{\ꞌ}ata} \textbf{taŋi} a Puakiva. \\
\textsc{ntr} more cry \textsc{prop} Puakiva \\

\glt 
‘Puakiva cried even more (than before).’ \textstyleExampleref{[R229.183]} 
\z

\ea\label{ex:7.93}
\gll Ka \textbf{{\ꞌ}ata} \textbf{hāpī}, {\ꞌ}ina ko hakarē. \\
\textsc{imp} more learn \textsc{neg} \textsc{neg.ipfv} leave \\

\glt
‘Learn more, don’t neglect it.’ \textstyleExampleref{[R242.093]} 
\z

The comparison may also be with respect to another entity as standard of comparison, though this rarely happens. Below is an example; as with adjectives, the standard of comparison is expressed with \textit{ki}:

\ea\label{ex:7.94}
\gll ¡E Māria, \textbf{{\ꞌ}ata} ha{\ꞌ}amaitai koe e te {\ꞌ}Atua \textbf{ki} te ta{\ꞌ}ato{\ꞌ}a ŋā vi{\ꞌ}e. \\
~\textsc{voc} Mary more bless \textsc{2sg} \textsc{ag} \textsc{art} God to \textsc{art} all \textsc{pl} woman \\

\glt
‘Mary, you are more blessed by God than all women.’ \textstyleExampleref{[Luke 1:42]}
\z

One could wonder if the verb has been adjectivised here; notice however that the Agent phrase \textit{e te {\ꞌ}Atua} is introduced by \textit{e}, which suggests that \textit{ha{\ꞌ}amaitai} retains its status of an agentive verb.

With transitive\is{Verb!transitive} verbs, \textit{{\ꞌ}ata} may indicate a multiplication of the object. For example, in \REF{ex:7.95} \textit{{\ꞌ}ata} semantically modifies \textit{tāropa}: ‘more baskets’.\footnote{\label{fn:344}Cf. the use of \textit{tahi} and \textit{rahi} in the verb phrase, sec. \sectref{sec:4.4.9} and \sectref{sec:4.4.7.2}.}

\ea\label{ex:7.95}
\gll E \textbf{{\ꞌ}ata} \textbf{ma{\ꞌ}u} te tāropa ana oho koe. \\
\textsc{exh} more carry \textsc{art} basket \textsc{irr} go \textsc{2sg} \\

\glt 
‘Take more baskets when you go.’ \textstyleExampleref{[MsE-064.013]}
\z

\ea\label{ex:7.96}
\gll ¡{\ꞌ}Ēē, ka \textbf{{\ꞌ}ata} \textbf{ao} mai ki a au! \\
~yes \textsc{imp} more serve\_food hither to \textsc{prop} \textsc{1sg} \\

\glt
‘Yes please, serve me some more.’ \textstyleExampleref{[R535.098]} 
\z

\textit{{\ꞌ}Ata} may also be used in a superlative\is{Superlative} sense:

\ea\label{ex:7.97}
\gll Te artículo \textbf{{\ꞌ}ata} pāpa{\ꞌ}i o tātou he ‘he’ {\ꞌ}e he ‘te’. \\
\textsc{art} article more write of \textsc{1pl.incl} \textsc{pred} \textit{~he} and \textsc{pred} \textit{~te} \\

\glt 
‘The articles we write most, are \textit{he} and \textit{te}.’ \textstyleExampleref{[R616.719]} 
\z

\subparagraph{\ref{sec:7.3.2}.3~ Placement} The exact position of \textit{{\ꞌ}ata}\is{ata ‘more’@{\ꞌ}ata ‘more’} and \textit{{\ꞌ}apa}\is{apa ‘part’@{\ꞌ}apa ‘part’} in relation to other preverbal elements varies, depending on their respective scope.

With causative\is{Causative} verbs, the degree modifier usually occurs before the causative\is{Causative} prefix \textit{haka}, as in \REF{ex:7.98}: ‘more [cause to be strong]’. However, it may also occur after \textit{haka}, in which case \textit{haka} has scope over the degree modifier. This is illustrated in \REF{ex:7.99}: ‘cause to be [more intelligent]’.

\ea\label{ex:7.98}
\gll Ko \textbf{{\ꞌ}ata} \textbf{haka} \textbf{pūai} {\ꞌ}ana te re{\ꞌ}o o Roŋotakahiu e pāta{\ꞌ}uta{\ꞌ}u era. \\
\textsc{prf} more \textsc{caus} strong \textsc{cont} \textsc{art} voice of Rongotakahiu \textsc{ipfv} recite \textsc{dist} \\

\glt 
‘Rongotakahiu sang louder (lit. strengthened his voice more when singing).’ \textstyleExampleref{[R476.014]} 
\z

\ea\label{ex:7.99}
\gll ...te tire e haŋa rō {\ꞌ}ā mo \textbf{haka} \textbf{{\ꞌ}ata} \textbf{māramarama} i a rāua.\\
~~~\textsc{art} Chilean \textsc{ipfv} want \textsc{emph} \textsc{cont} for \textsc{caus} more intelligent \textsc{acc} \textsc{prop} \textsc{3pl}\\

\glt
‘...Chileans who want to pass themself off as smarter (lit. to cause them to be smarter).’ \textstyleExampleref{[R428.006]} 
\z

With the constituent negator \textit{ta{\ꞌ}e}\is{tae (negator)@ta{\ꞌ}e (negator)}, either the negator or the degree particle may come first. In \REF{ex:7.100} the negator comes first and has scope over \textit{{\ꞌ}ata}\is{ata ‘more’@{\ꞌ}ata ‘more’}: ‘not [more high]’. In \REF{ex:7.101}, \textit{{\ꞌ}apa}\is{apa ‘part’@{\ꞌ}apa ‘part’} has scope over the negation: ‘somewhat [not listening]’.

\ea\label{ex:7.100}
\gll Te tāvini \textbf{ta{\ꞌ}e} \textbf{{\ꞌ}ata} \textbf{hau} ki te taŋata haka aŋa i a ia. \\
\textsc{art} servant \textsc{conneg} more exceed to \textsc{art} man \textsc{caus} work \textsc{acc} \textsc{prop} \textsc{3sg} \\

\glt 
‘A servant is not higher than his master (lit. the man who makes him work).’ \textstyleExampleref{[Mat. 10:24]}
\z

\ea\label{ex:7.101}
\gll Te māmoe nei māmoe vara kori {\ꞌ}e \textbf{{\ꞌ}apa} \textbf{ta{\ꞌ}e} \textbf{hakaroŋo}. \\
\textsc{art} sheep \textsc{prox} sheep usually play and part \textsc{conneg} listen \\

\glt 
‘This lamb used to play and was somewhat disobedient.’ \textstyleExampleref{[R536.009]} 
\z

\section{Evaluative markers}\label{sec:7.4}
\is{Evaluative marker}
The evaluative markers \textit{nō} and \textit{rō} occur in the same position in the verb phrase; they are mutually exclusive. 

\subsection{The limitative marker \textit{nō}}\label{sec:7.4.1}
\is{nō ‘just’|(}
\textit{Nō} originates in \is{Proto-Polynesian}PPN \textit{*noa}, which occurs as a postverbal marker in a number of languages throughout Polynesia. Rapa Nui is the only language in which the vowel sequence \textit{oa} assimilated to \textit{ō}, apart from \ili{Hawaiian} (\citealt[100]{ElbertPukui1979}).

\textit{Nō} is a limitative marker; its basic sense is ‘nothing else’.\footnote{\label{fn:345}Cf. \citet[146]{LazardPeltzer2000} on \ili{Tahitian} \textit{noa}: their basic gloss is ‘ne faire que’, from which they derive the different uses of \textit{noa}, which largely parallel those of Rapa Nui \textit{nō}.} The particle has several uses, which can all be related to this basic sense: ‘simply, just’ (nothing more), ‘still’ (a lack of change), ‘even so, yet’ (something happens, despite expectations to the contrary).

In this section, the use of \textit{nō} in the verb phrase is discussed. \textit{Nō} also occurs after other parts of speech, which are discussed elsewhere: nouns (\sectref{sec:5.8.2}), numerals (\sectref{sec:4.3.2.4}) and quantifiers\is{Quantifier} (\sectref{sec:4.4.10}).

\textit{Nō} may indicate that something just happens, without anything more. The implication is that something else or something more could happen, but does not actually happen. The context tells what this ‘something else’ would be:

\ea\label{ex:7.102}
\gll {\ꞌ}Ina a Tiare kai mate; \textbf{ko} \textbf{rerehu} \textbf{nō} \textbf{{\ꞌ}ā}. \\
\textsc{neg} \textsc{prop} Tiare \textsc{neg.pfv} die \textsc{prf} faint just \textsc{cont} \\

\glt 
‘Tiare was not dead; she had just fainted.’ \textstyleExampleref{[R481.086]} 
\z

\ea\label{ex:7.103}
\gll Mā{\ꞌ}aku {\ꞌ}ā e aŋa tahi; \textbf{ka} \textbf{oho} \textbf{nō} kōrua. \\
\textsc{ben.1sg.a} \textsc{ident} \textsc{ipfv} do all \textsc{imp} go just \textsc{2pl} \\

\glt
‘I myself will do everything; you guys just go.’ \textstyleExampleref{[R236.010]} 
\z

\textit{Nō} in this sense ‘just’ may have the connotation ‘without further ado, without thinking, without taking other considerations into account’.

\ea\label{ex:7.104}
\gll ¿Kai ha{\ꞌ}amā koe \textbf{i} \textbf{to{\ꞌ}o} \textbf{nō} koe i te mauku mo ta{\ꞌ}o i ta{\ꞌ}a {\ꞌ}umu?\\
~\textsc{neg.pfv} ashamed \textsc{2sg} \textsc{pfv} take just \textsc{2sg} \textsc{acc} \textsc{art} grass for cook  \textsc{acc} \textsc{poss.2sg.a} earth\_oven\\

\glt 
‘Weren’t you ashamed, that you just took the grass to (as fuel) to cook your earth oven (without asking, even though the grass was mine)?’ \textstyleExampleref{[R231.186]} 
\z

\ea\label{ex:7.105}
\gll Te me{\ꞌ}e nō, \textbf{ku} \textbf{oho} \textbf{nō} \textbf{{\ꞌ}ā} ki tai hī. \\
\textsc{art} thing just \textsc{prf} go just \textsc{cont} to sea to\_fish \\

\glt 
‘(Nowadays people don’t consider the moon and the wind.) On the contrary, they just go out to sea to fish.’ \textstyleExampleref{[R354.026]} 
\z

In the previous examples, a contrast is implied between what happens and what could have happened. Sometimes this sense of contrast is more prominent; the clause has a connotation of counterexpectation\is{Counterexpectation}: ‘even so, no matter, still’.\footnote{\label{fn:346}A contrastive sense of \textit{nō} is also found in expressions like \textit{Te me{\ꞌ}e nō} ‘however’ (see \REF{ex:5.153} on p.~\pageref{ex:5.153}) and in the conjunction\is{Conjunction} \textit{nōatu} (\sectref{sec:11.6.7}).}

\ea\label{ex:7.106}
\gll ...\textbf{e} \textbf{māuiui} \textbf{nō} \textbf{{\ꞌ}ana} te ŋā poki. \\
~~~\textsc{ipfv} sick just \textsc{cont} \textsc{art} \textsc{pl} child \\

\glt 
‘(Nowadays there are all kinds of things to take care of children,) but even so, children get sick.’ \textstyleExampleref{[R380.138]} 
\z

\ea\label{ex:7.107}
\gll Ka rahi atu tā{\ꞌ}aku poki, \textbf{e} \textbf{hāpa{\ꞌ}o} \textbf{nō} e au {\ꞌ}ā.\\
\textsc{cntg} many away \textsc{poss.1sg.a} child \textsc{ipfv} care\_for just \textsc{ag} \textsc{1sg} \textsc{ident}\\

\glt 
‘Even if I have many children, I will still take care of them myself.’ \textstyleExampleref{[R229.023]} 
\z

\textit{Nō} may be used in a continuous\is{Aspect!continuous} clause, emphasising that the action is still going on. In this sense, it is often used with the imperfective \textit{e}. 

\ea\label{ex:7.108}
\gll He u{\ꞌ}i i a Vaha, \textbf{e} \textbf{oho} \textbf{nō} mai era, \textbf{e} \textbf{{\ꞌ}amo} \textbf{nō} mai era i te poki tiŋa{\ꞌ}i era.\\
\textsc{ntr} look \textsc{acc} \textsc{prop} Vaha \textsc{ipfv} go just hither \textsc{dist} \textsc{ipfv} carry just hither \textsc{dist} \textsc{acc} \textsc{art} child kill \textsc{dist}\\

\glt 
‘He saw Vaha, who was still going and carrying the killed child.’ \textstyleExampleref{[Mtx-3-01.144]}
\z

\ea\label{ex:7.109}
\gll {\ꞌ}I te pō{\ꞌ}ā e oho era ki tā{\ꞌ}ana aŋa \textbf{e} \textbf{ha{\ꞌ}uru} \textbf{nō} {\ꞌ}ā a Eva. \\
at \textsc{art} morning \textsc{ipfv} go \textsc{dist} to \textsc{poss.3sg.a} work \textsc{ipfv} sleep just \textsc{cont} \textsc{prop} Eva \\

\glt
‘In the morning he went to his work, when Eva was still sleeping.’ \textstyleExampleref{[R210.025]} 
\z

An action marked with \textit{nō} is often unremarkable, routine, expected: something is simply going on, nothing significant has happened (yet). Often, the verb phrase expresses a lack of change with respect to a previous situation: the same thing described earlier is still going on. In this sense, \textit{nō} is common in progressive\is{Progressive} cohesive clauses\is{Clause!cohesive} (see \REF{ex:11.213} on p.~\pageref{ex:11.213}):

\ea\label{ex:7.110}
\gll \textbf{E} \textbf{iri} \textbf{nō} \textbf{{\ꞌ}ā} he take{\ꞌ}a e Te Manu e tahi hōŋa{\ꞌ}a māmari.\\
\textsc{ipfv} ascend just \textsc{cont} \textsc{ntr} see \textsc{ag} Te Manu \textsc{num} one nest eggs\\

\glt
‘(The two went up and looked for eggs...) While they were still going up, Manu saw a nest with eggs.’ \textstyleExampleref{[R245.202–203]}
\z

As discussed in \sectref{sec:5.8.2}, \textit{nō} in the noun phrase often serves to limit the reference of a noun phrase. Occasionally, \textit{nō} in the verb phrase has the same effect.\footnote{\label{fn:347}Cf. the use of \textit{tahi} ‘all’, which occurs in the verb phrase but determines the reference of a noun phrase in the clause (\sectref{sec:4.4.9}).} In \REF{ex:7.111}, \textit{nō} occurs after the (nominalised) verb \textit{kai}, signalling that the object noun phrase has limited reference.

\ea\label{ex:7.111}
\gll Ko ha{\ꞌ}umani {\ꞌ}ana {\ꞌ}i te \textbf{kai} \textbf{iŋa} \textbf{nō} i te moa. \\
\textsc{prf} bored \textsc{cont} at \textsc{art} eat \textsc{nmlz} just \textsc{acc} \textsc{art} chicken \\

\glt 
‘I’m tired of eating only chicken.’ \textstyleExampleref{[R229.123]} 
\z

\largerpage
After certain adjectival predicates, \textit{nō} signals that the object described has only the property in question, implicitly excluding other properties: ‘just, altogether’. So while being fundamentally limitative in nature, \textit{nō} in these cases underlines and emphasises the property expressed by the adjective: the object is entirely characterised by this property, to the exclusion of anything else. This use is only found with adjectives expressing a positive evaluation, like \textit{rivariva} ‘good’, \textit{nene} ‘sweet, delicious’, \textit{tau} ‘pretty’. The adjective is preceded by the aspectual \textit{he}.

\ea\label{ex:7.112}
\gll {\ꞌ}Ina he maŋeo, he \textbf{nene} \textbf{nō}. \\
\textsc{neg} \textsc{ntr} sour \textsc{ntr} sweet just \\

\glt 
‘(The orange) was not sour, just sweet.’ \textstyleExampleref{[Egt-02.135]}
\z

\ea\label{ex:7.113}
\gll Te pahī nei, he \textbf{nehenehe} \textbf{nō}. \\
\textsc{art} ship \textsc{prox} \textsc{ntr} beautiful just \\

\glt
‘This ship was just beautiful.’ \textstyleExampleref{[R239.022]} 
\z

Notice that \ili{English} ‘just’ can be used in the same way, as the translation of \REF{ex:7.113} shows.
\is{nō ‘just’|)}

\subsection{The asseverative marker \textit{rō}}\label{sec:7.4.2}
\is{ro (emphatic marker)@rō (emphatic marker)|(}
\textit{Rō} is an asseverative particle. It serves to underline the reality of the event and/or its significance in the course of events. (See also \citealt[41]{WeberR2003}.) While \textit{nō} underlines the expected, routine nature of the event (for example, because the situation has not changed), \textit{rō} underlines its significance, newsworthiness. In pragmatic terms: while \textit{nō} indicates a low information load, \textit{rō} indicates a high information load. In view of the diversity of its uses, \textit{rō} is glossed \textsc{emph}(atic).\footnote{\label{fn:348} \citet[37]{DuFeu1996} characterises \textit{rō} as a realis\is{Realis} particle, glossed as [+REA]; contrasted with \textit{rā} [-REA]. She points out that \textit{rā} is for example used in imperatives\is{Imperative}, when the speaker has no control over the outcome; \textit{rō}, on the other hand, is for example used in 1\textsuperscript{st} person imperatives\is{Imperative} (= hortatives) where speaker has greater control over the realisation of the event. While this is only part of the picture, and while \textit{rō} is actually not in a paradigmatic relation with the intensifier \textit{rā} (\sectref{sec:4.5.4.4}), this correctly underlines the asseverative character of \textit{rō}.}

Like \textit{nō}, \textit{rō} is the result of vowel assimilation: it is derived from PNP \textit{*loa}. Unlike \textit{nō}, \textit{rō} is not used in the noun phrase, but it does occur occasionally in numeral phrases (see \REF{ex:4.28} on p.~\pageref{ex:4.28}).

In the verb phrase, \textit{rō} is used in certain well-defined contexts, which are discussed elsewhere in this grammar:

\begin{itemize}
\item 
\textit{he V rō {\ꞌ}ai}\is{he (aspect marker)!he V rō {\ꞌ}ai} (\sectref{sec:7.2.3.3}), a construction which marks pivotal or climactic events in a narrative and events with a certain emotional intensity. 

\item 
\textit{e V rō}, which marks future events\is{e (imperfective)!e V rō} (\sectref{sec:7.2.5.3}). One could say that by using \textit{e V rō}\is{e (imperfective)!e V rō}, the speaker stresses the real, non-hypothetical character of the future event.

\item 
\textit{ka V rō}\is{ka (aspect marker)!ka V rō} (\sectref{sec:11.6.2.5}), a construction indicating the upper limit of an event (‘until’).

\item 
the existential \textit{e ai rō {\ꞌ}ā} (\sectref{sec:9.3.1}), which states the existence of a person of object which is new in the discourse, and therefore carries a high information load.

\item 
after \textit{{\ꞌ}o}\is{o ‘lest’@{\ꞌ}o ‘lest’} ‘lest’ (\sectref{sec:11.5.4}). 

\end{itemize}

But \textit{rō} is not limited to these constructions. Generally speaking, \textit{rō} marks events which are significant in discourse, for example because they are the culmination of a series of events, or because they change the course of events. This happens in the \textit{he V rō {\ꞌ}ai}\is{he (aspect marker)!he V rō {\ꞌ}ai} construction mentioned above; it is also found with \textit{i V ai} (\sectref{sec:7.6.5}). In the following example, Kainga produces a spear point which will play an important role in the events to follow.

\ea\label{ex:7.114}
\gll {\ꞌ}I rā pō {\ꞌ}ā a Kaiŋa... \textbf{i} \textbf{aŋa} \textbf{rō} \textbf{ai} e tahi matā rivariva. \\
at \textsc{dist} night \textsc{ident} \textsc{prop} Kainga \textsc{pfv} make \textsc{emph} \textsc{pvp} \textsc{num} one obsidian good:\textsc{red} \\

\glt
‘In that night Kainga made a good obsidian spearpoint.’ \textstyleExampleref{[R304.015]} 
\z

Events may also be significant by way of contrast:

\ea\label{ex:7.115}
\gll A Pea e ko rivariva mo hāpa{\ꞌ}o i a Puakiva;  \textbf{e} \textbf{oho} \textbf{rō} \textbf{{\ꞌ}ā} ki te aŋa.\\
\textsc{prop} Pea \textsc{ipfv} \textsc{neg.ipfv} good:\textsc{red} for care\_for \textsc{acc} \textsc{prop} Puakiva  \textsc{ipfv} go \textsc{emph} \textsc{cont} to \textsc{art} work\\

\glt
‘Pea was not able to take care of Puakiva; (rather,) he used to go to work.’ \textstyleExampleref{[R229.005]} 
\z

\textit{Rō} may emphasise the reality of a situation: ‘really’.\footnote{\label{fn:349}Examples such as \REF{ex:7.116} might suggest that \textit{rō} means ‘very’. However, \textit{rō} (unlike \ili{Tahitian} \textit{roa}) is not a common way to express a high degree; rather, this is expressed using \textit{hope{\ꞌ}a} ‘last’ or \textit{ri{\ꞌ}ari{\ꞌ}a} ‘terribly’.}

\ea\label{ex:7.116}
\gll Te parauti{\ꞌ}a, \textbf{e} \textbf{haŋa} \textbf{rō} \textbf{{\ꞌ}ā} a au ki a kōrua ko koro. \\
\textsc{art} truth \textsc{ipfv} love \textsc{emph} \textsc{cont} \textsc{prop} \textsc{1sg} to \textsc{prop} \textsc{2pl} \textsc{prom} Dad \\

\glt 
‘The truth is, I do love you and Dad.’ \textstyleExampleref{[R229.498]} 
\z

\ea\label{ex:7.117}
\gll A nua \textbf{e} \textbf{koa} \textbf{rō} \textbf{{\ꞌ}ā} {\ꞌ}i tū poki era {\ꞌ}ā{\ꞌ}ana. \\
\textsc{prop} Mum \textsc{ipfv} happy \textsc{emph} \textsc{cont} at \textsc{dem} child \textsc{dist} \textsc{poss.3sg.a} \\

\glt
‘Mum was really happy with her child.’ \textstyleExampleref{[R250.055]} 
\z

When \textit{rō} emphasises the reality of the clause, there may be a connotation of counterexpectation\is{Counterexpectation}. In \REF{ex:7.118} this happens in a question, in \REF{ex:7.119} as reply to a question.

\ea\label{ex:7.118}
\gll ¿\textbf{E} \textbf{haŋa} \textbf{rō} koe mo oho ki hiva mo hāpī? \\
~\textsc{ipfv} want \textsc{emph} \textsc{2sg} for go to continent for learn \\

\glt 
‘Do you (really) want to go to the continent to study?’ \textstyleExampleref{[R210.010]} 
\z

\ea\label{ex:7.119}
\gll ¿{\ꞌ}Ina {\ꞌ}ō pēaha kai ŋaro{\ꞌ}a e te nu{\ꞌ}u hūrio i te roŋo rivariva o te {\ꞌ}Atua? \textbf{I} \textbf{ŋaro{\ꞌ}a} \textbf{rō}. \\
~~\textsc{neg} really perhaps \textsc{neg.pfv} perceive \textsc{ag} \textsc{art} people Jew \textsc{acc} \textsc{art} message good:\textsc{red} of \textsc{art} God \textsc{pfv} perceive \textsc{emph} \\

\glt 
‘Have the Jews perhaps not heard the good news about God? They have heard it.’ \textstyleExampleref{[Rom. 10:18]}
\z

In several of the examples above, \textit{rō} occurs in the common construction \textit{e V rō}\is{e (imperfective)!e V {\ꞌ}ā} \textit{{\ꞌ}ā}, which expresses an ongoing event or situation (\sectref{sec:7.2.5.4}). In this construction, the asseverative sense of \textit{rō} is not always clear. Sometimes the clause does convey new, unexpected or even surprising information, as in the following example, where the subject does a somewhat unexpected discovery:

\ea\label{ex:7.120}
\gll I {\ꞌ}ara mai ai, {\ꞌ}i rote piha e tahi \textbf{e} \textbf{moe} \textbf{rō} \textbf{{\ꞌ}ā}... \\
\textsc{pfv} wake\_up hither \textsc{pvp} at inside\_\textsc{art} room \textsc{num} one \textsc{ipfv} lie \textsc{emph} \textsc{cont} \\

\glt
‘When she woke up, she was lying in a room...’ \textstyleExampleref{[R210.090]} 
\z

But in other cases the information load of the \textit{e V rō}\is{e (imperfective)!e V {\ꞌ}ā} \textit{{\ꞌ}ā} clause does not seem to be particularly high:

\ea\label{ex:7.121}
\gll Mo u{\ꞌ}i atu o te ŋā poki ki a Taparahi \textbf{e} \textbf{ha{\ꞌ}ere} \textbf{rō} \textbf{{\ꞌ}ā} a te ara he ri{\ꞌ}ari{\ꞌ}a.\\
if look away of \textsc{art} \textsc{pl} child to \textsc{prop} Taparahi \textsc{ipfv} walk \textsc{emph} \textsc{cont} by \textsc{art} road \textsc{ntr} afraid\\

\glt
‘When the children saw Taparahi walking by the road, they were afraid.’ \textstyleExampleref{[R250.190]} 
\z

Possibly, the sense of \textit{rō} in this construction is weakened, and \textit{e V rō}\is{e (imperfective)!e V {\ꞌ}ā} \textit{{\ꞌ}ā} has been developing into a fossilised construction expressing the ongoing duration of a situation.
\subsection{Conclusion}\label{sec:7.4.3}

\is{no ‘just’@nō ‘just’}
To give a general characterisation of \textit{rō} and \textit{nō}, one could say that they indicate the cognitive status of the information given in the clause: \textit{nō} indicates that the clause expresses something unchanged, which is often expected or even routine; \textit{rō} indicates that the clause expresses something new and unexpected, which may even be surprising. \textit{Rō} is reminiscent of a “mirative” marker \citep[255]{Payne1997}, though it may not be as strong as elements that have been identified as miratives in other languages.

Even though \textit{rō} and \textit{nō} are in a way opposites, both may involve counterexpectation. That \textit{rō} would express counterexpectation is no surprise, but \textit{nō} may involve a hint of counterexpectation as well: a situation continues to be true or an expected event still happens, despite factors to the contrary. When \textit{nō} involves counterexpectation, this is because of an (unexpected) continuity; \textit{rō} expresses counterexpectation because of a discontinuity.
\is{ro (emphatic marker)@rō (emphatic marker)|)}

\section{Directionals}\label{sec:7.5}
\is{Directional|(}\is{Deictic centre|(}
The directionals\is{Directional} \textit{mai} and \textit{atu} indicate direction with respect to a certain deictic centre or locus:

\newpage 
\begin{itemize}
\item 
\textit{Mai} indicates movement towards the deictic centre, hence the gloss ‘hither’;

\item 
\textit{Atu} indicates movement away from the deictic centre, hence the gloss ‘away’.

\end{itemize}

\textit{Mai} and \textit{atu} are the only reflexes in Rapa Nui of a somewhat larger system of directionals\is{Directional} in \is{Proto-Polynesian}Proto-Polynesian.\footnote{\label{fn:350}\citet[34]{Clark1976} identifies five directionals\is{Directional} in \is{Proto-Polynesian}PPN: \textit{*mai} ‘toward speaker’, \textit{*atu} ‘away from speaker’, \textit{*hake} ‘upward’, \textit{*hifo} ‘downward’, \textit{*ange} ‘along, obliquely’. Most languages preserved at least three of these, Rapa Nui only two. \textit{*hifo} was retained as \textit{iho}; however, this developed into an adverb\is{Adverb} meaning ‘just then’ (\sectref{sec:4.5.3.1}).
Ultimately, \textit{mai} and \textit{atu} stem from a set of directional\is{Directional} verbs in POc, which were used as the final verb in a serial verb\is{Serial verb} construction \citep[194]{Ross2004}.} 

The movement indicated by directionals\is{Directional} may be of different kinds. Three common types are:\is{Deictic centre}

\begin{itemize}
\item 
movement of the Agent, with motion verbs\is{Verb!motion} like \textit{oho} ‘go’;

\item 
movement of the Patient or another participant, with transfer verbs like \textit{va{\ꞌ}ai} ‘give’, \textit{to{\ꞌ}o} ‘take’, or verbs of carrying like \textit{ma{\ꞌ}u} ‘carry’.

\item 
flow of information from one participant to another, with speech verbs\is{Verb!speech} like \textit{kī} ‘say’.

\end{itemize}

The last type of movement is a metaphorical extension of the idea of movement. Other metaphorical extensions are possible, as will be shown below.

In \sectref{sec:7.5.1}, the main uses of directionals\is{Directional} are discussed, mainly based on three narrative texts (all of which include a considerable amount of direct speech). In \sectref{sec:7.5.2}, statistics are presented for the use of directionals\is{Directional} with certain categories of verbs in the text corpus as a whole. Finally, \sectref{sec:7.5.3} raises the question which factors prompt the use of a directional\is{Directional}.

\subsection{Use of directionals}\label{sec:7.5.1}
\is{Directional}\subsubsection[In direct speech]{In direct speech}\label{sec:7.5.1.1}

As indicated in the previous section, directionals\is{Directional} signal movement with respect to a deictic centre. In direct speech, the deictic centre is usually the speaker. This means that in a conversation, \textit{mai} usually indicates a movement towards the speaker as in \REF{ex:7.122} below, while \textit{atu} indicates a movement away from the speaker. The latter movement may be towards the addressee as in \REF{ex:7.123}, or away from the speaker in another direction as in \REF{ex:7.124}.

\ea\label{ex:7.122}
\gll ¿Ko ai koe e eke \textbf{mai} ena? \\
~\textsc{prom} who \textsc{2sg} \textsc{ipfv} go\_up hither \textsc{med} \\

\glt 
‘Who are you (who are) coming up?’ \textstyleExampleref{[R304.084]} 
\z

\ea\label{ex:7.123}
\gll {\ꞌ}Ī au he oho \textbf{atu}.\\
\textsc{imm} \textsc{1sg} \textsc{ntr} go away\\

\glt 
‘I am coming (towards you) right now.’ \textstyleExampleref{[R152.010]} 
\z

\ea\label{ex:7.124}
\gll He e{\ꞌ}a \textbf{atu} a au {\ꞌ}i te hora nei.\\
\textsc{ntr} go\_out away \textsc{prop} \textsc{1sg} at \textsc{art} time now\\

\glt
‘I am going away now.’ \textstyleExampleref{[R245.017]} 
\z

In the next examples, it is the direct object which moves: towards the speaker in \REF{ex:7.125}, away from the speaker towards the addressee in \REF{ex:7.126}. 

\ea\label{ex:7.125}
\gll ¡Ka hoa \textbf{mai} a nei! \\
~\textsc{imp} throw hither by \textsc{prox} \\

\glt 
‘Throw (the body) here!’ \textstyleExampleref{[R304.060]} 
\z

\ea\label{ex:7.126}
\gll ¡Ka haro \textbf{atu}! \\
~\textsc{imp} pull away \\

\glt
‘Pull (the net, from here) towards you!’ \textstyleExampleref{[R304.135]} 
\z

In the following examples, there is no physical movement of a participant or object, but a flow of information from the speaker to the addressee. In \REF{ex:7.127} the speaker is the subject, so the information moves away from the speaker; hence the use of \textit{atu}. In \REF{ex:7.128} the speaker is addressed by the subject of the clause, so the flow of information is directed towards the speaker; hence the use of \textit{mai}.

\ea\label{ex:7.127}
\gll {\ꞌ}O ira e haka {\ꞌ}ite \textbf{atu} ena i te roŋo rivariva nei. \\
because\_of \textsc{ana} \textsc{ipfv} \textsc{caus} know away \textsc{med} \textsc{acc} \textsc{art} message good:\textsc{red} \textsc{prox} \\

\glt 
‘Therefore (I) make this good news known (to you).’ \textstyleExampleref{[Acts 13:32]}
\z

\ea\label{ex:7.128}
\gll ¿He aha rā nei o te me{\ꞌ}e nei a koro ka kī \textbf{mai} nei? \\
~\textsc{pred} what \textsc{intens} \textsc{prox} of \textsc{art} thing \textsc{prox} \textsc{prop} Dad \textsc{cntg} say hither \textsc{prox} \\

\glt
‘What is this thing that Dad is saying (to us)?’ \textstyleExampleref{[R313.007]} 
\z

The movement may also be more implicit. The following example is spoken by fishermen, who tell what often happens to them: a tuna will come up towards them (i.e. towards the speaker, \textit{mai}), but then it will cut the line. The last verb ‘to cut’ is not a motion verb\is{Verb!motion}, yet the verb is followed by \textit{atu}: the action implies that the tuna swims away from the fishermen, i.e. away from the deictic centre.

\ea\label{ex:7.129}
\gll ...te kahi era {\ꞌ}i raro {\ꞌ}ā, e iri \textbf{mai} era,  he motu rō \textbf{atu} {\ꞌ}ai te kahi.\\
~~~\textsc{art} tuna \textsc{dist} at below \textsc{ident} \textsc{ipfv} ascend hither \textsc{dist}  \textsc{ntr} cut \textsc{emph} away \textsc{subs} \textsc{art} tuna\\

\glt 
‘...the tuna deep below, which when it comes up, the tuna cuts (the line).’ \textstyleExampleref{[R368.024]} 
\z

As directionals\is{Directional} indicate the direction of movement of a participant, object, or information, a directional\is{Directional} may be sufficient to indicate the recipient, addressee or goal of an event: \textit{kī mai} ‘say toward’ indicates that something was said to me (or us). Therefore, the recipient, addressee or goal does not need to be stated separately.\footnote{\label{fn:351}For this reason, \citet[4]{Wittenstein1978} calls \textit{mai} and \textit{atu} in Rapa Nui “directional\is{Directional} pronouns”.} This is the case in \REF{ex:7.128} above and in \REF{ex:7.130} below:

\ea\label{ex:7.130}
\gll —¡A Te Manu ho{\ꞌ}i i kī \textbf{mai} mo turu o māua!  —¡{\ꞌ}Ēē, a Te Manu ho{\ꞌ}i i kī \textbf{atu}!\\
~~~~\textsc{prop} Te Manu indeed  \textsc{pfv} say hither for go\_down of \textsc{1du.excl}  ~~~~yes~~ \textsc{prop} Te Manu indeed \textsc{pfv} say away\\

\glt
‘—Te Manu said (to me) that we (=he and I) should go down! —OK, so Te Manu told (you)!’ \textstyleExampleref{[R245.221]} 
\z

When the subject is also left implicit, the directional\is{Directional} \textit{mai} or \textit{atu} may be the only clue for participant reference. In \REF{ex:7.131}, the subject is not expressed. \textit{Atu} indicates that the request was directed from the speaker (‘we’) to Meherio, while the help requested went from Meherio to the speaker.\footnote{\label{fn:352}The start of the sentence is syntactically unusual. The syntax of \textit{{\ꞌ}āmui} (a borrowing from \ili{Tahitian}, where it is a verb ‘to get together, be united’) is very flexible in Rapa Nui. In this case, a \ili{Tahitian} construction seems to be used, in which \textit{i N VP} (‘to/at me said’) may function as a temporal clause (‘when I said’); this construction is not attested otherwise in Rapa Nui.}

\ea\label{ex:7.131}
\gll {\ꞌ}Āmui i a Meherio e kī \textbf{atu} era... mo hā{\ꞌ}ū{\ꞌ}ū \textbf{mai},  {\ꞌ}ina he hā{\ꞌ}ū{\ꞌ}ū rō \textbf{mai}.\\
moreover at \textsc{prop} Meherio \textsc{ipfv} say away \textsc{dist} for help hither  \textsc{neg} \textsc{ntr} help \textsc{emph} hither\\

\glt 
‘Moreover, when (we) told Meherio to help (us), (she) didn’t help (us).’ \textstyleExampleref{[R315.031]} 
\z

In the examples so far, \textit{atu} indicates either a movement from speaker to addressee, or a movement away from speaker and hearer in an unspecified direction as in \REF{ex:7.124}.\footnote{\label{fn:353}As mentioned above, according to \citet[34]{Clark1976}, \textit{*atu} in \is{Proto-Polynesian}PPN means ‘away from the speaker’.} However, \textit{atu} does not always imply a movement away from the speaker: it may indicate a movement from another place or participant towards the addressee. The following examples illustrate this.

\ea\label{ex:7.132}
\gll Mo haŋa ō{\ꞌ}ou mo haŋa \textbf{atu} o tētahi manu era, e haŋa ra{\ꞌ}e e koe. \\
if want \textsc{poss.2sg.o} for love away of other bird \textsc{dist} \textsc{exh} love first \textsc{ag} \textsc{2sg} \\

\glt 
‘If you want other birds to love (you), you should love (them) first.’ \textstyleExampleref{[R213.050]} 
\z

\ea\label{ex:7.133}
\gll E tu{\ꞌ}u haka{\ꞌ}ou \textbf{atu} {\ꞌ}ā a Hoto Vari,  e haka~poreko haka{\ꞌ}ou \textbf{atu} {\ꞌ}ā hai {\ꞌ}arero...\\
\textsc{ipfv} arrive again away \textsc{cont} \textsc{prop} Hoto Vari  \textsc{ipfv} stick\_out again away \textsc{cont} \textsc{ins} tongue\\

\glt
‘When Hoto Vari comes again (to you), and sticks out his tongue (to you)...’ \textstyleExampleref{[R304.020]} 
\z

An example like \REF{ex:7.133} is striking, because it is not at all clear how the location of origin can be considered the deictic centre: the place where Hoto Vari comes from, is not relevant at all in the story; it is not even mentioned. In other words, the use of \textit{atu} seems to be motivated entirely by its destination (the second person), not by a deictic centre. This may thus be an exception to the rule (formulated e.g. by \citealt[285]{Hooper2002} for \ili{Tokelauan}) that the use of directionals\is{Directional} always implies the existence of a deictic centre.

These example also show that the sense of \textit{atu} cannot be captured in a single definition: \textit{atu} does not always express movement away from the speaker (see (\ref{ex:7.132}–\ref{ex:7.133})), but neither does it always express movement towards the addressee (see \REF{ex:7.124}). Either one is a sufficient criterion for using \textit{atu}; neither is a necessary criterion.

\subsubsection{In third-person contexts}\label{sec:7.5.1.2}

The previous section discussed contexts where a speaker and/or addressee is involved and where movement takes place with respect to the speaker or addressee. As we saw, in these cases \textit{mai} indicates movement towards the speaker, while \textit{atu} indicates movement away from the speaker and/or towards the addressee.

Directionals are also used in third-person contexts, where no speaker or addressee is involved. In such cases, movement does not take place from the perspective of the speaker; rather, the deictic centre is a participant or location in the text. The speaker positions himself (and the hearers) at a certain location or near a certain participant, and events are regarded from the point of view of that location or participant.

There are no fixed rules for determining the deictic centre: it is to a certain extent up to the narrator to choose the perspective from which the text world is regarded.\footnote{\label{fn:354}\citet[105]{Tchekhoff1990} likewise stresses optionality and subjectivity for the use of directional\is{Directional} particles in \ili{Tongan}.} The deictic centre may be fairly constant throughout the story, or it may shift with each scene or even from sentence to sentence. Speakers may have a preference to identify the deictic centre with one central participant, or to vary the point of view. Speakers may also show a preference for \textit{mai} or \textit{atu} with certain verbs or classes of verbs, regardless the context and the subject of the verb. 

In other words, there are no hard and fast rules for the use of directionals\is{Directional}. However, certain clear tendencies can be observed. In this section some of these tendencies are discussed from individual stories, while \sectref{sec:7.5.2} gives statistical data from the whole text corpus. These statistics reveal a number of general tendencies and also show a number of diachronic shifts in the use of directionals\is{Directional}.

\paragraph[Example 1: a stable deictic centre]{Example 1: a stable deictic centre}\label{sec:7.5.1.2.1}

The story \textit{Nuahine Rima Roa}, ‘The old lady with the long arms’ (R368), tells about an old woman with enormous arms, who terrorises the village by stealing food, but who is eventually tricked into defeat by a group of fishermen. In this story, there is one central participant, the old lady; the other participants are hardly mentioned as individuals (they mostly act as a group), let alone mentioned by name. It is not surprising, therefore, that the deictic centre in most of the story is the old lady. Events are regarded from the perspective of wherever the old lady is. Numerous examples could be given, such as the following:

\ea\label{ex:7.134}
\gll He ra{\ꞌ}e ma{\ꞌ}u \textbf{mai} era i te kai... \\
\textsc{pred} first carry hither \textsc{dist} \textsc{acc} \textsc{art} food \\

\glt 
‘They first brought food (to her)...’ \textstyleExampleref{[R368.006]} 
\z

\ea\label{ex:7.135}
\gll {\ꞌ}I te {\ꞌ}ao era {\ꞌ}ā o tū ra{\ꞌ}ā era, he oho \textbf{mai} he tu{\ꞌ}u \textbf{mai}  ki tū rū{\ꞌ}au era.\\
at \textsc{art} dawn \textsc{dist} \textsc{ident} of \textsc{dem} day \textsc{dist} \textsc{ntr} go hither \textsc{ntr} arrive hither  to \textsc{dem} old\_woman \textsc{dist}\\

\glt
‘In the morning of that day, they came and arrived at that old lady.’ \textstyleExampleref{[R368.063]} 
\z

In direct speeches in the story the situation is different: here the deictic centre is the speaker, whether this is the old lady or another participant. But even outside direct speech, not all directionals\is{Directional} in this story presuppose the old lady as deictic centre. In the following example, the men come out of the house of the old lady, i.e. they move away from her; yet \textit{mai} is used:

\ea\label{ex:7.136}
\gll I e{\ꞌ}a haka{\ꞌ}ou \textbf{mai} era tū ŋāŋata era mai te hare era  o tū rū{\ꞌ}au era...\\
\textsc{pfv} go\_out again hither \textsc{dist} \textsc{dem} men \textsc{dist} from \textsc{art} house \textsc{dist}  of \textsc{dem} old\_woman \textsc{dist}\\

\glt
‘When those men came out again of the house of that old woman...’ \textstyleExampleref{[R368.056]} 
\z

This apparent exception to the rule may have to do with a general tendency of the verb \textit{e{\ꞌ}a} ‘go out’ to be followed by \textit{mai} rather than \textit{atu}. As discussed in \sectref{sec:7.5.2} below, \textit{e{\ꞌ}a} commonly takes \textit{mai} while it rarely takes \textit{atu}; a similar tendency is discernible for other motion verbs\is{Verb!motion}. This means that the directional\is{Directional} after \textit{e{\ꞌ}a} tends to point to the destination, the place where the subject is going to, as the centre of attention. In this way it provides the reader/hearer with a subtle signal that this location is significant as the location where the next events are going to happen. Notice that in \REF{ex:7.136} above, \textit{e{\ꞌ}a mai} occurs in a cohesive clause, which provides a bridge between the previous scene (in the house) and the next one (in the village). \textit{Mai} contributes to paving the way for the change of location and the next scene.

Such examples show that even in a narrative with one protagonist around whom the action revolves, the narrator may use directionals\is{Directional} as a device to focus the hearer’s attention on locations relevant in the development of the story.

\paragraph{Example 2: a shifting deictic centre}\label{sec:7.5.1.2.2}

The story \textit{He via o te Tūpāhotu} ‘The life of the Tupahotu’ (R304) tells about wars between two major tribes on the island, the Miru and the Tupahotu. There are various protagonists: Huri a Vai and his father Kainga of the Miru tribe, Hoto Vari and his father Poio of the Tupahotu tribe. These protagonists, as well as a few other characters, alternate in prominence in different parts of the story, and the deictic centre shifts accordingly.

In the first part of the story, the focus is on Huri a Vai. Not only is he mentioned more than other characters, the directionals\is{Directional} point towards him as the deictic centre:

\ea\label{ex:7.137}
\gll He take{\ꞌ}a i a Hoto Vari ka pū \textbf{mai}. \\
\textsc{ntr} see \textsc{acc} \textsc{prop} Hoto Vari \textsc{cntg} approach hither \\

\glt
‘(Huri a Vai) saw Hoto Vari coming towards him.’ \textstyleExampleref{[R304.004]} 
\z

Now what if the movement concerns the protagonist himself, i.e. when Huri a Vai himself moves to a different location? \citet[142–143]{Levinsohn2007} points out that in such cases, languages tend to use one of two strategies: the deictic centre is either a fixed geographical location or it is the next location, i.e. the destination of the movement. As it turns out, in Rapa Nui narrative the second strategy is predominant: when the protagonist moves, \textit{mai} is used to point to the location where the next events are going to happen.\footnote{\label{fn:355}In the preceding section, the same tendency was observed in the story \textit{Nuahine Rima Roa}; see example \REF{ex:7.136} and discussion.} The following example illustrates this:

\ea\label{ex:7.138}
\gll I ahiahi era he hoki \textbf{mai} a Huri {\ꞌ}a Vai ki te kona hare era. \\
\textsc{pfv} afternoon \textsc{dist} \textsc{ntr} return hither \textsc{prop} Huri a Vai to \textsc{art} place house \textsc{dist} \\

\glt
‘In the afternoon, Huri a Vai returned home.’ \textstyleExampleref{[R304.009]} 
\z

This corresponds to a general tendency in Rapa Nui: motion verbs\is{Verb!motion} are much more commonly followed by \textit{mai} than by \textit{atu}, as shown in \sectref{sec:7.5.2} below.

In the remainder of the story, the deictic centre shifts between various participants and locations. Sometimes one of the major participants is the deictic centre for a while; in the following example, four consecutive verbs are all followed by a directional\is{Directional} pointing towards Kainga, one of the protagonists, as deictic centre:

\ea\label{ex:7.139}
\gll {\ꞌ}Ī ka u{\ꞌ}i \textbf{atu} ena ko te {\ꞌ}ata o te taŋata ka kohu \textbf{mai}  {\ꞌ}i mu{\ꞌ}a i a ia. I hāhine \textbf{mai} era ki muri i a ia, he {\ꞌ}ui \textbf{atu}...\\
\textsc{imm} \textsc{cntg} look away \textsc{med} \textsc{prom} \textsc{art} shadow of \textsc{art} man \textsc{cntg} shade hither  at front at \textsc{prop} \textsc{3sg} \textsc{pfv} near hither \textsc{dist} to near at \textsc{prop} \textsc{3sg} \textsc{ntr} ask away\\

\glt
‘Then (Kainga) saw the shadow of a man falling in front of him. When (that man) was close to him, (Kainga) asked...’ \textstyleExampleref{[R304.095–096]}
\z

The deictic centre may also be a minor participant, provided this participant is significant in the scene in question. See \REF{ex:7.140} in the next section for an example. 

\subsubsection[Directionals with speech verbs]{Directionals with speech verbs}\label{sec:7.5.1.3}
\is{Verb!speech}
Directionals in Rapa Nui are commonly used in clauses introducing direct speech (‘he said: ...’). In such clauses, various strategies are possible:

%\setcounter{listWWviiiNumlviiileveli}{0}
\begin{enumerate}
\item 
do not use a directional\is{Directional};

\item 
use \textit{mai}, designating the addressee as deictic centre;

\item 
use \textit{atu}, designating the speaker as deictic centre;

\item 
in a dialogue, use \textit{mai} with one speaker and \textit{atu} with the other, i.e. one speaker is the deictic centre throughout.

\end{enumerate}

All these strategies are used to various degrees in Rapa Nui discourse. Strategy 1 is dominant overall: as the statistics in the next section will show, about 70\% of all speech verbs\is{Verb!speech} in the corpus do not have a directional\is{Directional}. In this section, the other strategies are illustrated from a couple of texts.

In the story \textit{He via} (R304), discussed in the previous section, a mix of strategies is used. In the following short conversation, the directionals\is{Directional} all point towards Oho Takatore as deictic centre (strategy 4). Oho Takatore is not a central participant in the story as a whole, but his presence is crucial at this point.

\ea\label{ex:7.140}
\gll He raŋi \textbf{atu} ia e {\ꞌ}Oho Takatore... Terā ia ka pāhono \textbf{mai} e Poio... I raŋi \textbf{mai} era e Poio pē ira...\\
\textsc{ntr} call away then \textsc{ag} Oho Takatore then then \textsc{cntg} answer hither \textsc{ag} Poio \textsc{pfv} call hither \textsc{dist} \textsc{ag} Poio like \textsc{ana}\\

\glt
‘Oho Takatore shouted... Then Poio answered... When Poio had called out like that...’ \textstyleExampleref{[R304.058-063]}
\z

In the following conversation, the deictic centre shifts halfway: in the first two clauses, Kainga is the deictic centre, but then it shifts to Vaha (strategies 2+4). 

\ea\label{ex:7.141}
\gll He {\ꞌ}ui \textbf{atu}... Terā, ka pāhono \textbf{mai} e Vaha... He pāhono \textbf{mai} ia e Kāiŋa...\\
\textsc{ntr} ask away then \textsc{cntg} answer hither \textsc{ag} Vaha \textsc{ntr} answer hither then \textsc{ag} Kainga\\

\glt
‘(Kainga) asked... Vaha replied... Kainga then replied...’ \textstyleExampleref{[R304.096]} 
\z

These examples show that the speaker has the choice from a variety of strategies.

In another story, \textit{Rā{\ꞌ}au o te rū{\ꞌ}au ko Mitimiti} ‘Medicine of the old woman Mitimiti’ (R313), the narrator has a general preference for \textit{atu}, both with speech verbs\is{Verb!speech} and other verbs (though not without exceptions). The general pattern in this story is for the first turn in a conversation to be unmarked or marked with \textit{mai}, whereas the following turns are marked with \textit{atu} (strategy 3).

\ea\label{ex:7.142}
\gll He kī o koro... {\ꞌ}E he kī tako{\ꞌ}a \textbf{atu} te re{\ꞌ}o o nua... He pāhono  \textbf{atu} ia te re{\ꞌ}o o tū ŋā repa era... He kī haka{\ꞌ}ou \textbf{atu} ia te re{\ꞌ}o o koro...\\
\textsc{ntr} say of Dad and \textsc{ntr} say also away \textsc{art} voice of Mum \textsc{ntr} answer  away then \textsc{art} voice of \textsc{dem} \textsc{pl} young\_man \textsc{dist} \textsc{ntr} say again away then \textsc{art} voice of Dad\\

\glt
‘Dad said... And Mum’s voice also said... The voice of those youngsters replied... Dad’s voice said again...’ \textstyleExampleref{[R313.009–015]}
\z

In this text, then, \textit{atu} functions as a sort of continuance marker, marking the next step in a series of speech turns.\footnote{\label{fn:356}This use may be influenced by \ili{Tahitian}, where \textit{atu ra} and \textit{mai ra} are extremely common to mark the next event in a narrative (see e.g. \citealt[134]{LazardPeltzer2000}).} 

\subsubsection[Lack of movement: more metaphorical uses]{Lack of movement: more metaphorical uses}\label{sec:7.5.1.4}

So far, various uses of directionals\is{Directional} have been discussed in which some kind of physical or metaphorical movement takes place: movement of a participant or object, or a flow of speech. This section deals with the use of directionals\is{Directional} in cases where no movement seems to be involved. In these cases the use of directionals\is{Directional} is extended even further than the metaphorical senses discussed so far. Various metaphorical extensions are possible, depending on the verb involved and subject to speaker preference. The examples discussed here do not cover all possibilities, but serve to illustrate the wide range of metaphorical uses of directionals.

Directionals may occur with verbs that do not indicate any movement, nor a transitive\is{Verb!transitive} action, but rather the absence of movement. In the story \textit{He via} (R304), a directional\is{Directional} is used twice with the verb \textit{piko} ‘to hide (intr.)’, once \textit{mai} and once \textit{atu}:

\ea\label{ex:7.143}
\gll ...tū pū era o Huri {\ꞌ}a Vai e \textbf{piko} \textbf{mai} era {\ꞌ}i roto. \\
~~~\textsc{dem} hole \textsc{dist} of Huri a Vai \textsc{ipfv} hide hither \textsc{dist} at inside \\

\glt 
‘...the hole where Huri a Vai was hiding.’ \textstyleExampleref{[R304.044]} 
\z

\ea\label{ex:7.144}
\gll He e{\ꞌ}a he oho mai a tū ara era {\ꞌ}i ira a Kāiŋa e \textbf{piko} \textbf{atu} era. \\
\textsc{ntr} go\_out \textsc{ntr} go hither by \textsc{dem} way \textsc{dist} at \textsc{ana} \textsc{prop} Kainga \textsc{ipfv} hide away \textsc{dist} \\

\glt
‘(Vaha) came out by that road where Kainga was hiding.’ \textstyleExampleref{[R304.094]} 
\z

There is a clear difference between these two examples: in \REF{ex:7.143}, Huri a Vai is hiding from his enemies, he is lying low to avoid being detected. In \REF{ex:7.144}, Kainga is not just hiding away; he is lying in ambush, waiting for Vaha to come by. In other words, the hiding in \REF{ex:7.143} is self-directed, oriented inwards, while the hiding in \REF{ex:7.144} is outward-looking, with the attention away from the person hiding. It is no coincidence that in the first case \textit{mai} is used, indicating orientation towards the subject as deictic centre, while in the second case \textit{atu} is used, pointing away from the subject. In other texts as well, \textit{piko mai} is commonly used when people hide from others, while \textit{piko atu} is used of people lying in ambush, spying on someone else (cf. \citealt[12]{Fuller1980}). While there is no movement involved, the directionals\is{Directional} indicate \textsc{orientation} with respect to the deictic centre.\footnote{\label{fn:357}\citet[1751]{Hooper2004} discusses a similar function of directionals\is{Directional} in \ili{Tokelauan}. She points out (with reference to \citealt{Jackendoff1983}) that a path (i.e. a directional\is{Directional} movement) may play various roles: an object may traverse a path, but it may also be oriented along a path, facing an entity (in this case, the deictic centre).} 

Directionals are also found with the verb \textit{noho} ‘to sit, stay’, a verb which would seem to epitomise lack of movement. At least some of these occurrences can be explained as indicating orientation: with \textit{mai} the participant faces inward, is self-oriented; with \textit{atu} the focus is outward. The following two examples illustrate the difference:

\ea\label{ex:7.145}
\gll E noho nō \textbf{mai} {\ꞌ}ā tū taŋata era, {\ꞌ}ī ka hakaroŋo atu ena....\\
\textsc{ipfv} stay just hither \textsc{cont} \textsc{dem} man \textsc{dist} \textsc{imm} \textsc{cntg} listen away \textsc{med}\\

\glt 
‘When that man was just staying (inside), suddenly he heard (a noise)...’ \textstyleExampleref{[R372.103]} 
\z

\ea\label{ex:7.146}
\gll Te vārua mo noho \textbf{atu} ō{\ꞌ}ou mo u{\ꞌ}i a ruŋa i a rāua... he u{\ꞌ}i kē.\\
\textsc{art} spirit if sit away \textsc{poss.2sg.o} if look by above at \textsc{prop} \textsc{3pl} \textsc{ntr} look different\\

\glt
‘The spirits, when you sit down and look at them, will look away.’ \textstyleExampleref{[R310.082]} 
\z

Orientation may also have to do with physical distance from the deictic centre, so that \textit{mai} is similar to ‘here’ and \textit{atu} is similar to ‘away’:

\ea\label{ex:7.147}
\gll E māmārū{\ꞌ}au, ka noho \textbf{mai} koe.\\
\textsc{voc} grandmother \textsc{imp} stay hither \textsc{2sg}\\

\glt 
‘Grandmother, stay here.’ \textstyleExampleref{[R313.177]} 
\z

\ea\label{ex:7.148}
\gll He oho atu he piri ki tētahi ŋā poki, he noho \textbf{atu} ananake.\\
\textsc{ntr} go away \textsc{ntr} join to other \textsc{pl} child \textsc{ntr} stay away together\\

\glt 
‘He went off (instead of going to school) and met other boys, and they stayed (away) together.’ \textstyleExampleref{[R250.034]} 
\z

\subsubsection{\textit{Atu} indicating extent}\label{sec:7.5.1.5}

Another metaphorical use concerns only \textit{atu}. With stative verbs\is{Verb!stative}, \textit{atu} may indicate the extent of a state or characteristic:

\ea\label{ex:7.149}
\gll Me{\ꞌ}e pararaha \textbf{atu} te oru era. \\
thing fat away \textsc{art} pig \textsc{dist} \\

\glt 
‘That pig is very fat.’ \textstyleExampleref{[Notes]}
\z

\ea\label{ex:7.150}
\gll E huri rō {\ꞌ}ā te {\ꞌ}āriŋa o Heru a ruŋa {\ꞌ}e ko tetea \textbf{atu} {\ꞌ}ā  te mata.\\
\textsc{ipfv} turn \textsc{emph} \textsc{cont} \textsc{art} face of Heru by above and \textsc{prf} \textsc{pl}:white away \textsc{cont}  \textsc{art} eye\\

\glt
‘Heru’s face was turned upwards and his eyes were very white.’ \textstyleExampleref{[R313.043]} 
\z

These examples suggest that in some cases \textit{atu} indicates a (light) degree of emphasis. It is not difficult to see how this use could arise: the basic meaning ‘away from a deictic centre’ may naturally develop into ‘away from a point of reference, beyond what is common or expected’.\footnote{\label{fn:358}\citet[291]{Hooper2002} signals a somewhat similar extension of the meaning of \textit{atu} in \ili{Tokelauan}, where it may signal a point in time beyond the time of reference (e.g. \textit{ananafi} ‘yesterday’, \textit{ananafi atu} ‘the day before yesterday’ – the same expression is found in \ili{Tahitian}).} 

The sense of an extent is also seen when \textit{atu} is used after the quantifiers\is{Quantifier} \textit{\mbox{tētahi}} ‘some/ others’, \textit{me{\ꞌ}e rahi} ‘many’ and \textit{rauhuru} ‘diverse’; in these cases, \textit{atu} emphasises the extent of a quantity (see discussion and examples in \sectref{sec:4.4.10}.) 

Finally, the sense of extent may also explain why \textit{atu} is common – at least for some speakers – in the construction \textit{he V rō {\ꞌ}ai}\is{he (aspect marker)!he V rō {\ꞌ}ai}, which indicates final or climactic events (\sectref{sec:7.2.3.3}).\footnote{\label{fn:359}In the corpus as a whole, \textit{rō atu {\ꞌ}ai} occurs 186 times, \textit{rō mai {\ꞌ}ai} only 60 times. Note that directionals\is{Directional} are by no means obligatory in this construction: \textit{rō {\ꞌ}ai} without directional\is{Directional} occurs 321 times.}  \textit{atu} simply makes the construction a little heavier, thereby adding to its prominence.

\ea\label{ex:7.151}
\gll \textbf{He} mate \textbf{rō} \textbf{atu} \textbf{{\ꞌ}ai} tū rū{\ꞌ}au era {\ꞌ}i te taŋi. \\
\textsc{ntr} die \textsc{emph} away \textsc{subs} \textsc{dem} old\_woman \textsc{dist} at \textsc{art} cry \\

\glt 
‘The old lady burst out in tears (lit. died with crying).’ \textstyleExampleref{[R313.104]} 
\z

\subsection{Directionals with motion, speech, and perception verbs}\label{sec:7.5.2}
\is{Verb!perception}
In the previous sections, the use of directionals\is{Directional} was analysed by looking at individual occurrences. One conclusion that could be drawn is that, while the basic meaning of the directionals\is{Directional} is clear, the speaker has a certain freedom, both in choosing the deictic centre and in applying directionals\is{Directional} in extended uses. 

Another method to analyse the use of directionals\is{Directional}, is to count the overall use of directionals\is{Directional} with different (classes of) verbs in the text corpus. As it turns out, statistical data shed additional light on the use of directionals\is{Directional}, revealing a number of general tendencies. These tendencies cannot be discovered by analysing individual texts, but only come to the surface when large numbers of occurrences (and non-occurrences) are taken into account.

This section discusses the use of directionals\is{Directional} with three classes of verbs that commonly take directionals\is{Directional}: motion verbs\is{Verb!motion}, speech verbs\is{Verb!speech} and perception verbs\is{Verb!perception}. Data are based on the whole corpus of old and new texts.

\subparagraph{\ref{sec:7.5.2}.1~ Motion verbs} One class of verbs which often takes a directional\is{Directional}, is the class of motion verbs\is{Verb!motion}. \tabref{tab:49} gives statistics for the use and non-use of directionals\is{Directional} with a number of common motion verbs\is{Verb!motion}.\footnote{\label{fn:360}This table includes counts for \textit{oho} ‘go’, \textit{tu{\ꞌ}u} ‘arrive’, \textit{e{\ꞌ}a} ‘go out’, \textit{turu} ‘go down’, \textit{iri} ‘go up’, \textit{uru} ‘go in’, \textit{tomo} ‘go ashore’.} Separate figures are given for old texts and newer texts. The most common directional\is{Directional} in each corpus is in bold.

\begin{table}
\begin{tabularx}{.75\textwidth}{L{24mm}R{12mm}R{12mm}R{12mm}R{12mm}}
\lsptoprule
 & \multicolumn{2}{c}{old} & \multicolumn{2}{c}{new}\\
\midrule
{\textit{mai}} &  {\bfseries 24.3\%}&  {\bfseries (863)}&  {\bfseries 26.8\%}&  {\bfseries (2001)}\\
{\textit{atu}} &  4.1\%&  (145)&  4.6\%&  (340)\\
{no~directional\is{Directional}} &  71.6\%&  (2546)&  68.7\%&  (5131)\\
total &  &  (3554)&  &  (7472)\\
\lspbottomrule
\end{tabularx}
\caption{Directionals with motion verbs}
\label{tab:49}
\end{table}

As these figures show, \textit{mai} is much more common than \textit{atu} with these verbs. In other words, when the direction of movement is indicated, in most cases the subject moves towards the deictic centre. Put differently, directionals\is{Directional} after motion verbs\is{Verb!motion} tend to point toward the location where the subject is heading as the deictic centre, subtly signalling that this new location is – or becomes – significant to the action. 

This tendency is strong in both older and newer texts. It is strongest for \textit{e{\ꞌ}a} ‘go out’, which is followed by \textit{mai} in a total of 415 cases and by \textit{atu} in only 32 cases. One could wonder if \textit{e{\ꞌ}a mai} is to a degree lexicalised, though we have to keep in mind that in two thirds of all occurrences, \textit{e{\ꞌ}a} does not have a directional\is{Directional} at all.

\subparagraph{\ref{sec:7.5.2}.2~ Speech verbs} As indicated in the previous section, directionals are also commonly used with verbs of speaking. These verbs imply a flow of information from the speaker\footnote{\label{fn:361}“Speaker” is here taken in the sense of “the person uttering the speech referred to by the speech verb”, not the speaker/narrator of the text as a whole.} to the addressee. The use of directionals with speech verbs\is{Verb!speech} points to one of the participants as the deictic centre of the speech act: \textit{mai} indicates a movement towards the addressee as deictic centre, \textit{atu} indicates a movement away from the speaker as the deictic centre. \is{Verb!speech}

\begin{table}
\begin{tabularx}{.75\textwidth}{L{24mm}R{12mm}R{12mm}R{12mm}R{12mm}}
\lsptoprule
 & \multicolumn{2}{c}{old} & \multicolumn{2}{c}{new}\\
\midrule
{\textit{mai}} &  {\bfseries 23.5\%}&  {\bfseries (397)}&  12.6\%&  (468)\\
{\textit{atu}} &  4.1\%&  (70)&  {\bfseries 17.7\%}&  {\bfseries (656)}\\
{no~directional\is{Directional}} &  72.4\%&  (1224)&  69.7\%&  (2582)\\
total &  &  (1691)&  &  (3706)\\
\lspbottomrule
\end{tabularx}
\caption{Directionals with speech verbs}
\label{tab:50}
\end{table}

Total occurrences for a number of common speech verbs\is{Verb!speech}\footnote{\label{fn:362}\textit{Kī} ‘say’, \textit{raŋi} ‘call’, \textit{pāhono} ‘answer’ (only in modern Rapa Nui), \textit{{\ꞌ}ui} ‘ask’.} are given in \tabref{tab:50}. This table shows a clear shift over time in the use of directionals\is{Directional}. Whereas in old texts \textit{mai} is by far the most common directional\is{Directional} and \textit{atu} is rare, in new texts \textit{atu} has become more frequent (though \textit{mai} is by no means uncommon). In other words, in older stories the speaking act is usually considered from the perspective of the addressee, whereas in newer stories it is more commonly seen from the perspective of the speaker (i.e. the subject). 

While the relative proportion of \textit{mai} and \textit{atu} has changed, the table also shows that the total use of directionals\is{Directional} has not changed much: in both corpora, roughly 30\% of the speech verbs\is{Verb!speech} under consideration are accompanied by a directional\is{Directional}.

\subparagraph{\ref{sec:7.5.2}.3~ Perception verbs} \is{Verb!perception}The most common verbs of perception\is{Verb!perception} in Rapa Nui are the following: for visual perception, \textit{take{\ꞌ}a}\is{tike{\ꞌ}a ‘to see’} or \textit{tike{\ꞌ}a}\footnote{\label{fn:363}Besides \textit{take{\ꞌ}a} and \textit{tike{\ꞌ}a}, there are also the less common variants \textit{tikera} and \textit{takera}. All four are synonymous.} ‘to see’ and \textit{u{\ꞌ}i} ‘to look, to watch’; for aural perception \textit{ŋaro{\ꞌ}a} ‘to hear, to perceive’ and \textit{hakaroŋo} ‘to listen’. All these verbs have two arguments: an experiencer (the perceiving entity) and a stimulus (the perceived entity). The experiencer is expressed as subject, the stimulus as direct object or as an oblique marked with \textit{ki}.

The first verb of each pair (\textit{take{\ꞌ}a} ‘see’ and \textit{ŋaro{\ꞌ}a} ‘hear’\is{nzaroa ‘to perceive’@ŋaro{\ꞌ}a ‘to perceive’}) indicates uncontrolled perception, i.e. the registration of a perceptual stimulus by one of the senses. The other two verbs (\textit{u{\ꞌ}i} ‘look’ and \textit{hakaroŋo} ‘listen’) express controlled perception, i.e. focused attention on the part of the subject (cf. \citealt[144]{Dixon2010-2}). In other words, whereas the subject of \textit{take{\ꞌ}a} and \textit{ŋaro{\ꞌ}a} is merely registrating a visual stimulus, the subject of \textit{u{\ꞌ}i}\is{ui ‘to look’@u{\ꞌ}i ‘to look’} and \textit{hakaroŋo}\is{haka roŋo ‘to listen’} is actively involved in the act of perception.\footnote{\label{fn:364}\citet{OsmondPawley2009} use the terms “sensing” and “attending”, respectively. In Rapa Nui, the two pairs of verb also show differences in subject marking (\sectref{sec:8.3.1.2}).}

This difference can be correlated to the direction of movement involved in the act of perception. Concentrating first on visual verbs: \textit{take{\ꞌ}a} ‘see’ indicates that a signal, originating from the stimulus, is perceived by the experiencer: there is a movement from the stimulus (the perceived object) to the experiencer (the subject). \textit{U{\ꞌ}i} ‘look’, on the other hand, indicates that the experiencer directs his/her attention towards the stimulus: there is a movement from the subject to the object.\footnote{\label{fn:365}See \citet[1745]{Hooper2004} for a similar description of the two possible trajectories.}

This has consequences for the use of directionals\is{Directional} with these verbs. When a directional\is{Directional} is used with \textit{u{\ꞌ}i}, this directional\is{Directional} tends to indicate a movement from the subject (the experiencer) to the object (the stimulus), whereas with \textit{take{\ꞌ}a} the directional\is{Directional} tends to indicate a movement from the object towards the subject. 

Now it is safe to assume that there is a tendency for the subject of the clause to act as the deictic centre, at least in first-person and third-person contexts: as discussed above, the deictic centre is usually either the speaker or a central participant in the discourse, both of which tend to be the subject of the clause. This leads us to expect that the controlled perception verb \textit{u{\ꞌ}i}\is{ui ‘to look’@u{\ꞌ}i ‘to look’} ‘look’ will predominantly take the directional\is{Directional} \textit{atu}: when the subject is the deictic centre, there is an outgoing movement from the subject/experiencer towards the object/stimulus. On the other hand, the uncontrolled perception verb \textit{take{\ꞌ}a} ‘see’ will predominantly take the directional\is{Directional} \textit{mai}: when the subject is the deictic centre, there is a movement from the stimulus towards the subject/experiencer. 

This expectation is borne out in newer texts, as shown in \tabref{tab:51}: \textit{u{\ꞌ}i} is followed by \textit{atu} in 339 cases and by \textit{mai} in only 118 cases; by contrast, \textit{take{\ꞌ}a} is followed by \textit{mai} in 91 cases and by \textit{atu} in just 4 cases.

In older texts, the difference is not as clear: with \textit{take{\ꞌ}a}\is{take{\ꞌ}a ‘to see’}, only \textit{mai} is used (though only in 18 cases), but with \textit{u{\ꞌ}i}, both directionals\is{Directional} are used with similar frequency. This corresponds to the phenomenon observed above with speech verbs\is{Verb!speech}: in older texts there is a general preference for \textit{mai}, while in newer texts \textit{atu} is more common.\footnote{\label{fn:366}Notice, however, that for motion verbs\is{Verb!motion} there is no such shift: \textit{mai} is predominant both in older and newer material.}

\begin{table}
\resizebox{\textwidth}{!}{
\begin{tabular}{lrrrr c rrrr} 
\lsptoprule
& \multicolumn{4}{c}{\textit{u{\ꞌ}i} ‘look, watch’} && \multicolumn{4}{c}{\textit{take{\ꞌ}a}\textit{, tike{\ꞌ}a} ‘see’}\\
& \multicolumn{2}{c}{old} & \multicolumn{2}{c}{new} && \multicolumn{2}{c}{old} & \multicolumn{2}{c}{new}\\
\cmidrule{2-5}
\cmidrule{7-10}
\textit{mai} &  {\bfseries 17.6\%}&  {\bfseries (88)}&  9.2\%&  (118)&&  {\bfseries 9.7\%}&  {\bfseries (18)}&  {\bfseries 13.9\%}&  {\bfseries (91)}\\
\textit{atu} &  14.8\%&  (74)&  {\bfseries 26.3\%}&  {\bfseries (339)}&&  0.0\%&  (0)&  0.6\%&  (4)\\
no~directional\is{Directional} &  67.7\%&  (339)&  64.5\%&  (830)&&  90.3\%&  (168)&  85.5\%&  (561)\\
total &  &  (501)& ~ &  (1287)&&  &  (186)&  &  (656)\\
\lspbottomrule
\end{tabular}
}
\caption{Directionals with verbs of seeing}
\label{tab:51}
\end{table}

Verbs of hearing show the same distinction in newer texts, as shown in \tabref{tab:52}: the controlled \textit{hakaroŋo} ‘to listen’ tends to take \textit{atu}, indicating outgoing attention from the subject as deictic centre, while the uncontrolled \textit{ŋaro{\ꞌ}a} ‘perceive’ usually takes \textit{mai}, indicating incoming perception towards the subject as deictic centre. Again, in older texts this tendency does not show up, though in the case of \textit{ŋaro{\ꞌ}a} data are scarce overall.

\begin{table}
\resizebox{\textwidth}{!}{
\begin{tabular}{lrrrr c rrrr} 
\lsptoprule
& \multicolumn{4}{c}{\textit{hakaroŋo}\is{haka roŋo ‘to listen’} ‘listen’} && \multicolumn{4}{c}{\textit{ŋaro{\ꞌ}a}\is{nzaroa ‘to perceive’@ŋaro{\ꞌ}a ‘to perceive’} ‘hear, perceive’\footnotemark{}}\\
& \multicolumn{2}{c}{old} & \multicolumn{2}{c}{new} && \multicolumn{2}{c}{old} & \multicolumn{2}{c}{new}\\
\cmidrule{2-5}
\cmidrule{7-10}
\textit{mai} &  {\bfseries 28.7\%}&  {\bfseries (25)}&  18.0\%&  (52)&&  {\bfseries 7.4\%}&  {\bfseries (2)}&  {\bfseries 10.8\%}&  {\bfseries (27)}\\
\textit{atu} &  16.1\%&  (14)&  {\bfseries 29.4\%}&  {\bfseries (85)}&&  0.0\%&  (0)&  2.0\%&  (5)\\
no~directional\is{Directional} &  55.2\%&  (48)&  52.6\%&  (152)&&  92.6\%&  (25)&  87.2\%&  (218)\\
total &  &  (87)&  &  (289)&&  &  (27)&  &  (250)\\
\lspbottomrule
\end{tabular}
}
\caption{Directionals with verbs of hearing}
\label{tab:52}
\end{table}

\footnotetext{ Both \textit{hakaroŋo} and \textit{ŋaro{\ꞌ}a} are predominantly used for aural perception, though \textit{ŋaro{\ꞌ}a} (and occasionally \textit{hakaroŋo}) may be used for perception in general (‘to feel, perceive’) as well.}

These tables also show that directionals\is{Directional} \textit{as such} are more common with the controlled perception verbs\is{Verb!active} \textit{u{\ꞌ}i}\is{ui ‘to look’@u{\ꞌ}i ‘to look’} and \textit{hakaroŋo} than with \textit{take{\ꞌ}a}\is{take{\ꞌ}a ‘to see’} and \textit{ŋaro{\ꞌ}a}: \textit{u{\ꞌ}i} takes a directional\is{Directional} in about 35\% of all occurrences, \textit{hakaroŋo} even over 45\%; on the other hand, \textit{take{\ꞌ}a} and \textit{ŋaro{\ꞌ}a} are followed by a directional\is{Directional} in less than 15\% of all occurrences. This is true in both older and newer texts.

\largerpage
In individual instances, the choice for \textit{mai} or \textit{atu} may be governed by other considerations: with any perception verb, the speaker may choose either the Experiencer or the Stimulus as deictic centre, depending on the dynamics of the discourse. But over the whole of the corpus, there is a clear correlation between verb type (controlled or uncontrolled perception) and the choice of directional\is{Directional}.

\subsection{To use or not to use a directional}\label{sec:7.5.3}
\is{Directional}
In \sectref{sec:7.5.1} and subsections, the use of directionals\is{Directional} has been discussed as a binary choice: a speaker may use either \textit{mai} or \textit{atu}. However, the statistics in \sectref{sec:7.5.2} show that verbs which take directionals\is{Directional} only do so in a minority of all cases. For example, only about 30\% of all motion verbs\is{Verb!motion} in the corpus are followed by a directional\is{Directional}. The speaker is thus faced with a ternary choice: \textit{mai}, \textit{atu}, or no directional\is{Directional} at all. One more question must therefore be addressed: which factors influence the choice between using a directional\is{Directional} and using no directional\is{Directional} at all?

Some of the factors which may play a role are the following:

\subparagraph{Directionality} Directionals are used when a movement (physical or metaphorical) is clearly directional\is{Directional} and when the speaker wishes to state so. In the following example, Eva first looks into a general direction; then she looks to a more precise location. Only the second verb is followed by a directional\is{Directional}.

\ea\label{ex:7.152}
\gll He \textbf{u{\ꞌ}i} a ruŋa i te henua... E \textbf{u{\ꞌ}i} \textbf{mai} era a tū kona kī era  e nua pē nei ē: ‘{\ꞌ}I {\ꞌ}Ohovehi mātou ka noho nei’.\\
\textsc{ntr} look by above at \textsc{art} land \textsc{ipfv} look hither \textsc{dist} by \textsc{dem} place say \textsc{dist}  \textsc{ag} Mum like \textsc{prox} thus ~at Ohovehi \textsc{1pl.excl} \textsc{cntg} stay \textsc{prox}\\

\glt
‘She looked towards the land... She kept looking towards the place about which Mum had said: ‘We will stay in Ohovehi’.’ \textstyleExampleref{[R210.082–083]}
\z

\subparagraph{Highlighting} Directionals subtly highlight the deictic centre of the text. The speaker may therefore choose to use directionals\is{Directional} to point to the deictic centre, whether this is constant or shifting. For example, the story \textit{Nuahine rima roa} (\sectref{sec:7.5.1.2.1}) contains numerous occurrences of \textit{mai} which point to the central participant, the old lady.\footnote{\label{fn:368}This may explain why \textit{mai} is more frequent overall than \textit{atu}, see the statistics in the previous sections. \citet[1742]{Hooper2004} mentions a 60/40 proportion for \textit{mai} and \textit{atu} in \ili{Tokelauan} discourse.}

\subparagraph{Participant reference} As discussed in \sectref{sec:7.5.1.1}, ex. (\ref{ex:7.130}–\ref{ex:7.131}), directionals\is{Directional} may play a role in participant reference: directionals\is{Directional} indicate whether a participant is at the origin or the goal of the movement, so they may be used instead of an overt subject or object. This accounts for many occurrences of directionals\is{Directional}, for example with speech verbs\is{Verb!speech} in direct discourse, as in \REF{ex:7.130}.

\subparagraph{Distance} Possibly, directionals\is{Directional} tend to be used when there is a significant distance between the origin and the goal of movement, e.g. between the speaker and the addressee. I have not found many instances where this is is the only factor involved, but there are examples which can plausibly be explained this way. In \REF{ex:7.153}, \textit{e{\ꞌ}a} ‘go out (of the house)’ is not marked by a directional\is{Directional}, while \textit{oho} ‘go’ is; the latter involves movement over a considerable distance, while the former does not. 

\ea\label{ex:7.153}
\gll Ka e{\ꞌ}a koe ka haka rivariva i te poki, ka oho \textbf{atu} kōrua ki Haŋa Piko. \\
\textsc{imp} go\_out \textsc{2sg} \textsc{imp} \textsc{caus} good:\textsc{red} \textsc{acc} \textsc{art} child \textsc{imp} go away \textsc{2pl} to Hanga Piko \\

\glt 
‘Go outside and prepare the child, and go to Hanga Piko.’ \textstyleExampleref{[R210.036]} 
\z

This list is not exhaustive, if only because it does not explain all occurrences of \textit{mai} and \textit{atu}. Moreover, many instances can be explained in more than one way. These factors are no more than possible considerations which may play a role; they influence rather than determine the choice for a directional\is{Directional}.
\is{Directional|)}
\is{Deictic centre|)}
\section{Postverbal demonstratives}\label{sec:7.6}
\is{Demonstrative!postverbal|(}\subsection{Introduction}\label{sec:7.6.1}

The postverbal demonstratives\is{Demonstrative!postverbal} (PVDs\is{Demonstrative!postverbal}) \textit{nei}, \textit{ena} and \textit{era} indicate spatial or temporal distance of the event with respect to a place and/or time of reference. The same forms also occur in the noun phrase (\sectref{sec:4.6.3}). Both in the noun phrase and in the verb phrase they have the following sense:

\begin{tabbing}
xxxx \= xxxxxx \= xxxxxxxxxxxxxx\kill
\> \textit{nei}\>  proximity, close to the speaker\\
\> \textit{ena} \> medial distance, close to the hearer\\
\> \textit{era} \> default PVD\is{Demonstrative!postverbal}; farther distance, removed from both speaker and hearer
\end{tabbing}

PVDs cannot be added to just any verb phrase: as the discussion of aspectuals in \sectref{sec:7.2} shows, PVDs\is{Demonstrative!postverbal} occur in certain syntactic contexts and convey certain syntactic nuances.

\begin{itemize}
\item 
PVDs are common after imperfective \textit{e}\is{e (imperfective)} to express a progressive\is{Progressive} or habitual\is{Aspect!habitual} action (\sectref{sec:7.2.5.4}).

\item 
The contiguous marker \textit{ka}\is{ka (aspect marker)} is often followed by a PVD\is{Demonstrative!postverbal}, both in main and subordinate clauses (\sectref{sec:7.2.6.2}–\ref{sec:7.2.6.3}).

\item 
With the perfect \textit{ko}\is{ko V {\ꞌ}ā (perfect aspect)} \textit{V {\ꞌ}ā}\is{ko V {\ꞌ}ā (perfect aspect)}, \textit{era} is occasionally used to express an action which is well and truly finished (\sectref{sec:7.2.7.3}).

\end{itemize}

In addition, PVD\is{Demonstrative!postverbal}’s\is{Demonstrative!postverbal} are used in relative clauses\is{Clause!relative!bare} (\sectref{sec:11.4.5}; see also \REF{ex:7.161} below).

The neutral aspectual \textit{he} is rarely followed by a PVD\is{Demonstrative!postverbal}. PVDs do not occur after the imperative markers \textit{ka} and \textit{e}. Neither are they found after negators \textit{kai} and \textit{e ko}, and after subordinators like \textit{mo} ‘in order to’ and \textit{ana} ‘irrealis’.

The use of PVDs after perfective \textit{i} warrants separate treatment. After \textit{i}\is{i (perfective)}, the verb is often followed by a PVD\is{Demonstrative!postverbal}; the list of PVDs\is{Demonstrative!postverbal} after \textit{i} also includes a fourth PVD: \textit{ai}, which is not used after other aspectuals. In fact, \textit{ai} is the default PVD\is{Demonstrative!postverbal} after \textit{i}, except in cohesive clauses\is{Clause!cohesive}. This will be discussed in \sectref{sec:7.6.5}.

\begin{table}
\begin{tabularx}{\textwidth}{L{12mm}R{12mm}R{12mm}p{2mm}R{12mm}R{12mm}p{2mm}R{12mm}R{12mm}}

\lsptoprule
 & \multicolumn{2}{c}{old texts} && \multicolumn{2}{c}{new texts} && \multicolumn{2}{c}{\textit{ {\textup{total}}}}\\
\midrule
{\textit{era}} &  69\%&  (455)& &  72\%&  (3,728)& & 72\%&  (4,183)\\
{\textit{ena}} &  10\%&  (67)& &  17\%&  (874)& & 16\%&  (941)\\
{\textit{nei}} &  21\%&  (142)& &  11\%&  (568)& &  12\%&  (710)\\
\lspbottomrule
\end{tabularx}
\caption{Frequencies of postverbal demonstratives}
\label{tab:53}
\end{table}

In the following subsections, the four PVDs\is{Demonstrative!postverbal} will be discussed in turn. First a statistical note. \tabref{tab:53} shows frequencies for the \textit{era}, \textit{ena} and \textit{nei} in all verb phrases in the text corpus. As this table shows, \textit{era} is far more frequent than \textit{ena} and \textit{nei}: 72\% of all PVDs\is{Demonstrative!postverbal} in the text corpus are \textit{era}. This suggests that \textit{era} is the default PVD\is{Demonstrative!postverbal}; it is used whenever a PVD\is{Demonstrative!postverbal} is called for and there is no reason to use \textit{nei} or \textit{ena}. For this reason, the use of \textit{era} will only be discussed as it relates to \textit{nei} and \textit{ena}.

\subsection{Proximal \textit{nei}} \label{sec:7.6.2}
\is{nei (proximal)!postverbal|(}
\textit{Nei} marks actions which are either performed by the speaker, take place close to the speaker, or happen at a time close to the time of speech. Any of these is sufficient to warrant the use of \textit{nei}; neither is a necessary condition.

\textit{Nei} often marks an action performed by the speaker, i.e. in the first person, as in \REF{ex:7.154}.

\ea\label{ex:7.154}
\gll {\ꞌ}O ira a au \textbf{i} \textbf{iri} \textbf{mai} \textbf{nei} ki a koe.\\
because\_of \textsc{ana} \textsc{prop} \textsc{1sg} \textsc{pfv} ascend hither \textsc{prox} to \textsc{prop} \textsc{2sg}\\

\glt
‘Therefore I have come up to you.’ \textstyleExampleref{[R229.208]} 
\z

Alternatively, the event may take place near the speaker as in \REF{ex:7.155}, or is directed towards the location of the speaker as in \REF{ex:7.156}:

\ea\label{ex:7.155}
\gll Pē nei \textbf{e} \textbf{kī} \textbf{nei} e te nu{\ꞌ}u nei: ko mate {\ꞌ}ana koe. \\
like \textsc{prox} \textsc{ipfv} say \textsc{prox} \textsc{ag} \textsc{art} people \textsc{prox} \textsc{prf} die \textsc{cont} \textsc{2sg} \\

\glt 
‘This is what these people say: you have died.’ \textstyleExampleref{[R229.316]} 
\z

\ea\label{ex:7.156}
\gll ¿{\ꞌ}I hē rā a Vaha \textbf{e} ta{\ꞌ}e \textbf{tu{\ꞌ}u} \textbf{mai} \textbf{nei}? \\
~at \textsc{cq} \textsc{intens} \textsc{prop} Vaha \textsc{ipfv} \textsc{conneg} arrive hither \textsc{prox} \\

\glt
‘Where is Vaha, that he doesn’t arrive?’ \textstyleExampleref{[R229.131]} 
\z

Occasionally \textit{nei} has a temporal rather than a spatial function. In \REF{ex:7.157}, the speaker talks about something habitually taking place in the present.

\ea\label{ex:7.157}
\gll Te vaka o te hora nei, \textbf{e} \textbf{haha{\ꞌ}o} \textbf{nei} te aroaro {\ꞌ}i ruŋa  o te kavakava mau {\ꞌ}ana.\\
\textsc{art} boat of \textsc{art} time \textsc{prox} \textsc{ipfv} insert \textsc{prox} \textsc{art} lining at above  of \textsc{art} rib really \textsc{ident}\\

\glt 
‘The boats of nowadays, they put the lining on top of the ribs.’ \textstyleExampleref{[R200.068]} 
\z

In narrative contexts, events usually do not take place close to the speaker, nor in the present. Even so, \textit{nei} occurs in narrative as well. By using \textit{nei}, the speaker indicates that the action is spatially close to the locus of discourse, or takes place near the time of reference:

\ea\label{ex:7.158}
\gll Mahana nei \textbf{i} \textbf{iri} \textbf{nei} ki te māmoe mo toke he ma{\ꞌ}urima  o tō{\ꞌ}ona pāpātio.\\
day \textsc{prox} \textsc{pfv} ascend \textsc{pfv} to \textsc{art} sheep for steal \textsc{ntr} catch  of \textsc{poss.3sg.o} uncle\\

\glt 
‘This day when he went to the sheep to steal, his uncle caught him.’ \textstyleExampleref{[R250.222]} 
\z

\ea\label{ex:7.159}
\gll \textbf{I} \textbf{e{\ꞌ}a} \textbf{nei} te taŋata nei {\ꞌ}i tū ra{\ꞌ}ā era he oho mai ē... \\
\textsc{pfv} go\_out \textsc{prox} \textsc{art} man \textsc{prox} at \textsc{dem} day \textsc{dist} \textsc{ntr} go hither on\_and\_on \\

\glt
‘When this man had gone out that day, he kept going...’ \textstyleExampleref{[R310.136]} 
\z

As these examples show, \textit{nei} in the verb phrase may co-occur with \textit{nei} in the subject or another noun phrase in the clause (cf. also \REF{ex:7.155} above).
\is{nei (proximal)!postverbal|)}

\subsection{Medial \textit{ena}}\label{sec:7.6.3}
\is{ena (medial distance)!postverbal|(}
\textit{Ena} usually indicates an action performed by the addressee, an event taking place close to the addressee, or an event at a me\is{Proximal}dial distance (i.e. not near the speaker, but not very far either). Either of these factors may trigger the use of \textit{ena}.

Often \textit{ena} marks an action performed by the addressee:

\ea\label{ex:7.160}
\gll ¿He aha koe \textbf{e} \textbf{taŋi} \textbf{ena}? \\
~\textsc{pred} what \textsc{2sg} \textsc{ipfv} cry \textsc{med} \\

\glt
‘Why are you crying?’ \textstyleExampleref{[R229.185]} 
\z

Sometimes the action takes place near the addressee, as in \REF{ex:7.161}, or at a little distance from both speaker and addressee, as in \REF{ex:7.162}. 

\ea\label{ex:7.161}
\gll tā{\ꞌ}ana vānaŋa \textbf{kī} \textbf{atu} \textbf{ena} ki a koe \\
\textsc{poss.3sg.a} word say away \textsc{med} to \textsc{prop} \textsc{2sg} \\

\glt 
‘the words he spoke to you’ \textstyleExampleref{[R229.079]} 
\z

\ea\label{ex:7.162}
\gll Mo kōrua ho{\ꞌ}i e u{\ꞌ}i, ana tu{\ꞌ}u mai a Hare mai tō{\ꞌ}ona kona ena \textbf{e} \textbf{ŋaro} \textbf{mai} \textbf{ena}. \\
for \textsc{2pl} indeed \textsc{ipfv} look \textsc{irr} arrive hither \textsc{prop} Hare from \textsc{poss.3sg.o} place \textsc{med} \textsc{ipfv} disappear hither \textsc{med} \\

\glt
‘You will see whether Hare comes from the place where he has disappeared.’ \textstyleExampleref{[R229.276]} 
\z

Notice that in \REF{ex:7.162}, postverbal \textit{ena} is paralleled by \textit{ena} in the preceding noun phrase.

\textit{Ena} may also have a temporal function: it refers to a moment somewhat removed from the present. This may be the near past as in \REF{ex:7.163}, or the near future as in \REF{ex:7.164}:

\ea\label{ex:7.163}
\gll te ŋā me{\ꞌ}e nei au i tataku \textbf{i} \textbf{oho} \textbf{atu} \textbf{ena} \\
\textsc{art} \textsc{pl} thing \textsc{prox} \textsc{1sg} \textsc{pfv} tell \textsc{pfv} go away \textsc{med} \\

\glt 
‘the things I have (just) been telling about’ \textstyleExampleref{[R360.037]} 
\z

\ea\label{ex:7.164}
\gll He mana{\ꞌ}u nō te me{\ꞌ}e nei ō{\ꞌ}oku \textbf{ka} \textbf{kī} \textbf{atu} \textbf{ena}. \\
\textsc{pred} thought just \textsc{art} thing \textsc{prox} \textsc{poss.1sg.o} \textsc{cntg} say away \textsc{med} \\

\glt 
‘What I am about to say, is just a thought.’ \textstyleExampleref{[R361.015]} 
\z

In narrative, \textit{ena} is especially used after the deictic particle \textit{{\ꞌ}ī}\is{i (deictic)@{\ꞌ}ī (deictic)} (\sectref{sec:4.5.4.1.1}), which signals a shift to the point of view of a participant in the story (often with a verb of perception). The use of \textit{ena} in this construction may be metaphorical, indicating that the reader is conceptually closer to the events in the story than usual, looking as it were through the eyes of the participant.

\ea\label{ex:7.165}
\gll \textbf{{\ꞌ}}\textbf{Ī} \textbf{ka} \textbf{u{\ꞌ}i} \textbf{atu} \textbf{ena} ko te {\ꞌ}ata o te taŋata... \\
\textsc{imm} \textsc{cntg} look away \textsc{med} \textsc{prom} \textsc{art} shadow of \textsc{art} man \\

\glt 
‘Then he saw the shadow of a man...’\textstyleExampleref{ [R304.095]}\textstyleExampleref{} 
\z
\is{ena (medial distance)!postverbal|)}
\subsection{Neutral/distal \textit{era}~}\label{sec:7.6.4}
\is{era (distal)!postverbal|(}
\textit{Era} is the default PVD\is{Demonstrative!postverbal}. It is especially common in narrative contexts, where proximity to speaker and hearer does not play a role. \textit{Era} occurs in numerous examples in the discussion of aspectuals in \sectref{sec:7.2}. 

Other PVDs\is{Demonstrative!postverbal} are only used when there is a specific reason to do so. As discussed above, \textit{nei} is used when the action is performed by the speaker, takes place close to the speaker, takes place in the present, or is metaphorically proximate in discourse. Likewise, \textit{ena} can be used when the action is performed by or near to the hearer, takes place at a moderate distance, or at a time somewhat close to the present. This does not mean that \textit{nei} or \textit{ena} is always used whenever one of these conditions is fulfilled. \textit{Era}, being the default PVD\is{Demonstrative!postverbal}, can be used for an action performed by the speaker as in (\ref{ex:7.166}–\ref{ex:7.167}), or an action performed by the hearer as in (\ref{ex:7.168}–\ref{ex:7.169}). In all these cases, however, distance is involved: in (\ref{ex:7.167}–\ref{ex:7.169}) the event takes place in the past; in \REF{ex:7.166} the event is hypothetical, and therefore also removed from the here and now.

\ea\label{ex:7.166}
\gll ...{\ꞌ}ai au \textbf{e} \textbf{u{\ꞌ}i} \textbf{mai} \textbf{era} mai ruŋa ki a koe. \\
~~~there \textsc{1sg} \textsc{ipfv} look hither \textsc{dist} from above to \textsc{prop} \textsc{2sg} \\

\glt 
‘(If I were that bird,) I would look at you from above.’ \textstyleExampleref{[R245.155]} 
\z

\ea\label{ex:7.167}
\gll A au ho{\ꞌ}i \textbf{i} \textbf{raŋi} \textbf{atu} \textbf{era} ki a koe... \\
\textsc{prop} \textsc{1sg} indeed \textsc{pfv} call away \textsc{dist} to \textsc{prop} \textsc{2sg} \\

\glt 
‘Indeed, I called out to you...’ \textstyleExampleref{[R229.499]} 
\z

\ea\label{ex:7.168}
\gll Pē nei koe \textbf{i} \textbf{kī} \textbf{mai} \textbf{era} ki a au: he tu{\ꞌ}u mai a Vaha {\ꞌ}arīnā. \\
like \textsc{prox} \textsc{2sg} \textsc{pfv} say hither \textsc{dist} to \textsc{prop} \textsc{1sg} \textsc{ntr} arrive hither \textsc{prop} Vaha later\_today \\

\glt 
‘You said to me that Vaha would arrive today.’ \textstyleExampleref{[R229.147]} 
\z

\ea\label{ex:7.169}
\gll ¿Pē hē kōrua \textbf{e} \textbf{vānaŋa} \textbf{era}? \\
~like \textsc{cq} \textsc{2pl} \textsc{ipfv} talk \textsc{dist} \\

\glt 
‘What were you(pl) talking about?’ \textstyleExampleref{[Ley-2-02.062]}
\z

To summarise: PVDs\is{Demonstrative!postverbal} are used in combination with aspectuals to convey certain aspectual nuances. The default PVD\is{Demonstrative!postverbal} is \textit{era} (except in certain contexts with the perfective marker \textit{i}, where \textit{ai} is more common, see \sectref{sec:7.6.5}). \textit{Nei} and \textit{ena} may be used to convey proximity and medial distance respectively; distance is usually defined in spatial terms with respect to a participant or locus of discourse, but may also have a temporal sense.
\is{era (distal)!postverbal|)}
\subsection{Postverbal demonstratives with perfective \textit{i}}\label{sec:7.6.5}

\is{ai (postverbal)|(}The perfective marker \textit{i} was discussed in \sectref{sec:7.2.4}. The examples in that section show, that an \textit{i}{}-marked verb is usually followed by a postverbal demonstrative\is{Demonstrative!postverbal} (PVD\is{Demonstrative!postverbal}). Besides \textit{era}, \textit{nei} and \textit{ena}, \textit{i} (unlike other aspectuals) allows a fourth PVD, \textit{ai}. In fact, \textit{ai} is by far the most common PVD after \textit{i}. Only in cohesive clauses\is{Clause!cohesive} (\sectref{sec:11.6.2.1}) is the verb usually followed by \textit{era}, while \textit{ai} is rare.

This raises the question what the function of \textit{ai} could be. Now the particle \textit{ai} is common in Polynesian languages. Rapa Nui is different from other languages in that \textit{ai} is not used after all aspectuals; apart from perfective \textit{i}, it is only found after the purpose marker \textit{ki} (\sectref{sec:11.5.3}). There is also a functional difference. \citet{Chapin1974} shows that in all languages except Rapa Nui, \textit{ai} is anaphoric\is{Anaphora}: it occurs when the verb is preceded by any constituent other than a nominative subject; it serves as a substitute for the preposed constituent. This does not hold in Rapa Nui: in many cases \textit{ai} occurs in verb-initial clauses, or in clauses where the verb is preceded by a subject\is{Subject!preverbal}. Even so, there is a correlation between the occurrence of preverbal constituents and the use of \textit{ai}. \tabref{tab:54} shows the occurrence of \textit{ai} and other PVDs\is{Demonstrative!postverbal} in \textit{i}{}-marked clauses (cohesive clauses\is{Clause!cohesive} excepted), differentiated for preverbal constituents: either a core argument (subject\is{Subject!preverbal} or direct object\is{Object!preverbal}), an oblique constituent (locative or temporal phrase, connective adverb\is{Adverb}, or question word), or none at all (verb-initial clauses): 

\begin{table}
\begin{tabularx}{\textwidth}{L{35mm}R{11mm}R{9mm}R{11mm}R{9mm}R{11mm}R{9mm}}
\lsptoprule
{preverbal constituent} & \multicolumn{2}{c}{\textit{ai}} & \multicolumn{2}{c}{other PVD}\is{Demonstrative!postverbal}& \multicolumn{2}{c}{ {no PVD}}\is{Demonstrative!postverbal}\\
\midrule
core argument(s) & 21\% & (11) & 19\%  &(10) & 60\%  &(31)\\
oblique constituent & 80\% & (73) & 18\%  &(16) & 2\%  &(2)\\
Ø (verb-initial) & 72\%  &(55) & 7\% & (5) & 21\% & (16)\\
total && (139) && (31) && (49)\\
\lspbottomrule
\end{tabularx}
\caption{Postverbal demonstratives with \textit{i}-marked verbs}
\label{tab:54}
\end{table}

As this table shows, when the verb is preceded by an oblique constituent, it is followed by \textit{ai} in 80\% of the cases. By contrast, when the verb is preceded by a core argument, \textit{ai} is relatively rare (21\%), while 60\% of the cases have no PVD\is{Demonstrative!postverbal} at all. These statistics show a similarity in the use of \textit{ai} between Rapa Nui and other Polynesian languages: \textit{ai} tends to be used after oblique constituents, but not after NP arguments.\footnote{\label{fn:369}Notice that, different from what \citet{Chapin1974} found in other languages, in Rapa Nui any NP argument, whether subject or object, disfavours the use of \textit{ai}.

In fact, \citet[299]{Chapin1974} found a similar correlation: counting occurrences of \textit{era} and \textit{ai} in Englert’s stories (Egt), concludes: “of the 26 cases discovered of verbs in \textit{i} tense with no PVD\is{Demonstrative!postverbal}, all but three or possibly four contain patterns which would lead one on comparative\is{Comparative} grounds not to expect \textit{ai.} Of the nearly 100 cases of post-verbal \textit{ai,} all but about a dozen appear according to comparative expectations.”} Still, the situation is much fuzzier than in other languages: \textit{ai} does occur after NP arguments, while after oblique constituents other PVDs\is{Demonstrative!postverbal} occur as well as \textit{ai}. 

In verb-initial clauses, \textit{ai} is almost as common as with oblique preverbal constituents (72\%), a situation not found in other languages. Possibly the use of \textit{ai} in these clauses can be explained to some extent in terms of inter-clausal (rather than intra-clausal) anaphora\is{Anaphora}. For example, in \REF{ex:7.170} \textit{ai} could be explained as providing an anaphoric\is{Anaphora} link with the preceding clause.

\ea\label{ex:7.170}
\gll ¿I mamae rō koe \textbf{i} \textbf{hiŋa} \textbf{ai}? \\
~\textsc{pfv} pain \textsc{emph} \textsc{2sg} \textsc{pfv} fall \textsc{pvp} \\

\glt
‘Did you get hurt when you fell down?’ \textstyleExampleref{[R481.131]} 
\z

On the other hand, many examples of \textit{ai} cannot be explained in this way.

Turning now to the other PVDs\is{Demonstrative!postverbal} \textit{nei}, \textit{ena} and \textit{era}, these are relatively rare with \textit{i}{}-marked verbs (except in cohesive clauses\is{Clause!cohesive}, see \sectref{sec:11.6.2.1}). As \tabref{tab:54} shows, out of 219 verbs, only 31 (14\%) are followed by one of these. Of these 31 cases, 13 have \textit{nei}, 10 have \textit{ena}, 8 have \textit{era}. These proportions are remarkable, as \textit{era} is much more frequent in general than \textit{nei} and \textit{ena}: as the statistics in \sectref{sec:7.6.1} show, \textit{era} accounts for 72\% of all occurrences of these three PVDs\is{Demonstrative!postverbal} overall, but in the constructions considered here, \textit{era} represents only 26\% of all three PVDs\is{Demonstrative!postverbal}. Even though the sample is small and therefore liable to skewing by a few aberrant examples, the difference is significant.

In other contexts, \textit{era} is the default PVD\is{Demonstrative!postverbal}; \textit{nei} and \textit{ena} are only used when there is a specific reason to use them, to indicate close distance (\textit{nei}) or medial distance (\textit{ena}) (\sectref{sec:7.6}). By contrast, with \textit{i}{}-marked verbs, \textit{ai} is the default PVD\is{Demonstrative!postverbal}. \textit{Nei} and \textit{ena} may be used to indicate close and medial distance; \textit{era} may either be a free (but relatively rare) alternative to \textit{ai}, or used only when the speaker wishes to emphasise distance. 
\is{Demonstrative!postverbal|)}\is{ai (postverbal)|)}
\section{Serial verb constructions}\label{sec:7.7}
\is{Serial verb|(}\subsection{Introduction}\label{sec:7.7.1}

Serial verb\is{Serial verb} constructions (SVCs) are constructions in which two or more verbs occur in a single clause, without being so closely linked that they form a verbal compound\is{Compound}.\footnote{\label{fn:370}On SVCs in general see \citet{Durie1988,Durie1997}; \citet{AikhenvaldDixon2006}; \citet{Sebba1987}. On SVCs in Oceanic languages, see \citet{Crowley2002} (+ reviews by \citet{Owens2002} and \citet{Bradshaw2004}); \citet{Senft2008}. These studies do not agree on a precise definition of SVCs (it is even uncertain if such a definition is possible, given the crosslinguistic variation in syntax and semantics of SVCs \citep[19]{Crowley2002}). They differ for example on the question whether SVCs necessarily constitute a single predicate. However, they do agree on the characteristics mentioned here.} Verbs within an SVC have the same specification for tense/aspect/mood and they usually share one or more arguments. They are not separated by a conjunction\is{Conjunction} or by anything marking a clause boundary. The events expressed within an SVC are closely linked: SVCs tend to express a single event, or a set of events considered to be part of a single “macro-event”.\footnote{\label{fn:371}The term \textit{macro-event} is discussed by \citet{Aikhenvald2006}.}  Certain verb combinations may be lexicalised in a language, but SVCs tend to be productive.

Serialisation is common in Austronesian languages, including Oceanic languages (see \citealt{Crowley2002}; \citealt{Senft2008}; \citealt{Durie1988}), but rare in Polynesian.\footnote{\label{fn:372}Both in \ili{Māori} \citep[150]{Harlow2007Maori} and \ili{Tahitian} (\citet[203]{AcadémieTahitienne1986}), the only traces of SVCs are motion verbs\is{Verb!motion} such as \textit{haere} ‘go’ modifying another verb: \ili{Māori} \textit{i tangi haere} ‘went weeping’. In \ili{Marquesan}, this modifying construction also occurs (\citealt[205–206]{Cablitz2006}). Cablitz also mentions bare complement clauses and clause chaining as examples of serialisation; however, clause chaining constructions are not monoclausal, hence they do not qualify as an SVC as defined above. The only reason to classify clause chaining constructions as SVCs is the absence of an A/M marker on the second verb. \citet[397]{MoselHovdhaugen1992}, using the same criterion, identify the same three constructions as SVCs in \ili{Samoan}. Finally, in \ili{Tuvaluan} \citep[538]{Besnier2000} SVCs occur on a limited scale; again, the second verb is not A/M-marked.} 

\subsection{The syntax of SVCs in Rapa Nui}\label{sec:7.7.2}

Rapa Nui is unusual among Polynesian languages in that SVCs are fairly common.\footnote{\label{fn:373}SVCs in Rapa Nui are discussed in \citet[67-75]{WeberR2003}; Weber uses the term \textit{verb nesting} (anidación de verbos) and especially discusses criteria to distinguish SVCs from clause conjunction\is{Conjunction}.} Moreover, it is – to my knowledge – the only Polynesian language in which all verbs in an SVC have an aspect/mood (A/M) marker; the A/M markers within an SVC are always identical.\footnote{\label{fn:374}While bare verbs can modify nouns (\sectref{sec:5.7.1}), they never modify other verbs without a preceding A/M marker. Some Polynesian languages have a V + V construction (see Footnote \ref{fn:372} above), but this does not occur in Rapa Nui.}

Apart from the A/M marker, nothing can occur between the verbs in an SVC. Postverbal particles – including obligatory particles – only occur after the last verb. Arguments of both verbs are placed after the last verb; preposed arguments occur before the first verb. The structure of a clause with serialisation is thus as follows:

\ea\label{ex:7.170a}
(constituent) ~ ~ [A/M\textsubscript{i}   V1   A/M\textsubscript{i}   V2   (particles) ]\textsubscript{VP} ~ ~  (constituents)
\z

Most SVCs have two verbs, but longer series occur. The verbs in an SVC usually share their S/A argument. In fact, the SVC as a whole has a single argument structure, which is determined by the verb with the highest valency: if both verbs are intransitive\is{Verb!intransitive}, the SVC as a whole is intransitive\is{Verb!intransitive}; if one verb is transitive\is{Verb!transitive}, the SVC is transitive\is{Verb!transitive}.

Below are a number of examples which illustrate the characteristics of SVCs. 

\ea\label{ex:7.171}
\gll {\ꞌ}Ēē, \textbf{ko} \textbf{ma{\ꞌ}u} \textbf{ko} \textbf{hoki} {\ꞌ}ā ki tō{\ꞌ}ona kona. \\
yes \textsc{prf} carry \textsc{prf} return \textsc{cont} to \textsc{poss.3sg.o} place \\

\glt 
‘Yes, they carried it back to its place.’ \textstyleExampleref{[R413.844]} 
\z

\ea\label{ex:7.172}
\gll \textbf{I} \textbf{hoki} \textbf{i} \textbf{turu} mai era ararua a rā {\ꞌ}ā... \\
\textsc{pfv} return \textsc{pfv} go\_down hither \textsc{dist} the\_two by \textsc{dist} \textsc{ident} \\

\glt 
‘When they returned together (with downwards movement) by that place...’ \textstyleExampleref{[R245.210]} 
\z

\ea\label{ex:7.173}
\gll \textbf{He} \textbf{ma{\ꞌ}u} \textbf{he} \textbf{iri} \textbf{he} \textbf{oho} i tū manu era ki te hare pure.\\
\textsc{ntr} carry \textsc{ntr} ascend \textsc{ntr} go \textsc{acc} \textsc{dem} animal \textsc{dist} to \textsc{art} house pray\\

\glt 
‘They carried the animal up to the church.’ \textstyleExampleref{[R178.053]} 
\z

\ea\label{ex:7.174}
\gll ...\textbf{{\ꞌ}o} \textbf{pe{\ꞌ}e} \textbf{{\ꞌ}o} \textbf{oho} te māuiui ki tētahi nu{\ꞌ}u sano ena e noho mai ena. \\
~~~lest infect lest go \textsc{art} sick to other people healthy \textsc{med} \textsc{ipfv} stay hither \textsc{med} \\

\glt
‘...lest the disease keeps infecting other people who are still healthy.’ \textstyleExampleref{[R398.017]} 
\z

Aspectuals – \textit{ko} in \REF{ex:7.171}, \textit{i} in \REF{ex:7.172}, \textit{he} in \REF{ex:7.173} – are repeated before each verb. Postverbal \textit{{\ꞌ}ā} \REF{ex:7.171} and \textit{era} \REF{ex:7.172} occur only after the second verb.\footnote{\label{fn:375}Both are obligatory, given the construction: perfect \textit{ko} is always accompanied by \textit{{\ꞌ}ā} (\sectref{sec:7.2.7}), while \textit{i V era} in \REF{ex:7.172} marks a cohesive clause (\sectref{sec:11.6.2.1}).} In \REF{ex:7.173} – a tripartite SVC – \textit{i tū manu era} is the direct object of the first verb \textit{ma{\ꞌ}u}, yet it occurs after the SVC construction as a whole. \REF{ex:7.174} shows that subordinators like \textit{{\ꞌ}o} are repeated in the same way as aspectuals. 

These examples also show how SVCs can be distinguished from coordinated clauses\is{Coordination}. As verb arguments are often omitted in discourse, a string of verbal clauses may consist of just \textit{A/M V A/M V...} (see e.g. \REF{ex:7.3} on p.~\pageref{ex:7.3}); such a string may at first sight be indistinguishable from an SVC. Diagnostics for SVCs are: the omission of postverbal particles after the first verb, and the placement of the direct object of a verb after the next verb (even when the latter is intransitive\is{Verb!intransitive}). SVCs can also be recognised by semantic criteria, as they often express a single event; this will be discussed in the next section.

In nominalised\is{Verb!nominalised} SVCs, the determiner is repeated. Any preposition preceding the nominalised verb\is{Verb!nominalised} is repeated as well, e.g. \textit{pe} in \REF{ex:7.176}:\footnote{\label{fn:376}\REF{ex:7.175} is an habitual\is{Aspect!habitual} actor-emphatic\is{Actor-emphatic construction} construction (\sectref{sec:8.6.3}).}

\ea\label{ex:7.175}
\gll O te naonao toretore \textbf{te} \textbf{haka} \textbf{pe{\ꞌ}e} \textbf{te} \textbf{oho} te māuiui he renke. \\
of \textsc{art} mosquito stripe \textsc{art} \textsc{caus} infect \textsc{caus} go \textsc{art} sick \textsc{pred} dengue \\

\glt 
‘It is the striped mosquito which keeps spreading dengue disease.’ \textstyleExampleref{[R535.051]} 
\z

\ea\label{ex:7.176}
\gll Te pūai, pa{\ꞌ}i, o rāua \textbf{pe} \textbf{te} \textbf{ŋaro} \textbf{pe} \textbf{te} \textbf{oho} nō.\\
\textsc{art} strong in\_fact of \textsc{3pl} toward \textsc{art} disappear toward \textsc{art} go just\\

\glt
‘Their power will gradually disappear.’ \textstyleExampleref{[1 Cor. 2:6]}
\z

There is only one situation in which V2 is unmarked: when the SVC functions as a bare relative clause\is{Clause!relative!bare} (\sectref{sec:11.4.5}), in which case neither verb in the SVC has an A/M marker.

\ea\label{ex:7.177}
\gll {\ꞌ}I te hora \textbf{turu} \textbf{oho} nei ō{\ꞌ}oku ki Haŋa Roa o Tai... \\
at \textsc{art} time go\_down go \textsc{prox} \textsc{poss.1sg.o} to Hanga Roa o Tai \\

\glt 
‘When I went down to Hanga Roa o Tai...’ \textstyleExampleref{[R230.059]} 
\z

\subsection{Semantics of SVCs}\label{sec:7.7.3}

Most SVCs refer to a single event, which is expressed by one verb (usually the first in the series) and modified in some way by the other verb(s) (categories 1–3 below).\footnote{\label{fn:377}The same is true crosslinguistically: directional\is{Directional} and aspectual SVCs are very common (\citealt{Aikhenvald2006}, who also mentions all the other categories found in Rapa Nui: manner, synonymy, sequential events)} Other SVCs express a series of closely connected events which are conceived as one macro-event (category 4).

\subparagraph{\ref{sec:7.7.3}.1~ Aspect} V2 may express an aspectual specification of the event. Only two verbs are used in this way.

\textit{Oho}\is{oho ‘to go’} ‘go’ is by far the most common V2 in SVCs. It often expresses extended duration\is{Aspect!continuous}, indicating that the action expressed by V1 goes on for a while. As \REF{ex:7.180} shows, when V1 is an adjective, the SVC expresses an ongoing process.

\ea\label{ex:7.178}
\gll Pē rā nō \textbf{e} \textbf{kai} \textbf{e} \textbf{oho} era. \\
like \textsc{dist} just \textsc{ipfv} eat \textsc{ipfv} go \textsc{dist} \\

\glt 
‘In that way he kept eating.’ \textstyleExampleref{[R310.225]} 
\z

\ea\label{ex:7.179}
\gll \textbf{I} \textbf{ta{\ꞌ}o} \textbf{i} \textbf{oho} nō \textbf{i} \textbf{ta{\ꞌ}o} \textbf{i} \textbf{oho} nō. \\
\textsc{pfv} cook \textsc{pfv} go just \textsc{pfv} cook \textsc{pfv} go just \\

\glt 
‘He just kept cooking and cooking.’ \textstyleExampleref{[R352.077]} 
\z

\ea\label{ex:7.180}
\gll He rahi te ta{\ꞌ}u {\ꞌ}e \textbf{he} \textbf{nuinui} \textbf{he} \textbf{oho} tū manu era. \\
\textsc{ntr} many \textsc{art} year and \textsc{ntr} big:\textsc{red} \textsc{ntr} go \textsc{dem} bird \textsc{dist} \\

\glt
‘Many years passed and the bird grew up (got bigger and bigger).’ \textstyleExampleref{[R447.012]} 
\z

\textit{Oti}\is{oti ‘to finish’} ‘finish’ is usually constructed with a complement clause (\sectref{sec:11.3.2.2}), but it may also function as V2 of an SVC, indicating that an action or process is completely carried out:

\ea\label{ex:7.181}
\gll \textbf{I} \textbf{tuha{\ꞌ}a} \textbf{i} \textbf{oti} era e Kaiŋa i tū kai era... \\
\textsc{prf} distribute \textsc{pfv} finish \textsc{dist} \textsc{ag} Kainga \textsc{acc} \textsc{dem} food \textsc{dist} \\

\glt 
‘When Kainga had finished distributing (=completely distributed) the food...’ \textstyleExampleref{[R304.116]} 
\z

\ea\label{ex:7.182}
\gll {\ꞌ}I te toru mahana \textbf{ko} \textbf{para} \textbf{ko} \textbf{oti} {\ꞌ}ana. \\
at \textsc{art} three day \textsc{prf} ripe \textsc{prf} finish \textsc{cont} \\

\glt 
‘On the third day (the bananas) are completely ripe.’ \textstyleExampleref{[R539-2.071]}
\z

\subparagraph{\ref{sec:7.7.3}.2~ Direction} V2 may be a motion verb\is{Verb!motion} specifying the direction in which the action expressed by V1 takes place. The motion verb\is{Verb!motion} may be \textit{iri} ‘go up’, \textit{turu} ‘go down’ or \textit{hoki} ‘go back, return’. The idea of movement itself may be expressed by V1 (e.g. \textit{haro} in \REF{ex:7.183}), but in other cases such as \REF{ex:7.184}, V1 by itself does not express movement.

\ea\label{ex:7.183}
\gll \textbf{I} \textbf{haro} \textbf{i} \textbf{iri} era he tu{\ꞌ}u ki ruŋa. \\
\textsc{pfv} pull \textsc{pfv} ascend \textsc{dist} \textsc{ntr} arrive to above \\

\glt 
‘When they had pulled (the net) up, it arrived on top.’ \textstyleExampleref{[R304.136]} 
\z

\ea\label{ex:7.184}
\gll \textbf{He} \textbf{kai} \textbf{he} \textbf{turu} i tā{\ꞌ}ana tūava. \\
\textsc{ntr} eat \textsc{ntr} go\_down \textsc{acc} \textsc{poss.3sg.a} goyava \\

\glt 
‘He went down, eating his goyavas.’ \textstyleExampleref{[R245.024]} 
\z

\ea\label{ex:7.185}
\gll Mo haŋa o kōrua, \textbf{he} \textbf{ma{\ꞌ}u} \textbf{he} \textbf{hoki} kōrua e au ki Tahiti. \\
if want of \textsc{2pl} \textsc{ntr} carry \textsc{ntr} return \textsc{2pl} \textsc{ag} \textsc{1sg} to Tahiti \\

\glt
‘If you (pl) want, I’ll take you back to Tahiti.’ \textstyleExampleref{[R231.102]} 
\z

\textit{Oho}\is{oho ‘to go’} ‘go’ is mostly used in SVCs to express duration (see 1 above); however, it may also express motion in a certain direction, without specifying the direction itself.\footnote{\label{fn:378}These examples are somewhat similar to category 3, in which the V1 specifies the manner in which V2 is performed. The difference is, that \textit{kau} and \textit{nekeneke} are themselves motion verbs\is{Verb!motion}, while the modifying verbs in category 3 are statives.} In these cases, no extensive duration is implied.

\ea\label{ex:7.186}
\gll Hora nei ho{\ꞌ}i \textbf{ku} \textbf{kau} \textbf{ku} \textbf{oho} mai {\ꞌ}ana ananake ki {\ꞌ}uta. \\
time \textsc{prox} indeed \textsc{prf} swim \textsc{prf} go hither \textsc{cont} together to inland \\

\glt 
‘Now they had swum to the shore together.’ \textstyleExampleref{[R361.032]} 
\z

\ea\label{ex:7.187}
\gll \textbf{I} \textbf{nekeneke} \textbf{i} \textbf{oho} mai era a tu{\ꞌ}a, he tito e tū {\ꞌ}uha era. \\
\textsc{pfv} crawl:\textsc{red} \textsc{pfv} go hither \textsc{dist} by back \textsc{ntr} peck \textsc{ag} \textsc{dem} chicken \textsc{dist} \\

\glt 
‘When he crawled backwards, the chicken pecked him.’ \textstyleExampleref{[R250.160]} 
\z

\subparagraph{\ref{sec:7.7.3}.3~ Manner} One verb in the SVC may be a stative verb indicating the manner in which the action expressed by the other verb is carried out. Usually the stative verb comes first, while the event itself is expressed by V2. 

\ea\label{ex:7.188}
\gll ...\textbf{i} \textbf{ke{\ꞌ}oke{\ꞌ}o} \textbf{i} \textbf{topa} \textbf{mai} \textbf{ai} mai ruŋa i tū tumu era. \\
~~~\textsc{pfv} hurry:\textsc{red} \textsc{pfv} descend hither \textsc{pvp} from above at \textsc{dem} tree \textsc{dist} \\

\glt 
‘...she hurried down from the tree.’ \textstyleExampleref{[R496.045]} 
\z

\ea\label{ex:7.189}
\gll \textbf{E} \textbf{hekaheka} \textbf{e} \textbf{eke} \textbf{e} \textbf{oho} nō {\ꞌ}ana te ika i haka hōriŋa rō ai. \\
\textsc{ipfv} soft:\textsc{red} \textsc{ipfv} ascend \textsc{ipfv} go just \textsc{cont} \textsc{art} fish \textsc{pfv} \textsc{caus} weary \textsc{emph} \textsc{pvp} \\

\glt 
‘The fish kept coming up easily, until it got tired of it.’ \textstyleExampleref{[R361.053]} 
\z

\subparagraph{\ref{sec:7.7.3}.4~ Other} In other cases, both verbs describe an event. The verbs may be closely related or near-synonyms as in \REF{ex:7.190}, both expressing the same event under different angles; they may also describe different aspects of the same event as in \REF{ex:7.191} (‘ask in writing’ or ‘write to ask’). Alternatively, they express sequential events considered to be part of the same macro-event, as in \REF{ex:7.192}.

\ea\label{ex:7.190}
\gll \textbf{Ko} \textbf{veveri} \textbf{ko} \textbf{{\ꞌ}ara} {\ꞌ}ana a au {\ꞌ}i te kona nei {\ꞌ}ana. \\
\textsc{prf} startled \textsc{prf} wake\_up \textsc{cont} \textsc{prop} \textsc{1sg} at \textsc{art} place \textsc{prox} \textsc{ident} \\

\glt 
‘I woke up with a start in this same place.’ \textstyleExampleref{[R539-1.764]}
\z

\ea\label{ex:7.191}
\gll Hora nei he pāhono atu au i te me{\ꞌ}e \textbf{pāpa{\ꞌ}i} \textbf{{\ꞌ}ui} mai era e kōrua.\\
time \textsc{prox} \textsc{ntr} answer away \textsc{1sg} \textsc{acc} \textsc{art} thing write ask hither \textsc{dist} \textsc{ag} \textsc{2pl}\\

\glt 
‘Now I will answer the things you wrote (and) asked me.’ \textstyleExampleref{[1 Cor. 7:1]}
\z

\ea\label{ex:7.192}
\gll \textbf{Ko} \textbf{te} \textbf{kimi} \textbf{ko} \textbf{te} \textbf{ohu} a nua. \\
\textsc{prom} \textsc{art} search \textsc{prom} \textsc{art} shout \textsc{prop} Mum \\

\glt 
‘Mum searched (the child), shouting.’ \textstyleExampleref{[R236.082]} 
\z
\is{Serial verb|)}
\section{Conclusions}\label{sec:7.8}

Verbs are preceded and followed by a range of particles which specify the event for aspect, mood, distance and direction. 

Aspect is primarily indicated by a set of five preverbal markers; the use of these markers is obligatory, unless the verb is preceded by a different marker (such as mood and negation) occurring in the same position. The aspectual markers are as follows: neutral \textit{he}, perfective \textit{i}, imperfective \textit{e}, contiguity \textit{ka} and perfect \textit{ko V {\ꞌ}ā}. The neutral marker \textit{he} is by far the most common one. It marks events which receive their aspectual value from the context in some way; in the absence of other contextual clues, a string of \textit{he-}clauses expresses sequential events in discourse.

The relationship between perfective \textit{i} and perfect \textit{ko V {\ꞌ}ā} calls for an explanation. Comparison of both markers in similar contexts suggest that \textit{ko V {\ꞌ}ā} is state-oriented, while \textit{i} is event-oriented. The state-oriented character of \textit{ko V {\ꞌ}ā} also shows up in its widespread use to mark a state which pertains at a time of reference (usually the present). This happens with typically stative verbs such as ‘be hot, big, poor, mad...’, but also with verbs of volition and cognition.

Finer aspectual distinctions are indicated by postverbal particles; different classes of particles play a role with different aspectuals:

\begin{itemize}
\item 
the evaluative marker \textit{rō}: \textit{e V rō} marks future, \textit{ka V rō} marks a temporal boundary ‘until’, etc.

\item 
postverbal demonstratives: \textit{i V era} marks perfective temporal clauses; \textit{e V era} marks habitual or continuous clauses;

\item 
the continuity marker \textit{{\ꞌ}ā}: \textit{e V {\ꞌ}ā} marks continuous or stative clauses.

\end{itemize}

One pair of postverbal particles operates entirely independently from aspect marking: the directional markers \textit{mai} and \textit{atu}. In direct speech, \textit{mai} indicates movement towards the speaker, while the use of \textit{atu} is varied: movement from the speaker towards the hearer, away from speaker and hearer, or from an unspecified source towards the hearer. 

In third-person discourse, the use of \textit{mai} and \textit{atu} marks a deictic centre. The speaker has a high degree of freedom in defining the deictic centre: it may be relatively fixed (often depending on the location of one or more protagonists in the story) or shift rapidly between different locations. Statistics show some general trends, though: with motion verbs, directionals tend to point to the destination of movement as the deictic centre. With perception verbs, there is a difference between controlled perception (‘to look, listen’) and uncontrolled perception (‘to see, hear’): with the former, directionals indicate a movement from the experiencer to the stimulus, i.e. directed attention; with the latter the direction is reversed, i.e. directionals signal the movement from the stimulus towards the experiencer.

\largerpage[2]
Finally, Rapa Nui is the only Polynesian language having a serial verb construction in which the preverbal marker is repeated. Apart from the preverbal marker, nothing may occur between the verbs in this construction. Serial verb constructions form a single predicate with a single argument structure; they often express a single event.
\is{Verb phrase|)}
