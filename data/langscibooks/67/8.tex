%8

\chapter{Clause combinations} \label{chap:8}
%\hypertarget{RefHeading22901935131865}

Some linguistic models, the mainstream generative grammar in particular, disregard the distinction between a clause and a sentence, but here the distinction is maintained. One of the main reasons is the medial clause system operating in Mauwake.  A simple sentence in Mauwake consists of one clause, but if that is a verbal clause, it must be a finite clause, not a medial one, as medial clauses only function within a sentence in combination with other clauses. Their distribution is restricted to non-final position in a sentence -- they may occur sentence-finally only if they are dislocated or the final clause is ellipted. Medial clauses also add the chaining structure to the clause combination possibilities (\sectref{sec:8.2}), besides regular coordination (\sectref{sec:8.1}) and subordination (\sectref{sec:8.3}). 

A sentence has the following features. It consists of one or more clauses. The end of a sentence is marked in speech by a falling intonation, or by a slightly rising intonation in polar questions, and normally a pause. The sentence-final falling intonation is clear, and can be distinguished from a less noticeable fall at the end of a non-final finite clause. In writing the end of a sentence is marked by a full stop, a question mark or an exclamation mark.

A simple sentence is the same as a clause, and was discussed in \chapref{chap:5}. When two main clauses are joined in a coordinate sentence, they are independent of each other as to their functional sentence type. In \REF{ex:8:x1352} the first clause is declarative and the second one interrogative; in \REF{ex:8:x1358} the first clause is imperative and the second one declarative, but the order could also be reversed. 

\ea%x1352
\label{ex:8:x1352}
\gll Yo  owora=ko  me  aaw-e-m,  no  moram  efa ma-i-n?\\
1s.\textsc{unm}  betelnut=\textsc{nf}  not  take-\textsc{pa}-1s  2s.\textsc{unm}  why  1s.\textsc{acc} say-\textsc{Np}-\textsc{pr}.2s\\
\glt `I didn't take the betelnut, why do you accuse me?'
\z


\ea%x1358
\label{ex:8:x1358}
\gll Ni  uf-owa  ikiw-eka,  yo  miatin-i-yem. \\
2p.\textsc{unm}  dance-\textsc{nmz}  go-\textsc{imp}.2p  1s.\textsc{unm}  dislike-\textsc{Np}-\textsc{pr}.1s\\
\glt `(You) go to dance, I don't want to.'
\z

In clause chaining (\sectref{sec:8.2}) and in complex clauses involving main and subordinate\linebreak clauses (\sectref{sec:8.3}), the situation is more complicated. Formally almost all of the subordinate and medial clauses are neutral/declarative. A subordinate clause typically lacks an illocutionary force of its own \citep[32]{Cristofaro2003} and conforms to the functional sentence type of the main clause. In the examples \REF{ex:8:x1357}--\REF{ex:8:x1898}, the subordinate clauses are in brackets.

\ea%x1357
\label{ex:8:x1357}
\gll {\ob}Ni  ifa  nia  keraw-i-ya  nain{\cb}  sira  kamenap on-i-man?\\
2p.\textsc{unm}  snake  2p.\textsc{acc}  bite-\textsc{Np}-\textsc{pr}.3s  that1  custom  what.like do-\textsc{Np}-\textsc{pr}.2p\\
\glt `When a snake bites you, what do you do?'
\z


\ea%x1897
\label{ex:8:x1897}
\gll Ni  {\ob}yapen  ...  wiar  in-em-ik-e-man  nain{\cb} kerer-omak-eka!\\
2p.\textsc{unm}  inland  {\dots}  3.\textsc{dat}  sleep-\textsc{ss}.\textsc{sim}-be-\textsc{pa}-2p  that1 arrive-\textsc{distr/pl}-2p.\textsc{imp}\\
\glt `Those (many) of you, who have stayed inland, arrive (back in your villages)!'
\z


\ea%x1898
\label{ex:8:x1898}
\gll {\ob}Ni  uf-ep-na{\cb}  ni  maadara  me iirar-eka.\\
2p.\textsc{unm} dance-\textsc{ss}.\textsc{seq}=\textsc{tp}  2p.\textsc{unm}  forehead.ornament  not remove-2p.\textsc{imp}\\
\glt `If/when you have danced, do not remove your forehead ornaments.'
\z

The non-polar questions are an exception, since the question word may also be in a subordinate clause \REF{ex:8:x1362}. When a subordinate clause contains a question word, the illocutionary force of a question spreads to whole sentence. 

\ea%x1362
\label{ex:8:x1362}
\gll No  {\ob}\textstyleEmphasizedVernacularWords{kaaneke}  \textstyleEmphasizedVernacularWords{ikiw-owa}{\cb}  efa  maak-i-n?\\
2s.\textsc{unm}  where.\textsc{cf}  go-\textsc{nmz}  1s.\textsc{unm}  tell-\textsc{Np}-\textsc{pr}.2s\\
\glt `You are telling me to go where?'
\z

A medial clause is coordinate with the main clause but dependent on it (\sectref{sec:8.2}). The imperative form is only possible in finite verbs, and the polar question marker only occurs sentence-finally.\footnote{As an alternative marker, the \textsc{qm} is used in non-final clauses as well (\sectref{sec:3.12.8}, \sectref{sec:8.1.2}).} Because of these formal restrictions, it is impossible to have an imperative or interrogative medial clause coordinated with a declarative main clause. A medial clause commonly conforms to the illocutionary force of the final clause, but it does not need to do so. In the examples \REF{ex:8:x1899} and \REF{ex:8:x1900} the bracketed medial clause is questioned with the main clause, in \REF{ex:8:x1901} and \REF{ex:8:x1902} it is not.  

\ea%x1899
\label{ex:8:x1899}
\gll {\ob}Maamuma  uruf-ap{\cb}  ma-i-n-i? \\
money  see-\textsc{ss}.\textsc{seq}  say-\textsc{pa}-2s=\textsc{qm}\\
\glt `Have you seen the money and (so) ask?'
\z

\ea%x1900
\label{ex:8:x1900}
\gll {\ob}Yo  pina  on-amkun=ko{\cb}  efa  uruf-a-man=i?\\
1s.\textsc{unm}  guilt  do-1s/p.\textsc{ds}=\textsc{nf}  2s.\textsc{acc}  see-\textsc{pa}-2p=\textsc{qm}\\
\glt `Did I do wrong and you saw me?'
\z

\ea%x1901
\label{ex:8:x1901}
\gll {\ob}Sande  erup  weeser-eya{\cb}  owowa  ekap-e-man=i? \\
week  two  finish-2/3s.\textsc{ds}  village  come-\textsc{pa}-2p=\textsc{qm}\\
\glt `When two weeks were finished, did you (then) come to the village?'
\z

\ea%x1902
\label{ex:8:x1902}
\gll {\ob}...ikoka  ekap-ep{\cb}  sira  nain  piipua-i-nan=i  e  weetak? \\
later  come-\textsc{ss}.\textsc{seq}  habit  that1  leave-\textsc{Np}-\textsc{fu}.2s=\textsc{qm}  or  no\\
\glt `{\dots}later when you come, will you drop that habit or not?'
\z


When a medial clause itself contains a question word, the illocutionary force spreads to the whole sentence \REF{ex:8:x1363}, \REF{ex:8:x1903}. 

\ea%x1363
\label{ex:8:x1363}
\gll {\ob}\textstyleEmphasizedVernacularWords{No}  \textstyleEmphasizedVernacularWords{maa}  \textstyleEmphasizedVernacularWords{mauwa}  \textstyleEmphasizedVernacularWords{uruf-ap}{\cb}  soran-ep  kirir-e-n?\\
2s.\textsc{unm}  thing  what  see-\textsc{ss}.\textsc{seq}  be.startled-\textsc{ss}.\textsc{seq}  shout-\textsc{pa}-2s\\
\glt `What did you see and (then) got startled and shouted?'
\z

\ea%x1903
\label{ex:8:x1903}
\gll {\ob}\textstyleEmphasizedVernacularWords{Naareke}  \textstyleEmphasizedVernacularWords{nia}  \textstyleEmphasizedVernacularWords{maak-eya}{\cb}  ekap-e-man? \\
who.\textsc{cf}  2p.\textsc{acc}  tell-2/3s.\textsc{ds}  come-\textsc{pa}-2p\\
\glt `Who told you to come?' (Lit: `Who told you and you came?')
\z


When the final clause is in the imperative mood, the implication of a command often extends backwards to a medial verb marked for the same subject \REF{ex:8:x1365}, but not so easily to one marked for a different subject. In (7.\ref{ex:7:x1364}) above, the command /request extends to the medial clause, whereas in \REF{ex:8:x1356} it does not. For more examples, see (7.\ref{ex:7:x1082})--(7.\ref{ex:7:x1083}) above.

\ea%x1365
\label{ex:8:x1365}
\gll {\ob}\textstyleEmphasizedVernacularWords{No}  \textstyleEmphasizedVernacularWords{nena}  \textstyleEmphasizedVernacularWords{maa}  \textstyleEmphasizedVernacularWords{fariar-ep}{\cb}  \textstyleEmphasizedVernacularWords{muuka}  \textstyleEmphasizedVernacularWords{nain} \textstyleEmphasizedVernacularWords{arim-ow-e}.\\
2s.\textsc{unm}  2s.\textsc{gen}  food  abstain-\textsc{ss}.\textsc{seq}  son  that1 grow-\textsc{caus}-\textsc{imp}.2s\\
\glt `Abstain from (certain) food(s) and bring up the son.'  
\z


\ea%x1356
\label{ex:8:x1356}
\gll {\ob}Nefa  war-iwkin{\cb}  \textstyleEmphasizedVernacularWords{naap}  \textstyleEmphasizedVernacularWords{  ma-e}. \\
2s.\textsc{acc}  shoot-2/3p.\textsc{ds}  thus  say-\textsc{imp}.2s      \\
\glt `(If/when) they shoot you, (then) say like that.'
\z

Although it is impossible to have an imperative verb form in a medial clause, a ``soft'' command/request (\sectref{sec:7.3}) may be used in medial clauses, as it takes the medial verb form. In \REF{ex:8:x1366}, the first clause is a request, the second one a statement.

\ea%x1366
\label{ex:8:x1366}
\gll Aite,  {\ob}\textstyleEmphasizedVernacularWords{i}  \textstyleEmphasizedVernacularWords{  aaya=ko}  \textstyleEmphasizedVernacularWords{  yia}  \textstyleEmphasizedVernacularWords{  aaw-om-aya}{\cb} enim-i-yan. \\
1s/p.mother  1p.\textsc{unm}  sugarcane=\textsc{\textsc{nf}}  1p.\textsc{acc}  get-\textsc{ben}-\textsc{bnfy}2.2/3s.\textsc{ds} eat-\textsc{Np}-\textsc{fu}.1p\\
\glt `Mother, get us sugarcane and we will eat it.'
\z

\section{Coordination of clauses}\label{sec:8.1}
%\hypertarget{RefHeading22921935131865}

Coordination links units of ``equivalent syntactic status'' \citep[93]{Crystal1997}. Clausal coordination commonly refers to the coordination of main clauses, as that is much more frequent than the coordination of subordinate clauses. In the following, it is assumed that the discussion is about main clause coordination unless stated otherwise.

The main clauses joined by coordination are independent in the sense that they could stand alone as individual sentences. Examples \REF{ex:8:x1352} and \REF{ex:8:x1358} above show that they can even manifest different functional sentence types. But they are called clauses firstly because they are coordinated within one sentence, and secondly for the sake of consistency, since the coordinated medial (\sectref{sec:8.2.1}) and subordinate clauses (\sectref{sec:8.3.7}) could not be called sentences.

As \citet[848]{Givon1990} points out, no clause in a text is truly independent from its context. Likewise, the coordination vs. subordination of clauses is in many languages a matter of degree rather than a clear-cut distinction. 

Although the chaining of medial and final clauses (\sectref{sec:8.2}) is the main strategy for combining clauses in Mauwake, coordination of main clauses is also common. It is used not only for the cross-linguistically typical cases of conjunction, disjunction, and adversative relations between clauses, but also for causal and consecutive relations.  

\subsection{Conjunction} \label{sec:8.1.1}
%\hypertarget{RefHeading22941935131865}

Conjunction is the most neutral form of coordination: two or more clauses are joined in a sentence, with or without a link between them. If there is a link, it is a pragmatic additive that does not specify the semantic relationship between the clauses. This sometimes allows different interpretations for the relationship, but usually the context constrains the interpretation considerably. 

\subsubsection{Juxtaposition}
%\hypertarget{RefHeading22961935131865}

In juxtaposition\footnote{Also called ``zero strategy'' by \citet[25]{Payne1985}.} two or more clauses are joined without any linking device at all. According to \citet[8]{Haspelmath2007} unwritten languages tend to lack their own coordinators and therefore use more juxtaposition and/or coordinators borrowed  from other, more prestigious languages. 

In Mauwake, juxtaposition is the most typical strategy for conjunction overall. Especially the coordination of verbless clauses is often symmetrical: the reversal of the conjuncts is possible without a change of meaning \REF{ex:8:x1367}, \REF{ex:8:x1390}. 

\ea%x1367
\label{ex:8:x1367}
\gll Wi  Yaapan  emeria  weetak,  mua  manek=iw. \\
3p.\textsc{unm}  Japan  woman  no  man  big=\textsc{lim}\\
\glt `The Japanese didn't have any wives, (they were) just the men.'
\z


\ea%x1390
\label{ex:8:x1390}
\gll Kuuten  wiawi  iperowa,  yo  auwa  kapa=ke. \\
Kuuten  3s/p.father  firstborn  1s.\textsc{unm}  1s/p.father  lastborn=\textsc{cf}      \\
\glt `Kuuten's father was the firstborn (son), my father was the lastborn.'
\z




When one of the conjuncts is a verbless clause and another is a verbal one, symmetrical conjunction is quite common \REF{ex:8:x1391}:

\ea%x1391
\label{ex:8:x1391}
\gll I  uruwa  miim-i-mik,  ni  sosora=ke.\\
1p.\textsc{unm}  loincloth  precede-\textsc{Np}-\textsc{pr}.1/3p  2p.\textsc{unm}  grass.skirt=\textsc{cf}\\
\glt `We father's side of the family (lit: loincloth) go first, you are mother's side (lit: grass skirt).'
\z




Symmetrical conjunction of verbal clauses may be used, when there is parallelism between the clauses \REF{ex:8:x1368}, \REF{ex:8:x1392}:

\ea%x1368
\label{ex:8:x1368}
\gll Na-emi  wi  afa  ar-omak-e-mik, osaiwa  ar-e-mik,  biri-birin-e-mik.\\
say-\textsc{ss}.\textsc{sim}  3p.\textsc{unm}  flying.fox  become-\textsc{distr}/\textsc{pl}-\textsc{pa}-1/3p bird.of.paradise  become-\textsc{pa}-1/3p  \textsc{rdp}-fly-\textsc{pa}-1/3p\\
\glt `Saying so, they became many flying foxes, they became birds of paradise, they flew (away).'
\z

\ea%x1392
\label{ex:8:x1392}
\gll Aria  makera  miirifa  okaiwi  soo=pa  kaik-i-mik, okaiwi  pia  kaik-i-mik.\\
alright  cane  end  other.side  trap=\textsc{loc}  tie-\textsc{Np}-\textsc{pr}.1/3p other.side  bamboo  tie-\textsc{Np}-\textsc{pr}.1/3p\\
\glt `Alright we tie one end of the cane to the trap, the other to a (piece of) bamboo.'
\z

In \REF{ex:8:x1851} the medial clause relates to both of the final clauses, not just to the first one:

\ea%x1851
\label{ex:8:x1851}
\gll Koora=pa  efa  uruf-am-ik-eya  \textstyleEmphasizedVernacularWords{ikiw}\textstyleEmphasizedVernacularWords{-}\textstyleEmphasizedVernacularWords{i}\textstyleEmphasizedVernacularWords{-}\textstyleEmphasizedVernacularWords{nen} \textstyleEmphasizedVernacularWords{ekap}\textstyleEmphasizedVernacularWords{-}\textstyleEmphasizedVernacularWords{i}\textstyleEmphasizedVernacularWords{-}\textstyleEmphasizedVernacularWords{nen}.\\
house=\textsc{loc}  1s.\textsc{acc}  see-\textsc{ss}.\textsc{sim}-be-2/3s.\textsc{ds}  go-\textsc{Np}-\textsc{fu}.1s come-\textsc{Np}-\textsc{fu}.1s\\
\glt `You see me from the house and/as I will go and come.'
\z

When the coordination is not symmetrical, the clause in the second conjunct is an example or an explanation of the first clause \REF{ex:8:x1370}, or it follows the first one in a temporal sequence \REF{ex:8:x1369}.

\ea%x1370
\label{ex:8:x1370}
\gll Auwa  aite  wia  karu-i-yen,  owowa=pa wia  uruf-u.\\
1s/p.father  1s/p.mother  3p.\textsc{acc}  visit-\textsc{Np}-\textsc{fu}.1p  village=\textsc{loc} 3p.\textsc{acc}  see-1d.\textsc{imp}\\
\glt `We'll visit my parents, let's see them in the village.'
\z


\ea%x1369
\label{ex:8:x1369}
\gll Miiw-aasa  um-eya  miiw-aasa  nain  on-am-ika-iwkin  \textstyleEmphasizedVernacularWords{epa}  \textstyleEmphasizedVernacularWords{kokom(a)-ar-e-k,}  \textstyleEmphasizedVernacularWords{epa}  \textstyleEmphasizedVernacularWords{iimeka}  \textstyleEmphasizedVernacularWords{tuun-e-k}.
\\
land-canoe  die-2/3s.\textsc{ds}  land-canoe  that1  do-\textsc{ss}.\textsc{sim}-be-2/3p.\textsc{ds}      place  dark-\textsc{inch}-\textsc{pa}-3s  place  ten  count?-\textsc{pa}-3s\\
\glt `The truck broke and while they were fixing the truck it became dark, (then) it was midnight.'
\z

A fairly common structure is one where the first conjunct is not directly followed by another finite clause but by one or more medial clauses before the final clause \REF{ex:8:x1371}:

\ea%x1371
\label{ex:8:x1371}
\gll \textstyleEmphasizedVernacularWords{Ikemika}  \textstyleEmphasizedVernacularWords{kaik-ow(a)}  \textstyleEmphasizedVernacularWords{mua}  \textstyleEmphasizedVernacularWords{nain}  \textstyleEmphasizedVernacularWords{nop-a-mik},  imen-ap   maak-iwkin  \textstyleEmphasizedVernacularWords{o}  \textstyleEmphasizedVernacularWords{miim-o-k}.\\
 wound  tie-\textsc{nmz}  man  that1  search-\textsc{pa}-1/3p  find-\textsc{ss}.\textsc{seq} tell-2/3p.\textsc{ds}  3s.\textsc{unm}  precede-\textsc{pa}-3s\\ 
\glt`They looked for the medical orderly, and when they found him and told him, he went ahead of them.'
\z


Juxtaposition in itself is neutral and only shows that the two or more clauses are somehow connected with each other, but it can be used when propositions joined by it have different semantic relationships with each other \REF{ex:8:x1404}, \REF{ex:8:x1425}.

\ea%x1404
\label{ex:8:x1404}
\gll Waaya  maneka  marew  pun,  mua  unowa  me  wia pepek-er-a-k.\\
 pig  big  no(ne)  also  man  many  not  3p.\textsc{acc} enough-\textsc{inch}-\textsc{pa}-3s\\
\glt`Also, the pig was not big, (so) it was not enough for many people.'
\z


\ea%x1425
\label{ex:8:x1425}
\gll Ni  iperuma  fain  me  enim-eka,  inasin(a)  mua=ke. \\
2p.\textsc{unm}  eel  this  not  eat-\textsc{imp}.2p  spirit  man=\textsc{cf}      \\
\glt`Don't eat this eel, (because) it is a spirit man.'
\z


\subsubsection{Conjunction with coordinating connectives} \label{sec:8.1.1.2}
%\hypertarget{RefHeading22981935131865}

Two of the three pragmatic connectives (\sectref{sec:3.11.1}) are used as clausal coordinators: the additive \textstyleStyleVernacularWordsItalic{ne}  and \textstyleStyleVernacularWordsItalic{aria}, `alright' which marks a break in the topic chain. \textstyleStyleVernacularWordsItalic{Ne} can be used in some of the contexts where mere juxtaposition is also used, but it is less frequent. If the second conjunct is an explanation or example of the first one, conjoining the clauses with \textstyleStyleVernacularWordsItalic{ne} is not allowed. Example \REF{ex:8:x1372} is a case of symmetrical coordination, but if the order of the two conjuncts were reversed, the adverbial \textstyleStyleVernacularWordsItalic{pun} `also', which has to be in the second conjunct, would not move to the first conjunct with the rest of the clause.

\ea%x1372
\label{ex:8:x1372}
\gll I  mua=ko  me  wia  furew-a-mik,  \textstyleEmphasizedVernacularWords{ne}  yiena  pun  mukuna=ko  me  op-a-mik.\\
1p.\textsc{unm}  man=\textsc{nf}  not  3p.\textsc{acc}  sense-\textsc{pa}-1/3p  \textsc{add}  1p.\textsc{gen}  also fire=\textsc{nf}  not  hold-\textsc{pa}-1/3p    \\
\glt`We didn't sense anyone there and we ourselves did not hold fire either.'
\z


The example \REF{ex:8:x1373} is syntactically neutral, but semantically it is interpreted as both temporal and consecutive sequence.

\ea%x1373
\label{ex:8:x1373}
\gll ...maa  wiar  fe-feef-omak-e-mik,  \textstyleEmphasizedVernacularWords{ne}  wi  ikiw-e-mik ...\\
food  3.\textsc{dat} \textsc{rdp}-spill-\textsc{distr}/\textsc{pl}-\textsc{pa}-1/3p  \textsc{add}  3p.\textsc{unm} go-\textsc{pa}-1/3p   \\
\glt`{\dots} they\textsubscript{i} spilled their\textsubscript{j} food, and (so/then) they\textsubscript{j} went (away) {\dots}'
\z


When there are more than two coordinated clauses in a sentence without any intervening medial clauses, it is common to have \textstyleStyleVernacularWordsItalic{ne}  joining the last two clauses \REF{ex:8:x1374}:

\ea%x1374
\label{ex:8:x1374}
\gll Mua  kuum-e-mik  nain  me  wia  kuuf-a-mik,  me wia  furew-a-mik,  \textstyleEmphasizedVernacularWords{ne}  me  wia  imen-a-mik. \\
man  burn-\textsc{pa}-1/3p  that1  not  3p.\textsc{acc}  see-\textsc{pa}-1/3p  not 3p.\textsc{acc}  sense-\textsc{pa}-1/3p  \textsc{add}  not  3p.\textsc{acc}  find-\textsc{pa}-1/3p     \\
\glt`We didn't see the men who burned it, we didn't sense them and we didn't find them.'
\z

The connective \textstyleStyleVernacularWordsItalic{ne}  is also used in sentences where an adversative interpretation can be applied.\footnote{Using \citegen[28]{Haspelmath2007} terms, \textit{ne} in the adversative function could be called an \textit{oppositive} coordinator, as the second coordinand does not cancel an expectation like it does in adversative clauses formed with either the demonstrative \textit{nain} or the topic marker -\textit{na} (\sectref{sec:8.3.4}).}  Example \REF{ex:8:x1375} describes a couple that stayed in the village during the war and placed some of their belongings outside their house to show that there were people living in the village, while many others ran away into the rainforest. 

\ea%x1375
\label{ex:8:x1375}
\gll Amina,  wiowa,  eka  napia  koor(a)  miira=pa iimar-aw-ikiw-e-mik,  \textstyleEmphasizedVernacularWords{ne}  wi  unowa  baurar-e-mik. \\
pot  spear  water  bamboo  house  front=\textsc{loc} stand-\textsc{caus}-go-\textsc{pa}-1/3p  \textsc{add}  3p.\textsc{unm}  many  flee-\textsc{pa}-1/3p    \\
\glt`We placed the pots, spears and bamboo water containers in line in front of the house, but many ran away.'
\z


The connective \textstyleStyleVernacularWordsItalic{aria} 'alright' may be used when there is a change of topic or an unexpected development within the sentence \REF{ex:8:x1376}, \REF{ex:8:x1377}.

\ea%x1376
\label{ex:8:x1376}
\gll Epa  wii-wiim-ik-ua,  \textstyleEmphasizedVernacularWords{aria}  wi  sawur=ke  ekap-ep  takira  nain  samapora  onaiya  akua  aaw-e-mik.\\
place  \textsc{rdp}-dawn-be-\textsc{pa}.3s  alright  3p.\textsc{unm}  spirit=\textsc{cf}  come-\textsc{ss}.\textsc{seq} boy  that1  bed  with  shoulder  take-\textsc{pa}-1/3p    \\
\glt`It was getting light, and spirits came and carried the boy with his bed (away) on their shoulders.'
\z


\ea%x1377
\label{ex:8:x1377}
\gll Iiriw  muuka  oko  wiawi  onak  urera  maa uup-e-mik,  \textstyleEmphasizedVernacularWords{aria}  maa  me  wu-om-a-mik  yon{\dots} \\
earlier  boy  other  3s/p.father  3s/p.mother  afternoon  food cook-\textsc{pa}-1/3p  alright  food  not  put-\textsc{ben}-\textsc{bnfy}2.\textsc{pa}-1/3p  perhaps     \\
\glt`Long ago, the parents of a boy cooked food in the afternoon, (but) perhaps they did not put any food for him {\dots}'
\z


It is also the default coordinator when a non-verbal constituent in two or more otherwise very similar conjuncts are contrasted \REF{ex:8:x1379}, or emphasized \REF{ex:8:x1380}, in coordinated clauses.

\ea%x1379
\label{ex:8:x1379}
\gll Yo  Malala  mauw-owa  nia  asip-i-yem,  \textstyleEmphasizedVernacularWords{aria} yena  owowa,  Moro  owowa  wia  asip-i-yem.\\
1s.\textsc{unm}  Malala  work-\textsc{nmz}  2p.\textsc{acc}  help-\textsc{Np}-\textsc{pr}.1s  alright 1s.\textsc{gen}  village  Moro  village  3p.\textsc{acc}  help-\textsc{Np}-\textsc{pr}.1s     \\
\glt`I help you Malala people with your work, and I help my village, Moro village.'
\z


\ea%x1380
\label{ex:8:x1380}
\gll Eema  pun  ekap-ep  yia  maak-e-k,  \textstyleEmphasizedVernacularWords{aria}  buburia  ona pun  ekap-ep  yia  maak-e-k. \\
Eema  also  come-\textsc{ss}.\textsc{seq}  1p.\textsc{acc}  tell-\textsc{pa}-3s  alright  bald  3s.\textsc{gen} also  come-\textsc{ss}.\textsc{seq}  1p.\textsc{acc}  tell-\textsc{pa}-3s     \\
\glt`Eema came and told us, and the bald man himself too came and told us.'
\z


\subsection{Disjunction} \label{sec:8.1.2}
%\hypertarget{RefHeading23001935131865}

The speech of the Mauwake people tends to be rather concrete in the sense that they do not speculate much on different abstract alternatives. So disjunction of clauses, although possible, is not common. Disjunction is marked by the connective \textstyleStyleVernacularWordsItalic{e} `or' placed between the conjuncts \REF{ex:8:x1385} (\sectref{sec:3.11.2}). 

\ea%x1385
\label{ex:8:x1385}
\gll Nain=ke  napum-ar-i-mik  \textstyleEmphasizedVernacularWords{e}  um-i-mik,  mua  oko napum-ar-e-k  nain  erewar-e-n. \\
that1=\textsc{cf}  sickness-\textsc{inch}-\textsc{Np}-\textsc{pr}.1/3p  or  die-\textsc{Np}-\textsc{pr}.1/3p  man  other sickness-\textsc{inch}-\textsc{pa}-3s  that1  foresee-\textsc{pa}-2s     \\
\glt`That is about people becoming sick or dying, you foresaw (in a dream) that some man became sick.'
\z


Sometimes the question marker -\textstyleStyleVernacularWordsItalic{i}  replaces the connective \REF{ex:8:x1387}.

\ea%x1387
\label{ex:8:x1387}
\gll Aria  no  ikoka  mua  owawiya  irak-ep=\textstyleEmphasizedVernacularWords{i}  kamenap  on-ap  yo  me  efar  kerer-e,  no nomokowa  Kululu  fan-e-k  a.\\
 alright  2s.\textsc{unm}  later  man  with  fight-\textsc{ss}.\textsc{seq}=\textsc{qm}  how do-\textsc{ss}.\textsc{seq}1s.\textsc{unm}  not  1s.\textsc{dat}  arrive-\textsc{imp}.2s  2s.\textsc{unm} 2s/p.brother  Kululu  here-\textsc{pa}-3s  \textsc{intj}\\
\glt`Alright, later when you fight with your husband or do something like that, do not come to me, your brother Kululu is right here.'
\z


Alternative questions (\sectref{sec:7.2.2}) have the question marker -\textstyleStyleVernacularWordsItalic{i} cliticized to the end of the clause at least in the first conjunct. Closed alternative questions leave the question mark out of the last conjunct \REF{ex:8:x1386}. 

\ea%x1386
\label{ex:8:x1386}
\gll Ikoka  ekap-ep  feeke  sira  nain  piipua-i-nan=\textstyleEmphasizedVernacularWords{i} \textstyleEmphasizedVernacularWords{e}  weetak?\\
later  come-\textsc{ss}.\textsc{seq}  here.\textsc{cf}  habit  that1  leave-\textsc{Np}-\textsc{fu}.2s=\textsc{qm} or  no\\
\glt`Later when you come, will you here leave that habit or not?'
\z


Open alternative questions have the question marker in all the conjuncts \REF{ex:8:x1384}.

\ea%x1384
\label{ex:8:x1384}
\gll Mua  oko  miira  inawera=pa  uruf-ap  ma-i-mik, mua  oko=ke  napuma  aaw-o-k=\textstyleEmphasizedVernacularWords{i}  \textstyleEmphasizedVernacularWords{e}  um-o-k=\textstyleEmphasizedVernacularWords{i}?\\
man  other  face  dream=\textsc{loc}  see-\textsc{ss}.\textsc{seq}  say-\textsc{Np}-\textsc{pr}.1/3p man  other=\textsc{cf} sickness  get-\textsc{pa}-3s=\textsc{qm}  or  die-\textsc{pa}-3s=\textsc{qm}\\
\glt`When we see some man's face in a dream we say, ``Has some other man become sick or died (or possibly neither)?'' '
\z


\subsection{Adversative coordination} \label{sec:8.1.3}
%\hypertarget{RefHeading23021935131865}

There is no adversative coordinator in Mauwake. It was mentioned above  (\sectref{sec:3.11.1}, 8.1.1.2) that the pragmatic additive connective \textstyleStyleVernacularWordsItalic{ne}, which is semantically neutral, is possible when there is a relationship between clauses that may be interpreted as contrastive \REF{ex:8:x1388}. 

\ea%x1388
\label{ex:8:x1388}
\gll Iir  nain  Kedem  manek  akena  keker  op-a-k \textstyleEmphasizedVernacularWords{ne}  Yoli  weetak.\\
time  that  Kedem  big  very  fear  hold-\textsc{pa}-3s \textsc{add}  Yoli  no \\
\glt`That time Kedem was very scared but Yoli wasn't.'
\z


There are two strategies that can be used when a strong adversative is needed. A `but'-protasis \citep[237]{Reesink1983b} may be marked by either the distal demonstrative \textstyleStyleVernacularWordsItalic{nain} `that' (\sectref{sec:3.6.2}), or the topic marker -\textstyleStyleVernacularWordsItalic{na} (§\sectref{sec:3.12.7.1}, \ref{sec:8.3.4}), added to a finite clause. Adversative clauses with the demonstrative \textstyleStyleVernacularWordsItalic{nain} differ from nominalized clauses functioning as complement clauses or relative clauses in the following respects. Intonationally, \textstyleStyleVernacularWordsItalic{nain} is the initial element in the second one of the contrasted clauses, rather than a final element in a subordinate clause, and it is often preceded by a short pause \REF{ex:8:x1395}. The protasis may even be a separate sentence \REF{ex:8:x728}. 

\ea%x1395
\label{ex:8:x1395}
\gll Panewowa  nain,  wi  iiriw  eno-wa  en-e-mik,  \textstyleEmphasizedVernacularWords{nain}  me onak-e-mik.\\
old.person  that1  3p.\textsc{unm}  earlier  eat-\textsc{nmz}  eat-\textsc{pa}-1/3p  that1  not give.3s-\textsc{pa}-1/3p\\
\glt`As for the old woman, they (aready) ate the meal earlier but did not give (any of it) to her to eat.'
\z


\ea%x728
\label{ex:8:x728}
\gll Yo  bom  koor  miira=pa  efar  or-om-ik-ua. \textstyleEmphasizedVernacularWords{Nain}  yo  me  baurar-em-ik-e-m. \\
1s.\textsc{unm}  bomb  house  face=\textsc{loc}  1s.\textsc{dat}  fall-\textsc{ss}.\textsc{sim}-be-\textsc{pa}.3s that1  1s.\textsc{unm}  not flee-\textsc{ss}.\textsc{sim}-be-\textsc{pa}-1s     \\
\glt`Bombs kept dropping in front of my house. But I didn't keep running away.'
\z


The examples \REF{ex:8:x1389} and \REF{ex:8:x1394} are structurally very similar to sentences with relative clauses (\sectref{sec:8.3.1.2}). But here the demonstrative \textstyleStyleVernacularWordsItalic{nain} is part of the adversative clause and is preceded by a pause. 

\ea%x1389
\label{ex:8:x1389}
\gll Mera  eka  enim-i-mik,  \textstyleEmphasizedVernacularWords{nain}  i  mangala me  enim-i-mik,  waaya  me  enim-i-mik.\\
fish  water  eat-\textsc{Np}-\textsc{pr}.1/3p  that1  1p.\textsc{unm}  shellfish not  eat-\textsc{Np}-\textsc{pr}.1/3p  pig  not  eat-\textsc{Np}-\textsc{pr}.1/3p\\
\glt`We eat fish soup, but we don't eat shellfish, (and) we don't eat pork.'
\z


\ea%x1394
\label{ex:8:x1394}
\gll I  nan  soomar-e-mik,  \textstyleEmphasizedVernacularWords{nain}  i  mukuna=ko  me op-a-mik.\\
1p.\textsc{unm}  there  walk-\textsc{pa}-1/3p  that1  1p.\textsc{unm}  fire=\textsc{nf} not hold-\textsc{pa}-1/3p     \\
\glt`We walked there, but we did not hold/have any fire.'
\z


Compare \REF{ex:8:x1394} with the relative clause \REF{ex:8:x1396}, where the demonstrative functions as a relative marker and comes at the end of the clause. This is shown by the slightly rising intonation on \textstyleStyleVernacularWordsxiiptItalic{nain}, as well as a pause following it in spoken text:\footnote{This similarity creates a problem with written texts that do not have adequate punctuation. Sometimes either interpretation is acceptable.} 

\ea%x1396
\label{ex:8:x1396}
\gll I  nan  soomar-e-mik \textstyleEmphasizedVernacularWords{nain},  i  mukuna=ko  me op-a-mik.\\
1p.\textsc{unm}  there  walk-\textsc{pa}-1/3p that1  1p.\textsc{unm}  fire=\textsc{nf}  not hold-\textsc{pa}-1/3p\\
\glt`We who walked there didn't hold/have any fire.' (Or: `When we walked there, we didn't hold/have any fire.')
\z

The adversative sentences formed with the topic marker -\textstyleStyleVernacularWordsItalic{na} are complex rather than coordinate sentences (\sectref{sec:8.3.4}).

\subsection{Consecutive coordination}
%\hypertarget{RefHeading23041935131865}

Within a sentence, clauses are typically connected by one of the syntactically neutral strategies, which leave the semantic relationship implied. Some sentences using juxtaposition \REF{ex:8:x1425}, the pragmatic additive \textstyleStyleVernacularWordsItalic{ne} \REF{ex:8:x1373} or clause chaining \REF{ex:8:x1412} can be interpreted as having a consecutive relationship between the clauses, although this does not show in the syntax. This section deals with the cases where the consecutive relationship is marked overtly.

Relationships of cause and effect, or reason and result,\footnote{Reason-result relationship presupposes the presence of reasoning in the process, cause-effect relationship does not.}  are central in the discussion of causal and consecutive clauses. It seems that currently Mauwake may be developing a distinction between cause and reason on one hand, and between effect and result on the other. But the tendency, if there, is not very strong (\sectref{sec:3.11.2}). 

Both the clauses in a sentence expressing a cause-effect or reason-result relationship are main clauses and are in a coordinate relationship with each other. It is common for the two clauses to form separate sentences rather than be within the same sentence. 

The tendency to present events in the same order that they occur, common to languages in general, is very strong in Papuan languages. Consequently, there is a strong preference to present a cause clause before an effect clause (\citealt[409]{Haiman1980}, \citealt[59]{Roberts1987}, \citealt{Reesink1987}). In Mauwake consecutive coordination is the default, unmarked strategy for those sentences that express cause-effect or reason-result relationships\linebreak overt\-ly, because their structure follows this principle \REF{ex:8:x1400}, whereas in causal coordination sentences the effect is stated before the cause. 

\ea%x1400
\label{ex:8:x1400}
\gll Emar,  nos=ke  yo  efa  kemal-ep  iripuma  fain ifakim-o-n,  \textstyleEmphasizedVernacularWords{naapeya}  iripuma  fain  ik-ep  enim-e.\\
1s/p.friend  2s.\textsc{cf}=\textsc{cf}  1s.\textsc{unm}  1s.\textsc{acc}  pity-\textsc{ss}.\textsc{seq}  iguana  this kill-\textsc{pa}-2s  therefore  iguana  this  roast-\textsc{ss}.\textsc{seq}  eat-\textsc{imp}.2s\\
\glt`Friend, it was you who pitied me and killed this iguana, therefore you roast and eat this iguana.'
\z


Effect and result clauses use \textstyleStyleVernacularWordsItalic{naapeya/naeya} `therefore, (and) so' (\sectref{sec:3.11.2}) as their connective \REF{ex:8:x1401}--\REF{ex:8:x1403}.

\ea%x1401
\label{ex:8:x1401}
\gll Koora  fuluwa  unowa  marew,  \textstyleEmphasizedVernacularWords{naapeya}  in-i-mik  nain dabela  me  senam  furew-i-mik.\\
house  hole  many  no(ne)  therefore  sleep-\textsc{Np}-\textsc{pr}.1/3p  that1 cold  not  too.much  sense-\textsc{Np}-\textsc{pr}.1/3p\\
\glt `The houses do not have many windows, so those who sleep (there) do not sense/feel the cold too much.'
\z


\ea%x1402
\label{ex:8:x1402}
\gll Pita  weke  wiar  um-o-k,  \textstyleEmphasizedVernacularWords{naapeya}  o  suule me  iw-a-k. \\
Pita  3s/p.grandfather  3.\textsc{dat}  die-\textsc{pa}-3s  therefore  3s.\textsc{unm} school not  go-\textsc{pa}-3s\\
\glt`Pita's grandfather died, so he (Pita) didn't go to school.'
\z


\ea%x1405
\label{ex:8:x1405}
\gll {\dots}pika  oona  me  kekan-ow-a-k,  \textstyleEmphasizedVernacularWords{naeya}  uura ewar  maneka=ke  kerer-emi  koora  nain  wiar teek-a-k.\\
...wall  support  not  be.strong-\textsc{caus}-\textsc{pa}-3s  therefore  night wind  big=\textsc{cf}  appear-\textsc{ss}-\textsc{sim}  house  that1  3.\textsc{dat} tear-\textsc{pa}-3s\\
\glt`He did not strengthen the wall supports, so at night a big wind arose and tore down his house.'
\z


\ea%x1408
\label{ex:8:x1408}
\gll No  nena  pun  pina  sira  naap  nain  on-i-n, \textstyleEmphasizedVernacularWords{naeya}  nos  pun  opora=pa  ika-i-nan.\\
2s.\textsc{unm}  2s.\textsc{gen}  also  guilt  custom  thus  that1  do-\textsc{Np}-\textsc{pr}.2s therefore  2s.\textsc{fc} also talk=\textsc{loc} be-\textsc{Np}-\textsc{fu}.2s\\
\glt`You yourself do bad things like that too, therefore you too will be under accusation.'
\z


\textstyleStyleVernacularWordsItalic{Naapeya} can also co-occur with the conjunctive coordinator \textstyleStyleVernacularWordsItalic{ne} \REF{ex:8:x1403}. 

\ea%x1403
\label{ex:8:x1403}
\gll Epa  nan  soomar-em-ik-ok  or-o-mik, \textstyleEmphasizedVernacularWords{ne}  \textstyleEmphasizedVernacularWords{naapeya}  pina  wi  wiar  korin-e-k.  \\
place  there  walk-\textsc{ss}.\textsc{sim}-be-\textsc{ss} descend-\textsc{pa}-1/3p \textsc{add}  therefore  guilt  3p.\textsc{unm} 3.\textsc{acc} stick-\textsc{pa}-3s\\
\glt`They were walking there in that place and came down, and so the guilt (for starting a forest fire) stuck to them.'
\z


The use of \textstyleStyleVernacularWordsItalic{naapeya} and \textstyleStyleVernacularWordsItalic{naeya} is both external and internal, i.e., they connect events in a situation and ideas in a text. The internal use of \textstyleStyleVernacularWordsItalic{ne naapeya} and \textstyleStyleVernacularWordsItalic{aria naapeya} is restricted to intersentential use. They refer to a longer stretch in the preceding text as their protasis \REF{ex:8:x1407}.

\ea%x1407
\label{ex:8:x1407}
\gll \textstyleEmphasizedVernacularWords{Aria}  \textstyleEmphasizedVernacularWords{naapeya}  wi  inasina  ook-i-mik sira  nain  me  wiar  ook-eka.  \\
alright  therefore  3p.\textsc{unm} spirit  follow-\textsc{Np}-\textsc{pr}.1/3p custom  that1  not  3.\textsc{dat} follow-\textsc{imp}.2p\\
\glt`So therefore do not follow the behavior of those who follow/believe in spirits.'
\z

 
As an internal connective \textstyleStyleVernacularWordsItalic{naeya} mainly connects full sentences \REF{ex:8:x1410}, only seldom clauses within a sentence \REF{ex:8:x1411}:

\ea%x1410
\label{ex:8:x1410}
\gll No  mua  woos  reen-owa=ke,  \textstyleEmphasizedVernacularWords{naeya}  no  kema  kir-owa miatin-i-n.\\
2s.\textsc{unm} man  head  dry-\textsc{nmz}=\textsc{cf} therefore  2s.\textsc{unm} liver turn-\textsc{nmz} dislike-\textsc{Np}-\textsc{pr}.2s\\
\glt`You are hard-headed, therefore you do not like to change your (bad) ways.'
\z


\ea%x1411
\label{ex:8:x1411}
\gll Ni  sira-sira  naap  on-i-man.  \textstyleEmphasizedVernacularWords{Naeya}  opora  iiriw ma-e-k  nain  pepek  akena  nia  ma-e-k.\\
2p.\textsc{unm} \textsc{rdp}-custom  thus  do-\textsc{Np}-\textsc{pr}.2p  therefore  talk  earlier say-\textsc{pa}-3s  that1  enough  very  2p.\textsc{acc} say-\textsc{pa}-3s\\
\glt`You do (bad) things like that. Therefore the talk that he already said about you is very accurate.'
\z


\textstyleStyleVernacularWordsItalic{Neemi}  is a consecutive coordinator that almost exclusively conjoins full sentences rather than clauses within a sentence: \REF{ex:8:x1409} is from translated text but considered natural. (3.\ref{ex:3:x736}) is repeated here as \REF{ex:8:x1904}. \textit{Neemi} is an internal connective, only used in reasoning. It requires some point of similarity between the two conjuncts.

\ea%x1904
\label{ex:8:x1904}
\gll Teeria  fain  K10  wu-a-mik.  \textstyleEmphasizedVernacularWords{Neemi}  wi  teeria  nain  pun K10  wu-a-mik.\\
group  this  K10  put-\textsc{pa}-1/3p  therefore  3p.\textsc{unm} group  that1  too K10  put-\textsc{pa}-1/3p\\
\glt`This group put down ten kina. Therefore that group put down ten kina, too.'
\z


\ea%x1409
\label{ex:8:x1409}
\gll Krais  sirir-owa  aaw-omak-e-k,  \textstyleEmphasizedVernacularWords{neemi}  is  pun unowiya  naap  aaw-i-mik.\\
Christ  hurt-\textsc{nmz} get-\textsc{distr}/\textsc{pl}-\textsc{pa}-3s  therefore  1p.\textsc{fc} also all  thus  get-\textsc{Np}-\textsc{pr}.1/3p\\
\glt`Christ received a lot of pain, so we all too get (pain) like that.'
\z


The connective \textstyleStyleVernacularWordsItalic{naap nain} is used almost only inter-sententially \REF{ex:8:x1905}. Between clauses in a sentence it is possible but rare \REF{ex:8:x1424}:

\ea%x1905
\label{ex:8:x1905}
\gll Naeya  nokar-e-mik, ``\textstyleEmphasizedVernacularWords{Naap}  \textstyleEmphasizedVernacularWords{nain}  no  naareke?'' \\
therefore  ask-\textsc{pa}-1/3p  thus  that1  2s.\textsc{unm}  who.\textsc{cf}\\
\glt`Therefore they asked, ``So then, who are you?'' '
\z


\ea%x1424
\label{ex:8:x1424}
\gll Wiam  arow  pepek  nan  urup-e-mik  nain, \textstyleEmphasizedVernacularWords{naap}  \textstyleEmphasizedVernacularWords{nain}  yo  moram  urup-e-m. \\
3p.\textsc{refl} three  enough  there  ascend-\textsc{pa}-1/3p  that1 thus  that1  1s.\textsc{unm} why/in.vain ascend-\textsc{pa}-1s    \\
\glt`(Since it is the case that) those three are enough and came up, so then why did I have to come up? (or: {\dots}so then I came up in vain).'
\z


\subsection{Causal coordination, ``afterthought reason''}
%\hypertarget{RefHeading23061935131865}

The causal coordination is a very marked structure, which shows in the unusual ordering of the clauses: the causal clause follows rather than precedes the consequent clause. The causal clause in Mauwake begins with the connective \textstyleStyleVernacularWordsItalic{moram} `because' (\sectref{sec:3.11.2}), which is originally the interrogative word for `why'.  There are two possible origins for this untypical structure. It may be a recent calque on the Tok Pisin causal construction, which uses \textstyleForeignWords{bilong wanem} `why/because' as the connector and the same ordering of the two clauses. The ordering of the clauses shows that it may also have originated as an ``afterthought reason'',\footnote{The term suggested by Ger Reesink.} even though  currently it is used when the cause or reason is emphasized \REF{ex:8:x1417}, \REF{ex:8:x1420}. 

\ea%x1417
\label{ex:8:x1417}
\gll Owowa  mamaiya  soora  weetak,  \textstyleEmphasizedVernacularWords{moram}  iwera isak-omak-e-mik.\\
village  near  forest  no  because  coconut plant-\textsc{distr}/\textsc{pl}-\textsc{pa}-1/3p\\
\glt`There is no forest near the village, because we have planted a lot of coconut palms.'
\z


\ea%x1420
\label{ex:8:x1420}
\gll Poh  San  uruf-ap  kema  ten-e-mik,  \textstyleEmphasizedVernacularWords{moram}  i  kema naap  suuw-a-mik,  napuma  me  sariar-owa  ik-ua.\\
Poh  San  see-\textsc{ss}.\textsc{seq} liver fall-\textsc{pa}-1/3p  because 1p.\textsc{unm} liver thus  push-\textsc{pa}-1/3p sickness  not heal-\textsc{nmz} be-\textsc{pa}.3s\\
\glt`We saw Poh San and were relieved (lit: liver fell), because we had thought that (her) sickness hadn't healed yet (but it had).'
\z


\textstyleStyleVernacularWordsItalic{Moram wia} is used almost exclusively between full sentences \REF{ex:8:x1906}; the example \REF{ex:8:x1421} is the only intra-sentential instance of \textstyleStyleVernacularWordsItalic{moram wia} in the data. I have not noticed any semantic difference caused by the addition of the negator.

\ea%x1906
\label{ex:8:x1906}
\gll ...maamuma  senam  aaw-e-mik.  \textstyleEmphasizedVernacularWords{Moram}  \textstyleEmphasizedVernacularWords{wia}, maa  ele-eliwa  sesek-a-mik.\\
money  too/very.much  get-\textsc{pa}-1/3p why not thing/food  \textsc{rdp}-good sell-\textsc{pa}-1/3p\\
\glt`{\dots}they got a lot of money. (That's) because they sold good food.'
\z

 
\ea%x1421
\label{ex:8:x1421}
\gll Iir  nain  yo  owowa=pa=ko  me  mauw-a-m, \textstyleEmphasizedVernacularWords{moram}  \textstyleEmphasizedVernacularWords{wia}  yo  Ukarumpa  urup-owa=ke  na-ep mauw-owa  miatin-e-m.\\
time  that1  1s.\textsc{unm} village=\textsc{loc}=\textsc{nf} not  work-\textsc{pa}-1s because  not  1s.\textsc{unm} Ukarumpa  ascend-\textsc{nmz}=\textsc{cf} say-\textsc{ss}.\textsc{seq} work-\textsc{nmz} dislike-\textsc{pa}-1s\\
\glt`That time I did not work in the village, because I thought that I was due to go up to Ukarumpa, and (so) I didn't like to work.'
\z


Both a causative and a consecutive connective can co-occur in the same sentence. When that happens, the consecutive clause occurs twice: first without a connective and after the causal clause with a connective \REF{ex:8:x1422}, \REF{ex:8:x1423}. This underlines the strong preference to keep the cause-effect (or reason-result) order.

\ea%x1422
\label{ex:8:x1422}
\gll I  epa  unowa=ko  me  soomar-e-mik,  \textstyleEmphasizedVernacularWords{moram}  owowa maneka,  \textstyleEmphasizedVernacularWords{naapeya}  soomar-owa  lebum(a)-ar-e-mik.\\
1p.\textsc{unm} place  many=\textsc{nf} not  walk-\textsc{pa}-1/3p  because  village big  therefore  walk-\textsc{nmz} lazy-\textsc{inch}-\textsc{pa}-1/3p\\
\glt`We didn't walk in many places, because the village/town was big, therefore we didn't care to walk.'
\z


\ea%x1423
\label{ex:8:x1423}
\gll Mua  lebuma  emeria  me  wi-i-mik,  \textstyleEmphasizedVernacularWords{moram}  emeria  muukar-eya  muuka  nain  maa  mauwa  enim-i-non, \textstyleEmphasizedVernacularWords{naapeya}  mua  lebuma  emeria  me  wi-i-mik.\\
man  lazy  woman  not  give.them-\textsc{Np}-\textsc{pr}.1/3p because  woman give.birth-2/3s.\textsc{ds} son  that1  food  what  eat-\textsc{Np}-\textsc{fu}.3s therefore  man  lazy  woman  not  give.them-\textsc{Np}-\textsc{pr}.1/3p\\
\glt`We do not give wives to lazy men, because when the woman bears a child what would it eat, therefore we do not give wives to lazy men.'
\z


\subsection{Apprehensive coordination} \label{sec:8.1.6}
%\hypertarget{RefHeading23081935131865}

A less common clause type, that of apprehensive clauses \citep[61]{Roberts1987}, also called negative purpose clauses (\citealt[444]{Haiman1980}, \citealt[188]{ThompsonEtAl1985}), is perhaps more commonly subordinate than coordinate. But in Mauwake the apprehensive clauses are coordinated finite clauses \REF{ex:8:x1426}, \REF{ex:8:x1427}, originally separate sentences \REF{ex:8:x1428}. The apprehension clause is introduced by the indefinite \textstyleStyleVernacularWordsItalic{oko} `other' (\sectref{sec:3.7.2}), which has also developed the meaning `otherwise'.  
% * <liisa.berghall@gmail.com> 2015-05-22T13:42:29.423Z:
%
%  Thompson EtAl has now been added to the bibl.
%
% ^ <liisa.berghall@gmail.com> 2015-07-27T13:35:28.523Z.

\ea%x1426
\label{ex:8:x1426}
\gll Ni  maa  uru-uruf-ami  ik-eka,  \textstyleEmphasizedVernacularWords{oko}  mua  oko=ke nia  peeskim-i-kuan.\\
2p.\textsc{unm} thing \textsc{rdp}-see-\textsc{ss}.\textsc{sim}  be-\textsc{imp}.2p  other  man  other=\textsc{cf} 2p.\textsc{acc}  cheat-\textsc{Np}-\textsc{fu}.3p\\
\glt`Watch out, otherwise/lest you get cheated.'
\z


\ea%x1427
\label{ex:8:x1427}
\gll Naap  on-owa  weetak, \textstyleEmphasizedVernacularWords{oko}  yiena  sira  puuk-i-yen. \\
thus  do-\textsc{nmz} no other 1p.\textsc{gen} custom  cut-\textsc{nps}-\textsc{fu}.1p\\
\glt`We must not do like that, otherwise/lest we break our custom/law (or: {\dots} lest we ourselves break the custom/law).'
\z


\ea%x1428
\label{ex:8:x1428}
\gll Naap  yo  aakisa  efa  uruf-i-n.  \textstyleEmphasizedVernacularWords{Oko} neeke soomar-ekap-em-ik-omkun  ma-i-nan,  ``  {\dots } ``\\
thus  1s.\textsc{unm} now 1s.\textsc{acc} see-\textsc{Np}-\textsc{pr}.2s other there.\textsc{cf} walk-come-\textsc{ss.sim}-be-1s/p.\textsc{ds} say-\textsc{Np}-\textsc{fu}.2s\\
\glt`So you see me now. Otherwise I'll be walking there and you will say, ``{\dots}'''
\z
% * <liisa.berghall@gmail.com> 2015-05-22T13:45:33.811Z:
%
%  What is inconsistent here? 'walk' is the best gloss for soomar-
%
% ^ <liisa.berghall@gmail.com> 2015-08-20T14:45:54.981Z.
% * <liisa.berghall@gmail.com> 2015-05-27T13:14:20.188Z:
%
%  OK, found the inconsistency: ss-sim should be ss.sim.  Corrected it
%
% ^ <liisa.berghall@gmail.com> 2015-08-20T14:46:00.292Z.

\section{Clause chaining} \label{sec:8.2}
%\hypertarget{RefHeading23101935131865}

Clause chaining is a feature typical of Papuan languages, and of the Trans-New Guinea languages in particular.\footnote{\citet[36]{Wurm1982} seems to consider clause chaining a genetic feature of the \textsc{tng} languages, but \citet[xlvii]{Haiman1980} suggests that it is an areal feature. \citet[122]{Roberts1997}, with the most data to date, suggests that there is a combination of both, but leaves the final decision open.}  A sentence may consist of several medial clauses\footnote{The terms \textit{medial} and \textit{final} clauses are well established in Papuan linguistics. }   where the verbs have medial verb inflection (\sectref{sec:3.8.3.5}), and a final clause where the verb has ``normal'' finite inflection (\sectref{sec:3.8.3.4}). Clause chaining indicates either temporal sequence or simultaneity between adjacent clauses.

The division into just medial and final clauses is not adequate for describing the system. \citet[xii]{HaimanEtAl1983} call the medial clauses \textstyleEmphasizedWords{{marking clauses}} and the clauses following them \textstyleEmphasizedWords{{reference clauses}}.\footnote{\citet{Comrie1983} and \citet{Roberts1997} call them \textit{marked clauses} and \textit{controlling clauses}, respectively.} Marking clause is simply another name for a medial clause and will not be used here. But a reference clause may be medial or finite\footnote{I prefer the term \textit{finite} to \textit{final} clauses (and verbs), as it is the finiteness rather than the position in the sentence that is important in their relation with medial clauses. Subordinate clauses are the most typical \textit{non-final} finite clauses, and they may also have medial clauses preceding them and relating to them.} -- what is important is that both the temporal relationship of the medial verb, and the person reference, is stated in relation to the reference clause. When a reference clause for a preceding medial clause is also a medial clause, it again has its own reference clause following it.

The medial clauses linked by clause chaining are sometimes called \textstyleEmphasizedWords{{cosubordinate}} (\citealt{Olson1981}, \citealt[257]{FoleyEtAl1984}\footnote{This is cosubordination at the \textit{peripheral} level; verb serialization is cosubordination at core or nuclear level.} or \textstyleEmphasizedWords{{coordinate-}}\textstyleEmphasizedWords{{dependent}} \citep[177]{Foley1986}, because they share features with both coordinate and subordinate clauses. Their relationship with each other and with the following finite clause is essentially coordinate,\footnote{\citet{Roberts1988a} brings several syntactic arguments to show that basically switch reference is indeed coordination rather than subordination. But he also argues for a separate subordinate switch reference in Amele and some other languages.} but the medial clauses are dependent on the finite clause both for their absolute tense, and, in the case of ``same subject'' forms, also for their person/number specification. 

Another term commonly used for the chained clauses, \textstyleEmphasizedWords{{switch-reference clauses}} (\textsc{sr}),\footnote{Clause chaining and switch reference are two separate strategies, but in Papuan languages the two very often  go together \citep[104]{Roberts1997}.} is related to their other function as a reference-tracking device \citep[ix]{HaimanEtAl1983}. They typically indicate whether their topic/subject is the same as, or different from, the topic/subject of the following clause. This is discussed below in \sectref{sec:8.2.3}. In this grammar the two terms are used interchangeably, as in Mauwake the medial verbs not only indicate a temporal relationship but are used for reference tracking as well.

\subsection{Chained clauses as coordinate clauses} \label{sec:8.2.1}
%\hypertarget{RefHeading23121935131865}

It is widely accepted that the relationship of medial clauses to their reference clauses is basically coordinate, but with some special features and exceptions.\footnote{E.g., \citet[175, 193]{Reesink1987}, \citet[51]{Roberts1988a}, \citet[51, 1997]{Roberts1988a}.} In Mauwake medial clauses are subordinate only if subordinated with the topic/conditional marker -\textstyleStyleVernacularWordsItalic{na}; otherwise they are coordinate. 

Instead of giving background information like subordinate clauses do, medial clauses are predications that carry on the foreground story line \REF{ex:8:x2000}. But they are also different from coordinate finite clauses. The similarities and differences are discussed in this section.

The pragmatic additives \textstyleStyleVernacularWordsItalic{ne} \REF{ex:8:x1485} and \textstyleStyleVernacularWordsItalic{aria} \REF{ex:8:x1486} (\sectref{sec:3.11.1}) can occur between a medial clause and its reference clause, as between normal coordinate clauses. This is uncommon, however.

\ea%x1444
\label{ex:8:x2000} % I have moved this from 8:x1444 to 8:x2000 to avoid multiple definitions. Fortunately, x1444/x2000 is NOT referenced anywhere. 
\gll Wiawi  ikiw-ep  maak-eya,  \textbf{ne}  wiawi=ke  maak-e-k  {\dots} \\
3s/p.father  go-\textsc{ss}.\textsc{seq} tell-2/3s.\textsc{ds} \textsc{add}  3s/p.father=\textsc{cf} tell-\textsc{pa}-3s\\
\glt`She went to her father and told him, and her father told her {\dots}'
\z


\ea%x1485
\label{ex:8:x1485}
\gll ...  wiena  en-emi,  epira  lolom  if-emi  \textbf{ne}  owowa p-urup-em-ik-e-mik.\\
...  3p.\textsc{gen} eat-\textsc{ss}.\textsc{sim} plate  mud  spread-\textsc{ss}.\textsc{sim} \textsc{add} village \textsc{PBx}-ascend-\textsc{ss}.\textsc{sim}-be-\textsc{pa}-1/3p\\
\glt`They ate it themselves, spread mud on the plates, and brought them up to the village.'
\z


\ea%x1442
\label{ex:8:x1486} % I have moved x1442 to x1486 in order to avoid conflicts. Fortunately, x1442/x1486 is not referenced anywhere.
\gll I  ikoka  yien=iw  urup-ep  nia  maak-omkun ora-iwkin,  \textstyleEmphasizedVernacularWords{aria}  owawiya  feeke  pok-ap  ik-ok  eka liiwa  muuta  en-ep  \textstyleEmphasizedVernacularWords{aria}  ni  soomar-ek-eka.\\
1p.\textsc{unm} later 1p.\textsc{gen}=\textsc{lim} ascend-\textsc{ss}.\textsc{seq} 2p.\textsc{acc} tell-1s/p.\textsc{ds} descend-2/3p.\textsc{ds} alright together here.\textsc{cf} sit-\textsc{ss}.\textsc{seq} be-\textsc{ss} water little  only  eat-\textsc{ss}.\textsc{seq} alright  2p.\textsc{unm} walk-go-2p.\textsc{imp}\\
\glt`Later we (by) ourselves will come up and tell you (to come), and when you come down we will sit here together and eat a little bit of soup and then you can walk back.'
\z


Coordinated main clauses are free in regard to their mood and, related to that, their functional sentence type. The medial clauses do not have any marking for mood.  They usually conform to that of the finite clause, but this is a pragmatic matter, not a syntactic requirement. 

When either the medial clause or the finite clause is a question, the whole sentence is interrogative, even if the other clause is a statement. In \REF{ex:8:x1449} the finite clause is a polar question, but the medial clause is not questioned. In the story that \REF{ex:8:x1452} is taken from, the killing is not questioned, only the manner. But since a medial clause cannot take the question marker, the verb in the finite clause has to carry the marking.

\ea%x1449
\label{ex:8:x1449}
\gll Sande  erup  weeser-eya  owowa  ekap-e-man=i? \\
week  two  finish-2/3s.\textsc{ds} village  come-\textsc{pa}-2p=\textsc{qm}\\
\glt`Two weeks were finished, and did you (then) come to the village?'\footnote{Another possible translation is `When the two weeks were finished, did you (then) come to the village?' but this does not reflect the coordinate relationship of the clauses in the original.}
\z


\ea%x1452
\label{ex:8:x1452}
\gll Naap  on-ap  ifakim-i-nen=i?\\
thus do-\textsc{ss}.\textsc{seq} kill-\textsc{Np}-\textsc{fu}.1s=\textsc{qm}\\
\glt`Shall I do like that and kill her?' (Or: `Is it in that way that I shall kill her?')
\z


A non-polar question can be in either a medial \REF{ex:8:x1451} or in a finite clause \REF{ex:8:x1450}.

\ea%x1451
\label{ex:8:x1451}
\gll No  sira  kamenap  on-eya  napuma  fain nefar  kerer-e-k?\\
2s.\textsc{unm} custom  how  do-2/3s.\textsc{ds} sickness  this 2s.\textsc{dat} appear-\textsc{pa}-3s\\
\glt`What did you do (so that) this sickness came to you?'
\z


\ea%x1450
\label{ex:8:x1450}
\gll No  karu-emi  kame  kaanek  ikiw-o-n? \\
2s.\textsc{unm} run-\textsc{ss}.\textsc{sim} side  where  go-\textsc{pa}-2s\\
\glt`You ran and where did you go?'
\z


For more examples, see (7.\ref{ex:7:x1082})--(7.\ref{ex:7:x1083}) in \sectref{sec:7.3} and the introductory section to \chapref{chap:8}. 

In regard to the scope of negation, the same-subject medial clauses differ from all other clauses.  Negative spreading (\sectref{sec:6.2.4}) in both directions is allowed only between \textstyleAcronymallcaps{\textsc{ss}} medial clauses and their reference clauses, and even there it is not very common. Backwards spreading is especially rare. In the following examples, negative spreading takes place in \REF{ex:8:x1443} and \REF{ex:8:x1447}, but not in \REF{ex:8:x1446} and \REF{ex:8:x1448}. Between other types of clauses negative spreading is not permitted at all. 

\ea%x1443
\label{ex:8:x1443}
\gll Nainiw  \textbf{ekap-ep}  maa  \textbf{me} \textbf{sesenar-e-mik}. \\
again  come-\textsc{ss}.\textsc{seq} food  not  sell-\textsc{pa}-1/3p\\
\glt`They did not come back and sell food.'
\z


\ea%x1447
\label{ex:8:x1447}
\gll Ikiw-em-ik-ok  \textbf{me} \textbf{kir-ep} \textbf{uruf-e}, no  oram  woolal-ikiw-em-ik-e.\\
go-\textsc{ss}.\textsc{sim}-be-\textsc{ss} not  turn-\textsc{ss}.\textsc{seq} look-\textsc{imp}.2s 2s.\textsc{unm} just  paddle-go-\textsc{ss}.\textsc{sim}-be-\textsc{imp}.2s\\
\glt`While going, don't turn and look back, just keep paddling.'
\z


\ea%x1446
\label{ex:8:x1446}
\gll Yaapan=ke  urup-em-ika-iwkin  wi  Australia=ke wia  uruf-ap  baurar-emi  \textstyleEmphasizedVernacularWords{me} \textstyleEmphasizedVernacularWords{yia}  \textstyleEmphasizedVernacularWords{maak-e-mik}.\\
Japan=\textsc{cf} ascend-\textsc{ss}.\textsc{sim}-be-2/3p.\textsc{ds} 3p.\textsc{unm} Australia=\textsc{cf} 3p.\textsc{acc} see-\textsc{ss}.\textsc{seq} flee-\textsc{ss}.\textsc{sim} not 1p.\textsc{acc} tell-\textsc{pa}-1/3p\\
\glt`When the Japanese were coming up the Australians saw them and ran away and/but did not tell us.'
\z


\ea%x1448
\label{ex:8:x1448}
\gll Iiriw  auwa=ke  sira  fain  \textbf{me}  \textbf{paayar-ep} muuka  momor  wiar  aaw-em-ik-e-mik.\\
earlier  1s/p.father=\textsc{cf} custom  this  not  understand-\textsc{ss}.\textsc{seq} son indiscriminately 3.\textsc{dat} get-\textsc{ss}.\textsc{sim}-be-\textsc{pa}-1/3p\\
\glt`Earlier our (fore)fathers didn't understand this custom, and (so) they adopted (lit: got/took) children indiscriminately.'
\z


Like coordinated main clauses and unlike subordinate clauses, medial clauses are not embedded as constituents in other clauses. However, a medial clause may interrupt its reference clause and appear inside it, if the subject or object noun phrase of the reference clause is fronted as the theme and thus precedes the interrupting medial clause \REF{ex:8:x1464}. For more examples, see (3.\ref{ex:3:x539}) and (3.\ref{ex:3:x540}). In the examples, the reference clause is bolded and the intervening medial clause is placed within square brackets.

\ea%x1464
\label{ex:8:x1464}
\gll Aria  \textstyleEmphasizedVernacularWords{yena} \textstyleEmphasizedVernacularWords{mua}  \textstyleEmphasizedVernacularWords{pun}  {\ob}irak-owa  kerer-owa  epa weeser-em-ik-eya{\cb}  \textstyleEmphasizedVernacularWords{iirar-iwkin}  owowa  ekap-o-k, o  amia  mua=pa  ik-ok.\\
alright  1s.\textsc{gen} man  too  fight-\textsc{nmz} appear-\textsc{nmz} time finish-\textsc{ss}.\textsc{sim}-be-2/3s.\textsc{ds} remove-2/3p.\textsc{ds} village come-\textsc{pa}-3s 3s.\textsc{unm} bow man=\textsc{loc} be-\textsc{ss}\\
\glt`Alright, the war was getting close and they dismissed my husband and he came to the village, after he had been a soldier.'
\z


In \REF{ex:8:x1465}, both the object and the subject are fronted. After the first medial clause, the object of the finite clause is fronted as the theme of the remainder of the sentence, and it pulls with it the subject, marked with the contrastive focus marker.  In the free translation, passive is used, because the object is fronted as a theme.

\ea%x1465
\label{ex:8:x1465}
\gll Sisina=pa  wu-ap  \textstyleEmphasizedVernacularWords{papako}\textsubscript{O}  \textstyleEmphasizedVernacularWords{mua=ke}\textsubscript{S}  {\ob}mera  saa urup-eya{\cb}  \textstyleEmphasizedVernacularWords{patopat=iw}  \textstyleEmphasizedVernacularWords{mik-i-mik}. \\
shallow.water=\textsc{loc} put-\textsc{ss}.\textsc{seq} some man=\textsc{cf} fish sand ascend-2/3s.\textsc{ds} fishing.spear=\textsc{inst} spear-\textsc{Np}-\textsc{pr}.1/3p\\
\glt`They drive (lit: put) them to the shallow water and the fish ascend to the beach and (then) some are speared by men with a fishing spear.'
\z


The examples \REF{ex:8:x1466}--\REF{ex:8:x1468} show that some of the same-subject medial clauses interrupting the reference clause, especially those that have a directional verb or the verb \textstyleStyleVernacularWordsItalic{aaw}- `take, get', may be in the process of grammaticalizing into serial verbs: 

\ea%x1466
\label{ex:8:x1466}
\gll \textstyleEmphasizedVernacularWords{I} \textstyleEmphasizedVernacularWords{iwer(a)} \textstyleEmphasizedVernacularWords{eka}  {\ob}iki(w-e)p{\cb}  \textstyleEmphasizedVernacularWords{nop-a-mik}. \\
1p.\textsc{unm} coconut  water  go-\textsc{ss}.\textsc{seq} fetch-\textsc{pa}-1/3p\\
\glt`We went and fetched coconut water.'
\z


\ea%x1467
\label{ex:8:x1467}
\gll \textstyleEmphasizedVernacularWords{Yo}  \textstyleEmphasizedVernacularWords{merena}  {\ob}fura  aaw-ep{\cb}  \textstyleEmphasizedVernacularWords{puuk-a-m}. \\
1s.\textsc{unm} leg  knife  take-\textsc{ss}.\textsc{seq} cut-\textsc{pa}-1s\\
\glt`I took a knife and cut (into) the leg. (Or: I cut into the leg with a knife.')
\z


\ea%x1468
\label{ex:8:x1468}
\gll Um-eya  \textstyleEmphasizedVernacularWords{merena}  \textstyleEmphasizedVernacularWords{ere-erup}  {\ob}ifara  aaw-ep{\cb}  \textstyleEmphasizedVernacularWords{kaik-ap} nabena  suuw-ap  akua  aaw-ep  or-o-m.\\
die-2/3s.\textsc{ds} leg \textsc{rdp}-two rope take-\textsc{ss}.\textsc{sim} tie-\textsc{ss}.\textsc{seq} carrying.pole push-\textsc{ss}.\textsc{seq} shoulder take-\textsc{ss}.\textsc{seq} descend-\textsc{pa}-1s\\
\glt`It (=a pig) died and I took a rope and tied its legs two and two together and pushed it to a carrying pole and carried it down on my shoulder.'
\z


\textstyleEmphasizedWords{{Right-dislocation}} of a medial clause is not unusual. One reason commonly given for right-dislocations is an afterthought: the speaker notices something that should be part of the sentence and adds it to the end \REF{ex:8:x1471}. Another reason is giving prominence to the dislocated clause, since the end of a sentence is a focal position. The right-dislocation of same-subject sequential medial clauses in particular breaks the iconicity between the events and the sentence structure, and has this effect. Consequently, the right-dislocated \textstyleAcronymallcaps{\textsc{ss}} sequential clauses, like the ones in examples \REF{ex:8:x1469} and \REF{ex:8:x1470}, are much more prominent than medial clauses in their normal position.

\ea%x1471
\label{ex:8:x1471}
\gll Or-op  naap  wia  uruf-a-mik,  {\ob}mua  oona,  eneka,  woosa kia  kir-em-ik-eya{\cb}. \\
descend-\textsc{ss}.\textsc{seq} thus 3p.\textsc{acc} see-\textsc{pa}-1/3p man  bone  tooth  head white  turn-\textsc{ss}.\textsc{sim}-be-2/3s.\textsc{ds}\\
\glt`They went down and saw them like that, the people's bones, teeth and heads turning white.'
\z


\ea%x1469
\label{ex:8:x1469}
\gll Aw-iki(w-e)m-ik-eya  wiena  mua  unowa  fiker(a)  epia nain  ook-i-kuan,  {\ob}wiowa  aaw-ep{\cb}.\\
burn-go-\textsc{ss}.\textsc{sim}-be-2/3s.\textsc{ds} 3p.\textsc{gen} man  many  kunai.grass  fire that1 follow-\textsc{Np}-\textsc{fu}.3p  spear  take-\textsc{ss}.\textsc{seq}\\
\glt`It keeps burning and many men follow the kunai grass fire, having taken spears.'
\z


\ea%x1470
\label{ex:8:x1470}
\gll Aaya  muuna  kuisow  enim-i-mik,  {\ob}aite=ke manina=pa  yia  aaw-om-iwkin{\cb}.   \\
sugarcane  joint  one  eat-\textsc{Np}-\textsc{pr}.1/3p 1s/p.mother=\textsc{cf} garden=\textsc{loc} 1p.\textsc{acc} get-\textsc{ben}-2/3p.\textsc{ds}\\
\glt`We eat one joint of sugarcane, when/after our mothers have gotten it for us from the garden.'
\z


\subsection{Temporal relations in chained clauses}
%\hypertarget{RefHeading23141935131865}

Clause chaining in Mauwake distinguishes between sequential and simultaneous actions in the clauses joined by chaining, but only when the clauses have the same subject (\sectref{sec:3.8.3.5.1}). The sequential action verb in \REF{ex:8:x1431} indicates that one action is finished before the next one starts. 

\ea%x1431
\label{ex:8:x1431}
\gll No  nainiw  kir-\textstyleEmphasizedVernacularWords{ep}  ikiw-\textstyleEmphasizedVernacularWords{ep}  owow  mua  wia maak-eya  urup-\textstyleEmphasizedVernacularWords{ep}  mukuna  nain  umuk-uk. \\
2s.\textsc{unm} again  turn-\textsc{ss}.\textsc{seq} go-\textsc{ss}.\textsc{seq} village  man  3p.\textsc{acc} tell-2/3s.\textsc{ds}  ascend-\textsc{ss}.\textsc{seq} fire  that1 extinguish-\textsc{imp}.3p\\
\glt`Turn around, go and tell the village men and let them come up and extinguish the fire.'
\z


When a clause has a simultaneous action medial verb \REF{ex:8:x1432}, it indicates at least some overlap with the action in the following clause. 

\ea%x1432
\label{ex:8:x1432}
\gll Or-\textstyleEmphasizedVernacularWords{omi}  yo  koka  koora=pa  nan  efa wu-\textstyleEmphasizedVernacularWords{ami}  ma-e-k,  `` ... '' \\
descend-\textsc{ss}.\textsc{sim} 1s.\textsc{unm} jungle house=\textsc{loc} there 1s.\textsc{acc} put-\textsc{ss}.\textsc{sim} say-\textsc{pa}-3s\\
\glt`As he went down, he put me in the jungle house and said, `` ... '' '
\z


Simultaneity vs. sequentiality is not always a choice between absolutes; sometimes it is a relative matter. Example \REF{ex:8:x1433} refers to a situation where a man came back home from a period of labour elsewhere and got married upon arrival. In actual life, there may have been a time gap of at least a number of days, possibly longer, but because the two events were so closely linked in the speaker's mind, the simultaneous action form was used when the story was told decades after the events took place.

\ea%x1433
\label{ex:8:x1433}
\gll Ekap-\textstyleEmphasizedVernacularWords{emi}  yo  efa  aaw-o-k. \\
come-\textsc{ss}.\textsc{sim} 1s.\textsc{unm} 1s.\textsc{acc} take-\textsc{pa}-3s\\
\glt`He came and married me.'
\z


The simultaneous action form is less marked than the sequential action form: when the relative order of the actions or events is not relevant, the simultaneous action form is used. In example \REF{ex:8:x1437}, the order of the preparations for a pighunt is not crucial, but the sequential action form on the last medial verb indicates that all the actions take place before leaving, rather than just at the time of leaving.  

\ea%x1437
\label{ex:8:x1437}
\gll Maa  en-ep-pu-\textstyleEmphasizedVernacularWords{ami}  top  aaw-\textstyleEmphasizedVernacularWords{emi}  moma unukum-\textstyleEmphasizedVernacularWords{emi}  kapit,  wiowa  aaw-\textstyleEmphasizedVernacularWords{ep}  fikera iw-i-mik.\\
food  eat-\textsc{ss}.\textsc{seq}-\textsc{cmpl}-\textsc{ss}.\textsc{sim} trap take-\textsc{ss}.\textsc{sim} taro wrap-\textsc{ss}.\textsc{sim} trap.frame spear take-\textsc{ss}.\textsc{seq} kunai.grass go-\textsc{Np}-\textsc{pr}.1/3p\\
\glt`We eat, take the trap, wrap taro, take the the trap frame and spear(s) and go to the kunai grass area.'
\z


A medial verb takes its temporal specification from the tense of the closest following finite clause \REF{ex:8:x1442}--\REF{ex:8:x1445}, or in the case of a right-dislocated medial clause, from the preceding finite clause \REF{ex:8:x1471}.

\ea%x1442
\label{ex:8:x1442}
\gll Nomokowa  maala  war-ep    ekap-ep  ifa  nain  ifakim-\textstyleEmphasizedVernacularWords{o}-k.\\
tree  long  cut-\textsc{ss}.\textsc{seq} come-\textsc{ss}.\textsc{seq} snake that1 kill-\textsc{pa}-3s\\
\glt`He cut a long stick, came and killed the snake.'
\z


\ea%x1444
\label{ex:8:x1444}
\gll Mua=ke  kais-ap  neeke  wu-ap  miiw-aasa  nop-ap miiw-aasa=ke  iwer(a)  ififa  nain  aaw-ep  p-ekap-ep epia  koora  mamaiya=pa  wu-eya  fook-\textstyleEmphasizedVernacularWords{i-mik}.\\
man=\textsc{cf} husk-\textsc{ss}.\textsc{seq} there.\textsc{cf} put-\textsc{ss}.\textsc{seq} land-canoe fetch-\textsc{ss}.\textsc{seq} land-canoe=\textsc{cf} coconut dry that1 take-\textsc{ss}.\textsc{seq} \textsc{\textsc{bp}x}-come-\textsc{ss}.\textsc{seq} fire house near=\textsc{loc} put-2/3s.\textsc{ds} split-\textsc{Np}-\textsc{pr}.1/3p\\
\glt`Men husk them (coconuts) and put them there and fetch a truck, and the truck takes the dry coconuts and brings them close to the drying shed (lit: fire house), and we split them.'
\z


\ea%x1445
\label{ex:8:x1445}
\gll Ikoka  mua  ar-ep  emeria  aaw-ep  kamenap  on-\textstyleEmphasizedVernacularWords{i-nan}? \\
later  man  become-\textsc{ss}.\textsc{seq} woman take-\textsc{ss}.\textsc{seq} how do-\textsc{Np}-\textsc{fu}.2s\\
\glt`Later when you become a man and take a wife, what will you do?'
\z


The \textstyleAcronymallcaps{\textsc{ds}} medial verbs (\sectref{sec:3.8.3.5.2}) do not differentiate between sequential and simultaneous action. Sequential action \REF{ex:8:x1502} is the default interpretation for verbs other than \textstyleStyleVernacularWordsItalic{ik}- `be', which is interpreted as simultaneous with the verb in the reference clause \REF{ex:8:x1503}. So in order to specify that two or more actions by different participants took place at the same time, the speaker needs to use the continuous aspect form \REF{ex:8:x1472}:

\ea%x1502
\label{ex:8:x1502}
\gll Maa  unowa  ifer-aasa=ke  p-urup\textstyleEmphasizedVernacularWords{-eya} miiw-aasa=ke fan  p-ir-am-ik-ua.\\
thing  many  sea-canoe=\textsc{cf} \textsc{bpx}-ascend-2/3s.\textsc{ds} land-canoe=\textsc{cf} here \textsc{bpx}-come-\textsc{ss}.\textsc{sim}-be-\textsc{pa}.3s\\
\glt`The cargo was brought up (to the coast) by ship(s), and (then) trucks kept bringing it here.'
\z


\ea%x1503
\label{ex:8:x1503}
\gll Wi  yapen=pa  \textstyleEmphasizedVernacularWords{ik-}omak\textstyleEmphasizedVernacularWords{-iwkin}  Amerika  kerer-e-mik.\\
3p.\textsc{unm} inland=\textsc{loc} be-\textsc{distr}/\textsc{pl}-2/3p.\textsc{ds} America appear-\textsc{pa}-1/3p\\
\glt`Many people were inland and the Americans arrived.'
\z


\ea%x1472
\label{ex:8:x1472}
\gll Ek-ap  umuk-i-nen  na-ep  on-am\textstyleEmphasizedVernacularWords{-ik-eya} ifa=ke  keraw-a-k,  ...\\
go-\textsc{ss}.\textsc{seq} extinguish-\textsc{Np}-\textsc{fu}.1s  say-\textsc{ss}.\textsc{seq} do-\textsc{ss}.\textsc{sim}-be-2/3s.\textsc{ds} snake=\textsc{cf} bite-\textsc{pa}-3s\\
\glt`He went and as he was trying to extinguish it (a fire), a snake bit him, {\dots}'
\z


Although the chaining structure itself only specifies the temporal relationship between the clauses and is otherwise neutral, it is open especially for causal/consecutive interpretation. \citet[237]{Reesink1983b} notes this for different-subject medial verbs in Usan, and although not very common in Mauwake in general, it is more frequent with \textsc{ds} predicates \REF{ex:8:x1434}, \REF{ex:8:x1412}  than with \textsc{ss} verbs.

\ea%x1434
\label{ex:8:x1434}
\gll Yo  maamuma  marew\textstyleEmphasizedVernacularWords{-eya} maak-e-m, {\textquotedbl}Iir  oko=pa  ni-i-nen.''\\
1s.\textsc{unm} money no(ne)-2/3s.\textsc{ds} tell-\textsc{pa}-1s time  other=\textsc{loc} give.you-\textsc{Np}-\textsc{fu}.1s\\
\glt`I had no money and I told him (or: Because I had no money I told him), ``I'll give it to you another time.'' '
\z


\ea%x1412
\label{ex:8:x1412}
\gll Iperowa=ke  kekan-\textstyleEmphasizedVernacularWords{iwkin}  ma-e-mik,  ``Aria,  ...'' \\
middle.aged=\textsc{cf} be.strong-2/3p.\textsc{ds} say-\textsc{pa}-1/3p alright\\
\glt`The elders insisted, and (so) we said, ``All right, {\dots}'' '
\z


The causal/consecutive interpretation is most common when the object of a transitive medial clause becomes the subject in an intransitive reference clause: in example \REF{ex:8:x1504} `the son' is the object of the first two clauses and the subject of the final clause.

\ea%x1504
\label{ex:8:x1504}
\gll {\ob}\textstyleEmphasizedVernacularWords{Muuka}{\cb}\textsubscript{O}  p-or-op \textstyleEmphasizedVernacularWords{p-er-iwkin}  \textstyleEmphasizedVernacularWords{yak-i-ya}. \\
son \textsc{bpx}-descend-\textsc{ss}.\textsc{seq} \textsc{bpx}-go-2/3p.\textsc{ds} bathe-\textsc{Np}-\textsc{pr}.3s\\
\glt`They bring the son down (from the house) and take him (to the well) and (so) he bathes.'
\z


Cognition verbs and feeling or experiential verbs seem to be the only ones that allow a causal/consecutive interpretation when a medial clause has a \textsc{ss} verb \REF{ex:8:x1440}--\REF{ex:8:x1484}:

\ea%x1440
\label{ex:8:x1440}
\gll Siiwa,  epa  maak-e-mik  nain  \textstyleEmphasizedVernacularWords{paayar-ep}  ma-e-k, ``Amerika  aakisa  irak-owa  kerer-e-mik.''\\
moon  place/time  tell-\textsc{pa}-1/3p that1  understand-\textsc{ss}.\textsc{seq} say-\textsc{pa}-3s America now fight-\textsc{nmz} appear-\textsc{pa}-1/3p\\
\glt`He understood the month and time/place that they (had) told him, and (so) he said, ``Now the Americans have come to fight.'' '
\z


\ea%x1441
\label{ex:8:x1441}
\gll ...  ne  wi  ikiw-e-mik,  \textstyleEmphasizedVernacularWords{kerewar-ep}  ikiw-e-mik. \\
...  \textsc{add} 3p.\textsc{unm} go-\textsc{pa}-1/3p become.angry-\textsc{ss}.\textsc{seq} go-\textsc{pa}-1/3p\\
\glt`{\dots} and they went; they were angry and (so) they went.'
\z


\ea%x1484
\label{ex:8:x1484}
\gll Mua  oko=ko  \textstyleEmphasizedVernacularWords{napum-ar-}\textstyleEmphasizedVernacularWords{ep}  ikemika  kaik-ow(a)  mua wiar  ikiw-o-k.\\
man  other=\textsc{cf} sickness-\textsc{inch}-\textsc{ss}.\textsc{seq} wound tie-\textsc{nmz} man 3.\textsc{dat} go-\textsc{pa}-3s\\
\glt`A man got sick and (so) he went to a doctor.'
\z


\subsection{Person reference in chained clauses} \label{sec:8.2.3}
%\hypertarget{RefHeading23161935131865}

The switch-reference marking tracks the referents in a different way from the person/ number marking in finite verbs. The medial verb suffix indicates whether the clause has the same subject/topic as the reference clause that comes after it, and the \textstyleAcronymallcaps{\textsc{ds}} suffixes also have some specification of the subject (\sectref{sec:3.8.5.2}). In \REF{ex:8:x1436}, the subjects are a man and his wife in the first two clauses and in the last one, and a spirit man in all the others:

\ea%x1436
\label{ex:8:x1436}
\gll Ikiw-\textstyleEmphasizedVernacularWords{ep}\textsubscript{i}  nan  ika-\textstyleEmphasizedVernacularWords{iwkin}\textsubscript{i}  inasina  mua\textsubscript{j}  ifa  puuk-\textstyleEmphasizedVernacularWords{ap}\textsubscript{j}  solon-\textstyleEmphasizedVernacularWords{ep}\textsubscript{j}  urup-\textstyleEmphasizedVernacularWords{ep}\textsubscript{j}  manina=pa  waaya puuk-\textstyleEmphasizedVernacularWords{ap}\textsubscript{j}  moma  wiar  en-em-ik-\textstyleEmphasizedVernacularWords{eya}\textsubscript{j} uruf-a-mik\textsubscript{i}.\\
go-\textsc{ss}.\textsc{seq} there be-2/3p.\textsc{ds} spirit  man  snake change.into-\textsc{ss}.\textsc{seq} crawl-\textsc{ss}.\textsc{seq} ascend-\textsc{ss}.\textsc{seq} garden=\textsc{loc} pig change.into-\textsc{ss}.\textsc{seq} taro 3.\textsc{dat} eat-\textsc{ss}.\textsc{sim}-be-2/3s.\textsc{ds} see-\textsc{pa}-1/3s\\
\glt`They went and were there, and a spirit man came and changed into a snake and crawled up and in the garden it changed into a pig and as it was eating their taro they saw it.'
\z


Because the switch reference marking relates to the subject/topic in two different clauses at the same time, this sometimes causes ambiguities that need to be solved. If the subjects in adjacent clauses are partially same and partially different, a choice has to be made whether they are marked as \textsc{ss} or \textsc{ds}; only a few Papuan languages have a choice of marking both \textsc{ss} and \textsc{ds} on the same verb \citep{Roberts1997}. Also, if the \textsc{sr} marking is considered to track the syntactic subject, there are a number of apparent irregularities in the marking. These have been discussed in particular by \citet{Reesink1983a} and \citet{Roberts1988b} with reference to Papuan languages. The next three subsections describe how Mauwake deals with these questions. 

\subsubsection{Partitioning of the participant set}
%\hypertarget{RefHeading23181935131865}

When one of the subjects is plural and the other is singular included in the plural, this mismatch theoretically allows for a number of different choices in the switch-reference marking, but in practice each language limits this choice in a way peculiar to it.\footnote{For a summary of how different Papuan languages treat this area of ambiguity, see \citet{Reesink1983a}, \citet[201--202]{Reesink1987} and \citet[87--91]{Roberts1988b}.} The following table \tabref{tab:15:switchref} shows this for Mauwake. 


\begin{table}
\caption{Switch-reference marking with partial overlap of subjects}
\label{tab:15:switchref}
\begin{tabular}{llll}
\mytoprule
\multicolumn{2}{l}{{\bfseries Singular to plural}}
 & \multicolumn{2}{l}{{\bfseries Plural to singular}}\\
\midrule
1s  {{\textgreater}}  1p & \textsc{ss} & 1p  {{\textgreater}}  1s & \textsc{ss}\\
2s  {{\textgreater}}  1p & \textsc{ds} & 1p  {{\textgreater}}  2s & \textsc{ss}\\
2s  {{\textgreater}}  2p & \textsc{ss}/\textsc{ds} & 1p  {{\textgreater}}  3s & \textsc{ss}\\
3s  {{\textgreater}}  1p & \textsc{ss}/\textsc{ds} & 2p  {{\textgreater}}  2s & \textsc{ss}\\
3s  {{\textgreater}}  2p & \textsc{ss}/\textsc{ds} & 2p  {{\textgreater}}  3s & \textsc{ss}\\
3s  {{\textgreater}}  3p & \textsc{ss}/\textsc{ds} & 3p  {{\textgreater}}  3s & \textsc{ss}\\
\mybottomrule
\end{tabular}
\end{table}


When a plural subject changes into a singular, the suffix is always the one used for same subject \REF{ex:8:x1438}. 

\ea%x1438
\label{ex:8:x1438}
\gll {\dots}owowa  urup-e-mik.  Owowa  urup-\textbf{ep } o  koora ikiw-o-k.\\
{\dots}village  ascend-\textsc{pa}-1/3p  village  ascend-\textsc{ss}.\textsc{seq} 3s.\textsc{unm} house go-\textsc{pa}-3s\\
\glt`{\dots}We came up to the village. After we came up to the village he went into the house.'
\z


When a singular subject changes into a plural there is more variation. First person singular changing into plural calls for same-subject marking \REF{ex:8:x1435}, but second person singular switching into first person plural requires different-subject marking even when this second person singular is part of the group denoted by the first person plural \REF{ex:8:x1439}. 

\ea%x1435
\label{ex:8:x1435}
\gll Mik-ap,  patot=iw  mik-ap,  aaw-ep, aasa=pa  wu-ap,  amap-urup-ep,  yena  koora=pa wu-ap,  uuriw  epa  wiim-eya  or-op, saa=pa  pa-\textstyleEmphasizedVernacularWords{ep}  uup-e-mik.\\
spear-\textsc{ss}.\textsc{seq} fishing.spear=\textsc{inst} spear-\textsc{ss}.\textsc{seq} take-\textsc{ss}.\textsc{seq} canoe=\textsc{loc} put-\textsc{ss}.\textsc{seq} \textsc{\textsc{bp}x}-ascend-\textsc{ss}.\textsc{seq} 1s.\textsc{gen} house=\textsc{loc} put-\textsc{ss}.\textsc{seq} morning place get.light-2/3s.\textsc{ds} descend-\textsc{ss}.\textsc{seq} sand=\textsc{loc} butcher-\textsc{ss}.\textsc{seq} cook-\textsc{pa}-1/3p\\
\glt`I speared it, I speared it with a fishing spear, and took it and put it in the canoe, brought it up and put it in my house, and in the morning when it was light I went down and butchered it on the beach, and \textstyleEmphasizedWords{we} cooked it.'
\z


\ea%x1439
\label{ex:8:x1439}
\gll Ekap-\textbf{eya} ikiw-i-yen.\\
come-2/3s.\textsc{ds} go-\textsc{Np}-\textsc{fu}.1p\\
\glt`When you come we (including you) will go.'
\z


When a second person plural switches into a first person plural (including the people indicated by the 2p), the marking has to be for different subject, but in the opposite case, the first person plural changing into the second person plural (again included in the 1p), the marking can be either for same or different subject. Both of these are exemplified in \REF{ex:8:x248}. Here the switch from first person plural to second person plural is marked with the \textsc{ss} marking.

\ea
\label{ex:8:x248}
\gll I  ikoka  yien=iw  urup-ep  nia  maak-omkun ora-\textstyleEmphasizedVernacularWords{iwkin}  aria  owawiya  feeke  pok-ap  ik-ok eka  liiwa  muuta  en-\textbf{ep}  aria  ni  soomar-ek-eka. \\
1p.\textsc{unm} later 1p.\textsc{gen}=\textsc{lim} ascend-\textsc{ss}.\textsc{seq} 2p.\textsc{acc} tell-1s/p.\textsc{ds} descend-2/3p.\textsc{ds} alright  together  here.\textsc{cf} sit-\textsc{ss}.\textsc{seq} be-\textsc{ss} water  a.little  only  eat-\textsc{ss}.\textsc{seq} alright 2p.\textsc{unm} walk-go-\textsc{imp}.2p\\
\glt `Later we (by) ourselves will come up and tell you (to come), and when you come down we will sit here together and eat a bit of something and then you (can) walk back.'
\z

% * <liisa.berghall@gmail.com> 2015-05-22T15:03:35.027Z:
%
%  In the pdf the example is OK
%
% ^ <liisa.berghall@gmail.com> 2015-07-27T15:16:38.516Z.

With the rest, the speaker has a choice between the two forms. This choice is probably pragmatic and depends on whether the speaker wants to stress the change or the continuity of the referents \citep[47]{Franklin1983}. 

\subsubsection{Tracking a subject high in topicality} 
%\hypertarget{RefHeading23201935131865}

\citet[xi]{HaimanEtAl1983} claim that it is strictly the syntactic subject whose reference is tracked, but this statement has been challenged and modified by several others.\footnote{\citet{Givon1983,Reesink1983a,Reesink1987,Roberts1988b,Roberts1997} and \citet{Farr1999} among others.} If it is accepted as such, both Mauwake and other Papuan languages present a number of irregularities that have to be explained somehow.

\citet[242--243]{Reesink1983a} suggests that the switch-reference system does monitor the subject co-referentiality in the medial clause and its reference clause, but topicality considerations cause apparent ``anomalies'' to the basic system.  \citet{Roberts1988b} makes a well supported claim for Amele that in fact it is the topic that is tracked rather than the syntactic subject, or semantic agent, and he tentatively extends the claim to cover other Papuan languages as well. His later survey \citep{Roberts1997} presents a more balanced view that \textstyleAcronymallcaps{\textsc{sr}} can be either agent-oriented or topic-oriented, while maintaining that in most Papuan languages it is topic-oriented. 

In a nominative-accusative language like Mauwake the syntactic subject, the semantic agent and the pragmatic topic coincide most of the time. The \textstyleAcronymallcaps{\textsc{sr}} marking tracks the subject, but when there is competition between a more topical and less topical subject in clause chains, it is the more topical one that is tracked. An object, even if it is the topic, does not participate in the \textstyleAcronymallcaps{\textsc{sr}} marking.

Competition between a more topical subject with a less topical one most commonly occurs when a clause with an inanimate subject intervenes between clauses where there is an animate/human subject. Even here the ``normal'' \textstyleAcronymallcaps{\textsc{sr}} strategy is used, if the inanimate subject is topical enough to control the \textstyleAcronymallcaps{\textsc{sr}} marking in the same way as animate subjects do. In the following examples, the drying of the soup \REF{ex:8:x1474} and the bending of the coconut palm \REF{ex:8:x1480} are important events in the development of the story and so the regular \textstyleAcronymallcaps{\textsc{sr}} marking is maintained. In \REF{ex:8:x1480}, the coconut palm can also be interpreted as a volitional participant, as it bends and straightens itself according to the needs of the people.
\largerpage

\ea%x1474
\label{ex:8:x1474}
\gll Uup-em-ika-\textstyleEmphasizedVernacularWords{iwkin}  maa  eka  saanar-em-ik-\textstyleEmphasizedVernacularWords{eya} iki(w-e)p  eka  un-ep  ekap-ep  amina=pa feef-am-ik-e-mik.\\
cook-\textsc{ss}.\textsc{sim}-be-2/3p.\textsc{ds} food water  dry-\textsc{ss}.\textsc{sim}-be-2/3s.\textsc{ds} go-\textsc{ss}.\textsc{seq}  water  draw-\textsc{ss}.\textsc{seq} come-\textsc{ss}.\textsc{seq} pot=\textsc{loc} pour-\textsc{ss}.\textsc{sim}-be-\textsc{pa}-1/3p\\
\glt`They were cooking it and the soup kept drying and they kept going and drawing water and coming and pouring it in the pot.'
% * <liisa.berghall@gmail.com> 2015-05-22T15:06:00.152Z:
%
%  Gap above the free translation too wide?
%
\z


\ea%x1480
\label{ex:8:x1480}
\gll Emeria  panewowa  nain  wiimasip  erup  wia  aaw-ep owow  uruma  or-op  iimar-ep  ika-\textstyleEmphasizedVernacularWords{iwkin} iwera  oko  mekemkar-ep  or-\textstyleEmphasizedVernacularWords{eya}  wi  iwera ir-\textstyleEmphasizedVernacularWords{iwkin}  nainiw  kaken  iimar-e-k.\\
woman  old  that1  3s/p.grandchild  two  3p.\textsc{acc} take-\textsc{ss}.\textsc{seq} village open.place descend-\textsc{ss}.\textsc{seq} stand.up-\textsc{ss}.\textsc{seq} be-2/3p.\textsc{ds} coconut  other  bend-\textsc{ss}.\textsc{seq} descend-2/3s.\textsc{ds} 3p.\textsc{unm} coconut climb-2/3p.\textsc{ds} again  straight  stand.up-\textsc{pa}-3s\\
\glt`The old woman took the two grandchildren and they went down to the village square and were standing there, and a coconut palm bent down and they climbed up the coconut palm and it stood up straight again.'
\z


When an inanimate subject is low in terms of topicality, the \textstyleAcronymallcaps{\textsc{sr}} marking of the previous clause disregards it and indicates same-subject continuation, but the verb in the inanimate clause has to indicate a change of subject, if a more topical subject follows. This structure in many Papuan languages is typical of temporal and climate expressions and other impersonal predications (\citealt{Reesink1983a}, \citealt{Roberts1988b}), which are often used for giving backgrounded\footnote{\citet[244]{Farr1999} calls this \textit{on-line background} to distinguish it from the off-line background information of subordinate clauses.} information. In examples \REF{ex:8:x1482} and \REF{ex:8:x1475} the verb in the initial medial clause predicating the action of human participants is marked with same subject following even when the following clause mentions the coming of darkness or dawn. Returning to the main line action requires different-subject marking. In the examples, the ``skipped'' medial clauses are in brackets.

\ea%x1482
\label{ex:8:x1482}
\gll Aria  maa  en-ep  naap  ik-\textstyleEmphasizedVernacularWords{ok}  {\ob}kokom-ar-\textstyleEmphasizedVernacularWords{e}\textstyleEmphasizedVernacularWords{y}\textstyleEmphasizedVernacularWords{a}{\cb} in-e-mik.\\
alright  food  eat-\textsc{ss}.\textsc{seq} thus be-\textsc{ss} dark-\textsc{inch}-2/3s.\textsc{ds} sleep-\textsc{pa}-1/3p\\
\glt`Alright we ate and stayed like that and (then) it became dark and we slept.'
\z


\ea%x1475
\label{ex:8:x1475}
\gll In-\textstyleEmphasizedVernacularWords{ep}  {\ob}epa  wiim-\textstyleEmphasizedVernacularWords{eya}{\cb}  onak  maak-e-mik,  ``{\dots''}\\
sleep-\textsc{ss}.\textsc{seq} place dawn-2/3s.\textsc{ds} 3s/p.mother  tell-\textsc{pa}-1/3p\\
\glt`They slept, and when it dawned they told their mother, ``{\dots}'' '
\z


If the impersonal predicate is important for the main story line, rather than providing backgrounded information, the impersonal verb itself is placed as a final verb, and the verb in the preceding medial clause is marked for different subject \REF{ex:8:x1492}. \citet[206]{Reesink1987} notes a similar rule for Usan. 

\ea%x1492
\label{ex:8:x1492}
\gll Kir-ep  ekap-em-ika-\textstyleEmphasizedVernacularWords{iwkin}  epa  wiim-o-k. \\
turn-\textsc{ss}.\textsc{seq} come-\textsc{ss}.\textsc{sim}-be-2/3p.\textsc{ds} place dawn-\textsc{pa}-3s\\
\glt`They turned and as they were coming, it dawned.'
\z


In many Papuan languages, the impersonal predications include a number of experiential verbs (\citealt[204]{Reesink1987}, \citealt{Roberts1997}). In Mauwake, most of the experiential expressions are adjunct plus verb constructions (\sectref{sec:3.8.5.2.1}), where the experiencer is a subject rather than an object; in chained clauses these behave in a regular manner. But those few experiential expressions that are impersonal do not trigger \textsc{ds} marking in the preceding medial clause, because the inanimate subject in the experiential clause is not topical enough to do it. In \REF{ex:8:x1491}, the first person singular subject of the medial clauses becomes the object of the final clause, but the medial clause has same subject marking:

\ea%x1491
\label{ex:8:x1491}
\gll Uuw-ap uuw-\textstyleEmphasizedVernacularWords{ap} oona=ke efa  sirir-i-ya.\\
work-\textsc{ss}.\textsc{seq} work-\textsc{ss}.\textsc{seq} bone=\textsc{qf} 1s.\textsc{acc} hurt-\textsc{Nc}-\textsc{pr}.3s\\
\glt`I worked and worked and my bones hurt.'
\z


The verb \textstyleStyleVernacularWordsItalic{weeser}- `finish' is often used in chained clauses to indicate the finishing of an action. In this function, its low-topicality subject, the nominalized form of the preceding verb, is never mentioned overtly, and the preceding medial clause has \textsc{ss} marking \REF{ex:8:x1483}:

\ea%x1483
\label{ex:8:x1483}
\gll Uup-\textstyleEmphasizedVernacularWords{ep}  {\ob}weeser-\textstyleEmphasizedVernacularWords{eya}{\cb}  aria  oposia  gelemuta  wiam  erup fain  wia  wu-om-a-m.\\
cook-\textsc{ss}.\textsc{seq} finish-2/3s.\textsc{ds} alright  meat  small  3p.\textsc{refl} two this 3p.\textsc{acc} put-\textsc{ben}-\textsc{bnfy}2.\textsc{pa}-1s     \\
\glt`I cooked it and when it was finished, all right, I put (aside) a little of the meat for these two (women).'
\z


In \REF{ex:8:x1476}, there are two intervening clauses with different low-topicality inanimate subjects. The same-subject marking of the first clause ``jumps over'' these two clauses and refers to the subject in the last clause. The two clauses in between both have \textsc{ds} marking. 

\ea%x1476
\label{ex:8:x1476}
\gll Maa  uup-\textstyleEmphasizedVernacularWords{ep} {\ob}fofola  urup-\textstyleEmphasizedVernacularWords{eya}{\cb}  {\ob}maa  op-\textstyleEmphasizedVernacularWords{iya}{\cb} iiw-o-k.\\
food  cook-\textsc{ss}.\textsc{seq} foam rise-2/3s.\textsc{ds} food  be.done-2/3s.\textsc{ds} dish.out-\textsc{pa}-3s\hspace{-1mm}\\
\glt`She cooked the food and when it boiled and was done she dished it out.'
\z


Although human subjects are typically high on the topicality hierarchy \citep[364]{Givon1984}, even a human subject may occasionally be so low in topicality that it gets overlooked in the \textstyleAcronymallcaps{\textsc{sr}} marking \REF{ex:8:x1477}, \REF{ex:8:x1478}.\footnote{\citet[236--237]{Reesink1983a} gives similar examples from other Papuan languages.} What is particularly striking with the example \REF{ex:8:x1477} is that the clause that is overlooked has a subject in first person singular, which is usually considered to be topically the highest possible subject. A plausible explanation is that politeness and hospitality requires the host of a big meal to downplay his own importance in this way. 

\ea%x1477
\label{ex:8:x1477}
\gll Efa  arew-\textstyleEmphasizedVernacularWords{ap}  {\ob}maa  eka  liiwa  muuta  on-\textstyleEmphasizedVernacularWords{amkun}{\cb} en-ep-pu-ami  soomar-ek-eka. \\
1s.\textsc{acc}  wait-\textsc{ss}.\textsc{seq} food water little only make-1s/p.\textsc{ds} eat-\textsc{ss}.\textsc{seq}-\textsc{cmpl}-\textsc{ss}.\textsc{sim} walk-go-\textsc{imp}.2p\\
\glt`Wait for me, and when I have made just a little soup you eat it and then you (may) go.'
\z


\ea%x1478
\label{ex:8:x1478}
\gll Ikiw-\textstyleEmphasizedVernacularWords{ep}  {\ob}mua  nain  urema  osarena=pa  iimar-ep ik-\textstyleEmphasizedVernacularWords{eya}{\cb} ona  mua  nain  ifakim-o-k. \\
go-\textsc{ss}.\textsc{seq} man  that1 bandicoot path=\textsc{loc} stand-\textsc{ss}.\textsc{seq} be-2/3s.\textsc{ds} 3s.\textsc{gen} man  that1 kill-\textsc{pa}-3s\\
\glt`She went and as the man was standing on the bandicoot path she killed that husband of hers.'
\z

  
In process descriptions, the identity of people performing the actions is not important, and their topicality is low. In \REF{ex:8:x1481}, the person watching the fire in the coconut drying shed is not mentioned in any way. This example is also like \REF{ex:8:x1476} above, in that there are two clauses with a different low-topicality subject, here one of them [+human], intervening between the second \textsc{ss} clause and the final clause, where the original subject is picked up.

\ea%x1481
\label{ex:8:x1481}
\gll Epia  wu-ap  ikiw-\textstyleEmphasizedVernacularWords{ep} {\ob}iwera  kuuf-am-ik-\textstyleEmphasizedVernacularWords{eya}{\cb} {\ob}iwera  reen-\textstyleEmphasizedVernacularWords{eya}{\cb} iwer  urupa  anum-i-mik.  \\
fire  put-\textsc{ss}.\textsc{seq} go-\textsc{ss}.\textsc{seq} coconut  watch-\textsc{ss}.\textsc{sim}-be-2/3s.\textsc{ds} coconut  dry-2/3s.\textsc{ds} coconut shell knock-\textsc{Np}-\textsc{pr}.1/3p     \\
\glt`We/they put them (the coconuts) on the fire and go, and (someone) keeps watching the coconuts and they dry and (then) we/they knock the shells away.'
\z


Even an inanimate subject may override an animate/human one in \textsc{sr} marking, if its topicality is high enough. In \REF{ex:8:x1479} the subject/topic is \textstyleForeignWords{kunai} grass and the burning of the grass, which is such an important part of a pighunt that the hunt itself is called \textstyleStyleVernacularWordsItalic{fiker(a) kuumowa} `kunai-burning'.  The grass is a continuing topic from the previous several sentences, so a noun phrase is not used for marking it. 

\ea%x1479
\label{ex:8:x1479}
\gll Kuum-\textstyleEmphasizedVernacularWords{iwkin}  aw-\textstyleEmphasizedVernacularWords{emi}  {\ob}mua  unow  maneka  iiwawun  fikera kuum-emi  saawirin-ow-\textstyleEmphasizedVernacularWords{iwkin}{\cb}  aria  fiker  epia aw-i-non.\\
burn-2/3p.\textsc{ds}  burn-\textsc{ss}.\textsc{sim} man many very altogether kunai.grass burn-\textsc{ss}.\textsc{sim} round-\textsc{caus}-2/3p.\textsc{ds} alright  kunai.grass  fire burn-\textsc{Np}-\textsc{fu}.3s\\
\glt`They burn it and it burns and all the men burn and surround the kunai grass, (and) alright the kunai fire will burn.'
\z


\subsubsection{Apparent mismatches of reference}
%\hypertarget{RefHeading23221935131865}

A medial verb with \textsc{ds} marking is used in two instances where it does not indicate a change of subject. Both types have two or more clauses with identical \textsc{ds} marking even though the subject is the same; only the last of those clauses really indicates a change of subject. One of them is recursion of a \textsc{ds} verb \REF{ex:8:x1493}, indicating continuity; the identification of the subject is suspended until the repetition ends \citep[201]{Reesink1987}. 

\ea%x1493
\label{ex:8:x1493}
\gll Wiawi  kuum-\textstyleEmphasizedVernacularWords{eya}  kuum-\textstyleEmphasizedVernacularWords{eya}  kuum-\textstyleEmphasizedVernacularWords{eya}  aw-ep eka  iw-a-k  na  wia,  eka=ke  saanar-e-k. \\
3s/p.father burn-2/3s.\textsc{ds} burn-2/3s.\textsc{ds} burn-2/3s.\textsc{ds} burn-\textsc{ss}.\textsc{seq} river enter-\textsc{pa}-3s but no river=\textsc{cf} dry-\textsc{pa}-3s\\
\glt`It kept burning and burning their father and he burned and entered the river but no, the river dried.'
\z


A medial clause that has the same subject as the following medial clause may have \textsc{ds} marking if both the medial clauses relate to the same finite clause as their reference clause, and the first of the medial clauses gets expanded or defined more closely in the second one. The \textsc{ds} verbs may be identical \REF{ex:8:x1494}, but they do not need to be \REF{ex:8:x1495}, \REF{ex:8:x1496}.

\ea%x1494
\label{ex:8:x1494}
\gll Efa  uruf-am-ik-\textstyleEmphasizedVernacularWords{eya},  koora=pa  efa  uruf-am-ik-\textstyleEmphasizedVernacularWords{eya} ikiw-i-nen  ekap-i-nen.\\
1s.\textsc{acc}  see-\textsc{ss}.\textsc{sim}-be-2/3.\textsc{ds} house=\textsc{loc} 1s.\textsc{acc} see-\textsc{ss}.\textsc{sim}-be-2/3.\textsc{ds} go-\textsc{Np}-\textsc{fu}.1s come-\textsc{Np}-\textsc{fu}.1s    \\
\glt`You will keep seeing me, you will keep seeing me from the house, and I will come and go.'
\z


\ea%x1495
\label{ex:8:x1495}
\gll ...pon  sisina=pa  ik-\textstyleEmphasizedVernacularWords{eya},  piipa  unowa=pa soomar-em-ik-\textstyleEmphasizedVernacularWords{eya}  mik-a-m. \\
{\dots}turtle  shallow.water=\textsc{loc} be-2/3s.\textsc{ds} seaweed  many=\textsc{loc} walk-\textsc{ss}.\textsc{sim}.be-2/3s.\textsc{ds} spear-\textsc{pa}-1s\\
\glt`{\dots} the turtle was in the shallow water, it was walking among a lot of seaweed and I speared it.'
\z


\ea%x1496
\label{ex:8:x1496}
\gll No  ikoka  era=pa  wia  far-\textstyleEmphasizedVernacularWords{eya}, owora  wia maak-\textstyleEmphasizedVernacularWords{eya}, aria  mua=ke  naap  me  nefa  ma-i-nok,  ``...'' \\
2s.\textsc{unm} later road=\textsc{loc} 3p.\textsc{acc} call-2/3s.\textsc{ds} betelnut 3p.\textsc{acc} tell-2/3s.\textsc{ds} alright man=\textsc{cf} thus  not  2s.\textsc{acc} say-\textsc{Np}-\textsc{fu}.3s\\
\glt`Later, when you see them on the road, when you ask them for betelnut, alright let your husband not say about you that {\dots}'
\z


The \textsc{ss} medial form of the verb `be' is used in the expression \textstyleStyleVernacularWordsItalic{naap ikok}  `it is/was thus (and)', regardless of the following subject/topic \REF{ex:8:x1500}, \REF{ex:8:x1501}. The construction seems to have grammaticalized as an expression of an indefinite time span.

\ea%x1500
\label{ex:8:x1500}
\gll \textstyleEmphasizedVernacularWords{Naap}  \textstyleEmphasizedVernacularWords{ik-ok}  wi  Saramun=ke  wiisa  uf-e-mik. \\
thus  be-\textsc{ss} 3p.\textsc{unm} Saramun=\textsc{cf} dance.name  dance-\textsc{pa}-1/3p      \\
\glt`It was like that and (then) the Saramun people danced \textstyleForeignWords{wiisa}.'
\z


\ea%x1501
\label{ex:8:x1501}
\gll ...mua  me  wia  imen-a-mik.  \textstyleEmphasizedVernacularWords{Naap} \textstyleEmphasizedVernacularWords{ik-ok} sarere  uura buburia  ona  amia  mua  wiar  kerer-ep opaimika=pa  yia  wu-a-k.\\
{\dots}man  not  3p.\textsc{acc} find-\textsc{pa}-1/3p  thus  be-\textsc{ss} Saturday night bald 3s.\textsc{gen} bow  man 3.\textsc{dat} appear-\textsc{ss}.\textsc{seq} talk=\textsc{loc} 1p.\textsc{acc} put-\textsc{pa}-3s\\
\glt`{\dots} we didn't find the men. It was like that, and on Saturday evening the bald man himself went to the police and accused us.'
\z


Even when the final clause is verbless \REF{ex:8:x1497}, \REF{ex:8:x1498}, or missing completely because of ellipsis \REF{ex:8:x1499}, a medial clause is still possible. In both cases, the \textsc{sr} marking is based on what the expected subject would be, if there were one. 

\ea%x1497
\label{ex:8:x1497}
\gll Naap  ik-\textstyleEmphasizedVernacularWords{ok}  uruf-am-ika-\textstyleEmphasizedVernacularWords{iwkin}  wia. \\
thus  be-\textsc{ss} see-\textsc{ss}.\textsc{sim}-be-2/3p.\textsc{ds}  no\\
\glt`He was like that and they were watching him, but no (he didn't get any better).'
\z


\ea%x1498
\label{ex:8:x1498}
\gll Iinan  aasa  gurun-owa  miim-\textstyleEmphasizedVernacularWords{ap}  eka=iw  umuk-owa ewur. \\
sky  canoe  rumble-\textsc{nmz} hear-\textsc{ss}.\textsc{seq} water=\textsc{inst} extinguish-\textsc{nmz} quickly\\
\glt`We heard the rumble of the airplane(s) and quickly extinguished (the fires) with water (lit: and the extinguishing with water quickly).'
\z


The two sentences preceding the example sentence \REF{ex:8:x1499} mention American airplanes that flew over and dropped messages during the Second World War. The ``same subject'' needs to be picked from there -- as the story continues without another reference to the Americans for a while -- and the elliptical clause construed as something like \textstyleStyleVernacularWordsItalic{naap onamik} `and they did so'.

\ea%x1499
\label{ex:8:x1499}
\gll Wi  Yaapan  nan  ik-e-mik  nain  wia  uruf-\textstyleEmphasizedVernacularWords{ap}.  \\
3p.\textsc{unm} Japan  there  be-\textsc{pa}-1/3p  that1  3p.\textsc{acc} see-\textsc{ss}.\textsc{seq}\\
\glt`They had seen that the Japanese were there (and so they {\ob}the Americans{\cb} did so).'
\z


\subsubsection{Medial clauses as a complementation strategy for perception verbs} \label{sec:8.2.3.4}
%\hypertarget{RefHeading23241935131865}

Perception verbs in Mauwake mostly use a medial clause as a complementation strategy \citep[371]{Dixon2010a}, when the object of the perception verb is an \textstyleEmphasizedWords{{activity}} \REF{ex:8:x1509}--\REF{ex:8:x1511}.\footnote{\citet[237]{Reesink1983b} notes this for Usan too.} Regular, nominalized complement clauses are only used with perception verbs when a \textstyleEmphasizedWords{{fact}} is reported (\sectref{sec:8.3.2.2}).

\ea%x1509
\label{ex:8:x1509}
\gll Moma  wiar  \textstyleEmphasizedVernacularWords{en-em-ik-eya}  uruf-a-mik. \\
taro  3.\textsc{dat} eat-\textsc{ss}.\textsc{sim}-be-2/3s.\textsc{ds}  see-\textsc{pa}-1/3p      \\
\glt`It was eating their taro, and they saw it.' (Or: `They saw that it was eating their taro.')
\z


\ea%x1510
\label{ex:8:x1510}
\gll Aara  \textstyleEmphasizedVernacularWords{muuk-ar-ep} \textstyleEmphasizedVernacularWords{ik-eya} uruf-a-mik.\\
hen  son-\textsc{inch}-\textsc{ss}.\textsc{seq} be-2/3s.\textsc{ds} see-\textsc{pa}-1/3p\\
\glt`The hen had laid an egg and we saw it.' (Or: `We saw that the hen had laid an egg.')
\z


\ea%x1511
\label{ex:8:x1511}
\gll Yo  me  baliwep  paayar-e-m,  oram  iperowa=ke \textstyleEmphasizedVernacularWords{nanar}\textstyleEmphasizedVernacularWords{-}\textstyleEmphasizedVernacularWords{iwkin}  miim-a-m.\\
1s.\textsc{unm} not  well  understand-\textsc{pa}-1s  just  middle.aged=\textsc{cf} tell.story-2/3p.\textsc{ds} hear-\textsc{pa}-1s     \\
\glt`I do not understand it well, I have just heard the older people tell stories about it.'
\z


\subsubsection{Tail-head linkage} \label{sec:8.2.3.5}
%\hypertarget{RefHeading23261935131865}

Tail-head linkage is a typical feature especially in oral texts\footnote{With the development of written style, this feature is becoming less prominent.}  in Papuan languages. It is an inter-sentential cohesive device and could be understood to belong outside ``syntax proper'', if syntax is defined very narrowly. It is mentioned here as it is an important linking device, and the chaining structure is used for it. In narratives and in descriptions of processes, tail-head linkage is utilized to tie sentences together within a thematic paragraph. 

 The tail-head link is formed when one sentence ends in a finite clause (``tail''), and the next sentence begins with a medial clause (``head'') that copies the verb but changes it into a medial one. The information in this medial clause is given rather than new, unlike in most other medial clauses. \citet[200--201]{Foley1986} claims for Yimas, and assumes for the rest of Papuan languages, that these medial clauses are subordinate, but at least in Mauwake they are not -- they are coordinate like other medial clauses. In a narrative, the final verbs, which then get recapitulated in the next sentence, carry the core of the story line \REF{ex:8:x1505}.

\ea%x1505
\label{ex:8:x1505}
\gll Wafur-a-k  na  weetak,  \textstyleEmphasizedVernacularWords{ufer-a-k}.  \textstyleEmphasizedVernacularWords{Ufer-ap} nainiw  burir  aaw-ep  woosa=pa  aruf-eya  waaya nain  \textstyleEmphasizedVernacularWords{in-e-k}.  \textstyleEmphasizedVernacularWords{In-eya}  yena  ikiw-emi  nainiw wiowa  erup  ar-ow-amkun  iiwawun  \textstyleEmphasizedVernacularWords{um-o-k}.  \textstyleEmphasizedVernacularWords{Um-eya} merena  ere-erup  kaik-ap  {\dots}\\
throw-\textsc{pa}-3s  but  no  miss-\textsc{pa}-3s  miss-\textsc{ss}.\textsc{seq} again  axe  take-\textsc{ss}.\textsc{seq} head=\textsc{loc} hit-2/3s.\textsc{ds} pig that1 lie.down-\textsc{pa}-3s lie.down-2/3s.\textsc{ds} 1s.\textsc{gen} go-\textsc{ss}.\textsc{sim} again spear  two  become-\textsc{caus}-1s/p.\textsc{ds} altogether die-\textsc{pa}-3s die-2/3s.\textsc{dn} leg \textsc{rdp}-two  tie-\textsc{ss}.\textsc{seq}\\
\glt`He threw it (=a spear) but no, he missed. He missed it and again took an axe and hit it on the head and the pig fell down. It fell down and I myself went and speared it twice and it died completely. It died and I tied its legs two and two together and {\dots}'
\z


The repeated verb retains its arguments, but there is a choice in how overtly they and the peripherals are marked in the medial clause. Retaining them makes the medial clause more emphatic, and the first element becomes a theme for the new sentence (\sectref{sec:9.1}). In \REF{ex:8:x1505} only the verbs are copied; \REF{ex:8:x1506} copies the subject as well, \REF{ex:8:x1507} the object and \REF{ex:8:x1508} the locative adverbial.

\ea%x1506
\label{ex:8:x1506}
\gll \textstyleEmphasizedVernacularWords{Miiw-aasa}  \textstyleEmphasizedVernacularWords{samor-ar-e-k.} \textstyleEmphasizedVernacularWords{Miiw-aasa}  \textstyleEmphasizedVernacularWords{samor-ar-eya}{\dots} \\
land-canoe  bad-\textsc{inch}-\textsc{pa}-3s land-canoe bad-\textsc{inch}-2/3s.\textsc{ds}\\
\glt`The car broke. The car broke and {\dots}'
\z


\ea%x1507
\label{ex:8:x1507}
\gll Owowa  or-op,  wuailal-ep  \textstyleEmphasizedVernacularWords{akia} \textstyleEmphasizedVernacularWords{ik-e-k}. \textstyleEmphasizedVernacularWords{Akia} \textstyleEmphasizedVernacularWords{ik-ep}  en-em-ik-ok, {\dots} \\
village  descend-\textsc{ss}.\textsc{seq} be.hungry-\textsc{ss}.\textsc{seq} banana  roast-\textsc{pa}-3s banana roast-\textsc{ss}.\textsc{seq}  eat-\textsc{ss}.\textsc{sim}-be-\textsc{ss}\\
\glt`He came down to the village, was hungry and roasted bananas. He roasted bananas and was eating them, and {\dots}'
\z


\ea%x1508
\label{ex:8:x1508}
\gll P-ikiw-ep  \textstyleEmphasizedVernacularWords{Bogia=pa} \textstyleEmphasizedVernacularWords{nan} \textstyleEmphasizedVernacularWords{wu-a-mik}. \textstyleEmphasizedVernacularWords{Bogia=pa} \textstyleEmphasizedVernacularWords{nan}  \textstyleEmphasizedVernacularWords{wu-ap}  i  kiiriw  ekap-e-mik. \\
\textsc{bpx}-go-\textsc{ss}.\textsc{seq} Bogia=\textsc{loc} there put-\textsc{pa}-1/3p Bogia=\textsc{loc} there  put-\textsc{ss}.\textsc{seq} 1p.\textsc{\textsc{unm}} again come-\textsc{\textsc{pa}}-1/3p     \\
\glt`We took it (=his body) and put/buried it in Bogia. We put it in Bogia and came back again.'
\z


Most commonly the derivational morphology in the two verbs is identical, but sometimes the derivation in the finite verb is dropped from the medial verb \REF{ex:8:x1513}, \REF{ex:8:x1514}. In \REF{ex:8:x1513}, there is a good reason for dropping the benefactive marking from the repeated verb: the spear was thrown for someone's benefit, but it missed, and consequently there was no benefit for anyone.

\ea%x1513
\label{ex:8:x1513}
\gll Olas=ke  ekap-emi  wiowa  \textstyleEmphasizedVernacularWords{wafur-om-a-k}. \textstyleEmphasizedVernacularWords{Wafur-a-k}  na  weetak,  ufer-a-k. \\
Olas=\textsc{cf}  come-\textsc{ss}.\textsc{sim} spear throw-\textsc{ben}-\textsc{bnfy}2.\textsc{pa}-3s throw-\textsc{pa}-3s  but  no  miss-\textsc{pa}-3s     \\
\glt`Olas came and threw a spear for him. He threw it but no, he missed.'
\z


\ea%x1514
\label{ex:8:x1514}
\gll Epa  wiim-eya  mua  \textstyleEmphasizedVernacularWords{karer-omak-e-mik}. \textstyleEmphasizedVernacularWords{Karer-a-p}  ma-e-mik,  ``{\dots''}\\
place  dawn-2/3s.\textsc{ds} man  gather-\textsc{distr}/\textsc{pl}-\textsc{pa}-1/3p gather-\textsc{ss}.\textsc{seq} say-\textsc{pa}-1/3p\\
\glt`It dawned and many men gathered. They gathered and said, ``{\dots}'' '
\z


Adding new derivation to the medial verb is possible, but rare: the example (3.\ref{ex:3:x237}) is repeated below as \REF{ex:8:x1515}.
% * <liisa.berghall@gmail.com> 2015-05-22T17:28:20.829Z:
%
%  The example is (3.460) since it is in ch. 3
%

\ea%x1515
\label{ex:8:x1515}
\gll Ikiwosa  wiar  pepekim-ep  \textstyleEmphasizedVernacularWords{kaik-a-m}.  \textstyleEmphasizedVernacularWords{Kaik-om-ap}{\dots} \\
head  3.\textsc{dat} measure-\textsc{ss}.\textsc{seq} tie-\textsc{pa}-1s  tie-\textsc{ben}-\textsc{bnfy}2.\textsc{ss}.\textsc{seq}\\
\glt`I measured her head and tied it (=headdress). I tied it for her and {\dots}'
\z


 Similarly, aspect marking normally stays the same in both the verbs, but it is also possible to have aspect marking on the medial verb, although the finite verb is without any aspect marking \REF{ex:8:x1516},  \REF{ex:8:x1517}. When new information is added to the verb either by derivation or aspect marking, it is less clear if this still is a true case of tail-head linkage.  

\ea%x1516
\label{ex:8:x1516}
\gll ...nomokowa  maala  war-ep,  ekap-ep  ifa  nain \textstyleEmphasizedVernacularWords{ifakim-o-k}.  \textstyleEmphasizedVernacularWords{Ifakim-em-ik-eya}  ifa  nain=ke siowa  wasirk-a-k.\\
{\dots}tree  long  cut-\textsc{ss}.\textsc{seq} come-\textsc{ss}.\textsc{seq} snake  that1 beat-\textsc{pa}-3s beat-\textsc{ss}.\textsc{sim}-be-2/3s.\textsc{ds} snake  that1=\textsc{cf} dog  release-\textsc{pa}-3s\\
\glt`{\dots}he cut a long stick, came, and beat up the snake. As he was beating it, the snake released the dog.'
\z


\ea%x1517
\label{ex:8:x1517}
\gll Moma  manina  mokomokoka  \textstyleEmphasizedVernacularWords{nop}\textstyleEmphasizedVernacularWords{-}\textstyleEmphasizedVernacularWords{i}\textstyleEmphasizedVernacularWords{-}\textstyleEmphasizedVernacularWords{mik}. \textbf{Nop}\textbf{-}\textbf{ap-pu}\textbf{-}\textbf{ap}  nomokowa  war-i-mik. \\
taro  garden  first  clear-\textsc{Np}-\textsc{pr}.1/3p clear-\textsc{ss}.\textsc{seq}-\textsc{cmpl}-\textsc{ss}.\textsc{seq} tree cut-\textsc{Np}-\textsc{pr}.1/3p     \\
\glt`First we clear (the undergrowth for) taro garden. When we have cleared it we cut the trees.'
\z


The tail-head linkage disregards right-dislocated items that come between the two verbs \REF{ex:8:x1518}, \REF{ex:8:x1519}. 

\ea%x1518
\label{ex:8:x1518}
\gll Ne  kiiriw  nan  Medebur  \textstyleEmphasizedVernacularWords{ek-a-mik},  mua  napuma  onaiya. \textstyleEmphasizedVernacularWords{Ek-ap}  Medebur=pa  neeke  {\dots}\\
\textsc{add} again  there  Medebur go-\textsc{pa}-1/3p  man  sick  with go-\textsc{ss}.\textsc{seq} Medebur=\textsc{loc} there.\textsc{cf}\\
\glt`And again from there they went to Medebur, with the sick man. They went and there in Medebur {\dots}'
\z


\ea%x1519
\label{ex:8:x1519}
\gll ...pok-ap  ika-iwkin  mua  wiar  \textstyleEmphasizedVernacularWords{ekap-e-mik}, wiinar-ep.  \textstyleEmphasizedVernacularWords{Ekap-emi}  wia  maak-e-mik, ``Maa  iiw-eka.'' \\
sit.down-\textsc{ss}.\textsc{seq} be-\textsc{ss}.\textsc{seq} man  3.\textsc{dat} come-\textsc{pa}-1/3p make.planting.holes-\textsc{ss}.\textsc{seq} come-\textsc{ss}.\textsc{sim} 3p.\textsc{acc} tell-\textsc{pa}-1/3p food dish.out-\textsc{imp}.2p\\
\glt`{\dots} they were sitting and their husbands came, having made the planting holes. They came and told them, ``Dish out the food.'' '
\z


A summary tail-head linkage with a generic verb \REF{ex:8:x1520}, a common feature in many \textstyleAcronymallcaps{tng} languages, is used very little in Mauwake. 

\ea%x1520
\label{ex:8:x1520}
\gll Or-omi  \textstyleEmphasizedVernacularWords{ma-em-ik-e-mik},  ``Eka  mamaiya  akena i  yoowa  me  aaw-i-yen.''  \textstyleEmphasizedVernacularWords{Naap} \textstyleEmphasizedVernacularWords{on-am-ika-iwkin} eka owowa kerer-ep {\dots}\\
descend-\textsc{ss}.\textsc{sim} say-\textsc{ss}.\textsc{sim}-be-\textsc{pa}-1/3p river near very 1p.\textsc{unm} hot  not  get-\textsc{Np}-\textsc{fu}.1p thus do-\textsc{ss}.\textsc{sim}-be-2/3p.\textsc{ds} river  village  appear-\textsc{ss}.\textsc{seq}\\
\glt`They went down and were saying, ``Very near the river we won't get hot.'' They were doing like that and (then) the river reached the village and {\dots}'
\z


\section{Subordinate clauses: embedding and hypotaxis} \label{sec:8.3}
%\hypertarget{RefHeading23281935131865}

Subordinate clauses are a problematic area to define both cross-linguistically (\citealt{HaimanEtAl1984}, \citealt[317]{MathiessenEtAl1988}) and even within one language \citep[848]{Givon1990}. It seems that there is a continuum from fully independent to embedded clauses (\citealt[207]{Reesink1987}, \citealt[189]{Lehmann1988}). 

Rather than treating subordinate clauses as one group, it is helpful to differentiate between embedding and hypotaxis. Embedded clauses have a function in the main clause: relative clauses as qualifiers within a \textstyleAcronymallcaps{np}, complement clauses as objects or subjects, and adverbial clauses as adverbials. Hypotactic clauses are also dependent on the main clause, but they do not function as a constituent in it (\citealt[219]{Halliday1994}, \citealt{Lehmann1988}). Even though subordination is ``a negative term which lumps together all deviations from some `main clause' norm'' \citep[510]{HaimanEtAl1984}%Thompson
, the term still has limited usefulness, as there are some rules that affect both embedded and hypotactic clauses.  

In Mauwake, subordinate clauses usually precede the main clauses, and they have a non-final intonation pattern. The initial position is related to the pragmatic function of topic that these clauses often have \citep[187]{Lehmann1988}; but when the subordinate clause is right-dislocated, it does not have a topic function.\footnote{For a discussion on the topic function of subordinate clauses see, e.g., \citet{Reesink1983b, Reesink1987}, \citet{MathiessenEtAl1988}, \citet{Lehmann1988}, and \citet{ThompsonEtAl2007}.}  The semantic function varies according to the type of subordinate clause. 

Embedded clauses in Mauwake are nominalized clauses: relative clause nominalization (\textstyleAcronymallcaps{rc}) (\sectref{sec:8.3.1}) is always done with the demonstrative \textstyleStyleVernacularWordsItalic{nain}\textstyleStyleVernacularWordsItalic `that' added to a finite clause, whereas complement clauses (\textstyleAcronymallcaps{cc}) (\sectref{sec:8.3.2}) can use either one of the two nominalization strategies (\sectref{sec:5.7}). The locative and temporal adverbial clauses (\sectref{sec:8.3.3}), like the relative clauses, are Type 2 nominalized clauses (\sectref{sec:5.7.2}). All of these clauses bear out \citegen[236]{Reesink1983b} claim that ``subordinate clauses, especially in sentence-initial position, are natural vehicles for the speaker's presuppositions''.\footnote{``Presuppositions'' here refer to pragmatic, not logical-semantic presuppositions.}  \citet[230]{Reesink1983b} also suggests that the origin of the relative clause is in a paratactic construction. At least in Mauwake this seems to be true not only of the relative clause but of the complement clause (\sectref{sec:8.3.2}) as well. 

The hypotactic conditional and concessive clauses are dependent on their main clause, but not embedded in it. 

\subsection{Relative clauses}  \label{sec:8.3.1}
%\hypertarget{RefHeading23301935131865}

I define a restrictive relative clause (\textsc{rc}),\footnote{This definition only applies to restrictive relative clauses; non-restrictive \textsc{rc}s (\sectref{sec:8.3.1.4}) are not real \textsc{rc}s although they are structurally similar to the real \textsc{rc}s.} following \citet[206]{Andrews2007b}, as a ``subordinate clause which delimits the reference of a \textsc{np} by specifying the role of the referent of that \textsc{np} in the situation described by the \textsc{rc}''. 
  
\newpage
The relative clause is a statement about some noun phrase in the main clause. That \textstyleAcronymallcaps{\textsc{np}} is here called the antecedent \textstyleAcronymallcaps{\textsc{np}} (\textstyleAcronymallcaps{\textsc{antnp}}),\footnote{This is often called Head NP, but because it is not grammatically a ``head'' of anything, I prefer to call it antecedent \textsc{np}. The name ``antecedent'' is also somewhat of a misnomer, as in Mauwake it does not \textit{precede} the \textsc{relnp}.}  since it is the unit that the the coreferential \textstyleAcronymallcaps{\textsc{np}} in the relative clause derives its meaning from \citep[20]{Crystal1997}. The coreferential \textstyleAcronymallcaps{\textsc{np}} in the \textstyleAcronymallcaps{\textsc{rc}} is called the relative \textstyleAcronymallcaps{\textsc{np}} (\textstyleAcronymallcaps{\textsc{relnp}}).\footnote{\citet[142]{Keenan1985} calls it a domain noun.}

Often the referent of the \textstyleAcronymallcaps{\textsc{antnp}} is assumed to be known to the hearer but not necessarily easily accessible, so the \textstyleAcronymallcaps{\textsc{rc}} gives background information to help the hearer identify the referent. 

The relative marker is the distal-1 demonstrative \textstyleStyleVernacularWordsItalic{nain} `that' (\sectref{sec:3.6.2}) occurring clause-finally in the relative clause \REF{ex:8:x1527}. It has a slightly rising non-final intonation indicating that the sentence continues; right-dislocated \textstyleAcronymallcaps{\textsc{rc}}s have sentence-final falling intonation. Givenness is an essential part of the meaning of the demonstrative, which is also used in \textstyleAcronymallcaps{\textsc{np}}s \REF{ex:8:x1528}. The demonstrative in effect makes the \textstyleAcronymallcaps{\textsc{rc}} into a noun phrase. The similarity of the two structures can be seen in the examples below.

\ea%x1527
\label{ex:8:x1527}
\gll {\ob}Takira  gelemuta  nain{\cb}\textsubscript{NP}  uruf-a-m.\\
boy  small  that1  see-\textsc{pa}-1s      \\
\glt`I saw that/the small boy.'
\z


\ea%x1528
\label{ex:8:x1528}
\gll {\ob}Takira  me  arim-o-k  nain{\cb}\textsubscript{\textsc{rc}}  uruf-a-m. \\
boy  not  grow-\textsc{pa}-3s that1 see-\textsc{pa}-1s \\
\glt`I saw the boy that has not grown.'
\z


\subsubsection{The type and position of the relative clause}
%\hypertarget{RefHeading23321935131865}

In typological terms, relative clauses in Mauwake are mostly replacive, also called internal  \REF{ex:8:x1529}--\REF{ex:8:x1546}. A normal finite clause is made into a noun phrase by the addition of the demonstrative \textstyleStyleVernacularWordsItalic{nain}, and the \textstyleAcronymallcaps{\textsc{relnp}} inside the \textstyleAcronymallcaps{\textsc{rc}} replaces the \textstyleAcronymallcaps{\textsc{antnp}}. Pre-nominal \textstyleAcronymallcaps{\textsc{rc}s,} where the\textstyleAcronymallcaps{ \textsc{rc}} precedes the \textstyleAcronymallcaps{\textsc{antnp},} are cross-linguistically more typical of \textstyleAcronymallcaps{\textsc{ov}} languages than replacive ones \citep[144]{Keenan1985}, but the latter are also common in Papuan languages (\citealt[229]{Reesink1983b} and \citealt[219]{Reesink1987}, \citealt[49]{Roberts1987}, \citealt[281]{Farr1999}, \citealt[193]{Whitehead2004}). Often both pre-nominal and replacive \textstyleAcronymallcaps{\textsc{rc}}s are possible, with one or the other being the dominant type.

\ea%x1529
\label{ex:8:x1529}
\gll {\ob}Ni  \textstyleEmphasizedVernacularWords{nomona}  unuf-a-man  nain{\cb},  aria  iimeka  kuisow  na-e-man. \\
2p.\textsc{unm} stone call-\textsc{pa}-2p that1 alright ten one say-\textsc{pa}-2p      \\
\glt`The money that you mentioned, alright you said ten (kina).'
\z


\ea%x1530
\label{ex:8:x1530}
\gll Ne  {\ob}eka  opora  \textstyleEmphasizedVernacularWords{biiris}  marew  nain{\cb}  wiena on-am-ik-e-mik. \\
\textsc{add} river  mouth  bridge  no(ne)  that1  3p.\textsc{gen} do-\textsc{ss}.\textsc{sim}-be-1/3p\\
\glt`And they themselves kept making bridges to river channels that didn't have them.' 
\z


\ea%x1545
\label{ex:8:x1545}
\gll {\ob}\textstyleEmphasizedVernacularWords{Mua}  kuum-e-mik  nain{\cb}  me  wia  kuuf-a-mik. \\
man  burn-\textsc{pa}-1/3p  that1  not  3p.\textsc{acc} see-\textsc{pa}-1/3p\\
\glt`We/They did not see the men that burned it.'
\z


\ea%x1546
\label{ex:8:x1546}
\gll Ne  {\ob}\textstyleEmphasizedVernacularWords{akia}  ik-e-k  nain{\cb}  me  en-e-k. \\
\textsc{add} banana  roast-\textsc{pa}-3s  that1  not  eat-\textsc{pa}-3s\\
\glt`And/but he did not eat the banana(s) that he roasted.'
\z


It is possible to retain the antecedent  \textstyleAcronymallcaps{\textsc{np}}, in which case the relative clause is not replacive but pre-nominal. In Mauwake this is not common; it is used when the noun phrase that is relativized is given extra emphasis \REF{ex:8:x1532}. 

\ea%x1532
\label{ex:8:x1532}
\gll {\ob}\textstyleEmphasizedVernacularWords{Fofa}  ikiw-e-mik  nain{\cb},  \textstyleEmphasizedVernacularWords{fofa}  nain  yo  me  paayar-e-m. \\
day  go-\textsc{pa}-1/3p  that1  day  that1  1s.\textsc{unm} not  know-\textsc{pa}-1s\\
\glt`The day that they went, I do not know the day/date.'
\z


Even though the \textstyleAcronymallcaps{\textsc{rc}} is usually embedded in the main clause, it can be right-dislocated. In that case the main clause contains the antecedent \textstyleAcronymallcaps{\textsc{np}}, and the relative \textstyleAcronymallcaps{\textsc{np}} is deleted from the \textstyleAcronymallcaps{\textsc{rc}}. This way the first one of the coreferential \textstyleAcronymallcaps{\textsc{np}}s is retained for easier processing. Reasons for right-dislocating a relative clause are firstly, a long \textstyleAcronymallcaps{\textsc{rc}}, which would be hard to process sentence-medially \REF{ex:8:x1533}, secondly, focusing on the \textstyleAcronymallcaps{\textsc{rc}}, or thirdly, an afterthought: something that the speaker still wants to add \REF{ex:8:x1534}.

\ea%x1533
\label{ex:8:x1533}
\gll \textstyleEmphasizedVernacularWords{Wi}  \textstyleEmphasizedVernacularWords{teeria}  \textstyleEmphasizedVernacularWords{papako}  o  asip-a-mik,  {\ob}ona  eka sesenar-ep  wienak-e-k  nain{\cb}. \\
3p.\textsc{unm} group  other  3s.\textsc{unm} help-\textsc{pa}-1/3p  3s.\textsc{gen} water buy-\textsc{ss}.\textsc{seq} feed.them-\textsc{pa}-3s that1\\
\glt`Another group helped him, (those) that he had bought and given beer to.'
\z


\ea%x1534
\label{ex:8:x1534}
\gll \textstyleEmphasizedVernacularWords{I} \textstyleEmphasizedVernacularWords{mua}  yiam  ikur,  {\ob}fikera  ikiw-e-mik  nain{\cb}. \\
1p.\textsc{unm} man 1p.\textsc{refl} five kunai.grass go-\textsc{pa}-1/3p that1\\
\glt`There were five of us men that went to the kunai grass (=pig-hunting).'
\z


In a very rare case the antecedent \textstyleAcronymallcaps{\textsc{np}} is deleted and the relative \textstyleAcronymallcaps{\textsc{np}} is retained in the right-dislocated \textstyleAcronymallcaps{\textsc{rc}}. What makes it possible in example \REF{ex:8:x1535} may be that the verb in the main clause can only have some food (or medicine/poison) as its object, so the object, although usually present, may also be left out.

\ea%x1535
\label{ex:8:x1535}
\gll Wi  mua  ...  ekap-iwkin  wienak-e-mik, {\ob}\textstyleEmphasizedVernacularWords{maa}  nop-a-mik  nain{\cb}.\\
3p.\textsc{unm} man  ...  come-2/3p.\textsc{ds} feed.them-\textsc{pa}-1/3p food  search-\textsc{pa}-1/3p  that1\\
\glt`The men {\dots} came, and we gave it to them to eat, (that is,) the food that we had searched for.'
\z


\citet[144--146]{Comrie1981} presents another typology based on how the role of the relative  \textstyleAcronymallcaps{\textsc{np}} is presented in the \textstyleAcronymallcaps{\textsc{rc}}. Basically Mauwake is of the ``gap type'', which ``does not provide any overt indication of the role of the head within the relative clause''. Noun phrases get very little case marking for their clausal role, and this is reflected in the \textstyleAcronymallcaps{\textsc{rc}} too. This results in ambiguous relative clauses when both a third person subject \textstyleAcronymallcaps{\textsc{np}} and a third person object \textstyleAcronymallcaps{\textsc{np}} are present in the \textstyleAcronymallcaps{\textsc{rc}} and the context does not make the meaning clear enough \REF{ex:8:x1548}:

\ea%x1548
\label{ex:8:x1548}
\gll {\ob}Siowa  kasi  keraw-a-k  nain{\cb}  um-o-k. \\
dog  cat  bite-\textsc{pa}-3s that1 die-\textsc{pa}-3s\\
\glt`The dog that bit the cat died.' Or: `The dog that the cat bit died.'
\z


The ambiguity can be avoided by adding the contrastive focus marker to the subject when the object is fronted to the theme position.\footnote{For some reason this is done in relative clauses mainly with human subjects, although the contrastive focus marker can be added to non-human subjects as well.}  Although this is not case marking, it can function as such, because the subject is the best candidate for contrastive focus marking \REF{ex:8:x1549} (\sectref{sec:3.12.7.2}).

\ea%x1549
\label{ex:8:x1549}
\gll {\ob}Mua  ona  emeria=ke  aruf-a-k  nain{\cb}  uruf-a-m. \\
man  3s.\textsc{gen} woman=\textsc{cf} hit-\textsc{pa}-3s that1 see-\textsc{pa}-1s\\
\glt`I saw the man whose wife hit him.'
\z


\citegen[140]{Comrie1981} ``non-reduction type'' is exhibited in Mauwake by those few cases where the relative \textstyleAcronymallcaps{\textsc{np}} keeps its oblique case marking. With overt case marking on the \textstyleAcronymallcaps{\textsc{relnp}}, the \textstyleAcronymallcaps{\textsc{antnp}} has to be retained \REF{ex:8:x1544}:

\ea%x1544
\label{ex:8:x1544}
\gll [\textbf{Burir=iw}  nomokowa  war-e-m  nain,]  burir  nain  duduw-ar-e-k.\\
axe=\textsc{inst} tree cut-\textsc{pa}-1s that1 axe that blunt-\textsc{inch}-\textsc{pa}-3s\\
\glt`The axe with which I cut trees became blunt.'
\z


But when the case marking does not appear in the \textstyleAcronymallcaps{\textsc{rc},} the \textstyleAcronymallcaps{\textsc{antnp}} is not present in the main clause either, and the \textstyleAcronymallcaps{\textsc{rc}} is a gapping-type relative clause \REF{ex:8:x1541}:

\ea%x1541
\label{ex:8:x1541}
\gll {\ob}\textbf{Burir}  nomokowa  war-e-m  nain=ke{\cb}  duduw-ar-e-k. \\
axe  tree  cut-\textsc{pa}-1s that1=\textsc{cf} blunt-\textsc{inch}-\textsc{pa}-3s\\
\glt`The axe with which I cut trees became blunt.'
\z


\subsubsection{The structure of the relative clause} \label{sec:8.3.1.2}
%\hypertarget{RefHeading23341935131865}

In Mauwake, the most typical relative clause is syntactically like a finite main clause, plus the distal-1 deictic \textstyleStyleVernacularWordsItalic{nain} functioning as a clause final relative marker. It was mentioned in \sectref{sec:5.7.2} that this is one strategy for nominalizing clauses in Mauwake. The demonstrative as a possible origin of a relative marker is well attested cross-linguistically (e.g. \citealt[342]{Dixon2010b}). 

The verb of the relative clause is a fully inflected finite verb. But when a non-verbal clause is a relative clause, it has no verb and is structurally like other non-verbal clauses \REF{ex:8:x1943}. 

\ea%x1943
\label{ex:8:x1943}
\gll Ne  {\ob}eka  opora  biiris  marew  nain{\cb}  wiena on-am-ik-e-mik.\\
\textsc{add} river  mouth  bridge  no(ne)  that1  3p.\textsc{gen} do-\textsc{ss}.\textsc{sim}-\textsc{pa}-1/3p\\
\glt`And they themselves kept making bridges to rivers that didn't have them.'
\z


The relative \textstyleAcronymallcaps{\textsc{np}} tends to be initial in the \textstyleAcronymallcaps{\textsc{rc}} regardless of its syntactic function. This is because it often has the pragmatic function of theme, which takes the clause-initial position. The initial position is easy to have also because a typical clause in Mauwake has so few noun phrases: in many \textstyleAcronymallcaps{\textsc{rc}}s the \textstyleAcronymallcaps{\textsc{relnp}} is the only noun phrase \REF{ex:8:x1552}. 

\ea%x1552
\label{ex:8:x1552}
\gll {\ob}Moma  p-or-o-mik  nain{\cb}  wiar  sesenar-e-mik.\\
taro  \textsc{bpx}-descend-\textsc{pa}-1/3p  that1  3.\textsc{dat} buy-\textsc{pa}-1/3p \\
\glt`They\textsubscript{i} bought from them\textsubscript{j} the taro that they\textsubscript{j} brought down.'
\z


When a personal pronoun functions as a subject and the relative \textstyleAcronymallcaps{\textsc{np}} in some other syntactic role, the pronoun tends to keep its initial position, thus maintaining the basic constituent order. The personal pronouns are high on the topicality hierarchy \citep[166]{Givon1976}, so it is natural that they tend to keep the clause-initial and also sentence-initial position.  Since the object \textstyleStyleVernacularWordsItalic{sirirowa} `pain' in \REF{ex:8:x1531} is not fronted, a temporal adverbial also keeps a place it would have in a neutral main clause. 

The tense in the \textstyleAcronymallcaps{\textsc{rc}} can be past \REF{ex:8:x1552} or present \REF{ex:8:x1559}, but not future. For future meaning, the present tense form has to be used \REF{ex:8:x1531}. 

\ea%x1531
\label{ex:8:x1531}
\gll {\ob}Yo  ikoka  sirir-owa  aaw-i-yem  nain{\cb},  nis  pun  eliw aaw-owen=i?\\
1s.\textsc{unm} later  hurt-\textsc{nmz} get-\textsc{Np}-\textsc{pr}.1s  that1  2p.\textsc{fc} also  well get-\textsc{fu}.2p=\textsc{qm}\\
\glt`Is it all right that you will also get the pain that I (will) later get?'
\z


As was mentioned above, the antecedent \textstyleAcronymallcaps{\textsc{np}} only rarely occurs overtly. But a relative \textstyleAcronymallcaps{\textsc{np}} can also be deleted if it is generic \REF{ex:8:x1561}, or recoverable from situational \REF{ex:8:x1555} or textual context \REF{ex:8:x1559}. In  \REF{ex:8:x1555}, the deleted \textstyleAcronymallcaps{\textsc{relnp}} can either be generic `what/whatever' or it may be \textstyleStyleVernacularWordsItalic{opora} `talk'; in \REF{ex:8:x1559}, the speaker is describing the process of making a fishtrap, which has already been mentioned in previous sentences.

\ea%x1561
\label{ex:8:x1561}
\gll {\ob}Iinan  aasa=pa  or-omi  kiikir  furew-a-mik  nain{\cb}  dabela.  \\
sky  canoe=\textsc{loc} descend-\textsc{ss}.\textsc{sim} first  sense-\textsc{pa}-1/3p that1 cold\\
\glt`What we first sensed/felt when we descended from the plane was the cold.'
\z


\ea%x1555
\label{ex:8:x1555}
\gll {\ob}Kululu  ma-e-k  nain{\cb}  kirip-i-yem. \\
Kululu  say-\textsc{pa}-3s that1 turn/reply-\textsc{Np}-\textsc{pr}.1s\\
\glt`I reply to what Kululu said.'
\z


\ea%x1559
\label{ex:8:x1559}
\gll Aria  {\ob}malol=pa  ifemak-i-mik  nain{\cb}  aana puuk-i-mik.\\
alright  deep.sea=\textsc{loc} press-\textsc{Np}-\textsc{pr}.1/3p  that1 cane cut-\textsc{Np}-\textsc{pr}.1/3p\\
\glt`Alright for those that we lower to the deep sea we cut cane.'
\z


In \REF{ex:8:x1556}, there is no other indication of the relative \textstyleAcronymallcaps{\textsc{np}} than the person suffix of the verb. The group of women referred to were mentioned as a noun phrase only near the beginning of the story, whereas the example is from close to the end:

\ea%x1556
\label{ex:8:x1556}
\gll Domora=pa  or-omi  nan  ik-e-mik,  {\ob}afa ar-e-mik  nain{\cb}. \\
Domora=\textsc{loc} descend-\textsc{ss}.\textsc{sim} there be-pa-1/3p  flying.fox become-\textsc{pa}-1/3p  that1\\
\glt`They went down from Domora and were there, those (women) who became flying foxes.'
\z


In the following two examples the \textstyleAcronymallcaps{\textsc{rc}s} are identical, but they have a different relative  \textstyleAcronymallcaps{\textsc{np}}. The \textstyleAcronymallcaps{\textsc{relnp}} of \REF{ex:8:x1557} is \textstyleStyleVernacularWordsItalic{mukuruna} `noise', but the \textstyleAcronymallcaps{\textsc{relnp}} of \REF{ex:8:x1558}, \textstyleStyleVernacularWordsItalic{wi} `they', only shows in the verbal suffix. The obligatory accusative pronoun in the main clause provides a key for the interpretation of \REF{ex:8:x1558}.

\ea%x1557
\label{ex:8:x1557}
\gll {\ob}Mukuruna  wua-i-mik  nain{\cb}  ikiw-ep  miim-eka. \\
noise  put-\textsc{Np}-\textsc{pr}.1/3p that1 go-\textsc{ss}.\textsc{sim} hear-\textsc{imp}.2p\\
\glt`Go and listen to the noise that they are making.'
\z


\ea%x1558
\label{ex:8:x1558}
\gll {\ob}Mukuruna  wua-i-mik  nain{\cb}  ikiw-ep  wia  miim-eka.\\
noise  put-\textsc{Np}-\textsc{pr}.1/3p that1 go-\textsc{ss}.\textsc{seq} 3p.\textsc{acc} hear-\textsc{imp}.2p\\
\glt`Go and listen to those who are making the noise.'
\z


The antecedent in most relative clauses has a specific reference. In Mauwake, when the reference is generic, a very generic noun is chosen as the head of the relativized \textstyleAcronymallcaps{\textsc{np}} and is modified by a question word \REF{ex:8:x1562}, \REF{ex:8:x1563}. So-called free \citep[213]{Andrews2007b} or condensed \citep[359]{Dixon2010b} relative clauses, which usually replace the whole \textstyleAcronymallcaps{\textsc{np}} with a generic or interrogative pronoun, are not used in Mauwake. 

\ea%x1562
\label{ex:8:x1562}
\gll {\ob}Maa  mauwa  maak-i-n  nain{\cb}  me  nefa  miim-i-non.\\
thing  what  tell-\textsc{Np}-\textsc{pr}.2s that1 not 2s.\textsc{acc} hear-\textsc{Np}-\textsc{fu}.3s\\
\glt`Whatever you tell him, he will not hear.'
\z


\ea%x1563
\label{ex:8:x1563}
\gll {\ob}Mua  naareke  kema  enek-ar-i-ya  nain{\cb}  eka  dabela enim-i-nok.\\
man  who.\textsc{cf} liver  tooth-\textsc{inch}-\textsc{Np}-\textsc{pr}.3s  that1  water  cold eat-\textsc{Np}-\textsc{imp}.3s\\
\glt`Whoever is thirsty must drink (cold) water.'
\z


When the antecedent is generic and human, there are two more possibilities for the \textsc{relnp}: it may be \textstyleStyleVernacularWordsItalic{mua} `man, person' \REF{ex:8:x1564} or the third person singular pronoun, plus the specifier \textstyleStyleVernacularWordsItalic{ena} \REF{ex:8:x1565} (\sectref{sec:3.12.7.1}). 

\ea%x1564
\label{ex:8:x1564}
\gll {\ob}Mua  ena  kema  enek-ar-i-ya  nain{\cb}  ... \\
man  \textsc{spec} liver  tooth-\textsc{inch}-\textsc{Np}-\textsc{pr}.3s  that1\\
\glt`Whoever is thirsty{\dots}'
\z


\ea%x1565
\label{ex:8:x1565}
\gll {\ob}O  ena  kema  enek-ar-i-ya  nain{\cb}  ... \\
3s.\textsc{unm} \textsc{spec} liver tooth-\textsc{inch}-\textsc{Np}-\textsc{pr}.3s  that1\\
\glt`Whoever is thirsty{\dots}'
\z


Non-verbal descriptive clauses can be made into relative clauses, but it is only in the negative that they are recognizable as such. In the affirmative, they are exactly like noun phrases with a demonstrative \REF{ex:8:x1550}, and because the meanings are so similar, it can be questioned whether there is such a thing as an affirmative non-verbal descriptive \textstyleAcronymallcaps{\textsc{rc}} at all in Mauwake.  

\ea%x1550
\label{ex:8:x1550}
\gll {\ob}Mua  eliwa  nain{\cb}  kookal-i-yem.\\
man  good  that1  like-\textsc{Np}-\textsc{pr}.1s\\
\glt`I like the good man.' Or `I like the man that is good.'
\z


In the negative, these clauses are different from the noun phrases because the negation is placed before the non-verbal predicate \REF{ex:8:x1551}.

\ea%x1551
\label{ex:8:x1551}
\gll {\ob}Koora  \textstyleEmphasizedVernacularWords{me}  maneka  nain{\cb}  uruf-a-m. \\
house  not  big  that1  see-\textsc{pa}-1s\\
\glt`I saw the house that is not big.'
\z


\subsubsection{Relativizable noun phrase positions}
%\hypertarget{RefHeading23361935131865}

Several \textstyleAcronymallcaps{\textsc{np}} functions can be relativized, and Mauwake conforms to \citegen{KeenanEtAl1977} Noun Phrase Accessibility Hierarchy:\footnote{As Mauwake adjectives do not have comparative forms there can be no relativization for an object of comparison, which in Keenan and Comrie's hierarchy is the hardest to relativize.}  the higher up an \textstyleAcronymallcaps{\textsc{np}} is in the hierarchy, the easier it is to relativize. Noun phrases with the following functions can be relativized: subject, object, recipient, beneficiary, instrument, comitative, object of genitive, temporal and locative. 

Subject \REF{ex:8:x1537} and object \REF{ex:8:x1538} are the most frequent functions of the \textstyleAcronymallcaps{\textsc{RelNP}}.

\ea%x1537
\label{ex:8:x1537}
\gll {\ob}\textstyleEmphasizedVernacularWords{Mesa}  \textstyleEmphasizedVernacularWords{asia}  fiker(a)  gone=pa  ika-i-ya  nain{\cb} aaw-em-ik-e-m.\\
wingbean  wild  kunai.grass  middle=\textsc{loc} be-\textsc{Np}-\textsc{pr}.3s that1 get-\textsc{ss}.\textsc{sim}-be-\textsc{pa}-1s\\
\glt`I kept picking wild wingbeans that are/grow in the middle of the kunai grass.'
\z


\ea%x1538
\label{ex:8:x1538}
\gll Muuka,  {\ob}yo  \textstyleEmphasizedVernacularWords{opora}  nefa  maak-i-yem  nain{\cb}  miim-ap ook-e.\\
son  1s.\textsc{unm} talk 2s.\textsc{acc} tell-\textsc{Np}-\textsc{pr}.1s that1 hear-\textsc{ss}.\textsc{seq} follow-\textsc{imp}.2s\\
\glt`Son, listen to and follow the talk that I am telling you.'
\z


Recipient \REF{ex:8:x1539} and beneficiary \REF{ex:8:x1547} are possible to relativize, but in natural texts beneficiary is very infrequent.

\ea%x1539
\label{ex:8:x1539}
\gll {\ob}\textstyleEmphasizedVernacularWords{Takira}  iwoka  iw-e-m  nain{\cb}  yena  aamun=ke.\\
boy  yam  give.him-\textsc{pa}-1s that1 1s.\textsc{gen} 1s/p.younger.sibling=\textsc{cf}\\
\glt`The boy that I gave yam to is my younger brother.'
\z


\ea%x1547
\label{ex:8:x1547}
\gll Ne  {\ob}\textstyleEmphasizedVernacularWords{wi}  \textstyleEmphasizedVernacularWords{emeria}  \textstyleEmphasizedVernacularWords{papako}  iiriw  sawur  wia iirar-om-a-k  nain{\cb}  {\dots}\\
\textsc{add} 3p.\textsc{unm} woman  some  earlier  bad.spirit 3p.\textsc{acc} remove-\textsc{ben}-\textsc{bnfy}2.\textsc{pa}-3s  that1\\
\glt`And some women, from (lit: for) whom he had removed bad spirits, {\dots}'
\z


When an instrument is relativized, the \textstyleAcronymallcaps{\textsc{relnp}} either takes the instrumental case marking \REF{ex:8:x1544} or has no case marking \REF{ex:8:x1553}: 

\ea%x1553
\label{ex:8:x1553}
\gll Aria  {\ob}\textstyleEmphasizedVernacularWords{maa} \textstyleEmphasizedVernacularWords{unowa}  wakesim-e-mik  nain{\cb}  sererk-a-mik.\\
alright  thing  many  cover-\textsc{pa}-1/3p  that1  distribute-\textsc{pa}-1/3p\\
\glt`Alright they distributed the many things with which they had covered her (body).'
\z


A comitative \textstyleAcronymallcaps{np} (\sectref{sec:4.1.3}) containing a comitative postposition may be relativized \REF{ex:8:x1542}, but one formed with a comitative clitic may not. 

\ea%x1542
\label{ex:8:x1542}
\gll {\ob}\textbf{Mua}  \textbf{nain}  \textbf{ikos}  ikiw-e-mik  nain{\cb}  napum-ar-e-k. \\
man  that1  with  go-\textsc{pa}-1/3p  that1 sick-\textsc{inch}-\textsc{pa}-3s\\
\glt`That man with whom I went became sick.'
\z


The object of genitive, or object of \textstyleEmphasizedWords{{possessive}} as it should be called when describing Mauwake grammar, only uses the dative pronoun (\sectref{sec:3.5.5}) when relativized \REF{ex:8:x1543}, not the unmarked (\sectref{sec:3.5.2.1}) or genitive (\sectref{sec:3.5.4}) pronoun.

\ea%x1543
\label{ex:8:x1543}
\gll {\ob}\textbf{Mua}  emeria  \textbf{wiar}  um-o-k  nain=ke{\cb}  baurar-ep owowa  oko  ikiw-o-k.\\
man  woman  3.\textsc{dat}  die-\textsc{pa}-3s that1=\textsc{cf} flee-\textsc{ss}.\textsc{seq} village  other go-\textsc{pa}-3s\\
\glt`The man whose wife died went away\footnote{Moving to another village after some misfortune is quite common, and the verb `flee' is used in this context but here it does not have a strongly negative connotation; this is reflected in the free translation.} to another village.'
\z


Temporal \REF{ex:8:x1554} and locative \REF{ex:8:x1560} \textstyleAcronymallcaps{\textsc{rc}}s are structurally identical to the other \textstyleAcronymallcaps{\textsc{rc}}s when the relativized temporal or locative \textstyleAcronymallcaps{\textsc{np}} does not have an adverbial function in the main clause. 

\ea%x1554
\label{ex:8:x1554}
\gll {\ob}Fofa  ikiw-e-mik  nain{\cb}  me  paayar-e-m. \\
day  go-\textsc{pa}-1/3p  that1  not  understand-\textsc{pa}-1s\\
\glt`I don't know the day that they went.'
\z


\ea%x1560
\label{ex:8:x1560}
\gll {\ob}Koora  maneka  wiena  opora  siisim-i-mik  nain{\cb}  uruf-a-mik. \\
house  big  3p.\textsc{gen} talk  write-\textsc{Np}-1/3p that1  see-\textsc{pa}-1/3p\\
\glt`We saw the big house where they write their talk (=printshop).'
\z


When the relativized temporal \textstyleAcronymallcaps{\textsc{np}} is a temporal in the main clause as well, the relative marker can optionally be replaced by the locative deictic \textstyleStyleVernacularWordsItalic{nan} or \textstyleStyleVernacularWordsItalic{neeke} `there' \REF{ex:8:x1625}.

\ea%x1625
\label{ex:8:x1625}
\gll [Aite  uroma  yaki-e-k  fofa  nain/nan/neeke] auwa  Madang  ikiw-o-k.\\
1s/p.mother  stomach  wash-\textsc{pa}-3s day that1/there/there.\textsc{cf} 1s/p.father Madang go-\textsc{pa}-3s\\
\glt`The day that mother gave birth, father went to Madang.'
\z


When the relativized locative \textstyleAcronymallcaps{\textsc{np}} is also a constituent in the main clause, the relative marker has to be replaced by \textstyleStyleVernacularWordsItalic{nan} or \textstyleStyleVernacularWordsItalic{neeke} \REF{ex:8:x1622}.

\ea%x1622
\label{ex:8:x1622}
\gll Or-op  {\ob}i  koora  ik-e-mik  neeke{\cb}  ekap-o-k.\\
descend-\textsc{ss}.\textsc{seq} 1p.\textsc{unm} house be-\textsc{pa}-1p there.\textsc{cf} come-\textsc{pa}-3s\\
\glt`It descended and came to the house/building where we were.'
\z


Temporal adverbial clauses, which are structurally close to relative clauses, are discussed below in \sectref{sec:8.3.3.1}, and locative adverbial clauses in \sectref{sec:8.3.3.2}.

\subsubsection{Non-restrictive relative clauses} \label{sec:8.3.1.4}
%\hypertarget{RefHeading23381935131865}

Non-restrictive, or appositional, relative clauses are structurally exactly like restrictive relative clauses, but their function is different. They do not delimit the reference of the antecedent \textstyleAcronymallcaps{\textsc{np}}. Instead, they give new information about it. Functionally they are like a coordinate clause added to the main clause.

Because of the structural and even intonational similarity, it is sometimes difficult to tell if a particular \textstyleAcronymallcaps{\textsc{rc}} is restrictive or non-restrictive. When the \textstyleAcronymallcaps{\textsc{antnp}} is a proper noun or when it includes a first or second person singular pronoun, the \textstyleAcronymallcaps{\textsc{rc}} is usually non-restrictive \REF{ex:8:x1567}, \REF{ex:8:x1568}:

\ea%x1567
\label{ex:8:x1567}
\gll Bang=ke  ekap-o-k,  {\ob}Ponkila  aaw-o-k  nain{\cb}.\\
Bang=\textsc{cf} come-\textsc{pa}-3s Ponkila get-\textsc{pa}-3s  that1\\
\glt`Bang came, (he) who married Ponkila.'
\z


\ea%x1568
\label{ex:8:x1568}
\gll Yo  nena  owowa  {\ob}moma  marew  nain{\cb}  miatin-i-yem. \\
1s.\textsc{unm} 2s.\textsc{gen} village  taro  no(ne)  that1 dislike-\textsc{Np}-\textsc{pr}.1s\\
\glt`I don't like your village, which doesn't have taro.'
\z


The proximate demonstrative \textstyleStyleVernacularWordsItalic{fain} `this' can also function as a relative marker in the non-restrictive \textstyleAcronymallcaps{\textsc{rc}}s but not in restrictive ones \REF{ex:8:x1536}:

\ea%x1536
\label{ex:8:x1536}
\gll Nomokowa  unowa  fan-e-mik,  {\ob}Simbine ekap-omak-e-mik  fain{\cb}.\\
2s/p.brother  many  here-\textsc{pa}-1/3p Simbine come-\textsc{distr}/\textsc{pl}-\textsc{pa}-1/3p this\\
\glt`Your many (clan) brothers are here, these Simbine people who came.'
\z


When the \textstyleAcronymallcaps{\textsc{antnp}} is a pronoun other than first or second singular, the \textstyleAcronymallcaps{\textsc{rc}} may be either restrictive or non-restrictive \REF{ex:8:x1570}.

\ea%x1570
\label{ex:8:x1570}
\gll I  mua  yiam  ikur,  {\ob}fikera  ikiw-e-mik  nain{\cb}. \\
1p.\textsc{unm} man 1p.\textsc{refl} five kunai.grass go-\textsc{pa}-1/3p that1\\
\glt`There were five of us men who went to the kunai grass (= pig-hunting).' Or: `We were five men, who went pig-hunting.'
\z


\subsection{Complement clauses and other complementation strategies} \label{sec:8.3.2}
%\hypertarget{RefHeading23401935131865}

The prototypical function of a complement clause is as a subject or object in a main clause. In Mauwake, a complement clause proper functions as an object of a complement-taking verb (\textstyleAcronymallcaps{\textsc{ctv}}), and occasionally as a subject in a non-verbal clause. Structurally it is a Type 2 nominalized clause: a finite clause that has the distal-1 demonstrative \textstyleStyleVernacularWordsItalic{nain} `that' occurring as a nominaliser clause-finally (\sectref{sec:5.7.2}). The complement clause precedes the complement-taking verb. The complement clause differs from the relative clause in that none of the noun phrases inside it is an \textstyleAcronymallcaps{\textsc{antnp}} or a \textstyleAcronymallcaps{\textsc{relnp}}. 

The division of complements into different types, ``Fact, Activity and Potential'', that \citet[371]{Dixon2010b} provides, is crucial for the use of the different complementation strategies in Mauwake.  A complement clause is normally used when a \textstyleAcronymallcaps{\textsc{ctv}} needs a fact-type object complement.

Besides the regular complement clause described above, Mauwake has other complementation strategies. The indirect speech clauses are ordinary sentences embedded in the utterance clause (\sectref{sec:8.3.2.1.2}). Medial clauses are used as the main complementation strategy for activity-type complements with perception verbs (\sectref{sec:8.3.2.2}). Clauses with a nominalized verb are used for potential-type complements with various \textstyleAcronymallcaps{\textsc{ctv}}s. The regular complement clause and the clause with a nominalized verb may occur as a subject of a clause (\sectref{sec:8.3.2.4}). 

Since one \textstyleAcronymallcaps{\textsc{ctv}} can take more than one complementation strategy, the main grouping below is done according to the \textstyleAcronymallcaps{\textsc{ctv}}s.

\subsubsection{Complements of utterance verbs} \label{sec:8.3.2.1}
%\hypertarget{RefHeading23421935131865}

Some utterance verbs (\sectref{sec:3.8.4.4.6}) are also used for thinking, so speech and thought are discussed as one group. 

The status of direct quote clauses (\sectref{sec:8.3.2.1.1}) as complement clauses is questionable, but they are discussed here because of their co-occurrence with the utterance verbs and their similarity with the indirect quotes (\sectref{sec:8.3.2.1.2}), which are complement clauses.  

The most important of the utterance verbs is \textstyleStyleVernacularWordsItalic{na}- `say, think'. It is used as the utterance verb for indirect quote complements, which in turn have grammaticalized, together with the same subject sequential form of the verb, as desiderative (\sectref{sec:8.3.2.1.3}) and purpose clauses (\sectref{sec:8.3.2.1.4}) and the conative construction (\sectref{sec:8.3.2.1.5}) 

\paragraph[Direct speech]{Direct speech} \label{sec:8.3.2.1.1}
%\hypertarget{RefHeading23441935131865}

It seems to be a universal feature of direct quote clauses that they behave independently of their matrix clauses. If they are considered complement clauses of utterance verbs, their independence sets them apart from all the other complement clauses \citep[303]{Munro1982}. \citet[398]{Dixon2010a} maintains that direct speech quotes are not any kind of complementation.

A direct quote may be a whole discourse on its own, not just a clause within a sentence.

It is rather typical in Papuan languages to have a strict quote formula both before and after a quotation, or at least before it (\citealt[120]{Franklin1971}, \citealt[1]{Davies1981}, \citealt[12]{Roberts1987}, \citealt{Farr1999}, \citealt[128]{Hepner2002}). It is also common that either there is no separate structure for indirect speech \citep[2]{Davies1981} or that direct and indirect speech are so similar that they are often hard to distinguish from each other \citep[14]{Roberts1987}. 

In the use of quote formulas, Mauwake is much freer than Papuan languages in general. A direct quotation in Mauwake is often preceded or followed by one of the utterance verbs. The verbs \textstyleStyleVernacularWordsItalic{na}- `say/think' and \textstyleStyleVernacularWordsItalic{naak}- `say/tell' are almost exclusively used after quotes. Enclosing a quote between two utterance verbs is not frequent \REF{ex:8:x1571}:

\ea%x1571
\label{ex:8:x1571}
\gll Ne  ona  mua  pun  \textbf{ma-e-k},  ``Eka  maneka  nain=ke iwa-mi  ifakim-o-k,''  \textstyleEmphasizedVernacularWords{na-e-k}.\\
\textsc{add} 3s.\textsc{gen} man  also  say-\textsc{pa}-3s river big  that1=\textsc{cf} come-\textsc{ss}.\textsc{sim} kill-\textsc{pa}-3s  say-\textsc{pa}-3s\\
\glt`Her husband also said, ``The big river came and killed her,'' he said.'
\z


Most commonly, only a speech verb precedes the quote \REF{ex:8:x1578}, \REF{ex:8:x1579}:

\ea%x1578
\label{ex:8:x1578}
\gll Panewowa=ke  \textbf{ma-e-k},  ``Yo  nia  maak-emkun  opaimika efa  fien-a-man.''\\
old=\textsc{cf} say-\textsc{pa}-3s 1s.\textsc{unm} 2p.\textsc{acc} tell-2/3p.\textsc{ds} talk 1s.\textsc{acc} disobey-\textsc{pa}-2p\\
\glt`The old (woman) said, ``When I told you, you disobeyed me.'' '
\z


\ea%x1579
\label{ex:8:x1579}
\gll Iiw-ep  wiipa  muuka  nain  wia  \textbf{maak-e-k}, ``Auwa  maa  p-ikiw-om-aka.\\
dish.out-\textsc{ss}.\textsc{seq} daughter  son  that1 3p.\textsc{acc} tell-\textsc{pa}-3s 1s/p.father food \textsc{bpx}-go-\textsc{ben}-\textsc{bnfy}2.\textsc{imp}.2p\\
\glt`She dished out (the food) and told the children, ``Take the food to father.''{}'
\z


A single utterance verb \REF{ex:8:x1580} or a whole quote-closing clause \REF{ex:8:x1583} may follow the quote. A quote-closing clause has to be used when the quotation consists of several sentences.

\ea%x1580
\label{ex:8:x1580}
\gll ``No  bom  fain=iw  mera  kuum-e,''  \textbf{naak-e-mik}. \\
2s.\textsc{unm} bomb this=\textsc{inst} fish burn-\textsc{imp}.2s  tell-\textsc{pa}-1/3p\\
\glt` ``Blast fish with this bomb,'' they told him.'
\z


\ea%x1583
\label{ex:8:x1583}
\gll ``I  muuka  marew  a,  wiipa  marew  a,''  \textbf{naap} \textbf{wia} \textstyleEmphasizedVernacularWords{maak-e-k}.\\
1p.\textsc{unm} son  no(ne)  ah  daughter  no(ne)  ah  thus 3p.\textsc{acc} tell-\textsc{pa}-3s\\
\glt` ``We have no son, we have no daughter,'' he told them like that.'
\z


In narratives where there are several exchanges between the participants, it is possible to leave out the utterance verb \REF{ex:8:x1581}, \REF{ex:8:x1582} and even the \textstyleAcronymallcaps{\textsc{np}} referring to the speaker of the utterance \REF{ex:8:x1581}, if the speaker is clear enough from the context. A good speaker creates variety in the text by utilizing all these different possibilities. 

\ea%x1581
\label{ex:8:x1581}
\gll Ne  onak=ke  \textstyleEmphasizedVernacularWords{{\O}},  ``A,  ifera  feeke  un-eka.'' Ne  wi  maak-e-mik,  ``Wia,  i  oro-or-op un-i-yan.''  ``A,  neeke-r=iw  un-eka.''  ``Weetak, i  oro-ora-i-yan.''\\
\textsc{add} 3s/p.mother  {\O}  Ah,  salt.water  here.\textsc{cf} fetch(water)-\textsc{imp}.2p \textsc{add} 3p.\textsc{unm} tell-\textsc{pa}-1/3p No  1p.\textsc{unm} \textsc{rdp}-descend-\textsc{ss}.\textsc{seq} fetch-\textsc{Np}-\textsc{fu}.1p Ah there.\textsc{cf}-{\O}=\textsc{lim} fetch-\textsc{imp}.2p no 1p.\textsc{unm} \textsc{rdp}-descend-\textsc{Np}-\textsc{fu}.1p\\
\glt`And their mother (said), ``Ah, fetch the sea water (from) here.'' But they told her, ``No, we'll go down (to the deep sea) and fetch it.'' ``Ah, fetch it right there.'' ``No, we'll go down a long way.'' '
\z


\ea%x1582
\label{ex:8:x1582}
\gll ``Mauwa ar-e-n, amia=iya nenar-e-mik=i?'' Sarak=ke \textstyleEmphasizedVernacularWords{{\O}}.\hspace{-1mm}\\
what become-\textsc{pa}-2s bow=\textsc{com} shoot.you-\textsc{pa}-1/3p=\textsc{qm} Sarak=\textsc{cf} {\O}\\
\glt```What happened to you, did they shoot you with a gun?'' Sarak (asked).'
\z


\paragraph[Indirect speech]{Indirect speech} \label{sec:8.3.2.1.2}
%\hypertarget{RefHeading23461935131865}

Indirect speech quotes, which report speech or thought, are objects of speech verbs. 

Most indirect quotes in Mauwake are syntactically identical to direct quotes. There is an intonational difference: the indirect quote is part of the intonation contour of the main clause, rather than having a contour of its own as a direct quote has. The quote is almost always followed by the utterance verb \textstyleStyleVernacularWordsItalic{na}- `say, think' \REF{ex:8:x1585}; but it is also possible for the verb \textstyleStyleVernacularWordsItalic{ma}- `say' to precede it, in which case both the utterance verb and the quote have their own intonation contour \REF{ex:8:x1587}.\footnote{In Amele, the absence of the speech verb before the quote is the main criterion for indirect speech \citep[14]{Roberts1987}. In Mauwake, this cannot be used as a criterion, as the occurrence of speech verbs with direct quotes varies so much.} An indirect quote is never enclosed between two utterance verbs. 

\ea%x1585
\label{ex:8:x1585}
\gll Aria,  Kalina,  {\ob}Amerika  ekap-e-mik{\cb}  na-i-mik. \\
alright  Kalina  America  come-\textsc{pa}-1/3p say-\textsc{Np}-\textsc{pr}.1/3p\\
\glt`Alright, Kalina, they say that the Americans have come.'
\z


\ea%x1587
\label{ex:8:x1587}
\gll Ma-e-m,  {\ob}nena  owowa=pa  ik-o-n{\cb}.\\
say-\textsc{pa}-1s 2s.\textsc{gen} village=\textsc{loc} be-\textsc{pa}-2s\\
\glt`I said (to her\textsubscript{i}) that you\textsubscript{j} are in your own village.'
\z


As direct quotes behave independently of their matrix clauses, they often have a separate deictic centre. But indirect quotes vary in this respect. Deictic elements, which get part or all of their interpretation from the situational context, are often the same in indirect quotes as they would be in direct quotes \REF{ex:8:x1584}, \REF{ex:8:x1280}: 

\ea%x1584
\label{ex:8:x1584}
\gll Aite=ke  {\ob}manina  yook-e{\cb}  na-eya  o ook-e.\\
1s/p.mother=\textsc{cf} garden follow.me-\textsc{imp}.2s say-2/3s.\textsc{ds} 3s.\textsc{unm} follow.her-\textsc{imp}.2s\\
\glt`When mother tells you to follow her to the garden, follow her.'
\z


\ea%x1280
\label{ex:8:x1280}
\gll Ni  Krais  {\ob}yena  teeria  efar  ik-eka{\cb}  na-ep nia  far-eya  ona  teeria  wiar  ik-e-man.\\
2p.\textsc{unm} Christ 1s.\textsc{gen} family 1s.\textsc{dat} be-\textsc{imp}.2p say-\textsc{ss}.\textsc{seq} 2p.\textsc{acc} call-2/3s.\textsc{ds} 3s.\textsc{gen} family 3.\textsc{dat} be-\textsc{pa}-2p\\
\glt`Christ called you to be his family and (now) you are his family.'
\z


But the deictic centre may also shift partly or completely towards that of the matrix clause. When this happens, the pronouns are the easiest to change, next the adverbs. In \REF{ex:8:x1586} a second person pronoun has replaced the proper name or third person pronoun that would have been used in a direct quote.

\ea%x1586
\label{ex:8:x1586}
\gll Sarak  oo,  Amerika  ekap-ep  Ulingan  nan  ik-e-mik, {\ob}\textstyleEmphasizedVernacularWords{nefa}  ikum-i-mik{\cb}  na-i-mik  oo. \\
Sarak \textsc{intj} America come-\textsc{ss}.\textsc{seq} Ulingan  there be-\textsc{pa}-1/3p 2s.\textsc{acc} wonder.about-\textsc{Np}-\textsc{pr}.1/3p say-\textsc{Np}-\textsc{pr}.1/3p \textsc{intj}\\
\glt`Sarak! The Americans have come and are in Ulingan and they say that they are wondering where you are!'
\z


When reported by the addressee of the example clause \REF{ex:8:x1281}, only the pronoun in the reported clause \REF{ex:8:x1282} is different:

\ea%x1281
\label{ex:8:x1281}
\gll No  owowa  ikiw-ep  buk  nain  sesek-om-e.\\
2s.\textsc{unm} village  go-\textsc{ss}.\textsc{seq} book that1  send-\textsc{ben}-\textsc{bnfy}1.\textsc{imp}.2s\\
\glt`When you go to the village, send the book to me.'
\z


\ea%x1282
\label{ex:8:x1282}
\gll {\ob}\textstyleEmphasizedVernacularWords{Yo} owowa \textstyleEmphasizedVernacularWords{ikiw-ep} buk  nain  sesek-om-e{\cb} efa  na-e-k. \\
1s.\textsc{unm} village go-\textsc{ss}.\textsc{seq} book that1 send-\textsc{ben}-\textsc{bnfy}1.\textsc{imp}.2s 1s.\textsc{acc} say-\textsc{pa}-3s\\
\glt`He told me to send that book to him (lit: me) when I would go to the village.'
\z


The verbs are most resistant to deictic shift. In \REF{ex:8:x1264}, even though the verb root changes, it still retains the tense and person marking of the direct quote \REF{ex:8:x1265}. Both the temporal adverb and the pronoun are shifted to reflect the deictic centre of the matrix clause.

\ea%x1264
\label{ex:8:x1264}
\gll Uurika  nefar  ikiw-i-nen. \\
tomorrow  2s.\textsc{dat} go-\textsc{Np}-\textsc{fu}.1s\\
\glt`Tomorrow I'll come (lit: go) to you.'
\z


\ea%x1265
\label{ex:8:x1265}
\gll {\ob}\textstyleEmphasizedVernacularWords{Ikoka}  \textstyleEmphasizedVernacularWords{efar}  \textstyleEmphasizedVernacularWords{ekap-}i-nen{\cb}  na-e-k  na  weetak.\\
Later(today)  1s.\textsc{dat} come-\textsc{Np}-\textsc{fu}.1s  say-\textsc{pa}-3s but no\\
\glt`He said that he would come to me today, but he hasn't.'
\z


Below in \REF{ex:8:x1266} also the person suffix is changed from that in \REF{ex:8:x1267} to reflect the situation of the new speech act participants. 

\ea%x1267
\label{ex:8:x1267}
\gll Ona  owowa=pa  ik-ua.\\
3s.\textsc{gen} village=\textsc{loc} be-\textsc{pa}.3s\\
\glt`She is in her own village.'
\z


\ea%x1266
\label{ex:8:x1266}
\gll Ma-e-m,  {\ob}\textstyleEmphasizedVernacularWords{nena} \textstyleEmphasizedVernacularWords{} owowa=pa  \textstyleEmphasizedVernacularWords{ik-o-n}{\cb}. \\
say-\textsc{pa}-1s 2s.\textsc{gen} village=\textsc{loc} be-\textsc{pa}-2s\\
\glt`I said (to her) that you are in your own village.'
\z


The deictic shift would need more study to ascertain if there are specific rules governing this variation in indirect quotes.

When the verb \textstyleStyleVernacularWordsItalic{na}- `say, think' indicates thinking, the complement clause is usually an indirect quote rather than a direct one \REF{ex:8:x1588}, \REF{ex:8:x1589}. 

\ea%x1588
\label{ex:8:x1588}
\gll {\ob}Muuka  ifera  me  enim-i-non{\cb} na-ep  me  uruf-a-m.\\
boy  salt.water  not  drink-\textsc{Np}-\textsc{fu}.3s think-\textsc{ss}.\textsc{seq} not look-\textsc{pa}-1s\\
\glt`Thinking that the boy wouldn't drown I didn't watch him.'
\z


\ea%x1589
\label{ex:8:x1589}
\gll Mua  pepena=ke  {\ob}menat=ke  ek-i-ya{\cb}  na-ep menat  ora-i-nan.\\
man  inexperienced=\textsc{cf} tide=\textsc{cf} go-\textsc{Np}-\textsc{pr}.3s think-\textsc{ss}.\textsc{seq} tide descend-\textsc{Np}-\textsc{fu}.2s\\
\glt`An inexperienced man will think that the tide is going down and will go to fish at low tide.'
\z


Indirect non-polar questions are similar to the corresponding direct questions apart from possible adjustments to deictic elements \REF{ex:8:x1592}, \REF{ex:8:x1590}.

\ea%x1592
\label{ex:8:x1592}
\gll {\ob}Wi  uf-ow(a)  epa  kaaneke  ik-ua{\cb}  na-e-k.\\
3p.\textsc{unm} dance-\textsc{nmz} place where.\textsc{cf} be-\textsc{pa}.3s say-\textsc{pa}-3s\\
\glt`He asked where their dancing place was.'
\z


\ea%x1590
\label{ex:8:x1590}
\gll {\ob}O  ikoka  sesa  kamenap  aaw-i-non{\cb}  na-e-k.\\
3s.\textsc{unm} later  price  what.like  get-\textsc{Np}-\textsc{fu}.3s  say-\textsc{pa}-3s\\
\glt`He asked what kind of wages he would get later.'
\z


Polar questions, when indirect, have to be alternative questions \REF{ex:8:x1591}. The verb \textstyleStyleVernacularWordsItalic{naep} may be deleted, when the indirect question is sentence-final \REF{ex:8:x1593}. 

\ea%x1591
\label{ex:8:x1591}
\gll {\ob}Beel-al-i-non=i  kamenion{\cb}  na-ep uruf-am-ik-ua.\\
rotten-\textsc{inch}-\textsc{Np}-\textsc{fu}.3s=\textsc{qm} or.what  think-\textsc{ss}.\textsc{seq} see-\textsc{ss}.\textsc{sim}-be-\textsc{pa}.3s\\
\glt`He was watching (thinking) whether it would rot or what would happen.'
\z


\ea%x1593
\label{ex:8:x1593}
\gll Wi  iwera  iinan=pa  ik-ok  iwer(a)  popoka wafur-am-ik-e-mik,  {\ob}eka  saanar-e-k=i  eewuar{\cb} {\O}. \\
3p.\textsc{unm} coconut top=\textsc{loc} be-\textsc{ss} coconut unripe throw-\textsc{ss}.\textsc{sim}-be-\textsc{pa}-1/3p water dry-\textsc{pa}-3s=\textsc{qm} not.yet\\
\glt`They were at the top of the coconut palm and threw unripe coconuts (thinking) whether the water had dried or not yet.'
\z


\paragraph[Desiderative clauses]{Desiderative clauses}\label{sec:8.3.2.1.3}
%\hypertarget{RefHeading23481935131865}

It is very common in Papuan languages that an indirect quote construction with the intended action verb in future tense, imperative or irrealis form expresses a want/wish\footnote{In Mauwake, the verb \textit{kookal}- `like, love, desire', is mostly used with an \textsc{np} object, but it can take a clausal complement as well. It does not indicate intention or purpose. The complement is either type of nominalized clauses (\sectref{sec:5.7}).}, desire or intention to do something. (\citealt[254--259]{Reesink1987}, \citealt[157]{Foley1986}, \citealt[112]{Hardin2002}, \citealt[76--77]{Hepner2002}).

In Mauwake, the future, imperative, counterfactual and nominalized forms of the main verb are possible in the complement clause. In desiderative clauses, the verb \textstyleStyleVernacularWordsItalic{na-} `say/think' is always in the medial same-subject sequential form \textstyleStyleVernacularWordsItalic{naep}; in purpose clauses, other forms are possible as well. Historically, there probably always used to be a clause with a finite verb following the clause expressing intention or desire \REF{ex:8:x367}; synchronically the finite clause is often missing \REF{ex:8:x368}, especially when the verb would be the same as in the complement.

\ea%x367
\label{ex:8:x367}
\gll Niena  {\ob}maa  enim-u{\cb}  na-ep  iiw-eka. \\
2p.\textsc{gen} food eat-\textsc{imp}.1d say-\textsc{ss}.\textsc{seq} dish.out-\textsc{imp}.2p\\
\glt`If you want/intend to eat food, dish it out (yourselves).'
\z


\ea%x368
\label{ex:8:x368}
\gll Yo  {\ob}opora  gelemuta=ko  ma-i-nen{\cb}  na-ep.\\
1s.\textsc{unm} talk  little=\textsc{nf} say-\textsc{nps}-\textsc{fu}.1s  say-\textsc{ss}.\textsc{seq}\\
\glt`I want to tell a little story.' Or: `I'm going to tell a little story.'
\z


The main verb in the complement clause is either marked for first person\footnote{Purpose clauses may use other person forms as well (\sectref{sec:8.3.2.1.4}).} or is nominalized. Mauwake uses the future \REF{ex:8:x368} or imperative form \REF{ex:8:x369} of the main verb for intention or a clear/certain wish, and the counterfactual form for a wish that has less potential to be realized. The latter is also the most polite form to use, if the wish indicates a request \REF{ex:8:x370}. 

\ea%x369
\label{ex:8:x369}
\gll {\ob}Haussik  p-ek-u{\cb}  na-ep  miiw-aasa  nop-a-mik.\\
aidpost \textsc{\textsc{bp}x}-go-\textsc{imp}.1d say-\textsc{ss}.\textsc{seq} land-canoe search-\textsc{pa}-1/3p\\
\glt`We/they wanted to take her to the aidpost and looked for a vehicle.'
\z


\ea%x370
\label{ex:8:x370}
\gll {\ob}Yo=ko  wia  uruf-ek-a-m{\cb} na-ep.\\
1s.\textsc{unm}=\textsc{nf} 3p.\textsc{acc} see-\textsc{cntf}-\textsc{pa}-1s say-\textsc{ss}.\textsc{seq}\\
\glt`I would like to see them.'
\z


The nominalized form is mostly used in complement clauses that can also be interpreted as purpose clauses. In ``pure'' desiderative clauses it is practical to use the nominalized form especially if the first person marking in the verb might make it harder to process the meaning \REF{ex:8:x1610}:

\ea%x1610
\label{ex:8:x1610}
\gll Ne  {\ob}o  uruf-owa{\cb}  ne  {\ob}maa  en-owa  asip-owa{\cb} na-ep=na  eliw  asip-uk.\\
\textsc{add} 3s.\textsc{unm} see-\textsc{nmz} \textsc{add} food  eat-\textsc{nmz} help-\textsc{nmz} say-\textsc{ss}.\textsc{seq}=\textsc{tp} well help-\textsc{imp}.3p\\
\glt`And if they want to see him and help him with food, let them help him.'
\z


\paragraph[Purpose clauses]{Purpose clauses} \label{sec:8.3.2.1.4}
%\hypertarget{RefHeading23501935131865}

Purpose is both conceptually close and structurally similar to desiderative, and in Mauwake many of the desiderative clauses can be interpreted as purpose clauses. This is particularly so when the main verb is in the nominalized form \REF{ex:8:x371}, \REF{ex:8:x345}. But a truly desiderative clause even with an action nominal is never right-dislocated, whereas a purpose clause \REF{ex:8:x372} often is. The nominalized form in the main verb is common:

\ea%x371
\label{ex:8:x371}
\gll {\ob}Weniwa=pa  en-owa{\cb}  na-ep  uuw-i-mik. \\
famine=\textsc{loc} eat-\textsc{nmz} say-\textsc{ss}.\textsc{seq} work-\textsc{Np}-\textsc{pr}.1/3p\\
\glt`We work in order to (be able to) eat during the time of hunger.'
\z


\ea%x345
\label{ex:8:x345}
\gll {\ob}Wi  Amerika  wiam=iya  irak-owa{\cb}  na-ep  ikiw-e-mik.\\
3p.\textsc{unm} America  3p=\textsc{com} fight-\textsc{nmz} say/think-\textsc{ss}.\textsc{seq} go-\textsc{pa}-1/3p\\
\glt`They went to fight with the Americans.'
\z


\ea%x372
\label{ex:8:x372}
\gll Ona  siowa  ikos  manina  ikiw-e-mik, {\ob}pika  on-owa{\cb} na-ep.\\
3s.\textsc{gen} dog  with  garden  go-\textsc{pa}-1/3p  fence  make-\textsc{nmz} say-\textsc{ss}.\textsc{seq}\\
\glt`He went to the garden with his dog, in order to make a fence.'
\z


Future and imperative forms are also used in the purpose clause. When the subject in the purpose clause is the same as the subject of the utterance verb and the main clause, the first person future form is used for singular \REF{ex:8:x1614}, \REF{ex:8:x1616} and first person dual imperative for plural \REF{ex:8:x1620}.

\ea%x1614
\label{ex:8:x1614}
\gll {\ob}Nain  nefa  maak-i-nen{\cb}  na-ep  yo  ep-a-m.\\
that1 2s.\textsc{acc} tell-\textsc{Np}-\textsc{fu}.1s  say-\textsc{ss}.\textsc{seq} 1s.\textsc{unm} come-\textsc{pa}-1s\\
\glt`I came to tell you that.'
\z


\ea%x1616
\label{ex:8:x1616}
\gll No  {\ob}owora  sesenar-i-nen{\cb}  na-ep  Kainantu  fofa ikiw-ep  neeke  aaw-i-nan.\\
2s.\textsc{unm} betelnut  buy-\textsc{Np}-\textsc{fu}.1s say-\textsc{ss}.\textsc{seq} Kainantu  market go-\textsc{ss}.\textsc{seq} there.\textsc{cf} get-\textsc{nps}-\textsc{fu}.2s\\
\glt`To buy betelnut you will (need to) go to Kainantu marker and get it \textstyleEmphasizedWords{{there}}.'
\z


\ea%x1620
\label{ex:8:x1620}
\gll Ne  {\ob}haussik  p-ek-u{\cb} na-ep miiw-aasa nop-a-mik.\\
\textsc{add} aidpost \textsc{bpx}-go-\textsc{imp}.1d say-\textsc{ss}.\textsc{seq} land-canoe search-\textsc{pa}-1/3p\\
\glt`And they searched for a truck (in order) to take him to the aidpost.'
\z


When the subject of the verb in the main clause differs from that of the purpose clause, the verb inside the purpose clause has to be in the imperative \REF{ex:8:x1062}--\REF{ex:8:x1627}. The whole purpose clause is structurally like a direct quote of the ``inner speech'' verb \textstyleStyleVernacularWordsItalic{naep}, so there is no deictic shift of the kind that may take place in indirect quotes.  

\ea%x1062
\label{ex:8:x1062}
\gll {\ob}Me  yiar-uk{\cb}  na-ep  koka=pa  ik-e-mik. \\
not  shoot.us-\textsc{imp}.3p say-\textsc{ss}.\textsc{seq} jungle=\textsc{loc} be-\textsc{pa}-1/3s\\
\glt`We stayed in the jungle so that they would not shoot us.'
\z


\ea%x346
\label{ex:8:x346}
\gll {\ob}Auwa=ke  o=ko  amukar-inok{\cb}  na-ep  maa  naap sirar-em-ik-e-mik.\\
1s/p.father=\textsc{cf} 3s.\textsc{unm}=\textsc{nf} scold-\textsc{imp}.3s say-\textsc{ss}.\textsc{seq} thing thus make-\textsc{ss}.\textsc{sim}-be-\textsc{pa}-1/3\\
\glt`They kept doing things like that so that father would scold \textstyleEmphasizedWords{him} (and not them).'
\z


\ea%x1615
\label{ex:8:x1615}
\gll Nain  {\ob}ni  amis-ar-eka{\cb}  na-ep  feenap on-i-yem.\\
that1 2p.\textsc{unm} knowledge-\textsc{inch}-\textsc{imp}.2p  say-\textsc{ss}.\textsc{seq} like.this do-\textsc{Np}-\textsc{pr}.1s\\
\glt`But I am doing this so that you would know.'
\z


\ea%x1617
\label{ex:8:x1617}
\gll {\ob}Efa  asip-e{\cb}  na-ep  ekap-e-m. \\
1s.\textsc{acc} help-\textsc{imp}.2s say-\textsc{ss}.\textsc{seq} come-\textsc{pa}-1s\\
\glt`I came so that you would help me.'
\z


\ea%x1618
\label{ex:8:x1618}
\gll {\ob}Feenap  nokar-eka{\cb}  na-ep  yia  sesek-a-k. \\
like.this  ask-\textsc{imp}.2p say-\textsc{ss}.\textsc{seq} 1p.\textsc{acc} send-\textsc{pa}-3s\\
\glt`He sent us to ask (you) like this.'
\z


\ea%x1619
\label{ex:8:x1619}
\gll {\ob}Yo efa miim-eka{\cb} na-ep wapena wu-ami ma-e-k,{\dots}\\
1s.\textsc{unm} 1s.\textsc{acc} hear-\textsc{imp}.2p say-\textsc{ss}.\textsc{seq} hand put-\textsc{ss}.\textsc{sim} say-\textsc{pa}-3s\\
\glt`He raised his hand for them to listen to him and said, {\dots}'
\z


\ea%x1627
\label{ex:8:x1627}
\gll Ne  wi  popor-ar-urum-ep  ik-ok  ifana  muutiw wu-am-ika-i-kuan,  {\ob}mua  unuma  wia  miim-u{\cb} na-ep.\\
\textsc{add} 3p.\textsc{unm} silent-\textsc{inch}-\textsc{distr}/\textsc{a}-\textsc{ss}.\textsc{seq} be-\textsc{ss} ear  only put-\textsc{ss}.\textsc{sim}-be-\textsc{Np}-\textsc{fu}.3p  man  name  3p.\textsc{acc} hear-\textsc{imp}.1d say-\textsc{ss}.\textsc{seq}\\
\glt`And they all will be quiet and listen carefully in order to hear the men's names.'
\z


There is no raising of negation from the subordinate to the main clause \REF{ex:8:x1623}. 

\ea%x1623
\label{ex:8:x1623}
\gll {\ob}Yo  me  pina=pa  nia  wu-ek-a-m{\cb}  na-ep ma-i-yem. \\
1s.\textsc{unm} not guilt=\textsc{loc} 2p.\textsc{acc} put-\textsc{cntf}-\textsc{pa}-1s say-\textsc{ss}.\textsc{seq} say-\textsc{Np}-\textsc{pr}.1s\\
\glt`I am not saying (this) to put guilt on you.' (=I am saying this, but not in order to put guilt on you.)
\z


A purpose clause does not always have the auxiliary \textstyleStyleVernacularWordsItalic{naep}. A clause with just a nominalized verb is used especially with the directional verbs \REF{ex:8:x1659}, \REF{ex:8:x1658}: 

\ea%x1659
\label{ex:8:x1659}
\gll {\ob}Yo  yena  emeria  aaw-owa{\cb}  urup-e-m.\\
1s.\textsc{unm} 1s.\textsc{gen} woman take-\textsc{nmz} ascend-\textsc{pa}-1s\\
\glt`I came up to take my wife.'
\z


\ea%x1658
\label{ex:8:x1658}
\gll Bogia  ikiw-e-mik,  {\ob}opaimika  aakun-owa{\cb}. \\
Bogia  go-\textsc{pa}-1/3p talk talk-\textsc{nmz}\\
\glt`We went to Bogia to talk.'
\z


A clause with a nominalized verb plus a clause-final distal-1 demonstrative \textstyleStyleVernacularWordsItalic{nain} 'that' is also possible, but less common \REF{ex:8:x1633}, \REF{ex:8:x1634}. I have not observed a functional difference between the different purpose structures.

\ea%x1633
\label{ex:8:x1633}
\gll Tunde=pa  {\ob}maa  muutitik  uruf-owa  nain{\cb}  soomar-e-mik.\\
Tuesday=\textsc{loc} thing all.kinds see-\textsc{nmz} that1  walk-\textsc{pa}-1/3p\\
\glt`On Tuesday we walked to see all kinds of things.'
\z


\ea%x1634
\label{ex:8:x1634}
\gll Ifemak-ep  nomona  iinan=pa  wua-i-nan,  {\ob}ikoka  ifera  me p-ikiw-owa  nain{\cb}. \\
press-\textsc{ss}.\textsc{seq} stone on.top=\textsc{loc} put-\textsc{Np}-\textsc{fu}.2s later sea not \textsc{bpx}-go-\textsc{nmz} that1\\
\glt`You press it down and put stones on top ( or: put it on top of stones/corals) so that the sea would not later take it away.'
\z


\paragraph[Conative clauses: `try' ]{Conative clauses: `try'}  \label{sec:8.3.2.1.5}
%\hypertarget{RefHeading23521935131865}

Instead of using a verbal construction with the verb `see' for conative modality -- expressing the attempt to do something -- which \citet[152]{Foley1986} claims as almost universal for Papuan languages, Mauwake makes use of a structure where the desiderative is followed by the verb \textstyleStyleVernacularWordsItalic{on-} `do' as the verb in its reference clause \REF{ex:8:x373}. Usan uses an identical construction for the same purpose \citep[258]{Reesink1987}. 

\ea%x373
\label{ex:8:x373}
\gll {\ob}Mukuna  umuk-u  na-ep  on-a-mik{\cb}=na  me  pepek.\\
fire  extinguish-\textsc{imp}.1d say-\textsc{ss}.\textsc{seq} do-\textsc{pa}-1/3p=\textsc{tp} not enough\\
\glt`We tried to extinguish the fire but were not able to.'
\z


When this structure is used, it is implied that somehow or other the effort fails \REF{ex:8:x374}, \REF{ex:8:x1606}:

\ea%x374
\label{ex:8:x374}
\gll {\ob}Emeria  aruf-i-nen  na-ep  on-am-ik-eya{\cb}  op-a-mik.\\
woman hit-\textsc{Np}-\textsc{fu}.1s say-\textsc{ss}.\textsc{seq} do-\textsc{ss}.\textsc{sim}-be-2/3s.\textsc{ds} hold-\textsc{pa}-1/3p\\
\glt`When he was trying to hit the woman, they grabbed him.'
\z


\ea%x1606
\label{ex:8:x1606}
\gll {\ob}Wia  uruf-ek-a-m  na-ep  on-a-k  on-a-k{\cb} weetak, o me wia  uruf-a-k.\\
3p.\textsc{acc} see-\textsc{cntf}-\textsc{pa}-1s say-\textsc{ss}.\textsc{seq} do-\textsc{pa}-3s do-\textsc{pa}-3s no 3s.\textsc{unm} not  3p.\textsc{acc} see-\textsc{pa}-3s\\
\glt`He tried and tried to see them, but no, he didn't see them.'
\z


The conative structure is not used when the effort is successful \REF{ex:8:x375}, and also when the `trying' is not so much an effort to do something as experimenting \REF{ex:8:x376}.  In these cases the verb \textstyleStyleVernacularWordsItalic{akim-} `try' is used, which is neutral as to the outcome. It requires a nominalized verb in the complement clause. 

\ea%x375
\label{ex:8:x375}
\gll {\ob}Aasa  keraw-owa{\cb}  akim-ap  akim-ap amis-ar-i-nan.\\
canoe  carve-\textsc{nmz} try-\textsc{ss}.\textsc{seq} try-\textsc{ss}.\textsc{seq} knowledge-\textsc{inch}-\textsc{Np}-\textsc{fu}.2s\\
\glt`After trying and trying to carve a canoe, you will know (how to do it).'
\z


\ea%x376
\label{ex:8:x376}
\gll {\ob}Weria  op-ap  wiinar-owa nain{\cb}  akim-am-ik-e.\\
planting.stick hold-\textsc{ss}.\textsc{seq} make.planting.holes-\textsc{nmz} that1 try-\textsc{ss}.\textsc{sim}-be-\textsc{imp}.2s\\
\glt`Keep trying/learning to make planting holes with the planting stick.'
\z


\paragraph[Complements of other utterance verbs ]{Complements of other utterance verbs} 
%\hypertarget{RefHeading23541935131865}

The verb \textstyleStyleVernacularWordsItalic{ma}- `say, talk' can take a regular complement clause, which is of the fact type \citep[389]{Dixon2010b}. This clause functions as an object of the verb in the same way as an \textstyleAcronymallcaps{\textsc{np}} with the head noun \textstyleStyleVernacularWordsItalic{opora} (or \textstyleStyleVernacularWordsItalic{opaimika}) `talk/story' in \REF{ex:8:x1595}:

\ea%x1595
\label{ex:8:x1595}
\gll {\ob}Opora  gelemuta=ko{\cb}\textsubscript{\textsc{np}}  ma-i-nen  na-ep.\\
talk  little=\textsc{nf} say-\textsc{Np}-\textsc{fu}.1s say/think-\textsc{ss}.\textsc{seq}\\
\glt`I want to tell a little story.'
\z


The complement clause says something about the contents of the story and functions as a kind of title. This type of structure is quite common in Papuan languages\footnote{\citet[231]{Reesink1987} treats them under relative clauses and considers them equivalents of English cleft sentences.} and is used mainly in an opening or closing formula in narrative texts \REF{ex:8:x1596}: 

\ea%x1596
\label{ex:8:x1596}
\gll Aria  yo  aakisa  {\ob}takira  en-owa  gelemuta  wia on-om-a-mik  nain{\cb}\textsubscript{\textsc{cc}} ma-i-yem.\\
alright  1s.\textsc{unm} now  child eat-\textsc{nmz} little 3p.\textsc{acc} make-\textsc{ben}-\textsc{bnfy}2.\textsc{pa}-1/3p  that1 say-\textsc{Np}-\textsc{pr}.1s\\
\glt`Alright now I tell about our making a feast for the children.'
\z


The complement ``clause'' may actually be a whole sentence, since it is possible to have medial clauses preceding the finite clause of the complement \REF{ex:8:x1597}:

\ea%x1597
\label{ex:8:x1597}
\gll {\ob}Tunde=pa  fikera  kuum-iwkin  ikiw-ep  waaya mik-a-m  nain{\cb}\textsubscript{\textsc{cc}} ma-i-yem.\\
Tuesday=\textsc{loc} kunai.grass burn-2/3p.\textsc{ds} go-\textsc{ss}.\textsc{seq} pig spear-\textsc{pa}-1s that1 say-\textsc{Np}-\textsc{pr}.1s\\
\glt`I tell about that when they burned kunai grass on Tuesday and I went and speared a pig.'
\z


Often the sentence has both a \textstyleAcronymallcaps{\textsc{np}} containing a word for `story' and the complement clause \REF{ex:8:x1594}. The relationship of these two \textstyleAcronymallcaps{\textsc{np}}s is not really appositional, because the nominalized clause modifies the other \textstyleAcronymallcaps{\textsc{np}}. But the nominalized clause is not a prototypical \textstyleAcronymallcaps{\textsc{rc}} either, in spite of identical structure, because \textstyleStyleVernacularWordsItalic{opora} is neither an antecedent \textstyleAcronymallcaps{\textsc{np}} nor a relative \textstyleAcronymallcaps{\textsc{np}} that would have a function in the \textstyleAcronymallcaps{\textsc{rc}.} I consider the nominalized clause a modifier of the other \textstyleAcronymallcaps{\textsc{np}}, and the whole comparable to the \textstyleAcronymallcaps{\textsc{np}} in \REF{ex:8:x1598}.\footnote{\citet{ComrieEtAl1995} present another alternative: treating complement clauses like this and relative clauses as a single construction, where the structure only indicates that the subordinate clause is connected to an \textsc{np}, and the interpretation of their relationship is done pragmatically. This possibility would need more investigation in Mauwake.} 

\ea%x1594
\label{ex:8:x1594}
\gll Aria  yo  aakisa  {\ob}fikera  ikum  kuum-e-mik  nain{\cb}\textsubscript{\textsc{cc}} opora  gelemuta=ko  ma-i-yem.\\
alright  1s.\textsc{unm} now  kunai.grass  illicitly  burn-\textsc{pa}-1/3p  that1 story  little=\textsc{nf} say-\textsc{Np}-\textsc{pr}.1s\\
\glt`Alright now I tell a little story about the kunai grass that was burned by arson.'
\z


\ea%x1598
\label{ex:8:x1598}
\gll manina  uuw-owa  opora \\
garden  work-\textsc{nmz} talk\\
\glt`garden work talk / talk (n.) about garden work'
\z


Another complementation strategy for utterance verbs is a clause with a nominalized verb. It is used when the event expressed in the clause is regarded as potential, rather than an actual activity or a fact. The following example has two levels of complementation, as the verb in the nominalized complement also takes a nominalized complement \REF{ex:8:x1599}:

\ea%x1599
\label{ex:8:x1599}
\gll I  {\ob}yiena  {\ob}miiw-aasa  muf-owa{\cb}  ikiw-owa{\cb} na-em-ik-omkun  o  ar-e-k. \\
1p.\textsc{unm} 1p.\textsc{gen} land-canoe pull-\textsc{nmz} go-\textsc{nmz} say-\textsc{ss}.\textsc{sim}-be-2/3p.\textsc{ds} 3s.\textsc{unm} become-\textsc{pa}-3s\\
\glt`While we were talking about our going to fetch a vehicle, she died (lit: became).'
\z


The same strategy is used with the verb \textstyleStyleVernacularWordsItalic{maak}- `tell' when it is used in the sense of ordering someone to do something \REF{ex:8:x1630}: 

\ea%x1630
\label{ex:8:x1630}
\gll Emar,  {\ob}no  muut  fain  uf-owa{\cb}  nefa  maak-e-m.\\
friend  2s.\textsc{unm} only  this  dance-\textsc{nmz} 2s.\textsc{acc} tell-\textsc{pa}-1s\\
\glt`Friend, I told you to dance only this.'
\z


\subsubsection{Complements of perception verbs} \label{sec:8.3.2.2}
%\hypertarget{RefHeading23561935131865}

It was mentioned above (\sectref{sec:8.2.3.4}) that the chaining structure is used with perception verbs in Mauwake as the main complementation strategy for perception verbs, when the complement is an activity \REF{ex:8:x1512} or event \REF{ex:8:x1600}. These are not genuine complement clauses, as they are not embedded in the main clause, but they perform the same function as regular complement clauses do. 

\ea%x1512
\label{ex:8:x1512}
\gll {\ob}Mukuruna  wu-am-ika-iwkin{\cb}  i  miim-a-mik.\\
noise  put-\textsc{ss}.\textsc{sim}-be-2/3p.\textsc{ds} 1p.\textsc{unm}  hear-\textsc{pa}-1/3p\\
\glt`We heard you making (the) noise.'
\z


\ea%x1600
\label{ex:8:x1600}
\gll {\ob}Urema  maneka  um-ep  ika-eya{\cb}  uruf-a-mik. \\
bandicoot  big  die-\textsc{ss}.\textsc{seq} be-2/3s.\textsc{ds}  see-\textsc{pa}-1/3p\\
\glt`They saw the big bandicoot dead (=having died).'
\z


A regular complement clause is only used with perception verbs about a past activity, when the complement clause reports a fact rather than an activity \REF{ex:8:x1628}, \REF{ex:8:x1629}. 

\ea%x1628
\label{ex:8:x1628}
\gll Iikir-ami  {\ob}iwera  nain  emeria  ar-e-p  ik-ua nain{\cb}\textsubscript{\textsc{cc}}  uruf-ap  {\dots}\\
get.up-\textsc{ss}.\textsc{sim} coconut that1 woman become-\textsc{pa}-3s be-\textsc{pa}.3s that1 see-\textsc{ss}.\textsc{seq}\\
\glt`He got up and saw that the coconut had become a woman, and {\dots}'
\z


\ea%x1629
\label{ex:8:x1629}
\gll {\ob}Yeesus  owow  iinan  urup-o-k  nain{\cb}\textsubscript{\textsc{cc}}  uruf-ap kemel-a-mik.\\
Jesus  village  above  ascend-\textsc{pa}-3s that1 see-\textsc{ss}.\textsc{seq} rejoice-\textsc{pa}-1/3p\\
\glt`They saw that Jesus ascended into heaven, and rejoiced.'
\z


When a perception verb takes an indirect question as a complement, it has to be a regular complement clause \REF{ex:8:x1631}:

\ea%x1631
\label{ex:8:x1631}
\gll Ni  {\ob}kakala  sira  kamenap  eliw-ar-i-ya  nain{\cb}\textsubscript{\textsc{cc}} uruf-eka.\\
2p.\textsc{unm} flower  custom  what.like  good-\textsc{inch}-\textsc{Np}-\textsc{pr}.3s that1 see-\textsc{imp}.2p\\
\glt`See how the flowers grow.'
\z


\subsubsection{Complements of cognitive verbs}
%\hypertarget{RefHeading23581935131865}

The verbs for knowing, \textstyleStyleVernacularWordsItalic{amisar}- and \textstyleStyleVernacularWordsItalic{paayar}- together cover the cognitive area of knowing facts and skills, coming to realize, and understanding. When the complement clause indicates contents of factual knowledge, it is usually a regular complement clause \REF{ex:8:x1602}.

\ea%x1602
\label{ex:8:x1602}
\gll O  {\ob}kaanek  aaw-ep  p-ekap-om-a-mik nain{\cb}  me  amis-ar-e-k.\\
3s.   where.\textsc{cf} get-\textsc{ss}.\textsc{seq} \textsc{bpx}-come-\textsc{ben}-\textsc{bnfy}2.\textsc{pa}-1/3p that1  not  knowledge-\textsc{inch}-\textsc{pa}-3s\\
\glt`He didn't know where they got it from and brought to him.'
\z


It seems that a clause with a nominalized verb is also used as a ``fact'' complement but only when it refers to pre-knowledge of an event \REF{ex:8:x1605}. It could also be understood as a ``potential'' type complement, in which case it is natural that it uses this complementation strategy. This requires more investigation. 

\ea%x1605
\label{ex:8:x1605}
\gll {\ob}O  ikiw-owa  nain{\cb}  amis-ar-e-n=i? \\
3s.\textsc{unm} go-\textsc{nmz} that1 knowledge-\textsc{inch}-\textsc{pa}-2s=\textsc{qm}\\
\glt`Did you know about his going?'
\z


When the complement is about knowing a skill, the verb in the complement clause is in nominalized form, or a medial clause is used \REF{ex:8:x1603}, \REF{ex:8:x1849}: 

\ea%x1603
\label{ex:8:x1603}
\gll {\ob}Nain  on-owa  (nain){\cb} me  amis-ar-e-m.\\
that1  do-\textsc{nmz} that1 not knowledge-\textsc{inch}-\textsc{pa}-1s\\
\glt`I don't know how to do that.'
\z


\ea%x1849
\label{ex:8:x1849}
\gll {\ob}Sawiter  inera  on-ap{\cb}  amis-ar-e-k.\\
Sawiter  basket  make-\textsc{ss}.\textsc{seq} knowledge-\textsc{inch}-\textsc{pa}-3s\\
\glt`Sawiter knows how to make baskets.'
\z


When the complement indicates lack of some experience, a construction with a medial clause is used. In this case, the main clause is in the negative, and the scope of the negation has to extend to the medial clause \REF{ex:8:x1604}:

\ea%x1604
\label{ex:8:x1604}
\gll {\ob}Owora  en-ep{\cb}  me  paayar-e-m.\\
betelnut  eat-\textsc{ss}.\textsc{seq} not understand-\textsc{pa}-1s\\
\glt`I'm not used to eating betelnut.' Or: `I don't know how to eat betelnut.'
\z


\subsubsection{Complement clauses as subjects}\label{sec:8.3.2.4}
%\hypertarget{RefHeading23601935131865}

Both types of a nominalized clause (§\sectref{sec:5.7.1}, \ref{sec:5.7.2}) may be used as subjects in verbless clauses, even though this function for complement clauses is not common. A clause with a nominalized verb is used when the activity is potential \REF{ex:8:x1636}, \REF{ex:8:x1637}. 

\ea%x1636
\label{ex:8:x1636}
\gll {\ob}Maa  wiar  ikum  aaw-owa{\cb}  eliwa=ki? \\
thing  3.\textsc{dat} illicitly take-\textsc{nmz} good=\textsc{cf}.\textsc{qm}\\
\glt`Is stealing from others good?'
\z


\ea%x1637
\label{ex:8:x1637}
\gll {\ob}Maa  eneka  me  en-owa{\cb}  maa  marew.\\
thing  tooth  not  eat-\textsc{nmz} thing  no(ne)\\
\glt`Not eating meat is all right.'
\z


A regular complement clause with a finite verb is used when the activity is considered a fact \REF{ex:8:x1639}:

\ea%x1639
\label{ex:8:x1639}
\gll {\ob}Ni  unuma  niam  p-ir-i-man  nain{\cb}  eliw(a) marew.\\
2p.\textsc{unm} name 2p.\textsc{refl} \textsc{bpx}-ascend-\textsc{Np}-\textsc{pr}.2p that1 good no(ne)\\
\glt`That you praise yourselves (lit: lift up your own name) is not good.'
\z


\subsection{Adverbial clauses} \label{sec:8.3.3}
%\hypertarget{RefHeading23621935131865}

Adverbial clauses are a very small group of subordinate clauses. They are Type 2 nominalized clauses (\sectref{sec:5.7.2}), and they perform the same function in a clause as a temporal or locative adverbial phrase. 

\subsubsection{Temporal clauses}  \label{sec:8.3.3.1}
%\hypertarget{RefHeading23641935131865}

The presence of the distal-1 demonstrative \textstyleStyleVernacularWordsItalic{nain} `that' indicates the pragmatic difference between the temporal clauses and those medial clauses that may get a temporal interpretation: the temporal clauses are presented as given information \REF{ex:8:x1540}--\REF{ex:8:x1624}, whereas the medial clauses usually present new information \REF{ex:8:x1632}, except when they occur in tail-head constructions.

\ea%x1540
\label{ex:8:x1540}
\gll Ni  {\ob}ifa  nia  keraw-i-ya  nain{\cb}  sira  kamenap on-i-man?\\
2p.\textsc{unm} snake 2p.\textsc{acc} bite-\textsc{Np}-\textsc{pa}.3s  that1 custom  what.like do-\textsc{Np}-\textsc{pr}.2p\\
\glt`When a snake bites you, what do you do?'
\z


\ea%x1569
\label{ex:8:x1569}
\gll {\ob}Maa  fain  pakak  na-e-k  nain{\cb}  yo  soran-e-m.\\
thing  this  bang  say-\textsc{pa}-3s  that1  1s.\textsc{unm}  be.startled-\textsc{pa}-1s\\
\glt`When this thing went ``bang!'' I got startled.'
\z


\ea%x1624
\label{ex:8:x1624}
\gll {\ob}Yo  napum-ar-e-m  nain{\cb}  eneka  maay-ar-e-m. \\
1s.\textsc{unm} sick-\textsc{inch}-\textsc{pa}-1s that1 tooth long-\textsc{inch}-\textsc{pa}-1s\\
\glt`When I got sick, I became hungry for meat (lit: my teeth got long).'
\z


\ea%x1632
\label{ex:8:x1632}
\gll Yo  napum-ar-ep  eneka  maay-ar-e-m.\\
1s.\textsc{unm} sick-\textsc{inch}-\textsc{ss}.\textsc{seq} tooth  long-\textsc{inch}-\textsc{pa}-1s\\
\glt`I got sick and became hungry for meat.'
\z


\subsubsection{Locative clauses} \label{sec:8.3.3.2}
%\hypertarget{RefHeading23661935131865}

Locative adverbial clauses use a clause-final deictic locative \textstyleStyleVernacularWordsItalic{nan} \REF{ex:8:x1621} or \textstyleStyleVernacularWordsItalic{neeke} \REF{ex:8:x1626} `there' instead of the demonstrative  \textstyleStyleVernacularWordsItalic{nain} `that'. Note that in \REF{ex:8:x1621} the locative noun \textstyleStyleVernacularWordsItalic{manina} `garden' is not a relative \textstyleAcronymallcaps{\textsc{np}}; if there were one, that would be \textstyleStyleVernacularWordsItalic{epa} `place' immediately preceding \textstyleStyleVernacularWordsItalic{nan} `there'. 

\ea%x1621
\label{ex:8:x1621}
\gll I  naap  ikiw-ep  {\ob}yiena  manina  on-a-mik  nan{\cb} ik-e-mik.\\
1p.\textsc{unm} thus  go-\textsc{ss}.\textsc{seq} 1p.\textsc{gen} garden  make-\textsc{pa}-1p  there be-\textsc{pa}-1p\\
\glt`We went there and stayed where we had made our gardens.'
\z


\ea%x1626
\label{ex:8:x1626}
\gll {\ob}Luuwa  niir-i-mik  neeke{\cb}  soomar-e-mik.\\
ball  play-\textsc{Np}-\textsc{pr}.1/3p there.\textsc{cf} walk-\textsc{pa}-1/3p\\
\glt`We walked (to) where they play football.'
\z


The following example is actually a locative relative clause, since it has a relative \textstyleAcronymallcaps{\textsc{np}} \textstyleStyleVernacularWordsItalic{kame} `side' that has a function in both clauses \REF{ex:8:x1638}:

\ea%x1638
\label{ex:8:x1638}
\gll {\ob}No  in-i-n  kame  nan{\cb}  urup-ep  tepak  iw-a-mik. \\
2s.\textsc{unm} sleep-\textsc{Np}-\textsc{pr}.2s side there ascend-\textsc{ss}.\textsc{seq} inside go-\textsc{pa}-1/3p\\
\glt`They climbed up on the side where you sleep and went inside.'
\z


\subsection{Adversative subordinate clause} \label{sec:8.3.4}
%\hypertarget{RefHeading23681935131865}

Coordinate adversative clauses were discussed in \sectref{sec:8.1.3}.

The topic marker -\textstyleStyleVernacularWordsItalic{na} (\sectref{sec:3.12.7.1}) marks an adversative clause when the main clause cancels an expectation, either expressed in the text or assumed to be in the hearer's mind. Because of this, this construction is used when some effort is frustrated \REF{ex:8:x729}, or when there is a strong element of surprise \REF{ex:8:x730} in the main clause. 

\ea%x729
\label{ex:8:x729}
\gll Mukuna  nain  umuk-a-mik=\textstyleEmphasizedVernacularWords{na}  me  pepek.\\
fire  that1 quench-\textsc{pa}-1/3p=\textsc{tp} not able\\
\glt`They tried to quench the fire, but couldn't.'
\z


\ea%x730
\label{ex:8:x730}
\gll Ekap-ep  uruf-a-k=\textstyleEmphasizedVernacularWords{na}  ifa  maneka=ke  siowa wasi-ep-pu-eya {\dots} \\
come-\textsc{ss}.\textsc{seq} see-\textsc{pa}-3s=\textsc{tp} snake  big=\textsc{cf} dog tie.around-\textsc{ss}.\textsc{seq}-\textsc{cmpl}-2/3s.\textsc{ds}\\
\glt`He came and looked, but a snake had tied itself around the dog, and/but {\dots}'
\z


In \REF{ex:8:x1393}, what the boys expect to see is a crocodile, but it turns out to be a turtle.

\ea%x1393
\label{ex:8:x1393}
\gll Takir(a)  oko=ke  pon  muneka  wu-ek-a-m  na-ep urup-em-ika-eya  uruf-ap  tuar=ke  na-ep alu-emi  baurar-e-k.  Takir(a)  unowa  ekap-ep uruf-a-mik=\textstyleEmphasizedVernacularWords{na}  pon=ke,  ne  unow=iya  op-ap kirip-a-mik.\\
boy  other=\textsc{cf} turtle egg  put-\textsc{cntf}-\textsc{pa}-1s  say-\textsc{ss}.\textsc{seq} ascend-\textsc{ss}.\textsc{sim}-be-2/3s.\textsc{ds} see-\textsc{ss}.\textsc{seq} crocodile=\textsc{cf} say-\textsc{ss}.\textsc{seq} shout-\textsc{ss}.\textsc{sim} flee-\textsc{pa}-3s boy  many come-\textsc{ss}.\textsc{seq} see-\textsc{pa}-1/3p=\textsc{tp} turtle=\textsc{cf} \textsc{add} many=\textsc{com} hold-\textsc{ss}.\textsc{seq} turn-\textsc{pa}-1/3p\\
\glt`A boy saw a turtle coming up (to the beach) to lay eggs and thought it was a crocodile, and shouted and fled. Many boys came and saw/looked, but it was a turtle, and they all together grabbed and turned it.'
\z


In \REF{ex:8:x1397}, a man talks to his son whom he wanted and expected to be a good person:

\ea%x1397
\label{ex:8:x1397}
\gll Aakisa  yo  nefa  uruf-i-yem=\textstyleEmphasizedVernacularWords{na} no  mua eliw marew.\\
now  1s.\textsc{unm} 2s.\textsc{acc} look-\textsc{Np}-\textsc{pr}.1s=\textsc{tp} 2s.\textsc{unm} man good no(ne)\\
\glt`I now look at you but you are not a good man.'
\z


Because these clauses express a cancellation or frustration of an expectation, a negator commonly follows as the first element in the main clause, and often the negator is the only element left of the main clause, as in \REF{ex:8:x1398}.

\ea%x1398
\label{ex:8:x1398}
\gll Marasin  wu-om-a-mik=\textstyleEmphasizedVernacularWords{na}  weetak.\\
medicine put-\textsc{ben}-\textsc{bnfy}2.\textsc{pa}-1/3p=\textsc{tp} no\\
\glt`They injected medicine in him, but no (it had no effect).'
\z


\subsection{Conditional clauses}  \label{sec:8.3.5}
%\hypertarget{RefHeading23701935131865}

\citet{Haiman1978} was the first one to clearly describe the close connection between conditionals and topics, and it has since then been attested in various languages \citep[292]{ThompsonEtAl2007}. In many Papuan languages, the connection is very evident (\citealt[235--244]{Reesink1987}, \citealt[304--308]{MacDonald1990}, \citealt[263]{Farr1999}. The protasis -- the subordinate clause expressing the condition -- provides the presupposition for the apodosis, the asserted main clause. In other words, ``it constitutes the framework which has been selected for the following discourse'' \citep[585]{Haiman1978}.

The conditional clauses in Mauwake can be grouped into three main groups on semantic and structural grounds: imaginative, predictive, and reality conditionals.\footnote{The terminology is from \citet[255]{ThompsonEtAl2007}.} Imaginative and predictive conditionals together belong to the unreality conditionals. Reality conditionals only include habitual/generic conditionals, as there are no present or past conditionals. 

The protasis clause, expressing the condition, is placed before the apodosis clause, which gives the consequence. Right-dislocation of the protasis is possible but rare. The verb forms in the protasis and the apodosis depend on the type of conditional. The topic marker -\textstyleStyleVernacularWordsItalic{na} is used as the conditional marker in the unreality conditional clauses, where it is cliticized to the last element of the protasis clause, usually the verb. Reality conditional clauses do not have a conditional marker, so structurally the protasis and apodosis are ordinary juxtaposed clauses. The intonation in the protasis has a slight rise towards the end, stronger with the topic marker -\textstyleStyleVernacularWordsItalic{na} than without it.  

In \textstyleEmphasizedWords{{imaginative conditional clauses}}, the verb in both the protasis and the apodosis is in the counterfactual mood, which is marked by the suffix -\textstyleStyleVernacularWordsItalic{ek}. The conditional/topic marker -\textstyleStyleVernacularWordsItalic{na} is always present. The same form is used for semantically counterfactual and hypothetical conditionals. The counterfactual interpretation \REF{ex:8:x1645} is more common, but especially if there is a reference to present \REF{ex:8:x1646} or future time \REF{ex:8:x1647}, it forces a hypothetical interpretation.

\ea%x1645
\label{ex:8:x1645}
\gll {\ob}Yo  Sek  haussik  ikiw-\textstyleEmphasizedVernacularWords{ek}-a-m=\textstyleEmphasizedVernacularWords{na}{\cb}  miiw-aasa=pa uroma  yaki-\textstyleEmphasizedVernacularWords{ek}-a-m.\\
1s.\textsc{unm}  Sek  hospital  go-\textsc{cntf}-\textsc{pa}-1s=\textsc{tp} land-canoe=\textsc{loc} stomach  wash-\textsc{cntf}-\textsc{pa}-1s\\
\glt`If I had gone to the Sek hospital, I would have given birth in the truck.'
\z


\ea%x1646
\label{ex:8:x1646}
\gll {\ob}Yena  aamun  aakisa  uruf-\textstyleEmphasizedVernacularWords{ek}-a-m=\textstyleEmphasizedVernacularWords{na}{\cb} kemel-\textstyleEmphasizedVernacularWords{ek}-a-m.\\
1s.\textsc{gen} 1s/p.younger.sibling  now  see-\textsc{cntf}-\textsc{pa}-1s=\textsc{tp} be.happy-\textsc{cntf}-\textsc{pa}-1s\\
\glt`If I saw my younger brother now, I would be happy.'
\z


\ea%x1647
\label{ex:8:x1647}
\gll {\ob}Morauta  iimar-ow(a)  mua  ik-\textstyleEmphasizedVernacularWords{ek}-a-k=\textstyleEmphasizedVernacularWords{na},{\cb} uurika ikiw-ep  maak-\textstyleEmphasizedVernacularWords{ek}-a-mik.\\
Morauta  stand.up-\textsc{nmz} man be-\textsc{cntf}-\textsc{pa}-3s=\textsc{tp} tomorrow go-\textsc{ss}.\textsc{seq} tell-\textsc{cntf}-\textsc{pa}-1/3p\\
\glt`If Morauta were the leader, we would go and talk to him tomorrow.'
\z


Usually the context determines the interpretation, but without a clear context the sentence may be ambiguous \REF{ex:8:x1648}:  

\ea%x1648
\label{ex:8:x1648}
\gll {\ob}Inasin  napuma  ik-\textstyleEmphasizedVernacularWords{ek}-a-k=\textstyleEmphasizedVernacularWords{na}{\cb}  sariar-\textstyleEmphasizedVernacularWords{ek}-a-k.\\
spirit/white.man sickness be-\textsc{cntf}-\textsc{pa}-3s=\textsc{tp} recover-\textsc{cntf}-\textsc{pa}-3s\\
\glt`If it were the white man's sickness\footnote{This is contrasted with \textit{owow napuma} `village sickness', caused by sorcery.} he would recover.' Or: `If it had been the white man's sickness, he would have recovered.'
\z


\textstyleEmphasizedWords{{Predictive conditionals}} are the most frequently used and show the greatest variation morphologically. The apodosis, and consequently the whole sentence, may be either a statement with a future tense verb, or a command with an imperative verb. The verb in the protasis may be in either present or future indicative, in imperative, or in medial form. The conditional/topic marker at the end of the protasis is obligatory. 

When the predictive conditional is a statement, the verb in both the protasis and in the apodosis is usually in the future tense \REF{ex:8:x1652}.

\ea%x1652
\label{ex:8:x1652}
\gll {\ob}No  oram  mokok=iw  \textstyleEmphasizedVernacularWords{ika-i-nan=na}{\cb}  ikoka  mua  lebuma \textstyleEmphasizedVernacularWords{ika}\textstyleEmphasizedVernacularWords{-}\textstyleEmphasizedVernacularWords{i}\textstyleEmphasizedVernacularWords{-}\textstyleEmphasizedVernacularWords{nan}.\\
2s.\textsc{unm} just  eye=\textsc{inst} be-\textsc{Np}-\textsc{fu}.2s=\textsc{tp} later  man  lazy be-\textsc{Np}-\textsc{fu}.2s\\
\glt`If you just watch with your eyes (without joining the work) you will be(come) a lazy man.'
\z


The protasis may have a medial verb form if the condition is likely to be fulfilled \REF{ex:8:x1654}, or when the protasis consists of two or more clauses that are in a medial-final relationship \REF{ex:8:x1653}.

\ea%x1654
\label{ex:8:x1654}
\gll {\ob}Emeria  \textstyleEmphasizedVernacularWords{sesenar-ek-a-m} \textstyleEmphasizedVernacularWords{na-ep=na}{\cb} waaya  ten  erup naap  wienak-i-non.\\
woman  buy-\textsc{cntf}-\textsc{pa}-1s say/think-\textsc{ss}.\textsc{seq}=\textsc{tp} pig  ten  two thus  feed.him-\textsc{Np}-\textsc{fu}.3s\\
\glt`If/when he wants to buy a wife, he will give him (=the bride's father) twenty or so pigs.'
\z


\ea%x1653
\label{ex:8:x1653}
\gll {\ob}Yaapan  me  \textstyleEmphasizedVernacularWords{piipu-ap=na}  anane  epaskun  ika-i-nan=na{\cb} no  iiwawun  weeser-i-nan.
\\
Japan  not  leave-\textsc{ss}.\textsc{seq}=\textsc{tp} always together  be-\textsc{Np}-\textsc{fu}.2s=\textsc{tp} 2s.\textsc{unm} altogether  finish-\textsc{Np}-\textsc{fu}.2s\\
\glt`If you don't leave the Japanese but are always together, you will be finished altogether.'
\z


Predictive conditionals allow right-dislocation of the protasis, but it is uncommon \REF{ex:8:x1662}: 

\ea%x1662
\label{ex:8:x1662}
\gll Owora  fain  aite  panewowa  onak-e,  {\ob}ekap-ep \textstyleEmphasizedVernacularWords{kerer-eya=na}{\cb}. \\
betelnut  this  1s/p.mother  old  feed-\textsc{imp}.2s come-\textsc{ss}.\textsc{seq} arrive-2/3s.\textsc{ds}=\textsc{tp}\\
\glt`Give these betelnuts to old mother to eat, if she comes and arrives here.'
\z


In those instances where the conditional marker is attached to a predicate that is not originally a verb, the predicate needs to have medial verb marking \REF{ex:8:x1660}, \REF{ex:8:x1661} (\sectref{sec:3.8.3.5}).

\ea%x1660
\label{ex:8:x1660}
\gll {\ob}\textstyleEmphasizedVernacularWords{Weetak-eya}\textstyleEmphasizedVernacularWords{=na}{\cb}  weetak.\\
no-2/3s.\textsc{ds}=\textsc{tp} no\\
\glt`If not, then not.'
\z


\ea%x1661
\label{ex:8:x1661}
\gll {\ob}Mauw-owa  \textstyleEmphasizedVernacularWords{manek-aya=na}{\cb}  yia  maak-i-non.\\
work-\textsc{nmz} big-2/3s.\textsc{ds}=\textsc{tp} 1p.\textsc{acc} tell-\textsc{Np}-\textsc{fu}.3s\\
\glt`If the work is big, he will tell us.'
\z


When the apodosis is in the imperative, there is normally some expectation that the the condition is to be fulfilled. When the likelihood is high, the medial form is used in the protasis \REF{ex:8:x1650}, \REF{ex:8:x1649}. Present tense \REF{ex:8:x1651} and imperative \REF{ex:8:x1656} indicate less, and future tense \REF{ex:8:x1657} the least likelihood for the condition to be fulfilled.

\ea%x1650
\label{ex:8:x1650}
\gll {\ob}Wia  \textstyleEmphasizedVernacularWords{uruf-ap=na}{\cb}  wia  maak-e.\\
3p.\textsc{acc} see-\textsc{ss}.\textsc{seq}=\textsc{tp} 3p.\textsc{acc} tell-\textsc{imp}.2s\\
\glt`If/when you see them, tell them.'
\z


\ea%x1649
\label{ex:8:x1649}
\gll {\ob}Maa  mauwa  nefa  \textstyleEmphasizedVernacularWords{maak-iwkin=na}{\cb}  opaimika  miim-e.\\
thing  what 2s.\textsc{acc} tell-2/3p.\textsc{ds}=\textsc{tp} talk  listen-\textsc{imp}.2s\\
\glt`Whatever they may tell you, listen to the talk.' (Lit: `If they tell you what(ever), listen to the talk.')
\z


\ea%x1651
\label{ex:8:x1651}
\gll Koora  pun  naap:  {\ob}mua  oko  naareke  koora  \textstyleEmphasizedVernacularWords{kua-i-ya=na}{\cb} o  asip-e.\\
house  also  thus  man  other  who.\textsc{cf} house  build-\textsc{Np}-\textsc{pr}.3s=\textsc{tp} 3p.\textsc{unm} help-\textsc{imp}.2p\\
\glt`A house is like that too: if/when any man builds a house, help him.'
\z


\ea%x1656
\label{ex:8:x1656}
\gll {\ob}Ni  kirip-owa  \textstyleEmphasizedVernacularWords{ika-inok}=\textstyleEmphasizedVernacularWords{na}{\cb} kirip-eka.\\
2p.\textsc{unm} reply-\textsc{nmz} be-\textsc{imp}.3s=\textsc{tp} reply-\textsc{imp}.2p\\
\glt`If you have something to reply, then reply.'
\z


\ea%x1657
\label{ex:8:x1657}
\gll {\ob}Wia  \textstyleEmphasizedVernacularWords{uruf-i-nan=na}{\cb} wia  maak-e.\\
3p.\textsc{acc} see-\textsc{Np}-\textsc{fu}.2s=\textsc{tp} 3p.\textsc{acc} tell-\textsc{imp}.2s\\
\glt`If you (happen to) see them, tell them.'
\z


\textstyleEmphasizedWords{{Reality conditional clauses}} are morpho-syntactically different from the other conditional clauses in that they are not marked with the topic marker. The protasis and apodosis are juxtaposed main clauses in future tense \REF{ex:8:x1644}, but this construction is mainly used to encode habitual or generic conditions. The protasis can never be right-dislocated, since it does not have the topic marker. 

\ea%x1644
\label{ex:8:x1644}
\gll {\ob}No  inasin(a)  unuma  me  unuf-i-nan{\cb},  mua  oko=ke  waaya nain  mik-ap  nefar  aaw-i-non.\\
2s.\textsc{unm} spirit  name  not  call-\textsc{Np}-\textsc{fu}.2s man  other=\textsc{cf}  pig that1  spear-\textsc{ss}.\textsc{seq} 2s.\textsc{dat} take-\textsc{Np}-\textsc{fu}.3s\\
\glt`If you don't call the spirit name, another man will spear the pig and take it from you.'
\z


If there are two protasis clauses, they may be juxtaposed without a connective \REF{ex:8:x1635} or joined with the pragmatic additive \textstyleStyleVernacularWordsItalic{ne} \REF{ex:8:x1643}.

\ea%x1635
\label{ex:8:x1635}
\gll {\ob}Nena  kuuf-i-nan,  parew-i-non{\cb},  eliw  perek-i-nan.\\
2s.\textsc{gen} see-\textsc{Np}-\textsc{fu}.2s mature-\textsc{Np}-\textsc{fu}.3s well harvest-\textsc{Np}-\textsc{fu}.2s\\
\glt`If you see it yourself and it is matured you may harvest it.'
\z


\ea%x1643
\label{ex:8:x1643}
\gll {\ob}Yo  um-i-nen  ne  yena  emeria  mua  oko aaw-i-non{\cb}, muuka  onaiya  me  ikiw-i-non.\\
1s.\textsc{unm} die-\textsc{Np}-\textsc{fu}.1s \textsc{add} 1s.\textsc{gen} woman  man  other take-\textsc{Np}-\textsc{fu}.3s son  with  not  go-\textsc{Np}-\textsc{fu}.3s\\
\glt`If I die and my wife takes another husband, she will not go (to him) with the son.'
\z


When a sentence contains alternatives expressed by two sets of reality conditional constructions, these are joined by the pragmatic additive \textstyleStyleVernacularWordsItalic{ne} \REF{ex:8:x1642}.

\ea%x1642
\label{ex:8:x1642}
\gll {\ob}Yo  auwa  miiwa=pa  mauw-i-nen{\cb},  irak-owa  marew,  ne {\ob}yo  aite  miiwa=pa  mauw-i-nen{\cb},  irak-owa  ika-i-non.\\
1s  1s/p.father  land=\textsc{loc} work-\textsc{Np}-\textsc{fu}.1s  fight-\textsc{nmz} no(ne) \textsc{add} 1s 1s/p.moher land=\textsc{loc}  work-\textsc{Np}-fu.1s fight-\textsc{nmz} be-\textsc{Np}-\textsc{fu}.3s\\
\glt`If I work on my father's land there is no fighting (over land), but if I work on my mother's land there will be fighting.'
\z


The same construction can encode a simple coordinate relationship, but it is less common. In spoken text a slightly falling intonation at the end of the first clause indicates a coordinate sentence \REF{ex:8:x1850}.

\ea%x1850
\label{ex:8:x1850}
\gll Oko=ke  pusun-emi  feeke  \textstyleEmphasizedVernacularWords{ikiw}\textstyleEmphasizedVernacularWords{-}\textstyleEmphasizedVernacularWords{i}\textstyleEmphasizedVernacularWords{-}\textstyleEmphasizedVernacularWords{non},  a  mua oko=ke  \textstyleEmphasizedVernacularWords{mik}\textstyleEmphasizedVernacularWords{-}\textstyleEmphasizedVernacularWords{i}\textstyleEmphasizedVernacularWords{-}\textstyleEmphasizedVernacularWords{non}.\\
other=\textsc{cf} run.loose-\textsc{ss}.\textsc{sim} here.\textsc{cf}  go-\textsc{Np}-\textsc{fu}.3s ah man other=\textsc{cf} spear-\textsc{Np}-\textsc{fu}.3s\\
\glt`Another (pig) will run loose and run this way, ah, another man will spear it.'
\z


\subsection{Concessive clauses}
%\hypertarget{RefHeading23721935131865}

Concessive clauses may look exactly like the predictive conditional clauses \REF{ex:8:x1655}. If the context is not clear enough, the phrase \textstyleStyleVernacularWordsItalic{nain pun} `that too' may be added between the protasis and the apodosis for clarification \REF{ex:8:x1430}.  

\ea%x1655
\label{ex:8:x1655}
\gll {\ob}Naapeya aara=ki e kasi=ke um-inok=na{\cb} ni nain kema bagiw-ir-ap  malaria sevis me wia iirar-eka.\\
therefore hen=\textsc{cf}.\textsc{qm} or cat=\textsc{cf} die-\textsc{imp}.3s=\textsc{tp} 2p.\textsc{unm} that1 liver hatred-rise-\textsc{ss}.\textsc{seq} malaria  service  not 3p.\textsc{acc} remove-\textsc{imp}.2p\\
\glt`Therefore, (even) if hens or cats die, do not get angry and drive away the Malaria Service people.'
\z


\ea%x1430
\label{ex:8:x1430}
\gll {\ob}Naap  yia  ma-ikuan=na{\cb}  \textstyleEmphasizedVernacularWords{nain}  \textstyleEmphasizedVernacularWords{pun}  ni  kekan-ep sira  eliwa  ook-eka.\\
thus  1p.\textsc{acc} say-\textsc{fu}.3p=\textsc{tp} that too 2p.\textsc{unm} be.strong-\textsc{ss}.\textsc{seq} custom good follow-\textsc{imp}.2p\\
\glt`Even if they talk about us like that, be strong and follow the good custom/ways.'
\z


\subsection{Coordination of subordinate clauses}\label{sec:8.3.7}
%\hypertarget{RefHeading23741935131865}

Subordinate clauses may also be coordinated with each other, although in normal speech the frequency of these constructions is low. The only subordinate clauses in the natural text data conjoined either by juxtaposition or with the additive \textstyleStyleVernacularWordsItalic{ne} are relative clauses \REF{ex:8:x1381}, \REF{ex:8:x1382}. The distal demonstrative \textstyleStyleVernacularWordsItalic{nain}, functioning as a relative marker, is attached to the end of each relative clause.

\ea%x1381
\label{ex:8:x1381}
\gll ...{\ob}\textstyleEmphasizedVernacularWords{waaya}  \textstyleEmphasizedVernacularWords{koka=pa}  \textstyleEmphasizedVernacularWords{ika-i-ya}  \textstyleEmphasizedVernacularWords{nain}{\cb}\textsubscript{\textsc{rc}}, {\ob}\textstyleEmphasizedVernacularWords{sokowa} \textstyleEmphasizedVernacularWords{maneka=pa} 
\textstyleEmphasizedVernacularWords{ika-i-ya}  \textstyleEmphasizedVernacularWords{nain}{\cb}\textsubscript{\textsc{rc}}  kanu-ep  aap-ekap-ep fikera=pa-r=iw  fiirim-eka.\\
pig  jungle=\textsc{loc} be-\textsc{Np}-\textsc{pr}.3s that1 grove  big=\textsc{loc} be-\textsc{Np}-\textsc{pr}.3s that1 chase-\textsc{ss}.\textsc{seq} \textsc{\textsc{bp}x}-come-\textsc{ss}.\textsc{seq} kunai.grass=\textsc{loc}-{\O}=\textsc{lim} gather-\textsc{imp}.2p\\
\glt`{\dots}chase the pigs that are in the jungle (and) that are in the big grove(s) and bring them and gather them right inside the kunai grass (area).'
\z


\ea%x1382
\label{ex:8:x1382}
\gll Ne {\ob}\textstyleEmphasizedVernacularWords{o} \textstyleEmphasizedVernacularWords{maa} \textstyleEmphasizedVernacularWords{kamenap} \textstyleEmphasizedVernacularWords{on-eya} \textstyleEmphasizedVernacularWords{wiar} \textstyleEmphasizedVernacularWords{uruf-i-n}  \textstyleEmphasizedVernacularWords{nain}{\cb}\textsubscript{\textsc{rc}}  \textstyleEmphasizedVernacularWords{ne} {\ob}\textstyleEmphasizedVernacularWords{wiar} \textstyleEmphasizedVernacularWords{miim-i-n} \textstyleEmphasizedVernacularWords{nain}{\cb}\textsubscript{\textsc{rc}} wia  maak-em-ika-i-nan.\\
\textsc{add} 3s.\textsc{unm} thing  how  do-\textsc{ss}.\textsc{seq} 3.\textsc{dat} see-\textsc{Np}-\textsc{pr}.2s that1 \textsc{add} 3.\textsc{dat} hear-\textsc{Np}-\textsc{pr}.2s that1 3p.\textsc{acc} tell-\textsc{ss}.\textsc{sim}-be-\textsc{Np}-\textsc{fu}.2s\\
\glt`And you will keep telling them that which you see and which you hear him do.'
\z


The chaining structure is also used to coordinate relative clauses \REF{ex:8:x1463} and complement clauses that have a nominalized verb (5.\ref{ex:5:x1845}), copied as \REF{ex:8:x1848} below: 

\ea%x1463
\label{ex:8:x1463}
\gll {\ob}\textstyleEmphasizedVernacularWords{Ni} \textstyleEmphasizedVernacularWords{manina} \textstyleEmphasizedVernacularWords{urup-ep} \textstyleEmphasizedVernacularWords{episowa} \textstyleEmphasizedVernacularWords{perek-a-man} \textstyleEmphasizedVernacularWords{nain}{\cb}\textsubscript{\textsc{rc}}  auwa  p-ikiw-om-aka.\\
2p.\textsc{unm} garden ascend-\textsc{ss}.\textsc{seq} tobacco  pick-\textsc{pa}-2p that 1s/p.father \textsc{\textsc{bp}x}-go-\textsc{ben}-\textsc{bnfy}2.\textsc{imp}.2p\\
\glt`Take to father the tobacco that you went up to the garden and picked.'
\z


\ea%x1848
\label{ex:8:x1848}
\gll Toiyan  iiriw  maak-ep-pu-a-mik, {\ob}\textbf{uuriw} \textbf{yia} \textstyleEmphasizedVernacularWords{aaw-ep} \textstyleEmphasizedVernacularWords{Madang}  \textstyleEmphasizedVernacularWords{ikiw-owa}{\cb}\textsubscript{\textsc{cc}} nain\\
Toiyan  already  tell-\textsc{ss}.\textsc{seq}-\textsc{cmpl}-\textsc{pa}-1/3p morning 1p.\textsc{acc} take-\textsc{ss}.\textsc{seq} Madang go-\textsc{nmz} that1\\
\glt`We already told Toiyan about taking us in the morning and going to Madang.' 
\z